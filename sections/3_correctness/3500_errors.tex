\chapter{Classifying Errors in Transformation Networks 
    \pgsize{15 p.}
}
\label{chap:errors}
\todo{Unterschied Klassifizierung/Kategorisierung: Klassifizierung benötigt Klassifizierungsdimension, Kategorisierung nicht}

\todo{Ziel dieses Kapitels: Wie hängen Korrektheitsdefinition voneinander ab (Hierarchie/Ebenen), was passiert, wenn einzelne Korrektheiten nicht erfüllt sind? Welche Fehler treten auf? Und wie häufig passiert das? (Letzteres eher als Evaluation verkaufen)}

\begin{copiedFrom}{ICMT}

\section{The Consistency Specification Process}
\todo{Process and levels from ICMT to identify how levels depend on each other (restructure according to relations (compatibility), individual transformations and networks), and derive which errors occur and how they depend on each other}

%\todoHeiko{Annahme: Immer nur Änderung an einem Modell (keine Synchronisation)}

The process of specifying consistency between $n>2$ types of models using a network of \acp{BX} can be separated into different conceptual levels.
We distinguish three such levels:
At the \emph{global level}, we describe the ($n$-ary) relations between all involved model types.
At the \emph{modularization level}, we split these global relations into modular, binary relations.
Finally, at the \emph{operationalization level}, we define preservation of consistency %after changes 
according to the modular relations.
That classification forms our contribution \ref{contrib:levels}.

All of these levels have to be considered during the consistency specification process.
A developer specifies consistency on one of these levels, depending on the abstraction level that the transformation language provides, and the transformation engine finally derives an operationalization from that.
Although a developer does not specify consistency on multiple levels, he or she has to think about the levels on and above the one consistency is specified on.
For example, to define an operationalization, the developer must be aware of the modular consistency relations.
%If a developer only specifies the modular consistency relations, the transformations engine has to derive an appropriate operationalization.
%In the end, to preserve consistency, an operationlization has to be derived from a specification on any of those levels.
The benefit of clearly separating these levels is that they have different potentials for mistakes, faults, and resulting failures. 
Consequently, avoiding a specific kind of mistake, which is related to one of the identified levels, completely prevents a specific category of failures.
We exemplify these levels in \autoref{fig:properties:levels_overview} and explain them in more detail in the following.

%\todoHeiko{Give an example for mistakes here! Maybe one example for every level as well?}

\subsection{Consistency Specification Levels}
\label{chap:properties:levels}

\begin{figure}
    \centering
%    \includegraphics[angle=-90, width=\textwidth]{figures/levels_overview.pdf}
    \newcommand{\modeltypesize}{3.6em}
\newcommand{\modelsize}{0.3em}

\begin{tikzpicture}[
    model type/.style={draw, circle, minimum width=\modeltypesize},
    model/.style={draw, fill, circle, minimum width=\modelsize, inner sep=0},
    label distance=-0.2em,
    every label/.append style={font=\small},
    legend/.style={font=\footnotesize},
    mininode/.style={inner sep=.25em},
]

\foreach \level in {1,2,3} {
    \node[model type, label=265:{$\mathcal{M}_1$}] at (\level*3.45*\modeltypesize, 0) (L\level_MM1) {};
    \node[model, above left=0.25*\modeltypesize and 0.2*\modeltypesize of L\level_MM1.center, anchor=center] (L\level_MM1_M1) {};
    \node[model, right=0.3*\modeltypesize of L\level_MM1.center, anchor=center] (L\level_MM1_M2) {};
    \node[model, below left=0.25*\modeltypesize and 0.2*\modeltypesize of L\level_MM1.center, anchor=center] (L\level_MM1_M3) {};
    
    \node[model type, label=280:{$\mathcal{M}_2$}] at (\level*3.45*\modeltypesize+1.7*\modeltypesize, 0) (L\level_MM2) {};
    \node[model, above right=0.25*\modeltypesize and 0.2*\modeltypesize of L\level_MM2.center, anchor=center] (L\level_MM2_M1) {};
    \node[model, left=0.3*\modeltypesize of L\level_MM2.center, anchor=center] (L\level_MM2_M2) {};
    \node[model, below right=0.25*\modeltypesize and 0.2*\modeltypesize of L\level_MM2.center, anchor=center] (L\level_MM2_M3) {};
    
    \node[model type, label=190:{$\mathcal{M}_3$}] at (\level*3.45*\modeltypesize+0.85*\modeltypesize, -1.35*\modeltypesize) (L\level_MM3) {};
    \node[model, above=0.25*\modeltypesize of L\level_MM3.center, anchor=center] (L\level_MM3_M1) {};
    \node[model, below left=0.25*\modeltypesize and 0.1*\modeltypesize of L\level_MM3.center, anchor=center] (L\level_MM3_M2) {};
    \node[model, below right=0.05*\modeltypesize and 0.25*\modeltypesize of L\level_MM3.center, anchor=center] (L\level_MM3_M3) {};
}

% CONSISTENCY L1
\draw[consistency relation] (L1_MM1_M1) -- (L1_MM2_M1);
\draw[consistency relation] ($(L1_MM1_M1)!0.13!(L1_MM2_M1)$) -- (L1_MM3_M2);
\draw[consistency relation] (L1_MM1_M2) -- (L1_MM2_M2);
\draw[consistency relation] ($(L1_MM1_M2)!0.5!(L1_MM2_M2)$) -- (L1_MM3_M1);

% CONSISTENCY L2
\draw[consistency relation] (L2_MM1_M1) -- (L2_MM2_M1);
\draw[consistency relation] (L2_MM1_M1) -- (L2_MM3_M2);
\draw[consistency relation] (L2_MM2_M1) -- (L2_MM3_M2);
\draw[consistency relation] (L2_MM1_M2) -- (L2_MM2_M2);
\draw[consistency relation] (L2_MM1_M2) -- (L2_MM3_M1);
\draw[consistency relation] (L2_MM2_M2) -- (L2_MM3_M1);

% CONSISTENCY L3
%\draw[consistency relation] (L3_MM1_M1) -- (L3_MM2_M1);
%\draw[consistency relation] (L3_MM1_M1) -- (L3_MM3_M2);
%\draw[consistency relation] (L3_MM2_M1) -- (L3_MM3_M2);
%\draw[consistency relation] (L3_MM1_M2) -- (L3_MM2_M2);
%\draw[consistency relation] (L3_MM1_M2) -- (L3_MM3_M1);
%\draw[consistency relation] (L3_MM2_M2) -- (L3_MM3_M1);

% USER CHANGE L3
\node[model, user changed element, above left=0.25*\modeltypesize and 0.2*\modeltypesize of L3_MM1.center, anchor=center] (L3_MM1_M1) {};
\node[model, user changed element, above right=0.25*\modeltypesize and 0.2*\modeltypesize of L3_MM2.center, anchor=center] (L3_MM2_M1) {};
\node[model, user changed element, below left=0.25*\modeltypesize and 0.1*\modeltypesize of L3_MM3.center, anchor=center] (L3_MM3_M2) {};
\draw[-latex, user changed element] (L3_MM1_M1) -- node[legend, above=0.2em] {$\Delta$} (L3_MM1_M2);

% CONSISTENCY PRESERVATION L3
\node[model, consistency changed element, right=0.3*\modeltypesize of L3_MM1.center, anchor=center] (L3_MM1_M2) {};
\node[model, consistency changed element, left=0.3*\modeltypesize of L3_MM2.center, anchor=center] (L3_MM2_M2) {};
\node[model, consistency changed element, above=0.25*\modeltypesize of L3_MM3.center, anchor=center] (L3_MM3_M1) {};
\draw[-latex, consistency changed element] (L3_MM2_M1) -- node[legend, below right=-0.3em] {$\Delta$} (L3_MM2_M2);
\draw[-latex, consistency changed element] (L3_MM3_M2) -- node[legend, below right=0 and -0.3em] {$\Delta$} (L3_MM3_M1);
\draw[-latex, dashed, consistency changed element, thin] ($(L3_MM1_M1)!0.4!(L3_MM1_M2)$) to[bend left=15] node[legend, above] {$\mathit{CPS}_{\mathit{CS}_{1,2}}$} ($(L3_MM2_M1)!0.4!(L3_MM2_M2)$);
\draw[-latex, dashed, consistency changed element, very thin] ($(L3_MM1_M1)!0.4!(L3_MM1_M2)$) to[bend right=15] node[legend, below left=0.5em and -0.5em] {$\mathit{CPS}_{\mathit{CS}_{1,3}}$} ($(L3_MM3_M2)!0.4!(L3_MM3_M1)$);

% CS LABELS
\node[consistency related element, above left=1.5em and -0.1em of L1_MM3.north, anchor=east, font=\footnotesize] {$\mathit{CS}$};
\node[consistency related element, above left=0.1em and 1.6em of L2_MM3, anchor=center, font=\footnotesize] {$\mathit{CS}_{1,3}$};
\node[consistency related element, above right=0.4em and 1.2em of L2_MM3, anchor=center, font=\footnotesize] {$\mathit{CS}_{2,3}$};
\node[consistency related element, above right=0.8em and 0.25*\modeltypesize of L2_MM1.east, anchor=south, font=\footnotesize] {$\mathit{CS}_{1,2}$};
%\node[consistency related element, above left=0.4em and 1.8em of L3_MM3, anchor=center, font=\footnotesize] {$CS_{1,3}$};
%\node[consistency related element, above right=0.4em and 1.4em of L3_MM3, anchor=center, font=\footnotesize] {$CS_{2,3}$};
%\node[consistency related element, above right=0.8em and 0.25*\modeltypesize of L3_MM1.east, anchor=south, font=\footnotesize] {$CS_{1,2}$};

% LEVEL LABELS
\node[anchor=north] (level1_label) at ([yshift=1.15*\modeltypesize]$(L1_MM1)!0.5!(L1_MM2)$) {Level 1: \emph{Global}};
\node[anchor=north] (level2_label) at ([yshift=1.15*\modeltypesize]$(L2_MM1)!0.5!(L2_MM2)$) {Level 2: \emph{Modularization}};
\node[anchor=north] (level3_label) at ([yshift=1.15*\modeltypesize]$(L3_MM1)!0.5!(L3_MM2)$) {Level 3: \emph{Operationalization}};

\coordinate (legend_anchor) at ([yshift=-0.2*\modeltypesize]L2_MM3.south);

\node[matrix, left=-0.2em of legend_anchor, anchor=north east, outer sep=0em, column sep=0.15em, row sep=-0.4em] (legend_left) {
    \node[model, anchor=center] (model_legend) {}; &
    \node[legend, anchor=west] {model of model type $\mathcal{M}_x$}; \\
    %
    \node[model, user changed element, anchor=center] (user_changed_model_legend)  {}; &
    \node[legend, anchor=west] {consistent models before user change}; \\
    %
    \node[model, consistency changed element, anchor=center] (consistency_preserved_model_legend)  {}; &
    \node[legend, anchor=west] {consistent models after consistency preservation}; \\
};

\node[matrix, left=0.2em of legend_anchor, anchor=north west, column sep=0.15em, row sep=-0.4em] (legend_right) {
    \draw[consistency relation] (0,0.25em) -- (1em,0.25em);
    \draw[consistency relation] (0.5em,0.25em) -- (0.5em,-0.2em); &
    \node[legend, consistency related element, anchor=west] {element of consistency specification}; \\% $\mathit{CS}$}; \\
    % 
    \draw[-latex, user changed element] (0,0) -- (1em,0); &
    \node[legend, user changed element, anchor=west] {user change introducing inconsistency}; \\
    %
    \draw[-latex, consistency changed element] (0,0) -- (1em,0); &
    \node[legend, consistency changed element, anchor=west, align=left] {execution of consistency preservation specification};\\ % $\mathit{CPS}_{\mathit{CS}}$}; \\
};
\coordinate (legend_left_dummy) at ([xshift=-0.3em]legend_left.west);
\node[draw=darkgray, inner sep=0em, fit=(legend_left_dummy)(legend_left)(legend_right)] {};


% \node[model] at (legend_anchor)(model_legend) {};
% \node[legend, right=0 of model_legend] {model of model type $\mathcal{M}_x$};
% \node[model, user changed element, below=1em of model_legend.center, anchor=center] (user_changed_model_legend)  {};
% \node[legend, right=0 of user_changed_model_legend] {consistent models before user change};
% \node[model, consistency changed element, below=1em of user_changed_model_legend.center, anchor=center]     (consistency_preserved_model_legend)  {};
% \node[legend, right=0 of consistency_preserved_model_legend] {consistent models after consistency preservation};
% %
% \draw[consistency relation] ([yshift=0.25em, xshift=20em]legend_anchor) -- ([yshift=0.25em, xshift=21em]legend_anchor);
% \draw[consistency relation] ([yshift=0.25em, xshift=20.5em]legend_anchor) -- ([yshift=-0.25em, xshift=20.5em]legend_anchor);
% \node[legend, consistency related element, anchor=west] at ([xshift=21em]legend_anchor) {element of consistency specification $CS$};
% %
% \draw[-latex, user changed element] ([yshift=-1em, xshift=20em]legend_anchor) -- ([yshift=-1em, xshift=21em]legend_anchor);
% \node[legend, user changed element, anchor=west] at ([yshift=-1em, xshift=21em]legend_anchor) {user change introducing inconsistency};
% %
% \draw[-latex, consistency changed element] ([yshift=-2em, xshift=20em]legend_anchor) -- ([yshift=-2em, xshift=21em]legend_anchor);
% \node[legend, consistency changed element, anchor=north west, align=left] at ([yshift=-1.2em, xshift=21em]legend_anchor) {execution of consistency preservation\\ specification $CPS_{CS}$};

\end{tikzpicture}

    \caption{Examples for Abstraction Levels in the Consistency Specification Process}
    \label{fig:properties:levels_overview}
\end{figure}

%\todoHeiko{Definieren was Correctness auf jedem der Levels heißt. System: Richtig bzgl. den tatsächlichen Konsistenzbeziehungen, Modularization: Richtig bzgl. der globalen Spezifikation, Operationalization: Korrekte Ergebnisse nach Änderungen bzgl. der modularen Spezifikationen}

\subsubsection*{Level 1 (\emph{Global}):}
At the most abstract level, we consider the knowledge about all actual consistency relations between the involved model types.
This knowledge can be represented by an $n$-ary relation between all model types, containing all tuples of consistent instances of the $n$ model types according to a consistency specification (\autoref{def:consistency_specification}). 
We refer to this as a \emph{global} consistency specification.

\subsubsection*{Level 2 (\emph{Modularization}):} 
At the second level, the global knowledge of the first level is separated into partial, binary consistency relations that, in combination, represent the overall knowledge about consistency in the system.
These relations should not contain any contradictions.
We do not necessarily need to describe relations between all pairs of model types, since some may not share information that may become inconsistent, or some may be represented transitively across other relations.
%This does also comprise the selection of a network structure that is capable of representing the full system knowledge.
%Mathematically speaking, on this level the knowledge on this level 
This knowledge 
can be represented by up to $\frac{n*(n-1)}{2}$ binary relations, each containing all pairs of instances of two of the model types that are consistent.
This corresponds to a set of binary consistency specifications according to \autoref{def:consistency_specification}.
We refer to these as \emph{modular} consistency specifications.\\[-1em]

\noindent\textit{Remark:} 
%\begin{remark*}
Although in theory not all kinds of $n$-ary relations can be separated into binary relations~\cite{stevens2017a}, we assume that all consistency relations considered in an automated consistency preservation process can be expressed by binary relations.
We shortly discussed why this is a reasonable assumption in \autoref{chap:properties:terminology}.
%As stated by \textcite{stevens2017a}, this is a reasonable assumption, because it is hard enough for people to think about and specify binary relations.
%Additionally, the separation into binary relations is the extreme case, which we explicitly consider here, nevertheless our finding also apply to a modularization into consistency specifications of higher arity.
%\end{remark*}

%\todoHeiko{Definieren, wie man mit mehrere CPS Konsistenz erreicht -> Zum Ziel der Arbeit kommen, dass man modulare CPS bel. Veschalten kann, um zu Konsistenz zu kommen.}  
\subsubsection*{Level 3 (\emph{Operationalization}):}
At this level, the consistency preservation is operationalized in terms of binary consistency preservation specifications according to \autoref{def:consistency_preservation_specification}. % for modular consistency specifications %after \autoref{def:consistency_specification} 
%on the second level.
%This requires the definition of update operations that restore consistency after user changes.
As discussed in \autoref{chap:properties:terminology}, we consider a set of consistency preservation specifications that can be composed to restore consistency.
In contrast to a single \ac{BX}, an operationalization in %arbitrary
networks of \acp{BX} has to deal with confluence of information.
This can lead to problems, such as overwrites or duplications of information, whenever a change can be propagated across at least two paths in the network of \acp{BX} to the same model.
%It especially requires an identification of matching elements in different consistency preservation specifications to avoid duplicate element creations or insertions. 
%For example, if an element is propagated from a model of $\mathcal{M}_1$ to a model of $\mathcal{M}_2$ and from this model to one of $\mathcal{M}_3$, then an additional consistency preservation specification from $\mathcal{M}_1$ to $\mathcal{M}_3$ must consider the already existing element instead of creating an additional or overwriting the existing one.
We have seen an example, in which such multiple transformation paths cannot be avoided, in \autoref{fig:properties:motivational_example}.
%We refer to these as modular consistency preservation specifications.

%\todoHeiko{Maybe call this recombination of modularized knowledge? Makes fokus more on operationalization/ consistency preservation, as that is where the problems occur}
%Mathematically speaking, on this level consistency preservation specifications according to \autoref{def:consistency_preservation_specification} are defined, which adhere to the consistency specifications according to \autoref{def:consistency_specification} on the second level.
%Mathematically speaking, this is the step from a consistency specification of relations to a consistency preservation specification of restore functions.
%We refer to these as modular consistency preservation specifications.
%OLD: partial knowledge (adding information to make partial specifications combinable without knowing about the others)


\subsection{Selecting the Specification Level}

A transformation language finally derives a consistency preservation specification from a specification on any of the levels and executes it. %specification is finally transformed into a consistency preservation specification by a transformation language, no matter on which level it is specified.
Imperative transformation languages expect specifications at the operationalization level, whereas rather declarative, usually bidirectional transformation languages expect specification at the modularization level.
Specifications at the global level are rather unusual, but could for example be expressed with multidirectional QVT-R~\cite{macedo2014a}, or the Commonalities language~\cite{gleitze2017a}.
A specification must finally be free of mistakes that can be made on any of those levels. 
The responsibility depends on the abstraction level the transformation language provides, as the developer is responsible for avoiding mistakes at or above the level at which he or she specifies consistency, whereas the transformation language is responsible for those below.

Specifications must especially be correct regarding all higher levels.
This means that an operationalization in consistency preservation specifications must preserve consistency according to the underlying modular consistency specifications.
So after changing a consistent set of models, the consistency preservation has to return another set of models that is consistent again, as shown in \autoref{fig:properties:levels_overview}.
Additionally, modular consistency specifications must be correct regarding the global specification in the sense that it must contain the same sets of models as the global specification. %exactly those sets of models that are in the relation of the modular specification are in the one of global specification.
Finally, the global consistency specification has to be correct regarding some, usually informal, notion of consistency for the considered model types.
Since this can usually not be validated, we assume a global specification to be correct. %made the assumption of having normative consistency specifications.
This conforms to the notion of \emph{correctness} already defined for \acp{BX}~\cite{stevens2010sosym}, but is used for the extension to networks of \acp{BX} here.

% A specification on one level must always be correct regarding all higher levels.
% This means, for example, that an operationalization in consistency preservation specifications must preserve consistency according to the underlying modular consistency specifications.
% So after changing a consistent set of models, the consistency preservation has to return another set of models that is consistent again, as shown in \autoref{fig:levels_overview}.
% Additionally, modular consistency specifications must be correct regarding the global specification in the sense that it must contain the same sets of models than the global specification. %exactly those sets of models that are in the relation of the modular specification are in the one of global specification.
% Finally, the global consistency specification has to be correct regarding some, usually informal notion of consistency for the considered model types.
% Since this can usually not be validated, we assume a global specification to be correct. %made the assumption of having normative consistency specifications.

% To preserve consistency for a set of models, all specification levels have to be considered.
% A developer defines a concrete specification on one of these levels, which usually depends on the used transformation language and especially the level of abstraction it provides.
% He therefore has to consider all levels on and above the one he specifies consistency on to ensure correctness, and especially has to avoid mistakes that can be made on those levels.
% On the other hand, the transformation language abstracts from all levels below the one consistency is specified on, and especially has to avoid all mistakes that can be made there.
% In the end, %for correct consistency preservation, 
% all mistakes on all levels must be avoided, but the responsibility depends on the level consistency is specified on.


% %Depending on level on which consistency is defined, potentials mistakes that can be made on one of the levels either have to be avoided by the developer or by the used transformation language and its engine.
% %The developer must avoid all mistakes that can arise from the conceptual levels on and above the one he specifies consistency one, whereas the transformation language must ensure that no mistakes occur on the lower levels, which it abstracts from.
% Imperative transformation languages expect specifications on the operationalization level, which means that the transformation developer has to ensure that he makes no mistakes on the Levels 1--3.
% Rather declarative, usually bidirectional transformation languages expect specifications on the modularization level, as they abstract from the operationalization. In that case, the developer must only deal with potential mistakes from Level 1--2 and possible operationalization mistakes have to be handled by the language and its engine.
% Finally, specifications on system level are rather unusual, but could for example be expressed with multidirectional QVT-R~\cite{macedo2014a}, or the Commonalities language~\cite{gleitze2017a}. % or the domain-specific DUALLY approach~\cite{eramo2012a}. %Nevertheless, knowledge about the overall consistency constraints in the system is still important, even when providing a specification on a lower level.
% \todoHeiko{Die letzten beiden Abschnitte möglicherweise tauschen, da im letzten eingängier erklärt wird, wie die Ebenen voneinandern abhängen. Der mittele Abschnitt ist sehr schwer verständlich.}
% %As we are focused on networks of \acp{BX}, whose specification happens on the modularization or operationalization level, we are especially concerned with those two levels. 
% %We therefore assume that the developer ensures that no mistakes on the system level are made by knowing about


\end{copiedFrom} % ICMT



\begin{copiedFrom}{ICMT}
%FORMERLY: \section{Issues in Networks of Bidirectional Transformations}
%\label{sec:classification}

In this section, we %first identify and 
categorize potential \emph{failures} that can occur when executing \acp{BX} in a network to preserve consistency.
We then consider \emph{mistakes} that a developer can make and that lead to \emph{faults} in the specifications of consistency and its preservation.
\todo{Heißt der folgende Satz nicht, dass auf jeder Ebene genau ein Typ von Fehler auftreten kann?}
We derive them from the specification levels introduced in \autoref{chap:properties:levels}, as each kind of mistake is specific for one of those levels.
We finally relate the mistakes to the failures that can occur while executing the operationalization of a faulty consistency specification.
That categorization forms our contribution \ref{contrib:issues}.
In the following, we only discuss failures and their causing mistakes, but no strategies to solve or avoid them.
Such strategies are discussed in \autoref{chap:prevention}.
%The identification and categorization in this section is based on argumentation. To show the correctness of identified mistakes, failures and their dependencies, we provide an appropriate evaluation in \autoref{sec:evaluation}.

%\todoHeiko{Introduce mistake, fault, failure} 

% \begin{itemize}
%     \item Define three essential abstraction levels in the development process
%     \item Levels depend on each other, so \emph{fulfillment} on one level is mandatory to investigate the next level
%     \item Mistakes on all levels may introduce failures in the execution of the operationalization of consistency constraint preservation
%     \item We summarize potential failures, identify their causes (mistakes and faults) and then categorize and relate them
% \end{itemize}

\section{Potential Failures}
\label{chap:errors:failures}

Mistakes in the specification of consistency, no matter on which of the specification levels, % (\autoref{sec:process:levels}),
can lead to failures when executing the preservation of consistency according to that specification. % on an actual system. 
Before identifying the causal mistakes, we first categorize the types of potential failures into three categories. We depict them in \autoref{fig:correctness:categorization}.

\begin{figure}
    \centering
    \newcommand{\labelcolumnwidth}{5em}
\newcommand{\contentcolumnwidth}{10.4em}

\begin{tikzpicture}[
    entry/.style={font=\small}
]

    % \node[fill=lightgray!60, anchor=north west, align=left, minimum height=6em, minimum width=\labelcolumnwidth+0.47*\contentcolumnwidth, text width=\labelcolumnwidth+0.4*\contentcolumnwidth] (system) {Level 1:\\ \mediumfont \emph{System}};
    % \node[below=6em of system.west, anchor=west, align=left, fill=lightgray!30, minimum height=6em, minimum width=\labelcolumnwidth+0.47*\contentcolumnwidth, text width=\labelcolumnwidth+0.4*\contentcolumnwidth] (modularization) {Level 2:\\ \mediumfont \emph{Modulari-}\\\emph{zation}};
    % \node[below=6em of modularization.west, anchor=west, align=left, fill=lightgray!60, minimum height=6em, minimum width=\labelcolumnwidth+0.47*\contentcolumnwidth, text width=\labelcolumnwidth+0.4*\contentcolumnwidth] (operationalization) {Level 3:\\ \mediumfont \emph{Operatio-}\\\emph{nalization}};
    
    \node[text depth=16.5em, minimum width=\contentcolumnwidth+\labelcolumnwidth, minimum height=18.75em] %, fill=lightgray!50]
    (errors) {\textbf{Mistakes}};
    \node[right=\contentcolumnwidth+0.5*\labelcolumnwidth of errors.center, anchor=center, text depth=16.5em, minimum width=\contentcolumnwidth, minimum height=18.75em] %, fill=lightgray!25] 
    (faults) {\textbf{Faults}};
    \node[right=\contentcolumnwidth of faults.center, anchor=center, text depth=16.5em, minimum width=\contentcolumnwidth, minimum height=18.75em] %, fill=lightgray!50] 
    (failures) {\textbf{Failures}};
    
    
    \node[below=5em of errors.north west, anchor=west, align=left, inner sep=0.7em] (system) {Level 1:\\ \emph{Global}};
    \node[below=5.5em of system.west, anchor=west, align=left, inner sep=0.7em] (modularization) {Level 2:\\ \emph{Modulari-}\\\emph{zation}};
    \node[below=5.5em of modularization.west, anchor=west, align=left, inner sep=0.7em] (operationalization) {Level 3:\\ \emph{Operatio-}\\\emph{nalization}};
    
    % \node[above right=5em and \labelcolumnwidth of system.west, anchor=north, text depth=18em, minimum width=\contentcolumnwidth, minimum height=20em, fill=lightgray!60] (errors) {Errors};
    % \node[right=\contentcolumnwidth of errors.center, anchor=center, text depth=18em, minimum width=\contentcolumnwidth, minimum height=20em, fill=lightgray!30] (faults) {Faults};
    % \node[right=\contentcolumnwidth of faults.center, anchor=center, text depth=18em, minimum width=\contentcolumnwidth, minimum height=20em, fill=lightgray!60] (failures) {Failures};
    \draw[thick] (errors.north west) -- ++(\labelcolumnwidth+3*\contentcolumnwidth, 0);
    \draw[very thin] ([yshift=2.75em, xshift=0.05*\contentcolumnwidth]system.west) -- ++(\labelcolumnwidth+0.9*\contentcolumnwidth,0);
    \draw[very thin] ([yshift=2.75em, xshift=\labelcolumnwidth+1.05*\contentcolumnwidth]system.west) -- ++(0.9*\contentcolumnwidth,0);
    \draw[very thin] ([yshift=2.75em, xshift=\labelcolumnwidth+2.05*\contentcolumnwidth]system.west) -- ++(0.9*\contentcolumnwidth,0);
    %\draw[dashed, very thin] ([yshift=3em]system.west) -- ++(\labelcolumnwidth+\contentcolumnwidth,0);
    \draw[dashed, very thin] ([yshift=2.75em, xshift=0.05*\contentcolumnwidth]modularization.west) -- ++(\labelcolumnwidth+0.9*\contentcolumnwidth,0);
    \draw[dashed, very thin] ([yshift=2.75em, xshift=0.05*\contentcolumnwidth]operationalization.west) -- ++(\labelcolumnwidth+0.9*\contentcolumnwidth,0);
    %\draw[dashed, very thin] ([yshift=-2.75em]operationalization.west) -- ++(\labelcolumnwidth+\contentcolumnwidth,0);
    \draw[thick] (errors.south west) -- ++(\labelcolumnwidth+3*\contentcolumnwidth, 0);
    
    % ERRORS
    \node[entry, right=\labelcolumnwidth+0.5*\contentcolumnwidth of system.west, anchor=south, align=center] (error_system_incomplete) {incomplete system\\ knowledge};
    \node[entry, below=0em of error_system_incomplete.south, anchor=north, align=center] (error_system_incorrect) {incorrect system\\ knowledge};
    
    \node[entry, right=\labelcolumnwidth+0.5*\contentcolumnwidth of modularization.west, anchor=south, align=center] (error_modular_incomplete) {incomplete modular\\ knowledge};
    \node[entry, below=0em of error_modular_incomplete.south, anchor=north, align=center] (error_modular_incorrect) {contradicting modular\\ knowledge};
    
    \node[entry, right=\labelcolumnwidth+0.5*\contentcolumnwidth of operationalization.west, anchor=center, align=center] (error_operationalization) {unknown connection of\\ modular specifications};
    
    % FAULTS
    \node[entry, right=\contentcolumnwidth of error_system_incomplete.north, anchor=north, align=center] (fault_missing) {missing\\ consistency constraint};
    \node[entry, below=1em of fault_missing.south, anchor=north, align=center] (fault_additional) {additional\\ consistency constraint};
    \node[entry, right=\contentcolumnwidth of error_modular_incorrect.north, anchor=north, align=center] (fault_contradicting) {contradicting\\ consistency constraint};
    \node[entry, right=\contentcolumnwidth of error_operationalization.north, anchor=north, align=center] (fault_matching) {missing element\\ matching};
    
    % FAILURES
    \node[entry, right=\contentcolumnwidth of fault_missing.north, anchor=north, align=center] (failure_inconsistent) {
    inconsistent termination\\
    \textbullet\ deterministic\\
    \textbullet\ non-deterministic\\
    };
    
    \node[entry, below=1.5em of failure_inconsistent.south, anchor=north, align=center] (failure_termination) {
    non-termination\\
    \textbullet\ alternating loop\\
    \textbullet\ diverging loop\\
    };
    
    \node[entry, below=1.5em of failure_termination.south, anchor=north, align=center] (failure_duplications) {
    duplications\\
    \textbullet\ multiple instantiations\\
    \textbullet\ multiple insertions\\
    };
    
    % ERROR -> FAULT
    \draw[-latex] ([yshift=0.4em]error_system_incomplete.east) -- ([yshift=0.4em]fault_missing.west);
    \draw[-latex] ([yshift=0.5em]error_system_incorrect.east) .. controls ++ (1.5em,-0.1em) and ($(fault_additional.west)-(1.5em,-0.1em)$) .. (fault_additional.west);
    
    \draw[-latex] ([yshift=0.5em]error_modular_incomplete.east) .. controls ++ (2em,0.1em) and ($(fault_missing.west)-(2em,0.3em)$) .. ([yshift=-0.4em]fault_missing.west);
    
    \draw[-latex] ([yshift=0.5em]error_modular_incorrect.east) .. controls ++ (1.5em,0.1em) and ($(fault_additional.west)-(1.5em,0.3em)$) .. ([yshift=-0.4em]fault_additional.west);
    \draw[-latex] ([yshift=0.5em]error_modular_incorrect.east) -- ([yshift=0.5em]fault_contradicting.west);
    
    \draw[-latex] (error_operationalization.east) -- (fault_matching.west);
    
    % FAULT -> FAILURE
    \draw[-latex] ([yshift=-0.3em]fault_missing.east) .. controls ++ (1.5em,-0.1em) and ($(failure_inconsistent.west)-(1.5em,-0.5em)$) .. ([yshift=0.4em,xshift=0.8em]failure_inconsistent.west);
    
    \draw[-latex] ([yshift=-0.4em]fault_contradicting.east) .. controls ++ (1.5em,0.1em) and ($(failure_inconsistent.west)-(1.5em,0.6em)$) .. ([yshift=-0.5em,xshift=0.8em]failure_inconsistent.west);
    \draw[-latex] ([yshift=-0.4em]fault_contradicting.east) .. controls ++ (2em,0.1em) and ($(failure_termination.west)-(2em,-1.3em)$) .. ([yshift=1.4em]failure_termination.west);
    
    \draw[-latex] ([yshift=0.5em]fault_matching.east) .. controls ++ (2.5em,0.1em) and ($(failure_duplications.west)-(1.5em,-1.3em)$) .. ([yshift=1.4em,xshift=2em]failure_duplications.west);
    
\end{tikzpicture}
    \caption{Categorization and Dependencies of Mistakes, Faults and Failures}
    \label{fig:correctness:categorization}
\end{figure}

First, consistency preservation can fail by \textbf{resulting in an inconsistent state}. This can either occur \emph{deterministically} or \emph{non-deterministically}, if the result depends on the execution order of the consistency preservation specifications.

Second, consistency preservation can fail by \textbf{not terminating}. This can either manifest in an \emph{alternating loop}, when a feature, e.g., an attribute, alternates between two or more values, or in a \emph{diverging loop}, when at least one feature value diverges, e.g., a number counting up or a string being repeatedly appended.

Third, consistency preservation can result in \textbf{duplications}. \emph{Multiple instantiation} can occur because different consistency preservation specifications instantiate an element multiple times, although all of them represent the same element. % and thus should be the same. 
For example, an element is created by transformations $\mathcal{M}_1 \rightarrow \mathcal{M}_2 \rightarrow \mathcal{M}_3$ and another is created by transformation $\mathcal{M}_1 \rightarrow \mathcal{M}_3$, although there should be only one element.
\emph{Multiple referencing} can occur due to the same reason because an element is inserted into a reference or attribute list several times, although it should be inserted only once. 
%Such duplications are a special kind of termination in inconsistent states.


% If mistakes are made during the specification of consistency, no matter on which of the levels introduced in \autoref{sec:process:levels}, this can finally lead to failures in the consistency preservation executed on an actual system. Before identifying the causing mistakes, we first give an overview on the types of failures that may occur and separate them into three categories.\\[-0.7em]

% \compactsubsection{Termination in inconsistent states}
% \begin{enumerate}[topsep=4pt]
%     \item \emph{Deterministic:} The consistency preservation process can deterministically terminate in a state that is not consistent. % wrt. the defined consistency specification.
%     \item \emph{Non-deterministic:} Consistency preservation can non-deterministically terminate in an inconsistent state, depending on the execution order of the binary consistency preservation specifications. %in which the partial consistency preservation rules are executed.
% \end{enumerate}

% \compactsubsection{Non-termination}
% \begin{enumerate}[resume, topsep=4pt]
%     \item \emph{Alternating loops:} Consistency preservation can be non-terminating, alternating between two or more values in at least one feature (e.g. a number or a String alternating between two values).
%     \item \emph{Diverging loops:} Consistency preservation can be non-terminating, having at least one feature with a diverging value (e.g. a number counting up or down, a String being always appended).
% \end{enumerate}

% \compactsubsection{Duplications}
% \begin{enumerate}[resume, topsep=4pt]
%     \item \emph{Multiple instantiation:} An element can be instantiated multiple times by different consistency preservation specifications, although all of them represent the same element and thus should be the same. E.g. an element is created by transforamtions $\mathcal{M}_1 \rightarrow \mathcal{M}_2 \rightarrow \mathcal{M}_3$ and another is created by transformations $\mathcal{M}_1 \rightarrow \mathcal{M}_3$, although the same element is meant.
%     \item \emph{Multiple referencing:} An element may also be inserted into a non-containment reference or an attribute list several times, although the same element is meant, within the same situations as multiple instantiation can occur.
% \end{enumerate}


\section{Mistakes and Faults}
\label{chap:errors:mistakes}

%\todoErik{Ich dachte immer, \enquote{Mistakes} machen nur Menschen}
Developers or the transformation engine can make different kinds of mistakes on each of the specification levels, which lead to faults in the specification and finally to different kinds of failures during consistency preservation.
In the following, we derive mistakes and faults from the specification levels, depicted in \autoref{fig:correctness:categorization}.

\subsection{Global Level}
Regarding global consistency specifications for a set of model types, two basic mistakes can be made. 
These mistakes concern compliance of the defined consistency specification with the actual notion of consistency between the involved model types.
First, a specification can be incomplete (\emph{underspecified}), which means that some consistency constraints are missed. 
As a result, the consistency specification according to \autoref{def:consistency_specification} would contain more tuples of models than are actually consistent to each other. 
%%As a result, if one would define the consistency specification according to \autoref{def:consistency_specification}, more tuples of models would be in the relation than are actually consistent to each other. 
%Incomplete consistency specifications can lead to \emph{false positives}, when investigating whether a given tuple of models is consistent or not.
Another potential mistake are too restricted (\emph{overspecified}) consistency specifications, which means that additional, faulty consistency constraints are considered. 
As a result, actually consistent tuples of models would be missing in the consistency specification according to \autoref{def:consistency_specification}. 
%As a result, if one would define the consistency specification according to \autoref{def:consistency_specification}, actually consistent tuples of models would not be in the relation. 
%This %, in contrast, 
%can lead to \emph{false negatives}, because actually consistent models are identified as inconsistent.

\begin{figure}[bt]
    \centering
%    \includegraphics[angle=-90, width=\textwidth]{figures/levels_overview.pdf}
    \newcommand{\modeltypesize}{3.6em}
\newcommand{\modelsize}{0.3em}

\begin{tikzpicture}[
    model type/.style={draw, circle, minimum height=\modeltypesize},
    model/.style={draw, fill, circle, minimum height=\modelsize, inner sep=0},
    label distance=-0.2em,
    every label/.append style={font=\small},
    legend/.style={font=\footnotesize}
]

\foreach \level in {1,2,3} {
    \node[model type, label=265:{$\mathcal{M}_1$}] at (\level*3.45*\modeltypesize, 0) (L\level_MM1) {};
    \node[model, above left=0.25*\modeltypesize and 0.2*\modeltypesize of L\level_MM1.center, anchor=center] (L\level_MM1_M1) {};
    \node[model, right=0.3*\modeltypesize of L\level_MM1.center, anchor=center] (L\level_MM1_M2) {};
    \node[model, below left=0.25*\modeltypesize and 0.2*\modeltypesize of L\level_MM1.center, anchor=center] (L\level_MM1_M3) {};
    
    \node[model type, label=280:{$\mathcal{M}_2$}] at (\level*3.45*\modeltypesize+1.7*\modeltypesize, 0) (L\level_MM2) {};
    \node[model, above right=0.25*\modeltypesize and 0.2*\modeltypesize of L\level_MM2.center, anchor=center] (L\level_MM2_M1) {};
    \node[model, left=0.3*\modeltypesize of L\level_MM2.center, anchor=center] (L\level_MM2_M2) {};
    \node[model, below right=0.25*\modeltypesize and 0.2*\modeltypesize of L\level_MM2.center, anchor=center] (L\level_MM2_M3) {};
    
    \node[model type, label=190:{$\mathcal{M}_3$}] at (\level*3.45*\modeltypesize+0.85*\modeltypesize, -1.35*\modeltypesize) (L\level_MM3) {};
    \node[model, above=0.25*\modeltypesize of L\level_MM3.center, anchor=center] (L\level_MM3_M1) {};
    \node[model, below left=0.25*\modeltypesize and 0.1*\modeltypesize of L\level_MM3.center, anchor=center] (L\level_MM3_M2) {};
    \node[model, below right=0.05*\modeltypesize and 0.25*\modeltypesize of L\level_MM3.center, anchor=center] (L\level_MM3_M3) {};
}

% CONSISTENCY L1
\draw[consistencyrel] (L1_MM1_M1) -- (L1_MM2_M1);
\draw[consistencyrel] ($(L1_MM1_M1)!0.13!(L1_MM2_M1)$) -- (L1_MM3_M2);
\draw[consistencyrel] (L1_MM1_M2) -- (L1_MM2_M2);
\draw[consistencyrel] ($(L1_MM1_M2)!0.5!(L1_MM2_M2)$) -- (L1_MM3_M1);

% CONSISTENCY L2
\draw[consistencyrel] (L2_MM1_M1) -- (L2_MM2_M1);
\draw[consistencyrel] (L2_MM1_M1) -- (L2_MM3_M2);
\draw[consistencyrel, dashed] (L2_MM2_M1) -- (L2_MM3_M2);
\draw[consistencyrel] (L2_MM1_M2) -- (L2_MM2_M2);
\draw[consistencyrel] (L2_MM1_M2) -- (L2_MM3_M1);
\draw[consistencyrel] (L2_MM2_M2) -- (L2_MM3_M1);
%\draw[consistency relation, dash dot] (L2_MM2_M1) -- (L2_MM3_M3);

% CONSISTENCY L3
%\draw[consistency relation] (L3_MM1_M1) -- (L3_MM2_M1);
%\draw[consistency relation] (L3_MM1_M1) -- (L3_MM3_M2);
%\draw[consistency relation] (L3_MM2_M1) -- (L3_MM3_M2);
%\draw[consistency relation] (L3_MM1_M2) -- (L3_MM2_M2);
%\draw[consistency relation] (L3_MM1_M2) -- (L3_MM3_M1);
%\draw[consistency relation] (L3_MM2_M2) -- (L3_MM3_M1);

% USER CHANGE L3
\node[model, user changed element, above left=0.25*\modeltypesize and 0.2*\modeltypesize of L3_MM1.center, anchor=center] (L3_MM1_M1) {};
\node[model, user changed element, above right=0.25*\modeltypesize and 0.2*\modeltypesize of L3_MM2.center, anchor=center] (L3_MM2_M1) {};
\node[model, user changed element, below left=0.25*\modeltypesize and 0.1*\modeltypesize of L3_MM3.center, anchor=center] (L3_MM3_M2) {};
\draw[-latex, user changed element] (L3_MM1_M1) -- node[legend, above=0.2em] {$\Delta$} (L3_MM1_M2);

% CONSISTENCY PRESERVATION L3
\node[model, consistency changed element, right=0.3*\modeltypesize of L3_MM1.center, anchor=center] (L3_MM1_M2) {};
\node[model, consistency changed element, left=0.3*\modeltypesize of L3_MM2.center, anchor=center] (L3_MM2_M2) {};
%\node[model, consistency changed element, above=0.25*\modeltypesize of L3_MM3.center, anchor=center] (L3_MM3_M1) {};
\node[model, consistency changed element, below right=0.05*\modeltypesize and 0.25*\modeltypesize of L3_MM3.center, anchor=center] (L3MM3_M3) {};
\draw[-latex, consistency changed element] (L3_MM2_M1) -- node[legend, below right=-0.3em] {$\Delta$} (L3_MM2_M2);
%\draw[-latex, consistency changed element] (L3_MM3_M2) -- node[legend, below right=0 and -0.3em] {$\Delta$} (L3_MM3_M1);
\draw[-latex, consistency changed element] (L3_MM3_M2) -- node[legend, below] {$\Delta$} (L3_MM3_M3);
\draw[-latex, dashed, consistency changed element, thin] ($(L3_MM1_M1)!0.4!(L3_MM1_M2)$) to[bend left=15] node[legend, above] {$\mathit{CPS}_{\mathit{CS}_{1,2}}$} ($(L3_MM2_M1)!0.4!(L3_MM2_M2)$);
%\draw[-latex, dashed, consistency changed element, very thin] ($(L3_MM1_M1)!0.4!(L3_MM1_M2)$) to[bend right=15] ($(L3_MM3_M2)!0.4!(L3_MM3_M1)$);
\draw[-latex, dashed, consistency changed element, very thin] ($(L3_MM1_M1)!0.4!(L3_MM1_M2)$) to[bend right=15] node[legend, below left=0.5em and -0.5em] {$\mathit{CPS}_{\mathit{CS}_{1,3}}$} ($(L3_MM3_M2)!0.4!(L3_MM3_M3)$);

% CS LABELS
\node[consistency related element, above left=1.5em and -0.1em of L1_MM3.north, anchor=east, font=\footnotesize] {$\mathit{CS}$};
\node[consistency related element, above left=0.1em and 1.6em of L2_MM3, anchor=center, font=\footnotesize] {$\mathit{CS}_{1,3}$};
\node[consistency related element, above right=0.4em and 1.2em of L2_MM3, anchor=center, font=\footnotesize] {$\mathit{CS}_{2,3}$};
\node[consistency related element, above right=0.8em and 0.25*\modeltypesize of L2_MM1.east, anchor=south, font=\footnotesize] {$\mathit{CS}_{1,2}$};
%\node[consistency related element, above left=0.4em and 1.8em of L3_MM3, anchor=center, font=\footnotefont] {$CS_{1,3}$};
%\node[consistency related element, above right=0.4em and 1.4em of L3_MM3, anchor=center, font=\footnotefont] {$CS_{2,3}$};
%\node[consistency related element, above right=0.8em and 0.25*\modeltypesize of L2_MM1.east, anchor=south, font=\footnotefont] {$CS_{1,2}$};

% LEVEL LABELS
\node[anchor=north] (level1_label) at ([yshift=1.15*\modeltypesize]$(L1_MM1)!0.5!(L1_MM2)$) {Level 1: \emph{Global}};
\node[anchor=north] (level2_label) at ([yshift=1.15*\modeltypesize]$(L2_MM1)!0.5!(L2_MM2)$) {Level 2: \emph{Modularization}};
\node[anchor=north] (level3_label) at ([yshift=1.15*\modeltypesize]$(L3_MM1)!0.5!(L3_MM2)$) {Level 3: \emph{Operationalization}};

\coordinate (legend_anchor) at ([yshift=-0.2*\modeltypesize]L2_MM3.south);

\node[matrix, left=-0.2em of legend_anchor, anchor=north east, outer sep=0em, column sep=0.15em, row sep=-0.4em] (legend_left) {
    \node[model, anchor=center] (model_legend) {}; &
    \node[legend, anchor=west] {model of model type $\mathcal{M}_x$}; \\
    %
    \node[model, user changed element, anchor=center] (user_changed_model_legend)  {}; &
    \node[legend, anchor=west] {consistent models before user change}; \\
    %
    \node[model, consistency changed element, anchor=center] (consistency_preserved_model_legend)  {}; &
    \node[legend, anchor=west] {consistent models after consistency preservation}; \\
};

\node[matrix, left=0.2em of legend_anchor, anchor=north west, column sep=0.15em, row sep=-0.4em] (legend_right) {
    \draw[consistencyrel] (0,0.25em) -- (1em,0.25em);
    \draw[consistencyrel] (0.5em,0.25em) -- (0.5em,-0.2em); &
    \node[legend, consistency related element, anchor=west] {element of consistency specification}; \\% $\mathit{CS}$}; \\
    % 
    \draw[-latex, user changed element] (0,0) -- (1em,0); &
    \node[legend, user changed element, anchor=west] {user change introducing inconsistency}; \\
    %
    \draw[-latex, consistency changed element] (0,0) -- (1em,0); &
    \node[legend, consistency changed element, anchor=west, align=left] {execution of consistency preservation specification};\\ % $\mathit{CPS}_{\mathit{CS}}$}; \\
};
\coordinate (legend_left_dummy) at ([xshift=-0.3em]legend_left.west);
\node[draw=darkgray, inner sep=0em, fit=(legend_left_dummy)(legend_left)(legend_right)] {};

\end{tikzpicture}

    \caption{Examples for Mistakes on Different Specification Levels}
    \label{fig:correctness:mistakes_specification_levels}
\end{figure}

\subsection{Modularization Level}
When developers modularize the global consistency specification by defining binary consistency specifications, these modular specifications can be non-compliant with the global one. 
Two kinds of mistakes, similar to those at the global level, can be distinguished, regarding compliance of modular and global specifications. %, but regarding compliance of modular and global specifications rather than between the global specification and the actual notion of consistency.
First, modular consistency specifications can be incomplete (\emph{underspecified}), so that there are global constraints which are not covered by them. 
The modular consistency specifications $\mathit{CS}_{1,2}$, $\mathit{CS}_{2,3}$ and $\mathit{CS}_{1,3}$ in \autoref{fig:correctness:mistakes_specification_levels} are incomplete iff
%For three model types $\mathcal{M}_1, \mathcal{M}_2$ and $\mathcal{M}_3$ with a global consistency specification $\mathit{CS}$, the binary specifications $\mathit{CS}_{1,2}$, $\mathit{CS}_{2,3}$ and $\mathit{CS}_{1,3}$, as depicted in \autoref{fig:levels_overview} are underspecified iff 
%\todoHeiko{Hier die Grafik aus dem Level-Kapitel übernehmen}
\begin{align*}
    & \exists M_1, M_2, M_3 : \\
    & \hspace{1em} (M_1, M_2) \in \mathit{CS}_{1,2} \land (M_2, M_3) \in \mathit{CS}_{2,3} \land (M_1, M_3) \in \mathit{CS}_{1,3} \land (M_1, M_2, M_3) \not\in \mathit{CS}
\end{align*}
This finally leads to \emph{false positives} when investigating whether a given tuple of models is consistent regarding the global specification. %as actually inconsistent models (regarding the global specification) are identified as consistent. %investigating whether a given set of models is consistent regarding the global specification or not, because actually inconsistent models regarding $\mathit{CS}$ are consistent according to all modular relations $\mathit{CS}_{1,2}$, $\mathit{CS}_{2,3}$ and $\mathit{CS}_{1,3}$.
%This finally leads to \emph{false positives} when investigating whether a given set of models is consistent regarding the global specification or not, because actually inconsistent models regarding $CS$ are consistent according to all modular relations $\mathit{CS}_{1,2}$, $\mathit{CS}_{2,3}$ and $\mathit{CS}_{1,3}$.
%\todoHeiko{Das folgende eher zu Avoidance Strategy? Überlapp vorhanden?}
%Such incomplete specifications can especially occur if constraints between two types of models are not expressed at all (so the consistency specification covers all model pairs), but are only transitively defined over two or more other relations. 
%For example, if $\mathit{CS}_{1,3}$ shall be omitted and transitively expressed across $\mathit{CS}_{1,2}$ and $\mathit{CS}_{2_3}$, the following must hold:
% \begin{align*}
%     & \forall M_1 \in \mathcal{M}_1 : \forall M_2 \in \mathcal{M}_2 : \forall M_3 \in \mathcal{M}_3 : \\
%     & \hspace{1em} \mathit{CS}(M_1, M_2, M_3) \iff \mathit{CS}_{1,2}(M_1, M_2) \land \mathit{CS}_{2,3}(M_2, M_3)
% \end{align*}
%\begin{align*}
    %& \forall M_1, M_2, M_3 : (M_1, M_2, M_3) \in \mathit{CS} \Leftrightarrow (M_1, M_2) \in \mathit{CS}_{1,2} \land (M_2, M_3) \in \mathit{CS}_{2,3}
%\end{align*}
%If this transitive relation misses or is even unable to express certain direct constraints, inconsistent models would be idenitified as consistent. %\todoHeiko{Das transitive muss man wohl an einem Beispiel erklären, am besten Ref. zu Intro}
Modular consistency specifications cannot only be incomplete because of an actual specification mistake, but also because of $n$-ary relations on the global level that cannot be expressed by a set of binary relations.
We excluded that case by our assumption made in \autoref{chap:properties:levels}, as otherwise a modularization into binary relations would not be possible at all.
If such cases have to be supported, the modularization would have to be extended to also consider $n$-ary relations.

Second, a modular specification can be too restricted (\emph{overspecified}) regarding the global consistency specification if additional constraints are added. 
The modular consistency specifications in \autoref{fig:correctness:mistakes_specification_levels} are overspecified iff
\begin{align*}
    & \exists M_1, M_2, M_3 : \\
    & \hspace{1em} (M_1, M_2, M_3) \in \mathit{CS} \land \big[ (M_1, M_2) \not\in \mathit{CS}_{1,2} \lor (M_2, M_3) \not\in \mathit{CS}_{2,3} \lor (M_1, M_3) \not\in \mathit{CS}_{1,3} \big]
\end{align*}
In \autoref{fig:correctness:mistakes_specification_levels}, omitting the dashed relation in $\mathit{CS}_{2,3}$ would lead to such an overspecifiation.
Overspecifications lead to additional constraints regarding the global specification, but also, and more severe, to contradicting constraints regarding other modular specifications.
In case of contradictions, the modular consistency specifications cannot be fulfilled at the same time.
In such a case, the graph of consistency relations %, as shown in \autoref{fig:mistakes_specification_levels}, 
would contain no cylces, i.e. sets of models that are consistent to each other.
We have discussed an example for such contradicting specifications %in the motivating example 
in \autoref{chap:properties:levels}, where constraints for transferring an employee name contradicted. % contains contradicting constraints for transferring the name. % to other types of models.
%In this case, when several binary specifications are combined to keep multiple models consistency, the resulting fault are incompatible binary specifications in the sense that different relations cannot hold at the same time because they are based on different global consistency specifications.
Such mistakes lead to \emph{false negatives} as actually consistent models (regarding the global specification) are identified as inconsistent. %, when investigating whether a given set of models is consistent regarding the global specification or not, because actually consistent models regarding $\mathit{CS}$ are not consistent according to the modular relations $\mathit{CS}_{1,2}$, $\mathit{CS}_{2,3}$ and $\mathit{CS}_{1,3}$.

%\begin{itemize}
    %\item inadequate structure
    %\item missing knowledge about other modular relation
%\end{itemize}

\subsection{Operationalization Level}
The types of mistakes that can be made at the operationalization level are different from those at the other levels, because this level does not concern the definition of consistency specifications (\autoref{def:consistency_specification}), but of consistency \emph{preservation} specifications (\autoref{def:consistency_preservation_specification}).
Such specifications are faulty if no composition of them exists that returns a consistent tuple of models for each possible change. % it does not lead to a consistent state after making modifications to a consistent tuple of models.
In \autoref{fig:correctness:mistakes_specification_levels}, an exemplary application of a single consistency preservation specification is depicted that leads to models that are not consistent according to the (global and modular) consistency specifications.
%If no concatenation of CPSs exists that finally returns a consistent set of models for each possible change, the specifications are faulty.
Let $\mathcal{CPS}$ be a set of consistency preservation specifications  %, e.g. $\mathcal{CPS} := \{\mathit{CPS}_{1}, \ldots, \mathit{CPS}_{m}\}$
for the binary consistency specifications $\mathcal{CS}$ % := \{\mathit{CS}_{1,2}, \mathit{CS}_{2,3}, \mathit{CS}_{1,3}\}$ %(where there can be more than one consistency preservation specifications for each consistency specification) 
%in \autoref{fig:mistakes_specification_levels}. %, metamodels $\mathcal{M}_0, \ldots, \mathcal{M}_n$, 
and
let $\mathfrak{M}_{\mathcal{CS}}$ be the set of model tuples that are consistent regarding $\mathcal{CS}$ (cf. \autoref{chap:properties:terminology}). 
The consistency preservation specifications are faulty iff
% \begin{align*}
%     & \exists M_0, M'_0 \in \mathcal{M}_0, M_1, M'_1 \in \mathcal{M}_1, M_2, M'_2 \in \mathcal{M}_2 : \forall \mathit{CPS}_0, \ldots, \mathit{CPS}_k \in \mathcal{CPS} : \\
%     & \hspace{1em} ((M_0, M''_0), (M_1, M''_1), (M_2, M''_2)) = \mathit{CPS}_0 \circ \dots \circ \mathit{CPS}_k((M_0, M'_0), (M_1, M'_1), (M_2, M'_2)) \\
%     & \hspace{1em} \Rightarrow \exists \mathit{CS}_{i,j} \in \mathcal{CS} : \neg \mathit{CS}_{i,j}(M''_i, M''_j)
% %    & \exists M_0, M'_0 \in \mathcal{M}_0, \ldots M_n, M'_n \in \mathcal{M}_n : \nexists \mathit{CPS}_0, \ldots, \mathit{CPS}_k \in \mathcal{CPS} : \\
% %    & \hspace{1em} ((M_0, M''_0), \dots, (M_n, M''n)) := \mathit{CPS}_0 \circ \dots \circ \mathit{CPS}_k((M_0, M'_0), \dots, (M_n, M'_n)) \\
% %    & \hspace{1em} \land \exists \mathit{CS}_{i,j} \in \mathcal{CS} : \neg \mathit{CS}_{i,j}(M''_i, M''_j)
% \end{align*}
\begin{align*}
    & \exists (M_1, \dots, M_n) \in \mathfrak{M}_{\mathcal{CS}}, (M'_1, \dots, M'_n) \in \mathcal{M}_1 \times \dots \times \mathcal{M}_n: \forall \mathit{CPS}_1, \ldots, \mathit{CPS}_k \in \mathcal{CPS} : \\
    %& \exists M_1, M'_1 \in \mathcal{M}_1, \dots, M_n, M'_n \in \mathcal{M}_n : \forall \mathit{CPS}_1, \ldots, \mathit{CPS}_k \in \mathcal{CPS} : \\
    & \hspace{1em} \mathit{CPS}_1 \circ \dots \circ \mathit{CPS}_k \big((M_1, M'_1), \dots, (M_n, M'_n) \big) = \big( (M_1, M''_1), \dots, (M_n, M''_n) \big)\\
    & \hspace{2em} \land \exists \mathit{CS}_{i,j} \in \mathcal{CS} : (M''_i, M''_j) \notin \mathit{CS}_{i,j}
%    & \exists M_0, M'_0 \in \mathcal{M}_0, \ldots M_n, M'_n \in \mathcal{M}_n : \nexists \mathit{CPS}_0, \ldots, \mathit{CPS}_k \in \mathcal{CPS} : \\
%    & \hspace{1em} ((M_0, M''_0), \dots, (M_n, M''n)) := \mathit{CPS}_0 \circ \dots \circ \mathit{CPS}_k((M_0, M'_0), \dots, (M_n, M'_n)) \\
%    & \hspace{1em} \land \exists \mathit{CS}_{i,j} \in \mathcal{CS} : \neg \mathit{CS}_{i,j}(M''_i, M''_j)
\end{align*}
%\todoHeiko{Das ist nicht so schön, weil nicht klar ist, welche CS to welcher CPS gehört und so}
%This means that there can be changes for which no execution of consistency preservation specifications is able to properly restore consistency. 

In practice, mistakes at the operationalization level occur due to missing identification of equal elements in different consistency preservation specifications. 
In our motivational example (\autoref{fig:properties:motivational_example}), %consider that an employee is created in the scheduling system for an employee created in the task management system after introducing one in the personnel management system.
consider that an employee is created in the personnel management system, transformed to the task management system and from that to the scheduling system.
The additional direct specification between personnel management and scheduling system has to consider the already created employee rather than instantiating a new one.
%We consider %this case and 
%options to avoid such problems in \autoref{sec:avoiding:matching}.
%In our example, if a class is created in Java after creating a UML class for a ADL component through appropriate consistency preservation specifications and the consistency preservation specification between ADL and Java also defines the creation of a class in Java, it is necessary that the already existing class is considered rather than creating a new class. We will consider this case and options to avoid such problems in \autoref{sec:avoiding:matching}.

% \begin{itemize}
%     \item unknown connection of elements in consistency specifications
%     \item Really, really make an example here, to distinguish from modularization level!!
% \end{itemize}

%In the following, we call all mistakes on modularization and operationalization level \emph{interoperability issues}, as they are all concerned with modularized specifications that have to interoperate. 


\section{Categorization and Discussion}
\label{chap:errors:categorization}

\begin{figure}[tb]
    \centering
    \newcommand{\classdistance}{5em}
\newcommand{\classwidth}{5em}
\newcommand{\objectwidth}{6.3em}

% #1: height
% #2: width
% #3: positioning options
% #4: name
\newcommand{\modeltypebgtextabove}[4]{\node[fill=lightgray!20, text depth=#1, minimum width=#2, #3] {\small \textit{#4}};}
\newcommand{\modeltypebgtextbelow}[4]{\node[fill=lightgray!20, text height=#1, minimum width=#2, #3] {\small \textit{#4}};}

\usetikzlibrary{positioning,arrows.meta,shapes.misc,matrix}

\begin{tikzpicture}[
    consistency preservation/.style={-latex, consistency changed element, font=\small},
    user change/.style={-latex, user changed element, font=\small},
    legend/.style={font=\footnotesize},
    mininode/.style={inner sep=.25em},
]
% requires styles: consistency related element, user changed element, consistency changed element
% requires tikzuml

\pgfdeclarelayer{bg}
\pgfsetlayers{bg,main}

% METAMODEL

\umlclassvarwidth{mm_task_employee}{}{Employee}{
name\\
\dots
}{\classwidth}    

\umlclassvarwidth[, below left=0.4*\classdistance and 2.5*\classdistance of mm_task_employee.north, anchor=north]{mm_personnel_employee}{}{
Employee}{
firstName\\
lastName\\
\dots
}{\classwidth}

\umlclassvarwidth[, right=5*\classdistance of mm_personnel_employee.north, anchor=north]{mm_scheduling_employee}{}{
Employee}{
name\\
\dots
}{\classwidth}

\draw[consistency relation] (mm_personnel_employee.north) |- node[above, pos=0.6] {name = firstName + "\textvisiblespace" + lastName} ([yshift=-0.7em]mm_task_employee.north west);
\draw[consistency relation]([yshift=-0.7em]mm_task_employee.north east) -| node[above, pos=0.25] {name = name} (mm_scheduling_employee.north);
\draw[consistency relation] ([yshift=-3em]mm_personnel_employee.north east) -- node[below, pos=0.5, align=left] {
\textit{Opt. 1}: name = lastName + ",\textvisiblespace" + firstname\\
\textit{Opt. 2}: name = firstName + "\textvisiblespace" + lastname
} ([yshift=-3em]mm_scheduling_employee.north west);

% LEGEND
\coordinate (legend_anchor) at ([yshift=-1.3*\classdistance]mm_task_employee);

\node[draw=darkgray, matrix, legend, nodes=mininode, below=0em of legend_anchor, anchor=north, outer sep=0, inner sep=0.4em, column sep=0.2em, row sep=-0.2em] (legend) {
    \draw[consistency relation] (0,0) -- (1.4em,0); &
    \node[consistency related element, anchor=west] {consistency constraint}; \\
    %
    \draw[-latex, user changed element] (0,0) -- (1.4em,0); &
    \node[user changed element, anchor=west] {user change}; \\
    %
    \draw[-latex, consistency changed element] (0,0) -- (1.4em, 0); &
    \node[consistency changed element, anchor=west, align=left] (legend_cpr_label) {consistency preservation};\\
};

% \draw[consistency relation] (legend_anchor) -- ([xshift=1.5em]legend_anchor);
% \node[legend, consistency related element, anchor=west] at ([xshift=1.5em]legend_anchor) {consistency constraint};

% \draw[-latex, user changed element] ([yshift=-1em]legend_anchor) -- ([yshift=-1em, xshift=1.5em]legend_anchor);
% \node[legend, user changed element, anchor=west] at ([yshift=-1em, xshift=1.5em]legend_anchor) {user change};

% \draw[-latex, consistency changed element] ([yshift=-2em]legend_anchor) -- ([yshift=-2em, xshift=1.5em]legend_anchor);
% \node[legend, consistency changed element, anchor=north west, align=left] (legend_cpr_label) at ([yshift=-1.3em, xshift=1.5em]legend_anchor) {consistency preservation};

%\coordinate (legend_upper_left) at ([xshift=-0.7em, yshift=0.7em]legend_anchor);
%\begin{pgfonlayer}{bg}
%    \node[fill=lightgray!30, fit=(legend_upper_left)(legend_cpr_label)] {};
%\end{pgfonlayer}


% LEVEL 2 ERROR
\coordinate (failure_l2_anchor) at ([xshift=-2.15*\classdistance, yshift=-2.65*\classdistance]mm_task_employee.north); %([xshift=4*\classdistance]mm_task_employee.north);

\umlobjectvarwidth[, consistency changed element, fill=white, below=0 of failure_l2_anchor, anchor=north]{l2_task_employee}{}{
: Employee}{
name="Alice Do"
}{\objectwidth}

\umlobjectvarwidth[, user changed element, fill=white, below left=1.2*\classdistance and 0.75*\classdistance of l2_task_employee.north, anchor=north] {l2_personnel_employee}{}{
: Employee}{
firstName="Alice"
lastName="Do"
}{\objectwidth}

\umlobjectvarwidth[, consistency changed element, fill=white, right=1.9*\classdistance of l2_personnel_employee.north, anchor=north] {l2_scheduling_employee}{}{
: Employee}{
name="Do, Alice"
}{\objectwidth}

\umlhuman{l2_human}{at ([xshift=1em, yshift=4.5em]l2_personnel_employee.north west)}{user changed element}{}{0.5}
\draw[user change] ([xshift=1em,yshift=3.3em]l2_personnel_employee.north west) -- node[above right=-0.7em and -0.2em, align=center] {1.\\ \tiny «create»} ([xshift=1em]l2_personnel_employee.north west);
\draw[consistency preservation] ([xshift=1em]l2_personnel_employee.north) -- node[left=0.2em] {2.} ([xshift=2.5em]l2_task_employee.south west);
\draw[consistency preservation] ([yshift=-2.5em]l2_personnel_employee.north east) -- node[below] {3.} ([yshift=-2.5em]l2_scheduling_employee.north west);

\draw[consistency preservation, dashed] ([xshift=-2em]l2_task_employee.south east) -- node[above right=0.3em and 0.2em, anchor=west] {4. \large \Lightning} ([xshift=-1em]l2_scheduling_employee.north);
\draw[consistency preservation, dashed] ([yshift=-1.5em]l2_scheduling_employee.north west) -- node[above] {5.}  ([yshift=-1.5em]l2_personnel_employee.north east);
\draw[consistency preservation, dashed] ([xshift=2em]l2_personnel_employee.north) -- node[right=0.2em] {6.} ([xshift=3.5em]l2_task_employee.south west);


% LEVEL 3 ERROR
\coordinate (failure_l3_anchor) at ([xshift=4.3*\classdistance]failure_l2_anchor);

\umlobjectvarwidth[, consistency changed element, fill=white, below=0 of failure_l3_anchor, anchor=north]{l3_task_employee}{}{
: Employee}{
name="Alice Do"
}{\objectwidth}

\umlobjectvarwidth[, user changed element, fill=white, below left=1.2*\classdistance and 1.15*\classdistance of l3_task_employee.north, anchor=north] {l3_personnel_employee}{}{
: Employee}{
firstName="Alice"
lastName="Do"
}{\objectwidth}

\umlobjectvarwidth[, consistency changed element, fill=white, below right=0.5em and 1.9*\classdistance of l3_personnel_employee.center, anchor=south] {l3_scheduling_employee}{}{
: Employee}{
name="Alice Do"
}{\objectwidth}

\umlobjectvarwidth[, consistency changed element, fill=white, below right=1em and 1.9*\classdistance of l3_personnel_employee.center, anchor=north] {l3_scheduling_employee_duplicate}{}{
: Employee}{
name="Alice Do"
}{\objectwidth}

\umlhuman{l3_human}{at ([xshift=1em, yshift=4.5em]l3_personnel_employee.north west)}{user changed element}{}{0.5}
\draw[user change] ([xshift=1em,yshift=3.3em]l3_personnel_employee.north west) -- node[above right=-0.7em and -0.2em, align=center] {1.\\ \tiny «create»} ([xshift=1em]l3_personnel_employee.north west);
\draw[consistency preservation] ([xshift=1em]l3_personnel_employee.north) -- node[left=0.3em] {2.} ([xshift=2em]l3_task_employee.south west);
\draw[consistency preservation] ([xshift=-2em]l3_task_employee.south east) -- node[above right=-0.5em and 0.2em] {3.} ([xshift=-1em]l3_scheduling_employee.north);
\draw[consistency preservation] ([yshift=1em]l3_personnel_employee.south east) -- node[below=0.2em] {4.} ([yshift=-1em]l3_scheduling_employee_duplicate.north west);

\begin{pgfonlayer}{bg}
    \modeltypebgtextabove{4.8em}{1.5*\classwidth}{above=1.7em of mm_task_employee.north, anchor=north}{Task Management}
    \modeltypebgtextbelow{6.4em}{1.5*\classwidth}{below=1.7em of mm_personnel_employee.south, anchor=south}{Personnel Data\vphantom{g}}
    \modeltypebgtextbelow{5.2em}{1.5*\classwidth}{below=1.7em of mm_scheduling_employee.south, anchor=south}{Scheduling}

    \modeltypebgtextabove{4.1em}{1.27*\objectwidth}{above=1.7em of l2_task_employee.north, anchor=north}{Task Management}
    \modeltypebgtextbelow{5.6em}{1.27*\objectwidth}{below=1.7em of l2_personnel_employee.south, anchor=south}{Personnel Data\vphantom{g}}
    \modeltypebgtextbelow{4.8em}{1.27*\objectwidth}{below=1.7em of l2_scheduling_employee.south, anchor=south}{Scheduling}
    
    \modeltypebgtextabove{4.1em}{1.27*\objectwidth}{above=1.7em of l3_task_employee.north, anchor=north}{Task Management}
    \modeltypebgtextbelow{5.6em}{1.27*\objectwidth}{below=1.7em of l3_personnel_employee.south, anchor=south}{Personnel Data\vphantom{g}}
    \modeltypebgtextbelow{8.3em}{1.27*\objectwidth}{below=1.7em of l3_scheduling_employee_duplicate.south, anchor=south}{Scheduling}
\end{pgfonlayer}


% LABELS AND LINES

\coordinate (y_verticalline) at ([yshift=2.4*\classdistance]l2_personnel_employee.west);
\draw[draw=gray, very thin] ([xshift=-0.05*\objectwidth]y_verticalline-|l2_personnel_employee.west) -- node[below=0.2em] (l2_label) {Level 2 Mistake \textit{(Opt. 1)}} ([xshift=-1em]y_verticalline-|legend.west);

%\node[below=0.4em, anchor=south west, align=left, font=\mediumfont] (l2_label) {Level 2 Mistake \textit{(Opt. 1)}}; 

%{Consistency Preservation with
%\node[above=2em of failure_l2_anchor, align=center, font=\smallerfont {Consistency Preservation with \textit{Opt. 1}\\ \textit{(Error on Level 2)}};

\draw[draw=gray, very thin] ([xshift=0.05*\objectwidth]y_verticalline-|l3_scheduling_employee.east) -- node[below=0.2em] (l3_label) {Level 3 Mistake \textit{(Opt. 2)}} ([xshift=1em]y_verticalline-|legend.east);
%\draw[dashed] ([yshift=2.4*\classdistance]l2_personnel_employee.west) -- node[below=0.2em, font=\mediumfont] (l2_label) {Level 2 Mistake \textit{(Opt. 1)}} ([xshift=-1em, yshift=2.4*\classdistance]l2_personnel_employee.west-|legend.west);

%\node[anchor=south east, align=left, font=\mediumfont] (l3_label) at (l2_label.south-|l3_scheduling_employee.east) {Level 3 Mistake \textit{(Opt. 2)}};
%\node[above=2em of failure_l3_anchor, align=center, font=\smallerfont] {Consistency Preservation with \textit{Opt. 2}\\ \textit{(Error on Level 3)}};
%\draw[dashed] ([yshift=0.4em]l3_scheduling_employee.east|-l3_label.north) -- ([xshift=1em, yshift=0.4em]l3_label.north-|legend.east);

\draw[draw=gray, very thin] ([yshift=-1em]legend.south) -- ([yshift=-2.8*\classdistance]legend.south);

\end{tikzpicture}


%    \includegraphics[angle=270, width=\textwidth]{figures/mistakes_examples_employee.pdf}
    \caption{Consistency Constraints on Metamodel Extract (top), Failure due to Mistake on Modularization Level (left), Failure due to Mistake on Operationalization Level (right)}
    \label{fig:correctness:mistake_effects_example}
\end{figure}

%We associated the mistakes presented in the previous section with the specification level they can occur on. Additionally, we summarized potential failures that can occur when executing final consistency preservation specification in the section before.
Although all failures occur during operationalization, the mistakes that lead to them can also be made at a higher specification level, such as the modularization or global level.
More importantly, each type of failure can be traced back to specific types of mistakes, or, vice versa, specific mistakes lead to specific kinds of failures.
\autoref{fig:correctness:mistake_effects_example} shows extracts of the three metamodels from our motivation, as well as consistency constraints between them.
There are two options for a constraint between personnel data and scheduling system.
The first option is contradictory to the one defined between personnel data and task management system, as already discussed in \autoref{chap:properties:levels}.
This demonstrates that contradictory constraints are a typical fault that can result from contradicting modular knowledge, when different persons define such constraints independently.
If, nevertheless, such a contradictory consistency specification is operationalized to a consistency preservation specification, the propagation of changes may never terminate.
This is shown in the left scenario in \autoref{fig:correctness:mistake_effects_example}, where
%Due to the contradicting constraints, 
the name is replaced repeatedly in an \emph{alternating loop} as indicated by the dashed arrows.

If no mistakes are made on the modularization level, so that no contradictions exist, %which especially means that the consistency specifications are free of contradictions, 
missing matching of equal elements in the consistency preservation specifications can still lead to duplicate element instantiations.
With the second option for the constraint in \autoref{fig:correctness:mistake_effects_example}, %no contradicting constraints and thus 
no mistakes on modularization level exist.
However, a missing matching of elements %in the consistency preservation specification 
can lead to the situation shown in the right scenario of \autoref{fig:correctness:mistake_effects_example}, in which two employees are instantiated across different transformation paths.
%We also demonstrated in the example that missing matching of equal elements in the consistency preservation specifications can lead to duplicate instantiations of elements.

These were two of several causal chains for mistakes and faults to resulting failures.
We give a full overview of those dependencies in \autoref{fig:correctness:categorization}.
Missing constraints lead to deterministic inconsistencies, because such inconsistencies are not modelled and thus resolved.
Additional consistency constraints do not lead to any actual failures, but reduce the set of consistent models. 
The only consequence is that consistency preservation does not consider models that would actually be consistent.
Contradicting constraints, which can arise from a faulty modularization, are more severe, as we have seen in the example:
They can either lead to non-deterministic inconsistencies, e.g., depending on the execution order of consistency preservation specifications, or to loops that alternate or diverge values.
Finally, the missing element matching at the operationalization level can lead to multiple instantiations, as we have seen in the example, or multiple insertions. %, if elements are added to a multi-valued reference multiple times.

%\todoHeiko{Tun wir das wirklich? Oder nur ein Level?}
%In the following, we discuss strategies to avoid mistakes at the different levels.
%Afterwards, we evaluate whether our identified categorizes of mistakes, faults and failures and their dependencies are actually correct.

%Categorize the detected Causes/Mistakes into three categories, which map to the steps identified in the first subsection. Two categories have to be resolved by user and, especially, are only resolvable in the moment when concrete transformations are combined (explain why!). One category can be solved by applying appropriate pattern, explained in the next section. Each category resolution is an assumption of the next (e.g. pattern matching does not make any sense when transformations are incompatible or at least the failures than can occur may differ).

\end{copiedFrom} % ICMT