%%% 
%%% SUMMARY
%%%
\section{Summary}

%\todo{Say that this is only an engineering consideration and not formally proven to be compliant with the partial-consistency-improving definition. Although the considerations are derived from that definition, we do not get a formal guarantee of specific properties. We therefore evaluate whether further problems that we did non consider here can occur.} -- Did that in the previous section

In this chapter, we have discussed how synchronizing transformations, as required in transformation networks, can be defined with existing transformation languages.
To this end, we defined synchronizing bidirectional transformations as an extension of bidirectional transformations specified in transformation languages.
%They need to fulfill the specific property of being \emph{partial-consistency-improving}.
We have formally proven that these transformations always terminate consistently and have equal expressiveness than synchronizing transformations.
Finally, we have identified properties and proposed an algorithm to be implemented by a transformation specified in a transformation language to be synchronizing.
We close this chapter with the following central insight.

\begin{insight}[Synchronization]
    Synchronizing transformations, as required in transformation networks, process pairs of models that both may have been and need to be modified.
    In contrast, ordinary bidirectional transformations consist of two unidirectional consistency preservation rules, each of them accepting changes in one model and updating the other.
    We have shown that if changes have been performed to both models, the consistency preservation rules cannot be sequenced such that they produce consistent results.
    By requiring that a bidirectional transformation fulfills a notion of being \emph{partial-consistency-improving}, we were able to define an execution algorithm for it that delivers consistent models after a finite number of execution steps.
    %To achieve that, a bidirectional transformation has to fulfill a defined notion of \emph{partial consistency improvement}.
    In return, we were able to formally prove that such transformations have equal expressiveness than synchronizing transformations as required for transformation networks. %, and that they always terminate consistently.
    Finally, we found that a transformation developer needs to consider only few situations explicitly to make a bidirectional transformation partial-consistency-improving. 
    The most important situation is that a transformation creates elements that already exist, because another transformation already created them, for which we provide an algorithm to avoid issues due to duplicate element creation already by construction.
    % Having ordinary transformations and consistency preservation rules for both directions, executing only one or one after another does not necessarily lead to a consistent result, thus the transformations are not correct in context of a transformation network when both models may have been modified.
    % We, however, found that their sequential execution leads to a consistent result for all possible combinations of changes, if identity of elements is handled correctly.
    % In contrast to ordinary incremental transformation, which assume that elements were created by the user or the transformation itself, in a transformation network other transformation may have already created appropriate elements.
    % In consequence, a transformation needs to identify if an element already exists upon its creation.
    % To achieve that, it needs to define key information for identifying that element.
    % With that addition, executing ordinary incremental transformations in both directions, each of the consistency preservation rules being correct, the emulated synchronizing transformation is correct.
    In consequence, synchronizing transformations can be constructed with existing transformation languages by fulfilling an additional property for which we provide a constructive strategy and without knowing about other transformations to combine them with.
\end{insight}

%\todo{Is this true? We can have cycles there, so it may not be correct. Just consider the confluence case?}