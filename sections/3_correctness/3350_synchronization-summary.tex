%%% 
%%% SUMMARY
%%%
\section{Summary}

\begin{insight}[Synchronization]
    When constructing synchronizing transformations, both models may have been and need to be modified in contrast to ordinary bidirectional transformations, which update only one of the models.
    Having ordinary transformations and consistency preservation rules for both directions, executing only one or one after another does not necessarily lead to a consistent result, thus the transformations are not correct in context of a transformation network when both models may have been modified.
    We, however, found that their sequential execution leads to a consistent result for all possible combinations of changes, if identity of elements is handled correctly.
    In contrast to ordinary incremental transformation, which assume that elements were created by the user or the transformation itself, in a transformation network other transformation may have already created appropriate elements.
    In consequence, a transformation needs to identify if an element already exists upon its creation.
    To achieve that, it needs to define key information for identifying that element.
    With that addition, executing ordinary incremental transformations in both directions, each of the consistency preservation rules being correct, the emulated synchronizing transformation is correct.
    In consequence, synchronizing transformations can be constructed with existing transformation languages not considering synchronization and without knowing about other transformations to combine them with.
\end{insight}

\todo{Is this true? We can have cycles there, so it may not be correct. Just consider the confluence case?}