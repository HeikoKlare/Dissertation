\chapter{Constructing Synchronizing Transformations
   \pgsize{50 p.}
}
\label{chap:synchronization}

%\todo{Vermeidbarkeit hängt insb. auch von der verwendeten Sprache ab. Z.B. sind Mappings / QVT-R analysierbar, aber Reactions / QVT-O nicht.}

%\todo{Ergebnis sollte sein: Synchronisierungseigenschaft ist erreichbar per Konstruktuion}

%\todo{Diskutieren, dass bei unidirektionalen CPRs sogar nicht berücksichtigt wird, dass die Rückrichtung noch ausgeführt wird. Normalerweise reicht durch die Korrektheit eine Richtung. Wenn man die Rückrichtung ausführt, also auf die Änderungen, reagieren lässt, die schon zu Konsistenz geführt haben, muss die Transformation aber auch damit klar kommen, dass das keine Nutzeränderungen sind, für die sie Konsistenz herstellen muss, sondern die korrespndieren Elemente möglicherweise schon da sind (hierfür sind die expliziten Checks).}

\mnote{Transformation correctness}
Transformations are the central artifacts of which a transformation network is composed.
We have introduced them as \emph{synchronizing transformations} in \autoref{def:synchronizingtransformation}, which are combinations of consistency relations with a consistency preservation rules that preserves them.
Correctness of such a transformation was then defined as the property of the consistency preservation rule to preserve consistency of given models according to the consistency relations (cf.\ \autoref{def:synchronizingtransformationcorrectness}).
In theory, a correct transformation can simply be achieved by adhering to that definition.

%Even if we consider that one may only define a consistency preservation rule, which then induces the consistency relation as its image, thus being correct by construction, the rule may not behave as expected.

% Problem statement: Problem is a practical one, not a theoretical one.
% Ordinary transformations may be correct if used on their own (according to Stevens), but in context of a transformation network, when other transformation have modified the "target" model as well, they do not lead to a consistent result anymore, i.e., they are not correct.
% Start with example of duplicated creation and overwrite.

\mnote{Unidirectional rules in transformation languages}
Using existing transformation languages, the defined transformations will, however, not follow the definition of a synchronizing transformation.
Transformation languages usually allow the specification of unidirectional consistency preservation rules, i.e., rules that restore consistency by updating one model if the other was modified.
Even if transformation languages allow bidirectional specifications, they still derive unidirectional consistency preservation rules from such a specification, such as forward and backward transformation (which may be incremental or not) derived from \glspl{TGG} rules~\cite{leblebici2014IncrementalTGGSurvey-GTVMT}.
In the following, we refer to such transformations as \emph{ordinary transformations} and give a more precise definition of them later on.
Synchronizing transformation, as we assume in transformation networks, are able to process changes made in both models and, in turn, also produce changes for both models.
This is an inevitable property in transformation networks, because both models involved in a transformation may have been modified due to different sequences of transformations having modified both of them.
The case that developers modify multiple models concurrently is sometimes also referred to as \emph{synchronization}, although the term is sometimes even used for the simple case of incremental updates.
If we consider that scenario, we will refer to it as \emph{concurrent editing} to avoid confusion.

\mnote{Ordinary transformations as synchronizing ones}
In this chapter, we aim to close this gap between synchronizing transformations as required in transformation networks and ordinary transformations with unidirectional consistency preservation rules used by transformation languages.
We investigate which requirements such an ordinary transformation has to fulfill to emulate a synchronizing transformation and thus be used in a transformation network.
This chapter thus constitutes our contribution \autoref{contrib:correctness:synchronization}, which consists of four subordinate contributions: a discussion of the formal basis for the gap between synchronizing and ordinary transformations; a discussion of different strategies to combine unidirectional consistency preservation rules of ordinary transformations to emulate a synchronizing transformation; a derivation of requirements to ordinary transformations to be synchronizing; and finally techniques to ensure that ordinary transformations fulfill these requirements.
It answers the following research question:

\researchquestionrepeat{rq:correctness:synchronization}

\mnote{Benefits due to reusability}
The benefit of enabling the definition of ordinary transformations that can be used as synchronizing ones instead of providing an approach or language for the specification of synchronizing transformations is that existing and well-researched transformation languages and knowledge about them can be reused.
Additionally, it reduces the complexity, because the definition of two unidirectional consistency preservation rules is less cumbersome than the definition of a single synchronizing transformation, which has to consider all possible combinations of changes in two models.
We will see that this is founded by the insight that only few combinations of changes are problematic and have to be considered explicitly.

\mnote{Publication of contributions}
We have already published parts of the contributions in this chapter in \cite{klare2018docsym} and \cite{klare2019icmt}.
The identification of essential issues when constructing synchronizing transformations from ordinary transformations defined in existing transformation languages has already been discussed in \cite{klare2018docsym}.
In the Master's thesis of \textcite{syma2018ma}, which was supervised by the author of this thesis, several issues in transformation networks have been identified, and for the category of changes arising from the combination of unidirectional transformation specifications a constructive solution has been proposed.
We have published that work in \cite{klare2019icmt} and present the results especially in \autoref{chap:synchronization:achieving}.

%%%
%%% FORMAL GAP IDENTIFICATION
%%%
\section{Gap between Synchronizing and Ordinary Transformations}

\todo{Exchange "ordinary" with "unidirectional"}
We have already introduced that there is both a formal and practical gap between synchronizing transformations, as we have defined as a component of transformation networks, and ordinary transformations, which are unidirectional and non-synchronizing, as used by most transformation languages.
In the following, we first give an example for faulty behavior if we simply used ordinary transformations in a transformation network.
Afterwards, we discuss options to sequence ordinary transformations and finally come up with a precise description of the formal gap and a practical approach to close it.

%%
%% MOTIVATING FAULT EXAMPLE WITH ORDINARY TRANSFORMATIONS
%%
\subsection{Behavior of Ordinary Transformations in Networks}
We have already sketched the example of creating a class in UML and Java after adding a component to a \gls{PCM} model in \autoref{chap:introduction:challenges:correctness:synchronization}.
In that scenario, it was possible that for a created \gls{PCM} component first a UML class is generated, which is then transformed into a Java class.
Additionally, the transformation between \gls{PCM} and Java creates another Java class, as it does not consider that there may be another transformation that already created that class.
Such scenarios can lead to the duplication of elements, or, in case of Java, an overwrite of an already created element, because the source file of the class will be placed at the same location in the file system.
Overwriting the previously created file may also overwrite and thus remove information that was already added to that class by the transformations across UML.

\begin{figure}
    \centering
    \includegraphics[width=0.8\textwidth]{figures/correctness/synchronization/duplicate_creation_example.png}    
    \caption{Duplicate creation of a resident by two sequences of consistency preservation rules}
    \label{fig:synchronization:duplicate_creation_example}
\end{figure}

An analogous example can be given for the running example of persons, employees and residents depicted in \autoref{fig:networks:three_persons_example}.
We consider the consistency relations $\consistencyrelation{R}{PE}, \consistencyrelation{R}{ER}$ and $\consistencyrelation{R}{PR}$.
As discussed in \autoref{chap:compatibility}, these relations are compatible, thus for any given person, employee or resident, there is a consistent set of models containing it.
Thus, the relations do not prevent transformations from finding consistent models whenever a person, employee or resident is added.
If we now consider ordinary transformations with unidirectional consistency preservation rules, they react to the changes in one model and update another accordingly.
In case of adding a person, this may look as depicted in \autoref{fig:synchronization:duplicate_creation_example}.
For each of the given consistency relations, we assume unidirectional consistency preservation rules that preserve consistency according to them.
They especially create an employee for each added person, and a resident for each created employee and person, respectively.
Since the transformations assume the models to be consistent before applying the changes, they always add a corresponding element when one of the elements is added.
This leads to the situation that both $\consistencypreservationrule{\consistencyrelation{CR}{ER},\rightarrow}$ as well as $\consistencypreservationrule{\consistencyrelation{CR}{PR},\rightarrow}$ create a resident upon creation of a person.
In consequence, there exist two residents with the same name, which does not fulfill the consistency relations.

It is our goal to find out how such a situation can be avoided by proper definition of consistency preservation rules in existing transformation languages.
A simple solution in this example would have been to look for the existence of elements to create first.
This can either be done by using a trace model, which most existing transformation language use to store corresponding elements, or by searching for an appropriate element in the other model.
Using a trace model, however, has some drawbacks and pitfalls, which we will investigate later.


% \begin{copiedFrom}{DocSym}

% % \section{Binary Transformation Interoperability}

% % Multi-model consistency preservation can be a achieved by combining binary transformations to graphs, %of transformations, 
% % with the transformations being executed transitively.
% % Since all binary transformations are developed independently of each other, it is necessary that they interoperate properly in a \emph{non-intrusive} way, thus without the necessity for the developer to understand and modify them, which we refer to as \emph{black-box combination}.

% Even under the assumption that, in contrast to our introductory motivation, all specifications are free of contradictions, it is easy to see that problems arise when combining binary transformations by transitively executing them.
% For example, consider the relations in \autoref{fig:prologue:binary_combination_example}.
% If a component is added to the \ac{ADL}, causing a \ac{UML} class creation due to \ref{fig:prologue:binary_combination_example:R1}, which in turn causes a Java class creation due to \ref{fig:prologue:binary_combination_example:R2}, the transformation for relation \ref{fig:prologue:binary_combination_example:R3} does not know that an appropriate class was already created, if the transformations are treated as black boxes.
% Consequently, the transformation will create the same class again, which may override the existing one, depending on the implementation and execution order.
% A simple solution for this example would be to have all transformations use a common trace model and check for existing elements before creating them in a transformation.
% Nevertheless, independently developed transformations will usually not assume that %this only applies if the transformation considers possibly preexisting transformation results, which it will not do in general if it does not assume other transformations to create corresponding elements.
% other transformations may already have created corresponding elements.
% Additionally, the trace model must allow the transformation engine to retrieve transitive traces.
% However, it is unclear if transitive resolution of traces can always be performed, as it can depend on whether the transitive trace belongs the considered consistency relation or another.
% %If, in another scenario, the transformation in the example was actually supposed to create an additional class, it would have to ignore the existing trace.

% As can be seen in the example, especially the correct handling of trace information in interdependent transformations has to be researched.
% This applies not only to element creations, but also other change types, such as attribute or reference changes, especially if they are multi-valued.
% In our thesis, we will therefore apply transitively executed binary transformations in different case studies to identify these and potential further problems.
% We then want to come up with a catalog of such problems %preventing the black-box combination of transformations 
% together with solution patterns for them.
% For example, to avoid duplicate element creations, a simple pattern could be to always check for already existing traces for that consistency relation in the transformations.
% In consequence, the integration of those patterns into a transformation language or the application of them as a transformation developer is supposed to achieve black-box combinability of the transformations.

% \end{copiedFrom} % DocSym


\subsection{Consistency Preservation for Fine-grained Relations}

\mnote{Consistency preservation defined for model-level consistency relations}
In our definition of consistency preservation rules in \autoref{def:consistencypreservationrule}, we used the coarse-grained notion of \modellevelconsistencyrelations, which describes consistency between two models in terms of a single relations.
In consequence, such a \modellevelconsistencypreservationrule ensures consistency to a single consistency relation.

\mnote{Fine-grained consistency relations allow to define relation between unidirectional preservation rules}
In \autoref{chap:compatibility:formal_notion}, we discussed that consistency relations can be considered in a fine-grained way that is able to reflect different notion of consistency in both directions of a relation.
We thus refined the notion of consistency relations in \autoref{def:consistencyrelation} to be defined at the level of model elements rather than complete models.
%This fine-grained notion of consistency does also fit well to how specifications in transformation languages consider consistency, as they define rules that relate only some classes by relations or routines to preserve their consistency.
We thus base further consideration on consistency preservation rules on such fine-grained consistency relations.
Nevertheless, we did also discuss in that chapter that all fine-grained relations can also be translated into \modellevelconsistencyrelations, thus all insights we already had for those model-level relations still apply to the considerations regarding fine-grained ones.

\mnote{Transformation languages use fine-grained relations and preservation rules}
This fine-grained notion of consistency does also fit well to how specifications in transformation languages consider consistency.
They allow to define rules that relate only some classes by relations, conforming to fine-grained consistency relations, from which then fine-grained consistency preservation rules are derived, or they directly allow to define routines to preserve consistency between specific classes.
These rules are often called \emph{transformation rules} and composed to a transformation that consists of multiple such rules, each encoding a consistency relations and a preservation rule for it.
%We will, however, stick to the coarse-grained notion of consistency preservation rules, because, first, it is difficult to describe how such fine-grained consistency preservation rules can be composed, and second, the coarse-grained notion is sufficient for our considerations anyway.

\mnote{Stick to coarse-grained notion of preservation rules}
It may happen easily that the execution of one transformation rule leads to the violation of the consistency relation of another one, which introduced dependencies between the individual transformation rules.
Thus, a combination of such transformation rules to a transformation has to ensure correctness, i.e., that the consecutive execution of the rules leads to a consistent state of the models.
Languages such as \gls{QVTR} and \gls{QVTO} therefore precisely specify in which order single transformation rules can and need to be executed. \todo{Add reference}
It is also a dedicated topic of research to ensure that the rules of a single transformation conform to each other \todo{Add references}, thus we assume that a transformation has that property.
To avoid the necessity of specifying this conformance property for transformation rules, we stick to the existing notion of coarse-grained consistency preservation rules, as it is sufficient for our considerations.

\mnote{New transformation notion based on fine-grained consistency relations}
In consequence, from now we will consider a synchronizing transformation as a set of fine-grained consistency relations according to \autoref{def:consistencyrelation} and a consistency preservation rule that preserves consistency according to the set of relations $\consistencyrelationset{CR}$ rather than a single \modellevelconsistencyrelation $\consistencyrelation{CR}{}$.
The consistency preservation rule $\consistencypreservationrule{\consistencyrelationset{CR}}$ and also the complete transformation are thus still considered correct if applying it to a consistent pair of models and changes to them, applying the resulting changes to the models again delivers a pair of models that is consistent to all consistency relations, i.e.:
\begin{align*}
    &
    \forall \model{m}{1} \in \metamodelinstanceset{M}{1}, \model{m}{2} \in \metamodelinstanceset{M}{2}, \change{\metamodel{M}{1}} \in \changeuniverse{\metamodel{M}{1}}, \change{\metamodel{M}{2}} \in \changeuniverse{\metamodel{M}{2}} : \tupled{\model{m}{1},\model{m}{2}} \consistenttomath \consistencyrelationset{CR} \\
    & \formulaskip
    \land \exists \change{\metamodel{M}{1}}' \in \changeuniverse{\metamodel{M}{1}}, \change{\metamodel{M}{2}}' \in \changeuniverse{\metamodel{M}{2}} : \tupled{\change{\metamodel{M}{1}}', \change{\metamodel{M}{2}}'} = \consistencypreservationrule{\consistencyrelationset{CR}}(\model{m}{1}, \model{m}{2}, \change{\metamodel{M}{1}}, \change{\metamodel{M}{2}}) \\
    & \formulaskip\formulaskip
    \Rightarrow \tupled{\change{\metamodel{M}{1}}'(\model{m}{1}),\change{\metamodel{M}{2}}'(\model{m}{2})} \consistenttomath \consistencyrelationset{CR}
\end{align*}
Note that being consistent to all fine-grained consistency relations is equivalent to being consistent to the single \modellevelconsistencyrelation induced by the fine-grained relations.

% In our definition of consistency preservation rules in \autoref{def:consistencypreservationrule}, we used the coarse-grained notion of \modellevelconsistencyrelations and, accordingly, defined \modellevelconsistencypreservationrules that consider models as whole.
% We refined the notion of consistency relations in \autoref{def:consistencyrelation} to be defined at the level of model elements rather than complete models.
% This fine-grained notion of consistency fits well to how specifications in transformation languages consider consistency, as they define rules that relate only some classes by relations or routines to preserve their consistency.
% In the following, we thus also stick to this fine-grained notion of consistency.
% We will, however, stick to the coarse-grained notion of consistency preservation rules given in \autoref{def:consistencypreservationrule}

% \subsection{Feingranulare CPR}
% Jede CPR stellt Konsistenz bzgl. einer unidirektionalen CR wieder her.
% Ein CPR darf keine Änderung machen, die eine andere CR verletzt.
% Schwierig, da CPR voneinander abhängen können (was wiederum dagegen spricht, dass eine CPR nicht eine andere CR verletzen darf), wie also Reihenfolge festlegen?
% Ist keine Option, insbesondere Annahme, dass sich CPR nicht gegenseitig beeinflussen nicht nachvollziehbar (obwohl in der Praxis dasselbe Probleme existiert, da CPR sowieso in feingranulare Regeln zerlegt werden, die natürlich widersprüchlich sein könnten).



\subsection{Unidirectional Consistency Preservation Rules}

\mnote{Formal definition of unidirectional consistency preservation rules}
Before discussing options how unidirectional consistency preservation rules can be used to emulate the behavior of synchronizing consistency preservation rules, we first need to define them to formally compare the two of them.
In contrast to a synchronizing consistency preservation rule as defined in \autoref{def:consisistencypreservationrule}, a unidirectional consistency preservation rule does only receive changes made to one of the two models and returns changes to the other models instead of receiving and returning changes to both.

\begin{definition}[Unidirectional Consistency Preservation Rule]
    \label{def:unidirectionalconsistencypreservationrule}
    Let $\metamodel{M}{1}, \metamodel{M}{2}$ be two metamodels and $\consistencyrelationset{CR}$ a set of consistency relations between elements of those metamodels.
    A \emph{unidirectional consistency preservation rule} $\consistencypreservationrule{\consistencyrelationset{CR}}$ for the relation set $\consistencyrelationset{CR}$ is a function:
    \begin{align*}
        \consistencypreservationrule{\consistencyrelationset{CR}} : (\metamodelinstanceset{M}{1}, \metamodelinstanceset{M}{2}, \changeuniverse{\metamodel{M}{1}}) \rightarrow \changeuniverse{\metamodel{M}{2}}
    \end{align*}
\end{definition}

\mnote{Preservation rules as defined in or derived in transformation languages}
This is how the consistency preservation rules defined in or derived from existing transformation languages operate.
They take two models and changes to one of them and generate changes for the other.
Most of them even directly apply the changes instead of returning a dedicated change artifact.

\mnote{Correctness of unidirectional rules analogous to common notions}
In addition, they usually assume the input models to be consistent and then ensure that applying the input and the output changes to the models, the resulting models are consistent again.
This conforms to the common notion of \emph{correctness} for consistency preservation rules, like for the state-based (rather than our delta-based) notion of consistency preservation rules defined in \autoref{stevens2010sosym}.
This is even compliant to the correctness notion that we defined for synchronizing consistency preservation rules in \autoref{def:consistencypreservationrulecorrectness}.
Thus, we define correctness of such a unidirectional consistency preservation rule as follows.

\begin{definition}[Unidirectional Consistency Preservation Rule Correctness]
    \label{def:unidirectionalconsistencypreservationrulecorrectness}
    Let $\consistencypreservationrule{\consistencyrelationset{CR}}$ be a \emph{unidirectional consistency preservation rule}.
    We call $\consistencypreservationrule{\consistencyrelationset{CR}}$ \emph{correct} if the resulting models when applying the generated changes are consistent to $\consistencyrelationset{CR}$ again:
    \begin{align*}
        &
        \forall 
        \model{m}{1} \in \metamodelinstanceset{M}{1}, 
        \model{m}{2} \in \metamodelinstanceset{M}{2},
        \change{\metamodel{M}{1}} \in \changeuniverse{\metamodel{M}{1}} : \\
        & \formulaskip
        \tupled{\model{m}{1}, \model{m}{2}} \consistenttomath \consistencyrelationset{CR} \\
        & \formulaskip
        \land \exists 
        \change{\metamodel{M}{2}} \in \changeuniverse{\metamodel{M}{2}} :
        \change{\metamodel{M}{2}} = \consistencypreservationrule{\consistencyrelation{CR}{}}(\model{m}{1}, \model{m}{2}, \change{\metamodel{M}{1}}) \\
        & \formulaskip\formulaskip
        \Rightarrow
        \tupled{\change{\metamodel{M}{1}}(\model{m}{1}), \change{\metamodel{M}{2}}(\model{m}{2})} \consistenttomath \consistencyrelationset{CR}
    \end{align*}
\end{definition}


\subsection{Bidirectional Transformations}

In practice unidirectional CPR are defined as pairs for both directions. This is usually denoted as bidirectional transformations.
They properly work for both directions individually.
\todo{Define Bidirectional Transformation, being composed of two unidirectional ones. Define correctness for the case that only one model was modified.}


\subsection{Unidirectional Relations and Preservation Alignment}
A unidirectional consistency relations require preservation rules in both directions (add/delete)
\begin{itemize}
    \item A single unidirectional consistency relation may impose preservation rules in both directions: For each employee, a resident is required, but not every resident needs to be an employee. When then have only one unidirectional consistency relation describing that circumstance. We need, however, consistency preservation rules in both directions, because if, for example, a resident is removed, the employee needs to be removed as well, whereas a resident has to be added whenever an employee is added.
    \item With ordinary unidirectional rules, we then have that after a change to model 1, the unidirectional rule has to change model 2 so that they are consistent to both unidirectional relations (and vice versa)
\end{itemize}
Konsequenz: Wir können nicht für die unidirektionalen Konsistenzrelationen jeweils die CPR in die Richtung angeben, sondern jede unidirektionale CPR muss immer die Konsistenzrelationen in beide Richtungen berücksichtigen.

Define mapping between change types and directionality of involved consistency relation (delete in opposite direction than add or modification). Make example at witness structures.


\paragraph{Synchronizing Transformation cannot be Unidirectional}
Es ist einfach zu sehen, dass synchronisierende Transformationen nicht so einfach unidirektional definiert werden können.
Wird m1 beliebig geändert und in m2 ein Element gelöscht, welches für Konsistenz zu m1 notwendig war, dann kann die unidirektionale CPR m2 nicht wieder so anpassen, dass es konsistent zu m1 ist, außer indem es das gelöschte Element wieder hinzufügt. Hier wäre es richtig das Element in m1 durch die gegenläufige CPR zu löschen. 
Das bedeutet aber das Korrektheit hier nicht definiert werden kann als die Eigenschaft Konsistenz bzgl. aller CR durch eine unidirektionale CPR herzustellen.

Das gilt allerdings nur, wenn man den Korrektheitsbegriff beibehält. Wir werden sehen, dass es andere Möglichkeiten gibt diesen Begriff einer unidirektionalen synchronisierenden Transformation zu definieren.


\subsection{Partial Definition of Consistency Preservation Rules}
1. Schon eine synchronisierende Transformation muss nicht für jede Eingabe definiert sein. Es könnte natürlich sein, dass sich gleichzeitige Änderungen nicht sinnvoll in Einklang bringen lassen. Theoretisch ist es aber möglich eine total definierte Funktion anzugeben, da immer ein beliebigen (im dümmsten Fall immer dasselbe / leere Modell) zurückgeliefert werden kann.

\todo{Conclude that an open issue is how to combine to unidirectional CPRs to emulate a synchronizing transformation, i.e. react to changes in both models.}



\section{Combining Unidirectional Consistency Preservation Rules}

\mnote{Make bidirectional transformation synchronizing}
We have introduced that bidirectional transformations, as we assume to be the notion for practically usable transformation specifications, can only be applied after changes to one model and update the other to restore consistency.
This induces a gap to synchronizing transformations, as required in transformation networks, which are able to accept changes made in both models and update both models to restore consistency.
To close this gap, we discuss options to combine the unidirectional consistency preservation rules of a bidirectional transformation, such that it considers changes made to both models and thus acts like a synchronizing transformation.


%%
%% OPTIONS TO COMBINE TRANSFORMATIONS / CPRS
%%
\subsection{Options for Combination}
\label{chap:synchronization:combination:options}

\mnote{Conflicting concurrent changes}
Existing work already proposed strategies to synchronize concurrent changes between two models.
This includes techniques for processing concurrent changes with \glspl{TGG}~\cite{hermann2012concurrentSynchronization-FASE,orejas2020IncrementalConcurrentSynchronization-FASE} and specific algorithms for a general notion of synchronizing transformations according to our definition~\cite{xiong2013SynchronizingConcurrentUpdates-SoSym,xiong2009parallelUpdates-ICMT}.
All these approaches, however, deal with the more general case that arbitrary changes may have been made.
This especially includes conflicting updates by one or more users, which need to be resolved and potentially require one of the changes to be reverted.

\mnote{Specific concurrent changes in transformation networks}
We are, however, in the situation that transformations do not perform arbitrary changes and that changes of other transformations may need to be revised but not reverted.
For example, it may be necessary to update an attribute value again, because the interval of consistent values of the currently executed transformation is smaller than the one of a transformation executed before.
It will, however, not be necessary to completely revert the modification of the attribute value, because the modification was necessary for another transformation to restore consistency.
Thus, the causal change for which consistency was restored would need to be reverted as well.
Finally, this would result in reverting a user change, which should never happen.

\mnote{Assumed compatibility of relations}
We assume the consistency relations of transformations to be compatible according to \autoref{def:compatibility}, which excludes contradictions that may prevent transformations from finding a consistent result for specific changes.
This assumptions reduces the potential conflicts that may occur when changes of different transformations need to be synchronized.

\mnote{Execution of both preservation rules}
A bidirectional transformation according to \autoref{def:bidirectionaltransformation} consists of two unidirectional consistency preservation rules.
We have discussed in \autoref{chap:synchronization:gap:alignment} that it is not possible to extend those consistency preservation rules to be synchronizing such that the execution of a single unidirectional consistency preservation rule restores consistency to all consistency relations after changes to both models.
In fact, it will be necessary to execute both preservation rules at least once to restore consistency.
Different options to apply the rules exist, each having individual benefits and drawbacks.

\mnote{Independent execution and merge}
We have sketched two scenarios for executing multiple consistency preservation rules in \autoref{chap:correctness:notions_consistency:preservation}, which can be transferred to the case of executing the two consistency preservation rules of a bidirectional transformation.
A first option is to independently apply the consistency preservation rules and then merge the results.
Imagine models $\model{m}{1}$ and $\model{m}{2}$ and changes $\change{\metamodel{M}{1}}$ and $\change{\metamodel{M}{2}}$ to them.
Applying the two unidirectional consistency preservation rules independently yields $\change{\metamodel{M}{2}}' = \consistencypreservationrule{\consistencyrelationset{CR}}^{\rightarrow}(\model{m}{1},\model{m}{2}, \change{\metamodel{M}{1}})$ and $\change{\metamodel{M}{1}}' = \consistencypreservationrule{\consistencyrelationset{CR}{}}^{\leftarrow}(\model{m}{2}, \model{m}{1}, \change{\metamodel{M}{2}})$.
It is, however, not guaranteed that $\tupled{\change{\metamodel{M}{1}}'(\change{\metamodel{M}{1}}(\model{m}{1})), \change{\metamodel{M}{2}}'(\change{\metamodel{M}{2}}(\model{m}{2}))}$ is consistent to $\consistencyrelationset{CR}$.
It is even not guaranteed that the changes, such as $\change{\metamodel{M}{1}}$ and $\change{\metamodel{M}{1}}'$, can be concatenated at all, since $\change{\metamodel{M}{1}}'$ was generated for $\model{m}{1}$ and not for $\change{\metamodel{M}{1}}(\model{m}{1})$.
As an example, $\change{\metamodel{M}{1}}$ may remove an element from $\model{m}{1}$, which $\change{\metamodel{M}{1}}'$ changes.
Even if the change is still defined for that modified model, the result may not be consistent, because the necessary change produced by $\consistencypreservationrule{\consistencyrelationset{CR}}^{\rightarrow}$ cannot be applied anymore.
Thus merging the changes of both consistency preservation rules does not necessarily yield a consistent result.

\mnote{Sequential execution}
Another option is to sequence the execution.
In a first step, we generate the change $\change{\metamodel{M}{2}}' = \consistencypreservationrule{\consistencyrelationset{CR}}^{\rightarrow}(\model{m}{1},\model{m}{2}, \change{\metamodel{M}{1}})$ as before.
Then, $\tupled{\change{\metamodel{M}{1}}(\model{m}{1}), \change{\metamodel{M}{2}}'(\model{m}{2})}$ is consistent due to correctness of $\consistencypreservationrule{\consistencyrelationset{CR}}^{\rightarrow}$.
Afterwards, we apply the second consistency preservation rule to the newly generated consistent models and the original change $\change{\metamodel{M}{2}}$ to $\model{m}{2}$, thus $\change{\metamodel{M}{1}}' = \consistencypreservationrule{\consistencyrelationset{CR}}^{\leftarrow}(\change{\metamodel{M}{2}}'(\model{m}{2}), \change{\metamodel{M}{1}}(\model{m}{1}), \change{\metamodel{M}{2}})$.
As a result, we receive $\tupled{\change{\metamodel{M}{1}}'(\change{\metamodel{M}{1}}(\model{m}{1})), \change{\metamodel{M}{2}}(\change{\metamodel{M}{2}}'(\model{m}{2}))}$, which is consistent to $\consistencyrelationset{CR}$.
This means that $\change{\metamodel{M}{2}}$ is not applied to $\model{m}{2}$ anymore, in which the changes were performed originally, but needs to be applied to $\change{\metamodel{M}{2}}'(\model{m}{2})$.
It is, again, unclear whether the change can be applied to that state, i.e., whether $\change{\metamodel{M}{2}}$ is defined for $\change{\metamodel{M}{2}}'(\model{m}{2})$.
However, if the changes are applicable, all original changes are reflected in the result.
In addition, the resulting models are consistent because of correctness of the consistency preservation rules.

\mnote{Sequential execution with less drawbacks}
Both discussed options have the drawback that they cannot guarantee to produce a result, as it is possible that the involved changes cannot be concatenated.
In addition, the first option of independently applying the consistency preservation rules and then merging the results cannot even guarantee that the resulting models are consistent if changes can be concatenated.
Thus, we only consider the second option of sequencing the execution of consistency preservation rules and further discuss it in the following.


%%%
%%% SEQUENCING OF CPRS
%%%
\subsection{Sequencing of Consistency Preservation Rules}
\label{chap:synchronization:combination:sequencing}

\begin{figure}
    \centering
    \newcommand{\hdistance}{12em}
\newcommand{\vdistance}{5em}

\begin{tikzpicture}[
    correspondence/.style={consistency relation, -}
]

\node[schematic model] (m1) {};
\node[left=0.1em of m1, anchor=east] {$\model{m}{1}$};
\node[schematic model, below=\vdistance of m1.center, anchor=center] (dm1) {};
\node[left=0.1em of dm1, anchor=east] {$\change{\metamodel{M}{1}}(\model{m}{1})$};
\node[schematic model, below=\vdistance of dm1.center, anchor=center] (ddm1) {};
\node[left=0.1em of ddm1, anchor=east] {$\change{\metamodel{M}{1}}'(\change{\metamodel{M}{1}}(\model{m}{1}))$};

\node[schematic model, right=\hdistance of m1] (m2) {};
\node[right=0.1em of m2, anchor=west] {$\model{m}{2}$};
\node[schematic model, below=\vdistance of m2.center, anchor=center] (dm2) {};
\node[right=0.1em of dm2, anchor=west] {$\change{\metamodel{M}{2}}'(\model{m}{2})$};
\node[schematic model, below=\vdistance of dm2.center, anchor=center] (ddm2) {};
\node[right=0.1em of ddm2, anchor=west] {$\change{\metamodel{M}{2}}(\change{\metamodel{M}{2}}'(\model{m}{2}))$};

\draw[correspondence] (m1) -- node[above] {consistent to $\consistencyrelationset{CR}$} (m2);
\draw[correspondence] (dm1) -- node[above] {consistent to $\consistencyrelationset{CR}$} (dm2);
\draw[correspondence] (ddm1) -- node[above] {consistent to $\consistencyrelationset{CR}$} (ddm2);

\draw[consistency execution, -latex] (m1) -- node[left] (d1) {$\change{\metamodel{M}{1}}$} (dm1);
\draw[consistency execution, -latex] (dm1) -- node[left] (dd1) {$\change{\metamodel{M}{1}}'$} (ddm1);
\draw[consistency execution, -latex] (m2) -- node[right] (d2) {$\change{\metamodel{M}{2}}'$} (dm2);
\draw[consistency execution, -latex] (dm2) -- node[right] (dd2) {$\change{\metamodel{M}{2}}$} (ddm2);

\draw[consistency execution, -latex] 
    ([xshift=0.3em]d1.east)
    --
    node[above] {$\consistencypreservationrule{\consistencyrelationset{CR}}^{\rightarrow}$}
    ([xshift=-0.3em]d2.west);
\draw[consistency execution, -latex]
    ([xshift=-0.3em]dd2.west)
    --
    node[above] {$\consistencypreservationrule{\consistencyrelationset{CR}}^{\leftarrow}$}
    ([xshift=0.3em]dd1.east);

\end{tikzpicture}
    %\includegraphics[width=0.8\textwidth]{figures/correctness/synchronization/sequencing_schema.png}
    \caption[Sequencing unidirectional consistency preservation rules]{Schema for sequencing unidirectional consistency preservation rules after concurrent changes. Circles denote model states, blue lines connect consistent models, and green lines with arrowheads denote the execution of changes or consistency preservation.}
    \label{fig:synchronization:sequencing_schema}
\end{figure}

\mnote{Changes affecting disjoint element sets}
The sequential application of original changes and execution of consistency preservation rules is depicted schematically in \autoref{fig:synchronization:sequencing_schema}.
It has two important properties. 
First, it ensures that all original changes are applied to the models and, second, it guarantees that the resulting models are consistent.
It is, however, only applicable in specific situations.
The optimal case, in which the approach is always applicable, is if $\consistencypreservationrule{\consistencyrelationset{CR}}^{\rightarrow}$ produces changes for the second model that affect a disjoint set of elements in $\consistencyrelationset{CR}$ compared to the original changes to the second model $\change{\metamodel{M}{2}}$.
If two changes affect completely disjoint sets of elements, they can obviously be consecutively applied.
It would then not even make a difference in which order they are applied.

\begin{figure}
    \centering
    \newcommand{\classwidth}{5em}
\newcommand{\objectwidth}{5.2em}
\newcommand{\hdistance}{(19em+0.4*\difftoafiveimage)}
\newcommand{\vdistance}{19em}

\begin{tikzpicture}

\umlclassvarwidth{employee_class}{}{Employee}{
    name
}{\classwidth}
\umlclassvarwidth[, right=1.4*\hdistance of employee_class.north west, anchor=north]{resident_class}{}{Resident}{
    name
}{\classwidth}
\node[above=0.2em of employee_class.north, anchor=south] {$\metamodel{M}{1}$};
\node[above=0.2em of resident_class.north, anchor=south] {$\metamodel{M}{2}$};

\draw[directed consistency relation]
    (employee_class)
    --
    node[pos=0, below right] {$e$}
    node[pos=1, below left] {$r$}
    node[above] {$\consistencyrelation{CR}{ER} = \setted{\tupled{e,r} \mid \mathvariable{e.name} = \mathvariable{r.name.toLower}}$}
    node[below, align=center] {$\consistencyrelationset{CR} = \setted{\consistencyrelation{CR}{ER}, \consistencyrelation{CR}{ER}^T}$}
    (resident_class);

% FIRST SCENARIO
\coordinate (begin_first) at ([yshift=-0.2*\vdistance]employee_class.north west);
\draw[lightgray] (begin_first) -- (begin_first-|resident_class.east);

\umlobjectvarwidth[, below right=0.11*\vdistance and 0.35*\hdistance of begin_first, anchor=north]{employee_first}{}{: Employee}{
    name = "alice"
}{\objectwidth}
\umlobjectvarwidth[, right=0.85*\hdistance of employee_first.center, anchor=center]{resident_first}{}{: Resident}{
    name = "alice"
}{\objectwidth}
\umlobjectvarwidth[, below=0.2*\vdistance of resident_first.center, anchor=center]{resident2_first}{}{: Resident}{
    name = "Alice"
}{\objectwidth}
\node[above=0.2em of employee_first.north, anchor=south] {$\model{m}{1}$};
\node[above=0.2em of resident_first.north, anchor=south] {$\model{m}{2}$};

\draw[consistency execution]
    ([xshift=-0.2*\hdistance]employee_first.west)
    --
    node[above, align=center] {$\change{\metamodel{M}{1}}$}
    (employee_first.west);
\draw[consistency execution]
    ([yshift=0.5em]employee_first.east)
    --
    node[above] {$\change{\metamodel{M}{2}}' = \consistencypreservationrule{\consistencyrelationset{CR}}^{\rightarrow}$}
    ([yshift=0.5em]resident_first.west);
\draw[consistency execution]
    ([xshift=0.2*\hdistance]resident2_first.east)
    --
    node[above, align=center] {$\change{\metamodel{M}{2}}$}
    (resident2_first.east);
\draw[correspondence]
    ([yshift=-0.5em]employee_first.east)
    --
    node[below, pos=0.4] {consistent to $\consistencyrelationset{CR}$}
    ([yshift=-0.5em]resident_first.west);
\draw[correspondence]
    (employee_first.south)
    |-
    node[below, pos=0.75] {inconsistent to $\consistencyrelationset{CR}$}
    (resident2_first.west);
\coordinate (cross_first) at ([xshift=-0.05*\hdistance]resident2_first.west);
\draw[correspondence]
    (cross_first)
    |-
    ([yshift=-1em]resident_first.west);
\filldraw[consistency related element] (cross_first) circle (0.15em);

\end{tikzpicture}
    %\includegraphics[width=0.9\textwidth]{figures/correctness/synchronization/non_transformability.png}
    \caption[Non-transformability in sequencing scenario]{Example for non-transformability when sequencing the application of unidirectional consistency preservation rules and concurrent changes.
    Blue lines without arrowheads connect elements that are \mbox{(in-)}consistent to $\consistencyrelationset{CR}$, and green lines with arrowheads indicate changes.}
    \label{fig:synchronization:non_transformability}
\end{figure}

\mnote{Issues when sequencing preservation rule application}
Unfortunately, the change $\change{\metamodel{M}{2}}'$ produced by $\consistencypreservationrule{\consistencyrelationset{CR}}^{\rightarrow}$ and the original one $\change{\metamodel{M}{2}}$ produced by other transformations do not necessarily affect disjoint sets of elements.
In that case, the two following problems can occur.
\begin{properdescription}
    \item[Non-Applicability:] The most obvious problem, which we have already discussed, is that the original change to the second model $\change{\metamodel{M}{2}}$ cannot be applied to the model changed by $\change{\metamodel{M}{2}}'$ as the result of $\consistencypreservationrule{\consistencyrelationset{CR}}^{\rightarrow}$. 
    This can, for example, happen when $\change{\metamodel{M}{2}}'$ removes an element that is affected by $\change{\metamodel{M}{2}}$.
    Since the element was changed in $\change{\metamodel{M}{2}}$, it is part of a condition element in another transformation that was executed before.
    As $\consistencypreservationrule{\consistencyrelationset{CR}}^{\rightarrow}$ removed that element, the condition element does no longer exist anyway, thus this removal has to be propagated back by the transformation that originally introduced the change $\change{\metamodel{M}{2}}$.
    In consequence, the modification in $\change{\metamodel{M}{2}}$ can simply be ignored.
    In the worst case, all elements affected by $\change{\metamodel{M}{2}}$ were removed by $\change{\metamodel{M}{2}}'$.
    Then, $\change{\metamodel{M}{2}}$ can be completely ignored, because all condition elements of the involved consistency relations were removed.
    Thus, we can always ensure that the changes, at least those that are still relevant, can still be applied.
    
    \item[Non-Transformability:] Even if the change $\change{\metamodel{M}{2}}$ can be applied to $\change{\metamodel{M}{2}}'(\model{m}{2})$, this does not guarantee that $\consistencypreservationrule{\consistencyrelationset{CR}}^{\leftarrow}$ is able to process the given change.
    In fact, this requirement applies to all changes, even including original user changes, but there are special circumstances in this situation that make the transformation prone to not being able to transform the changes.
    Whenever $\change{\metamodel{M}{2}}'$ adds condition elements that were already added by $\change{\metamodel{M}{2}}$, their concatenation can lead to a duplication of those elements.
    Consider the scenario depicted in \autoref{fig:synchronization:non_transformability} with consistency relations $\consistencyrelationset{CR} = \setted{\consistencyrelation{CR}{ER}, \consistencyrelation{CR}{ER}^T}$. 
    An employee \enquote{alice} is added by the original change to $\model{m}{1}$.
    The consistency preservation rule then generates an appropriate resident with the same name to fulfill the consistency relation.
    The original change to $\model{m}{2}$ adds a resident \enquote{Alice}, which was generated by another transformation, e.g., the one that created an appropriate person and changed the capitalization of the name.
    Applying this original change leads to two residents with different name capitalizations.
    Now it is impossible for $\consistencypreservationrule{\consistencyrelationset{CR}}^{\leftarrow}$ to generate a change $\change{\metamodel{M}{1}}'$ for the first model to restore consistency. The employee corresponds to both residents, as both fulfill the constraint of the consistency relation. 
    But there is no additional employee that could be added to achieve a unique mapping between corresponding elements.
    A synchronizing transformation would have been able to produce a consistent result by considering both original changes at once and then simply not performing any additional changes, as the originally added resident is already consistent to the originally added employee.
    In consequence, if the unidirectional consistency preservation rule had known that the resident was already added, it would not have performed any changes.
\end{properdescription}

\mnote{Non-transformable changes of users unavoidable}
As remarked before, the situation that certain changes cannot be processed by the consistency preservation rules cannot be avoided. 
If the user had added the second resident in the previous scenario, there would have also been no possibility for the consistency preservation rule to generate changes that restore consistency.
The difference is, however, that in this case it is fine that no result is found.
In case of the scenario discussed above, the original changes could have been reasonably processed to a consistent result if the unidirectional consistency preservation rule would have considered that there was already a change that restored consistency.

\mnote{Necessity to process inconsistent inputs}
In consequence, it is inevitable that consistency preservation rules need to be able to deal with the situation that the target model was already modified, such that the given models are not initially consistent.
This is necessary to reflect the changes that have already been made and to integrate them into consistency preservation.
In consequence, we finally have to relax our requirements for the input of consistency preservation rules to be able to consider the changes to both models.
This means that we need to make further requirements to the preservation rules, because we do not yet assume the consistency preservation rules to produce results for inputs that are not consistent.
We have already given examples for scenarios in which it is not possible to restore consistency by one unidirectional consistency preservation rule after changes in both models.

\mnote{Relaxed notion affecting number of executions}
Before we define a precise notion of further requirements to consistency preservation rules that accept inconsistent inputs, we first discuss how often it may be necessary to execute both consistency preservation rules to restore consistency, as this directly affects the requirements we have to define.


%%%
%%% NON-TERMINATION
%%%
\subsection{Execution Bounds}
\label{chap:synchronization:combination:bounds}

\mnote{Unidirectional preservation rules cannot always be correct}
Correctness of unidirectional consistency preservation rules ensures that after executing such a rule the resulting models are consistent.
It is easy to see that this correctness notion cannot be fulfilled for certain sets of consistency relation sets.
This is exemplified at the artificial scenario depicted in \autoref{fig:synchronization:multiple_unidirectional_execution}.
We consider two consistency relations $\consistencyrelation{CR}{1}$ and $\consistencyrelation{CR}{2}$ and their transposed relations, i.e., $\consistencyrelationset{CR} = \setted{\consistencyrelation{CR}{1}, \consistencyrelation{CR}{1}^T, \consistencyrelation{CR}{2}, \consistencyrelation{CR}{2}^T}$.
$\consistencyrelation{CR}{1}$ requires that for each \modelelement{A} an instance of \modelelement{B} exists that has the same value of $i$ incremented by $1$.
The only exception is that if $i$ in \modelelement{A} is $4$ (or any other arbitrary value), then no corresponding element \modelelement{B} is required.
$\consistencyrelation{CR}{2}$ requires that for each \modelelement{A} an instance of \modelelement{B} exists, which has the same value of $i$.
We want to define a bidirectional transformation of two unidirectional consistency preservation rules $\consistencypreservationrule{\consistencyrelationset{CR}}^{\rightarrow}$ for propagating changes in models with instances of \modelelement{A} to one with instances of \modelelement{B} and $\consistencypreservationrule{\consistencyrelationset{CR}}^{\leftarrow}$ to propagate changes in the opposite direction. 

\begin{figure}
    \centering
    \newcommand{\hdistance}{24em}
\newcommand{\vdistance}{9.5em}
\newcommand{\classwidth}{3em}

\begin{tikzpicture}[
    relation/.style={consistency relation, font=\footnotesize}
]

\umlclassvarwidth{A}{}{A\sameheight}{
i
}{\classwidth}

\umlclassvarwidth[,right=\hdistance of A.north, anchor=north]{B}{}{B\sameheight}{
i
}{\classwidth}

% CONSISTENCY RELATIONS
\draw[relation] (A.east) -- node[pos=0, above right] {$a$} node[below, align=left] {
    $\consistencyrelation{CR}{1} = \setted{\tupled{a,b} \mid a.i, b.i \ge 0 \land b.i = a.i + 1 \neq 5}$\\
    $\consistencyrelation{CR}{2} = \setted{\tupled{a,b} \mid a.i = b.i}$
} node[pos=1, above left] {$b$} (B.west);

\end{tikzpicture}
    %\includegraphics[width=0.7\textwidth]{figures/correctness/synchronization/multiple_unidirectional_execution.png}
    \caption[Multiple execution of consistency preservation rules]{Two consistency relations requiring multiple executions of unidirectional consistency preservation rules to find a consistent result.}
    \label{fig:synchronization:multiple_unidirectional_execution}
\end{figure}

\mnote{Example scenario}
Consider the following scenario: If an \modelelement{A} with $i = 0$ is added to an empty model, $\consistencypreservationrule{\consistencyrelationset{CR}}^{\rightarrow}$ cannot perform any changes in an (also empty) model with instances of \modelelement{B} that restore consistency.
Because of $\consistencyrelation{CR}{1}$, a \modelelement{B} with $i = 1$ has to be created, and because of $\consistencyrelation{CR}{2}$, a \modelelement{B} with $i = 0$ has to be created.
While this also fulfills $\consistencyrelation{CR}{1}^T$, the existence of \modelelement{B} with $i = 1$ requires the existence of an \modelelement{A} with $i = 1$ due to $\consistencyrelation{CR}{2}^T$.
Since $\consistencypreservationrule{\consistencyrelationset{CR}}^{\rightarrow}$ cannot modify the model with instances of \modelelement{A}, it is impossible for $\consistencypreservationrule{\consistencyrelationset{CR}}^{\rightarrow}$ to restore consistency in that case.

\mnote{Multiple executions leading to consistent result}
Allowing the consistency preservation rules to react to each other multiple times can, however, lead to a consistent result.
If $\consistencypreservationrule{\consistencyrelationset{CR}}^{\leftarrow}$ adds an \modelelement{A} with $i = 1$ in response to the previous execution of $\consistencypreservationrule{\consistencyrelationset{CR}}^{\rightarrow}$, all consistency relations except $\consistencyrelation{CR}{1}$ are fulfilled.
$\consistencypreservationrule{\consistencyrelationset{CR}}^{\rightarrow}$ can then create a \modelelement{B} with $i = 2$, which is iteratively processed by $\consistencypreservationrule{\consistencyrelationset{CR}}^{\leftarrow}$.
This process terminates as soon as $\consistencypreservationrule{\consistencyrelationset{CR}}^{\leftarrow}$ adds an \modelelement{A} with $i = 4$, as then $\consistencyrelation{CR}{1}$ is also fulfilled, because it does not require a corresponding \modelelement{B} for an \modelelement{A} with $i = 4$.

\mnote{Arbitrary high number of necessary executions}
We have seen that it is possible to execute unidirectional consistency preservation rules multiple times to achieve a consistent state and that it is not always possible to ensure consistency with only one execution of such a rule.
In fact, the number of necessary executions of consistency preservation rules can be arbitrarily high.
The value of $5$ in $\consistencyrelation{CR}{1}$ of the example can be exchanged by any value requiring an arbitrary high number of executions.
We may only circumvent this by requiring that $\consistencypreservationrule{\consistencyrelationset{CR}}^{\rightarrow}$ must perform changes such that $\consistencypreservationrule{\consistencyrelationset{CR}}^{\leftarrow}$ can then restore consistency with a single execution.
In our scenario, this would mean that $\consistencypreservationrule{\consistencyrelationset{CR}}^{\rightarrow}$ adds all instances of \modelelement{B} with $i \leq 4$.
Anyway, such a behavior requires a relaxation of the correctness requirement for consistency preservation rules, because the execution of $\consistencypreservationrule{\consistencyrelationset{CR}}^{\rightarrow}$ can never result in a consistent state.

\mnote{Preservation rules complete partial condition elements}
Additionally, it may be desired that elements of a consistency relation are created by a consistency preservation rule, although a condition element was only created partially so far.
In that case, the partial condition element has to be completed in one model in addition to the creation of the corresponding condition element in the other model.
Thus, changes in both models are required, which can only be achieved by executing both consistency preservation rules and accepting that executing the first one does not result in consistent models.
An example for such a scenario could be the consistency relation between a component in the \gls{PCM} and its realization as a package and a class in Java.
It may be desired that a package at a specific place, e.g., within a \enquote{components} package, or with a specific name, e.g., containing \enquote{Component}, in the Java code is identified as a component.
Creating such a package shall then lead to the creation of a component in the the \gls{PCM} model as well as of the implementation class in Java.
In that case, there is no complete condition element created in Java, because this would also require the existence of an appropriate class.
If the elements shall still be created, both models have to be changed.
Thus, the first consistency preservation rule introduces the \gls{PCM} component, which introduces an inconsistency between the models, as the corresponding Java class is missing.
This is then corrected by the consistency preservation rule in opposite direction adding the implementation class.

\mnote{Necessity of termination guarantee}
Finally, it is questionable whether such scenarios should be considered in the formal framework or if it should be up to a developer to implement such a scenario without having specific guarantees regarding termination of the consistency preservation rules or regarding consistency of the models after executing the rules a specific number of times.
Since we need to relax the requirement of consistency preservation rules to always produce consistent results after one execution in the synchronization scenario where both models have been modified, we will allow the consistency preservation rules to be executed more than once anyway.
Regarding the example in \autoref{fig:synchronization:multiple_unidirectional_execution}, if we started with an \modelelement{A} with $i = 6$ and let the consistency preservation rules operate as discussed above, i.e., always adding the elements with $i$ incremented by one, this process would never terminate.
We thus need to ensure that such an execution terminates.
Since the consistency preservation rules depend on each other, this will, however, be a property of the bidirectional transformation rather than the individual consistency preservation rule.


\subsection{Necessity for Synchronization Extension}

\mnote{Processing inconsistent input models}
In the previous subsections, we have discussed that after changes to two models, these changes and the ones produced by consistency preservation rules that restore consistency between these models cannot be sequenced in a way such that we receive consistent models in all cases the consistency preservation rules are able to handle.
We especially found that it is necessary for a unidirectional consistency preservation rule to consider the changes made to the model it is supposed to modify.
Thus, we need to enable consistency preservation rules to deal with the situation that the input models are inconsistent.
In our current definition, no behavior of a consistency preservation rule and the encapsulating bidirectional transformation for such a situation is defined.
Thus, we discuss an appropriate extension of bidirectional transformations that support this scenario of synchronization in the following section.

\mnote{Guaranteeing termination}
Additionally, we found that consistency preservation rules may need to be executed multiple times.
This is obviously necessary to make bidirectional transformations synchronizing, as they need to be able to change both models after both of them may have been modified.
Therefore, we consider how we can achieve execution bounds, such that the termination of multiple executions of the consistency preservation rules of a bidirectional transformation is guaranteed.



%%%%
%%%% VERSCHIEDENE VERSUCHE FÜR STRATEGIEN ZUR KOMBINATION VON CPRS
%%%%

% \subsection{Consideration at Condition Element Level}
% % CONSIDERATIONS DO NOT WORK PROPERLY
% \begin{itemize}
%     \item Unidirectional synchronizing: correct if applied to changes d1 to m1, but produces changes d2' that can be applied to d2(m2) as well, and vice versa produces changes d1' for changes d2 to model m2 that can be applied to d1(m1) as well. (Property: Sequentializability, produced changes can handle arbitrary other changes added before)
%     \item For each condition element in d1(m1), i.e., each element for which a consistency relation applies, and for each condition element in d2'(m2), we find exactly one corresponding element in the other model (this is what correctness means). Additionally, in d2'(d2(m2)) $\cap$ d2'(m2), i.e., those elements that are not affected by d2, we also find corresponding 
    
%     \item Take all condition elements in m1 and m2. Take all those for which still only one corresponding condition element exists between m1 and d2(m2), i.e., all the ones not affected by d2. For all of them present in d1(m1) and d2'(d2(m2)), there is still exactly one corresponding condition element. For the ones in d1(m1), which were not in m1, there is a corresponding element in d2'(d2(m2)) and for the ones in d2'(m2), which are also present in d2'(d2(m2)), there is one in d1(m1).
%     \item What if d2'(d2(m2)) produces a new condition element that was not present in m2 and d2(m2) and d2'(m2)? We need to show that this cannot occur?
%     \item Non-synchronizing: d1 and d2 may induce violations of consistency relations. d1' and d2' restore fulfillment of these consistency relations. We consider how consistency relations can be violated when we put d2 in front of d2' and d1 in front of d1' other than in the case when its applied directly.
% \end{itemize}


% \subsection{Versuch über ursächliche Condition Element Änderungen zu unterscheiden}
% Fälle:
% 1. Condition Element wird neu erzeugt
% 2. Condition Element wird geändert, sodass nun ein anderes Element der gleichen Condition vorhanden ist
% 3. Condition Element wird gelöscht
% Es können bei einem Change mehrere davon bzgl. verschiedener Conditions auftreten (also bzgl. verschiedene Consistency Relations)

% Fall 1: Die Vereinigung der Condition Elements aus d2(m2) und d2'(m2) ist gleich derer in d2'(d2(m2)). Somit wird durch die Kombination kein neues Condition Element eingeführt, für das Konsistenz hergestellt werden müsste.

% Fall 2: Die Vereinigung der Condition Elements aus d2(m2) und d2'(m2) ist ungleich derer in d2'(d2(m2)). Somit wird durch die Kombination ein neues Condition Element eingeführt, für das Konsistenz hergestellt werden müsste.

% NEUE STORY:

% Allgemeine Betrachtung von Änderungen: Es geht immer darum, dass Condition Elemente geändert/gelöscht/hinzugefügt wurden und die CPR entsprechend reagieren muss, um das Vorhandensein einer entsprechenden Witness-Struktur zu garantieren. Dabei können entsprechend folgende Fälle auftreten:
% 1. Änderungen führen dazu, dass ein neues Condition Element im Modell existiert, dass zuvor nicht vorhanden war.
% 2. Änderungen führen dazu, dass Elemente eines bereits vorhandenes Condition Elementes geändert werden und dadurch ein andere Condition Element derselben Condition instanziieren.
% 3. Änderungen führen dazu, dass ein vorher existierendes Condition Element nicht mehr im Modell auftaucht.

% Es gibt hierzu drei entsprechende Reaktionen der CPR:
% 1. Im anderen Modell werden, falls nicht vorhanden, entsprechende Elemente erzeugt, um für das neue Condition Element ein eindeutiges korrespondierendes Condition Element zu erzeugen und somit eine Witness-Struktur aufzubauen. (Erzeugungs-Propagation)
% 2. Das gem. Witness-Struktur korrespondierende Condition Element im anderen Modell wird so angepasst, dass wieder eine valide Witness-Struktur entsteht. (Änderungs-Propagation)
% 3. In dem anderen Modell werden die Elemente des korrespondierenden Condition Elementes entfernt (oder zumindest Teile davon), sodass entsprechend der Konsistenzregeln keine weiteren Elemente vorhanden sein müssen, d.h. wieder eine valide Witness-Struktur vorhanden ist.

% % DIE FOLGENDEN ZWEI PARAGRAPHEN SIND NUN BEI DER STRIKTEN SEQUENTIALISIERUNG
% Wenn wir die unidirektionalen CPR sequentialisieren (also erst d2' erzeugen, dann d2 darauf anwenden), kann es sein, dass d2'(m2) Änderungen an Condition Elements in m2 vornimmt oder neue hinzufügt, um Konsistenz wiederherzustellen, die ebenfalls von d2 eingefügt werden, sodass es nicht mehr möglich ist, durch die rückwärtige CPR Änderungen vorzunehmen, die eine valide Witness-Struktur induzieren.
% Z.B. könnte d2'(m2) einen Resident hinzufügen, der bereits durch d2 eingefügt wurde (weil er über einen andere Pfad erstellt wurde, siehe \autoref{fig:synchronization:duplicate_creation_example}). Wird nun d2 auf d2'(m2) angewendet, würde ggf. in einem Container, in dem die Residents gespeichert werden, zwei Residents mit gleichem Namen eingefügt. Für diese kann aber keine Änderung in m1 (sei es das Employee-Modell) erzeugt werden, durch die eine valide Witness-Struktur entsteht. Das Einfügen eines zweiten Employee mit dem gleichen Namen führt dazu, dass jeder Employee und jeder Resident zu zwei Residents bzw. Employees korrespondiert, was keine eindeutige Witness-Struktur induziert.
% Um dies zu vermeiden, müssen die CPR sicherstellen, dass in den Änderungen am anderen Modell nicht bereits entsprechende Condition Elements erzeugt wurden.
% Alle anderen Änderungen sind unproblematisch, da Änderungen die d2 an bestehenden Condition Elements, die nicht zu neuen Condition Elements führen, durchführt mittels 2->1 propagiert werden können, indem die Condition Elements in m1 angepasst werden.

% Insgesamt ist die Situation die gleiche, als würde ein Nutzer eine entsprechende Änderung machen. Auch er kann natürlich einen zweiten Resident mit demselben Namen einführen. Hier würde die CPR selbstverständlich fehlschlagen. Während das für Nutzeränderungen erwünscht ist, da die doppelte Erzeugung desselben Elementes hier schon vom Nutzer durchgeführt wurde und für einen entsprechende Nutzeränderung kein konsistentes Modell generiert werden kann, ist dies innerhalb des Transformationsnetwerkes unerwünscht, da die CPR natürlich eine konsistente Modellmenge finden können und die doppelte Erzeugung lediglich daher kommt, dass die Transformationen nicht, wie verlangt, synchronisieren sind. In letztem Fall wäre sichergestellt, dass nach entsprechenden Änderungen an beiden Modellen (also Erzeugung von Employee in einem, Erzeugung des passenden Resident im anderen) keine Änderungen gemacht werden, da bereits eine passende Witness-Struktur vorhanden ist.
% Die unidirektionalen CPR schaffen das jedoch nicht, da Ihnen die entsprechende Information fehlt.


% \subsection{Erster Versuch zur Partiellen Konsistenz}
% % DIE FOLGENDEN PARAGRAPHEN SIND IN DER PARTIELLEN KONSISTENZ AUFGEGANGEN, NUR DAS PARTIELLE KONSISTENZ ÜBER MODELLE UND NICHT ÜBER CPR DEFINIERT IST
% Was kann nun passieren, wenn wir die CPR mit dem modifizierten Ziel-Modell aufrufen?
% Wir müssen Def anpassen, da es nun nicht mehr reicht, wenn CPR für konsistente Eingabe korrekt ist.

% m1 und d2(m2) sind ja immer noch partiell konsistent. Wir betrachten für jedes Konsistenzrelation alle condition elements in m1 und d2(m2).
% Diejenigen, für die es ein eindeutiges korrespondierendes Element gibt, also eine maximale Menge (es gibt keine Menge, von der sie eine Teilmenge ist) für die es eine Witness-Struktur gibt, und die immer noch in d1(m1) bzw. d2'(d2(m2)) vorkommen, muss es auch darin ein eindeutiges korrespondierendes Element geben.
% Außerdem muss für alle Condition Elements in d1(m1) $\setminus$ m1 und in d2'(d2(m2)) $\setminus$ d2(m2) ein eindeutiges korrespondierendes Element existieren.
% Dies ist bereits dadurch sichergestellt, dass ja die CPR immer auf das Erzeugen/Ändern/Löschen eines Condition Elementes reagieren, d.h. für die Element ein d1(m1) $\setminus$ m1 stellt es Konsistenz sicher und für d2'(d2(m2)) auch, da es sonst eine neue Inkonsistenz induzieren würde.
% D.h. nur für Elemente, die vorher nicht konsistent waren, ist keine Konsistenz verlangt.

% Property: Always-preserving
% CPR erhält Konsistenz für solche Elemente, die vorher konsistent waren. D.h. wenn es eine Witness-Struktur für eine Teilmenge der Modelle gibt, dann gibt es sie auch nach den Änderungen (also in d1(m1) und d2'(d2(m2))) für dieselben Teilmengen (bzw. das was noch davon da ist), egal ob die Modelle vorher konsistent waren oder nicht. (es ist schwer diese Eigenschaft für Transformationen zu zeigen, aber die empirische Evaluation zeigt, dass die Annahme dort zumindest gilt)

% Property: Delta-Correcting
% CPR stellt Konsistenz für solche Elemente her, die durch das Delta im Quellmodell und Zielmodell hinzugefügt werden. Also für alle Condition Elements in d1(m1) $\setminus$ m1 und in d2'(d2(m2)) $\setminus$ d2(m2) muss ein eindeutiges korrespondierendes Element existieren.

% \subsection{Zweiter Ansatz zur partiellen Konsistenz auf Condition Element Level}
% Wir fordert für alle Condition Elements in d1(m1) und alle in m1, die noch immer in d1(m1) vorkommen, dass sie wieder ein eindeutiges korrespondierendes in d2'(d2(m2)) haben, wenn sie in m1 vorhanden waren und dort ein korrespondierendes in d2(m2) hatten.
% Fallunterscheidung:
% * Sie waren in m1 vorhanden, aber in d1(m1) nicht mehr: Dann wurden sie geändert. Wurde das korrespondierende Element in d2(m2) nicht gegenüber m2 geändert, muss es mit d2'(d2(m2)) von CPRr angepasst werden, da sonst auch ohne d2(m2) CPRr die Änderung nicht vorgenommen hätte und damit nicht korrekt wäre. Wurde das Element in d2(m2) gegenüber m2 geändert, so kümmert sich CPRl noch um diese Änderung. \todo{Das können wir nicht transitiv so machen}
% * \dots


% \subsection{Co-occurring Changes to Corresponding Elements}
% Two problem cases: 
% 1. Both d1 and d2 affect corresponding condition elements (otherwise show that it is unproblematic); 
% 2. d2'(d2(m2)) introduced new condition elements that were neither present in m2, nor in d2'(m2), nor d2(m2), so they are neither consistent due to correctness of forward preservation rule, nor processed by backward preservation rule.
% \begin{itemize}
%     \item If d1 affects a condition element (be it a change of an existing, the creation of a new one or the removal of an old one), then preservation needs to generate a d2' that updates/creates/removes the corresponding condition elements appropriately, such that even elements that are potentially part of another condition element, fulfill consistency (due to correctness). If d2 does not affect any of the corresponding or other changes elements, everything is fine, because then we can simply sequence changes.
%     \item 
% \end{itemize}
    
% Szenarien:
% Auf jeder Seite wurde ein Condition Element modifiziert.
% 1. 1->2 und 2->1 ändern jeweils Elemente aus komplett disjunkten Condition Elements: Witness-Struktur ergibt sich aus alter Witness-Struktur und für entfernen von Element und korrespondierendem das Entfernen der Korrespondenz, Hinzufügen und Element und korrespondierendem das Hinzufügen der Korrespondenz, sowie Ändern eines Condition Elementes zu einem anderen das Ändern der entsprechenden Korrespondenz.
% 2. 1->2 (resp.~2->1) fügen durch Änderungen neues Condition Element ein: Sind per Korrektheit verpflichtet das richtig aufzulösen
% 3. \dots

\section{Synchronizing Bidirectional Transformations}

\subsection{Partial Consistency}

A notion of partial consistency may be defined in two different ways.
We may either say that two models only fulfill the consistency relations partially, or we may say that only extracts of two models fulfill the consistency relations.

In the first option, we could consider that the given models are only consistent to a subset of the given consistency relations.
There may, however, be only a single element in the models that leads to the violation of all consistency relations.
Thus, we would call the models totally inconsistent just because of a single element.
To circumvent hat, we would need to define a notion of partial consistency relations, which allows us to define that models are consistent to parts of consistency relations.
Such a notion would have to be defined at the level of consistency relation pairs and their condition elements within the consistency relations.
It would, however, not make sense to consider subsets of consistency relations, i.e., only a subset of their consistency relation pairs, because when analyzing consistency of two models those pairs are not independent.
The existence of the condition elements of one consistency relation pair within two models may prevent the existence of the condition elements of another consistency relation pair, because in \autoref{def:consistency}, we require a unique mapping between condition elements in terms of witness structure.
If models are consistent to a consistency relation in which we removed one (or more) consistency relation pairs, this does not give any reasonable indication on how the models violate consistency.
This may be due to the reason that an element is missing in the models or that an additional element prevents from finding a witness structure.
It does, however, not mean that adding a missing element or removing the additional element ensures that a proper witness structure can be found, because these elements may still be relevant other consistency relation pairs in the witness structure.
These interdependencies of consistency relation pairs are the reason why consistency to partial consistency relations does not provide insights on the reasons for models being inconsistent.

In the second option, we could consider that only parts of the given models are consistent to all given consistency relations.
In addition to the missing ability of the first option to give reasonable insights on inconsistencies, this, intuitively, is a more reasonable notion, because it explicitly defines that parts of the models are consistent, whereas other parts of them are not.
We thus define partial consistency as models having subsets that are actually consistent.
To identify how far models are partially consistent, we also define a metrics.
It is based on the idea to find maximal subsets of the models that are consistent.

\begin{definition}[Partial Consistency]
    Let $\consistencyrelationset{CR}$ be a set of consistency relations.

    Given two models $\model{m}{1} \in \metamodelinstanceset{M}{1}$ and $\model{m}{2} \in \metamodelinstanceset{M}{2}$, we define their \emph{maximal consistent subsets} $\model{m}{1}^p \in \metamodelinstanceset{M}{1}$ and $\model{m}{2}^p \in \metamodelinstanceset{M}{2}$ with regards to $\consistencyrelationset{CR}$ as the subsets of $\model{m}{1}$ and $\model{m}{2}$ that are consistent and larger than all other consistent subsets:
    \begin{align*}
        & 
        \tupled{\model{m}{1}^p, \model{m}{2}^p} \consistenttomath \consistencyrelationset{CR} \land
        \model{m}{1}^p \subseteq \model{m}{1} \land \model{m}{2}^p \subseteq \model{m}{2}  \\
        & \formulaskip
        \land \forall \model{m}{1}^{p'} \in \metamodelinstanceset{M}{1}, \model{m}{2}^{p'} \in \metamodelinstanceset{M}{2} : \\
        & \formulaskip\formulaskip
        \bigl(\model{m}{1}^{p'} \subseteq \model{m}{1} \land \model{m}{2}^{p'} \subseteq \model{m}{2} 
        \land \tupled{\model{m}{1}^{p'}, \model{m}{2}^{p'}} \consistenttomath \consistencyrelationset{CR} \\
        & \formulaskip\formulaskip
        \Rightarrow 
        \abs{\model{m}{1}^{p'}} + \abs{\model{m}{2}^{p'}} \leq \abs{\model{m}{1}^{p}} + \abs{\model{m}{2}^{p}} \bigr)
    \end{align*}
    We define partial consistency of two models with respect to $\consistencyrelationset{CR}$ as the ratio between the size of the maximal consistent subsets and the size of the models in $\function{cons}_{\consistencyrelationset{CR}}$:
    \begin{align*}
        \function{cons}_{\consistencyrelationset{CR}}: \; 
        & (\metamodelinstanceset{M}{1}, \metamodelinstanceset{M}{2}) \rightarrow [0,1] \\
        & 
        (\model{m}{1}, \model{m}{2}) \mapsto \frac{\abs{\model{m}{1}^{p}} + \abs{\model{m}{2}^{p}}}{\abs{\model{m}{1}} + \abs{\model{m}{2}}}
    \end{align*}
    % with:
    % \begin{align*}
    %     & 
    %     \exists \model{m}{1}^p \in \metamodelinstanceset{M}{1}, \model{m}{2}^p \in \metamodelinstanceset{M}{2} : \\
    %     & \formulaskip
    %     \tupled{\model{m}{1}^p, \model{m}{2}^p} \consistenttomath \consistencyrelationset{CR} \land
    %     \model{m}{1}^p \subseteq \model{m}{1} \land \model{m}{2}^p \subseteq \model{m}{2} : \\
    %     & \formulaskip
    %     \land \forall \model{m}{1}^{p'} \in \metamodelinstanceset{M}{1}, \model{m}{2}^{p'} \in \metamodelinstanceset{M}{2} : \\
    %     & \formulaskip\formulaskip
    %     \bigl(\model{m}{1}^{p'} \subseteq \model{m}{1} \land \model{m}{2}^{p'} \subseteq \model{m}{2} 
    %     \land \tupled{\model{m}{1}^{p'}, \model{m}{2}^{p'}} \consistenttomath \consistencyrelationset{CR} \\
    %     & \formulaskip\formulaskip
    %     \Rightarrow 
    %     \abs{\model{m}{1}^{p'}} + \abs{\model{m}{2}^{p'}} \leq \abs{\model{m}{1}^{p}} + \abs{\model{m}{2}^{p}} \bigr) \\
    %     &\formulaskip
    %     \land \function{cons}_{\consistencyrelationset{CR}}(\model{m}{1}, \model{m}{2}) = \abs{\model{m}{1}^{p}} + \abs{\model{m}{2}^{p}}
    % \end{align*}
\end{definition}

Such maximal consistent subsets do always exist.
In the extreme case, it is $\model{m}{1}^p = \model{m}{2}^p = \emptyset$, because empty models are always consistent by definition.
In that case, partial consistency of the models is $0$, whereas in cases when models are actually consistent the maximal consistent subsets are the models themselves, which is why partial consistency is $1$.

% \paragraph{Partielle Konsistenz}
% Gegeben m1, d2(m2) und d1.
% d1(m1) und d2(m2) sind partiell konsistent, d.h. es gibt Teilmengen d1(m1)p und d2(m2)p, für die alle Konsistenzrelationen erfüllt sind.
% Dies kann für mehrere Teilmengen gelten, wir betrachten die, die zusammen maximal groß sind, d.h. |d1(m1)p| + d2(m2)p| > |d1(m1)p'| + d2(m2)p'| für beliebige andere Teilmenge, die konsistent sind.
% TODO: Metrik definieren für partielle Konsistenz. Dann können nämlich einfach sagen, dass sich die partielle Konsistenz erhöhen muss (indem entweder die Modelle kleiner werden oder die konsistenten Teile größer)


\subsection{Transformation Requirements}

\paragraph{Partial Consistency Preservation}
Wir verlangen von CPRr, dass nachher ist Konsistenz für eine Obermenge der partiell konsistenten Modelle gilt, also für Obermengen von d1(m1)p und eine Obermenge von d2(m2)p erfüllt. Das verbietet insbesondere, dass eine CPR in m2 beliebig Elemente löscht, die vorher konsistent waren, also verlangt dass d2(m2)p $\subseteq$ d2'(d2(m2))
Eigenschaft: Partial-consistency preserving
Wir verlangen von CPRl dasselbe.
Es gibt nach CPRr auf jeden Fall zwei Teilmengen der Modelle die konsistent sind und größer als die ursprünglich konsistenten Modellteile.
Somit muss CPRl für diese (oder eine ggf. andere mindestens genauso große Teilmenge) wieder die Eigenschaft erfüllen.
Für CPRr:
|d1(m1)p| + |d2'(d2(m2))p| >= |d1(m1)p| + |d2(m2)p|

\paragraph{Partial Consistency Improving}
Wir verlangen außerdem, dass für alle m1, m2, d1, d2 mit d2' = CPRr(m1, d2(m2), d1) und d1' = CPRl(m2, d1(m1), d2)eines der folgenden gilt:
1. |d1(m1)| + |d2'(d2(m2))| < |d1(m1)| + |d2(m2)|
2. |d1'(d1(m1))| + |d2(m2)| < |d1(m1)| + |d2(m2)|
3. |d1'(d1(m1))p| + |d2(m2)p| > |d1(m1)p| + |d2(m2)p|
4. |d1(m1)p| + |d2'(d2(m2))p| > |d1(m1)p| + |d2(m2)p|
Das bedeutet, dass für Änderungen an beiden Modellen eine der beiden CPR entweder dafür sorgt, dass eines der Modelle kleiner wird, oder dass sich die partielle Konsistenz erhöht (also die Teilmengen, die konsistent sind, größer werden).

\paragraph{Non-Inconsistency Introducing}
1. Elemente in d2'(d2(m2)) $\setminus$ d2(m2) müssen in d2'(d2(m2))p sein
2. Elemente in d2(m2) $\setminus$ d2'(d2(m2)) dürfen nicht in d2(m2)p sein
Das heißt, alles was neu hinzu kommt muss auch partiell konsistent sein, und alles was entfernt wird darf vorher nicht partiell konsistent gewesen sein.
Das ist sinnvoll, da sonst eine CPR ja bereits konsistente Teile inkonsistent machen würde und sonst beliebigen Quatsch hinzufügen könnte, der nicht konsistent ist. Letztes ist insbesondere wichtig, damit eine CPR nicht die Modelle ein bisschen konsistenter macht, aber durch Hinzufügen weiterer Elemente eigentlich viel inkonsistenter.
(Die zweite Anforderung sagt, dass keine vorher konsistenten Elemente entfernt werden dürfen und ergibt sich eigentlich auch schon aus der Partial-consistency preserving Eigenschaft. Allerdings nicht ganz, da partial-consistency preserving nur Aussagen über die Größe macht.)

\paragraph{Informelle Beschreibung}
Informell gesagt wollen wir, dass die beiden CPR 1. keine vorhandene Konsistenz zerstören und immer ein bisschen Konsistenz hinzufügen (d.h. partielle Konsistenz erhöhen) und 2. insbesondere keine neue Inkonsistenz einführen indem sie Elemente erzeugen, die neuer Konsistenz bedürfen.
Dies ist lediglich die Erkenntnis, dass CPR-Implementierungen so etwas können müssen. Wie das praktisch erreicht werden kann ist damit nicht angegeben.
Es ist aber nur natürlich, dass eine CPR so definiert ist, dass sie keine neuen Inkonsistenzen einführt und dass bei Änderungen, die eine Inkonsistenz induzieren, eine der beiden CPR sich dieser Änderung annimmt und Konsistenz wiederherstellt.



\subsection{Transformation Termination}

FOLGERUNG:
Wenn wir zwei CPR haben, die diese Eigenschaft erfüllen, konvergiert deren sequentielle Ausführung (ggf. auch mehrfach) zu einem konsistenten Zustand, da sie monoton und beschränkt sind.

\todo{Theorem zur konsistenten Terminierung}

\paragraph{Randnotiz}
Die maximalen Mengen sind eine formale Voraussetzung. In der Praxis hat man i.d.r. explizite Trace-Modelle, die quasi als Witness-Struktur fungieren und damit genau beschrieben, welche Elemente zusammengehören. Das kann dazu führen, dass nicht die maximalen Teilmengen tatsächlich die Teile der Modelle darstellen, die als konsistent betrachtet werden, sondern kleinere Mengen.
Das ist o.B.d.A. aber kein Problem, sofern beide CPR Monotonie für immer die gleichen Mengen erfüllen.

\paragraph{Alte Idee}
Anforderung an CPR:
Hippokratisch und assoziativ, d.h.
d2' = CPR(m1, m2, d1), d2'' CPR(d1(m1), d2'(m2), d1')
und d2''' = CPR(m1, m2, d1' o d1) mit d2''' = d2'' o d2'
Dadurch kann ich eine Änderung beliebig zerlegen und nacheinander CPR anwenden mit demselben Ergebnis.
Das bedeutet insb. auch, dass wenn d1(m1) un d2(m2) konsistent ist CPR(m1, d2(m2), d1) nichts tut, weil es hippokratisch ist, also d2' = id und CPR(m1, d2(m2), d1' o d1) = CPR(m1, d2(m2), d1')










\section{Old contents to integrate}

%%
%% THE DIRECTIONALITY GAP: Formal framework considers bidirectional transformations, practically we have unidirectional ones
%%
\subsection{The Directionality Gap}

\mnote{Unidirectional synchronizing preservation rules to close the gap}
To come up with an approach to combine unidirectional consistency preservation rules to behave as a single synchronizing one, we first identify how we can decompose synchronizing consistency preservation rules into unidirectional ones.
We then relate these unidirectional synchronizing rules to the non-synchronizing ones defined in transformation languages.

\mnote{Fine-grained consistency relations allow to define relation between unidirectional preservation rules}
In \autoref{chap:compatibility:formal_notion}, we also discussed that consistency relation can be considered in a fine-grained way that is able to reflect different notion of consistency in both directions of a relation.
We will base our notion of unidirectional synchronizing rules on those fine-grained relations to be able to find a proper notion of how the rules in both directions are supposed to work together.
We did, however, also discuss in that chapter that all fine-grained relations can also be translated into \modellevelconsistencyrelations, thus the insights we already had for those model-level relations still apply to the considerations regarding fine-grained ones.

\mnote{Stick to coarse-grained notion of preservation rules}
According to the consideration of fine-grained consistency relations, transformation languages also allow the specification of or derive fine-grained consistency preservation rules from a declarative specification.
They are often called \emph{transformation rules} and composed to a transformation that consists of multiple such rules, each encoding a consistency relations and a preservation rule for it.
We will, however, stick to the coarse-grained notion of consistency preservation rules, as they are sufficient for our considerations.

\mnote{New transformation notion based on fine-grained consistency relations}
In consequence, from now we will consider a synchronizing transformation as a set of fine-grained consistency relations according to \autoref{def:consistencyrelation} and a consistency preservation rule that preserves consistency according to the set of relations.
The consistency preservation rule and also the complete transformation are thus still considered correct if applying it to a consistent pair of models and changes to them, applying the resulting changes to the models again delivers a pair of models that is consistent to all consistency relations.
Note that being consistent to all fine-grained consistency relations is equivalent to being consistent to the single \modellevelconsistencyrelation induced by the fine-grained relations.


\subsubsection{Unidirectional Synchronizing Transformations}

\mnote{Decomposition of consistency preservation rules into two unidirectional ones}
To reflect the unidirectional notion of consistency preservation provided by transformation languages in our formalism, we propose a decomposition of consistency preservation rules according to \autoref{def:consistencypreservationrule} into two unidirectional ones.

\begin{definition}[Unidirectional Consistency Preservation Rule]
    \label{def:unidirectionalconsistencypreservationrule}
    Let $\metamodel{M}{1}, \metamodel{M}{2}$ be two metamodels and $\consistencyrelationset{CR}$ a set of consistency relations between elements of those metamodels.
    A \emph{unidirectional consistency preservation rule} $\consistencypreservationrule{\consistencyrelationset{CR}}$ for the relation set $\consistencyrelationset{CR}$ is a function:
    \begin{align*}
        \consistencypreservationrule{\consistencyrelationset{CR}} : (\metamodelinstanceset{M}{1}, \metamodelinstanceset{M}{2}, \changeuniverse{\metamodel{M}{1}}, \changeuniverse{\metamodel{M}{2}}) \rightarrow \changeuniverse{\metamodel{M}{2}}
    \end{align*}
\end{definition}

\mnote{Notation for two related unidirectional rules}
To be able to explicitly reference the consistency preservation rules for both directions between two metamodels $\metamodel{M}{1}$ and $\metamodel{M}{2}$, we denote the set of consistency relations between $\metamodel{M}{1}$ and $\metamodel{M}{2}$ as $\consistencyrelationset{CR}_{\rightarrow}$ and the one in the opposite direction between $\metamodel{M}{2}$ and $\metamodel{M}{1}$ as $\consistencyrelationset{CR}_{\leftarrow}$.
We then refer to the two unidirectional consistency preservation rules as:
\begin{align*}
    \consistencypreservationrule{\consistencyrelationset{CR}_{\rightarrow}} : (\metamodelinstanceset{M}{1}, \metamodelinstanceset{M}{2}, \changeuniverse{\metamodel{M}{1}}) \rightarrow \changeuniverse{\metamodel{M}{2}}\\
    \consistencypreservationrule{\consistencyrelationset{CR}_{\leftarrow}} : (\metamodelinstanceset{M}{2}, \metamodelinstanceset{M}{1}, \changeuniverse{\metamodel{M}{2}}) \rightarrow \changeuniverse{\metamodel{M}{1}}
\end{align*}

\mnote{Correctness of unidirectional rules analogous to original rules}
We define correctness of such a unidirectional consistency preservation rule in the same way it was defined for synchronizing consistency preservation rules according to \autoref{def:consistencypreservationrule} and compliant to existing correctness notions for non-synchronizing consistency preservation rules, such as~\cite{stevens2010sosym}.

\begin{definition}[Unidirectional Consistency Preservation Rule Correctness]
    \label{def:unidirectionalconsistencypreservationrulecorrectness}
    Let $\consistencypreservationrule{\consistencyrelationset{CR}}$ be a \emph{unidirectional consistency preservation rule}.
    We call $\consistencypreservationrule{\consistencyrelationset{CR}}$ \emph{correct} if the resulting models when applying the generated changes are consistent to $\consistencyrelationset{CR}$ again:
    \begin{align*}
        &
        \forall 
        \model{m}{1} \in \metamodelinstanceset{M}{1}, 
        \model{m}{2} \in \metamodelinstanceset{M}{2},
        \change{\metamodel{M}{1}} \in \changeuniverse{\metamodel{M}{1}},
        \change{\metamodel{M}{2}} \in \changeuniverse{\metamodel{M}{2}} : \\
        & \formulaskip
        \tupled{\model{m}{1}, \model{m}{2}} \consistenttomath \consistencyrelationset{CR} \\
        & \formulaskip
        \land \exists 
        \change{\metamodel{M}{2}}' \in \changeuniverse{\metamodel{M}{2}} :
        \change{\metamodel{M}{2}}' = \consistencypreservationrule{\consistencyrelation{CR}{}}(\model{m}{1}, \model{m}{2}, \change{\metamodel{M}{1}}, \change{\metamodel{M}{2}}) \\
        & \formulaskip\formulaskip
        \Rightarrow
        \tupled{\change{\metamodel{M}{1}}(\model{m}{1}), \change{\metamodel{M}{2}}'(\model{m}{2})} \consistenttomath \consistencyrelationset{CR}
    \end{align*}
\end{definition}

\mnote{Unidirectional synchronizing transformation encapsule a preservation rule for each direction}
Based on such a unidirectional notion of consistency preservation rule, we can also define a unidirectional notion of transformations, which then consists of two sets of unidirectional consistency relations and two unidirectional consistency preservation rules.

\begin{definition}[Unidirectional Synchronizing Transformation]
    \label{def:unidirectionalsynchronizingtransformation}
    Let $\metamodel{M}{1}$ and $\metamodel{M}{2}$ be two metamodels and $\consistencyrelationset{CR}_{\rightarrow}$ a set of consistency relations between $\metamodel{M}{1}$ and $\metamodel{M}{2}$, as well as $\consistencyrelationset{CR}_{\leftarrow}$ a set of consistency relations between $\metamodel{M}{2}$ and $\metamodel{M}{1}$.
    Additionally, let $\consistencypreservationrule{\consistencyrelationset{CR}_{\rightarrow}}$ and $\consistencypreservationrule{\consistencyrelationset{CR}_{\leftarrow}}$ be unidirectional consistency preservation rules for both consistency relation sets.
    A \emph{unidirectional synchronizing transformation} is a quadruple $\transformation{T} = \tupled{\consistencyrelationset{CR}_{\rightarrow}, \consistencyrelationset{CR}_{\leftarrow},\consistencypreservationrule{\consistencyrelationset{CR}_{\rightarrow}}, \consistencypreservationrule{\consistencyrelationset{CR}_{\leftarrow}}}$.
\end{definition}

\mnote{Correctness of unidirectional synchronizing transformations}
We call such a unidirectional synchronizing transformation correct if both consistency preservation rules are correct, i.e., they both preserve consistency according to the underlying consistency relation set.

\begin{definition}[Unidirectional Synchronizing Transformation Correctness]
    \label{def:unidirectionalsynchronizingtransformationcorrectness}
    Let $\transformation{T} = \tupled{\consistencyrelationset{CR}_{\rightarrow}, \consistencyrelationset{CR}_{\leftarrow},\consistencypreservationrule{\consistencyrelationset{CR}_{\rightarrow}}, \consistencypreservationrule{\consistencyrelationset{CR}_{\leftarrow}}}$ be a unidirectional synchronizing transformation.
    We call $\transformation{T}$ correct if, and only if, $\consistencypreservationrule{\consistencyrelationset{CR}_{\rightarrow}}$ and $\consistencypreservationrule{\consistencyrelationset{CR}_{\leftarrow}}$ are both correct according to \autoref{def:unidirectionalconsistencypreservationrulecorrectness}.
\end{definition}

\mnote{Each unidirectional rule only preserves consistency in one direction}
With a unidirectional synchronizing transformation that adheres to the given correctness definition, we are able to preserve consistency for the consistency relations in each direction between two models.
Executing either of the unidirectional consistency preservation rules of the transformation does, however, not ensure that the consistency relations for the other direction are fulfilled as well.
That is the purpose of the other unidirectional consistency preservation rule.

% \begin{itemize}
    % \item We assume consistency preservation rules according to fine-grained consistency relations introduced for compatibility
    % \item So a synchronizing transformation considers fine-grained relations, in fact a transformation then consists of multiple relations, two for each fine-grained relation (each direction). Their combination induces the \modellevelconsistencyrelation for the two metamodels.
    % \item Although the consistency preservation rule may in practice also be defined in terms fine-grained rules, which together with the fine-grained consistency rules then forms what is often called \emph{transformation rules}, we do not need to have a more fine-grained notion here.
    % \item The transformation then is still correct as defined before, when the preservation rule preserves consistency to the relation, but now according to all fine-grained relations (and thus also the induced monolithic one) instead to the single model-level one.
    % \item To reflect the notion of unidirectional consistency preservation rules, as often defined in transformation languages, which are still synchronizing, i.e., are able to react to changes made to both models, we may define:
% \end{itemize}
% \begin{align*}
%     \consistencypreservationrule{\consistencyrelationset{CR}_{\rightarrow},\rightarrow} : (\metamodelinstanceset{M}{1}, \metamodelinstanceset{M}{2}, \changeuniverse{\metamodel{M}{1}}, \changeuniverse{\metamodel{M}{2}}) \rightarrow \changeuniverse{\metamodel{M}{2}})\\
%     \consistencypreservationrule{\consistencyrelationset{CR}_{\leftarrow},\leftarrow} : (\metamodelinstanceset{M}{1}, \metamodelinstanceset{M}{2}, \changeuniverse{\metamodel{M}{1}}, \changeuniverse{\metamodel{M}{2}}) \rightarrow \changeuniverse{\metamodel{M}{1}})
% \end{align*}

\subsubsection{Alignment of Unidirectional Synchronizing Transformations}

\mnote{Sequential execution of unidirectional rules does not guarantee consistency to relations in both directions}
It is now possible to execute both consistency preservation rules of a unidirectional synchronizing transformation one after another, as each is able to reflect the changes produced by the other due to the ability to process changes made to both models.
Thus, we could sequentially calculate:
\begin{align*}
    &
    \change{\metamodel{M}{2}}' = \consistencypreservationrule{\consistencyrelationset{CR}_{\rightarrow}}(\model{m}{1},\model{m}{2},\change{\metamodel{M}{1}},\change{\metamodel{M}{2}})\\
    &
    \change{\metamodel{M}{1}}' = \consistencypreservationrule{\consistencyrelationset{CR}_{\leftarrow}}(\model{m}{2},\model{m}{1},\change{\metamodel{M}{2}}',\change{\metamodel{M}{1}})
\end{align*}
We then receive the resulting model pair as $\tupled{\change{\metamodel{M}{1}}'(\model{m}{1}), \change{\metamodel{M}{2}}'(\model{m}{2})}$.
This gives us the guarantee that the resulting model pair is consistent to $\consistencyrelationset{CR}_{\leftarrow}$, as its consistency preservation rule was executed last.
However, the definitions do not give any guarantee that the model pair is consistent to $\consistencyrelationset{CR}_{\rightarrow}$, as the execution of $\consistencypreservationrule{\consistencyrelationset{CR}_{\leftarrow}}$ may violate some of those relations.

\mnote{Not violating relations of the other direction is desirable for unidirectional rules}
It is, however, a desirable property of a unidirectional synchronizing transformation that the consistency preservation rules for the two directions between the same metamodels are aligned with each other in a way that executing the rule in one direction does not lead to further violations of the consistency relations in the other direction.
This is especially necessary to produce the same behavior as a synchronizing transformation.
We call such a property \emph{inverse-preserving}, as it ensures that fulfillment of consistency relations of the inverse direction are preserved.

\begin{definition}[Inverse-Preserving Unidirectional Synchronizing Transformation]%Unidirectional Consistency Preservation Rule}
    \label{def:inversepreservingunidirectionalsynchronizingtransformation}
    Let $\transformation{T} = \tupled{\consistencyrelationset{CR}_{\rightarrow}, \consistencyrelationset{CR}_{\leftarrow},\consistencypreservationrule{\consistencyrelationset{CR}_{\rightarrow}}, \consistencypreservationrule{\consistencyrelationset{CR}_{\leftarrow}}}$ be a correct unidirectional synchronizing transformation for two metamodels $\metamodel{M}{1}$ and $\metamodel{M}{2}$.
    We say that $\transformation{T}$ is \emph{inverse-preserving} if, and only if, executing one of the consistency preservation rules ensures that the resulting models are still consistent to all consistency relations of the opposite direction to which the models were consistent before.
    For $\consistencypreservationrule{\consistencyrelationset{CR}_{\rightarrow}}$ this is given by the following (for $\consistencypreservationrule{\consistencyrelationset{CR}_{\leftarrow}}$ analogously):
    \begin{align*}
        & \forall \model{m}{1} \in \metamodelinstanceset{M}{1}, \model{m}{2} \in \metamodelinstanceset{M}{2}, \change{\metamodel{M}{1}} \in \changeuniverse{\metamodel{M}{1}}, \change{\metamodel{M}{2}} \in \changeuniverse{\metamodel{M}{2}} : \\
        & 
        \exists \change{\metamodel{M}{2}}' \in \changeuniverse{\metamodel{M}{2}} : \change{\metamodel{M}{2}}' = \consistencypreservationrule{\consistencyrelationset{CR}_{\rightarrow}}(\model{m}{1},\model{m}{2},\change{\metamodel{M}{1}},\change{\metamodel{M}{2}}) \\
        & \formulaskip
        \Rightarrow
        \forall \consistencyrelation{CR}{} \in \consistencyrelationset{CR}_{\leftarrow}: 
        \bigl(
        \tupled{\change{\metamodel{M}{2}}(\model{m}{2}),\change{\metamodel{M}{1}}(\model{m}{1})} \consistenttomath \consistencyrelation{CR}{} \\
        & \formulaskip\formulaskip
        \Rightarrow
        \tupled{\change{\metamodel{M}{2}}'(\model{m}{2}), \change{\metamodel{M}{1}}(\model{m}{1})} \consistenttomath \consistencyrelation{CR}{}
        \bigr)
    \end{align*}
\end{definition}

\mnote{Inverse-preserving transformations define reasonable alignment}
This is a reasonable property, because the consistency relations in both directions are usually not disaligned, but only give the freedom to define different consistency notions in both directions, such as more options for consistent elements in one direction than in the other to support different kinds of abstraction.
It should, however, never be the case that preserving consistency in one direction violates consistency relations in the other direction if transformations are defined properly.

\mnote{Inverse-preserving transformations can be sequences}
Given an inverse-preserving unidirectional synchronizing transformation, executing the two unidirectional consistency preservation rules one after another to given consistent models and changes to them preserves consistency to all consistency relations.
This is a direct consequence of the inverse-preserving property, because no directional rule is allowed to violate consistency that was already ensured in the other direction.

\begin{theorem}[Inverse-Preserving Unidirectional Synchronizing Transformation Sequencing Correctness]
    \label{theorem:sequencinginversepreservingtransformations}
    Let $\transformation{T} = \tupled{\consistencyrelationset{CR}_{\rightarrow}, \consistencyrelationset{CR}_{\leftarrow}, \consistencypreservationrule{\consistencyrelationset{CR}_{\rightarrow}}, \consistencypreservationrule{\consistencyrelationset{CR}_{\leftarrow}}}$ be a correct, inverse-preserving unidirectional synchronizing transformation.
    Then sequentially executing both consistency preservation rules to given models and changes restores consistency to both consistency relation sets, i.e., given models $\model{m}{1}, \model{m}{2}$ and changes to them $\change{\metamodel{M}{1}}, \change{\metamodel{M}{2}}$, it holds hat:
    \begin{align*}
        &
        \tupled{\model{m}{1}, \model{m}{2}} \consistenttomath \consistencyrelationset{CR}_{\rightarrow}
        \land \tupled{\model{m}{2}, \model{m}{1}} \consistenttomath \consistencyrelationset{CR}_{\leftarrow}\\
        &
        \land \exists \change{\metamodel{M}{2}}' \in \changeuniverse{\metamodel{M}{2}}' : 
        \change{\metamodel{M}{2}}' = \consistencypreservationrule{\consistencyrelationset{CR}_{\rightarrow}}(\model{m}{1}, \model{m}{2}, \change{\metamodel{M}{1}}, \change{\metamodel{M}{2}}) \\
        &
        \land \exists \change{\metamodel{M}{1}}' \in \changeuniverse{\metamodel{M}{1}}' :
        \change{\metamodel{M}{1}}' = \consistencypreservationrule{\consistencyrelationset{CR}_{\leftarrow}}(\model{m}{2}, \model{m}{1}, \change{\metamodel{M}{2}}', \change{\metamodel{M}{1}}) \\
        &
        \Rightarrow \tupled{\change{\metamodel{M}{1}}'(\model{m}{1}), \change{\metamodel{M}{2}}'(\model{m}{2})}\consistenttomath \consistencyrelationset{CR}_{\rightarrow}  \\
        & \formulaskip
        \land \tupled{\change{\metamodel{M}{2}}'(\model{m}{2}), \change{\metamodel{M}{1}}'(\model{m}{1})} \consistenttomath \consistencyrelationset{CR}_{\leftarrow}
    \end{align*}
\end{theorem}
\begin{proof}
    Given models $\model{m}{1}, \model{m}{2}$ that are consistent to $\consistencyrelationset{CR}_{\rightarrow}$ and $\consistencyrelationset{CR}_{\leftarrow}$ and changes $\change{\metamodel{M}{1}}, \change{\metamodel{M}{2}}$, we assume that $\change{\metamodel{M}{1}}'$ and $\change{\metamodel{M}{2}}'$ exist as the results of the consistency preservation rules according to the theorem.
    If this was not the case, the consistency preservation rules are not defined for the given inputs and thus are not able to produce consistent results, which evaluates the left side of the implication to false, making the whole statement of the theorem true.
    Now we show correctness of both implied statements:
    \begin{properenumerate}
        \item $\tupled{\change{\metamodel{M}{1}}'(\model{m}{1}), \change{\metamodel{M}{2}}'(\model{m}{2})}\consistenttomath \consistencyrelationset{CR}_{\rightarrow}$:
        As a direction implication of correctness of $\consistencypreservationrule{\consistencyrelationset{CR}_{\rightarrow}}$, we know that $\tupled{\change{\metamodel{M}{1}}(\model{m}{1}), \change{\metamodel{M}{2}}'(\model{m}{2})} \consistenttomath \consistencyrelationset{CR}_{\rightarrow}$.
        Now the inverse-preserving property of $\transformation{T}$ ensures for the given input of $\consistencypreservationrule{\consistencyrelationset{CR}_{\leftarrow}}$ according to \autoref{def:inversepreservingunidirectionalsynchronizingtransformation} that 
        \begin{align*}
        & \forall \consistencyrelation{CR}{} \in \consistencyrelationset{CR}_{\rightarrow}: 
        \bigl(
        \tupled{\change{\metamodel{M}{1}}(\model{m}{1}),\change{\metamodel{M}{2}}'(\model{m}{2})} \consistenttomath \consistencyrelation{CR}{} \\
        & \formulaskip\formulaskip
        \Rightarrow
        \tupled{\change{\metamodel{M}{1}}'(\model{m}{2}), \change{\metamodel{M}{2}}'(\model{m}{1})} \consistenttomath \consistencyrelation{CR}{}
        \bigr)
        \end{align*}
        Since the left side of the implication is true for all consistency relations in $\consistencyrelationset{CR}_{\rightarrow}$ due to correctness of $\consistencypreservationrule{\consistencyrelationset{CR}_{\rightarrow}}$, the right side is also true for all consistency relations in $\consistencyrelationset{CR}_{\rightarrow}$.
        In consequence, we know that $\tupled{\change{\metamodel{M}{1}}'(\model{m}{1}), \change{\metamodel{M}{2}}'(\model{m}{2})} \consistenttomath \consistencyrelationset{CR}_{\rightarrow}$
        \item $\tupled{\change{\metamodel{M}{2}}'(\model{m}{2}), \change{\metamodel{M}{1}}'(\model{m}{1})} \consistenttomath \consistencyrelationset{CR}_{\leftarrow}$:
        This directly follows from correctness of $\consistencypreservationrule{\consistencyrelationset{CR}_{\leftarrow}}$.
    \end{properenumerate}
    In consequence, sequentially applying both consistency preservation rules of an inverse-preserving unidirectional synchronizing transformation ensures consistency to the consistency relations in both directions.
\end{proof}

\mnote{Inverse-preserving unidirectional transformation can emulate synchronizing ones}
As a consequence of the theorem, we know that we can emulate a synchronizing transformation according to \autoref{def:synchronizingtransformation} with an inverse-preserving unidirectional synchronizing transformation according to \autoref{def:inversepreservingunidirectionalsynchronizingtransformation}.
We can execute the unidirectional consistency preservation rules one after another to achieve that the resulting models are consistent to all consistency relations in both direction.

\mnote{Ensuring inverse-preserving property is already a problem of bidirectional transformations}
In fact, it is not trivial to ensure that two unidirectional transformations are inverse-preserving, even if the consistency relations in both directions are the same, which we will also see in our evaluation of errors in \autoref{chap:errors}.
This problem, however, already arises when defining bidirectional transformations.
Transformation languages may derive two unidirectional preservation rules from one specification, so that they are inherently inverse-preserving, or they may allow individual specification of the directions and provide some support for checking that they are aligned with each other, e.g., in the sense that they are inverse-preserving.
This is, however, an isolated and existing topic of research \todo{Add references for that} and a challenge that already has to be solved for a single bidirectional transformation rather than a network, which is why we do not discuss this problem in more detail here and therefore assume given transformations to be inverse-preserving.

\mnote{Directionality gap was discussed, synchronization gap remains}
We have discussed the gap between synchronizing transformations and ordinary transformations defined in transformation languages regarding directionality by decomposing synchronizing transformations into unidirectional ones.
In the following, we discuss the remaining gap of the formalized transformations being synchronizing, whereas practically defined transformations do not have that property.


%%
%% THE SYNCHRONIZATION GAP: After the directionality gap, we need to discuss how to come from synchronizing to non-synchronizing transformations
%%
\subsection{The Synchronization Gap}

\mnote{Practical transformations are not synchronizing}
Still, there is a gap to practical approaches for defining transformations, as existing approaches do usually not support synchronization, i.e., they are not able to process changes in both models, but only in one of them.
Transformation languages usually assume that changes are either made by the developer and are then to be propagated to the other model by the transformation, or they are made by the transformation in reaction to changes to the other model.
The case that developers modify multiple models is sometimes also referred to as a synchronization scenario (although the term is sometimes even used for the simple case of incremental update).
If we consider that scenario, we will refer to it as \emph{concurrent editing} to avoid confusion.

\mnote{In contrast to arbitrary concurrent changes, transformation networks produce less conflicts}
Although the two cases have in common that both instead of only one model involved in a transformation may have been modified, they have a specific difference.
While user changes to both models can be arbitrarily conflicting, changes performed by other transformations in a network should, in general, not be conflicting, especially if the underlying relations are compatible, as discussed in \autoref{chap:compatibility}.
For example, if a user changes an element $A$, whose information needs to be propagated to element $B$, but removes element $B$ as well, this cannot be resolved easily, apart from potentially removing element $A$ as well.
However, as we know from existing approaches for concurrent editing with tools like Git, conflict resolution is not a trivial task~\todo{add cite for difficulty of conflict resolution}.
Such a scenario may, however, not occur in a transformation network, because if transformations remove elements that are to be updated by others, there will obviously be some conflicts in the transformations manifested in an incompatibility of their consistency relations.

\mnote{Definition of ordinary transformations}
An ordinary unidirectional consistency preservation rule as defined in or produced by a transformation language looks as follows:
\begin{align*}
    \consistencypreservationrule{\consistencyrelationset{CR}_{\rightarrow}} : (\metamodelinstanceset{M}{1}, \metamodelinstanceset{M}{2}, \changeuniverse{\metamodel{M}{1}}) \rightarrow \changeuniverse{\metamodel{M}{2}}
\end{align*}
Such a rule is not synchronizing, i.e., it does not consider that the second model was modified as well, like we defined for unidirectional consistency preservation rules in \autoref{def:unidirectionalconsistencypreservationrule}.

\mnote{Passing changed models to consistency preservation rule is not possible}
Given models $\model{m}{1},\model{m}{2}$ and changes $\change{\metamodel{M}{1}},\change{\metamodel{M}{2}}$, if we simply pass the changed model $\model{m}{2}$ to the preservation rule, we call $\consistencypreservationrule{\consistencyrelationset{CR}_{\rightarrow}}(\model{m}{1},\change{\metamodel{M}{2}}(\model{m}{2}),\change{\metamodel{M}{1}})$.
Then, in general, the behavior of the function is undefined.
As defined in \autoref{def:consistencypreservationrulecorrectness}, we only required the function to return a change such that applying all changes produces consistent models if the original models were consistent.
In this case, however, the given models are not necessarily consistent to each other.

%\subsection{Closing the Gap}
\mnote{Emulate synchronizing transformations with non-synchronizing ones}
We thus want to achieve a slight adaptation of those non-synchronizing unidirectional consistency preservation rules, such that they support the case that they support the synchronization case, i.e., that the second model has already been modified.
Informally speaking, we want to emulate unidirectional synchronizing consistency preservation rules with non-synchronizing ones.
Additionally, we want to ensure that if such a rule is inverse-preserving for consistent inputs, it is also inverse-preserving for the case that the second model was already modified, such that executing both rules consecutively ensures consistency to the relations in both directions, as we have already proven for inverse-preserving synchronizing transformations in \autoref{theorem:sequencinginversepreservingtransformations}.

\mnote{Find requirements for non-synchronizing transformations to impose same behavior as synchronizing ones}
We thus want to find out what has to be considered to define a pair of ordinary, non-synchronizing consistency preservation rules $\consistencypreservationrule{\consistencyrelationset{CR}_{\rightarrow}}$ $\consistencypreservationrule{\consistencyrelationset{CR}_{\leftarrow}}$ to emulate a unidirectional synchronizing transformation.
This means that $\consistencypreservationrule{\consistencyrelationset{CR}_{\rightarrow}}$ is able to process an input $\tupled{\model{m}{1},\change{\metamodel{M}{2}}\model{m}{2},\change{\metamodel{M}{1}}}$ and return $\change{\metamodel{M}{2}}'$, such that if $\tupled{\model{m}{1},\model{m}{2}}$ is consistent to $\consistencyrelationset{CR}_{\rightarrow}$, then $\tupled{\change{\metamodel{M}{1}}(\model{m}{1}),\change{\metamodel{M}{2}}' \concatfunction \change{\metamodel{M}{2}}(\model{m}{2})}$ is consistent to $\consistencyrelationset{CR}_{\rightarrow}$ as well (and $\consistencypreservationrule{\consistencyrelationset{CR}_{\leftarrow}}$ analogously).
Additionally, it requires that the transformation induced by the two non-synchronizing consistency preservation rules is inverse-preserving according to \autoref{def:inversepreservingunidirectionalsynchronizingtransformation}, such that sequentially executing them ensures consistency to all consistency relations in both directions.

\mnote{Ensure correctness and inverse-preserving}
To achieve that goal, we discuss all possible combinations of changes made to $\model{m}{1}$ and $\model{m}{2}$ that the consistency preservation rules may need to consider.
We consider the most atomic types of changes that can be performed and ensure that the finding hold for arbitrary change combinations by composition.
For each combination of changes processed by $\consistencypreservationrule{\consistencyrelationset{CR}_{\rightarrow}}$ , we need to find out the following (for $\consistencypreservationrule{\consistencyrelationset{CR}_{\leftarrow}}$ analogously):
\begin{enumerate}
    \item The consistency preservation rule operates correctly, i.e., the result is still consistent to $\consistencyrelationset{CR}_{\rightarrow}$.
    \item The transformation is inverse-preserving, i.e., no relation in $\consistencyrelationset{CR}_{\leftarrow}$ that was fulfilled before is violated afterwards.
\end{enumerate}

\mnote{Case distinction for all change type combinations necessary}
In the following section, we perform a case distinction for all possible combinations of changes, specifically for EMOF-based models as the most common and general formalism to describe metamodels and models.
This allows us to derive which combinations of changes are problematic at all and thus have to be explicitly considered when defining ordinary non-synchronizing transformations to be able to properly use them in a transformation network.
In consequence, it enables transformation developers to construct transformations with ordinary transformation languages that behave like synchronizing transformations required in networks.



% \begin{itemize}
%     \item Define unidirectional consistency preservation rules as above
%     \item Define correctness of unidirectional consistency preservation rules
%     \item Say that we want to investigate what happens when we apply CPR to m1, d(m2) with m1, m2 consistent instead of to m1, m2
% \end{itemize}

% Define ordinary transformations to take deltas in model 1 and produce deltas in model 2
% Say that unidirectional synchronizing transformations take deltas in both models and update the deltas in one of them.
% Refer to fine-grained formalization regarding compatibility, where consistency relations are directional, thus each directional preservation rule preserves consistency according to the consistency relations in one direction.
% Having synchronizing unidirectional transformation, executing both preserves consistency to both unidirectional consistency relations. However, their must be some kind of conformance of the unidirectional transformation to each other (define how this conformance looks like!), so that executing each once does not lead to violations in the other direction. In general, that may not be possible. In fact, each unidirectional transformation should consider the unidirectional consistency relations of both directions.

% So:
% Ordinary unidirectional transformation for Rr: m1, m2, d1 -> d2, such that (d1(m1), d2(m2)) consistent to all relations in Rr (Rl respectively)
% Synchronizing unidirectional transformation: m1, m2, d1, d2 -> d2', such that (d1(m1), d2'(m2)) consistent to all relations Rr and consistent to all relations in Rl to which is was consistent before (thus no violation of further consistency relations)

% Direct consequence: Executing one transformation after the other ensures that models are consistent to alle relations in Rr and relation

% Based on that, we derive how we can use languages that take deltas in model 1 and produce them in model 2 to emulate synchronizing unidirectional transformations that are able to manage deltas in both models and produce deltas in one of them.
% For that, we make the case distinction and derive the creation pattern.
% For each change merge case, we consider that we somehow "merge" the changes. In general, the change of the transformation to m2 will overwrite the previous change to m2. Then we consider that there is another consistency relation affected by the new change. We show whether/why not the other consistency relation can be violated by that change and discuss how to avoid that.


%%%
%%% AVOIDANCE PATTERNS
%%%
\section{Achieving Synchronization} % of Bidirectional Transformations}
% Das Praktische Problem
\label{chap:synchronization:achieving}

\mnote{Synchronizing bidirectional transformation}
We have introduced the notion of synchronizing bidirectional transformations, which can be used within transformation networks in place of synchronizing transformations.
They are composed of two unidirectional consistency preservation rules, which fits to the way how transformations are specified in transformation languages.
In contrast to only being correct, as commonly required of transformations, they need to fulfill the notion of being partial-consistency-improving to be used instead of synchronizing transformations.

\mnote{Partial consistency improvement no canonically achievable}
The knowledge about this requirement, theoretically, gives a transformation developer the ability to define appropriate transformations to be used in transformation networks.
We have discussed that the requirement for transformations to be partial-consistency-improving is reasonable as it reflect intuitive requirements to transformations to always restore more consistency than is violated by their execution.
There is, however, still no canonical way to fulfill that requirement.
It may be possible to define analyses for transformations or even appropriate transformation languages that guarantee that property by construction.
This could, however, even lead to severe restrictions in expressiveness, if analyzability is the primary goal.
In addition, research about synchronizing concurrent changes (e.g.~\cite{hermann2012concurrentSynchronization-FASE,orejas2020IncrementalConcurrentSynchronization-FASE,xiong2013SynchronizingConcurrentUpdates-SoSym,xiong2009parallelUpdates-ICMT} already addresses a comparable problem.
Thus, we do not discuss or investigate such approaches in this thesis.

\mnote{Systematic avoidance of synchronization problems}
We leave it up to transformation developers to thoroughly define their transformations such that they fulfill the required property.
Having precise knowledge about the property that needs to be fulfilled by the transformations already provides a benefit regarding the baseline of using ordinary transformations in a transformation network without knowing how the transformations have to be improved to work properly.
In addition, we discuss a distinction of possible scenarios that can occur when changes need to be synchronized and come up with engineering considerations how to systematically deal with these scenarios.
We identify one essentially problematic scenario and propose a strategy to avoid that problem by proper construction of transformations.
In our evaluation in \autoref{chap:correctness_evaluation}, we will see that it is actually the only occurring and thus most relevant problem scenario that transformation developers have to deal with when developing synchronizing bidirectional transformations.

% \todo{Special case: Changes witness inconsistency. In the previous definition, we can give arbitrary models and even empty changes to the models and still require them to get consistent.}

% We just introduced requirements a developer has to fulfill in his unidirectional transformation. This, theoretically, gives him the ability to define transformation of unidirectional rules to be used as synchronizing ones.
% There is, however, no canonical way to fulfill all the requirements.
% Building a language that ensures all the requirements will be a cumbersome task and is thus out of scope of this thesis, and may, potentially, even lead to severe restrictions to expressiveness, reducing its practical applicability.
% Thus, it can be possible that it is up to engineers to thoroughly define their transformations to ensure these properties.
% Nevertheless, we will make an engineering consideration to distinguish different scenarios that may occur and what has to be considered there in general.
% This leads us to a pattern to follow when developing transformations to avoid a typical problem.
% We will see in our evaluation, that this is actually the most severe problems that transformation developers have to deal with when developing unidirectional CPRs to be used as synchronizing transformations.

% Während wir es dem Entwickler überlassen eine unidirektionale synchronisierende Transformation zu entwickeln, die die entsprechend notwendigen Eigenschaften hat, widmen wir uns noch einem praktischen Problem.
% Konsistenz herstellen tun CPR sowieso. Sie reagieren auf Änderungen und erzeugen Änderungen im Zielmodell, sodass die partielle Konsistenz erhöht wird.
% Bleibt also zu klären, wie das Einführen von Inkonsistenzen durch das Hinzufügen von Elementen verhindert werden kann.
% Dies können wir auf Basis der konkreten Änderungen an Modellen machen oder auf Basis der möglichen Änderungen von Condition Elements.


\subsection{Synchronization Scenarios}

\mnote{Inconsistency introduced by changes}
For the execution of synchronizing bidirectional transformations, we have assumed that inconsistencies are only introduced by changes.
Thus, defining a consistency preservation rule that processes changes in one model, it has to deal with the situation that the other model has been changed as well.
Although this might intuitively lead to the expectation that distinguishing the different types of changes, such as element insertions and removals, helps to identify relevant scenarios, actually the modification of condition elements of the consistency relations rather than individual elements is relevant.

\mnote{No case distinction by models changes types}
If we process a change $\change{\metamodel{M}{1}}$ to model $\model{m}{1}$, and $\model{m}{2}$ was changed by $\change{\metamodel{M}{2}}$ as well,
a consistency preservation rule $\consistencypreservationrule{}^{\rightarrow}$ from $\metamodel{M}{1}$ to $\metamodel{M}{2}$ of a synchronizing bidirectional transformation $\transformation{t}$ produces a change $\change{\metamodel{M}{2}}'$ in the execution of the synchronizing execution step $\function{SyncEx}_{\transformation{t}}^1$.
If we assume that $\change{\metamodel{M}{1}}$ performs a change that introduces a new condition element, $\consistencypreservationrule{}^{\rightarrow}$ is responsible for adding a corresponding element to $\change{\metamodel{M}{2}}(\model{m}{2})$ such that partial consistency between the two is improved, and in the best case already be restored to $1$.
$\consistencypreservationrule{}^{\rightarrow}$ must also consider the change $\change{\metamodel{M}{2}}$, which may have already added an appropriate corresponding element, such that adding a further one may reduce instead of improve partial consistency.
Adding a condition element to a model can, however, not only be the result of adding an element, but also of different types of changes, such as also the change of an attribute or reference.
In fact, it must only be considered that a condition element was added, but not which kind of change introduced it.

% \subsection{Case Distinction for Changes}
% % Erkenntnis: Fallunterscheidung über Changes bringt nichts
% Es ist leicht einzusehen, dass eine Fallunterscheidung über Changes nichts bringt.
% Beispielsweise könnte ein externer Change (d2) und ein Change per CPRr (d2') völlig unabhängige Elemente betreffen (z.B. einmal ein Attribut einer Klasse, einmal eine Referenz einer anderen Klasse), insofern können wir sie nacheinander ausführen. Allerdings können sie zusammen ein neues Condition Element induzieren. Die Rücktransformation CPRl müsste hierfür ein entsprechendes Condition Element in d1(m1) hinzufügen, was induktiv wieder ein neues Condition Element provozieren kann.
% Und das kann passieren, obwohl jede CPR für sich in der Lage ist Konsistenz in einem Schritt herzustellen.
% Somit ist es nicht sinnvoll über Änderungen zu unterscheiden, sondern über Condition Elements, da relevant ist, ob eine Kombination von Änderungen dazu führt, dass sich ein Condition Element ändert, ein neues entsteht oder ein altes entfernt wird.

% Die Kombination zweier Changes kann niemals dazu führen, dass ein Condition Element entfernt wird, was nicht eh entfernt worden wäre, da bereits beim Nicht-Vorhandensein eines Elementes aus dem Condition Element das Condition Element schon nicht mehr im Modell ist.
% Gleiches gilt für die Änderung eines Condition Element. Ein Condition Element ist genau eine Menge von Modellelementen. Entweder ein Change ändert dieses oder nicht, aber nicht erst eine Kombination von Änderungen kann dazu führen.
% Interessant ist die Erzeugung eines Condition Elementes, denn diese kann durch die Kombination mehrere Änderungen entstehen. Erst wenn alle Modellelemente innerhalb eines Condition Elementes erzeugt wurde induziert dies Konsistenzanforderungen.

% \subsection{Case Distinction for Condition Elements}
%\todo{We assume transformation to be correct, because they have to be correct anyway. Thus, if only one model is changed, consistency is achieved after one execution of the transformation.
%Thus, we only have to consider which scenarios can occur when the second model has also been modified}

% Die Unterscheidung über Condition Elements ist sinnvoller, denn genau dann wenn ein solches Element betroffen ist hat das Auswirkungen auf die Konsistenz. Durch welche Art von Änderung das genau entsteht, ist zweitrangig. Man könnte lediglich noch untersuchen, welche Modelländerung zu welcher Änderung eines Condition Elementes führen kann, um dem Entwickler bei der Einschätzung zu helfen.
% \todo{Das könnten wir wirklich noch tun!}

\mnote{Case distinction fpr condition element change types}
We have already discussed in \autoref{chap:synchronization:gap:alignment} that the addition, removal and change of condition elements are the relevant scenarios that can lead to consistency violation.
In case of adding a condition element, an appropriate corresponding element for it may be missing, such that no witness structure for consistency is given.
This requires an appropriate element to be added.
In case of removing a condition element, the element was corresponding to another one, which now has no corresponding element anymore.
This requires the corresponding condition element to be removed.
Changing a condition element can be seen as a modification of model elements such that they represent another condition element of the same condition, thus still belonging to the same consistency relation.
The consistency preservation rule must then update the corresponding condition element appropriately.

\mnote{Intuitive behavior ensuring partial consistency improvement}
That behavior is what consistency preservation rules are actually supposed to implement.
A bidirectional transformation with such consistency preservation rules is inherently supposed to fulfill the property of being partial-consistency-improving, because the elements that have no corresponding elements due to the modification are not part of the maximal consistent subsets before executing the consistency preservation rule.
After executing it, either the corresponding element is removed and thus the model size decreases, or a corresponding element is added, such that the size of maximal consistent subsets improves.

\mnote{Interference of condition elements}
In addition to the above considerations, a transformation may be prevented from being partial-consistency-improving because the addition or removal of a condition element to improve consistency affects further condition elements.
This can occur because these condition elements overlap, i.e., some model elements may be part of several condition elements.
Then, if all elements of a condition element are removed, the other condition element is not present anymore as well.
A consistency preservation rule must thus be carefully defined such that removing one condition element does not lead to the removal of another one, which was actually part of the maximal consistent subset.
Otherwise the consistency preservation rule introduces a new violation of consistency.
The same applies to the scenario of adding condition elements. 
If the addition leads to the introduction of an additional condition element, because some objects in the added condition element together with other existing objects form a condition element of another consistency relation, this introduces an inconsistency if no corresponding element exists yet, thus reducing partial consistency.
If the previously existing elements within the induced condition element were part of the maximal consistent subset, the consistency preservation rule is actually not correct.
If the models were consistent before and only the change to one model is performed, correctness of the consistency preservation rule requires the result to be consistent.
It, however, introduces a further condition element that has no corresponding element, thus the result is not consistent.
If, on the other hand, the previously existing elements within the induced condition element were not part of the maximal consistent subset, it is fine that these elements are still inconsistent, as the consistency preservation rules still need to process them anyway.
These problems are comparable to those of fine-grained transformation rules, as discussed in \autoref{chap:correctness:finegrained:relations}, which need to be defined such that one rule does not lead to the violation of the consistency relation of another.

\mnote{Conflicts between condition element changes}
The previous considerations reflected the case that only one model was changed.
If the other model was changed as well, the combinations of changes can lead to specific situations that have to be handled differently.
We therefore distinguish the addition, removal or change of a condition element to be processed by the consistency preservation rule and discuss what conflicts may occur by changes performed in the other model.
Changes of condition elements are, in practice, traced by the usage of trace models that store trace links between corresponding elements.
It can be seen as a representation of the witness structure we defined for identifying consistency.
If elements become changed, the trace links still exist and indicate which corresponding elements need to be adapted.
According to the defined notion of consistency, these potential conflicts are just based on the question whether appropriate condition elements exist or not.
\begin{properdescription}
    \item[Addition:] Whenever a condition element is added to one model, it must be ensured that a corresponding condition element in the other model exists.
    In the case that both models were consistent before, the corresponding element cannot already be present in the other model and thus has to be added.
    If the other model has been changed, an appropriate corresponding element may already have been added.
    That scenario has to be explicitly considered to avoid a duplicate creation of the condition element, which then may lead to a violation of consistency that cannot be resolved by adding further elements anymore.
    \item[Removal:] Whenever a condition element is removed from one model, the corresponding condition element must be removed from the other model, as otherwise its corresponding element is missing, which would violate consistency.
    If the models were consistent before, the corresponding element must necessarily exist and thus can be removed.
    If the corresponding condition element is not present because it was removed from the other model already, the element can but also does not need to be removed anymore.
    It must only be considered that the existence of the corresponding element cannot be assumed.
    \item[Change:] When model elements are changed such that they represent a different condition element of the same condition as before, they usually also require the corresponding element to be updated to represent the condition element of an applicable consistency relation pair.
    If the corresponding element is removed, the opposite consistency preservation rule will remove the changed condition element anyway to restore consistency.
    Thus, the consistency preservation rule must only consider that the corresponding element may have been removed, but must perform no changes.
    If the corresponding element was changed, which is identified by the trace model still containing a link to a changed element, it must be adapted such that both elements form a consistency relation pair again.
    The modification to the corresponding element will then be propagated back by the opposite consistency preservation rule.
\end{properdescription}

\mnote{Problematic scenario is duplicate addition of condition elements}
In summary, we have to deal with two specific situations that can occur when the target model of a consistency preservation rule may have been changed.
First, when adding condition elements, their corresponding elements may already exist in the other model.
Second, when removing condition elements, their corresponding elements may have already been removed from the other model.
While the second scenario is easy to handle by doing nothing whenever the corresponding elements of removed elements are not present anymore, the first scenario requires an approach to identify whether corresponding elements already exist.
While existing corresponding elements can be retrieved from a trace model, no trace links exist for these elements yet.
In the following, we discuss an approach to find corresponding elements.

% Inkonsistenzen werden dadurch eingeführt, dass neue Condition Elements hinzugefügt werden, für die es keine eindeutigen korrespondierenden Elemente gibt, d.h. keine Witness-Struktur aufgebaut werden kann.
% Dies ist der Fall sein, wenn Elemente erzeugt werden, um notwendige Konsistenz herzustellen, dadurch aber neue Condition Elements induziert werden, die wieder Konsistenzhaltung bedürfen.
% Entsteht das neue Condition Element aus Modellelementen, die alle im partiell konsistenten Teilmodell liegen, dann ist die CPR falsch.
% Für dieses partielle Modell, welches vorher konsistent war, und die zugehörige Änderung müsste die CPR gem. Definition Korrektheit ein konsistentes Ergebnis produzieren, kann also keine Condition Elements in m2 induzieren, die keine korrespondierenden Elemente in m2 haben, da die Modelle dann nicht konsistent sind.
% Entsteht das neue Condition Element mit Modellelementen, die nicht im partiell konsistenten Teilmodell liegen, ist das okay, da hierfür keine Konsistenz verlangt ist.

% Probleme mit denen CPR durch nebenläufige Änderung umgehen können muss:
% 1. Neue Condition Elements sind bereits vorhanden
% 2. Condition Elements wurden auf Zielseite modifiziert
% 3. Condition Elements wurden auf Zielseite gelöscht

% 1. Das müssen wir herausfinden, siehe Trace-Modell
% 2. CPR muss wissen, welche korrespondierenden Elemente es vorher gab und diese auf der Zielseite entsprechend anpassen. Im Prinzip muss die CPR nur betrachten, welche Bedingungen die Elemente erfüllen müssen und diese wiederherstellen. Hier müssen aber nur die Änderungen in beide Richtungen entsprechend übertragen werden.
% 3. CPR muss nichts tun, da durch das Löschen auf der Zielseite auch ein Löschen auf der Quellseite nötig ist, um wieder eine passende Witness-Struktur zu induzieren. Das kann nur die gegenläufige CPR sicherstellen.

% Konsequenz: 1. ist der Fall, den wir uns noch genauer anschauen müssen.

% % ZUSAMMENFASSUNG DER SZENARIEN BEI BEARBEITUNG VON MODIFIZIERTEM M2
% Mögliche Situationen wenn wir an 1->2 statt m2 direkt d2(m2) übergeben:
% 1. d2(m2) ändert Elemente in einem Condition Element, die 1->2 auch ändern muss. Dann ändert 1->2 einfach partiell die Elemente, und 2->1 macht dann nachher den Rest (nachweisen, dass das geht)
% 2. d2(m2) fügt neue Condition Elemente hinzu: Unproblematisch, wenn sie nicht mit Elemente in d1(m1) zusammenhängen (d.h. keine Witness-Struktur zwischen d1(m1) und d2(m2) die Elemente verbindet), dann werden sie von 2->1 bearbeitet. Wenn sie mit Element in d1(m1) zusammenhängen (also es eine Witness-Struktur gibt, die Elemente verbindet, auch wenn das noch nicht im Trace-Modell steht), muss hier das Matching passieren! Hier ist natürlich von einer passenden Granularität der Konsistenzrelationen auszugehen. Bspw. kann es reichen, dass eine eine Person/Resident/... mit einem passenden Namen vorhanden ist, ohne dass irgendwelche anderen Werte übereinstimmen (das könnte dann in weiteren Konsistenzrelationen stehen). Dann gäbe es z.B. eine Konsistenzrelation, die angibt, dass für jede Person ein Resident mit dem gleichen Namen vorhanden sein muss, und dann noch eine die beschreibt, dass für Person/Resident-Paare mit dem gleichen Namen auch andere Attribute passend abgebildet werden müssen. Dies ist aber in Transformationssprachen eh implizit so realisiert und lässt sich mit unserem Formalismus für feingranulare Konsistenzrelationen problemlos abbilden.
% 3. d2(m2) entfernt Condition Elemente: Unproblematisch, wenn die Verarbeitung von d1 nicht auf die entsprechenden Condition Elemente zugreifen muss, also nichts in m1 geändert wurde was zu gelöschten Elementen in d2(m2) korrespondiert. Ansonsten können keine Informationen übertragen werden, da die Witness-Struktur nicht mehr gilt, also macht 1->2 an der Stelle nicht weiter und 2->1 übernimmt den Abbau der Elemente in m1.


\subsection{Identification of Existing Corresponding Elements}
\label{chap:synchronization:achieving:identification}

\mnote{Identification of corresponding elements}
Whenever a condition element is added, which requires a corresponding element to exist in the other model, the consistency preservation rule will usually create appropriate elements in the other model.
This is due to the reason that in the case when that model may not have been modified as well, these elements cannot already exist.
In the synchronization case, however, the change to the other model may have already introduced those elements, thus it is necessary to find them to avoid their duplicate creation.

\mnote{Trace links}
In previous work~\owncite{klare2019icmt}, we have proposed a strategy to identify such corresponding elements.
Transformation languages usually use trace models to store the information which elements are corresponding to each other.
Thus, whenever the consistency preservation rule in the opposite direction added the element whose addition is currently processed, a trace link already exists.
When the corresponding elements were created by different transformations, however, there will not be a trace link between them.

\mnote{Transitive trace links}
An intuitive attempt would be to use the trace links of the other transformations across which they were created.
For example, if for a \gls{PCM} component a \gls{UML} class is created and for this \gls{UML} class a Java class is created, then there are trace links between the \gls{PCM} component and the \gls{UML} class, as well as between the \gls{UML} class and the Java class.
Synchronizing the addition of the \gls{PCM} component and the Java class should not result in a redundant addition of, for example, a further Java class.
Resolving the existing trace links transitively is, however, not a solution.
In this case, a unique one-to-one mapping exists that actually traces the \gls{PCM} component to the corresponding Java class.
It would, however, also be possible that a \gls{PCM} component has trace links to several elements in the Java model across \gls{UML}.
If those elements are even multiple classes, such as one public and one internal utility class, but the consistency relation between \gls{PCM} and Java only requires one Java class for a \gls{PCM} component, it would be unclear which to select.

\mnote{Semantics of trace links}
Transformation languages usually tag trace links with additional information, for example, containing the transformation rule that created them, to distinguish links to instances of the same class.
Since these tags are created by other transformations, considering them would violate our assumption of independent development of transformations and modular reuse.
Even worse, it could also be the case that another third class is required by the consistency relation between \gls{PCM} and Java.
Finally, it is up to the actual consistency relation to define when elements are to be considered corresponding, because there may be more semantics beyond the types of the elements related by a trace link that determines how they belong together.

\mnote{Information for identifying corresponding elements is given in consistency relation}
Thus, whether corresponding elements already exist cannot be identified by transitively resolving trace links of other transformations, but only by considering the two involved models.
The information to identify whether elements can be considered corresponding is precisely given in the consistency relation.
For example, if the relation specifies that, in a very simplified notion, a \gls{PCM} component is consistent to all Java classes that have the same name, no matter what implementation the class contains, then if any class with the name of the \gls{PCM} component is found in the Java code, it can be considered corresponding.

\mnote{Corresponding element identification level}
We come up with three levels of identifying corresponding elements:
\begin{properdescription}
    \item[Explicit Unique:] The information that elements correspond is unique and represented explicitly, e.g., within a trace model. %Existing transformation languages usually use this technique.
    \item[Implicit Unique:] The information that elements correspond is unique, but represented implicitly, e.g., in terms of key information within the models, such as element names. %types and element names.
    \item[Non-Unique:] Without unique information, heuristics based on ambiguous information or transitive resolution of indirect trace links must be used.
\end{properdescription}

\mnote{Explicit and unique identification by trace links}
In the best case, a trace link already exists between the corresponding elements. This can be because a consistency preservation rule in one direction created the corresponding element and added the trace link. Then the consistency preservation rule in the other direction processes the change that introduced the corresponding element, but now can already retrieve the trace link.
This is what we call \emph{explicit unique} information, because the information is represented explicitly and unambiguously in the trace model.

\mnote{Implicit and unique identification by key information}
If no trace link exists, like in the synchronization scenario, the information specified in consistency relations to identify corresponding elements needs to be used.
This can be considered key information, because the information is used as the key to identify corresponding elements.
To this end, the model has to be queried for elements with the given information.
The transformation language \gls{QVTR} already provides a language construct to specify such key information within transformation rules~\cite[7.10.2.]{qvt}.
%\todo{Compare to QVT-R description in~\cite[7.10.2]{qvt}, which specifies that elements are created if no elements match the defined key property values specified in the object template}
We call this information \emph{implicit unique}, because elements can be unambiguously identified but rely on implicit information within the models rather than explicit traces.
Note that in case that multiple corresponding elements are found by matching key information, any of them can be selected.
It is up to the consistency preservation rule for the other direction to add further elements such that corresponding elements for all of them are added, such that a valid witness structure is induced.

\mnote{Non-unique identification by heuristics}
In the worst case, no unique information is given.
Precisely following our formalism, this scenario can never occur, because each consistency relation defines the necessary key information.
Thus, this scenario can only occur in practice with a relaxed notion of consistency.
This can be the case when for an element a corresponding one is always created, containing some related information, but no unique information to identify that the two are corresponding is given.
In that case, only trace links identify that the elements are corresponding.
Thus, if other transformations created the element and thus no direct trace link exists, it is impossible to identify that these elements are to be corresponding.
Since no information to identify that the elements should be corresponding is present anyway and since this requires a relaxed consistency notion, we assume this scenario unlikely to occur at all and did not face it in our evaluation at any time.
If, nevertheless, this scenario occurs, only heuristics can be used to identify corresponding elements without any guarantee of success.
It would also be possible to involve the developer and let him decide whether an element should be considered corresponding.

\mnote{Extension of ordinary transformations}
In summary, it is necessary that transformation developers use key information for identifying corresponding elements based on \emph{implicit unique} information in addition to the usage of \emph{explicit unique} information in terms of trace links. %, like already used for ordinary transformations.
In case that corresponding elements are found based on implicit unique information, they need to establish a trace link for the elements.
We define this behavior in \autoref{algo:synchronization:find_corresponding_elements}, which is an extended version of an algorithm~\owncite[Alg.~1]{saglam2020ma} defined in the Master's thesis of \citeauthor{saglam2020ma}, which was supervised by the author of this thesis, and adapted to our formalism.

\begin{algorithm}
    \begin{algorithmic}[1]
    \Procedure{\function{FindCorresponding}}{$\consistencyrelation{CR}{}, \conditionelement{c}{l}, \model{m}{2}, \model{traces}{}$}
        \State $\mathvariable{tracedElements}$ $\leftarrow$ $\setted{\conditionelement{c}{r} \mid \tupled{\conditionelement{c}{l}, \conditionelement{c}{r}} \in \model{traces}{}}$
        \For{$\conditionelement{c}{r} \in \mathvariable{tracedElements}$} \label{algo:synchronization:findcorrespondingelements:line:explicit}
            \If{$\tupled{\conditionelement{c}{l}, \conditionelement{c}{r}} \in \consistencyrelation{CR}{}$}
                \State \Return{\textsc{true}}
            \EndIf
        \EndFor
        \For{$\conditionelement{c}{r} \in \mathcal{P}(\model{m}{2})$} \label{algo:synchronization:findcorrespondingelements:line:implicit}
            \If{$\tupled{\conditionelement{c}{l}, \conditionelement{c}{r}} \in \consistencyrelation{CR}{}$}
                \State $\model{trace}{}$ $\leftarrow$ $\model{trace}{} \cup \setted{\tupled{\conditionelement{c}{l},\conditionelement{c}{r}}}$
                \State \Return{\textsc{true}}
            \EndIf 
        \EndFor
        \State \Return{\textsc{false}}
    \EndProcedure
\end{algorithmic}
    \caption[Retrieval of corresponding elements]{Retrieval of corresponding elements.}
    \label{algo:synchronization:find_corresponding_elements}
\end{algorithm}

\mnote{Algorithm for finding corresponding elements}
\autoref{algo:synchronization:find_corresponding_elements} receives the consistency relation for which corresponding elements shall be found, the condition element $\conditionelement{c}{l}$ of the condition $\condition{c}{l,\consistencyrelation{CR}{}}$ that was added to model $\model{m}{1}$ for which corresponding elements shall be found or created, the second model $\model{m}{2}$ in which the corresponding elements shall be searched, and the trace model $\model{m}{\mathvariable{traces}} \subseteq \mathcal{P}(\model{m}{1}) \times \mathcal{P}(\model{m}{2})$ containing pairs of elements in $\model{m}{1}$ and $\model{m}{2}$, which represents a combination of witness structures for consistency relations between metamodels $\metamodel{M}{1}$ and $\metamodel{M}{2}$. 
The algorithm first retrieves all corresponding elements for the condition element from the trace model and then in the loop in \autoref{algo:synchronization:find_corresponding_elements:line:explicit} checks whether any of the corresponding elements according to the trace model is a corresponding element in the consistency relation $\consistencyrelation{CR}{}$.
If this is the case, a corresponding element $\conditionelement{c}{r}$ is found and the procedure returns $\conditionelement{c}{r}$.
Otherwise, model $\model{m}{2}$ is browsed for the existence of a corresponding element in the loop starting in \autoref{algo:synchronization:find_corresponding_elements:line:implicit}.
It considers all subset of $\model{m}{2}$, i.e., the potency set $\mathcal{P}(\model{m}{2})$, of which each could be such a corresponding element.
If one of them is corresponding according to $\consistencyrelation{CR}{}$, then the pair $\tupled{\conditionelement{c}{l},\conditionelement{c}{r}}$ is added to the trace model $\model{m}{\mathvariable{trace}}$ as an appropriate trace link and the procedure returns the found element $\conditionelement{c}{r}$.
If no such element is found, the procedure returns $\bot$ to indicate that no corresponding element is found and thus has to be created by the consistency preservation rule.

\mnote{Efficient identification by key}
The loop in \autoref{algo:synchronization:find_corresponding_elements:line:implicit} is defined in a rather inefficient way, but describes its purpose in the most general way.
In a practical implementation it may not consider every subset of the model $\model{m}{2}$, but instead retrieve all candidate elements, for example, by filtering the model elements by their class.
Depending on the implementation of the modeling framework, different possibilities to efficiently find specific elements can be used.
The implementation of \gls{EMF}, for example, provides functions that yield all instances of a specific class.

\mnote{Application situation of algorithm}
The transformation developer has to apply this algorithm every time he or she specifies the creation of corresponding elements because a change adds a condition element.
This ensures that applying the bidirectional transformation to the synchronization case properly handles the situation that a change has already created the corresponding elements to ensure that the resulting transformation is partial-consistency-improving.

\mnote{Partial consistency improvement not proven}
In contrast to the insights of the previous sections, the engineering considerations we have made in this section are not completely formally founded and proven.
Thus, we have not proven that if a transformation developer follows the discussed rules for the construction of consistency preservation rules and applies the $\function{FindCorresponding}$ function whenever condition elements are created leads to a synchronizing bidirectional transformation, i.e., a bidirectional transformation that fulfills the requirement of begin partial-consistency-improving.
Although we derived the insights from thorough argumentation, we also validate them in the evaluation in \autoref{chap:correctness_evaluation}.

% \begin{copiedFrom}{ICMT}

% FORMERLY: \subsection{Matching Elements in Operationalizations}
%\subsection{Identification of Existing Corresponding Elements}
% \label{chap:prevention:interoperability:matching}

% To avoid failures due to mistakes at the operationalization level, transformations must respect that other transformations may have already created elements.
% In the binary case, this is unnecessary.
% A single incremental \ac{BX} can assume that elements are either created by the user, %and then are input of the transformations
% or were created by the transformation itself.
% To identify corresponding elements, transformation languages usually use trace models, which are created by the transformations.
% When \acp{BX} are combined to networks, %elements may also be created by other transformations.
% %In consequence, 
% direct trace links may be missing because a sequence of other transformations created the elements and trace links only indirectly across elements in other models.
% %Thus, it is necessary to establish direct trace links between corresponding elements.´
% In this scenario, corresponding elements can be matched by information at three levels:
% %Such element matching can be performed on three levels:
% \begin{enumerate}
%     \item \emph{Explicit unique}: The information that elements correspond is unique and represented explicitly, e.g., within a trace model. %Existing transformation languages usually use this technique.
%     \item \emph{Implicit unique}: The information that elements correspond is unique, but represented implicitly, e.g., in terms of key information within the models such as element names. %types and element names.
%     \item \emph{Non-unique}: If no unique information exists, heuristics must be used, e.g., based on ambiguous information or transitive resolution of indirect trace links.
% \end{enumerate}
% \todo{Give examples for each case to show that they actually occur}

% Indirect trace links, which link elements transitively across other models, usually exist for elements that correspond, because other transformations have already created them.
% Nevertheless, indirect trace links cannot be used to unambiguously identify such elements.
% An element can correspond to multiple elements in another model, which is why most transformation languages offer tagging of trace links with additional information to identify the correct element.
% %For example, a component in an architecture description could be mapped to two classes in an object-oriented design, one providing the component implementation and one providing utilities.
% %The relevant corresponding element can be retrieved if the traces are tagged with the information that one class is the implementation and one is a utility.
% For example, a language may tag trace links with the transformation rule they were instantiated in.
% This is helpful in the bidirectional case, but when links are resolved transitively, these tags have been created by other, independently developed transformations, and are thus unknown.
% %If such tags would be considered, transformations would depend on tags of other transformations and could thus not be developed independently anymore.
% Therefore, resolving indirect trace links is only a heuristic, but does not unambiguously retrieve corresponding elements.

% % Explain how to match rules on three different levels, what the levels can provide etc.

% % \begin{enumerate}
% %     \item Direct Correspondences
% %     \item Key information
% %     \item Heuristics: Indirect correspondences, potentially ambiguous information
% % \end{enumerate}

% Finally, it is up to the transformation engine or the transformation developer %, depending on the provided abstraction level, 
% to ensure that elements are correctly matched.
% In contrast to the bidirectional case, direct trace links cannot be assumed in case of networks of \acp{BX}.
% Therefore, key information within the models must always be considered to identify matching elements.
% Whenever direct trace links or unique key information exists, relevant elements can be unambiguously matched.
% In all other cases, heuristics must be used, which potentially leads to failures.

% \end{copiedFrom} % ICMT


\subsection{Model Changes To Condition Element Changes}
\label{chap:synchronization:achieving:changes}

\mnote{Condition element changes induced by model element changes}
The previous discussions distinguished different change scenarios for condition elements, as those are relevant when considering the synchronization case of bidirectional transformations.
A transformation does, however, not receive changes of condition elements but changes of actual model elements.
These then eventually lead to the addition, removal or change of a condition element.
Thus, a transformation developer needs to decide which model changes introduce which modifications of condition elements to determine appropriate behavior of the consistency preservation rules.

\mnote{EMOF- and Ecore-based models}
The possible types of model changes are induced by the used modeling formalism, as the \metametamodel defines which changes can be performed in models.
The \gls{EMOF} is the most common standard describing a modeling formalism, on which Ecore, the \metametamodel of the \gls{EMF}, is based.
Our modeling formalism introduced in \autoref{chap:networks:models} is conforming to \gls{EMOF}, which is why we consider changes in \gls{EMOF}- and Ecore-based models.

% TAKEN FROM SYNCHRONIZATION SCENARIOS SECTION
% We map model changes to potential changes of condition elements, so we find out when a developer has to consider what may happen
%Since we consider the practical realization of such preservation rules with ordinary transformation languages, we also specifically consider the changes that can be processed by those transformation languages.
%Thus, we focus on the types of changes that can be performed in EMOF-based and conforming Ecore-based models.

\begin{figure}
    \centering
    \begin{forest}%
for tree={parent anchor=south,
         child anchor=north,
%          l+=1cm,
%          fill=black!10,
         draw,
         delay={content={\strut #1}},
%          node distance=2ex and 1ex,
         },
featuremandatory/.style={tikz={\node[draw,fill=black!60,inner sep=2pt,circle,anchor=south,yshift=-1pt]at(.north){};}},
featureoptional/.style={tikz={\node[draw,fill=white,inner sep=2pt,circle,anchor=south,yshift=-1pt]at(.north){};}},
[Change
   [Atomic
      [Operation
         [Content
            [Additive]
            [Subtractive]
         ]
         [Order
            [Permute,featuremandatory]
         ]
      ]
      [Target,featuremandatory
         [Root]
         [Feature
            [Type,featuremandatory
               [Attribute]
               [Reference]
            ]
            [Cardinality,featuremandatory
               [Single]
               [Multi]
            ]
         ]
      ]
      [Existential,featureoptional
         [Create]
         [Delete]
      ]
   ]
   [Compound
      [Unset]
      [Move]
      [Replace]
   ]
]      
% fill angles 
\foreach \i/\j/\k in {!1/!/!2,!21/!2/!23,!111/!11/!112,!121/!12/!122,!12211/!1221/!12212,!12221/!1222/!12222,!131/!13/!132}
{
\coordinate (A) at (\i.north);
\coordinate (O) at (\j.south);
\coordinate (B) at (\k.north);
\featurexor{A}{O}{B}
}
\foreach \i/\j/\k in {!1111/!111/!1112}
{
\coordinate (A) at (\i.north);
\coordinate (O) at (\j.south);
\coordinate (B) at (\k.north);
\featureor{A}{O}{B}
}
\node [yshift=-6ex,xshift=-30ex,anchor=north west] at (!12211) {constraints:};
\node [yshift=-9ex,xshift=-30ex,anchor=north west] at (!12211) {1.\ Permute $\Rightarrow$ Multi};
\node [yshift=-12ex,xshift=-30ex,anchor=north west] at (!12211) {2.\ (Multi $\wedge$ Content) $\Rightarrow$ (Additive $\oplus$ Subtractive)};
\node [yshift=-15ex,xshift=-30ex,anchor=north west] at (!12211) {3.\ Single $\Rightarrow$ (Additive $\wedge$ Subtractive)};
\node [yshift=-6ex,anchor=north west] at (!12221) {4.\ Existential $\Rightarrow$ (Root $\oplus$ Reference)};
\node [yshift=-9ex,anchor=north west] at (!12221) {5.\ Create $\Rightarrow$ (Additive $\oplus$ Root)};
\node [yshift=-12ex,anchor=north west] at (!12221) {6.\ Delete $\Rightarrow$ (Subtractive $\oplus$ Root)};
\node [yshift=-15ex,anchor=north west] at (!12221) {7.\ Root $\Rightarrow$ (Additive $\oplus$ Subtractive)};
% \begin{enumerate}[leftmargin=*]
%  \item Permute $\shortimplies$ Multi
%  \item (Multi $\wedge$ Content) $\shortimplies$ (Additive XOR Subtractive)
%  \item Single $\shortimplies$ (Additive $\wedge$ Subtractive)
% \end{enumerate}
\end{forest}
    \caption[Feature model for changes in Ecore-based models]{Feature model for changes in Ecore-based models. Adapted from~\cite[Fig.~5.3]{kramer2017a}.}
    \label{fig:synchronization:change_feature_model}
\end{figure}

\mnote{Possible change types}
\citeauthor{kramer2017a} proposes feature models that express all kinds of possible changes in \gls{EMOF}-based models~\cite[Fig. 5.2]{kramer2017a} and Ecore-based models~\cite[Fig.~5.3]{kramer2017a}.
Since \gls{EMOF} and Ecore are rather similar (see \autoref{chap:foundations:formalisms:ecore}), we focus on Ecore as the practically realized modeling formalism.

\mnote{Corrections of existing feature model}
We depict a modified version of the feature model for changes in Ecore-based models in \autoref{fig:synchronization:change_feature_model}.
In comparison to the original model by \textcite[Fig. 5.3]{kramer2017a}, we made the following changes:
\begin{properdescription}
    \item[No Compound Changes:] We do not consider compound changes, because they are simply compositions of atomic changes and thus do not need to be considered explicitly.
    \item[No Permutation:] We removed the \emph{Permutation} feature, because it can be considered as a compound change of a subtractive and additive multi-valued feature change. Whether or not permutation rather than the removal and addition is relevant is up to the interpretation of the change sequence and is comparable to moving an element from one reference to another, which is also modeled as a compound change.
    \item[Mandatory Content:] We made the \emph{Content} feature mandatory, as due to the removal of the permutation every change is either additive or subtractive.
    \item[Constraints Reduction:] We reduced the constraints to those that are still relevant after performing the previously discussed changes.
    \item[Error Corrections:] We fixed an error in the constraints of the original model. They required a \emph{Create} change of a root element to be subtractive, which does not make sense. We corrected that error by simplification.
\end{properdescription}

\mnote{Case distinction of model changes}
The set of all possible change types in Ecore-based models is given by enumerating all valid configurations of the feature model.
We discuss for each of the resulting changes the types of condition element changes it may induce.
\begin{properdescription}
    \item[Additive Root Change (Possibly Create):] Adding a root element can lead to the addition of a condition element, which consists only of this root element. It may not induce a change or removal of a condition element.
    \item[Subtractive Root Change (Possibly Delete):] The removal of a root element can lead to the removal of a condition element, which involves the root element. Since it completely removes an element, it can neither lead to a change nor to an addition of a condition element.
    \item[Single-valued Attribute Change:] Changing an attribute can lead to either an addition, removal or change of a condition element. The change may lead to an element that now, potentially together with other elements, forms a condition element. It may also lead to a different condition element of the same condition, e.g., by renaming an element. Finally, it can also lead to an element that is not present in a condition anymore. This applies no matter whether the attribute change is only additive, only subtractive, or both, thus whether it adds, removes or replaces the attribute value.
    \item[Additive Multi-valued Attribute Change:] Adding a value to a multi-valued attribute can lead to either an addition, removal or change of a condition element. The change can lead to the situation that the element is now part of a condition element, is not part of a condition element anymore, or that it represents a different condition element of the same condition and is thus comparable to the change of a single-valued attribute.
    \item[Subtractive Multi-valued Attribute Change:] Removing a value from a multi-valued attribute can lead to either an addition, removal or change of a condition element, due to the same reasons as the additive multi-valued attribute change.
    \item[Single-valued Reference Change (Possibly Create/Delete):] Changing a reference can lead to either an addition, removal or change of a condition element, due to the same reasons as for single-valued attribute changes. This is even independent from whether the added or removed element is created or deleted, respectively.
    \item[Additive Multi-valued Reference Change (Possibly Create):] Adding a value to a multi-valued reference can lead to either an addition, removal or change of a condition element, due to the same reasons as for adding an attribute to a multi-valued attribute. Like for single-valued reference changes this is even independent from whether the element was created or already existed before.
    \item[Subtractive Multi-valued Reference Change (Possibly Delete):] The removal of a value from a multi-valued reference can lead to either an addition, removal or change of a condition element, due to the same reasons as for removing an attribute from a multi-valued attribute. Like for single-valued reference changes and additive multi-valued reference changes this is even independent from whether the element was created or already existed before.
\end{properdescription}

\mnote{Relation between condition and model element changes}
It is easy to see that except for root changes each type of model change can lead to any kind of condition element change, because almost every type of change can lead to the situation that model elements form a condition element or do not form a condition element anymore.
There may be a missing reference or attribute value, or even a superfluous reference or attribute value, after whose change the model elements form a condition element.
This conforms to the notion of creating a corresponding element whenever all conditions for some model elements are fulfilled in the \gls{QVTR}-like \emph{Mappings Language}~\cite[p.~283]{kramer2017a}.
Since all types of changes can lead to the fulfillment of conditions, the addition of a condition element is not tied to a specific type of change.

\mnote{Relevant changes restriction depending on consistency relation}
Depending on the specific consistency relation, there may, however, be some change types that are not relevant in that case.
For example, if a consistency relation puts two model elements having only reference values into relation, then no attribute change will lead to the addition, removal or change of a condition element of that consistency relation.

\mnote{Application cases for corresponding elements identification}
The specific case of identifying corresponding elements during synchronization discussed in the previous subsection needs to be considered whenever a condition element was added.
Since this can occur because of any type of change except for removals of root elements, we cannot make any general restrictions on the types of model changes that need to be explicitly considered for the synchronization case.
The transformation developer must decide after which changes a condition element may be created, independent from whether corresponding elements may already exist or not.
Thus, he or she makes this decision anyway and must only extend the existing logic for finding corresponding elements according to the given algorithm.

% FROM MAX:
% Different information and case distinctions are necessary to describe all possible model changes for modelling languages that follow the Essential Meta Object Facility (EMOF) standard or the Ecore variant. 
% Both meta-modelling languages and the differences between them are described in \autoref{chap:foundations:modeling:models}.
% Only two differences have a major effect on our change modelling language and the specifications language that use them:
% \begin{enumerate}%[1.]
% \item In EMOF, properties can be typed using metaclasses or using other data types, but in Ecore these are distinguished as references and attributes.
% \item Ecore requires that all elements except for a root element are contained in exactly one container and EMOF only requires that all elements have at most one container~\cite[pp.~31-32]{mof}.
% \end{enumerate}
% If we only consider these two differences, then Ecore can be seen as a refinement of EMOF, which only adds a more fine-grained distinction of properties and further containment restrictions.
% Because of this refinement relation, we will first describe which information is necessary to represent model changes of EMOF-based models and then add further information and distinctions for Ecore-based models.
% Finally, we briefly explain how we made all this information available in practice using a change modelling language.


%%% 
%%% SUMMARY
%%%
\section{Summary}

In this chapter, we have discussed how synchronizing transformations, as required in transformation networks, can be defined with existing transformation languages.
To this end, we have defined synchronizing bidirectional transformations as an extension of bidirectional transformations specified in transformation languages.
We have formally proven that these transformations always terminate consistently and have equal expressiveness than synchronizing transformations.
Finally, we have identified properties and proposed an algorithm to be implemented by a transformation specified in a transformation language to be synchronizing.
We close this chapter with the following central insight.

\begin{insight}[Synchronization]
    Synchronizing transformations, as required in transformation networks, process pairs of models that both may have been and need to be modified.
    In contrast, ordinary bidirectional transformations consist of two unidirectional consistency preservation rules, each of them accepting changes in one model and updating the other.
    We have shown that if changes have been performed to both models, the consistency preservation rules cannot be sequenced such that they produce consistent results.
    By requiring that a bidirectional transformation fulfills a notion of being \emph{partial-consistency-improving}, we were able to define an execution algorithm for it that delivers consistent models after a finite number of execution steps.
    In return, we were able to formally prove that such transformations have equal expressiveness than synchronizing transformations as required for transformation networks.
    Finally, we found that a transformation developer needs to consider only few situations explicitly to make a bidirectional transformation partial-consistency-improving. 
    The most important situation is that a transformation creates elements that already exist, because another transformation already created them, for which we provide an algorithm to avoid issues due to duplicate element creation already by construction.
    In consequence, synchronizing transformations can be constructed with existing transformation languages by fulfilling an additional property for which we provide a constructive strategy and without knowing about other transformations to combine them with.
\end{insight}
