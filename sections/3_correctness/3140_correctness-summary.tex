\section{Summary}
\label{chap:correctness:summary}

\mnote{Insight}
In this chapter, we have discussed notions of correctness for transformation networks and the artifacts it consist of, and we have precisely defined that notion that is relevant for the context of this thesis.
In summary, we provided the following insight in this chapter.

\begin{insight}[Correctness Notion]
    A reasonable notion of correctness for networks of modular, independently developed transformations consists of correctness of the single transformations, which need to be synchronizing, and correctness of the application function that determines an execution order of the transformations.
    An application function may not be able to return a result due to different reasons, such as transformations not being applicable to specific changes, the absence of an execution order of the transformations that leads to consistent models, or the inability to find such an order.
    Thus, in comparison to correctness, the degree of conservativeness is the more important property of an application function, which indicates how often the function does not deliver a result although there is an order of transformations that would restore consistency.
    Additionally, although theoretically not relevant for correctness, the relations defining when models are considered consistent have to fulfill some notion of compatibility to be useful, as they can otherwise prevent transformations from finding consistent models.
\end{insight}

\mnote{Achieving correct transformation networks}
In the following chapters, we will thus define a notion of compatibility for consistency relations, discuss how correctness of the individual synchronizing transformations for achieving local consistency can be achieved and finally how a correct and appropriate application function to perform the orchestration for achieving global consistency can be defined.
In summary, these following contributions together will allow to develop what we defined as a \emph{correct} transformation network.



% \section{Summary}

% Central Insights:
% \begin{itemize}
%     \item In networks, we need compatible consistency relations -> first RQ
%     \item In networks, we need synchronizing rather than bidirectional transformations -> second RQ
%     \item In networks, we need orchestration functions -> third RQ
%     \item Correctness is not the problem, optimality is the problem
%     \item We can only check dynamically whether a consistent state was reached due to Halting Problem. We cannot guarantee to always find a consistent state
% \end{itemize}