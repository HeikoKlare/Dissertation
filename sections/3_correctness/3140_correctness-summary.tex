\section{Summary}
\label{chap:correctness:summary}

In this chapter, we have discussed notions of correctness for transformation networks and precisely defined the one that is relevant for the context of this thesis.
In summary, we made the following insight in this chapter.

\begin{insight}[Correctness Notion]
    A reasonable notion of correctness for networks of modular, independently developed transformations consists of correctness of the single transformations, which need to be synchronizing, and correctness of the application function that determines an execution order of the transformations.
    The more important property than correctness of the application function is its degree of conservativeness.
    Additionally, although theoretically not relevant for correctness, the relations defining when models are considered consistent have to fulfill some notion of compatibility to be useful.
\end{insight}

\mnote{Outlook: achieving correct transformation networks}
In the following chapters, we will thus define such a notion of compatibility, discuss how correctness of the individual synchronizing transformations can be achieved and finally how a correct and appropriate application function to perform the orchestration can be defined.
In summary, these following contributions together will allow to develop what we defined as a \emph{correct} transformation network.



% \section{Summary}

% Central Insights:
% \begin{itemize}
%     \item In networks, we need compatible consistency relations -> first RQ
%     \item In networks, we need synchronizing rather than bidirectional transformations -> second RQ
%     \item In networks, we need orchestration functions -> third RQ
%     \item Correctness is not the problem, optimality is the problem
%     \item We can only check dynamically whether a consistent state was reached due to Halting Problem. We cannot guarantee to always find a consistent state
% \end{itemize}