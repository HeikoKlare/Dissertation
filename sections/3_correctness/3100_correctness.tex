\chapter{Correctness in Transformation Networks
    \pgsize{45 p.}
}
\label{chap:correctness}

%\todo{Discuss hippocraticness (although it may not be necessary). Bereits implizit in der consistency notion eingeführt. Auch auf Idempotenz der Funktion verweisen.}

\mnote{Chapter goals}
In this chapter, we will first discuss a rather informal notion of consistency and its preservation. It is supposed to describe the different dimensions in which consistency and its preservation can be considered to then discuss how \emph{correctness} can be reasonably defined.
After identifying the correctness notion that is relevant in the context of our work, we define a suitable formal notion of consistency.
We formally define correctness of different artifacts relevant for that notion of consistency.
Finally, we present a refined notion of consistency, which we do not require for the initial overview, but which we will later use for several detailed considerations.

\mnote{Contributions}
This chapter thus constitutes our contribution \autoref{contrib:correctness:notion}, which is composed of four subordinate contributions: a discussion of consistency notions; a discussion and determination of correctness notions for consistency specifications; a formalization of a relevant correctness notion; and finally a refinement of our consistency notion for later detailed considerations.
It answers the following research question:

\researchquestionrepeat{rq:correctness:notions}

\mnote{Publication of contributions}
Parts of the contributions in this chapter have already been published in \owncite{klare2020Vitruv-JSS}, \owncite{klare2020compatibility-report} and \owncite{klare2018docsym}.
In \owncite{klare2020Vitruv-JSS}, we have used a simplified version of the formalization that we introduce in this chapter and especially identified the challenge of orchestration, which is central for the formalization of transformation networks.
\owncite{klare2020compatibility-report} discussed compatibility and introduced the fine-grained consistency notion, which was required for detailed statements about compatibility.
In \owncite{klare2018docsym}, we especially motivated and informally derived the correctness notion that we formalize in the following and give an overview of the goal regarding correctness of transformation networks.

%%
%% CONSISTENCY NOTIONS
%%
\section{Notions of Consistency and its Preservation}
\label{chap:correctness:notions_consistency}

We begin with an informal discussion of different ways to consider consistency and its preservation. This involves \emph{intensional} and \emph{extensional}, as well \emph{monolithic} and \emph{modular} notions, and different execution strategies.

\subsection{Intensional and Extensional Consistency Notions}
\label{chap:correctness:notions_consistency:intensional_extensional}

\mnote{Intensional notion}
When we consider a tuple of models, we may intuitively assume it to be consistent if it fulfills some kind of constraints.
Defining these constraints to derive or check whether a given tuple of models is consistent constitutes an \emph{intensional specification} of consistency, because the set that contains all consistent model tuples is intensionally represented by these constraints and can be derived from it.
We can consider a set of constraints as a predicate, i.e., a Boolean-valued function $P$, which indicates whether a model tuple $\modeltuple{m} \in \metamodeltupleinstanceset{M}$ fulfills the constraints $P: \metamodeltupleinstanceset{M} \rightarrow \setted{\truemath, \falsemath}$. Then we can say that:
\begin{align*}
    \modeltuple{m} \consistenttomath P \equivalentperdefinition P(\modeltuple{m}) = \truemath
\end{align*}

\mnote{Extensional notion}
Alternatively, one can enumerate the (possibly infinite number of) consistent tuples of models.
Thus, a model tuple is considered consistent if that enumeration contains it.
This constitutes an \emph{extensional specification} of consistency.
Given such an enumeration $E = \setted{\modeltuple{m} \mid \modeltuple{m} \mathtextspacebefore{is consistent}}$, we can say that:
\begin{align*}
    \modeltuple{m} \consistenttomath E \equivalentperdefinition \modeltuple{m} \in E
\end{align*}

\mnote{Equivalence of notions}
Both kinds of specification have equal expressiveness. For each intensional specification, the extensional one can be derived by enumerating all models that fulfill the constraints:
\begin{align*}
    E = \setted{\modeltuple{m} \mid P(\modeltuple{m}) = true}
\end{align*}
An extensional specification can also be transferred to an intensional one by defining constraints that are fulfilled by exactly the enumerated instances:
\begin{align*}
    P(\modeltuple{m}) \mapsto 
    \begin{cases} 
        \truemath, & \modeltuple{m} \in E\\
        \falsemath, & \modeltuple{m} \not\in E\\
    \end{cases}
\end{align*}
For us, it will only be relevant that an intensional specification can be transformed into an extensional one.

\mnote{We use extensional specifications}
A developer who defines consistency usually wants to use an intensional specification, as tools like transformation languages allow the specification of constraints rather than enumerating consistent instances.
Since there is usually an infinite number of consistent models, he or she cannot explicitly enumerate them but only define constraints that allow to derive them.
From a theoretical perspective, however, we prefer to consider extensional specifications, because they allow to directly apply set theory in a concise way.
Due to the fact that each intensional specification can be transformed into an extensional one, we can make theoretical statements about extensional specifications that also hold for intensional ones.
In the following, we always consider extensional specifications unless otherwise stated.
So we define which models are considered consistent in terms of relations, which we also call \emph{consistency relations}.


\subsection{Monolithic and Modular Consistency Notions}
\label{chap:correctness:notions_consistency:monolithic_modular}

\mnote{Intuitive notion}
Consistency, be it specified intensionally or extensionally, can be considered in an either monolithic or modular way.
Having a single specification of consistency for an arbitrary number of models constitutes a \emph{monolithic} notion of consistency.
Like discussed for intensional and extensional consistency specifications, this can be expressed by a tuple of models fulfilling constraints or being contained in a relation.
A \emph{modular} notion of consistency considers several relations for subsets of the relevant metamodels, which together define when models are considered consistent.

\mnote{Modular specification example}
For an extensional notion of consistency between three metamodels $\metamodel{M}{1}, \metamodel{M}{2}$ and $\metamodel{M}{3}$, a modular specification could manifest in three relations $\consistencyrelation{CR}{1,2}, \consistencyrelation{CR}{1,3}$ and $\consistencyrelation{CR}{2,3}$ defining the model pairs that are considered consistent.
If two models are consistent to one of the relations, we can say that they are \emph{locally} consistent to that relation.
We are, however, interested in whether models are \emph{globally} consistent to all these relations, so we say:
\parameterizeformat{
    \begin{align*}
        & \model{m}{1}, \model{m}{2}, \model{m}{3} \mathtextspacearound{are consistent} \equivalentperdefinition 
        #2
        \tupled{\model{m}{1},\model{m}{2}} \in \consistencyrelation{CR}{1,2} \land \tupled{\model{m}{1},\model{m}{3}} \in \consistencyrelation{CR}{1,3} \land \tupled{\model{m}{2},\model{m}{3}} \in \consistencyrelation{CR}{2,3}
    \end{align*}
}{}{\\ & \formulaskip}%

\mnote{Modular specification for multiary relations}
Due to the assumptions of independent development and modular reuse, which we have defined in \autoref{chap:introduction:objective:assumptions}, we are interested in a modular notion of consistency.
In the example, we have considered a modular notion based on binary relation. Such a modular notion, however, can also be based on multiple multiary relations. 
But even with multiary relations, modularity is necessary for reasons of independent development and reuse.
For reasons of simplicity, we stick to modular notions of binary relations, although most of our considerations can be transferred to multiary ones.


\subsection{Consistency Preservation}
\label{chap:correctness:notions_consistency:preservation}

\mnote{Consistency preservation goal}
Consistency preservation is the process of ensuring that models stay consistent.
Based on a notion of consistency relations that describe when models are considered consistent, this process ensures that models stay in that relation. 
If models become changed such they that are not in the relation anymore, consistency preservation updates the models such that they are in that relation again.
In consequence, consistency preservation is always relative to relations defining consistency.

\mnote{Consistency preservation as function}
Consistency preservation can be considered as a function $\function{Cp}$ that takes (potentially inconsistent) models and returns a consistent tuple of models:
\parameterizeformat{
\begin{align*}
    & \function{Cp} : \metamodeltupleinstanceset{M} \rightarrow \metamodeltupleinstanceset{M} 
    #1 #2
    \forall \modeltuple{m} \in \metamodeltupleinstanceset{M} : \function{Cp}(\modeltuple{m}) \mathtextspacebefore{is consistent}
\end{align*}
}{\qquad\qquad}{\\ &}%
The definition of \emph{is consistent} depends on whether we rely on a monolithic or modular notion of consistency.
Thus it may require the models to be in one or multiple relations.
For example, given a monolithic relation $\consistencyrelation{CR}{}$, $\function{Cp}$ is supposed to fulfill that: 
\begin{align*}
    \forall \modeltuple{m} \in \metamodeltupleinstanceset{M} : \function{Cp}(\modeltuple{m}) \in \consistencyrelation{CR}{}
\end{align*}
Since these functions define how consistency is preserved, we also call them \emph{consistency preservation rules}.

\begin{figure}
    \centering
    \newcommand{\hdistance}{6em}

% #1: Position m1
% #2: Prefix
% #3: Suffix
% #4-#6: Model names
% #7-#8: Relation formatting
\newcommand{\network}[8][]{
    \node[model, #1] (#2_m1_#3) {#4};
    \node[model, above right=0.4*\hdistance and 0.6*\hdistance of #2_m1_#3.center, anchor=center] (#2_m2_#3) {#5};
    \node[model, right=1*\hdistance of #2_m1_#3.center, anchor=center] (#2_m3_#3) {#6};
    \draw[correspondence, #7] (#2_m1_#3) -- (#2_m2_#3);
    \draw[correspondence, #8] (#2_m1_#3) -- (#2_m3_#3);
}

\begin{tikzpicture}[
    every node/.append style={font=\footnotesize},
    model/.style={schematic model, minimum size=2em, inner sep=0.15em},
    violated/.style={densely dashed, color=darkred}
]

% INDEPENDENT
% Models
\network{ind}{1}{$\model{m}{1}$}{$\model{m}{2}$}{$\model{m}{3}$}{}{}
\network[below=1.2*\hdistance of ind_m1_1.center, anchor=center]{ind}{2}{$\model{m}{1}'$}{$\model{m}{2}$}{$\model{m}{3}$}{violated}{violated}
\network[below left=1.3*\hdistance and 0.8*\hdistance of ind_m1_2.center, anchor=center]{ind}{3}{$\model{m}{1}''$}{$\model{m}{2}'$}{$\model{m}{3}$}{}{violated}
\network[below right=1.3*\hdistance and 0.8*\hdistance of ind_m1_2.center, anchor=center]{ind}{4}{$\model{m}{1}'''$}{$\model{m}{2}$}{$\model{m}{3}'$}{violated}{}
\node[below right=0.2*\hdistance and 1.3*\hdistance of ind_m1_3.center, anchor=north, align=center] {unclear whether $\model{m}{1}'' = \model{m}{1}'''$ \\ or whether/how they can be merged};
% Execution
\draw[consistency execution] ([xshift=0.1*\hdistance, yshift=-0.1*\hdistance]ind_m1_1.south) -- node[left, align=center] {user\\ change} ([xshift=0.1*\hdistance, yshift=0.2*\hdistance]ind_m1_2.north);
\draw[consistency execution] ([xshift=0.45*\hdistance, yshift=-0.1*\hdistance]ind_m1_2.south) -- node[left=0.2em] {$\function{Cp}_{1,3}$} ([xshift=0.2*\hdistance, yshift=0*\hdistance]ind_m2_3.north);
\draw[consistency execution] ([xshift=0.55*\hdistance, yshift=-0.1*\hdistance]ind_m1_2.south) -- node[right=0.2em] {$\function{Cp}_{2,3}$} ([xshift=-0.4*\hdistance, yshift=0*\hdistance]ind_m2_4.north);

% CONSECUTIVE
% Models
\network[left=2.5*\hdistance+0.5*\difftoafiveimage of ind_m1_1.center, anchor=center]{cons}{1}{$\model{m}{1}$}{$\model{m}{2}$}{$\model{m}{3}$}{}{}
\network[below=1.2*\hdistance of cons_m1_1.center, anchor=center]{cons}{2}{$\model{m}{1}'$}{$\model{m}{2}$}{$\model{m}{3}$}{violated}{violated}
\network[below=1.2*\hdistance of cons_m1_2.center, anchor=center]{cons}{3}{$\model{m}{1}''$}{$\model{m}{2}'$}{$\model{m}{3}$}{}{violated}
\network[below=1.2*\hdistance of cons_m1_3.center, anchor=center]{cons}{4}{$\model{m}{1}'''$}{$\model{m}{2}'$}{$\model{m}{3}'$}{violated}{}
\node[below right=0.2*\hdistance and 0.5*\hdistance of cons_m1_4.center, anchor=north, align=center] {potentially $\tupled{\model{m}{1}''', \model{m}{2}'} \not\in \consistencyrelation{CR}{1,2}$};
% Execution
\draw[consistency execution] ([xshift=0.1*\hdistance, yshift=-0.1*\hdistance]cons_m1_1.south) -- node[left, align=center] {user\\ change} ([xshift=0.1*\hdistance, yshift=0.2*\hdistance]cons_m1_2.north);
\draw[consistency execution] ([xshift=0.1*\hdistance, yshift=-0.1*\hdistance]cons_m1_2.south) -- node[left] {$\function{Cp}_{1,2}$} ([xshift=0.1*\hdistance, yshift=0.2*\hdistance]cons_m1_3.north);
\draw[consistency execution] ([xshift=0.1*\hdistance, yshift=-0.1*\hdistance]cons_m1_3.south) --  node[left] {$\function{Cp}_{2,3}$} ([xshift=0.1*\hdistance, yshift=0.2*\hdistance]cons_m1_4.north);

% Labels
\node[above left=0.25*\hdistance and 0.1*\hdistance of ind_m2_1.center, anchor=south, font=\bfseries\small] {Independent Execution\sameheight};
\node[above left=0.25*\hdistance and 0.1*\hdistance of cons_m2_1.center, anchor=south, font=\bfseries\small] {Consecutive Execution\sameheight};

\end{tikzpicture}
    %\includegraphics[width=\textwidth]{figures/correctness/notion/concurrent_consecutive_execution.png}
    \caption[Execution alternatives of consistency preservation rules]{Scenarios for independently executing consistency preservation rules on input models and consecutively executing them on the results of other rules. Circles denote models, lines between models denote fulfilled (solid blue) or violated (dashed red) consistency relations, and arrows between the model states (unidirectional green) denote the conduction of user changes or consistency preservation execution.}
    \label{fig:correctness:concurrent_consecutive_execution}
\end{figure}

\mnote{Modular consistency preservation}
Like for the proposed notion of consistency, we can also consider consistency preservation in an either monolithic or modular way.
With a modular notion of consistency preservation, we may have multiple consistency preservation rules that preserve consistency, each of them for a consistency relation that defines consistency for a subset of the involved models.
Unlike for the relations defining consistency, which can be evaluated independently to identify whether models are consistent, the functions, i.e., consistency preservation rules, cannot be evaluated independently.
If each function is executed independently, each of them returns new models that may need to be merged. 
This is exemplified in the following scenario, which is also depicted in \autoref{fig:correctness:concurrent_consecutive_execution}.
Imagine two functions $\function{Cp}_{1,2}$ and $\function{Cp}_{1,3}$ that preserve consistency for relations $\consistencyrelation{CR}{1,2}$ and $\consistencyrelation{CR}{1,3}$, respectively.
Consider the input models $\tupled{\model{m}{1}, \model{m}{2}, \model{m}{3}}$ that are not consistent to $\consistencyrelation{CR}{1,2}$ and $\consistencyrelation{CR}{1,3}$, i.e., $\tupled{\model{m}{1},\model{m}{2}} \not\in \consistencyrelation{CR}{1,2}$ and $\tupled{\model{m}{1},\model{m}{3}} \not\in \consistencyrelation{CR}{1,3}$.
This can, for example, occur because $\model{m}{1}$ was changed by a user.
Now if we apply the functions independently, we have $\function{Cp}_{1,2}(\tupled{\model{m}{1}, \model{m}{2}}) = \tupled{\model{m}{1}', \model{m}{2}'} \in \consistencyrelation{CR}{1,2}$ and 
$\function{Cp}_{1,3}(\tupled{\model{m}{1}, \model{m}{3}}) = \tupled{\model{m}{1}'', \model{m}{3}'} \in \consistencyrelation{CR}{1,3}$.
It is now unclear how to unify $\model{m}{1}'$ and $\model{m}{1}''$ to $\model{m}{1}'''$, such that $\tupled{\model{m}{1}''', \model{m}{2}'} \in \consistencyrelation{CR}{1,2}$ and  $\tupled{\model{m}{1}''', \model{m}{3}'} \in \consistencyrelation{CR}{1,3}$.

\mnote{Consecutive function execution}
An intuitive approach to execute the functions is their composition, i.e., a consecutive execution that does not apply the functions for consistency preservation to the original models but to the models delivered by the previous executions of the functions, which is also exemplarily depicted in \autoref{fig:correctness:concurrent_consecutive_execution}.
If we consecutively apply the two given functions, we know that $\function{Cp}_{1,2}(\tupled{\model{m}{1}, \model{m}{2}}) = \tupled{\model{m}{1}', \model{m}{2}'} \in \consistencyrelation{CR}{1,2}$ and 
$\function{Cp}_{1,3}(\tupled{\model{m}{1}', \model{m}{3}}) = \tupled{\model{m}{1}'', \model{m}{3}'} \in \consistencyrelation{CR}{1,3}$.
It is, however, unclear whether $\tupled{\model{m}{1}'', \model{m}{2}'} \in \consistencyrelation{CR}{1,2}$, so it may be necessary to execute $\function{Cp}_{1,2}$ again.
In fact, we need some method to decide in which order and how often the consistency preservation rules are applied to result in a consistent tuple of models.
We call this an \emph{orchestration}.
The challenge to find an execution order of transformations without leading to execution cycles has also been identified by \textcite[Sec.~3.9]{kramer2017a}.

\mnote{Unification or orchestration necessary}
Even if consistency preservation rules were supposed to only modify one model instead of two, the same problems of unifying changes of their independent execution or orchestrating their consecutive execution occur as soon as there are two sequences of consistency preservation rules that change the same models.

\mnote{Benefits of consecutive execution}
In our work, we follow the approach of orchestrating and consecutively executing consistency preservation rules.
The benefits of this approach are twofold. First, there is no additional logic required for unifying the changes performed by independently executed consistency preservation rules. 
Second, the unification may deliver a model that is not consistent to any of the consistency relations anymore, whereas consecutive execution at least guarantees that the models are consistent to the last applied consistency preservation rule.
With this approach, the repeated execution of consistency preservation rules can be seen as a negotiation of a solution by reacting to the changes the other consistency preservation rules performed.

\begin{remark} 
\mnote{Monolithic notions as degraded modular ones}
Finally, every monolithic notion of consistency and its preservation can be considered a special case of a modular notion. Having only one consistency relation and one function that preserves it degrades the problem by making the necessity to perform an orchestration of functions obsolete.
\end{remark}

\mnote{Realization options for consistency preservation rules}
For now, the introduced consistency preservation rules can be any kind of functions that return consistent models. 
Their realization may, for example, be transformations that define how to react to certain changes for restoring consistency, or constraint solvers that find consistent models by solving consistency constraints. 
We do not yet need to consider how these functions are realized to derive consistent models, although we later focus on transformation-based approaches.


\subsection{Declarative and Imperative Specifications}
\label{chap:correctness:notions_consistency:declarative_imperative}

\begin{figure}
    \centering
    \begin{tikzpicture}

\node[schematic metamodel] (instances1) {$\metamodel{M}{1}$};
\node[schematic metamodel, right=22em of instances1] (instances2) {$\metamodel{M}{2}$};

\draw[consistency relation] (instances1) -- node[above] (cr) {$\consistencyrelation{CR}{1,2}$} (instances2);
\draw[transformation] (instances1) -- ++(0,-5em) -| node[pos=0.25,below] (cpr) {$\function{CP}_{1,2}$} (instances2);

\draw[-latex] ([xshift=-0.5em]cr.south) -- node[left=0.2em, align=center] {declarative specification \\ (ambiguous)} ([xshift=-0.5em]cpr.north);
\draw[-latex] ([xshift=0.5em]cpr.north) -- node[right=0.2em, align=center] {imperative specification \\ (unambiguous)}  ([xshift=0.5em]cr.south);

\end{tikzpicture}
    %\includegraphics[width=0.5\textwidth]{figures/correctness/notion/declarative_imperative}
    \caption[Declarative and imperative consistency specification]{Declarative and imperative specification of consistency relations and consistency preservation rules for two metamodels $\metamodel{M}{1}$ and $\metamodel{M}{2}$.}
    \label{fig:correctness:declarative_imperative}
\end{figure}

\mnote{Only define relation or function}
We have discussed that consistency preservation can be considered as functions, called consistency preservation rules, that preserve consistency according to some relations.
In practice, however, one will usually not specify both the consistency relation and the consistency preservation rule that preserves it.
Instead, one artifact is given and the other is implied or derived.
This leads to the two approaches of \emph{declarative} and \emph{imperative} consistency specifications, depending on whether the specification defines \emph{how} consistency is achieved.
The relation between the two approaches regarding a consistency relation and a consistency preservation rule is depicted in \autoref{fig:correctness:declarative_imperative}.

\mnote{Ambiguity in deriving rules}
As a first option, a developer may only define relations that specify consistency. Functions that preserve these relations can be derived from that.
This is called a \emph{declarative} specification, because it only declares when models are consistent but not \emph{how} consistency is achieved.
In general, there are multiple valid options for deriving a consistency preservation rule from a relation.
It can, for example, calculate the result with minimal differences to the input according to some defined metric.
Or, especially if there is an intensional specification of the relations, the approach may consider the type of input change and calculate an appropriate change according to the constraints in the intensional specification.
This approach is followed by many declarative transformation languages, such as \gls{QVTR}~\cite{qvt} or \glspl{TGG}~\cite{anjorin2014EfficientSynchronizationTGG-ECMFA}.

\mnote{Relations as fixed points}
As a second option, a developer can define consistency preservation rules without explicitly specifying the consistency relations to which they preserve consistency.
Instead, these functions imply the underlying consistency relations that they preserve, at least if we assume that a consistency preservation rule does not perform changes when the input models are already consistent.
Given a function $\function{Cp}$, the relation $\consistencyrelation{CR}{}$ it preserves is implied by its fixed points: $\consistencyrelation{CR}{} = \setted{\modeltuple{m} \mid \function{Cp}(\modeltuple{m}) = \modeltuple{m}}$.
If a function preserving consistency does not perform any changes, the models are, by definition, consistent.
Usually, we will assume that such a function returns consistent models with a single application.
Thus, if it does not perform changes when the input models are already consistent, the function is idempotent and then the consistency relation is given by its image, i.e., $\consistencyrelation{CR}{} = \setted{\modeltuple{m} \mid \exists \modeltuple{m}' : \function{Cp}(\modeltuple{m}') = \modeltuple{m}}$.
This is called an \emph{imperative} specification, because it declares \emph{how} consistency can be achieved.
Such an approach is followed by many imperative transformation languages, such as \gls{QVTO}~\cite{qvt}.


\subsection{Consistency Preservation Artifacts}

\mnote{Four artifacts}
We have discussed that consistency can be considered in a monolithic or modular way. We have, however, also mentioned that the monolithic case can be considered as a special case of the modular one.
For the general case, we thus know from the previous considerations that in a consistency preservation process at least specifications that define consistency, called \emph{consistency relations}, functions that preserve consistency, called \emph{consistency preservation rules}, and a function for orchestrating the functions, in the following called \emph{orchestration function}, are necessary. Finally, we also need a function that applies the consistency preservation rules in the order that is determined by the orchestration function, which we call the \emph{application function}.
To summarize, we consider the following four artifacts necessary to handle consistency preservation.
\begin{properdescription}
    \item[Consistency Relations:] Binary relations that specify which pairs of models shall be considered consistent.
    \item[Consistency Preservation Rules:] Functions that restore consistency for a pair of models that became inconsistent by modification.
    \item[Orchestration Function:] A function that determines the execution order of the consistency preservation rules to restore consistency.
    \item[Application Function:] A function that applies the consistency preservation rules in the order determined by the orchestration function. 
\end{properdescription}

\begin{figure}
    \centering
    \newcommand{\distance}{2em}

% #1: position left model
% #2: identifier prefix
% #3: relation style
% #4-#7: relations names 
\newcommand{\network}[7][]{
    \node[schematic metamodel, #1] (#2_m1) {};
    \node[schematic metamodel, above right=\distance and \distance of #2_m1.center, anchor=center] (#2_m2) {};
    \node[schematic metamodel, right=2*\distance of #2_m1.center, anchor=center] (#2_m3) {};
    \node[schematic metamodel, below right=\distance and \distance of #2_m1.center, anchor=center] (#2_m4) {};
    \draw[#3] (#2_m1) -- node[above left] {#4} (#2_m2);
    \draw[#3] (#2_m2) -- node[above right] {#5} (#2_m3);
    \draw[#3] (#2_m2) -- node[fill=white, opacity=0.7, text opacity=1] {#6} (#2_m4);
    \draw[#3] (#2_m1) -- node[below left] {#7} (#2_m4);    
}

\begin{tikzpicture}[
    every node/.style={font=\footnotesize},
    annotation/.style={font=\itshape\footnotesize}
]

% INPUTS
\network{input}{consistency relation}{$\consistencyrelation{CR}{1}$}{$\consistencyrelation{CR}{2}$}{$\consistencyrelation{CR}{3}$}{$\consistencyrelation{CR}{4}$}
\node[below left=0.2em and 0em of input_m1.west, anchor=east] {$\model{m}{1}$};
\node[right=1*\distance of input_m3.center, anchor=center] {+};
\node[right=2*\distance of input_m3.center, anchor=center, font=\small] (changes) {$\Delta_{\model{m}{1}}$};
\node[below=0.3em of input_m4.south, anchor=north, align=center, annotation] (text_input_models) {models consistent\\ to $\consistencyrelation{CR}{1}, \consistencyrelation{CR}{2}, \dots$};
\node[right=3*\distance of text_input_models.north, anchor=north, align=center, annotation] (text_input_changes) {changes to \\ models(s)};
\node[consistency execution element, above left=0.3*\distance and 2.5*\distance of input_m2.north, anchor=north, align=center] {Consistency\\ Relations};

% OUTPUT
\network[right=9.5*\distance of input_m1.center, anchor=center]{output}{consistency relation}{$\consistencyrelation{CR}{1}$}{$\consistencyrelation{CR}{2}$}{$\consistencyrelation{CR}{3}$}{$\consistencyrelation{CR}{4}$}
\node[below=0.3em of output_m4.south, anchor=north, align=center, annotation] (text_output_models) {changed models\\ consistent to \\ $\consistencyrelation{CR}{1}, \consistencyrelation{CR}{2}, \dots$};

\draw[consistency execution, double distance=0.15em] ([xshift=1em]changes.east) -- node[below] {Application Function} ([xshift=-1em]output_m1.west);

% ORCHESTRATION FUNCTION
\draw[consistency execution element, decorate,decoration={brace,amplitude=10pt}] 
    (text_input_changes.south east) 
    -- 
    (text_input_changes.south -| text_input_models.south west);
\coordinate (brace_midpoint) at ($(text_input_models.south west)!0.5!(text_input_changes.south east)$);
\draw[consistency execution, double distance=0.15em] 
    ([yshift=-10pt]brace_midpoint) 
    |- 
    node[pos=0.7,below] (orchestration_function) {Orchestration Function} 
    ++(3*\distance,-1*\distance);

% CONSISTENCY PRESERVATION
\network[below left=5.7*\distance and 1.7*\distance of input_m1.center, anchor=center]{cprs}{transformation}{$\consistencypreservationrule{1}$}{$\consistencypreservationrule{2}$}{$\consistencypreservationrule{3}$}{$\consistencypreservationrule{4}$}
\node[consistency execution element, above=0.3em of cprs_m2.north, anchor=south, align=center] {Consistency\\ Preservation Rules};
\draw[consistency preservation element, -latex] 
    ([xshift=0.5em]cprs_m3.east) 
    -| 
    node[pos=0.25, below, annotation] {parametrization} 
    (orchestration_function.south);

% ORCHESTRATION SEQUENCE
\coordinate (orchestration_start) at ([xshift=3*\distance, yshift=-2.7\distance]brace_midpoint);
\draw[transformation] 
    ([xshift=\distance-0.3*\distance,yshift=-0.3*\distance]orchestration_start) 
    -- 
    node[left=0.5*\distance, anchor=center] {$\langle$}
    node[below=0.35*\distance, anchor=north] {1}
    node[right=0.4*\distance, anchor=north] {,}
    ++(0.6*\distance, 0.6*\distance);
\draw[transformation] 
    ([xshift=1.8*\distance-0.3*\distance,yshift=0.3*\distance]orchestration_start) 
    -- 
    node[below=0.35*\distance, anchor=north] {2} 
    node[right=0.4*\distance, anchor=north] {,}
    ++(0.6*\distance, -0.6*\distance);
\draw[transformation] 
    ([xshift=2.6*\distance,yshift=0.3*\distance]orchestration_start) 
    -- 
    node[below=0.35*\distance, anchor=north] {3}
    node[right=0.4*\distance, anchor=north] (orchestration_middle) {,}
    ++(0, -0.6*\distance);
\draw[transformation] 
    ([xshift=3.4*\distance-0.3*\distance,yshift=-0.3*\distance]orchestration_start) 
    -- 
    node[below=0.35*\distance, anchor=north] {1}
    node[right=0.4*\distance, anchor=north] {,}
    ++(0.6*\distance, 0.6*\distance);
\draw[transformation] 
    ([xshift=4.2*\distance-0.3*\distance,yshift=0.3*\distance]orchestration_start) 
    -- 
    node[below=0.35*\distance, anchor=north] {4}
    node[right=0.4*\distance, anchor=north] {,}
    ++(0.6*\distance, -0.6*\distance);
\draw[transformation] 
    ([xshift=5.0*\distance,yshift=0.3*\distance]orchestration_start) 
    -- 
    node[below=0.35*\distance, anchor=north] {3}
    node[right=0.3*\distance, anchor=center] {$\rangle$}
    ++(0, -0.6*\distance);
\node[consistency preservation element, below=0.7em of orchestration_middle, anchor=north, annotation] {orchestrated sequence};
\draw[consistency preservation element, -latex] 
    ([xshift=\distance, yshift=0.5*\distance]orchestration_start) 
    -- 
    node[right, annotation] {parametrization} 
    ++(0,2.7*\distance);

% LABELS
\node[above right=0.5em and 1.2*\distance of input_m2.north, anchor=south, font=\bfseries\small] {Process Inputs};
\node[above=0.5em of output_m2.north, anchor=south, font=\bfseries\small] {Process Output};

\end{tikzpicture}
    %\includegraphics[width=\textwidth]{figures/correctness/notion/execution_process.png}
    \caption[Consistency specification execution process and artifacts]{Execution process and artifacts for a modular consistency specification. Central artifacts are annotated in (green) normal font.}
    \label{fig:correctness:execution_process}
\end{figure}

\mnote{Distinction between orchestration and application}
We explicitly distinguish the orchestration and the application to be able to make more fine-grained statements about the responsibilities for the orchestration and its actual execution.
This is particularly useful to determine the behavior in cases in which no orchestration of transformations that results in consistent models can be found.
The process is depicted in \autoref{fig:correctness:execution_process}. Given models that are consistent according to some consistency relations and changes to them that lead to inconsistencies, the orchestration function delivers an order of consistency preservation rules, which is used to parametrize the application function that executes these rules in the given order.
The result is, in the best case, a model tuple that is consistent to the relations again.

%%
%% CORRECTNESS NOTIONS
%%
\section{Notions of Correctness for Consistency Specifications}
\label{chap:correctness:notions_correctness}

\mnote{Different notions of correctness}
Before we formally define the above introduced artifacts, such as consistency relations, consistency preservation rules, an orchestration function and an application function, we first discuss different notions of \emph{correctness} for them.
Since there are different dimensions of correctness, we need to clarify which of them is relevant in the context of our research questions and will be defined in the formalization.


\subsection{Relative Correctness Notions}

\mnote{Intuitive notion of correctness}
The overall objective regarding correctness of consistency preservation is to find models that are actually consistent.
Intuitively speaking, artifacts are correct if they fulfill their intended purpose. 
In our case, this means that consistency relations should consider models consistent whenever they are actually supposed to be considered consistent.
Consistency preservation rules should return models that are consistent according to a consistency relation to be considered correct.
This also conforms to existing notions of correctness for transformations~\cite{stevens2010sosym}, which realize consistency preservation rules.
And finally, the orchestration and application functions should execute the consistency preservation rules such that they yield models that are consistent according to all relations afterwards.

\mnote{Correctness is relative}
Correctness of an artifact is usually considered with respect to some other specification, be it formally defined or only an informal notion.
For example, consistency relations may be considered correct with respect to some informal notion of correctness that is collected by domain experts and requirements engineers.
A consistency preservation rule should always be consistent with respect to a consistency relation. As discussed before, this relation may either be defined explicitly and the preservation rule has to be correct with respect to it, or it may be induced by the fixed points of the preservation rule.
In the latter case, the consistency preservation rule will always be correct by construction.


\subsection{Correctness regarding Global Knowledge}

\mnote{Correctness of modular w.r.t. monolithic specification}
We previously distinguished between monolithic and modular consistency notions.
In the above considerations, we have related the artifacts of a modular specification to each other.
Another notion of correctness can be defined by relating a modular artifact to a corresponding monolithic artifact.
For example, a set of modular consistency relations may be considered correct with respect to a monolithic relation when it considers the same model tuples consistent.
For three metamodels $\metamodel{M}{1}, \metamodel{M}{2}, \metamodel{M}{3}$ with three modular consistency relations $\consistencyrelation{CR}{1,2}, \consistencyrelation{CR}{1,3}, \consistencyrelation{CR}{2,3}$ between them, as well as a ternary consistency relation $\consistencyrelation{CR}{1,2,3}$, we could say that $\consistencyrelation{CR}{1,2}, \consistencyrelation{CR}{1,3}, \consistencyrelation{CR}{2,3}$ are correct (with respect to $\consistencyrelation{CR}{1,2,3}$) if, and only if,
\begin{align*}
    & \forall \model{m}{1} \in \metamodel{M}{1}, \model{m}{2} \in \metamodel{M}{2}, \model{m}{3} \in \metamodel{M}{3}: \tupled{\model{m}{1}, \model{m}{2}, \model{m}{3}} \in \consistencyrelation{CR}{1,2,3} \\
    & \formulaskip
    \equivalent \tupled{\model{m}{1}, \model{m}{2}} \in \consistencyrelation{CR}{1,2} \land \tupled{\model{m}{1}, \model{m}{3}} \in \consistencyrelation{CR}{1,3} \land \tupled{\model{m}{2}, \model{m}{3}} \in \consistencyrelation{CR}{2,3}
\end{align*}
%
We may, analogously, define correctness for consistency preservation rules, an orchestration function, and an application function with respect to a monolithic preservation rule by defining that both deliver the same results for the same inputs or at least return a consistent result in the same cases.

\begin{figure}
    \centering
    \newcommand{\distance}{4em}

% #1: position left model
% #2: identifier prefix
\newcommand{\network}[2][]{
    \node[schematic metamodel, #1] (#2_m1) {};
    \node[schematic metamodel, right=\distance of #2_m1.center, anchor=center] (#2_m2) {};
    \node[schematic metamodel, below=\distance of #2_m1.center, anchor=center] (#2_m3) {};
    \node[schematic metamodel, below right=\distance and \distance of #2_m1.center, anchor=center] (#2_m4) {};
}

\begin{tikzpicture}[
    every node/.style={font=\footnotesize},
    correctness relation/.style={-latex, font=\itshape\footnotesize},
    irrelevant/.style={gray}
]

% MONOLITHIC RELATION
\network{mono_relations}
\draw[consistency relation] (mono_relations_m1) -- (mono_relations_m4);
\draw[consistency relation] (mono_relations_m2) -- (mono_relations_m3);
\filldraw[consistencycolor1]
    ([xshift=0.5*\distance, yshift=-0.5*\distance]mono_relations_m1.center)
    circle
    (0.04*\distance)
    node[above=0.5em] {$\consistencyrelation{CR}{}$};

% MONOLITHIC TRANSFORMATION
\network[below=2.7*\distance of mono_relations_m1.center, anchor=center]{mono_transformations}
\draw[transformation] (mono_transformations_m1) -- (mono_transformations_m4);
\draw[transformation] (mono_transformations_m2) -- (mono_transformations_m3);
\filldraw[consistencypreservationcolor]
    ([xshift=0.5*\distance, yshift=-0.5*\distance]mono_transformations_m1.center)
    circle
    (0.04*\distance)
    node[above=0.5em] {$\consistencypreservationrule{}$};

% MODULAR RELATIONS
\network[right=3.2*\distance+0.7*\difftoafiveimage of mono_relations_m1.center, anchor=center]{modu_relations}
\draw[consistency relation] 
    (modu_relations_m1) 
    -- 
    node[above] {$\consistencyrelation{CR}{1}$}
    (modu_relations_m2);
\draw[consistency relation] 
    (modu_relations_m1) 
    -- 
    node[left] (cr2) {$\consistencyrelation{CR}{2}$}
    (modu_relations_m3);
\draw[consistency relation] (modu_relations_m1) -- (modu_relations_m4);
\draw[consistency relation] (modu_relations_m2) -- (modu_relations_m3);
\draw[consistency relation] 
    (modu_relations_m2)
    --
    node[right] {$\consistencyrelation{CR}{3}$}
    (modu_relations_m4);
\draw[consistency relation] 
    (modu_relations_m3)
    --
    node[below] {$\dots$}
    (modu_relations_m4);

% MODULAR TRANSFORMATIONS
\network[below=2.7*\distance of modu_relations_m1.center, anchor=center]{modu_transformations}
\draw[transformation]
    (modu_transformations_m1) 
    -- 
    node[above] {$\consistencypreservationrule{1}$}
    (modu_transformations_m2);
\draw[transformation]
    (modu_transformations_m1) 
    -- 
    node[left] (cpr2) {$\consistencypreservationrule{2}$}
    (modu_transformations_m3);
\draw[transformation] (modu_transformations_m1) -- (modu_transformations_m4);
\draw[transformation] (modu_transformations_m2) -- (modu_transformations_m3);
\draw[transformation]
    (modu_transformations_m2)
    --
    node[right] {$\consistencypreservationrule{3}$}
    (modu_transformations_m4);
\draw[transformation] 
    (modu_transformations_m3)
    --
    node[below] {$\dots$}
    (modu_transformations_m4);
\node[consistencypreservationcolor, right=1.2*\distance of modu_transformations_m2.north, anchor=north, align=center] (orc_function) {+ orchestration / \\ application \\ function};

% CORRECTNESS RELATIONS
\draw[correctness relation]
    ([xshift=0.5*\distance]mono_transformations_m1.center)
    --
    node[pos=0.4, left] {correct w.r.t.}
    ([xshift=0.5*\distance,yshift=-0.7*\distance]mono_relations_m1.center);
\draw[correctness relation, -, irrelevant, decorate, decoration={brace,amplitude=0.5em}] 
    ([xshift=-1.8em]modu_relations_m3.south west) 
    -- 
    ([xshift=-1.8em]modu_relations_m1.north west);
\draw[correctness relation, irrelevant]
    ([xshift=-2.3em, yshift=-0.5*\distance]modu_relations_m1.west)
    --
    node[pos=0.4, above] {correct w.r.t.}
    ([xshift=-0.3*\distance,yshift=-0.5*\distance]mono_relations_m2.center);
\draw[correctness relation, -, irrelevant, decorate, decoration={brace,amplitude=0.5em}] 
    ([xshift=-1.8em]modu_transformations_m3.south west) 
    -- 
    ([xshift=-1.8em]modu_transformations_m1.north west);
\draw[correctness relation, irrelevant]
    ([xshift=-2.3em, yshift=-0.5*\distance]modu_transformations_m1.west)
    --
    node[pos=0.4, above] {correct w.r.t.}
    ([xshift=-0.3*\distance,yshift=-0.5*\distance]mono_transformations_m2.center);
\draw[correctness relation] 
    (cpr2)
    --
    node[left, align=center] {correct w.r.t. \\ (locally)}
    (cr2);
\draw[correctness relation, -,, decorate, decoration={brace,amplitude=0.5em,aspect=0.192}] 
    ([yshift=1em]modu_transformations_m1.north west)
    -- 
    ([yshift=1em]modu_transformations_m1.north west-|orc_function.north east);
\draw[correctness relation, -,, decorate, decoration={brace,amplitude=0.5em}] 
    ([yshift=-0.6em]modu_relations_m4.south east)
    --
    ([yshift=-0.6em]modu_relations_m3.south west);
\draw[correctness relation] 
    ([xshift=0.5*\distance, yshift=1.5em]modu_transformations_m1.north)
    --
    node[right, align=center] {correct w.r.t. \\ (globally)}
    ([xshift=0.5*\distance, yshift=-1.1em]modu_relations_m3.south);

% LABELS
\node[above right=1em and 0.5*\distance of mono_relations_m1.north, anchor=south, font=\bfseries\small] {Monolithic};
\node[above right=1em and 0.5*\distance of modu_relations_m1.north, anchor=south, font=\bfseries\small] {Modular};
\node[consistencycolor1, below left=0.5*\distance and 0.5em of mono_relations_m1.center, anchor=east, align=center] {Consistency \\ Relations};
\node[consistencypreservationcolor, below left=0.5*\distance and 0.5em of mono_transformations_m1.center, anchor=east, align=center] {Consistency \\ Preservation \\ Rules};

\end{tikzpicture}

    %\includegraphics[width=\textwidth]{figures/correctness/notion/correctness_notions.png}
    \caption[Notions of correctness for consistency and its preservation]{Different notions of correctness for consistency and its preservation. Circles denote metamodels with arrows between them representing consistency relations and consistency preservation rules. Further unidirectional arrows denote different notions of correctness of one or more artifacts with respect to others.}
    \label{fig:correctness:correctness_notions}
\end{figure}


\subsection{Dimensions of Correctness}
\label{chap:correctness:notions_correctness:dimensions}

\mnote{Two dimension of correctness notions}
The discussed correctness notions induce two dimensions: First, correctness can be considered between artifacts within a monolithic or modular specification. Second, correctness can be considered between artifacts of a modular specification and corresponding artifacts of a monolithic specification. These dimensions are depicted in \autoref{fig:correctness:correctness_notions}.
The former dimension is depicted vertically. Consistency preservation rules need to be correct with respect to their consistency relations.
In the modular case, in addition to each preservation rule being \emph{locally} correct with respect to its relation, the combination of preservation rules by an orchestration and application function must also be \emph{globally} correct with respect to the combination of all relations.
The latter dimension is depicted horizontally. Each modular artifact must be consistent with respect to a corresponding monolithic artifact.

\mnote{Drawbacks of global specification notions}
Although correctness of modular with respect to monolithic artifacts can be interesting from a theoretical perspective, its practical relevance is limited.
That notion of correctness assumes that there is some kind of global truth that has to be reflected by a modular specification.
This, however, has the following two essential drawbacks.
\begin{properdescription}
    \item[Validation Artifacts:] The artifacts to validate correctness against, i.e., the global, monolithic consistency relation as well as an appropriate monolithic consistency preservation rule, do usually not exist. If they existed, they could directly be used to preserve consistency. Thus, it is impossible to validate a set of consistency relations and consistency preservation rules against such a global specification.
    \item[Modular Knowledge:] This notion of correctness requires that the developers have some global knowledge that represents a monolithic consistency relation and its consistency preservation rule. We assume the knowledge about relations between models to usually be distributed across several persons. Thus, there will be no such global knowledge, and not even an implicit notion of the necessary artifacts to validate the modular specifications against exists.
\end{properdescription}
%
Since this conflicts with our assumption of distributed knowledge about relations and independently developed, modular specifications, we do not further consider this notion of consistency.
We focus on correctness between the artifacts of a modular consistency specification.
We have discussed this correctness notion as correctness between a \emph{modularization level} and a \emph{global level} of consistency specifications in previous work~\owncite{klare2019icmt}.


\subsection{Correctness of Consistency Relations}
\label{chap:correctness:notions_correctness:relations}

\mnote{Correctness of relations}
The consistency notion that we consider in the following especially requires that consistency preservation rules and the functions to orchestrate and apply them must be correct with respect to consistency relations.
This notion does, however, not define when consistency relations are considered \emph{correct}.
One option is to only consider correctness with respect to monolithic artifacts for the case of consistency relations, as we have proposed in previous work~\owncite{klare2019icmt}.
This, however, suffers from the discussed drawback of requiring a global notion of consistency.
Another notion of correctness would be conformance of the specified relations with what developers expect to be consistent, i.e., a validation of requirements.
For example, a consistency relation between \gls{UML} and Java may only be considered correct if it fulfills some \enquote{natural} notion of consistency, as developers know how elements are related because they represent similar things, such as classes, or because a standard like the \gls{UML}~\cite{uml} prescribes it.
In this work we do not consider such a correctness notion with respect to external, maybe not formally specified artifacts, as it is part of separate research on requirements engineering and validation.

\mnote{No correctness by construction}
In consequence, we might say that consistency relations are simply \emph{correct by construction}.
Thus, relations would normatively define what is to be considered consistent.
However, a consequence of not assuming a global knowledge of consistency is that different domain experts may have different and even conflicting notions of when models are to be considered consistent.
Consider for three metamodel $\metamodel{M}{1}, \metamodel{M}{2}, \metamodel{M}{3}$ the three modular consistency relations $\consistencyrelation{CR}{1,2} = \setted{\tupled{\model{m}{1},\model{m}{2}}}$, $\consistencyrelation{CR}{1,3} = \setted{\tupled{\model{m}{1},\model{m}{3}}}$, and $\consistencyrelation{CR}{2,3} = \setted{\tupled{\model{m}{2},\model{m}{3}'}}$. 
Then there is no triple of models that is considered consistent to all relations. 
Although we still do not want to assume a global knowledge about consistency to which the modular one must conform, we might say that these relations are \emph{incompatible}, as we do not want to combine relations that induce an empty set of consistent model tuples.
Identifying an appropriate notion of \emph{compatibility} and how to check it constitutes \autoref{rq:correctness:compatibility} and will be discussed as our contribution \autoref{contrib:correctness:compatibility} in \autoref{chap:compatibility}.

\mnote{Induction of monolithic relation}
In fact, every set of modular consistency relations induces a monolithic one.
The monolithic relation $\consistencyrelation{CR}{}$ for metamodels $\metamodelsequence{M}{n}$ and pairwise relations $\consistencyrelation{CR}{i,j}$ is defined by:
\begin{align*}
    \consistencyrelation{CR}{} = \setted{\tupled{\model{m}{1}, \dots, \model{m}{n}} \mid \bigwedge\limits_{1 \le i < j \le n} \tupled{\model{m}{i}, \model{m}{j}} \in \consistencyrelation{CR}{i,j}}
\end{align*}
At least if this induced relation is empty, we probably want to consider the modular relations incompatible, because if no models are considered consistent, we cannot describe any system consistently.


% Keeping multiple models consistent by means of transformations imposes either a single multidirectional transformation or a combination of several bi- or multidirectional transformations.
% Each of these transformations is able to take a consistent tuple of models and a change to them and to return a new consistent tuple of models.

% \mnote{Correctness implicitly covered by definitions}
% \textcite{stevens2010sosym} proposes an explicit notion of \emph{correctness} for transformations. This is based on the fact that her definition of a transformation does only specify that for two given model, which may be inconsistent because one was modified, an update of the other model is returned.
% The requirement that the originally modified model and the one returned by the transformation have to conform to some consistency relations is specified externally as a notion of \emph{correctness}.
% We directly relate a consistency preservation rule that restores consistency to an according consistency relation, thus a consistency preservation rule that follows our definition is correct by construction in terms of the correctness definition by \citeauthor{stevens2010sosym}.
% The same applies to the consistency preservation application function, which we consider \emph{correct} if it fulfills its definition, as that definition already covers all requirements to that function.

% \mnote{Different notions of correctness}
% In general, correctness can be considered in two ways: First, an artifact may be correct if it simply follows its definition.
% While for consistency relations, changes and the generic \modellevelconsistencypreservationrule generalization function correctness can be canonically achieved, this is not that simple for a consistency preservation rule and the consistency preservation application function, as they have to fulfill some constraints with respect to consistency relations they rely on.
% Second, an artifact may be correct if it fulfills some, maybe only implicitly known specification. For example, a consistency relation between UML and Java may only be considered correct if it fulfills some \enquote{natural} notion of consistency, as people know how elements have to be related because they represent similar things, such as classes, or because a standard like the UML~\cite{uml} prescribes that.

% \mnote{Correctness regarding global specifications}
% In this work we do not consider the latter correctness notion with respect to external, maybe not formally specified artifacts, which is part of separate research on validation.
% However, when considering consistency of multiple models it may be standing to reason that a modular specification of consistency and its preservation has to be correct with regards to some global, monolithic specification. More precisely, there may be a multiary relation putting several metamodels into relation, which the developers at least implicitly know, and a set of binary relations somehow has to respect that multiary relation, i.e., be \emph{correct} with respect to that relation.
% The same can be imagines for consistency preservation. One may define a multidirectional transformation for a multiary relation, taking a tuple of changes to consistent models and retuning a new tuple of changes, which applied to the models delivers a consistent set of models again. In fact, this would be a realization of the behavior of the consistency preservation application function without relying on modular \modellevelconsistencypreservationrules.

% \begin{description}
%     \item[Validation Artifacts:] The artifacts to check correctness against, i.e., the global, multiary consistency relation as well as an appropriate multidirectional transformation, do usually not exist. If they existed, they could directly be used to preserve consistency. Thus is impossible to validate a set of consistency relations and preservation rules against such a global specification.
%     \item[Modular Knowledge:] This notion of correctness requires that the developers have some global knowledge that represents a monolithic, multiary consistency relation and their preservation rules. Usually, this will not be the case, so there is even no implicit notion of the necessary artifacts to validate the modular specifications against, not to be mention an explicit representation.
% \end{description}

%\todo{Add an image for that relation}

% Overall Goals:
% \begin{itemize}
%     \item Define correctness of a transformation network: termination in consistent state
% \end{itemize}


%%
%% FORMALIZATION
%%
\section{A Formal Notion of Transformation Networks}
\label{chap:correctness:formalization}

\mnote{Formalization of transformation networks and correctness}
We have so far discussed a general notion of consistency and its preservation with a focus on a modular way of specifying it.
This notion was introduced in a rather informal way to first be able to discuss correctness notions and determine which notion is relevant for the considerations in this thesis.
In the following, we define a formal notion of consistency and its preservation, based on the informal explanation given before.
It extends the one we have presented in previous work~\owncite{klare2021Vitruv-JSS}.
We also give a precise definition of notions for correctness between the artifacts of a modular specification.
Furthermore, we now focus on transformation-based approaches, i.e., we consider specifications that transform changes within one or more models into changes in one or more other models, as a specialization of the general notion for consistency preservation used before.

%Allgemeine Definition Transformationsnetzwerke:
%\begin{itemize}
    %\item Definition Transformation aus Relation und Wiederherstellungsroutinen; Routinen nehmen n Modelle und n Deltasequenzen (eine pro Modell) und liefern n Deltasequenzen zurück.
    %\item Im Allgemeinen könnte eine Transformation beliebige dieser Deltasequenzen modifizieren. Wir verlangen jedoch, dass eine Transformation nur Deltas anhängt, also die Sequenzen länger werden
    %\item Genauer beschränken wir auch, welche Sequenzen eine Transformation sehen und ändern darf, genau gesagt darf sie die Sequenzen von zwei Modellen sehen und eine davon verlängern.
    %\item Hier kommt bereits der Unterschied zu bisherigen Transformationen, denn die sehen nur Deltas an einem Modell und erzeugen Deltas an dem anderen. Das ist bei uns schon gänzlich anders. Bidirektionale Transformationen unterstützen das im Übrigen auch nicht, sondern sind nur Spezifikationen, aus denen sich Wiederherstellungsroutinen für beide Richtungen ableiten lassen (siehe Stevens 2010)
    %\item Relationen in erster Instanz auf Modellebene (also bzgl. ganzer Modelle, nicht einzelner Modellelemente) definieren
    %\item Direkt als multidirektionale Transformation definieren, also beliebig viele geändert Ein- und Ausgabemodelle (oder jeweils nur eins?)
    %\item Korrektheit einer Transformation (nach Stevens) definieren!
    %item Versuchen den Konkatenationsoperator zu definieren ohne dass er alle Metamodelle referenzieren muss (also Transformation wählt aus einer großen Eingabemenge relevanten Modelle aus, ändert relevante und dann fügt der Operator sie in die große Menge ein)
    %\item Definition Transformationsnetzwerk als Tupel aus Metamodellen, Transformationen und einer Ausführungsfunktion. 
    %\item Die Ausführungsfunktion führt für eine gegebene Änderung eine Auswahl der Transformationen nacheinander aus.
    %\item Korrektheit eines Netzwerkes definieren: Die Ausführungsfunktion erzeugt eine Transformationssequenz, die angewendet auf eine Änderung für alle Änderungen ein korrektes Ergebnisse produziert, d.h. die Modelle sind konsistent bzgl. allen Konsistenzrelationen.
%\end{itemize}


\subsection{Modular Consistency Specification}

\mnote{Extensional specifications are relations}
As discussed informally before, an extensional specification of consistency defines a relation between models by enumerating all tuples of models that are considered consistent.

\begin{definition}[\ModelLevelConsistencyRelation]
    \label{def:modellevelconsistencyrelation}
    Given a tuple of metamodels $\metamodeltuple{M} = \tupled{\metamodelsequence{M}{n}}$, a \emph{\modellevelconsistencyrelation} $\consistencyrelation{CR}{}$ is a relation for instances of the metamodels $\consistencyrelation{CR}{} \subseteq \metamodeltupleinstanceset{M} = \metamodelinstanceset{M}{1} \times \dots \times \metamodelinstanceset{M}{n}$.

    For a tuple of models $\modeltuple{m} \in \metamodeltupleinstanceset{M}$, we say that:
    \begin{align*}
        \modeltuple{m} \consistenttomath \consistencyrelation{CR}{} \equivalentperdefinition \modeltuple{m} \in \consistencyrelation{CR}{}
    \end{align*}
    Otherwise, we call $\modeltuple{m} \inconsistenttomath \consistencyrelation{CR}{}$.
\end{definition}

\mnote{Start with coarse-grained model-level relations}
We consider a tuple of models consistent if the consistency relation contains it.
This conforms to existing consistency definitions for bidirectional transformations~\cite{stevens2010sosym}.
We denote this kind of consistency relation as \emph{model-level}, because we later need to refine the notion of consistency relations to the level of \metaclasses and distinguish them.

\mnote{Modular notions of consistency}
If a single relation describes consistency between all relevant models, consistency is defined by means of model tuples being contained in that relation. We call such a relation \emph{monolithic}.
If a relation only defines consistency between some of the relevant models and the global consistency relation is defined by a combination of several such relations, we need an explicit definition of such a \emph{modular} notion of consistency.
For the sake of simplicity, we focus on \emph{binary} relations as a modular representation of consistency.

\begin{definition}[Model-Level Consistency] 
    \label{def:modellevelconsistency}
    Let $\metamodeltuple{M} =  \tupled{\metamodelsequence{M}{n}}$ be metamodels and let $\consistencyrelation{CR}{i,k} \subseteq \metamodelinstanceset{M}{i} \times \metamodelinstanceset{M}{k}$ be a binary \modellevelconsistencyrelation for $\metamodel{M}{i}, \metamodel{M}{k} \in \metamodeltuple{M}$. 
    We say that a model tuple $\modeltuple{m} = \tupled{\model{m}{1}, \dots, \model{m}{n}} \in \metamodeltupleinstanceset{M}{}$ is \emph{consistent to} $\consistencyrelation{CR}{i,k}$ if, and only if, the instances of $\metamodel{M}{i}$ and $\metamodel{M}{k}$ are in that relation:
    \begin{align*} 
        &
        \modeltuple{m} \consistenttomath \consistencyrelation{CR}{i,k} \equivalentperdefinition 
        \tupled{\model{m}{i},\model{m}{k}} \in \consistencyrelation{CR}{i,k}
    \end{align*}
    For a set of binary \modellevelconsistencyrelations $\consistencyrelationset{CR}$ for metamodels $\metamodeltuple{M}$, we say that a tuple of models $\modeltuple{m} \in \metamodeltupleinstanceset{M}{}$ is \emph{consistent to} $\consistencyrelationset{CR}$ if, and only if, it is consistent to each consistency relation in that set:
    \begin{align*} 
        &
        \modeltuple{m} \consistenttomath \consistencyrelationset{CR} \equivalentperdefinition
        \forall \consistencyrelation{CR}{} \in \consistencyrelationset{CR} : \modeltuple{m} \consistenttomath \consistencyrelation{CR}{}
    \end{align*}
\end{definition}

\mnote{Exemplary binary relations}
The definition states that models are consistent to a set of \modellevelconsistencyrelations if they are consistent to each relation in that set.
Consider, for example, for $\model{m}{i} \in \metamodelinstanceset{M}{i}$ the relations $\consistencyrelation{CR}{1} = \setted{\tupled{\model{m}{1}, \model{m}{2}}}$, $\consistencyrelation{CR}{2} = \setted{\tupled{\model{m}{2},\model{m}{3}}}$, and $\consistencyrelation{CR}{3} = \setted{\tupled{\model{m}{1}, \model{m}{3}}}$. Then the model tuple $\tupled{\model{m}{1}, \model{m}{2}, \model{m}{3}}$ is consistent to these relations.
These consistency relations are equivalent to a monolithic relation $\consistencyrelation{CR}{} = \setted{\tupled{\model{m}{1}, \model{m}{2}, \model{m}{3}}}$, because a model tuple $\modeltuple{m}$ is consistent to $\consistencyrelation{CR}{}$ exactly when it is consistent to $\setted{\consistencyrelation{CR}{1}, \consistencyrelation{CR}{2}, \consistencyrelation{CR}{3}}$.

\mnote{Relation uniqueness assumption}
For reasons of simplicity, we assume only one consistency relation between each pair of metamodels.
This also includes that there are no two consistency relations $\consistencyrelation{CR}{i,j}$ and $\consistencyrelation{CR}{j,i}$ for metamodels $\metamodel{M}{i}$ and $\metamodel{M}{j}$, which means that the relations do not have a direction.
This assumption is without loss of generality, because two relations between the same metamodels are, independent from their direction, equivalent to only considering their intersection, i.e., only the model pairs that are considered consistent by both relations.

% \subsection{Expressiveness of Modular Consistency Specifications}
% \label{chap:correctness:formalization:expressiveness}

\mnote{Expressiveness of modular specifications}
Although in the preceding exemplary case the binary relations are equivalent to a monolithic relation, such an equivalence is not always given. In general, two interesting insights come along with the definition of consistency based on modular relations. First, expressiveness of defining consistency modularly by a set of relations is not equivalent to defining one monolithic relation. Second, a modular definition of consistency can easily contain contradictions, which can lead to an empty tuple of consistent models.

\begin{figure}
    \centering
    \newcommand{\metamodelsize}{5.5em}

\begin{tikzpicture}[
    model/.style={schematic model, minimum size=1.8em, inner sep=0.1em},
    metamodel/.style={schematic metamodel, circle, minimum size=\metamodelsize}
]

% MONOLITHIC METAMODELS
\node[metamodel] (mono_mm1) {};
\node[model, below=0.1*\metamodelsize of mono_mm1.center] (mono_mm1_1) {$\model{m}{1}$};
\node[model, above=0.1*\metamodelsize of mono_mm1.center] (mono_mm1_2) {$\model{m}{1}'$};
\node[right=0.1em of mono_mm1.west, anchor=west] {$\metamodelinstanceset{M}{1}$};

\node[metamodel, below left=1.2*\metamodelsize and 0.65*\metamodelsize of mono_mm1.center, anchor=center] (mono_mm2) {};
\node[model, above left=0.1*\metamodelsize and 0.03*\metamodelsize of mono_mm2.center] (mono_mm2_1) {$\model{m}{2}$};
\node[model, below right=0.1*\metamodelsize and 0.03*\metamodelsize of mono_mm2.center] (mono_mm2_2) {$\model{m}{2}'$};
\node[right=0.3em of mono_mm2.west, anchor=north west] {$\metamodelinstanceset{M}{2}$};

\node[metamodel, below right=1.2*\metamodelsize and 0.65*\metamodelsize of mono_mm1.center, anchor=center] (mono_mm3) {};
\node[model, above right=0.1*\metamodelsize and 0.03*\metamodelsize of mono_mm3.center] (mono_mm3_1) {$\model{m}{3}$};
\node[model, below left=0.1*\metamodelsize and 0.03*\metamodelsize of mono_mm3.center] (mono_mm3_2) {$\model{m}{3}'$};
\node[left=0.3em of mono_mm3.east, anchor=north east] {$\metamodelinstanceset{M}{3}$};

% MONOLITHIC RELATIONS
\coordinate (mono_cr1) at ([yshift=0.1em]mono_mm1.center);
\draw[consistency related element] (mono_cr1) -- (mono_mm1_2);
\draw[consistency related element] (mono_cr1) -- (mono_mm2_1);
\draw[consistency related element] (mono_cr1) -- (mono_mm3_1);
\filldraw[consistency related element] (mono_cr1) circle (0.12em);

\coordinate (mono_cr2) at ([xshift=0.4*\metamodelsize,yshift=0.45*\metamodelsize]mono_mm2);
\draw[consistency related element] (mono_cr2) -- (mono_mm1_1);
\draw[consistency related element] (mono_cr2) -- (mono_mm2_1);
\draw[consistency related element] (mono_cr2) -- (mono_mm3_2);
\filldraw[consistency related element] (mono_cr2) circle (0.15em);

\coordinate (mono_cr3) at ([xshift=-0.4*\metamodelsize,yshift=0.45*\metamodelsize]mono_mm3);
\draw[consistency related element] (mono_cr3) -- (mono_mm1_1);
\draw[consistency related element] (mono_cr3) -- (mono_mm2_2);
\draw[consistency related element] (mono_cr3) -- (mono_mm3_1);
\filldraw[consistency related element] (mono_cr3) circle (0.15em);

\node[consistency related element, anchor=center] at ($(mono_cr2)!0.5!(mono_cr3)$) {$\consistencyrelation{CR}{}$};

% MODULAR METAMODELS
\node[metamodel, right=3.2*\metamodelsize of mono_mm1.center, anchor=center] (modu_mm1) {};
\node[model, below=0.1*\metamodelsize of modu_mm1.center] (modu_mm1_1) {$\model{m}{1}$};
\node[model, above=0.1*\metamodelsize of modu_mm1.center] (modu_mm1_2) {$\model{m}{1}'$};
\node[right=0.3em of modu_mm1.west, anchor=south west] {$\metamodelinstanceset{M}{1}$};

\node[metamodel, below left=1.2*\metamodelsize and 0.65*\metamodelsize of modu_mm1.center, anchor=center] (modu_mm2) {};
\node[model, above left=0.1*\metamodelsize and 0.03*\metamodelsize of modu_mm2.center] (modu_mm2_1) {$\model{m}{2}$};
\node[model, below right=0.1*\metamodelsize and 0.03*\metamodelsize of modu_mm2.center] (modu_mm2_2) {$\model{m}{2}'$};
\node[right=0.3em of modu_mm2.west, anchor=north west] {$\metamodelinstanceset{M}{2}$};

\node[metamodel, below right=1.2*\metamodelsize and 0.65*\metamodelsize of modu_mm1.center, anchor=center] (modu_mm3) {};
\node[model, above right=0.1*\metamodelsize and 0.03*\metamodelsize of modu_mm3.center] (modu_mm3_1) {$\model{m}{3}$};
\node[model, below left=0.1*\metamodelsize and 0.03*\metamodelsize of modu_mm3.center] (modu_mm3_2) {$\model{m}{3}'$};
\node[left=0.3em of modu_mm3.east, anchor=north east] {$\metamodelinstanceset{M}{3}$};

% MODULAR RELATIONS
\draw[consistency related element] (modu_mm1_1) -- (modu_mm2_1);
\draw[consistency related element] (modu_mm1_1) -- (modu_mm2_2);
\draw[consistency related element] (modu_mm1_2) -- node[left] {$\consistencyrelation{CR}{1}$} (modu_mm2_1);

\draw[consistency related element] (modu_mm2_1) -- node[below=1.8em] {$\consistencyrelation{CR}{2}$} (modu_mm3_1);
\draw[consistency related element] (modu_mm2_1) -- (modu_mm3_2);
\draw[consistency related element] (modu_mm2_2) -- (modu_mm3_1);

\draw[consistency related element] (modu_mm1_1) -- (modu_mm3_1);
\draw[consistency related element] (modu_mm1_1) -- (modu_mm3_2);
\draw[consistency related element] (modu_mm1_2) -- node[right] {$\consistencyrelation{CR}{3}$} (modu_mm3_1);

% LABELS
\node[above=0.5em of mono_mm1.north, anchor=south, font=\bfseries\small] {Monolithic};
\node[above=0.5em of modu_mm1.north, anchor=south, font=\bfseries\small] {Modular};


\draw[-latex]
    ([xshift=0.3*\metamodelsize, yshift=0*\metamodelsize]mono_mm1.north east)
    --
    node[above] {requires binary relations}
    ([xshift=-0.3*\metamodelsize, yshift=0*\metamodelsize]modu_mm1.north west);

\draw[-latex]
    ([xshift=-0.3*\metamodelsize, yshift=-0.15*\metamodelsize]modu_mm1.north west)
    --
    node[below, align=center] {induces further elements\\ in monolithic relation, \\e.g., $\tupled{\model{m}{1}, \model{m}{2}, \model{m}{3}}$}
    ([xshift=0.3*\metamodelsize, yshift=-0.15*\metamodelsize]mono_mm1.north east);

\end{tikzpicture}
    %\includegraphics[width=\textwidth]{figures/correctness/notion/binary_definable.png}
    \caption[Monolithic consistency relation that cannot be modularized]{A monolithic consistency relation that cannot be expressed by binary relations. Small circles denote models and blue, solid hyperedges relate tuples of consistent models.}
    \label{fig:correctness:binary_definable}
\end{figure}

\mnote{Binary relations reduce expressiveness}
Obviously, combining binary relations has not the same expressiveness as defining a monolithic relation.
For example, binary relations cannot express the monolithic relation $\consistencyrelation{CR}{} = \setted{\tupled{\model{m}{1}, \model{m}{2}, \model{m}{3}'}, \tupled{\model{m}{1}, \model{m}{2}', \model{m}{3}}, \tupled{\model{m}{1}', \model{m}{2}, \model{m}{3}}}$, as depicted in \autoref{fig:correctness:binary_definable}.
The binary relations necessarily need to contain $\tupled{\model{m}{1}, \model{m}{2}}$ because $\tupled{\model{m}{1}, \model{m}{2}, \model{m}{3}'} \in \consistencyrelation{CR}{}$, $\tupled{\model{m}{1}, \model{m}{3}}$ because $\tupled{\model{m}{1}, \model{m}{2}', \model{m}{3}} \in \consistencyrelation{CR}{}$, and $\tupled{\model{m}{2}, \model{m}{3}}$ because $\tupled{\model{m}{1}', \model{m}{2}, \model{m}{3}} \in \consistencyrelation{CR}{}$. 
However, this would mean that $\tupled{\model{m}{1}, \model{m}{2}, \model{m}{3}}$ is consistent to the binary relations although it is not consistent to the monolithic relation $\consistencyrelation{CR}{}$.
Thus, using sets of binary relations in contrast to a single monolithic relation reduces expressiveness.
\textcite{stevens2020BidirectionalTransformationLarge-SoSym} discusses the property of a multiary relation to be expressible by binary ones as \emph{binary-definable} in detail.
She proposes restrictions to binary relations that may be sufficient and still practical for expressing consistency, such as a notion of \emph{binary-implemented} relations.
We have reasoned the assumption that relations are specified independently and thus modularly, thus we have to accept these theoretic restrictions in expressiveness anyway.

\begin{figure}
    \centering
    \newcommand{\metamodelsize}{5.5em}

\begin{tikzpicture}[
    model/.style={schematic model, minimum size=1.8em, inner sep=0.1em},
    metamodel/.style={schematic metamodel, circle, minimum size=\metamodelsize}
]

% METAMODELS 1
\node[metamodel] (first_mm1) {};
\node[model, right=0.1*\metamodelsize of first_mm1.center, anchor=center] (first_mm1_1) {$\model{m}{1}$};
\node[right=0.3em of first_mm1.west, anchor=west] {$\metamodelinstanceset{M}{1}$};

\node[metamodel, below left=1.1*\metamodelsize and 0.7*\metamodelsize of first_mm1.center, anchor=center] (first_mm2) {};
\node[model, right=0.1*\metamodelsize of first_mm2.center, anchor=center] (first_mm2_1) {$\model{m}{2}$};
\node[right=0.3em of first_mm2.west, anchor=west] {$\metamodelinstanceset{M}{2}$};

\node[metamodel, below right=1.1*\metamodelsize and 0.7*\metamodelsize of first_mm1.center, anchor=center] (first_mm3) {};
\node[model, above right=0.1*\metamodelsize and 0.03*\metamodelsize of first_mm3.center] (first_mm3_1) {$\model{m}{3}$};
\node[model, below left=0.05*\metamodelsize and 0.03*\metamodelsize of first_mm3.center] (first_mm3_2) {$\model{m}{3}'$};
\node[left=0.3em of first_mm3.east, anchor=north east] {$\metamodelinstanceset{M}{3}$};

% RELATIONS 1
\draw[consistency related element] (first_mm1_1) -- node[pos=0.47, left] {$\consistencyrelation{CR}{1}$} (first_mm2_1);
\draw[consistency related element] (first_mm2_1) -- node[pos=0.55, below] {$\consistencyrelation{CR}{2}$} (first_mm3_2);
\draw[consistency related element] (first_mm1_1) -- node[pos=0.47, right] {$\consistencyrelation{CR}{3}$} (first_mm3_1);

% METAMODELS 2
\node[metamodel, right=2.7*\metamodelsize+0.6*\difftoafiveimage of first_mm1.center, anchor=center] (second_mm1) {};
\node[model, right=0.1*\metamodelsize of second_mm1.center, anchor=center] (second_mm1_1) {$\model{m}{1}$};
\node[right=0.3em of second_mm1.west, anchor=west] {$\metamodelinstanceset{M}{1}$};

\node[metamodel, below left=1.1*\metamodelsize and 0.7*\metamodelsize of second_mm1.center, anchor=center] (second_mm2) {};
\node[model, right=0.1*\metamodelsize of second_mm2.center, anchor=center] (second_mm2_1) {$\model{m}{2}$};
\node[right=0.3em of second_mm2.west, anchor=west] {$\metamodelinstanceset{M}{2}$};

\node[metamodel, below right=1.1*\metamodelsize and 0.7*\metamodelsize of second_mm1.center, anchor=center] (second_mm3) {};
\node[model, above right=0.1*\metamodelsize and 0.03*\metamodelsize of second_mm3.center] (second_mm3_1) {$\model{m}{3}$};
\node[model, below left=0.05*\metamodelsize and 0.03*\metamodelsize of second_mm3.center] (second_mm3_2) {$\model{m}{3}'$};
\node[left=0.3em of second_mm3.east, anchor=north east] {$\metamodelinstanceset{M}{3}$};

% RELATIONS 2
\draw[consistency related element] (second_mm1_1) -- node[pos=0.47,  left] {$\consistencyrelation{CR}{1}$} (second_mm2_1);
\draw[consistency related element] (second_mm2_1) -- (second_mm3_1);
\draw[consistency related element] (second_mm2_1) -- node[pos=0.55, below] {$\consistencyrelation{CR}{2}$} (second_mm3_2);
\draw[consistency related element] (second_mm1_1) -- node[pos=0.47, right] {$\consistencyrelation{CR}{3}$} (second_mm3_1);

\end{tikzpicture}
    %\includegraphics[width=0.9\textwidth]{figures/correctness/notion/contradictions_example.png}
    \caption[Example for incompatible consistency relations]{Modular consistency relations, which together cannot be fulfilled (left) or which cannot be fulfilled for some of the consistent model pairs (right). Small circles denote models and (blue) lines relate consistent model pairs.}
    \label{fig:correctness:contradictions_example}
\end{figure}

\mnote{Contradictions of binary relations}
Additionally, it can easily occur that multiple binary relations can be fulfilled by certain models, but no tuple of models exists that is consistent to all of them. Consider the relations $\consistencyrelation{CR}{1} = \setted{\tupled{\model{m}{1}, \model{m}{2}}}$, $\consistencyrelation{CR}{2} = \setted{\tupled{\model{m}{2}, \model{m}{3}'}}$, and $\consistencyrelation{CR}{3} = \setted{\tupled{\model{m}{1}, \model{m}{3}}}$, which are also depicted at the left of \autoref{fig:correctness:contradictions_example}.
Although for each of these relations a consistent pair of models exists, which is exactly the one defined in each relation, no tuple of models exists that fulfills their combination.
This example illustrates the worst case, in which no consistent models exist for a set of relations.
In other cases, only for some models that are consistent according to one or some of the relations no model tuple may exist that is consistent to all relations.
Consider the relations $\consistencyrelation{CR}{1} = \setted{\tupled{\model{m}{1}, \model{m}{2}}}$, $\consistencyrelation{CR}{2} = \setted{\tupled{\model{m}{2}, \model{m}{3}}, \tupled{\model{m}{2}, \model{m}{3}'}}$, and $\consistencyrelation{CR}{3} = \setted{\tupled{\model{m}{1}, \model{m}{3}}}$, which are also depicted at the right of \autoref{fig:correctness:contradictions_example}.
In this case, the tuple $\tupled{\model{m}{1}, \model{m}{2}, \model{m}{3}}$ is considered consistent to the relations, but although $\tupled{\model{m}{2}, \model{m}{3}'} \in \consistencyrelation{CR}{2}$ there exists no consistent model tuple containing $\model{m}{3}$, i.e., there is no $\model{m}{1}^* \in \metamodelinstanceset{M}{1}$ such that $\tupled{\model{m}{1}^*, \model{m}{2}, \model{m}{3}'}$ is consistent to all relations.

\mnote{Forbidden models through relations}
It is easy to see that one monolithic relation can be equally represented by an arbitrary number of sets of binary relations by simply adding model pairs to these binary relations that are never consistent to the other relations, like we have seen for the pair $\tupled{\model{m}{2}, \model{m}{3}'}$ in the previous example.
This means that the combination of relations can lead to the situation that some models are actually forbidden (like $\model{m}{3}'$ in the example before) due to the combination of consistency relations.
Whether such a situation is intended can eventually depend on the semantics of the models and relations, but we will discuss which situations are unintended in general.
We have informally discussed this as a notion of \emph{compatibility}, for which we investigate in \autoref{chap:compatibility} how far this behavior should be expected.


\subsection{Incremental Consistency Preservation}
\label{chap:correctness:formalization:incremental_inductive}

\mnote{Preserving consistency}
While the previous discussion only concerned when models are considered consistent, it is of particular interest to ensure that consistency of models is preserved.
We informally introduced such specifications as consistency preservation rules.
In the following, we will restrict ourselves to \emph{incremental} and \emph{inductive} consistency preservation and give a precise definition for that.
This means that we make the following assumptions to the process.
\begin{properdescription}
    \item[Information Preservation (Incrementality):] After a change to one model, the others are not generated from scratch but updated according to the performed changes. This ensures that information that cannot be generated but was added by users to the other models is preserved.
    \item[Consistency Assumption (Induction):] We assume models to be consistent before a change is processed by consistency preservation rules. Otherwise, the preservation rules would need to be able to handle arbitrary states of the models and intentions of performed changes could not be incorporated to restore consistency.
\end{properdescription}
Incrementality is an essential requirement whenever consistency shall be preserved to avoid information loss. Otherwise, if for example Java code is always generated anew after changes to a \gls{UML} model instead of adapting it incrementally, all implementations of methods in Java get lost every time the \gls{UML} model is changed.
Inductivity, on the other hand, may not be necessary, as consistency preservation rules could also be defined to restore consistency from arbitrarily inconsistent states.
We, however, make this assumption to avoid requiring from the consistency preservation rules that they need to be able to process an inconsistent state without knowing which changes introduced it.
From a theoretical point of view, we could omit that requirement, but this would make the specification of consistency preservation rules impractically complicated, such that omitting that requirement is not practically relevant anyway.

\mnote{Monolithic and modular consistency preservation}
Like we have discussed for consistency preservation rules in general, incremental preservation rules can be realized in an either monolithic or modular way.
A monolithic consistency preservation rule takes a tuple of models that is consistent to a consistency relation and a change to these models, and it returns a tuple of models that is consistent again.
In a modular specification of consistency preservation rules, a set of such rules is given of which each preserves consistency of a subset of the given models according to a modular consistency relation.
In our case, we consider such rules for two models, each of them restoring consistency according to a binary consistency relation.

\mnote{Drawbacks of existing consistency preservation approaches}
In existing terminology for transformations, a consistency preservation rule that restores consistency of models according to a consistency relation in one direction is called \emph{directional transformation}~\cite{stevens2010sosym} or \emph{consistency restorer}~\cite{stevens2020BidirectionalTransformationLarge-SoSym}.
That terminology usually considers model states instead of changes and defines a consistency preservation rule $\consistencypreservationrule{}$ for metamodels $\metamodel{M}{1}$ and $\metamodel{M}{2}$ to modify the instance of $\metamodel{M}{2}$ for restoring consistency as:
\begin{align*}
    \consistencypreservationrule{}: \metamodelinstanceset{M}{1} \times \metamodelinstanceset{M}{2} \rightarrow \metamodelinstanceset{M}{2}
\end{align*}
This notion, however, has two properties that imply essential drawbacks:
\begin{properdescription}
    \item[State-Based:] Information about the performed changes that led to the inconsistent state is missing. Thus the specification is not aware of how the inconsistent state was reached.
    \item[Unidirectional:] The specification is unidirectional, which always requires to only update one model to restore consistency.
\end{properdescription}
State-based transformations suffer from not knowing which changes were made that led to an inconsistent state, and reconstructing them from the difference between two states is only a heuristic approximation~\cite{diskin2011StateToDeltaSymmetric-MODELS}.
This, for example, includes that information about elements that were moved or renamed can potentially not be reconstructed, leading to elements that are deleted and created anew and losing all information that was potentially added to them.
Unidirectionality may be reasonable when assuming that only one of the models was modified. In that case, it is sufficient to update the other model to restore consistency.
With a modular specification of consistency preservation, however, several consistency preservation rules modifying the same models may need to be executed.

\begin{figure}
    \centering
    \newcommand{\distance}{6.5em}

\begin{tikzpicture}

\umlhuman{person}{}{}{}{0.7}
\node[below=0em of person.south, anchor=north, align=center] {System \\ Developer};

\node[schematic model, right=\distance of person] (m1) {$\model{m}{1}$};
\node[schematic model, above right=0.6*\distance and \distance of m1.center, anchor=center] (m2) {$\model{m}{2}$};
\node[schematic model, right=2*\distance of m1.center, anchor=center] (m3) {$\model{m}{3}$};

\draw[consistency execution, -latex]
    (person)
    --
    node[above] {«modifies»}
    (m1);
\draw[consistency execution, -latex]
    (m1)
    --
    node[below right] {1.}
    node[above left] {$\consistencypreservationrule{1}$}
    (m2);
\draw[consistency execution, -latex]
    (m2)
    --
    node[below left] {2.}
    node[above right] {$\consistencypreservationrule{2}$}
    (m3);
\draw[consistency execution, latex-latex]
    (m1)
    --
    node[above] {3.}
    node[below] {$\consistencypreservationrule{3}$}
    (m3);

\draw[decorate,decoration={brace,amplitude=10pt}] 
    ([yshift=-0.3em]m3.south west) 
    --
    node[below=1em, align=center] {not unidirectional, because\\ $\model{m}{1}$ and $\model{m}{3}$ have both been changed}
    ([yshift=-0.3em]m1.south east);


\end{tikzpicture}
    %\includegraphics[width=0.7\textwidth]{figures/correctness/notion/not_unidirectional.png}
    \caption[Unidirectional consistency preservation in networks]{Execution of consistency preservation rules of which at least one cannot be unidirectional, because both involved models (circles) have been modified by a user or other consistency preservation rules.}
    \label{fig:correctness:unidirectionality_example}
\end{figure}

\mnote{Counterexample for unidirectionality}
\autoref{fig:correctness:unidirectionality_example} depicts an example in which unidirectional consistency preservation rules cannot be applied when used in combination with other such rules.
If the depicted consistency preservation rules $\consistencypreservationrule{1}$ and $\consistencypreservationrule{2}$ are executed first, $\consistencypreservationrule{3}$ cannot be unidirectional, because both involved models $\model{m}{1}$ and $\model{m}{3}$ have been modified by either the user or another consistency preservation rule.
Thus, it is, in general, not possible to only consider changes in one model and unidirectionally propagate them to the other model.
In consequence, the preservation rules need to be able to deal with changes performed in both models and, consequentially, need to update both models to reflect the changes in each other.

\mnote{Synchronizing consistency preservation rules}
To be able to combine several consistency preservation rules without the discussed drawbacks, we define a \emph{synchronizing} rather than a unidirectional notion of them.
Those rules can react to changes in both models and produce changes in both models again.
This is sometimes also called the capability of handling \emph{concurrent} modifications (e.g.~\cite{leblebici2014IncrementalTGGSurvey-GTVMT}).
To precisely define this behavior, we introduce a notion of \emph{changes} and \emph{consistency preservation rules}, which we also refer to as \emph{synchronizing} consistency preservation rules.

\mnote{Changes as functions}
As motivated before, we base our notion of consistency preservation on changes to explicitly express how an inconsistent state was derived from a previously consistent one.
We consider these changes as functions that take a model and return a new one.
They are not restricted to a specific model but defined for all instances of a metamodel, because a change is supposed to represent how specific elements are modified, such as adding, removing or modifying them.
Thus, they can be applied to any models containing these affected elements.
This is also how actual implementations, such as the one in the \gls{EMF} behave.
When elements affected by a change are not present in a model, applying the change may fail.
For that reason, we consider the function describing a change to be partial.
We denote partiality by returning $\bot$ for inputs the function is undefined for.
\begin{definition}[Change]
    \label{def:change}
    Given a metamodel $\metamodel{M}{}$, a change $\change{\metamodel{M}{}}$ is a partial function that takes an instance of that metamodel and returns another one or $\bot$:
    \begin{align*}
        \change{\metamodel{M}{}}: \metamodelinstanceset{M}{} \rightarrow \metamodelinstanceset{M}{} \cup \setted{\bot}
    \end{align*}
    We denote the identity change, i.e., the one always returning the input model, as $\identitychange$:
    \begin{align*}
        \identitychange(x) \equalsperdefinition x
    \end{align*}
    We denote the universe of all changes in $\metamodel{M}{}$, i.e., all injective subsets of $\metamodelinstanceset{M}{} \times \metamodelinstanceset{M}{}$, as:
    \begin{align*}
        \changeuniverse{\metamodel{M}{}} \equalsperdefinition \setted{\change{\metamodel{M}{}} \subseteq \metamodelinstanceset{M}{} \times \metamodelinstanceset{M}{} \mid
        \forall \tupled{\model{m}{1}, \model{m}{2}}, \tupled{\model{m}{1}, \model{m}{2}'} \in \change{\metamodel{M}{}} : \model{m}{2} = \model{m}{2}'}
    \end{align*}
\end{definition}
\begin{definition}[Change Tuple]
    \label{def:changetuple}
    For a given metamodel tuple $\metamodeltuple{M} = \tupled{\metamodelsequence{M}{n}}$, we denote a tuple of changes to an instance of each metamodel as:
    \begin{align*}
        \changetuple{\metamodeltuple{M}} = \tupled{\change{\metamodel{M}{1}}, \dots, \change{\metamodel{M}{n}}} \in \changeuniverse{\metamodel{M}{1}} \times \dots \times \changeuniverse{\metamodel{M}{n}}
    \end{align*}
    We define the universe of change tuples in instance tuples of 
    $\metamodeltuple{M}$ as:
    \begin{align*}
        \changeuniverse{\metamodeltuple{M}} \equalsperdefinition \changeuniverse{\metamodel{M}{1}} \times \dots \times \changeuniverse{\metamodel{M}{n}}
    \end{align*}
    We define the application of a change tuple $\changetuple{\metamodeltuple{M}} = \tupled{\change{\metamodel{M}{1}}, \dots, \change{\metamodel{M}{n}}}$ to a model tuple $\modeltuple{m} = \tupled{\modelsequence{m}{n}} \in \metamodeltupleinstanceset{M}$ as the element-wise application:
    \begin{align*}
        \changetuple{\metamodeltuple{M}}(\modeltuple{m}) \equalsperdefinition \tupled{\change{\metamodel{M}{1}}(\model{m}{1}), \dots, \change{\metamodel{M}{n}}(\model{m}{n})}
    \end{align*}
\end{definition}

\mnote{Change behavior}
For us, it does not matter how the function behaves in cases in which the encoded change cannot be applied, e.g., because the changed or removed element does not exist. The function may do nothing, i.e., return the identical model, or even be undefined for those models, i.e., be partial and return $\bot$.
%
%\mnote{Reasonable change behavior}
In fact, we do not restrict the actual behavior of a change in any way.
It may return an empty model regardless of the input, or it may perform arbitrary changes to different models instead of affecting only specific elements.
Since we do not need such restrictions, they are not reflected in the formalism.

\mnote{Consistency preservation rules}
With that notion of changes, we can define consistency preservation rules as functions that receive two models and changes to them, and that return new changes to both models.
While the general definition does not prescribe this, we assume the resulting changes to include the input changes such that not both of them have to be executed consecutively.
This will also be reflected by a correctness notion for such rules.

\begin{definition}[Consistency Preservation Rule]
    \label{def:consistencypreservationrule}
    Let $\consistencyrelation{CR}{} \subseteq \metamodelinstanceset{M}{1} \times \metamodelinstanceset{M}{2}$ be a binary \modellevelconsistencyrelation between metamodels $\metamodel{M}{1}$ and $\metamodel{M}{2}$.
    A \emph{consistency preservation rule} $\consistencypreservationrule{\consistencyrelation{CR}{}}$ for the relation $\consistencyrelation{CR}{}$ is a function:
    \begin{align*}
        \consistencypreservationrule{\consistencyrelation{CR}{}} : (\metamodelinstanceset{M}{1}, \metamodelinstanceset{M}{2}, \changeuniverse{\metamodel{M}{1}}, \changeuniverse{\metamodel{M}{2}}) \rightarrow (\changeuniverse{\metamodel{M}{1}}, \changeuniverse{\metamodel{M}{2}}) \cup \setted{\bot}
    \end{align*}
\end{definition}

\mnote{Partial consistency preservation rules}
For reasons of practical applicability, the rules need to be partial, as we may not want to require them to be able to process arbitrary models and changes.
Like for changes, we denote this partiality by allowing the function to return $\bot$.
First, this is because we do not require it to produce changes when the input models were not consistent.
Second, even if the input models are consistent, it may not be possible to preserve consistency for the given changes.
For example, if conflicting changes in both changes are made, i.e., changes that require one of them to be reverted, it may be desired that the consistency preservation rule does not return an unexpected result but to indicate a failure by returning $\bot$.
Our formalism does not restrict such a behavior, in fact it even allows to always return the same changes or to return changes that always deliver empty models.
Finally, it is up to the developer to define reasonable consistency preservation rules and to define in which cases the function does not return a result.

\mnote{Conformance to existing definitions}
This notion of synchronizing consistency preservation conforms to the definition of \emph{synchronizers} given by \textcite{xiong2013SynchronizingConcurrentUpdates-SoSym}, which also reflect the case that both models have been modified and can be updated by the consistency preservation rule.
They do, however, encode the changes in terms of new model states rather than explicit changes.

\mnote{Correctness notion}
To consider a consistency preservation rule \emph{correct}, it has to return changes that, when applied to the input models, result in models that are consistent according to the \modellevelconsistencyrelation for which the preservation rule is defined.
This conforms to the notion of correctness defined for bidirectional transformations~\cite{stevens2010sosym} and the notion of consistency given for synchronizers by \textcite{xiong2013SynchronizingConcurrentUpdates-SoSym}.

\begin{definition}[Consistency Preservation Rule Correctness]
    \label{def:consistencypreservationrulecorrectness}
    We call a consistency preservation rule $\consistencypreservationrule{\consistencyrelation{CR}{}}$ \emph{correct} if, and only if, it either returns $\bot$ or changes that applied to the input models yield models that are consistent to $\consistencyrelation{CR}{}$:
    \begin{align*} &
        \consistencypreservationrule{\consistencyrelation{CR}{}} \correctmath \equivalentperdefinition
        \forall 
        \model{m}{1} \in \metamodelinstanceset{M}{1}, 
        \model{m}{2} \in \metamodelinstanceset{M}{2},
        \change{\metamodel{M}{1}} \in \changeuniverse{\metamodel{M}{1}},
        \change{\metamodel{M}{2}} \in \changeuniverse{\metamodel{M}{2}} : \\
        & \formulaskip
        \bigl( \exists 
        \change{\metamodel{M}{1}}' \in \changeuniverse{\metamodel{M}{1}},
        \change{\metamodel{M}{2}}' \in \changeuniverse{\metamodel{M}{2}} :
        \consistencypreservationrule{\consistencyrelation{CR}{}}(\model{m}{1}, \model{m}{2}, \change{\metamodel{M}{1}}, \change{\metamodel{M}{2}}) = (\change{\metamodel{M}{1}}', \change{\metamodel{M}{2}}') \\
        & \formulaskip\formulaskip
        \Rightarrow
        \tupled{\change{\metamodel{M}{1}}'(\model{m}{1}), \change{\metamodel{M}{2}}'(\model{m}{2})} \consistenttomath \consistencyrelation{CR}{} \bigr)
    \end{align*}
\end{definition}

\mnote{Reasonable consistency preservation rules}
This definition does not restrict how the input and output changes are related.
In fact, a valid (and especially correct) consistency preservation rule could always return identity changes.
In consequence, the rule would simply revert all input changes to achieve a consistent state.
Although this may not be the expected behavior, there is no reason to restrict this behavior by definition.
Actually, the developer should specify a preservation rule in a \emph{reasonable} way, such that it defines an expected behavior.

\mnote{Transformations}
We have discussed that consistency preservation rules can be derived from consistency relations and that consistency preservation rules can imply the consistency relations by their image, i.e., the set of all models that can be derived by applying the consistency preservation rule to any models and changes for which it is defined.
In practice there will only be one of these specifications and the other is implied or derived.
We thus define a \emph{synchronizing transformation}, in extension to \emph{bidirectional transformations}~\cite{stevens2010sosym}, as an artifact that encapsulates a \modellevelconsistencyrelation together with a consistency preservation rule, no matter which of them is defined and which is derived or implied.

\begin{definition}[Synchronizing Transformation]
    \label{def:synchronizingtransformation}
    Let $\consistencyrelation{CR}{}$ be a \modellevelconsistencyrelation and $\consistencypreservationrule{\consistencyrelation{CR}{}}$ a consistency preservation rule that restores consistency according to that relation.
    A \emph{synchronizing transformation} is a pair $\transformation{t} = \tupled{\consistencyrelation{CR}{}, \consistencypreservationrule{\consistencyrelation{CR}{}}}$.
\end{definition}

\mnote{Transformation correctness}
We also use the short term \emph{transformation} for a synchronizing transformation.
Correctness of a transformation is then given by correctness of its consistency preservation rule.

\begin{definition}[Synchronizing Transformation Correctness]
    \label{def:synchronizingtransformationcorrectness}
    Let $\transformation{t} = \tupled{\consistencyrelation{CR}{}, \consistencypreservationrule{\consistencyrelation{CR}{}}}$ be a synchronizing transformation.
    We say that $\transformation{t}$ is \emph{correct} if, and only if, $\consistencypreservationrule{\consistencyrelation{CR}{}}$ is correct according to \autoref{def:consistencypreservationrulecorrectness}:
    \begin{align*}
        &
        \transformation{t} \correctmath \equivalentperdefinition \consistencypreservationrule{\consistencyrelation{CR}{}} \correctmath
    \end{align*}
\end{definition}

\mnote{Hippocraticness}
Transformations are usually expected to by \emph{hippocratic}~\cite{stevens2010sosym}.
This means that a transformation, or more precisely its consistency preservation rule, does not perform any changes if the input changes applied to the input models already yield consistent models.
We define the application of hippocraticness to synchronizing transformations as follows.
\begin{definition}[Hippocratic Synchronizing Transformation]
    \label{def:hippocratictransformation}
    Let $\transformation{t} = \tupled{\consistencyrelation{CR}{}, \consistencypreservationrule{\consistencyrelation{CR}{}}}$ be a transformation for metamodels $\metamodel{M}{1}$ and $\metamodel{M}{2}$.
    We say that $\transformation{t}$ is \emph{hippocratic} if, and only if, it returns the input changes if their application to the input models yields consistent models:
    \parameterizeformat{
    \begin{align*}
        &
        \transformation{t} \mathtextspacearound{hippocractic} \equivalentperdefinition
        \forall
        \model{m}{1} \in \metamodelinstanceset{M}{1}, \model{m}{2} \in \metamodelinstanceset{M}{2}, 
        \change{\metamodel{M}{1}} \in \changeuniverse{\metamodel{M}{2}}, \change{\metamodel{M}{2}} \in \changeuniverse{\metamodel{M}{2}} : \\
        & #1 #2
        \tupled{\change{\metamodel{M}{1}}(\model{m}{1}), \change{\metamodel{M}{2}}(\model{m}{2})} \in \consistencyrelation{CR}{} \Rightarrow \consistencypreservationrule{\consistencyrelation{CR}{}}(\model{m}{1}, \model{m}{2}, \change{\metamodel{M}{1}}, \change{\metamodel{M}{2}}) = (\change{\metamodel{M}{1}}, \change{\metamodel{M}{2}})
    \end{align*}
    }{\formulaskip}{\formulaskip[0.5em]}%
\end{definition}

\mnote{Benefits of hippocraticness}
Although hippocraticness is not a necessary requirement for our considerations in most cases, it is usually a desired property in practice~\cite{stevens2010sosym}.
One benefit of hippocraticness with regards to transformations is given if a transformation is only defined by its consistency preservation rule and thus implies the underlying consistency relation as its fixed points, as discussed in \autoref{chap:correctness:notions_consistency:declarative_imperative}.
Actually, a consistency preservation rule according to our definition does not have fixed points, because the signatures of definition and value set of the function are different due to the models only occurring in the definition set.
Transferred to our definition, the consistency relation is implied by iteratively applying the function to each pair of models and changes with the changes delivered by the function until they are not modified by the function anymore.
In case that the transformation is correct and hippocratic, it does always deliver changes that yield consistent models already upon its first execution and does not modify them upon further applications, thus the consistency relation is implied by applying the function to each pair of models and changes only once.

\mnote{Generalization to transformations}
In the following, we only refer to transformations rather than consistency relations and consistency preservation rules if the distinction is not necessary.
We thus also say that models are consistent to a transformation, which is supposed to mean that they are consistent to the consistency relation encapsulated by that transformation.

\begin{definition}[Consistency to Transformation]
    \label{def:consistencytransformation}
    Let $\transformation{t} = \tupled{\consistencyrelation{CR}{}, \consistencypreservationrule{\consistencyrelation{CR}{}}}$ be a synchronizing transformation.
    We say that a tuple of models $\modeltuple{m}$ is \emph{consistent to} $\transformation{t}$ if, and only if, it is consistent to its consistency relation:
    \begin{align*}
        \modeltuple{m} \consistenttomath \transformation{t} \equivalentperdefinition \modeltuple{m} \consistenttomath \consistencyrelation{CR}{}
    \end{align*}
    %
    For a set of transformations $\transformationset{T}$, we say that a model tuple $\modeltuple{m}$ is \emph{consistent to} $\transformationset{T}$ if, and only if, it is consistent to all transformations in it:
    \begin{align*}
        \modeltuple{m} \consistenttomath \transformationset{T} \equivalentperdefinition \forall \transformation{t} \in \transformationset{T} : \modeltuple{m} \consistenttomath \transformation{t}
    \end{align*}
\end{definition}

\mnote{Synchronization problem}
Although \autoref{def:synchronizingtransformationcorrectness} precisely defines correctness of a transformation, it is unclear how to define a transformation that fulfills that property.
In particular, most existing transformation languages are restricted to input changes to one model or to delivering changes to one model.
We thus discuss how we can achieve a correct synchronizing transformation with such a restricted formalism.
This question was introduced as \autoref{rq:correctness:synchronization}, and an approach for that constitutes our contribution \autoref{contrib:correctness:synchronization}, which we discuss in \autoref{chap:synchronization}.


\subsection{Transformation Orchestration}

\mnote{Multiple transformations need orchestration}
Preserving consistency between instances of multiple metamodels after changes with multiple transformations requires their orchestration, i.e., the decision in which order to execute them.
We have discussed in \autoref{chap:correctness:notions_consistency:preservation} that transformations, or more precisely their consistency preservation rules, may be executed independently, which requires their results to be unified, or to execute them consecutively.
We have identified the drawbacks of concurrent execution, including the necessity to define unification operators and the missing guarantee of consistency after unification.
This is why we follow the approach of consecutively executing transformations.

\mnote{Orchestration function to determine execution order}
To consecutively execute transformations, an execution order has to be determined.
While in practice a dynamic algorithm will determine that order, from a theoretical perspective that algorithm realizes a function that returns the execution order.
We call this an \emph{orchestration function}, as it is responsible for orchestrating the transformation execution.

\begin{definition}[Transformation Orchestration Function] \label{def:orchestrationfunction}
    Let $\transformationset{T}$ be a set of transformations for metamodels $\metamodeltuple{M}$.
    A transformation orchestration function $\orcfunction{\transformationset{T}}$ for these transformations is a function that delivers a sequence of transformations for given models and changes:
    \begin{align*}
        &
        \orcfunction{\transformationset{T}} : (\metamodeltupleinstanceset{M}, \changeuniverse{\metamodeltuple{M}}) \rightarrow \transformationset{T}^{< \mathbb{N}}
    \end{align*}
    $\transformationset{T}^{< \mathbb{N}}$ denotes all finite sequences in $\transformationset{T}$, i.e., $\transformationset{T}^{< \mathbb{N}} \equalsperdefinition \bigcup_{i=0}^{\infty} \transformationset{T}^i$
\end{definition}

\mnote{Repetitions in orchestration}
The orchestration function returns a sequence of transformations and determines that their consistency preservation rules need to be executed in the given order. 
This especially includes that transformations may occur more than once in such a sequence.

\mnote{Existence of reasonable orchestrations}
Without further restrictions to the transformations, an orchestration function may not always find an execution order that yields a consistent model tuple for given transformations, models, and changes to them.
Such an order may not exist, because due to the transformations making local decisions to restore consistency for two models that are never consistent with the other transformations.
Additionally, even if such an order exists, it may not be possible to find it.
%
%\mnote{Decidability of existence of orchestration}
We discuss these problems in detail in \autoref{chap:orchestration} and prove that the decision problem whether an orchestration that leads to a consistent result exists is undecidable without further restrictions.
For that reason, the definition does not require that an orchestration of transformations has to lead to a consistent result.

\mnote{Explicit function for transformation application}
An orchestration function only determines an order of transformations.
Consistency for given models and changes can be preserved by requesting an orchestration from that function and executing the transformations in that order.
We make this process explicit by defining an \emph{application function} that performs consistency preservation based on given transformations, an orchestration function for them and the actual models and changes.

\mnote{Generalization function for transformation concatenation}
Before defining that application function, we first need to define an auxiliary function to concatenate transformations, more precisely their contained consistency preservation rules.
Consistency preservation rules according to \autoref{def:consistencypreservationrule} are restricted to the two metamodels they are defined for.
Additionally, they require initial models and changes as input, but only return changes.
For these two reasons, the functions describing the preservation rules cannot be easily concatenated.
This, however, is necessary to compose them to formally describe their consecutive execution.
We define a \emph{generalization function} for transformations, which generalizes them to arbitrary metamodel tuples and a conforming signature for their input and output, which eases the description of their concatenation.

\begin{definition}[Transformation Generalization Function]
    \label{def:generalizationfunction}
    Let $\metamodeltuple{M} = \tupled{\metamodel{M}{1}, \dots, \metamodel{M}{i}, \dots, \metamodel{M}{k}, \dots, \metamodel{M}{n}}$ be a metamodel tuple and let $\transformation{t} = \tupled{\consistencyrelation{CR}{}, \consistencypreservationrule{\consistencyrelation{CR}{}}}$ be a transformation for metamodels $\metamodel{M}{i}, \metamodel{M}{k}$.
    A transformation generalization function $\generalizationfunction{\metamodeltuple{M},\transformation{t}}$ for metamodels $\metamodeltuple{M}$ and transformation $\transformation{t}$ is a partial function:
    \begin{align*}
        \generalizationfunction{\metamodeltuple{M},\transformation{t}} : (\metamodeltupleinstanceset{M}, \changeuniverse{\metamodeltuple{M}}) \rightarrow (\metamodeltupleinstanceset{M}, \changeuniverse{\metamodeltuple{M}}) \cup \setted{\bot}
    \end{align*}
    It generalizes the consistency preservation rule $\consistencypreservationrule{\consistencyrelation{CR}{}}$ of $\transformation{t}$ such that it can be applied to changes in $\metamodeltuple{M}$ instead of $\metamodel{M}{i}$ and $\metamodel{M}{k}$, i.e., it applies the changes delivered by $\consistencypreservationrule{\consistencyrelation{CR}{}}$ for the corresponding models to the given change tuple.
    Let $\modeltuple{m} \in \metamodeltupleinstanceset{M}$ be a model tuple and let $\changetuple{\metamodeltuple{M}} = \tupled{\change{\metamodel{M}{1}}, \dots, \change{\metamodel{M}{i}}, \dots, \change{\metamodel{M}{k}}, \dots, \change{\metamodel{M}{n}}}$ be a change tuple.
    We define $\tupled{\change{\metamodel{M}{i}}', \change{\metamodel{M}{k}}'} \equalsperdefinition \consistencypreservationrule{\consistencyrelation{CR}{}}(\model{m}{i}, \model{m}{k}, \change{\metamodel{M}{i}}, \change{\metamodel{M}{k}})$.
    Then we define:
    \parameterizeformat{
    \begin{align*}
        &
        \generalizationfunction{\metamodeltuple{M},\transformation{t}}(\modeltuple{m}, \changetuple{\metamodeltuple{M}})
        #2
        \equalsperdefinition
        \begin{cases}
            \bot, & 
                \ifmath \tupled{\change{\metamodel{M}{i}}', \change{\metamodel{M}{k}}'} = \bot \\
            (\modeltuple{m}, \tupled{\change{\metamodel{M}{1}}, \dots, \change{\metamodel{M}{i}}', \dots, \change{\metamodel{M}{k}}', \dots, \change{\metamodel{M}{n}}}), & 
                \otherwisemath
        \end{cases}
    \end{align*}
    }{}{\\ & \formulaskip}
\end{definition}

\mnote{Partial and universal generalization function}
Like consistency preservation rules, a generalization function must be partial and return $\bot$ for inputs it is undefined for to reflect cases in which it cannot return a result.
This is a direct consequence of consistency preservation rules being partial, thus a generalization function is defined to return $\bot$ in the same cases as the consistency preservation rule it generalizes.
%
%\mnote{Generalization function is universal}
The generalization function is a universally-defined auxiliary function only necessary for formalizing the concepts.
It must neither be specialized for each transformation, nor must a transformation developer specify it at all.

\mnote{Dealing with unresolvable cases}
Finally, either the orchestration function or an application function must be able to reflect the cases in which no execution order of transformations that restores consistency can be found.
In accordance to existing terminology~\cite{stevens2020BidirectionalTransformationLarge-SoSym}, we call these cases \emph{unresolvable}.
From a theoretical perspective, it does not matter whether the orchestration or application function makes that decision, as the orchestration function could even be encoded into the application function.
From a practical perspective, however, we may want to determine an execution order even if there is no order that results in a consistent state.
This supports finding out why no such order is found, e.g., which transformation induces that problem.

\mnote{Application function}
We define a transformation application function that applies transformations to a given tuple of models and changes according to an order delivered by an orchestration function.
This function is partial to allow it to indicate that no result with consistent models could be found, e.g., because the input models were inconsistent or because a transformation within the orchestration delivered $\bot$.
We indicate those cases with the result $\bot$.

\begin{definition}[Transformation Application Function] \label{def:applicationfunction}
    Let $\transformationset{T}$ be a synchronizing transformations set for consistency relations $\consistencyrelationset{CR}$ on metamodels $\metamodeltuple{M}$ and $\orcfunction{\transformationset{T}}$ an orchestration function.
    A transformation application function $\appfunction{\orcfunction{\transformationset{T}}}$ for them is a partial function:
    \begin{align*}
        &
        \appfunction{\orcfunction{\transformationset{T}}} : (\metamodeltupleinstanceset{M}, \changeuniverse{\metamodeltuple{M}}) \rightarrow \metamodeltupleinstanceset{M} \cup \setted{\bot}
    \end{align*}
    The function takes a consistent tuple of models and a tuple of changes that was performed on them and returns a changed tuple of models by acquiring changes from the consistency preservation rules of $\transformationset{t}$.
    Thus, it has to fulfill the following condition:
    \parameterizeformat{
    \begin{align*}
        &
        \forall \modeltuple{m} \in \metamodeltupleinstanceset{M} \mid \modeltuple{m} \consistenttomath \consistencyrelationset{CR} : \forall \changetuple{\metamodeltuple{M}} \in \changeuniverse{\metamodeltuple{M}} : \\
        & #1 #2
        \big[
            \big(\exists \modeltuple{m'} \in \metamodeltupleinstanceset{M} : 
            \appfunction{\orcfunction{\transformationset{T}}}(\modeltuple{m}, \changetuple{\metamodeltuple{M}}) = \modeltuple{m'} \big) \Rightarrow\\
            & #1 #2 \formulaskip
            \big(\exists \transformation{t}_{1}, \dots, \transformation{t}_{m} \in \transformationset{T} :
            \exists \changetuple{\metamodeltuple{M}}' \in \changeuniverse{\metamodeltuple{M}} :
            \orcfunction{\transformationset{T}}(\modeltuple{m}, \changetuple{\metamodeltuple{M}}) = \sequenced{\transformation{t}_{1}, \dots, \transformation{t}_{m}} \\
            & #1 #2 \formulaskip
            \land \generalizationfunction{\metamodeltuple{M},\transformation{t}_{m}} \concatfunction \dots \concatfunction \generalizationfunction{\metamodeltuple{M},\transformation{t}_{1}}(\modeltuple{m}, \changetuple{\metamodeltuple{M}}) = (\modeltuple{m}, \changetuple{\metamodeltuple{M}}') 
            \land \changetuple{\metamodeltuple{M}}'(\modeltuple{m}) = \modeltuple{m'}
        \big)\big]
    \end{align*}
    }{\formulaskip}{\formulaskip[0.5em]}%
\end{definition}

\mnote{Weak notion of correctness}
While the previous definition does not restrict in which cases $\bot$ and in which an actual tuple of models is returned, we define when we consider an application function \emph{correct}.
Correctness can be defined in several ways.
For example, we might say that the function is correct if it returns a consistent tuple of models whenever there is an order of transformations that leads to those consistent models.
As we will see later, this correctness notion is, however, inappropriate, because the underlying decision problem is undecidable.
In consequence, the application function needs to operate conservatively, i.e., it may return $\bot$ even if there is a sequence of transformations whose application leads to consistent models.
As an alternative, we might require the function to return consistent models whenever the orchestration function delivers a sequence of transformations whose application leads to a consistent tuple of models.
Since we have to deal with conservativeness anyway, this, however, does not provide any benefits.
In fact, the above discussed requirements encode a kind of \emph{optimality} for the functions, which we will specify more precisely in \autoref{chap:orchestration}.
For now, we stick to the simple notion of correctness that the application function does never return inconsistent models, i.e., if a tuple of models is returned, it must be consistent.

\begin{definition}[Transformation Application Function Correctness]
    \label{def:applicationfunctioncorrectness}
    Let $\appfunction{\orcfunction{\transformationset{T}}}$ be an application function for an orchestration function $\orcfunction{\transformationset{T}}$ for transformations $\transformationset{T}$.
    Let $\consistencyrelationset{CR}$ be the set of consistency relations of transformations in $\transformationset{T}$.
    We say that $\appfunction{\orcfunction{\transformationset{T}}}$ is \emph{correct} if, and only if, its result is either $\bot$ or consistent to $\consistencyrelationset{CR}{}$:
    \begin{align*}
        &
        \appfunction{\orcfunction{\transformationset{T}}} \correctmath \equivalentperdefinition
        \forall \modeltuple{m} \in \metamodeltupleinstanceset{M} \mid \modeltuple{m} \consistenttomath \consistencyrelationset{CR} : \forall \changetuple{\metamodeltuple{M}} \in \changeuniverse{\metamodeltuple{M}} : \\
        & \formulaskip
        \appfunction{\orcfunction{\transformationset{T}}}(\modeltuple{m}, \changetuple{\metamodeltuple{M}}) = \bot \lor \appfunction{\orcfunction{\transformationset{T}}}(\modeltuple{m}, \changetuple{\metamodeltuple{M}}) \consistenttomath \consistencyrelationset{CR}
    \end{align*}
\end{definition}

\mnote{Conservativeness more relevant}
This is, in fact, a rather weak notion of correctness.
An application function that always returns $\bot$ is correct according to that definition.
Because the orchestration and application function have to operate conservatively, a binary correctness notion is less relevant than a gradual one anyway.
The question how to determine such an orchestration was introduced as \autoref{rq:correctness:orchestration}.
We present and discuss a concrete approach as contribution \autoref{contrib:correctness:orchestration} in \autoref{chap:orchestration}.


\subsection{Transformation Networks}

\mnote{Transformation networks}
Based on the previous definitions of transformations, orchestration and application functions, we define what we consider a \emph{transformation network} and when we consider it \emph{correct}.
A transformation network is composed of transformations, an orchestration and an application function.
Although we define these artifacts specifically for one transformation network, i.e., an orchestration and application function according to their definitions are specific for one set of transformations, the goal will be to find an orchestration and application function that is independent from the actual transformations.

\begin{definition}[Transformation Network]
    \label{def:transformationnetwork}
    Let $\transformationset{T}$ be a transformation set, $\orcfunction{\transformationset{T}}$ an orchestration function for these transformations, and $\appfunction{\orcfunction{\transformationset{T}}}$ an application function.
    A transformation network $\transformationnetwork{N}$ is a triple:
    \begin{align*}
        \transformationnetwork{N} \equalsperdefinition \tupled{\transformationset{T}, \orcfunction{\transformationset{T}}, \appfunction{\orcfunction{\transformationset{T}}}}
    \end{align*}
\end{definition}

\mnote{Correctness of transformation networks}
Correctness of a transformation network is given by correctness of the individual transformations and the application function, according to \autoref{def:synchronizingtransformationcorrectness} and \autoref{def:applicationfunctioncorrectness}.
%We say that the transformations ensure \emph{local consistency}, because they locally achieve consistency for two models, whereas the application function achieves \emph{global consistency} by applying the transformations such that all models are consistent to all transformations.

\begin{definition}[Transformation Network Correctness]
    \label{def:transformationnetworkcorrectness}
    Let $\transformationnetwork{N} = \tupled{\transformationset{T}, \orcfunction{\transformationset{T}}, \appfunction{\orcfunction{\transformationset{T}}}}$ be a transformation network.
    We say that $\transformationnetwork{N}$ is \emph{correct} if, and only if, its transformations in $\transformationset{T}$ and the application function $\appfunction{\orcfunction{\transformationset{T}}}$ are correct:
    \begin{align*}
        & 
       \transformationnetwork{N} \correctmath \equivalentperdefinition
        \forall \transformation{t} \in \transformationset{T} : \transformation{t} \correctmath \land \appfunction{\orcfunction{\transformationset{T}}} \correctmath
    \end{align*}
\end{definition}

\mnote{Conservativeness and compatibility}
We have already discussed that we will show that the application function has to operate conservatively, which is why correctness is an essential property but not the most interesting one to achieve.
Additionally, we discussed that the consistency relations of the transformations can be considered correct by definition, but that we will discuss a notion of \emph{compatibility} to reflect when those relations contain unintended contradictions.


%%% THIS VERSION MAKES ORCHESTRATION ON ITS OWN AND DEFINES THE REQUIREMENTS FOR THAT. IT MAY REQUIRE THE RESULT TO BE CONSISTENT OR NOT
% \begin{definition}[Consistency Preservation Application Function]
%     \todo{Define for transformations instead?}
%     Let $\consistencypreservationruleset{}$ be a set of consistency preservation rules for a set of consistency relations $\consistencyrelationset{CR}$ on metamodels $\metamodeltuple{M} = \tupled{\metamodelsequence{M}{n}}$.
%     A consistency preservation application function $\consistencyappfunction{\consistencypreservationruleset{}}$ for these rules is a partial function:
%     \begin{align*}
%         &
%         \consistencyappfunction{\consistencypreservationruleset{}} : (\metamodeltupleinstanceset{M}, \changeuniverse{\metamodeltuple{M}}) \rightarrow (\metamodeltupleinstanceset{M})
%     \end{align*}
%     The function takes a consistent tuple of models and a tuple of changes that was performed on them and returns a changed tuple of models by acquiring changes from the consistency preservation rules in $\consistencypreservationruleset{}$. It is partial, because it is only defined for consistent input model tuples and may not return a result for all possible changes to any model. It has to fulfill the following conditions:
%     \begin{align*}
%         &
%         \forall \modeltuple{m} \in \metamodeltupleinstanceset{M} : \forall \changetuple{\metamodeltuple{M}} = \tupled{\change{\metamodel{M}{1}}, \dots, \change{\metamodel{M}{n}}} \in \changeuniverse{\metamodeltuple{M}} :
%         \modeltuple{m} \consistenttomath \consistencyrelationset{CR} \Rightarrow \\
%         & \formulaskip
%         \exists \modeltuple{m'} \in \metamodeltupleinstanceset{M} :
%         \modeltuple{m'} = \consistencyappfunction{\consistencypreservationruleset{}}(\modeltuple{m}, \changetuple{\metamodeltuple{M}}) \\
%         %\land \modeltuple{m'} \consistenttomath \consistencyrelationset{CR} \\
%         & \formulaskip
%         \land \exists \consistencypreservationrule{1}, \dots, \consistencypreservationrule{m} \in \consistencypreservationruleset{} : 
%         \exists \changetuple{\metamodeltuple{M}}' = \tupled{\change{\metamodel{M}{1}}', \dots, \change{\metamodel{M}{n}}'} \in \changeuniverse{\metamodeltuple{M}} :\\
%         & \formulaskip \formulaskip 
%         \cprgeneralizationfunction{\consistencypreservationrule{1}} \concatfunction \dots \concatfunction \cprgeneralizationfunction{\consistencypreservationrule{m}}(\modeltuple{m}, \changetuple{\metamodeltuple{M}}) = (\modeltuple{m}, \changetuple{\metamodeltuple{M}}')\\
%         & \formulaskip \formulaskip
%         \land \tupled{\change{\metamodel{M}{1}}'(\model{m}{1}), \dots, \change{\metamodel{M}{n}}'(\model{m}{n})} = \modeltuple{m'}
%     \end{align*}
% \end{definition}


% It is obvious that we can define consistency preservation rules for which no execution order can be specified that returns a consistent tuple of models after certain changes. Consider the example in \autoref{fig:formal:noexecutionorder}. There exists no execution order for any input value that terminates. The transformations will always increase the value, although the defined relations could be fulfilled for the input value, but the transformations never find that solution.

% \begin{figure}
%     \centering
%     \includegraphics[width=\textwidth]{figures/correctness/formal/divergence_example.png}
%     \caption{Example for divergence}
%     \label{fig:formal:noexecutionorder}
% \end{figure}

% Although we will discuss restrictions to relations and transformations that reduce the chance that no solution can be found, it will not be possible to ensure that such a solution can always be found. This is due to the reason that transformations can perform arbitrary changes given the transformations Turing-completeness, which should not be restricted, because it is unclear which restrictions could be made without forbidding scenarios that should actually we supported. Thus, we assume that transformations are Turing complete.

% Finally, this makes it necessary that a function that applies \modellevelconsistencypreservationrules may not find an execution order that returns a consistent model, thus is should be able to also return $\bot$ as an indicator for that situation.

% We first give a basic definition for such a function without further specifying in which cases the function is expected to return a result other than $\bot$.

% \begin{definition}[Consistency Preservation Application Function]
%     \todo{Define for transformations instead?}
%     Let $\consistencypreservationruleset{}$ be a set of consistency preservation rules for a set of consistency relations $\consistencyrelationset{CR}$ on metamodels $\metamodeltuple{M} = \tupled{\metamodelsequence{M}{n}}$.
%     A consistency preservation application function $\consistencyappfunction{\consistencypreservationruleset{}}$ for these rules is function:
%     \begin{align*}
%         &
%         \consistencyappfunction{\consistencypreservationruleset{}} : (\metamodeltupleinstanceset{M}, \changeuniverse{\metamodeltuple{M}}) \rightarrow (\metamodeltupleinstanceset{M})
%     \end{align*}
%     The function takes a consistent tuple of models and a tuple of changes that was performed on them and returns a changed tuple of models by acquiring changes from the consistency preservation rules in $\consistencypreservationruleset{}$. Thus, it has to fulfill the following conditions:
%     {\setlength{\mathindent}{0em}
%     \begin{align*}
%         &
%         \consistencyappfunction{\consistencypreservationruleset{}}(\modeltuple{m}, \changetuple{\metamodeltuple{M}}) = 
%         \begin{cases}
%             \modeltuple{m'}, & \begin{array}{l@{}}
%                 \exists \changetuple{\metamodeltuple{M}}' = \tupled{\change{\metamodel{M}{1}}', \dots, \change{\metamodel{M}{n}}'} \in \changeuniverse{\metamodeltuple{M}} :\\
%                 \exists \consistencypreservationrule{1}, \dots, \consistencypreservationrule{m} \in \consistencypreservationruleset{} : \\
%                 \cprgeneralizationfunction{\consistencypreservationrule{1}} \concatfunction \dots \concatfunction \cprgeneralizationfunction{\consistencypreservationrule{m}}(\modeltuple{m}, \changetuple{\metamodeltuple{M}}) = (\modeltuple{m}, \changetuple{\metamodeltuple{M}}') \\
%                 \land \tupled{\change{\metamodel{M}{1}}'(\model{m}{1}), \dots, \change{\metamodel{M}{n}}'(\model{m}{n})} = \modeltuple{m'} 
%             \end{array} \\
%             \bot, & otherwise
%         \end{cases}
%     \end{align*}
%     }
% \end{definition}

% \begin{definition}[Consistency Preservation Application Function]
%     \todo{Define for transformations instead?}
%     Let $\consistencypreservationruleset{}$ be a set of consistency preservation rules for a set of consistency relations $\consistencyrelationset{CR}$ on metamodels $\metamodeltuple{M} = \tupled{\metamodelsequence{M}{n}}$.
%     A consistency preservation application function $\consistencyappfunction{\consistencypreservationruleset{}}$ for these rules is function:
%     \begin{align*}
%         &
%         \consistencyappfunction{\consistencypreservationruleset{}} : (\metamodeltupleinstanceset{M}, \changeuniverse{\metamodeltuple{M}}) \rightarrow (\metamodeltupleinstanceset{M})
%     \end{align*}
%     The function takes a consistent tuple of models and a tuple of changes that was performed on them and returns a changed tuple of models by acquiring changes from the consistency preservation rules in $\consistencypreservationruleset{}$. Thus, it has to fulfill the following conditions:
%     {\setlength{\mathindent}{1em}
%     \begin{align*}
%         &
%         \consistencyappfunction{\consistencypreservationruleset{}}(\modeltuple{m}, \changetuple{\metamodeltuple{M}}) = 
%         \begin{cases}
%             \modeltuple{m'}, & \begin{array}{l@{}}
%                 \exists \changetuple{\metamodeltuple{M}}' = \tupled{\change{\metamodel{M}{1}}', \dots, \change{\metamodel{M}{n}}'} \in \changeuniverse{\metamodeltuple{M}} :\\
%                 \exists \consistencypreservationrule{1}, \dots, \consistencypreservationrule{m} \in \consistencypreservationruleset{} : \\
%                 \cprgeneralizationfunction{\consistencypreservationrule{1}} \concatfunction \dots \concatfunction \cprgeneralizationfunction{\consistencypreservationrule{m}}(\modeltuple{m}, \changetuple{\metamodeltuple{M}}) = (\modeltuple{m}, \changetuple{\metamodeltuple{M}}') \\
%                 \land \tupled{\change{\metamodel{M}{1}}'(\model{m}{1}), \dots, \change{\metamodel{M}{n}}'(\model{m}{n})} = \modeltuple{m'} 
%             \end{array} \\
%             \bot, & otherwise
%         \end{cases}
%     \end{align*}
%     }
% \end{definition}



% \begin{definition}[Correct Consistency Preservation Application Function]
%     Let $\consistencyappfunction{\consistencypreservationruleset{}}$ be a consistency preservation application function for a set of \modellevelconsistencypreservationrules $\consistencypreservationruleset{}$ for a set of \modellevelconsistencyrelations $\consistencyrelationset{CR}$.
%     We say that:
%     \begin{align*}
%         &
%         \consistencyappfunction{\consistencypreservationruleset{}} \mathtext{is correct} \equivalentperdefinition \\
%         & \formulaskip
%         \forall \modeltuple{m} \in \metamodeltupleinstanceset{M} : \forall \changetuple{\metamodeltuple{M}} = \tupled{\change{\metamodel{M}{1}}, \dots, \change{\metamodel{M}{n}}} \in \changeuniverse{\metamodeltuple{M}} :
%         \modeltuple{m} \consistenttomath \consistencyrelationset{CR} \Rightarrow \\
%         & \formulaskip
%         \consistencyappfunction{\consistencypreservationruleset{}}(\modeltuple{m}, \changetuple{\metamodeltuple{M}}) \consistenttomath \consistencyrelationset{CR}
%     \end{align*}
% \end{definition}




%Now it is obvious that the consistency preservation rules can actually do anything to achieve consistency, including returning always the same set of models that is consistent, although that may not be expected. We will discuss later which reasonable assumptions can be made to the behavior to on the hand not restrict the possibilities of the transformation developer and on the other hand be able to ensure some properties of the transformations and their execution.


%Two levels of correctness:
%\begin{enumerate}
    %\item Local correctness: a consistency relation is correct to the global relation and the CPR is to the relation, i.e., given two models and changes in them, the transformation can produce a change that restores consistency regarding the global consistency relation of these two models (i.e., there are some other models with which these two models would be consistent regarding the global specification) --> a network is locally correct, if this property is fulfilled
    %\item Global correctness: the binary relations together are equal to the global one and the execution function is able to find consistency models after a change to initially consistent models --> network is globally correct, if this property is fulfilled
%\end{enumerate}
%Potentiell ist lokale Korrektheit (zumindest einer CPR zu ihrer CR per Konstruktion) herstellbar -- das war auch das Ergebnis bisheriger Studien --, eventuell auch von einer CR zu einer globalen CR, obwohl die ja eigentlich meist nicht existiert, daher nehmen wir das als gegeben an.
%Dann zeigen, dass die globale Beziehung der Relationen nicht äquivalent ist zu den einzelnen lokalen, daher kommt hier zusätzliche Komplexität rein (Kompatibilitätsbegriff).
%Final muss noch die Ausführungsfunktion korrekt sein, hier aber Problem der Turing-Vollständigkeit. 
%Daher Einschränkungen an Transformationen finden bzw. ingenieurmäßige Ausführungsreihenfolge festlegen, die möglichst oft richtige Lösungen findet und sonst konservativ mit einem Fehler terminiert.


% \textbf{On top of ordinary bx correctness:}
% \begin{itemize}
    %\item Transformations need to be synchronizing
    %\item Consistency relations need to fulfill a notion of correctness
    %\item Exkurs:
    % \begin{itemize}
    % %\item Is compatibility a subclass of correctness? Is every correct set of relations compatible as well?
    % \item Problematisch: unser Konsistenzbegriff für Relationen (feingranulare Relationen) schließt keine Modelle aus, der Konsistenzbegriff hier aber schon. Wie realisiere ich die feingranularen Relationen, die dafür sorgen, dass nur genau ein Tupel von Modellen konsistent ist?
    % \item Wir müssen bei der Ableitung unseres Kompatibilitätsbegriffes erklären, dass bei uns der vollständige Ausschluss bestimmter Modelle nicht Teil einer feingranularen Konsistenzrelation sein darf, sondern Teil einer weiteren Spezifikation, die angibt, welche Modelle überhaupt valide sind. Denn so ist es in Transformationssprachen tatsächlich auch.
    % \end{itemize}
    %\item Execution function needs to be defined, which potentially induces requirements to the transformations.
% \end{itemize}


% Voraussetzungen:
% \begin{itemize}
%     \item Relationen müssen korrekt sein, d.h. sie müssen bzgl. einer globalen (meist eher implizit bekannten) n-stelligen Relation zwischen allen Modellen identisch sein. Eine n-stellige Relation lässt sich nicht immer zerlegen (siehe Stevens), aber wir nehmen das an.
%     \item Die einzelne Transformation muss bzgl. ihrer Relation korrekt sein, d.h. sie muss bei Änderungen in beiden Modellen ein zur Relation konsistentes Modell liefern.
% \end{itemize}

%Ebenen der Korrektheit:
%\begin{itemize}
    %\item Relationen müssen korrekt sein, d.h. gegeben eine Nutzeränderung muss es überhaupt möglich sein eine konsistente Menge an Modellen zu finden. Wenn Transformationen etwas beliebigen tun dürfen geht das immer. Wir nehmen an, dass eine Nutzeränderung nicht rückgängig gemacht werden soll (bzw. wenn sie rückgängig gemacht werden würde eigentlich die Änderung invalide war, d.h. keine Konsistenz im Netzwerk hergestellt werden kann). Daher sind Relationen nur korrekt, wenn für fixierte Elemente, die durch eine Nutzeränderung entstehen können, eine Modellmenge abgeleitet werden kann, die bzgl. der Relationen konsistent ist. D.h. gegeben einige Elemente muss es eine Modellmenge geben, die in allen Relationen liegt und die diese Elemente enthält (-> Kompatibilitätsbegriff). Wir betrachten in Kapitel ?, wie man Kompatibilität präzise definieren und feststellen/garantieren kann.\\
    %Resultat: Gegeben eine Änderung ist es möglich eine Transformation anzugeben, die aus der Änderung ein konsistentes Modell produziert.
    %\item Einzelne Transformationen müssen korrekt sein: Wir fordern Korrektheit der Transformation sowieso. Allerdings machen in einem Netzwerk verschiedene Transformationen Änderungen an allen Modellen, d.h. wir müssen nicht den "normalen" Transformationsfall unterstützen, dass Deltas in einem Modell ins andere übertragen werden, um Konsistenz herzustellen, sondern die Transformationen müssen \emph{synchronisierend} sein, also Deltas in beiden Modelle annehmen und dann Konsistenz herstellen. Wir definieren diese Synchronisationseigenschaft und betrachten in Kapitel ?, welcher zusätzlichen Anforderungen sich dadurch bzgl. EMOF-Modellen ergeben. Der Input sind Deltas in zwei Modellen, und einzelne Deltas sind potentiell als "authoritative" definiert, was bedeutet, dass die erzeugten/geänderten Elemente nicht noch einmal geändert/gelöscht werden dürfen. Das realisiert die Anforderung, dass Nutzeränderungen nicht rückgängig gemacht werden dürfen. \\
    %Resultat: Gegeben Änderungen in zwei Modellen (mit potentiell authoritativen Änderungen) gibt die Transformation ein konsistentes (bzgl. der Konsistenzrelation) Modellpaar zurück. 
    %\item Korrektheit der Anwendungsfunktion: Die Anwendungsfunktion muss die Transformationen in einer 
%\end{itemize}


% \todo{Überlegen, wo hier die Definition von (undirektionalen Relationen) rein muss.}
% Präzisere Eigenschaften:
% \begin{itemize} 
%     \item Synchronisationseigenschaft: Eine Transformation kann mit Änderungen an mehreren Modellen umgehen, d.h. gegeben zwei konsistente Modelle + Änderungen an beiden resultiert in zwei Modellen, die konsistent bzgl. der Relation(en) zwischen den Metamodellen sind
%     \item 
% \end{itemize}  

% \begin{itemize}
%     \item Kompatibilität entsprechend Modularisierungsebene
%     \item Synchronisation auf Operationalisierung-Ebene: Abwägen, dass eine Transformation verschiedene Zustände sehen könnte, auf denen sie ausgeführt wird. Aber letztendlich muss sie damit klarkommen, dass zwei Modelle geändert wurden. 
% \end{itemize}

%TODO:
%\begin{itemize}
    %\item Authoritative Modelle (bzw. eher authoritative Regionen) diskutieren (Verweis Stevens)
%\end{itemize}



% \section{Local Correctness}

% Simple solution: we define a transformation which normatively implies a relation, thus it is correct by construction. From a theoretical perspective this is easy to reach, from a practical it is not.
% However, in contrast to our definition of synchronizing transformations, ordinary transformations are only able to process changes in one model and update the other accordingly. Together with the assumption that both models were consistent before does not fit with our scenario, because if one model is modified, the other may be modified as well by another transformation across another path, before a transformation is executed. Thus, both models may have been modified.
% We consider the following situation: Models A and B were consistent. Model A was changed an we have the changes at hand. Additionally, B was modified because there were other changes propagated through the network. 
% We distinguish all cases of modifications to B that may have violated a consistency relation between A and B (according to our fine-grained consistency notion) and consider what we have to do there (e.g., find-or-create-pattern).
% Put empirical analysis here.


% \section{Correct APP function}

% We make the following approach: Always assume there is a solution and start executing the transformation (for now in any order). Finally, the network has to terminate at a fixed point. We investigate, what the reasons may be that it does not try to avoid them.

% These reasons can lie in the relations:
% - relations cannot be completely unfulfillable, as the empty models are always consistent, thus there can always be CPRs that result in a consistent set of models
% - however, if relations contain pairs that can never be in any consistent model tuple they improve proneness to errors, because a CPR may return that pair, which will never fit to any result of any other transformation. Thus, this should not be allowed -> compatibility

% These reasons can also lie in the transformations:
% - Transformations can make choices and they make choices that are always incompatible to other (refer to example)

% Essentially there are two problems: alternation and divergence


% \subsection{Other thought}
% If each element occurs in each relation only once (so always 1:1 mappings) and if we have compatibility, then any transformation order would return exactly the one model tuple that fits.
% However: In that case we would have confluence, every information must directly be available in B from A without a transitive propagation over C. This is not what we want. So there must in general be more than one option a transformation is fine with that to reflect the information that another transformation may add or change.


%\todo{Hippocraticness is not necessary but needs to be discussed}

% Goal:
% - Find a solution in as much cases as possible, abort in the others (conservatively)
% - To do so: reduce cases in which there is no such function
% - To do so: ensure that relations are defined in a way such that they do not allow a locally correct set of CPRs that has no APP solution. If there is a pair of models (or elements of a fine-grained relation) in a relation, a CPR may return it. But if there is no consistent tuple of models containing these two, it does not make any sense to consider these elements (even worse, if we have monotony, adding these elements makes the network unsolvable). For that reason, we need compatibility.

\section{A Fine-Grained Notion of Consistency}
\label{chap:correctness:finegrained}

\mnote{Fine-grained understanding of consistency}
We have up to now given a common definition of consistency~\cite{stevens2010sosym} by enumerating consistent pairs of models in a relation.
That notion is sufficient for defining transformation networks, correctness of their artifacts, and also the essential considerations regarding orchestration, as presented in the preceding section.
Domain experts and transformation developers, however, usually think in terms of a more fine-grained notion of consistency.
They do not consider when complete models are consistent, but when specific relations between some of their elements are fulfilled, i.e., which other elements they require to exist if some elements are present in models.
For example, they consider consistency between architectural components and object-oriented classes instead of complete models containing these elements.

\mnote{Representation in transformation languages}
This is also reflected by transformation languages, such as \gls{QVTR}.
First, they require relations to be defined at the level of classes and their properties. They define how properties of some classes are related to properties of other classes.
Second, they are defined in an \emph{intensional} way, i.e., constraints specify which elements are consistent rather than enumerating all consistent instances in an \emph{extensional} specification.
We have already discussed that intensional and extensional specifications have equal expressiveness and can be transformed into each other, which is why we stick to extensional specifications for reasons of simplicity.
However, we reuse the concept of specifying relations at the level of classes and their properties.

\mnote{Benefits of fine-grained notion}
This reflects a natural understanding of consistency and, in particular, makes it easier to make statements about dependencies between consistency relations, which we need to make statements about compatibility of consistency relations.
Thus, we introduce an appropriate, fine-grained notion of consistency relations in the following.
Finally, from such a fine-grained specification, a \modellevelconsistencyrelation can always be derived by enumerating all models that fulfill all the fine-grained specifications, thus it does not restrict expressiveness in any way and can be seen as a \emph{compositional approach} for defining consistency, which is only a refinement of the notion of \modellevelconsistencyrelations.
We have presented the following definitions of a fine-grained consistency notion, partly literally, in previous work~\owncite{klare2020compatibility-report}. 
The definitions are based on those proposed in the work of \textcite[Sec.~2.3.2, 4.1.1]{kramer2017a} and \textowncite{klare2021Vitruv-JSS}.


\subsection{Fine-Grained Consistency Relations}
\label{chap:correctness:finegrained:relations}

\mnote{Conditions}
The central idea of the fine-grained consistency notion is to have consistency relations that contain pairs of objects and, broadly speaking, requires that if the objects in one side of the pair occur in a model, the others have to occur in another model as well.
A \emph{condition} encapsulates such objects, for which we require objects in another model to occur.

\begin{definition}[Condition]
    A condition $\condition{c}{}$ for a class tuple $\classtuple{C}{\condition{c}{}} = \tupled{\class{C}{\condition{c}{},1}, \dots, \class{C}{\condition{c}{},n}}$ is a set of object tuples with: 
    \begin{align*}
    &
    \forall \tupled{\object{o}{1}, \dots, \object{o}{n}} \in \condition{c}{}: \forall i \in \setted{1, \dots, n} : \object{o}{i} \in \instances{\class{C}{\condition{c}, i}}
    \end{align*}
    An element $\conditionelement{c}{} \in \condition{c}{}$ is called a \emph{condition element}.
    %
    For a model tuple $\modeltuple{m} \in \metamodeltupleinstanceset{M}$ of a metamodel tuple $\metamodeltuple{M}$ and a condition element $\conditionelement{c}{}$, we say that: 
    \begin{align*}
        &
        \modeltuple{m} \containsmath \conditionelement{c}{} \equivalentperdefinition
        \exists \model{m}{} \in \modeltuple{m} : \conditionelement{c}{} \subseteq \model{m}{}
    \end{align*}
\end{definition}

\mnote{Models containing conditions}
\emph{Conditions} represent object tuples, called \emph{condition elements}, that instantiate the same tuple of classes. 
They are supposed to occur in models that fulfill a certain condition regarding consistency and thus require elements in other models to exist, as subsequently defined by consistency relations.
We say that a tuple of models contains a condition element if any of the models contains all the objects within the condition element.
This implies that such a model's metamodel has to contain all the classes in the class tuple of the condition.
We use conditions to define consistency relations as the co-occurrence of condition elements.

\begin{definition}[Consistency Relation]
\label{def:consistencyrelation}
    Let $\classtuple{C}{l,\consistencyrelation{CR}{}}$ and $\classtuple{C}{r,\consistencyrelation{CR}{}}$ be two class tuples.
    A consistency relation $\consistencyrelation{CR}{}$ is a subset of pairs of condition elements in conditions $\condition{c}{l,\consistencyrelation{CR}{}}, \condition{c}{r,\consistencyrelation{CR}{}}$ with
    $\classtuple{C}{l,\consistencyrelation{CR}{}} = \classtuple{C}{\condition{c}{l,\consistencyrelation{CR}{}}}$ and $\classtuple{C}{r,\consistencyrelation{CR}{}} = \classtuple{C}{\condition{c}{r,\consistencyrelation{CR}{}}}$ :
    \begin{align*}
        & 
        \consistencyrelation{CR}{} \subseteq \condition{c}{l,\consistencyrelation{CR}{}} \times \condition{c}{r,\consistencyrelation{CR}{}}
    \end{align*}
    We call a pair of condition elements $\tupled{\conditionelement{c}{l}, \conditionelement{c}{r}} \in \consistencyrelation{CR}{}$ a \emph{consistency relation pair}. 
    For a model tuple $\modeltuple{m}$ and a consistency relation pair $\tupled{\conditionelement{c}{l}, \conditionelement{c}{r}}$, we say that:
    \begin{align*}
        & 
        \modeltuple{m} \containsmath \tupled{\conditionelement{c}{l}, \conditionelement{c}{r}} \equivalentperdefinition \modeltuple{m} \containsmath \conditionelement{c}{l} \land \modeltuple{m} \containsmath \conditionelement{c}{r}
    \end{align*}
\end{definition}

\mnote{Consistency relations}
A consistency relation is a set of pairs of condition elements, which indicate the tuples of objects that are considered consistent with each other. 
This means that if a model contains one of the left condition elements that occurs in the relation, another model must contain one of the related right condition elements.
It bases on two conditions that define relevant object tuples in instances of each of the two metamodels and defines the ones that are related to each other.
Without loss of generality, we assume that each condition element of both conditions occurs in at least one consistency relation pair:
\parameterizeformat{
\begin{align*}
    & 
    \forall \conditionelement{c}{} \in \condition{c}{l,\consistencyrelation{CR}{}} : \exists \tupled{\conditionelement{c}{l}, \conditionelement{c}{r}} \in \consistencyrelation{CR}{} : \conditionelement{c}{} = \conditionelement{c}{l} 
    \\ &
    \land \forall \conditionelement{c}{} \in \condition{c}{r,\consistencyrelation{CR}{}} : \exists \tupled{\conditionelement{c}{l}, \conditionelement{c}{r}} \in \consistencyrelation{CR}{} : \conditionelement{c}{} = \conditionelement{c}{r}
\end{align*}
}{}{\\ &}%
Based on these consistency relations, we can define a fine-grained notion of consistency.

\begin{definition}[Consistency] \label{def:consistency}
    Let $\consistencyrelation{CR}{}$ be a consistency relation and let $\modeltuple{m} \in \metamodeltupleinstanceset{M}$ be a tuple of models of the metamodels in $\metamodeltuple{M}$.
    We say that:
     \begin{align*}
        & 
        \modeltuple{m} \consistenttomath \consistencyrelation{CR}{} \equivalentperdefinition \\
        & \formulaskip
        \exists \consistencyrelation{W}{} \subseteq \consistencyrelation{CR}{} : 
        \big[
        \forall \tupled{\conditionelement{c}{l,1}, \conditionelement{c}{r,1}}, \tupled{\conditionelement{c}{l,2}, \conditionelement{c}{r,2}} \in \consistencyrelation{W}{} : \\
        & \formulaskip\formulaskip\formulaskip
        \big( \tupled{\conditionelement{c}{l,1}, \conditionelement{c}{r,1}} = \tupled{\conditionelement{c}{l,2}, \conditionelement{c}{r,2}} \lor 
        ( \conditionelement{c}{l,1} \neq \conditionelement{c}{l,2} \land \conditionelement{c}{r,1} \neq \conditionelement{c}{r,2}) \big) \\
        & \formulaskip\formulaskip
        \land \forall \tupled{\conditionelement{c}{l}, \conditionelement{c}{r}} \in \consistencyrelation{W}{} : \big( \modeltuple{m} \containsmath \conditionelement{c}{l} \land \modeltuple{m} \containsmath \conditionelement{c}{r} \big) \\
        & \formulaskip\formulaskip
        \land \forall \conditionelement{c}{l}' \in \condition{c}{l,\consistencyrelation{CR}{}} : \big( \modeltuple{m} \containsmath \conditionelement{c}{l}' \Rightarrow \conditionelement{c}{l}' \in \condition{c}{l,\consistencyrelation{W}{}} \big)
        \big]
    \end{align*}
    We call such a $\consistencyrelation{W}{}$ a \emph{witness structure} for consistency of $\modeltuple{m}$ to $\consistencyrelation{CR}{}$, and for all pairs $\tupled{\conditionelement{w}{l}, \conditionelement{w}{r}} \in \consistencyrelation{W}{}$, we call $\conditionelement{w}{l}$ and $\conditionelement{w}{r}$ \emph{corresponding to} each other.
    
    For a set of consistency relations $\consistencyrelationset{CR} = \setted{\consistencyrelation{CR}{1}, \consistencyrelation{CR}{2}, \dots}$, we say that:
    \begin{align*}
        \formulaskip &
        \modeltuple{m} \consistenttomath \consistencyrelationset{CR} \equivalentperdefinition
        \forall \consistencyrelation{CR}{} \in \consistencyrelationset{CR} : \modeltuple{m} \consistenttomath \consistencyrelation{CR}{}
    \end{align*}
\end{definition}

\mnote{Consistency by fine-grained relations}
A consistency relation $\consistencyrelation{CR}{}$ relates one condition element at the left side to one or more other condition elements at the right side of the relation.
The definition of consistency ensures that if one condition element $\conditionelement{c}{} \in \condition{c}{l,\consistencyrelation{CR}{}}$ at the left side of the relation occurs in a tuple of models, exactly one of the condition elements related to it by a consistency relation $\consistencyrelation{CR}{}$ occurs in another model to consider the tuple of models consistent.
If another element that is related to $\conditionelement{c}{}$ occurs in the models, this one has to be, in turn, related to another condition element $\conditionelement{c}{}' \in \condition{c}{l,\consistencyrelation{CR}{}}$ of the left side of condition elements by $\consistencyrelation{CR}{}$ that also occurs in the models.
This ensures that a condition element contained in a model uniquely corresponds to another element to which it is considered consistent according to $\consistencyrelation{CR}{}$.

\begin{figure}
    \centering
    \newcommand{\hdistance}{(23em+0.5*\difftoafiveimage)}
\newcommand{\classwidth}{6em}

\begin{tikzpicture}

% Employee
\umlclassvarwidth{employee}{}{Employee\sameheight}{
name
}{\classwidth}

% Resident
\umlclassvarwidth[, right=\hdistance of employee.north, anchor=north]{resident}{}{Resident\sameheight}{
name
}{\classwidth}

% CONSISTENCY RELATIONS
\draw[directed consistency relation] (employee.east) -- node[pos=0, above right] {$e$} node[pos=0.5, below, align=center] {
$\consistencyrelation{CR}{} = \setted{ \tupled{e,r} \mid \mathvariable{e.name.toLower} = \mathvariable{r.name}}$} node[pos=1, above left] {$r$} (resident.west);

\end{tikzpicture}
    %\includegraphics[width=0.8\textwidth]{figures/correctness/notion/witness_uniqueness.png}
    \caption[Example for necessity of a witness structure]{A consistency relation derived from \autoref{fig:networks:three_persons_example}, which depicts the necessity of a witness structure to ensure that only one employee out of those with differently capitalized names is allowed to correspond to a resident with the same name.}
    \label{fig:correctness:witness_uniqueness}
\end{figure}

\mnote{Witness structure}
Consider the exemplary consistency relation in \autoref{fig:correctness:witness_uniqueness}, which is derived from the one in our running example in \autoref{fig:networks:three_persons_example}.
The relation requires for each resident an employee with an appropriate name to exist and vice versa.
It assumes that resident names are stored lowercase and allows the employee name to be written in arbitrary capitalization.
Thus, for example, both the employees with names \enquote{Alice} and \enquote{alice} would be considered consistent to a resident with name \enquote{alice}.
Without the restriction defined by the auxiliary witness structure $\consistencyrelation{W}{}$, an employee model containing the employees with both capitalizations would be considered consistent to a resident model containing a corresponding resident with the same name written in lowercase.
The witness structure, however, ensures that for each employee one corresponding resident exists, thus there can only exist one employee with one of the allowed capitalizations, as each of them is corresponding to the resident with the lowercase name.
In general, the witness structure restriction ensures that if several alternatives for a corresponding element exists, only one is actually allowed to be present.

\begin{figure}
    \centering
    \newcommand{\hdistance}{(20em+0.5*\difftoafiveimage)}
\newcommand{\vdistance}{1em}
\newcommand{\internalvdistance}{0.3em}
\newcommand{\classwidth}{6em}
\newcommand{\objectwidth}{6.7em}
\newcommand{\leftshift}{5em}
\renewcommand{\sameheight}{\vphantom{dy}}

\begin{tikzpicture}[
    witness/.style={correspondence},
    witness fault/.style={witness, color=darkred, dashed}
]

% Employee
\umlclassvarwidth{employee}{}{Employee\sameheight}{
name
}{\classwidth}

% Resident
\umlclassvarwidth[, right=\hdistance of employee.north, anchor=north]{resident}{}{Resident\sameheight}{
name
}{\classwidth}

% CONSISTENCY RELATIONS
\draw[directed consistency relation] ([yshift=0.5em]employee.east) -- node[pos=0, above right] {$e$} node[pos=0.5, below, align=center] {
$\begin{aligned}
    \consistencyrelation{CR}{} = \setted{ \tupled{e,r} \mid \mathvariable{e.name} = \mathvariable{r.name} \\[-0.4em]
	\lor \mathvariable{e.name.toLower} = \mathvariable{r.name} }
\end{aligned}$} node[pos=1, above left] {$r$} ([yshift=0.5em]resident.west);

% EXAMPLE 1
% Employee
\umlobjectvarwidth[, below right=1.8*\vdistance and \leftshift of employee.south west, anchor=north west]{instance1_employee}{}{Employee\sameheight}{
	name = "Alice"
}{\objectwidth}
% Resident
\umlobjectvarwidth[, below=1.8*\vdistance of resident.south east, anchor=north east]{instance1_resident}{}{Resident\sameheight}{
	name = "Alice"
}{\objectwidth}

% EXAMPLE 2
% Employee
\umlobjectvarwidth[, below=\vdistance of instance1_employee.south, anchor=north]{instance2_employee}{}{Employee\sameheight}{
	name = "Alice"
}{\objectwidth}
% Resident
\umlobjectvarwidth[, below=\vdistance of instance1_resident.south, anchor=north]{instance2_resident}{}{Resident\sameheight}{
	name = "alice"
}{\objectwidth}


% EXAMPLE 3
% Employee
\umlobjectvarwidth[, below=\vdistance of instance2_employee.south, anchor=north]{instance3_employee1}{}{Employee\sameheight}{
	name = "alice"
}{\objectwidth}
\umlobjectvarwidth[, below=\internalvdistance of instance3_employee1.south, anchor=north]{instance3_employee2}{}{Employee\sameheight}{
	name = "Alice"
}{\objectwidth}
% Resident
\umlobjectvarwidth[, below=\vdistance of instance2_resident.south, anchor=north]{instance3_resident1}{}{Resident\sameheight}{
	name = "alice"
}{\objectwidth}
\umlobjectvarwidth[, below=\internalvdistance of instance3_resident1.south, anchor=north]{instance3_resident2}{}{Resident\sameheight}{
	name = "Alice"
}{\objectwidth}

% EXAMPLE 4
% Employee
\umlobjectvarwidth[, below=\vdistance of instance3_employee2.south, anchor=north]{instance4_employee1}{}{Employee\sameheight}{
	name = "Alice"
}{\objectwidth}
% Resident
\umlobjectvarwidth[, below=\vdistance of instance3_resident2.south, anchor=north]{instance4_resident1}{}{Resident\sameheight}{
	name = "Alice"
}{\objectwidth}
\umlobjectvarwidth[, below=\internalvdistance of instance4_resident1.south, anchor=north]{instance4_resident2}{}{Resident\sameheight}{
	name = "Bob"
}{\objectwidth}

% EXAMPLE 5
% Resident
\umlobjectvarwidth[, below=\vdistance of instance4_resident2.south, anchor=north]{instance5_resident1}{}{Resident\sameheight}{
	name = "Alice"
}{\objectwidth}
\umlobjectvarwidth[, below=\internalvdistance of instance5_resident1.south, anchor=north]{instance5_resident2}{}{Resident\sameheight}{
	name = "alice"
}{\objectwidth}
% Employee
\umlobjectvarwidth[, left=\hdistance-\leftshift+(\classwidth-\objectwidth) of instance5_resident1.north, anchor=north]{instance5_employee1}{}{Employee\sameheight}{
	name = "Alice"
}{\objectwidth}

% EXAMPLE 6
% Resident
\umlobjectvarwidth[, below=\vdistance of instance5_resident2.south, anchor=north]{instance6_resident1}{}{Resident\sameheight}{
	name = "alice"
}{\objectwidth}
% Employee
\umlobjectvarwidth[, left=\hdistance-\leftshift+(\classwidth-\objectwidth) of instance6_resident1.north, anchor=north]{instance6_employee1}{}{Employee\sameheight}{
	name = "alice"
}{\objectwidth}
\umlobjectvarwidth[, below=\internalvdistance of instance6_employee1.south, anchor=north]{instance6_employee2}{}{Employee\sameheight}{
	name = "Alice"
}{\objectwidth}

\node[left=0.5*\leftshift of instance1_employee.west] {\textbf{1.}};
\node[left=0.5*\leftshift of instance2_employee.west] {\textbf{2.}};
\node[left=0.5*\leftshift of instance3_employee1.west] {\textbf{3.}};
\node[left=0.5*\leftshift of instance4_employee1.west] {\textbf{4.}};
\node[left=0.5*\leftshift of instance5_employee1.west] {\textbf{5.}};
\node[left=0.5*\leftshift of instance6_employee1.west] {\textbf{6.}};

\draw[gray] ([xshift=-\leftshift,yshift=0.5*\vdistance]instance1_employee.north west) -- ([yshift=0.5*\vdistance]instance1_resident.north east);
\draw[gray] ([xshift=-\leftshift,yshift=0.5*\vdistance]instance2_employee.north west) -- ([yshift=0.5*\vdistance]instance2_resident.north east);	 
\draw[gray] ([xshift=-\leftshift,yshift=0.5*\vdistance]instance3_employee1.north west) -- ([yshift=0.5*\vdistance]instance3_resident1.north east);	 
\draw[gray] ([xshift=-\leftshift,yshift=0.5*\vdistance]instance4_employee1.north west) -- ([yshift=0.5*\vdistance]instance4_resident1.north east);	 
\draw[gray] ([xshift=-\leftshift,yshift=0.5*\vdistance]instance5_employee1.north west) -- ([yshift=0.5*\vdistance]instance5_resident1.north east);
\draw[gray] ([xshift=-\leftshift,yshift=0.5*\vdistance]instance6_employee1.north west) -- ([yshift=0.5*\vdistance]instance6_resident1.north east);	


% WITNESS STRUCTURE
\draw[witness] (instance1_employee) -- (instance1_resident);
\draw[witness] (instance2_employee) -- (instance2_resident);
\draw[witness] (instance3_employee1) -- (instance3_resident1);
\draw[witness] (instance3_employee2) -- (instance3_resident2);
\draw[witness] (instance4_employee1) -- (instance4_resident1);
\draw[witness fault] (instance5_employee1) -- (instance5_resident1);
\draw[witness fault] (instance5_employee1) -- (instance5_resident2);
\draw[witness fault] (instance6_employee1) -- (instance6_resident1);
\draw[witness fault] (instance6_employee2) -- (instance6_resident1);

\end{tikzpicture}
    \caption[Examples for fine-grained consistency relations]{A consistency relation between employee and resident and six example model pairs: pairs 1--4 consistent with an appropriate witness structure $\consistencyrelation{W}{}$ shown in blue, solid lines, and pairs 5 and 6 inconsistent with an inappropriate mapping structure shown in red, dashed lines. Adapted from~\owncite[Fig.~2]{klare2020compatibility-report}.}
    \label{fig:correctness:consistency_example}
\end{figure}

\begin{example}
The definition of consistency is exemplified in \autoref{fig:correctness:consistency_example}, which is an alternation of an extract of \autoref{fig:networks:three_persons_example} only considering employees and residents. Models with employees and residents are considered consistent if for each employee exactly one resident with the same name or the name in lowercase exists.
The model pairs $1$--$3$ are obviously consistent according to the definition, because there is always a pair of objects that fulfills the consistency relation.
In model pair $4$, there is a consistent resident for each employee, but there is no appropriate employee for the resident with $\mathvariable{name} = "\mathvariable{Bob}"$. However, our definition of consistency only requires that for each condition element at the left side of the relation that appears in the models, an appropriate right element occurs, but not vice versa. Thus, a relation is interpreted unidirectionally, which we subsequently discuss in more detail.
In model pair $5$, there are two residents with names in different capitalizations, which would both be considered consistent to the employee according to the consistency relation.
Comparably, in model pair $6$, there is a resident that fulfills the consistency relations for both employees, each having a different but matching capitalization. 
However, the consistency definition requires that each model element for which consistency is defined by a consistency relation must only have one corresponding element. 
In this case, there are two residents or employees that could be considered consistent to the employee or resident, respectively, thus there is no witness structure with a unique mapping between the elements as required by the definition.
\end{example}

\mnote{Unidirectional notion}
As mentioned in the example, the definition considers consistency in a unidirectional way, which means that a consistency relation may define that some elements $\conditionelement{c}{r}$ are required to occur in a tuple of models if some elements $\conditionelement{c}{l}$ occur, but not vice versa.
Such a unidirectional notion can also be reasonable in our example, as it could make sense to require a resident for each employee, but not every resident might be employed and thus also represent an employee.
To achieve a bijective consistency definition, for each consistency relation $\consistencyrelation{CR}{}$ its transposed relation $\consistencyrelation{CR}{}^T = \setted{\tupled{\conditionelement{c}{l}, \conditionelement{c}{r}} \mid \tupled{\conditionelement{c}{r}, \conditionelement{c}{l}} \in \consistencyrelation{CR}{}}$ can be considered as well.
Regarding \autoref{fig:correctness:consistency_example}, if we consider the relation between employees and residents as well as its transposed, the model pair $4$ would also be considered inconsistent, because an appropriate employee for each resident is required by the transposed relation.
We call sets of consistency relations that contain only bijective definitions of consistency \emph{symmetric}.

\begin{definition}[Symmetric Consistency Relation Set]
    Let $\consistencyrelationset{CR}$ be a set of consistency relations.
    We say that $\consistencyrelationset{CR}$ is \emph{symmetric} if, and only if, for each contained relation its transposed one is also contained:
    \begin{align*}
        &
        \consistencyrelationset{CR} \mathtextspacearound{is symmetric} \equivalentperdefinition
        \forall \consistencyrelation{CR}{} \in \consistencyrelationset{CR} :
        \consistencyrelation{CR}{}^T \in \consistencyrelationset{CR}
    \end{align*}
\end{definition}

\mnote{Reasons for unidirectionality}
Any description of bijective consistency relations can be defined with a symmetric consistency relation set.
We have defined consistency in a unidirectional way for the two following reasons.
\begin{longenumerate}
    \item Some relevant consistency relations are actually not bijective. 
    Apart from the simple example concerning residents and employees, this situation always occurs when objects at different levels of abstraction are related.
    Consider a relation between components and classes, requiring for each component an implementation class but not vice versa, or a relation between \gls{UML} models and object-oriented code, requiring for each \gls{UML} class an appropriate class in code but not vice versa.
    These relations could not be expressed if consistency relations were always considered bijective.
    \item We consider networks of consistency relations, in which a combination of multiple bijective consistency relations does not necessarily imply a bijective consistency relation again. 
    Thus, we need a unidirectional notion of consistency relations anyway.
\end{longenumerate}

\mnote{Explicit trace models}
One might argue that consistency is usually traced by means of a \emph{trace model}, which stores the pairs of element tuples in models that fulfill a consistency relation.
A trace model can be seen as an explicit representation of a witness structure as specified in \autoref{def:consistency}.
We do, however, not explicitly consider such an explicit trace model in this formalism for two reasons also discussed in previous work~\owncite{klare2021Vitruv-JSS}.
First, a trace model is only necessary in practice if no identifying information for related elements is present, or if performance is to be improved.
However, we assume such identifying information to exist without loss of generality, as introduced in \autoref{chap:networks:models:assumption}.
Second, a trace model can, from a theoretical perspective, be treated as a usual model by defining consistency between one concrete and one trace model. This conforms to the fact that each multiary relation can be expressed by binary relations to an additional model (in this case the trace model), as discussed in existing work~\cite{stevens2020BidirectionalTransformationLarge-SoSym, cleve2019dagstuhl}.
We discuss practical benefits of having an explicit trace model for consistency preservation in \autoref{chap:synchronization} to distinguish modifications of elements from their removal and addition.
But this does, as discussed, not restrict applicability of our formalism.


\subsection{Expressiveness of Fine-Grained Relations} % Equivalence to \modellevelconsistencyrelations

\mnote{Expressiveness of fine-grained consistency}
The model-level consistency notion of \autoref{def:modellevelconsistency} is established and based on notions used by several researchers.
The fine-grained consistency notion according to \autoref{def:consistency} is based on the insight that practical approaches to describe consistency and its preservation use fine-grained rules rather than enumerating consistent model pairs.
We did, however, only provide examples that justify specific decisions in the definitions, such as the witness structure for corresponding elements, but we did not argue if and why fine-grained relations are an actual refinement, such that statements about \modellevelconsistencyrelations also apply to fine-grained relations.

\mnote{Implication of model-level relation}
To show that every set of fine-grained consistency relations can be expressed by a single \modellevelconsistencyrelation, we can use the same constructive approach that we have used to define consistency according to multiple consistency relations, be they at the model level or fine-grained.
Given fine-grained consistency relations $\consistencyrelationset{CR} = \setted{\consistencyrelation{CR}{1}, \dots, \consistencyrelation{CR}{k}}$, we can construct an equivalent \modellevelconsistencyrelation $\consistencyrelation{CR}{}$ as follows:
\begin{align*}
    \consistencyrelation{CR}{} = \setted{\modeltuple{m} \mid \modeltuple{m} \consistenttomath \consistencyrelationset{CR}}
\end{align*}

\mnote{Model-level relations more expressive}
A \modellevelconsistencyrelation can, however, not necessarily be expressed by fine-grained consistency relations.
The most simple construction approach would define a single fine-grained consistency relation to express a \modellevelconsistencyrelation, which contains the complete models instead of extracts of them.
The definition of consistency is, however, different for the two types of relations.
While at the model level consistency is defined as two (or more) models being in a relation (see \autoref{def:modellevelconsistency}), fine-grained consistency relations do only describe that if an element at the left side of the relation occurs in a model, then any of the related elements at the right side has to occur in another.
If two models are considered consistent by a \modellevelconsistencyrelation, they are also consistent to the accordingly constructed fine-grained relation, because there is a witness structure that contains exactly the two consistent models.
If there is a model that is not considered consistent to any other model in the \modellevelconsistencyrelation, thus the \modellevelconsistencyrelation does not contain any pair with that model, then there will also be no such pair in the fine-grained relation.
According to \autoref{def:consistency} of consistency for fine-grained relations, if there is no condition element in the relation, then consistency is not constrained for the contained model elements.
In consequence, such a model would be considered consistent to every other model.

\mnote{Additional semantics in consistency relations}
While, at first, this may seem inappropriate, it actually is appropriate for two reasons.
First, the formalism can only express that for some elements other elements need to exist, but not that specific elements are not allowed to exist if other elements exist.
This is reasonable, because consistency between models is supposed to ensure that the overlap of information is represented uniformly, thus to express that information in one model needs to be represented in another one as well.
Expressing that some elements are not allowed to exist because of others, e.g., being an employee in one model, the same person cannot be a student in another model, is actually not a consistency constraint for information shared between models.
This is actually additional information that should be stored in a specific model representing these semantics.
Thus, we do not consider this case at all.

\mnote{Restriction of valid models}
Second, the formalism for fine-grained consistency relations can not prevent specific elements from existing at all.
For example, a consistency relation may define that for a component in an architecture model a corresponding class in the object-oriented design model has to exist, but it may not restrict that only components of specific names are allowed.
Such restrictions should and actually are separate specifications not related to consistency between models but restricting a model on its own.
Thus, the metamodel or some additional specification for it should provide such restrictions of valid models, which we have discussed as a restriction of $\metamodelinstanceset{M}{}$ for a metamodel $\metamodel{M}{}$ in \autoref{chap:networks:models}.

\mnote{Insight transferability between notions}
Summarizing, we found that we can express each set of fine-grained consistency relations by a \modellevelconsistencyrelation.
Additionally, we know that there are specific kinds of restrictions that can be encoded in \modellevelconsistencyrelations but not in fine-grained consistency relations.
We have, however, discussed why they are not relevant for the designated application area of consistency preservation.
In consequence, all insights made for \modellevelconsistencyrelations can also be applied to fine-grained consistency relations and, if specific restrictions are excluded, vice versa.

%\begin{itemize}
    %\item Is compatibility a subclass of correctness? Is every correct set of relations compatible as well?
    %\item Problematisch: unser Konsistenzbegriff für Relationen (feingranulare Relationen) schließt keine Modelle aus, der Konsistenzbegriff hier aber schon. Wie realisiere ich die feingranularen Relationen, die dafür sorgen, dass nur genau ein Tupel von Modellen konsistent ist?
    %\item Wir müssen bei der Ableitung unseres Kompatibilitätsbegriffes erklären, dass bei uns der vollständige Ausschluss bestimmter Modelle nicht Teil einer feingranularen Konsistenzrelation sein darf, sondern Teil einer weiteren Spezifikation, die angibt, welche Modelle überhaupt valide sind. Denn so ist es in Transformationssprachen tatsächlich auch.
%\end{itemize}


\subsection{Application to Consistency Preservation Rules}
\label{chap:correctness:finegrained:rules}

\mnote{Fine-grained notion in transformation languages}
As mentioned before, the fine-grained notion of consistency fits well to how transformation languages consider consistency.
They allow to define rules that relate only some classes by relations, conforming to fine-grained consistency relations, from which fine-grained consistency preservation rules are derived.
Alternatively, they directly allow to define rules to preserve consistency between specific classes.
These rules are often called \emph{transformation rules} and composed to a transformation that consists of multiple such rules, each encoding a consistency relation and a preservation rule.

\mnote{Conflicts between transformation rules}
It may easily happen that the execution of one transformation rule leads to the violation of the consistency relation of another, which induces dependencies between the individual transformation rules.
Thus, a combination of transformation rules to a transformation has to ensure correctness, i.e., that the consecutive execution of the rules leads to a consistent state of the models.
Languages such as \gls{QVTR} and \gls{QVTO} therefore specify that transformation rules may not be conflicting~\cite[Sec.~7.10.2.]{qvt}.
It is also a dedicated topic of research to ensure that the rules of a single transformation conform to each other, e.g.~\cite{cuadrado2017tse,cabot2010VerificationInvariants-JSS}, which is why we assume that transformations fulfill that property.

\mnote{Extension of Consistency preservation rules}
To avoid the necessity of specifying this conformance property for transformation rules, we stick to the existing notion of coarse-grained consistency preservation rules, as it is sufficient for our considerations.
Still, consistency preservation rules were defined for \modellevelconsistencyrelations in \autoref{def:consistencypreservationrule}.
This can, however, be easily extended to fine-grained consistency relations, as we simply need to require the rule to consider consistency to a set of fine-grained relations according to \autoref{def:consistency} rather than consistency to a single \modellevelconsistencyrelation according to \autoref{def:modellevelconsistency}.

\begin{figure}
    \centering
    \newcommand{\hdistance}{12em}
\newcommand{\vdistance}{8em}

\begin{tikzpicture}[
    model style/.style={color=darkgray},
    concept/.style={draw, minimum width=7em, inner sep=0.5em},
    model concept/.style={model style, concept},
    metamodel concept/.style={concept},
    model relation/.style={model style},
    metamodel relation/.style={},
]

\node[model concept] (model) {Model};
\node[model concept, right=\hdistance of model.north, anchor=north] (view) {View};
\node[model concept, right=\hdistance of view.north, anchor=north] (vsum) {V-SUM};

\node[metamodel concept, below=\vdistance of model.north, anchor=north] (metamodel) {Metamodel};
\node[metamodel concept, right=\hdistance of metamodel.north, anchor=north] (viewtype) {View Type};
\node[metamodel concept, right=\hdistance of viewtype.north, anchor=north] (vsumm) {V-SUM Metamodel};

\node[metamodel concept, below=\vdistance of metamodel.north, anchor=north] (cr) {Consistency Rule};
\node[metamodel concept, right=\hdistance of cr.north, anchor=north] (cpr) {Consistency Preservation Rule};
\node[metamodel concept, right=\hdistance of cpr.north, anchor=north] (app) {Application Function};

\draw[model relation, -latex] (view) -- node[uml cardinality start, very near start, above right] {*} node[uml association name, below] {is projection of} node[uml cardinality end, above left] {1..*} (model);
\draw[model relation, -latex] (view) -- node[uml cardinality start, very near start, above left] {*} node[uml association name, below] {projected from} node[uml cardinality end, very near end, above right] {1} (vsum);
\draw[model relation, open diamond-latex] (vsum) -- node[uml cardinality start, right] {1} ++(0,3em) -| node[uml association name, pos=0.25, below] {consists of} node[uml cardinality end, very near end, left] {*} (model);

\draw[model relation, dashed, -latex] (model) -- node[uml cardinality start, above left] {*} node[uml association name, right] {\guillemotleft instance of\guillemotright} node[uml cardinality end, below left] {1} (metamodel);
\draw[model relation, dashed, -latex] (view) -- node[uml cardinality start, above left] {*} node[uml association name, right] {\guillemotleft instance of\guillemotright} node[uml cardinality end, below left] {1} (viewtype);
\draw[model relation, dashed, -latex] (vsum) -- node[uml cardinality start, above left] {*} node[uml association name, right] {\guillemotleft instance of\guillemotright} node[uml cardinality end, below left] {1} (vsumm);

\draw[metamodel relation, open diamond-latex] ([xshift=1em]viewtype.north west) -- node[uml cardinality start, pos=0.25, right] {*} ++(0,2em) -| node[uml association name, pos=0.25, below] {has} node[uml cardinality end, pos=0.75, left] {1} ([xshift=-1em]metamodel.north east);
\draw[metamodel relation, -latex] (viewtype) -- node[uml cardinality start, very near start, above right] {*} node[uml association name, below] {projects from} node[uml cardinality end, above left] {1..*} (metamodel);
\draw[metamodel relation, -latex] (viewtype) -- node[uml cardinality start, very near start, above left] {*} node[uml association name, below] {used for} node[uml cardinality end, above right] {1..*} (vsumm);

\draw[metamodel relation, open diamond-latex] (cr) -- node[uml cardinality start, below left] {*} node[uml association name, pos=0.4, right, align=left] {describes\\ consistency\\ between} node[uml cardinality end, above left] {2} (metamodel);
\draw[metamodel relation, open diamond-latex] (cpr) -- node[uml cardinality start, above right] {*} node[uml association name, below] {preserves} node[uml cardinality end, very near end, above left] {1} (cr);
\draw[metamodel relation, open diamond-latex] (app) -- node[uml association name, below] {orchestrates} node[uml cardinality end, very near end, above left] {*} (cpr);

\draw[metamodel relation, open diamond-latex] ([xshift=-4em]vsumm.south) -- node[uml cardinality start, pos=0.5, left] {*} ++(0,-2em) -| node[uml association name, pos=0.25, below] {consists of} node[uml cardinality end, pos=0.75, left] {*} ([xshift=-1em]metamodel.south east);
\draw[metamodel relation, open diamond-latex] ([xshift=-2em]vsumm.south) -- node[uml cardinality start, pos=0.25, left] {*} ++(0,-4em) -| node[uml association name, pos=0.25, below] {consists of} node[uml cardinality end, pos=0.75, below left] {*} (cpr.north);
\draw[metamodel relation, open diamond-latex] (vsumm) -- node[uml cardinality start, above left] {*} node[uml association name, right] {consists of} node[uml cardinality end, below left] {1} (app);

\end{tikzpicture}

    %\includegraphics[width=0.85\textwidth]{figures/correctness/notion/conceptual_model.png}
    \caption[Conceptual model for transformation networks]{A conceptual model for the terms and artifacts introduced for transformation networks and their relations. Adapted from~\owncite[Fig.~5]{klare2021Vitruv-JSS}.}
    \label{fig:correctness:conceptual_model}
\end{figure}

\mnote{Consistency preservation for fine-grained relations}
A consistency preservation rule $\consistencypreservationrule{\consistencyrelationset{CR}}$ for a set of consistency relations $\consistencyrelationset{CR}$ according to \autoref{def:consistencyrelation} is thus still considered correct if it only returns changes when they yield models that are consistent to all consistency relations if applied to the input models, in accordance with \autoref{def:consistencypreservationrulecorrectness}:
\begin{align*}
    &
    \forall \model{m}{1} \in \metamodelinstanceset{M}{1}, \model{m}{2} \in \metamodelinstanceset{M}{2}, \change{\metamodel{M}{1}} \in \changeuniverse{\metamodel{M}{1}}, \change{\metamodel{M}{2}} \in \changeuniverse{\metamodel{M}{2}} : \\
    & \formulaskip
    \forall \change{\metamodel{M}{1}}' \in \changeuniverse{\metamodel{M}{1}}, \change{\metamodel{M}{2}}' \in \changeuniverse{\metamodel{M}{2}} : 
    \big(
        \consistencypreservationrule{\consistencyrelationset{CR}}(\model{m}{1}, \model{m}{2}, \change{\metamodel{M}{1}}, \change{\metamodel{M}{2}}) = \tupled{\change{\metamodel{M}{1}}', \change{\metamodel{M}{2}}'}\\
        & \formulaskip\formulaskip
        \Rightarrow \tupled{\change{\metamodel{M}{1}}'(\model{m}{1}),\change{\metamodel{M}{2}}'(\model{m}{2})} \consistenttomath \consistencyrelationset{CR} 
    \big)
\end{align*}
Note that being consistent to all fine-grained consistency relations is equivalent to being consistent to the \modellevelconsistencyrelation induced by the fine-grained relations.

\mnote{Transformations for fine-grained relations}
Likewise, we consider a synchronizing transformation according to \autoref{def:synchronizingtransformation} as a pair of fine-grained consistency relations and a consistency preservation rule for them, thus $\transformation{t} = \tupled{\consistencyrelationset{CR}, \consistencypreservationrule{\consistencyrelationset{CR}}}$.
Again, in conformance with \autoref{def:synchronizingtransformationcorrectness}, we call such a transformation $\transformation{t}$ correct if, and only if, its consistency preservation rule is correct.

\section{Summary}
\label{chap:correctness:summary}

\mnote{Insight}
In this chapter, we have discussed notions of correctness for transformation networks and the artifacts they consist of, and we have precisely defined the notion that is relevant for the context of this thesis.
We give an overview of the introduced concepts and their relations in the conceptual model in \autoref{fig:correctness:conceptual_model}.
In summary, we provided the following insight in this chapter.

\begin{insight}[Correctness Notion]
    A reasonable notion of correctness for networks of modular, independently developed transformations consists of correctness of the single transformations, which need to be synchronizing, and correctness of the application function that determines an execution order of the transformations.
    An application function may not be able to return a result for different reasons, such as transformations not being applicable to specific changes, the absence of an execution order of the transformations that leads to consistent models, or the inability to find such an order.
    Thus, in comparison to correctness, the degree of conservativeness is the more important property of an application function, which indicates how often the function does not deliver a result although there is an order of transformations that would restore consistency.
    Additionally, although theoretically not relevant for correctness, the relations defining when models are considered consistent must fulfill some notion of compatibility to be useful, as they can otherwise prevent transformations from finding consistent models.
\end{insight}

\mnote{Achieving correct transformation networks}
In the following chapters, we thus define a notion of compatibility for consistency relations, discuss how correctness of the individual synchronizing transformations for achieving local consistency can be achieved, and finally how a correct and appropriate application function to perform the orchestration for achieving global consistency can be defined.
In summary, these following contributions together allow to develop what we defined as a \emph{correct} transformation network.

For visualizing examples of consistency relations, consistency preservation rules, and their execution throughout the next chapters, we use a notation according to the example depicted in \autoref{fig:correctness:visualization_example}.
We visualize consistency relations with blue arrows and a definition of the conditions for consistency relation pairs forming that relation.
In the example, the consistency relation contains all pairs of employees and residents having the same name, except for those with an empty name.
The arrows of such a relation indicate whether we only consider a directional consistency relation or also its transposed one.
We depict consistency preservation rules with orange arrows and denote which changes it produces because of which input change.
In the example, we denote that the addition of an employee ($+e$) leads to the addition of a resident with the same name, specified by the according property assignment $r(\mathvariable{name} = \mathvariable{e.name})$.
In addition, we annotate conditions to the consistency preservation rules, such as $\mathvariable{e.name} \neq ""$ in the example, which restricts the resident creation to the case in which the employee name is not empty.
We usually specify only parts of a consistency preservation rule if the other cases are not relevant in the specific context.
In the example, we only specify the behavior for the case of adding an element but not of modifying or removing it.
Finally, we denote the execution of any changes, including consistency preservation rules, with green arrows.
In the example, we visualize the addition of an employee by a user, denoted with a \enquote{$+$}, which leads to the addition of a resident because of the execution of the above introduced consistency preservation rule.

\begin{figure}
    \centering
    \newcommand{\hdistance}{22em}
\newcommand{\classwidth}{5em}

\begin{tikzpicture}

% METAMODEL
\umlclassvarwidth{employee}{}{Employee\sameheight}{
    name
}{\classwidth}
\umlclassvarwidth[, right=\hdistance of employee.north, anchor=north]{resident}{}{Resident\sameheight}{
    name
}{\classwidth}

% RELATIONS AND TRANSFORMATIONS
\draw[consistency relation]
    ([yshift=1em]employee.east)
    --
    node[pos=0, above right] {$e$}
    node[above, align=center] {$\consistencyrelation{CR}{} = \{ \tupled{e,r} \mid$ \\ $\mathvariable{e.name} = \mathvariable{r.name} \land \mathvariable{e.name} \neq ""\}$}
    node[pos=1, above left] {$r$}
    ([yshift=1em]resident.west);
\draw[transformation]
    ([yshift=-1em]employee.east)
    --
    node[pos=0, above right] {$e$}
    node[above] {$\consistencypreservationrule{}$}
    node[below, align=center] {$+e \xrightarrow{\mathvariable{e.name} \; \neq \; ""} +r (\mathvariable{name} = \mathvariable{e.name})$}
    node[pos=1, above left] {$r$}
    ([yshift=-1em]resident.west);

% SEPARATION
\draw[lightgray] ([yshift=-2.5em]employee.south west) -- ([yshift=-2.5em]resident.south east);

% MODELS
\umlhuman{person}{}{below=0.3*\hdistance of employee.center, anchor=center}{}{0.5}
\umlobjectvarwidth[, below right=0.3*\hdistance and 0.45*\hdistance of employee.north, anchor=north]{employee_instance}{}{Employee\sameheight}{
    name = "Alice"
}{\classwidth}
\umlobjectvarwidth[, right=0.55*\hdistance of employee_instance.north, anchor=north]{resident_instance}{}{Resident\sameheight}{
    name = "Alice"
}{\classwidth}

% EXECUTIONS
\draw[consistency execution, -latex]
    (person.east|-employee_instance.west)
    --
    node[above] {+}
    (employee_instance.west);
\draw[consistency execution, -latex]
    (employee_instance.east)
    --
    node[above] {+}
    node[below] {$\consistencypreservationrule{}$}
    (resident_instance.west);

\end{tikzpicture}
    %\includegraphics[width=0.8\textwidth]{figures/correctness/notion/visualization_example.png}
    \caption[Example for concept visualizations]{Example for the visualization of consistency relations, consistency preservation rules, and the execution of changes by users or consistency preservation.}
    \label{fig:correctness:visualization_example}
\end{figure}

% \section{Summary}

% Central Insights:
% \begin{itemize}
%     \item In networks, we need compatible consistency relations -> first RQ
%     \item In networks, we need synchronizing rather than bidirectional transformations -> second RQ
%     \item In networks, we need orchestration functions -> third RQ
%     \item Correctness is not the problem, optimality is the problem
%     \item We can only check dynamically whether a consistent state was reached due to Halting Problem. We cannot guarantee to always find a consistent state
% \end{itemize}

