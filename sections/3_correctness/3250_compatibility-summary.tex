\section{Summary}

In this chapter, we have discussed the challenge regarding compatibility of consistency relations, which are encoded in transformations.
We have derived and precisely defined a notion of compatibility and presented a formal approach that is able to validate it for given relations.
The approach is proven to be correct.
Based on the formal approach, we have developed a practical approach that validates compatibility of relations defined with \gls{QVTR} and \gls{OCL}.
We conclude this chapter with the following central insight.

\begin{insight}[Compatibility]
    Transformations that are supposed to preserve contradictory consistency relations easily lead to problems when combining them to a network, because their relations cannot be fulfilled at the same time.
    The relations preserved by transformations should thus be \emph{compatible}, i.e., they should not restrict consistency for elements such that no consistent set of models can be found by the transformation network.
    That notion of compatibility can be proven for given transformations by considering their preserved consistency relations, finding redundant relations, and removing them until only a tree of relations remains. Since we were able to prove that consistency relation trees are inherently compatible and removing redundant relations is compatibility-preserving, this approach is proven correct.
    Compatibility is a property of the network and not a single transformation, thus it cannot be achieved by construction of the individual transformations, but it can only be analyzed for a given transformation network.
\end{insight}
