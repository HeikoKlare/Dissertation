\section{A Practical Approach to Prove Compatibility}
\label{chap:compatibility:practical_approach}

%Goal: Applies formal framework for extensional specification to actual languages with intensional specification (MA Aurélien)

\begin{copiedFrom}{SoSym MPM4CPS}

%\section{Decomposing Transformations}
%\label{sec:decomposition}

The formal approach adopted in this article demonstrates that deriving a consistency relation tree from a set of consistency relations $\consistencyrelationset{CR}$ is an effective way to prove compatibility. It is especially a consequence of \autoref{theorem:treecompatibility}. 
Given that proving compatibility amounts to the construction of a consistency relation tree, this result lends itself well to an operationalization. To this end, we propose an algorithm that turns the proof of compatibility into an operational procedure. %This constitutes our contribution \ref{contrib:operationalizedapproach}.
For the most part, this algorithm is based on results previously developed and described in detail in a master's thesis~\cite{pepin2019ma} and, as introduced before, is mostly taken from the article~\owncite{klare2020compatibility-report} describing those results.

Constructing a consistency relation tree can be achieved by finding and virtually removing every redundant consistency relation in $\consistencyrelationset{CR}$. Such a result facilitates the development of software systems. First, developers build transformations independently, resulting in a transformation network. Then,~they regularly run a procedure that assesses the compatibility of consistency relations specified in transformations by checking the existence of a consistency relation tree.

Removing redundant relations in a consistency relation set to generate a tree, or a set of independent trees, is called \textit{decomposition}. Designing a decomposition procedure requires to represent consistency relations in actual model transformation languages and to provide a way to test the redundancy of a consistency relation. We first highlight a mapping between the consistency framework developed in this article and the QVT-R transformation language through the use of \textit{predicates}. As a consequence, results achieved with consistency relations become also applicable with QVT-R. Then, we design a fully automated decomposition procedure that takes a \emph{consistency specification}, i.e., a set of QVT-R transformations, as an input and virtually removes as many redundant consistency relations as possible. In the decomposition procedure, each consistency relation removal is a two-step process. First, a potentially redundant relation and an alternative concatenation of consistency relations are identified. Then, a redundancy test is performed: it answers whether it is possible or not to remove the candidate relation using the alternative concatenation. Uncoupling the search for candidates from the decision-making makes it possible to plug in different strategies to test redundancy. This section focuses on the first step, i.e., setting up a structure suited to the detection of possibly redundant relations and finding candidates for the redundancy test.

%% 5.1.
\subsection{Consistency Relations in Transformation Languages}

According to \autoref{def:consistencyrelation}, consistency relations are built by enumerating valid co-occurring condition elements. However, developers do not enumerate valid models when writing transformations. %It is even sometimes impossible, for example when there are infinitely many consistent metamodel instances. 
They rather describe patterns for models to be considered consistent and sometimes how consistency is restored after a model was modified. %the transformation should be performed when a non-consistent model needs to be updated. 
In relational transformation languages, developers define consistency as a set of criteria that models must fulfill. Criteria are expressed using metamodel elements (i.e., class properties), as objects are only distinguished by their contents. For example, an \texttt{Employee} object and a \texttt{Person} object are considered consistent if their \texttt{name} attributes are equal.

Criteria are equivalent to predicates, i.e., Boolean-valued filter functions: consistency relations are then defined as sets of pairs of condition elements for which the predicate evaluates to \textsc{true}. Therefore, we move from an extensional to an intensional, programming-like definition of consistency relations, making it easier to link consistency relations with QVT-R transformations.
In \autoref{chap:correctness:notions_consistency:intensional_extensional}, we have shown that both types of specification provide equal expressiveness.

\subsubsection{Properties, Property Values and Predicates}

We first define concepts that allow the intensional construction of consistency relations. The main idea is to select some properties in each metamodel and to define a predicate that filters values of these properties.

\begin{definition}[Property Set]
A property set for a class $\class{C}{}$ is a subset $\propertyset{P}{\class{C}{}}$ of properties of $\class{C}{}$, i.e., $\propertyset{P}{\class{C}{}} = \{P_{\class{C}{}, 1}, \dots, P_{\class{C}{}, n}\}$ such that $P_{\class{C}{}, i} \in \class{C}{}$.
\end{definition}

A property set models a choice of properties that play a role in the definition
of a predicate in order to distinguish consistent and non-consistent condition elements. Not all properties have to be used to describe consistency, particularly in incremental model transformations. 

\begin{definition}[Tuple of Property Sets]
For a class tuple $\classtuple{C}{}$, it is possible to build a tuple of property sets by defining a property set for every class, i.e., $\propertysettuple{P}{\classtuple{C}{}} = \tupled{\propertyset{P}{\class{C}{1}}, \dots, \propertyset{P}{\class{C}{n}}} = \tupled{\{P_{\class{C}{1}, 1}, \dots, P_{\class{C}{1}, m}\}, \dots, \{P_{\class{C}{n}, 1}, \dots, P_{\class{C}{n}, k}\}}$.
\end{definition}

Tuples generalize the use of property sets to class tuples, because conditions themselves are made up of class tuples.

\begin{definition}[Property Value Set]
A property value set $\propertyvalueset{p}{\class{C}{}}$ for a property set $\propertyset{P}{\class{C}{}}$ is a set in which each property in $\propertyset{P}{\class{C}{}}$ is instantiated, i.e., $\propertyvalueset{p}{\class{C}{}} = \{p_{\class{C}{}, 1}, \dots, p_{\class{C}{}, n}\}$ with $p_{\class{C}{}, i} \in I_{P_{\class{C}{}, i}}$. Similarly, a tuple of property value sets can be built from a tuple of property sets by instantiating each property set in it.
\end{definition}

Just as a property set is a subset of properties of a class $\class{C}{}$, a property value set is a subset of property values of an object $\object{o}{}$ that instantiates $\class{C}{}$. The property value set is a fragment of $\object{o}{}$ that provides enough information to evaluate consistency.

\begin{definition}[Predicate]
A predicate for two class tuples $\classtuple{C}{l}$ and $\classtuple{C}{l}$ is a triple $\pi = (\propertysettuple{P}{\classtuple{C}{l}}, \propertysettuple{P}{\classtuple{C}{r}}, f_\pi)$ where $\propertysettuple{P}{\classtuple{C}{l}}$ (resp. $\propertysettuple{P}{\classtuple{C}{r}}$) is a tuple of property sets of $\classtuple{C}{l}$ (resp. $\classtuple{C}{r}$) and $f_\pi$ is a Boolean-valued function that takes instances of $\tuple{P}{\classtuple{C}{l}}$ and $\tuple{P}{\classtuple{C}{r}}$ as an input, i.e., $f_\pi \colon I_{\propertysettuple{P}{\classtuple{C}{l}}} \times I_{\propertysettuple{P}{\classtuple{C}{r}}} \to \{\textsc{true}, \textsc{false}\}$.
\end{definition}

For readability purposes, it is sometimes useful to group all the properties used by a predicate within the same set. As a consequence, the property collection $\propertycollection{\pi}$ of a predicate $\pi = (\propertysettuple{P}{\classtuple{C}{l}}, \propertysettuple{P}{\classtuple{C}{r}}, f_\pi)$ is defined as:
\begin{align*}
    \formulaskip &
    \propertycollection{\pi} = (\bigcup_{j}\ \tuple{P}{\classtuple{C}{l}, j})\ \cup\ (\bigcup_{k}\ \tuple{P}{\classtuple{C}{r}, k})  
\end{align*}

%The predicate function $f_\pi$ takes two tuples of tuples property value sets as an input and returns \textsc{true} if values of the left tuple match values of the right tuple according to some arbitrary criteria.
The definition of a predicate involves the choice of some properties for each class that occurs in one of the two class tuples of a consistency relation $CR$. It also involves the definition of an appropriate function $f_\pi$ that answers whether instances of these properties, i.e., property values, evaluate to \textsc{true} or \textsc{false}. In the former case, objects containing these property values match the predicate: the associated consistency relation pair is in $CR$. In the latter case, objects do not match the predicate and are not considered consistent, i.e., do not occur in $CR$. The expression of $f_\pi$ is the choice of the developer who defines it according to consistency criteria.

Predicates model the way consistency relations are defined in %usable 
model transformation languages. Objects can only be distinguished by their property values. Thus, the distinction between consistent and non-consistent pairs of condition elements is always based on some attribute or reference values.

\subsubsection{Predicate-Based Consistency Relations}
\label{sec:predicatebasedconsistencyrelations}

\begin{definition}[Property Matching]
A property value set $\propertyvalueset{p}{\class{C}{}} = \{p_{\class{C}{}, 1}, \dots, p_{\class{C}{}, n}\}$ matches an object $\object{o}{}$ if and only if
\begin{align*}
    \formulaskip &
    \object{o}{} \in I_{\class{C}{}} \wedge \forall p_{\class{C}{}, i} : p_{\class{C}{}, i} \in \object{o}{}
\end{align*}
%
Similarly, a tuple of property value sets $\propertyvaluesettuple{p}{\classtuple{C}{}} = \tupled{\propertyvalueset{p}{\class{C}{1}}, \dots, \propertyvalueset{p}{\class{C}{n}}}$ matches a tuple of objects $\tuple{o}{} = \tupled{\object{o}{1}, \dots, \object{o}{k}}$ if and only if $|\tuple{p}{\classtuple{C}{}}| = |\tuple{o}{}|$ and $\forall i : \propertyvalueset{p}{\class{C}{i}}\ \text{matches}\ \object{o}{i}$.
\end{definition}

% Let $P_{\classtuple{C}{\condition{c}{l}}}$ and $P_{\classtuple{C}{\condition{c}{r}}}$ be tuples of property sets for $\classtuple{C}{\condition{c}{l}}$ and $\classtuple{C}{\condition{c}{r}}$.
\begin{definition}[Predicate-Based Consistency Relation]
Let $\condition{c}{l}$ and $\condition{c}{r}$ be two conditions for two class tuples $\classtuple{C}{\condition{c}{l}}$ and $\classtuple{C}{\condition{c}{r}}$. 
Let $\Pi$ be a set of predicates for $\classtuple{C}{\condition{c}{l}}$ and $\classtuple{C}{\condition{c}{r}}$. A $\Pi$-based consistency relation $CR_{\Pi}$ is a subset of pairs of condition elements such that:
% $$CR_{\pi} = \{(c_l, c_r) \mid c_l \in \condition{c}{l} \wedge c_r \in \condition{c}{r} \wedge \exists TPVSl, TPVSr : \}$$
% \begin{align*}
% \formulaskip &
% CR_{\pi} = \{(c_l, c_r) \mid \exists \tupled{p_{l, i}}, \tupled{p_{r, i}} : \tupled{p_{l, i}}\  \text{matches}\ c_l \\
% & \formulaskip
% \wedge \tupled{p_{r, i}}\ \text{matches}\ c_r \\
% & \formulaskip 
% \wedge f_{\pi}(p_l, p_r) = \textsc{true}\}
% \end{align*}
\begin{align*}
\formulaskip &
CR_{\Pi} = \{(c_l, c_r) \mid \forall (\tuple{P}{\classtuple{C}{\condition{c}{l}}}, \tuple{P}{\classtuple{C}{\condition{c}{r}}}, f_\pi) \in \Pi : \\
& \formulaskip\formulaskip
\exists \tuple{p}{\classtuple{C}{\condition{c}{l}}} \in \tuple{P}{\classtuple{C}{\condition{c}{l}}},
\tuple{p}{\classtuple{C}{\condition{c}{r}}} \in \tuple{P}{\classtuple{C}{\condition{c}{r}}}:\\
& \formulaskip
\propertyvaluesettuple{p}{\classtuple{C}{\condition{c}{l}}}\  \text{matches}\ c_l \\
& \formulaskip
\wedge \propertyvaluesettuple{p}{\classtuple{C}{\condition{c}{r}}}\ \text{matches}\ c_r \\
& \formulaskip 
\wedge f_{\pi}(p_{\classtuple{C}{\condition{c}{l}}}, p_{\classtuple{C}{\condition{c}{r}}}) = \textsc{true}\}
\end{align*}
\end{definition}

%% ONE OR MORE PREDICATES?
%% Is the conjunction of predicates a predicate itself?

The construction of a predicate-based consistency relation is of importance for the practicality of model transformation languages. The developer can produce a consistency specification by retaining some object properties and imposing conditions on values of these properties via a predicate function. Then, the construction of the consistency relation fully amounts to the evaluation of the predicate function.

\begin{example}
The following example demonstrates how to build a consistency relation $\consistencyrelation{CR}{PR}$ based on predicates between \texttt{Person} and \texttt{Resident} metamodels, according to the example in \autoref{fig:compatibility:three_persons_example_extended}. $\consistencyrelation{CR}{PR}$ ensures that the name of a \texttt{Resident} object is the concatenation of the first name and the last name of a \texttt{Person} object. It also ensures that both objects have the same address. First, $\consistencyrelation{CR}{PR}$ involves one class in
each metamodel, resulting in two class tuples: $\classtuple{C}{P} = \tupled{Person}$ and $\classtuple{C}{R} = \tupled{Resident}$. There are two conditions to achieve consistency, which are equal names and equal addresses, so $\consistencyrelation{CR}{PR}$ will be made up of two predicates. The first predicate needs the \texttt{firstname} and \texttt{lastname} attributes in \texttt{Person} and the \texttt{name} in \texttt{Resident}, so $\propertysettuple{P}{\classtuple{C}{P}, 1} = \tupled{\{firstname, lastname\}}$ and $\propertysettuple{P}{\classtuple{C}{R}, 1} = \tupled{\{name\}}$. Similarly, $\propertysettuple{P}{\classtuple{C}{P}, 2} = \tupled{\{address\}}$ and $\propertysettuple{P}{\classtuple{C}{R}, 2} = \tupled{\{address\}}$.
The functions of the predicate, shortly denoting \texttt{name} as $n$, \texttt{firstname} as $fn$, \texttt{lastname} as $ln$, as well as \texttt{address} of \texttt{Person} as $a_P$ and of \texttt{Resident} as $a_R$, look as follows:
\begin{align*}
   \formulaskip &
   f_{\pi, 1}(\tupled{\{n\}}, \tupled{\{fn, ln\}}) = \begin{cases} 
      \textsc{true} & \text{if}\ n = fn + `\ ` + ln \\
      \textsc{false} & \text{otherwise}
   \end{cases} \\
   &
   f_{\pi, 2}(\tupled{\{a_P\}}, \tupled{\{a_R\}}) = \begin{cases} 
      \textsc{true} & \text{if}\ a_P = a_R \\
      \textsc{false} & \text{otherwise}
   \end{cases}
\end{align*}

$\consistencyrelation{CR}{PR}$ is a $\Pi$-based consistency relation where $\Pi$ is the set of predicates $\{(\propertysettuple{P}{\classtuple{C}{P}, 1}, \propertysettuple{P}{\classtuple{C}{R}, 1}, f_{\pi, 1}), (\propertysettuple{P}{\classtuple{C}{P}, 2}, \propertysettuple{P}{\classtuple{C}{R}, 2}, f_{\pi, 2})\}$.
\end{example}

\subsubsection{Consistency Relations in QVT-R Transformations}

There is a variety of model transformation languages~\cite{czarnecki2003a}. Like programming languages, transformation languages can be divided into two main paradigms: \textit{declarative} languages that focus on \textit{what} transformations should perform and \textit{imperative} languages that describe \textit{how} transformations should be performed. The QVT standard~\cite{qvt} %OMG's \textit{Query/View/Transformation} (QVT) standard
provides three transformation languages: Operational, Core and Relations.
% QVT is a 

\begin{figure}
\begin{embeddedqvtcode}[frame=bt, numbers=none, mathescape=true, caption={Simplified structure of a QVT-R transformation. Taken from \owncite{klare2020compatibility-report}.},label={qvt:structure},captionpos=b]
import $M_1$ : 'path_m1.ecore';
import $M_2$ : 'path_m2.ecore';

transformation T($M_1$, $M_2$) {
    [top] relation $R_1$ {
        [variable declarations]
        domain M a : A { $\pi_{M}$ }
        domain N b : B { $\pi_{N}$ }
        [when { PRECOND }] [where { INVARIANT }]
    }
    
    [top] relation $R_2$ { ... }
}
\end{embeddedqvtcode}
\end{figure}

The most relevant language for consistency specification is QVT Relations (QVT-R). It is a declarative and relational language that shares many concepts with the consistency framework developed in this article. It lends itself well to mathematical formalization~\cite{stevens2010sosym}. As with consistency relations, QVT-R supports bidirectionality. Transformations written in QVT-R can be used with two execution modes. First, a \textit{checkonly} mode to check that models fulfill consistency relations. Second, an \textit{enforce} mode to repair consistency in a given direction if not all relations are fulfilled. The simplified structure of a QVT-R transformation is as follows and also depicted in \autoref{qvt:structure}. 

A QVT-R \texttt{transformation} can check or repair consistency of models it receives as parameters. Models are typed models, i.e., their structure conforms to a type defined by the metamodel. Each \texttt{transformation} is composed of \texttt{relation}s, which define the rules for objects of both models to be consistent. Relations are only invoked if they are prefixed by the \texttt{top} keyword, if they belong to the precondition (\texttt{when}) of a relation to be invoked, or if they belong to the invariant (\texttt{where}) of a relation already invoked. The QVT-R mechanism for checking consistency is based on pattern matching. Shared information between objects of different models is represented by variables assigned to class properties. These variables contain values that must remain consistent from one object to another. Therefore, there must exist some assignment that matches all patterns at the same time for classes in a relation to be consistent. 

More precisely, each QVT-R \texttt{relation} contains two \texttt{domain}s that contain themselves \textit{domain patterns}. In QVT terminology, a domain pattern is a variable instantiating a class. Values that this variable can take are constrained by conditions on its properties. These conditions, known as \textit{property template items} (PTIs), are OCL constraints~\cite{ocl}. OCL operations provide the ability to describe more complex constraints than equalities between property values and variables. In \autoref{qvt:domainpatterns}, each domain has one pattern. These patterns filter \texttt{Person} objects (with three PTIs) and \texttt{Employee} objects (with two PTIs), respectively. For two objects to be consistent, there must exist values of \texttt{fstn}, \texttt{lstn} and \texttt{inc} that match property values of these objects, thus ensuring the fact that the name of the employee equals the concatenation of the first name and the last name of the person and the fact that both instances have the same income. If objects are inconsistent, e.g., if the person and the employee have different incomes, then there is no such variable assignment.

%% USE DIRECTLY ONE OF THE THREE RELATIONS?
\begin{figure}
\begin{embeddedqvtcode}[frame=bt, numbers=none, mathescape=true, caption={Two domains, each with one domain pattern. Taken from \owncite{klare2020compatibility-report}.},label={qvt:domainpatterns},captionpos=b]
fstn: String; lstn: String;
inc: Integer;

domain pers p:Person {
    firstname=fstn, lastname=lstn, 
    income=inc
};

domain emp e:Employee {
    name=fstn + ' ' + lstn,
    salary=inc
};
\end{embeddedqvtcode}
\end{figure}
QVT-R relations are defined intensionally. In \textit{checkonly} mode, a \texttt{relation} does not check that metamodel instances are consistent by looking for them in an existing set of pairs of consistent models. It rather evaluates the existence of a value that fulfills all property template items in domain patterns. These patterns can be regarded as predicates. Thus, it is relevant to correlate QVT-R relations and predicate-based consistency relations. One relation in QVT-R can be translated into one or more predicates. The main idea is to extract properties that are bound to the same QVT-R variables: having QVT-R variables in common means that values of these properties are interrelated. Properties are separated to build two tuples of property sets, one for each metamodel. Then, a predicate function is generated by extracting OCL constraints. The triplet that groups these objects together is a predicate. A formal construction of predicates from QVT-R
will be presented in the subsequent section.

As a result, QVT-R is a language suitable for writing consistency relations according to our formalism. The decomposition procedure presented in this section treats a set of (binary) QVT-R \texttt{transformation}s as a consistency relation set and checks its compatibility.
The \qvtr transformations for the example in \autoref{fig:compatibility:three_persons_example_extended} are depicted in \autoref{lst:correctness:prevention:running_example_qvtr}.

%% Three consistency relations
\definecolor{Gray}{gray}{0.96}
\newcolumntype{a}{>{\columncolor{Gray}}p{0.30\textwidth}}
    
\begin{figure*}
    \centering
    \begin{tabular}{a|a|a}
        \begin{embeddedqvtcode}[basicstyle=\scriptsize\ttfamily, frame=none, numbers=none, mathescape=true, linewidth=0.30\textwidth, breaklines=true]
import personMM   : 'personmm.ecore';
import employeeMM : 'employeemm.ecore';

transformation PersonEmployee(
    person: personMM,
    employee: employeeMM) {
    
	top relation PE {
		fstn: String;
		lstn: String;
		inc: Integer;
		
		domain person p:Person {
		    firstname=fstn,
		    lastname=lstn,
		    income=inc};
		domain employee e:Employee {
		    name=fstn + ' ' + lstn,
		    salary=inc};
	}
}
        \end{embeddedqvtcode}&
        \begin{embeddedqvtcode}[basicstyle=\scriptsize\ttfamily, frame=none, numbers=none, mathescape=true, linewidth=0.30\textwidth, breaklines=true]
import personMM : 'personmm.ecore';
import residentMM : 'residentmm.ecore';

transformation PersonResident(
    person: personMM,
    resident: residentMM) {
    
	top relation PR {
		fstn: String;
		lstn: String;
		addr: String;
		
		domain person p:Person {
		    firstname=fstn,
		    lastname=lstn,
		    address=addr};
		domain resident r:Resident {
		    name=fstn + ' ' + lstn,
		    address=addr};
	}
}
        \end{embeddedqvtcode}&
        \begin{embeddedqvtcode}[basicstyle=\scriptsize\ttfamily,frame=none, numbers=none, mathescape=true, linewidth=0.30\textwidth, breaklines=true]
import employeeMM : 'employeemm.ecore';
import residentMM : 'residentmm.ecore';

transformation EmployeeResident(
    employee: employeeMM,
    resident: residentMM) {
    
	top relation ER {
		n: String;
		ssn: Integer;
		
		domain employee e:Employee {
		    name=n,
		    socsecnumber=ssn};
		domain resident r:Resident {
		    name=n,
		    socsecnumber=ssn};
	}
}
        \end{embeddedqvtcode}
    \end{tabular}
    \caption[\qvtr transformations for the running example]{Three binary QVT-R transformations forming a consistency specification, based on the relations in \autoref{fig:compatibility:three_persons_example_extended}. Taken from \owncite{klare2020compatibility-report}.}
    \label{lst:correctness:prevention:running_example_qvtr}
\end{figure*}

\todo{Take subsubsections of the following as subsection}
%%%%%%%%%%%%%%%%%%%%%%%%%%%%%%%%%%%%%%%%%%%%%%%%%%%%%%%%%%%%%%%%%%%%
\subsection{Decomposition Procedure}

%% ---------------------------------------------
%% /!\ Pay attention not to mix terminology with QVT-R vocabulary
%% /!\ "metagraph"? confusion?

In this section, we introduce a fully automated procedure to achieve the decomposition of transformation networks. The procedures takes a consistency relation set as an input and virtually deletes as many redundant consistency relations as possible. The result is a simplified, possibly tree-like transformation network, which 
is equivalent to %models the same consistency specification as 
the input network. 
The topology of the resulting network gives information about compatibility of its relations. If the resulting network is a consistency relation tree, compatibility is inherent. Otherwise, developers can focus on consistency relations in remaining cycles to detect possible incompatibilities.
The procedure considers transformations written in QVT-R, whose mapping to consistency relations was exposed above.
%To make our approach operational and usable in a model-driven engineering context, 
%the procedure considers transformations written using the QVT-R language. %The shift from QVT-R transformations to consistency relations was exposed in the previous section.

The decomposition procedure relies on an algorithmic way to detect redundant consistency relations. Redundancy occurs in two ways. First, it requires the existence of (sequences of) relations that relate the same metamodels. Second, it indicates that definitions of these consistency relations overlap in some way. In fact, this captures both aspects of consistency specifications. First, transformation networks give an insight into the specification structure, i.e.\ \textit{which} metamodels are related with each other. This global point of view helps to check compatibility: a cycle in the network may indicate contradictory transformations, whereas compatibility is inherent in a tree topology.
However, identifying redundant transformations requires to know exactly \textit{how} metamodels are related to each other. It is necessary to see how consistency is defined with class properties at the metamodel element level to find out if some transformation definitions are contradictory. In predicate-based consistency relations, these definitions are made explicit through the use of predicates. This is a local point of view: given a set of transformations forming a cycle in a transformation network, each transformation definition is retrieved and compared to others so as to assess compatibility. Such a comparison is called a \textit{redundancy test}.
%%          ↑ (and to perform a decomposition)
As a result, transformations are both edges of a graph and sets of consistency relations with exact definitions. Although both aspects are essential for decomposing transformations, the graph can be generated from consistency relation definitions. That is, vertices of the graph are metamodels and whenever a consistency relation definition relates two elements from two different metamodels, there is an edge between these metamodels. To take advantage of both aspects, we propose to perform redundancy tests on a single structure: a graph of class properties labeled by predicates that define consistency relations. The procedure operates in two phases. First, the decomposition procedure creates this structure out of \qvtr transformations. Then, it refers to consistency relation definitions inside it to detect redundant relations and check compatibility of the consistency specification. 

\subsubsection{From Consistency Specification to Property Graph}

The decomposition procedure uses an intermediate representation of transformation networks, i.e., a single data structure that combines both aspects of transformations. This data structure, called a \textit{property graph}, brings the graph characterization of transformations at the level of class properties and predicates. Such a structure can be represented as a hypergraph with a labeling.

\begin{definition}[Property Graph]
Let $\consistencyrelationset{CR} = \{\consistencyrelation{CR}{i}\}_{i = 1}^{n}$ be a set of consistency relations where each consistency relation $\consistencyrelation{CR}{i}$ is based on a set of predicates $\Pi_i$. 
A property graph is a couple $\mathcal{M} =(\mathcal{H}, l)$, such that $\mathcal{H} = (V_{\mathcal{H}}, E_{\mathcal{H}})$ is a hypergraph and $l\colon E_{\mathcal{H}}\to \{\textsc{true}, \textsc{false}\}^{I_{\propertysettuple{P}{\classtuple{C}{l}}} \times I_{\propertysettuple{P}{\classtuple{C}{r}}}}$ is a hyperedge labeling:
        \begin{itemize}
            \item $V_{\mathcal{H}}$ is the set of vertices, i.e., the set of all properties used in all predicates:
\[V_{\mathcal{H}} = \bigcup_{i = 1}^{n} \bigcup_{\pi\ \in\ \Pi_{i}} \propertycollection{\pi}\]
            \item $E_{\mathcal{H}}$ is the set of hyperedges, i.e., $E_{\mathcal{H}} \subseteq \mathcal{P}(V_{\mathcal{H}}) \setminus \{\varnothing\}$. For a property graph, hyperedges are made up of properties that occur in the same predicate:
\[E_{\mathcal{H}} = \bigcup_{i = 1}^{n} \bigcup_{\pi\ \in\ \Pi_{i}} \left\{\propertycollection{\pi}\right\}\]
            \item $l$ is a function that labels each hyperedge with its corresponding predicate function:
\begin{align*}
    &\forall i \in \{1, \dots, n\}, \forall \pi = (\propertysettuple{P}{\classtuple{C}{l}},\ \propertysettuple{P}{\classtuple{C}{r}},\ f_\pi)\ \in\ \Pi_{i} :\\
    &\formulaskip l(\propertycollection{\pi}) = f_\pi
\end{align*}
        \end{itemize}
        \label{def:propertygraph}
\end{definition}

The idea behind the property graph is to group properties that participate in the definition of the same predicate. When the consistency relation set is not a tree, some properties may be used in the definition of multiple consistency relations. For example, an employee's name must be consistent with a resident's name and a person's first and last name. When properties are vertices, such groups form hyperedges. For a consistency relation to be redundant, there must be other relations that share properties with it. As a consequence, the property graph is useful to detect independent sets of consistency relations as well as cycles of hyperedges (which form alternative concatenations for redundant relations). Hyperedges only address the structural aspect of consistency relation definitions. For this reason, each hyperedge is labeled with its corresponding predicate function.

\begin{figure}
    \centering
    \resizebox{\linewidth}{!}{\DeclareRobustCommand{\hlblue}[1]{{\sethlcolor{kit-blue30}\hl{#1}}}
\DeclareRobustCommand{\hlgreen}[1]{{\sethlcolor{kit-green30}\hl{#1}}}
\DeclareRobustCommand{\hlyellow}[1]{{\sethlcolor{kit-yellow30}\hl{#1}}}
\DeclareRobustCommand{\hlorange}[1]{{\sethlcolor{kit-orange30}\hl{#1}}}
\DeclareRobustCommand{\hlred}[1]{{\sethlcolor{kit-red30}\hl{#1}}}

    \begin{tikzpicture}%[show background grid]
        \tikzstyle{every node}=[font=\Large]
        %% METAGRAPH EDGES
        \filldraw[fill=kit-orange30, draw=kit-orange100, thin, fill opacity=0.6] (-0.5, -0.5)
            to[out=0, in=270, looseness=0.2] (8, 4)
            to[out=90, in=0, looseness=0.2] (6, 6)
            to[out=180, in=180, looseness=0.2] node[midway, name=orangeN] {} (-0.5, -0.5);

        \filldraw[fill=kit-yellow30, draw=kit-yellow100, thin, fill opacity=0.6] (-0.5, 0)
            to[out=-45, in=225, looseness=0.3] node[midway, name=yellowN] {} (9.5, -1.1)
            to[out=45, in=0, looseness=0.2] (6, 0.5)
            to[out=180, in=90, looseness=0.2] (-0.5, 0);

        \coordinate (greenN) at (9, 2);
        \filldraw[fill=kit-green30, draw=kit-green100, thin, fill opacity=0.6] (9.1, -1.5)
            to[out=135, in=270, looseness=0.4] (5.7, 5.4)
            to[out=0, in=90, looseness=0.4] (9, 2)
            to[out=270, in=0, looseness=0.4] (9.1, -1.5);

        \filldraw[fill=kit-blue30, draw=kit-blue100, thin, fill opacity=0.6] (-1.5, 1.5)
            to[out=0, in=300, looseness=0.3] (4.5, 6.5)
            to[out=180, in=90, looseness=0.2] (1, 5.2)
            to[out=225, in=90, looseness=0.3] (-1.5, 1.5);
        
        \filldraw[fill=kit-blue30, draw=kit-blue100, thin, fill opacity=0.6] (0.4, -2)
            to[out=90, in=90, looseness=0.2] (8.7, -3.1)
            to[out=270, in=0, looseness=0.3] (4.5, -3.2)
            to[out=180, in=-45, looseness=0.3] (0.4, -2);
            
        \filldraw[fill=kit-blue30, draw=kit-blue100, thin, fill opacity=0.6] (10, 4.5)
            to[out=225, in=135, looseness=0.2] (11, -0.5)
            to[out=0, in=270, looseness=0.3] (11.6, 3)
            to[out=100, in=0, looseness=0.4] (10, 4.5);

        %% METAGRAPH NODES
        \node[circle, blueshading, name=circleResN] at ($(0, 0)$) {};
        \node[above = 0.1cm of circleResN] {\texttt{Resident::name}};

        \node[circle, blueshading, name=circleEmpN] at ($(9, -1)$) {};
        \node[above = 0.1cm of circleEmpN] {\texttt{Employee::name}};

        \node[circle, blueshading, name=circlePersF] at ($(7, 4)$) {};
        \node[below = 0.1cm of circlePersF] {\texttt{Person::firstname}};

        \node[circle, blueshading, name=circlePersN] at ($(6, 5)$) {};
        \node[above = 0.1cm of circlePersN] {\texttt{Person::lastname}};
        
        %% METAGRAPH NODES (part 2)
        \node[circle, blueshading, name=circlePersA] at ($(4, 6)$) {};
        \node[above = 0.1cm of circlePersA] {\texttt{Person::address}};

        \node[circle, blueshading, name=circlePersI] at ($(10.5, 4)$) {};
        \node[above = 0.1cm of circlePersI] {\texttt{Person::income}};
        
        \node[circle, blueshading, name=circleEmpSo] at ($(8, -3)$) {};
        \node[above = 0.1cm of circleEmpSo] {\texttt{Employee::socsecnumber}};
        
        \node[circle, blueshading, name=circleEmpSa] at ($(11, 0)$) {};
        \node[above = 0.1cm of circleEmpSa] {\texttt{Employee::salary}};
        
        \node[circle, blueshading, name=circleResA] at ($(-1, 2)$) {};
        \node[above = 0.1cm of circleResA] {\texttt{Resident::address}};
        
        \node[circle, blueshading, name=circleResS] at ($(1, -2)$) {};
        \node[above = 0.1cm of circleResS] {\texttt{Resident::socsecnumber}};
        
        \node at (4, 3) {\textbf{1}};
        \node at (8, 2) {\textbf{2}};
        \node at (5, -0.5) {\textbf{3}};
        \node at (1, 4) {\textbf{4}};
        \node at (11, 2) {\textbf{5}};
        \node at (4, -2.5) {\textbf{6}};
        
        \node[] () [below = 2cm of circleEmpSo] {};
    \end{tikzpicture}}
    \resizebox{\linewidth}{!}{\begin{tikzpicture}
    \tikzstyle{every node}=[font=\Large]
        % METAGRAPH SYSTEMS
    \node[draw=kit-orange100, dashed, name=n1, minimum height=2.5cm] at (0, -6) {%
        $\begin{cases}
            \textup{\texttt{ Person::firstname=fstn}} \\
            \textup{\texttt{ Person::lastname=lstn}} \\
            \textup{\texttt{ }\lstinline[language=QVT]{Resident::name=fstn + ' ' + lstn}}
        \end{cases}$
    };

    \node[draw=kit-green100, dashed, name=n2, minimum height=2.5cm] at (0, -9.7) {%
        $\begin{cases}
            \textup{\texttt{ Person::firstname=fstn}} \\
            \textup{\texttt{ Person::lastname=lstn}} \\
            \textup{\texttt{ }\lstinline[language=QVT]{Employee::name=fstn + ' ' + lstn}}
        \end{cases}$
    };

    \node[draw=kit-yellow100, dashed, name=n3, minimum height=2.5cm, minimum width=9.5 cm] at (0, -13.4) {%
        $\begin{cases}
            \textup{\texttt{ Resident::name=n}} \\
            \textup{\texttt{ Employee::name=n}}
        \end{cases}$
    };
    
    \node[draw=kit-blue100, dashed, name=n4, minimum height=2.5cm, minimum width=8.5 cm, anchor=west] at (5.5, -6) {%
        $\begin{cases}
            \textup{\texttt{ Resident::address=addr}} \\
            \textup{\texttt{ Employee::address=addr}}
        \end{cases}$
    };
    
    \node[draw=kit-blue100, dashed, name=n5, minimum height=2.5cm, minimum width=8.5 cm, anchor=west] at (5.5, -9.7) {%
        $\begin{cases}
            \textup{\texttt{ Person::income=inc}} \\
            \textup{\texttt{ Employee::salary=inc}}
        \end{cases}$
    };
    
    \node[draw=kit-blue100, dashed, name=n6, minimum height=2.5cm, minimum width=8.5 cm, anchor=west] at (5.5, -13.4) {%
        $\begin{cases}
            \textup{\texttt{ Person::socsecnumber=ssn}} \\
            \textup{\texttt{ Employee::socsecnumber=ssn}}
        \end{cases}$
    };

    \node[below = 0.1cm of n1] {1};
    \node[below = 0.1cm of n2] {2};
    \node[below = 0.1cm of n3] {3};
    \node[below = 0.1cm of n4] {4};
    \node[below = 0.1cm of n5] {5};
    \node[below = 0.1cm of n6] {6};
\end{tikzpicture}}
    \caption[Property graph for the running example]{Property graph for the QVT-R example in \autoref{lst:correctness:prevention:running_example_qvtr} based on the relations in \autoref{fig:compatibility:three_persons_example_extended}. Taken from \owncite{klare2020compatibility-report}.}
    \label{fig:correctness:prevention:propertygraph_re}
\end{figure}


The need for hypergraphs arises from the fact that predicates can relate more than two properties: the predicate in the relation $\consistencyrelation{CR}{PE}$, which ensures equality of an employee's name and the concatenation of first and last name of a person, contains three properties.
% Metamodel elements participating in the consistency specification are vertices of the hypergraph. Then, metamodel elements that appear in the same consistency specification. Each edge is labeled with the QVT-R relation(s) (i.e.\ fine-grained consistency relations) from which it is derived, so that no information is lost during the construction of the metagraph.
The first phase of the procedure is to set up such a structure using QVT-R transformations. 

\paragraph{Input and Transformation Parsing}
The procedure takes a set of well-formed QVT-R files as inputs. Each file contains one or more transformations.
Transformations and metamodels are then parsed to provide a logical view of the consistency specification. %Access to metamodels and the specification is read-only because the removal of redundant consistency relations is only virtual. %: no QVT-R file is modified during the execution of the procedure. 
Decomposition is intended to help the developer by either proving compatibility or highlighting possible causes of incompatibility. 
It does, however, not update the specification, thus access to the specification is read-only.

% \paragraph{Input Parameters}
% The procedure takes a set of well-formed QVT-R files as inputs. Each file contains one or more transformations, as well as \texttt{import} statements for metamodels involved in transformations. Having one transformation per file is a good practice, as it fosters separation of concerns.

% \paragraph{Transformation Parsing}
% Transformations and metamodels are then parsed to provide a logical view of the consistency specification. Access to metamodels and the specification is read-only because the removal of redundant consistency relations is only virtual: no QVT-R file is modified during the execution of the procedure. Decomposition is intended to help the developer by either proving compatibility or highlighting possible causes of incompatibility. It does, howver, not update the specification.
%Although metamodels may be imported in multiple transformations, they are only parsed once. This is achieved by gathering all metamodels in a set before parsing them.

\paragraph{Traversal Order Among Transformations}
In order to build the property graph, each transformation must be processed to retrieve relations it contains. The decomposition procedure processes one transformation at a time. For compatibility checking purposes, transformations are independent of each other. They can be processed separately and in any order. The reason behind this is that QVT-R relations are not executed on models but only read. Therefore, they are side-effect free.

\paragraph{Traversal Order Inside Transformations}
A transformation is a set of QVT-R relations. In general, each relationship deals with the consistency of only a small part of each metamodel, for example the consistency between two classes. 
%This design pattern, known as \textit{phased construction}, promotes modularity~\cite{lano2014a}. 
Unlike QVT-R transformations, QVT-R relations cannot be processed in any order. There are two types of relations: top-level relations and non-top-level relations. 
Top-level relations are always invoked, whereas non-top-level relations are only invoked in \texttt{where} or \texttt{when} clauses of other relations through a mechanism similar to a function call. Only relations that would be invoked during the execution of transformations have to be processed for the decomposition, as non-invoked relations cannot cause incompatibilities. To determine a relevant processing order and retain only QVT-R relations that can be invoked, one solution is to create the call graph of the transformation.
%during its execution. 
We impose a restriction on the specification to make this representation easier: relations may only be invoked in \texttt{where} clauses. Starting from top-level relations, relations are visited using a depth-first traversal. A relation $R_2$ can be visited from a relation $R_1$ if $R_1$'s \texttt{when} clause invokes $R_2$. As a result, a relation is only visited if it is top-level or if a relation invoking it was itself visited before.
%% WITHOUT LOSS OF GENERALITY?

% \todo[inline, color=kit-orange100]{AP: add an example of call graph? or refer to the MA?}

\begin{figure} % EQC
\begin{embeddedqvtcode}[frame=bt, numbers=none, mathescape=true, caption={Structure of a QVT-R relation with property template items. Taken from \owncite{klare2020compatibility-report}.},captionpos=b,label={lst:correctness:prevention:rel_to_hyperedge}]
relation R {
    [variable declarations]
    
    domain $MM_1$ a : A {$P_{A, 1}$=$e_{A, 1}$, $P_{A, 2}$=$e_{A, 2}$}
    domain $MM_2$ b : B {$P_{B, 1}$=$e_{B, 1}$}
    [when { PRE }] [where { INV }]
}
\end{embeddedqvtcode}
\end{figure}

\paragraph{From QVT-R Relation to Hyperedge}
At the beginning, the property graph is an empty object. Each time a QVT-R relation is processed, the property graph may get new vertices and a new hyperedge. In accordance with \autoref{def:propertygraph}, hyperedges can be generated from predicates. Therefore, it is relevant to translate each QVT-R relation into a set of predicates. \autoref{lst:correctness:prevention:rel_to_hyperedge} depicts the structure of an abstract QVT-R relation between two metamodels $MM_1$ and $MM_2$. Defining a predicate from a QVT-R relation amounts to find important properties for each metamodel and definitions that bind them. Class tuples are composed of classes that occur in each domain, i.e., $\classtuple{C}{MM_1} = \tupled{A}$ and $\classtuple{C}{MM_2} = \tupled{B}$. Each class in each class tuple is associated with a set of property template items.
Important properties for the consistency specification are in the left-hand side of each property template item. For example, the property template item $P_{A, 1} = e_{A, 1}$ indicates that the property $P_{A, 1}$ must match the OCL expression $e_{A, 1}$ in which there are QVT-R variables. Not all properties are related to each other within the same QVT-R relation. For example, constraints on \texttt{Employee.salary} and \texttt{Employee.name} are independent, because consistency of one does not depend on consistency of the other. There is a simple criterion in QVT-R to identify interrelated properties. Pattern matching indicates which properties have to be grouped together to build a predicate. If two properties depend on the same \qvtr variable, they are interrelated, because a value assignment must satisfy both property template items. Predicates can then be generated from sets of interrelated properties.
OCL expressions can also occur in \texttt{when} and \texttt{where} clauses. As with relation invocations, we focus on invariants (\texttt{where}), which we limit to the manipulation of QVT-R variables, given that properties can be limited to domain patterns without loss of generality. The processing of an invariant is similar to that of property template items: properties that depend on QVT-R variables occurring in the same invariant have to be grouped together.

\begin{algorithm}[t]
    \algrenewcommand\alglinenumber[1]{\texttt{\footnotesize #1}}
\algdef{SE}[DOWHILE]{Do}{doWhile}{\algorithmicdo}[1]{\algorithmicwhile\ #1}%

\begin{algorithmic}[1]

\Procedure{\textsc{Merge-Consistency-Variables}}{$\{({p}, V_{\{p\}})\}$}
    \State $stopMerge$ $\leftarrow$ \textsc{true} 
    \State $entries$ $\leftarrow$ $\tupled{(\{p\}, V_{\{p\}})}$\\

    \Do
        \State $stopMerge$ $\leftarrow$ \textsc{true}
        \State $results$ $\leftarrow$ $\tupled{}$\\
        
        \While{$entries \neq \tupled{}$}
        % \While{not entries.isEmpty()}
            \State $ref = (P_{\textup{ref}}, V_{p_{\textup{ref}}}) \leftarrow$ $entries[0]$ %\Comment{Reference pair to be compared with others}
            \State $others$ $\leftarrow$ $entries[1:]$
            \State $entries$ $\leftarrow$ $\tupled{}$\\
            
            \For{$(P, V_P) \in others$}
                \If{$V_P \cap V_{P_\textup{ref}} = \varnothing$}
                    \State $entries$ $\leftarrow$ $entries$ $\cup$ $\{(P, V_P)\}$
                \Else
                    \State $stopMerge$ $\leftarrow$ \textsc{False} \label{lst:line:stopMerge}%\Comment{Not all $V_{e_i}$ are pairwise disjoint}
                    \State $ref$ $\leftarrow$ $(P \cup P_{\textup{ref}}, V_P \cup V_{P_{\textup{ref}}})$
                \EndIf
            \EndFor
            \State $results$ $\leftarrow$ $results$ $\cup$ $\{$ref$\}$
        \EndWhile
        \State $entries$ $\leftarrow$ $results$
    \doWhile{$\neg stopMerge$}\\
    \State \Return{set($entries$)}
\EndProcedure

\end{algorithmic}
    \caption[Merge algorithm combining properties to predicates]{Merge algorithm combining properties to predicates. Taken from \owncite{klare2020compatibility-report}.}
    \label{algo:mergeproperties}
\end{algorithm}

\autoref{algo:mergeproperties} formalizes the way properties are grouped to form predicates. At the beginning of the algorithm, each property is associated with QVT-R variables that occur in the corresponding property template item. This association, called an \textit{entry}, is a couple $(\{p\}, V_{\{p\}})$ where $\{p\}$ is a singleton containing the property $p$ and $V_{\{p\}}$ a set of QVT-R variables. The entry of an invariant is composed of variables in it and all properties associated with these variables through property template items. At each iteration, the algorithm chooses a reference entry and merges all other entries with it if the intersection of their sets of QVT-R variables is nonempty. The algorithm stops when all sets of QVT-R variables are pairwise disjoint.

\begin{example}
There are five properties in the relation \texttt{PE} of the QVT-R transformation \texttt{PersonEmployee} in \autoref{lst:correctness:prevention:running_example_qvtr}, which can be described with the following entries:
\begin{align*}
\formulaskip
&(\{\propdisplay{firstname}\}, \{fstn\}), (\{\propdisplay{lastname}\}, \{lstn\}),\\ &(\{\propdisplay{income}\}, \{inc\}), (\{\propdisplay{name}\}, \{fstn, lstn\}),\\ &(\{\propdisplay{salary}\}, \{inc\})
\end{align*}
After the execution of the algorithm, properties are merged as follows:
\begin{align*}
\formulaskip
&(\{\propdisplay{firstname}, \propdisplay{lastname}, \propdisplay{name}\}, \{fstn, lstn\}),\\
&(\{\propdisplay{income}, \propdisplay{salary}\}, \{inc\})
\end{align*}
This results in two sets of properties.
\end{example}

At the end of the algorithm, each entry can be transformed into a hyperedge. To do so, properties of the entry are assigned to the classes out of which they originate to form property sets. These property sets are grouped into two tuples. The predicate function is the conjunction of all OCL expressions associated with properties of the entry.
When all transformations and all QVT-R relations have been processed, the property graph is correctly initialized. It is now invariable, in the sense that the procedure cannot add new vertices or hyperedges to the graph. Only hyperedges identified as redundant can then be removed. Moreover, all the information needed to assess compatibility in the consistency specification is translated into the property graph. There is no need to query metamodels or QVT-R transformations anymore.

%\begin{itemize}
    % \item \textbf{Phase IV}. \textit{QVT-relation -> Hyperedge}. Main idea: "metamodel elements are interrelated if they are bound to the same QVT-R variables", mechanism of pattern matching. Consequence: group together metamodel elements depending on variables they are bound to. (Algorithm: \formalize{this kind of transitive closure}) => to hyperedges
    % \item \textbf{Phase V}. \textit{Hyperedge + Labeling}. Reason: hyperedges only indicate that instances of elements must be consistent but it does not specify under which rules they should be. (+ processing of invariants and preconditions)
    %\item \textbf{RESULT}. At this stage, the metagraph is permanent (no metamodel element, no hyperedge can be added). It is also the only structure to perform a decomposition: no more queries on metamodels and QVT-R transformations.  
    % \item \textbf{INPUT}. A set of QVT-R (well-formed) files: metamodels (\texttt{import}) and transformations. The construction of the metagraph reproduces the execution of every transformation but results in another view of the consistency specification rather than checking/enforcing consistency.
    % \item \textbf{Phase I}. Read-only access to metamodels and specification (because of \textit{virtual} removal). [Implem details: relies on Ecore's Resource]
    % \item \textbf{Phase II}. \textit{Inter-transformations}. Transformations are independent of each other, i.e.\ they can be processed separately and in any order. Reasons: relations are side-effect free and transformations are the most outer objects in QVT-R. Benefits: independent specification of transformations for developers. The decomposition procedure processes one transformation at a time.
%\end{itemize}

\subsubsection{From Property Graph to Decomposition}

Given a property graph $\mathcal{M} = (\mathcal{H}, l)$, decomposition is accomplished by removing redundant consistency relations (hyperedges of $\mathcal{H}$) until all relations have been tested once or until the property graph is only composed of trees.
The hypergraph $\mathcal{H}$ provides valuable information about the nature of the consistency specification. First, a necessary condition for a consistency relation between two metamodels $\metamodel{M}{1}$ and $\metamodel{M}{2}$ to be redundant according to \autoref{def:redundancy} is the existence of an alternative concatenation of relations that links $\metamodel{M}{1}$ and $\metamodel{M}{2}$ too. In terms of graph, there must exist a path between $\metamodel{M}{1}$ and $\metamodel{M}{2}$. Second, consistency relation definitions can be independent from each other, in the sense that they share no properties. Following \autoref{theorem:independencecompatibility}, the union of two independent and compatible consistency relations is also compatible. Independency in the hypergraph is given by connected components. An important aspect of the decomposition procedure is to find consistency relation definitions that can be tested for redundancy. Taking advantage of the structure of the graph is useful for listing these relations.

\paragraph{Consistency Relation Preprocessing}
As a special kind of hypergraph, a property graph has the advantage of being expressive when it comes to model consistency relation definitions involving multiple properties. The downside is that common graph algorithms (such as graph traversal) become harder to define and to apply. The choice between graphs and hypergraphs is a balance between abstraction and usability. 
For purposes of implementation, the property graph is replaced by its dual, i.e., a simple graph that is equivalent to it. The dual of the property graph is the graph whose vertices are hyperedges of the property graph. If hyperedges of the property graph share at least one property, their corresponding vertices in the dual are linked. An example for the dual of a property graph is given in \autoref{fig:correctness:prevention:dual_propertygraph_re}.

\todoDiss{Prove that if the dual of the meta graph is a tree, it is a consistency relation tree}


\begin{definition}[Dual of a Property Graph]
Let $\mathcal{M} = (\mathcal{H}, l)$ be a property graph. The dual of the property graph $\mathcal{M}$, denoted $\mathcal{M^{*}}$ is a tuple $(\mathcal{G}, v, l)$ with a simple graph $\mathcal{G}$ and two functions $v$ and $l$ such that:
    \begin{itemize}
        \item $V_{\mathcal{G}} = E_{\mathcal{H}}$
        \item $E_{\mathcal{G}} = \{\{E_1, E_2\} \mid \forall (E_1, E_2) \in E_{\mathcal{H}}^2 : E_1 \cap E_2 \neq \varnothing\}$
        \item $\forall (E_1, E_2) \in E_{\mathcal{G}} : v(\{E_1, E_2\}) = E_1 \cap E_2$
    \end{itemize}
\end{definition}

Each edge $\{E_1, E_2\}$ in the dual is labelled with the set of properties that occur both in $E_1$ and $E_2$. The dual contains all the information necessary to build the property graph again. Given a dual $\mathcal{M^{*}} = (\mathcal{G}, v, l)$, the property graph $\mathcal{M} = (\mathcal{H}, l)$ can be built by defining $V_{\mathcal{H}} = \bigcup_{V \in V_{\mathcal{G}}} V$ and $E_{\mathcal{H}} = V_{\mathcal{G}}$. 

\begin{figure}
    \centering
    \resizebox{0.8\linewidth}{!}{%% TITLE: From QVT-R files to {metamodels, specifications}
%% Don't forget to compile the main document!

\DeclareRobustCommand{\hlblue}[1]{{\sethlcolor{kit-blue30}\hl{#1}}}
\DeclareRobustCommand{\hlgreen}[1]{{\sethlcolor{kit-green30}\hl{#1}}}
\DeclareRobustCommand{\hlyellow}[1]{{\sethlcolor{kit-yellow30}\hl{#1}}}
\DeclareRobustCommand{\hlorange}[1]{{\sethlcolor{kit-orange30}\hl{#1}}}
\DeclareRobustCommand{\hlred}[1]{{\sethlcolor{kit-red30}\hl{#1}}}

\begin{tikzpicture}%[show background grid]
    \tikzstyle{every node}=[font=\Large]
    %% GRAPH NODES
    \node[circle, fill=kit-green30, draw=kit-green100, name=greenN] at (10, 3) {};
    \node[circle, fill=kit-orange30, draw=kit-orange100, name=orangeN] at (0, 2) {};
    \node[circle, fill=kit-yellow30, draw=kit-yellow100, name=yellowN] at (5, -2) {};

    \node[circle, fill=kit-blue30, draw=kit-blue100, name=salary] at (-0.5, -2) {};
    
    \node[circle, fill=kit-blue30, draw=kit-blue100, name=socsecnumber] at (-2, 0) {};
    
    \node[circle, fill=kit-blue30, draw=kit-blue100, name=address] at (10, -1) {};

    \draw[-] (greenN) -- (orangeN) node [midway, above, sloped] (T1) {\texttt{Person::firstname, Person::lastname}};
    \draw[-] (greenN) -- (yellowN) node [midway, above, sloped] (T2) {\texttt{Resident::name}};
    \draw[-] (orangeN) -- (yellowN) node [midway, above, sloped] (T1) {\texttt{Employee::name}};

    \node[left = 0.2cm of orangeN] {\textbf{1}};
    \node[right = 0.2cm of greenN] {\textbf{2}};
    \node[below = 0.2cm of yellowN] {\textbf{3}};
    \node[left = 0.2cm of address] {\textbf{4}};
    \node[right = 0.2cm of salary] {\textbf{5}};
    \node[right = 0.2cm of socsecnumber] {\textbf{6}};

    \node[] () [below = 2cm of yellowN] {};

    %% METAGRAPH SYSTEMS
    % \node[draw=kit-orange100, dashed, name=n1, minimum height=2.5cm] at (0, -6) {%
    %         $\begin{cases}
    %             \textup{\texttt{ Person::firstname=fstn}} \\
    %             \textup{\texttt{ Person::lastname=lstn}} \\
    %             \textup{\texttt{ }\lstinline[language=QVT]{Resident::name=fstn + ' ' + lstn}}
    %         \end{cases}$
    %     };

    %     \node[draw=kit-green100, dashed, name=n2, minimum height=2.5cm] at (0, -10) {%
    %         $\begin{cases}
    %             \textup{\texttt{ Person::firstname=fstn}} \\
    %             \textup{\texttt{ Person::lastname=lstn}} \\
    %             \textup{\texttt{ }\lstinline[language=QVT]{Employee::name=fstn + ' ' + lstn}}
    %         \end{cases}$
    %     };

    %     \node[draw=kit-yellow100, dashed, name=n3, minimum height=2.5cm, minimum width=9.5 cm] at (0, -14) {%
    %         $\begin{cases}
    %             \textup{\texttt{ Resident::name=n}} \\
    %             \textup{\texttt{ Employee::name=n}}
    %         \end{cases}$
    %     };
        
    %     \node[draw=kit-blue100, dashed, name=n4, minimum height=2.5cm, minimum width=9.5 cm] at (10.5, -6) {%
    %         $\begin{cases}
    %             \textup{\texttt{ Resident::address=addr}} \\
    %             \textup{\texttt{ Employee::address=addr}}
    %         \end{cases}$
    %     };
        
    %     \node[draw=kit-blue100, dashed, name=n5, minimum height=2.5cm, minimum width=9.5 cm] at (10.5, -10) {%
    %         $\begin{cases}
    %             \textup{\texttt{ Person::income=inc}} \\
    %             \textup{\texttt{ Employee::salary=inc}}
    %         \end{cases}$
    %     };
        
    %     \node[draw=kit-blue100, dashed, name=n6, minimum height=2.5cm, minimum width=9.5 cm] at (10.5, -14) {%
    %         $\begin{cases}
    %             \textup{\texttt{ Person::socsecnumber=ssn}} \\
    %             \textup{\texttt{ Employee::socsecnumber=ssn}}
    %         \end{cases}$
    %     };

    % %% METAGRAPH LINES
    % % \draw[-, kit-orange100] (n1) -- (orangeN);
    % % \draw[-, kit-green100] (n2) -- (greenN);
    % % \draw[-, kit-yellow100] (n3) -- (yellowN);

    % \node[below = 0.1cm of n1] {1};
    % \node[below = 0.1cm of n2] {2};
    % \node[below = 0.1cm of n3] {3};
    % \node[below = 0.1cm of n4] {4};
    % \node[below = 0.1cm of n5] {5};
    % \node[below = 0.1cm of n6] {6};
\end{tikzpicture}}
    \resizebox{\linewidth}{!}{\begin{tikzpicture}
    \tikzstyle{every node}=[font=\Large]
        % METAGRAPH SYSTEMS
    \node[draw=kit-orange100, dashed, name=n1, minimum height=2.5cm] at (0, -6) {%
        $\begin{cases}
            \textup{\texttt{ Person::firstname=fstn}} \\
            \textup{\texttt{ Person::lastname=lstn}} \\
            \textup{\texttt{ }\lstinline[language=QVT]{Resident::name=fstn + ' ' + lstn}}
        \end{cases}$
    };

    \node[draw=kit-green100, dashed, name=n2, minimum height=2.5cm] at (0, -9.7) {%
        $\begin{cases}
            \textup{\texttt{ Person::firstname=fstn}} \\
            \textup{\texttt{ Person::lastname=lstn}} \\
            \textup{\texttt{ }\lstinline[language=QVT]{Employee::name=fstn + ' ' + lstn}}
        \end{cases}$
    };

    \node[draw=kit-yellow100, dashed, name=n3, minimum height=2.5cm, minimum width=9.5 cm] at (0, -13.4) {%
        $\begin{cases}
            \textup{\texttt{ Resident::name=n}} \\
            \textup{\texttt{ Employee::name=n}}
        \end{cases}$
    };
    
    \node[draw=kit-blue100, dashed, name=n4, minimum height=2.5cm, minimum width=8.5 cm, anchor=west] at (5.5, -6) {%
        $\begin{cases}
            \textup{\texttt{ Resident::address=addr}} \\
            \textup{\texttt{ Employee::address=addr}}
        \end{cases}$
    };
    
    \node[draw=kit-blue100, dashed, name=n5, minimum height=2.5cm, minimum width=8.5 cm, anchor=west] at (5.5, -9.7) {%
        $\begin{cases}
            \textup{\texttt{ Person::income=inc}} \\
            \textup{\texttt{ Employee::salary=inc}}
        \end{cases}$
    };
    
    \node[draw=kit-blue100, dashed, name=n6, minimum height=2.5cm, minimum width=8.5 cm, anchor=west] at (5.5, -13.4) {%
        $\begin{cases}
            \textup{\texttt{ Person::socsecnumber=ssn}} \\
            \textup{\texttt{ Employee::socsecnumber=ssn}}
        \end{cases}$
    };

    \node[below = 0.1cm of n1] {1};
    \node[below = 0.1cm of n2] {2};
    \node[below = 0.1cm of n3] {3};
    \node[below = 0.1cm of n4] {4};
    \node[below = 0.1cm of n5] {5};
    \node[below = 0.1cm of n6] {6};
\end{tikzpicture}}
    \caption[Dual of the property graph for the running example]{Dual of the property graph for the QVT-R example in \autoref{lst:correctness:prevention:running_example_qvtr} based on the relations in \autoref{fig:compatibility:three_persons_example_extended}. Taken from \owncite{klare2020compatibility-report}.}
    \label{fig:correctness:prevention:dual_propertygraph_re}
\end{figure}

\paragraph{Independent Subsets of Consistency Relations}
In a consistency specification, consistency relations can form independent sets, in the sense that the consistency of one set can be checked and repaired independently without affecting the consistency of the other set. This occurs at two levels. 
First, at the metamodel level, when there exist two sets of metamodels such that no metamodel of one set is bound to a metamodel of the other set through a consistency relation. 
Second, at the metamodel element level, when two consistency relation sets of the same metamodel relate objects that are independent of each other. In terms of consistency relations and predicates, such sets are made up of relations that do not share any property.
Independence is characterized in the same way for both levels in the property graph. This results in two subhypergraphs\footnote{A subhypergraph of a hypergraph $\mathcal{H} = (V_{\mathcal{H}}, E_{\mathcal{H}})$ is a hypergraph $\mathcal{S} = (V_{\mathcal{S}}, E_{\mathcal{S}})$ such that $E_{\mathcal{S}} \subseteq E_{\mathcal{H}}$ and $V_{\mathcal{S}} = \bigcup_{E \in E_{\mathcal{S}}} E$} such that there is no path (i.e., sequence of incident hyperedges) from one to the other.
In the dual of the property graph, this results in two subgraphs that are not connected to each other as well. Otherwise, there would exist two consistency relations (one from each subgraph) that share at least one property, thus contradicting the hypothesis of independence. Independent subsets of hyperedges in the property graph can be processed independently, since one's compatibility has no influence on compatibility of the others.

Once the property graph is converted to its dual, the decomposition procedure computes independent subsets of relations. This can be achieved by computing connected components in the dual. Connected components are maximal subgraphs such that there exists a path between any two vertices in it. We use Tarjan's algorithm to compute them in linear time~\cite{tarjan1972depth}. Incompatibilities can then only occur within a connected component. Therefore, we prove that a consistency relation set is compatible by proving compatibility of every connected component of the dual of the property graph.
% same characterization of dual and hypergraph

\paragraph{Generation of Candidate Relations}
For a connected component to be compatible, there must be no redundant predicate in it. Like consistency relations at the metamodel level, predicates at the metamodel element level are said to be redundant if consistency specifications with and without it are equivalent. In the property graph, this requires the existence of an alternative sequence of hyperedges that relate the same properties as the possibly redundant hyperedge. 
%Removing a hyperedge without having an alternative path would result in properties missing from all other predicates, thus weakening the consistency specification. 
Note that the existence of an alternative path is a necessary but not a sufficient condition. For instance, a predicate ensuring that two \texttt{String} attributes are equal could not be replaced by a sequence of predicates only ensuring that these strings have the same length. That is why the possibly redundant predicate and the alternative path of predicates are subject to a redundancy test (see \autoref{chap:prevention:compatibility:redundancies}).

Given that connected components are independent, a predicate can only be replaced by other predicates in the same component. Moreover, \autoref{theorem:treecompatibility} also applies to the dual of the property graph: once the component is a tree, it is inherently compatible. As a result, the dual proves compatibility of the consistency specification if it is only composed of independent trees. Such a graph is called a \textit{forest}.

In the property graph, an alternative path for a hyperedge $E$ (i.e., a predicate) is a sequence of pairwise incident hyperedges such that the first and the last are also incident to $E$. Hyperedges of the property graph become vertices in the dual. Therefore, an alternative path for a predicate $E$ in the dual is characterized by a cycle that contains $E$. If the vertex sequence of such a cycle is $\tupled{E, E_1, \dots, E_n, E}$, the alternative path is $\tupled{E_1, \dots, E_n}$. Ultimately, the generation of candidate relations for a redundancy test amounts to the enumeration of pairs $(E, \tupled{E_i})$, where $E$ is a possibly redundant predicate (i.e., a hyperedge in the property graph and a vertex in its dual) and $\tupled{E_i}$ is an alternative sequence of predicates that may replace $E$. There may be multiple alternative paths for a given predicate in the dual of the property graph, hence the need to find multiple cycles. The problem of finding all simple cycles in an undirected graph is called \textit{cycle enumeration}.

\begin{algorithm}
    %\let\oldalgorithmic\algorithmic 
%\renewcommand\algorithmic{\ttfamily\fontseries{l}\selectfont\oldalgorithmic}
%\renewcommand{\Comment}[2][.3\linewidth]{\leavevmode\hfill\makebox[#1][l]{//~#2}}

    %\algrenewcommand\alglinenumber[1]{\texttt{\footnotesize #1}}
    \makeatletter
    \algnewcommand{\LineComment}[1]{\Statex \hskip\ALG@thistlm \(// \) #1}
\makeatother

\algdef{SE}[DOWHILE]{Do}{doWhile}{\algorithmicdo}[1]{\algorithmicwhile\ #1}%

\begin{algorithmic}[1]
\Procedure{\textsc{Enumerate-FoundCycles}}{$\text{Dual}\ \mathcal{M}^{*}, $p$ \in V_{\mathcal{M}^{*}}$}
    \State $\{base_1,\dots,base_n\}$ $\leftarrow$ \textsc{Paton-Algorithm}($\mathcal{M}^{*}$)
    \State $foundCycles$ $\leftarrow$ $\{base_1\}$
    \State $currentCycles$ $\leftarrow$ $\varnothing$, $currentCycles^{*}$ $\leftarrow$ $\varnothing$\\

    \For{$base \in \{base_2, \dots, base_n\}$}
        \For{$foundCycle \in foundCycles$}
            \State $newCycle \leftarrow foundCycle \oplus base$
            \If{$foundCycle \cap base \neq \varnothing$}
                \State $currentCycles \leftarrow currentCycles \cup \{newCycle\}$
            \Else
                \State $currentCycles^{*} \leftarrow currentCycles^{*} \cup \{newCycle\}$
            \EndIf
        \EndFor
        \LineComment{Remove non-simple cycles from $currentCycles$}
        \For{$cycle_1, cycle_2 \in currentCycles$}
            \If{$cycle_2 \subset cycle_1$}
                \State $currentCycles \leftarrow currentCycles \setminus \{cycle_1\}$
                \State $currentCycles^{*} \leftarrow currentCycles^{*} \cup \{cycle_1\}$
            \EndIf
        \EndFor
        \LineComment{New valid cycles are in $currentCycles \cup \{base\}$}
        \For{$cand \in currentCycles \cup \{base\}$}\label{lst:line:newCycles}
            \If{$p \in cand$ \textbf{and} \textsc{Replace-Hyperedge}$(p, cand)$}
                \State remove $p$ and its incident edges from $\mathcal{M}^{*}$
                \State \textbf{break}
            \EndIf
        \EndFor
        \State $foundCycles \leftarrow foundCycles \cup currentCycles$
        \State $foundCycles \leftarrow foundCycles \cup currentCycles^{*} \cup \{base\}$
        \State $currentCycles \leftarrow \varnothing$, $currentCycles^{*} \leftarrow \varnothing$
    \EndFor
    %\State \Return{$S$}
\EndProcedure
\end{algorithmic}
    \caption[Enumeration of alternative paths]{Enumeration of alternative paths. Taken from \owncite{klare2020compatibility-report}.}
    \label{algo:cycleenumeration}
\end{algorithm}

The cycle enumeration algorithm used in the decomposition procedure relies on a \textit{cycle basis}. In an undirected graph, a cycle basis is a set of simple cycles that can be combined to generate all other simple cycles of the graph. The cycle basis is first computed using Paton's algorithm~\cite{paton1969algorithm}. For a given predicate, the enumeration processes each cycle from the cycle basis and merges it with all yet processed cycles. In the context of the decomposition procedure, a cycle must also go through the predicate to analyze. \autoref{algo:cycleenumeration} is a slightly modified version of Gibb's algorithm to enumerate simple cycles in an undirected graph using a cycle basis~\cite{gibbs1969cycle}. In this algorithm, every cycle is represented as a set of edges. We denote the symmetric difference with the $\oplus$ sign, i.e., $A \oplus B$ is the set of edges that are in $A$ or in $B$ but not in both. The set $foundCycles$ contains all linear combinations of cycles that were yet processed. Merged with cycles of the basis $base_1, \dots, base_n$, these linear combinations are used to merge more than two cycles of the basis. At each iteration of \autoref{algo:cycleenumeration}, processing a new cycle from the cycle basis $base$, new simple cycles are in $currentCycles \cup \{base\}$. Edge-disjoint or non-simple cycles are stored in $currentCycles^*$. Note that redundancy tests can be performed as new cycles are generated, as shown on line \ref{lst:line:newCycles}. For the given predicate $p$, it is checked whether one of the newly generated cycles is redundant, i.e., it contains the predicate $p$ and $p$ can be replaced by the concatenation of the other edges. By doing so, it is not necessary to store all cycles and wait for the end of the algorithm before starting redundancy tests. More interestingly, if the redundancy test is positive for one alternative sequence of predicates, there is no need to test others. The initial predicate can be removed and the algorithm can be used with another possibly redundant predicate.
  
\paragraph{Stopping Criterion}
The decomposition procedure stops when each predicate has been tested once. Note that if the connected component becomes a tree after a few removals of predicates, then the last tests of remaining predicates are trivial. As there are no more cycles in the dual of the connected component, no redundancy test is performed.

\subsubsection{Summary}
In this section, we have presented an algorithm for proving compatibility of relations in a consistency specification written in \qvtr.
We defined property graphs and their dual as a representation of consistency relations and explained how they can be derived from a specification in \qvtr.
We discussed how a consistency relation tree manifests in such a representation and how candidates for redundancies in connected components of such a representation can be found by computing cycle basis.
Based on \autoref{theorem:independencecompatibility} and \autoref{theorem:treecompatibility}, as well as \autoref{corollary:transitiveredundancycompatibility}, this algorithm is able to prove compatibility by removing redundant relations, such that the resulting network is a composition of independent trees.
However, we still need to discuss how redundancy of relations, in terms of redundant predicates in the property graph, can be identified, according to the \textsc{Replace-Hyperedge} method in \autoref{algo:cycleenumeration}.We will discuss that in the subsequent section.

% \todo[inline, color=kit-orange100]{AP: note on cycles complexity + heuristics (or evaluation, or future work?)} 

% \begin{itemize}
%     \item \textbf{Phase III}. Which cycle to opt for? Some cycles are better than others. In particular, having exponential many cycles implies that it is a good idea to test the cycles that are most likely to show redundancy first.
%         \begin{itemize}
%             \item Necessary condition: candidate is in the cycle
%             \item Heuristics (unproven for now): shorter cycles, cycles whose meta-edges share many metamodel elements (information on edges in the dual)
%         \end{itemize}
    % \item \textbf{Pre-processing}. Dual of the metagraph: mostly for convenience in graph processing (balance between expressiveness and usability). Solution: use the dual (equivalent up to isomorphism). Consequences of the dual:
    %     \begin{enumerate}
    %         \item Independent subsets => independent subsets
    %         \item Path problem => cycle problem
    %         \item Edges of the dual: common metamodel elements (useful for heuristics)
    %     \end{enumerate}
    % \item \textbf{Phase I}. Independent subsets of meta-edges (through connected components in the metagraph) => computing connected components in the metagraph or in its dual leads to the same result (develop proof?). Each subset can be processed separately.
    %     \item \textbf{Phase II}. Generation of combinations of meta-edges, i.e. a set of meta-edges to replace another meta-edge. IN THE METAGRAPH: finding combinations amounts to finding a path of meta-edges whose endpoints share vertices with the combination to be replaced. (Problem: many characterizations of path algorithms in hypergraphs and more complex algorithms), hence the dual.
    % \item IN THE DUAL: finding combinations amounts to finding \textit{simple} cycles. We have to generate many of them (all?), not just one to test redundancy. The problem of finding all simple cycles in an undirected graph is called \textit{cycle enumeration}. [Complexity details: (-) number of cycles can be exponential in the number of vertices > no polynomial algorithm; (+) only one suitable combination is necessary; (+) using heuristics]. 
%\end{itemize}

%% simplified syntax FROM https://nmacedo.github.io/pubs/FASE13.pdf
% \begin{center}
%     \begin{tabular}{c}
%         \begin{lstlisting}[numbers=none, mathescape=true]
% [top] relation R {
%     [variable declarations]
%     domain M a : A { $\pi_{M}$ }
%     domain N b : B { $\pi_{N}$ }
%     [when { PRECOND }] [where { INVARIANT }] }
%         \end{lstlisting}
%     \end{tabular}
% \end{center}

\iffalse % COMMENTS
    OUTLINE:
        - 
        
    =============================================
    CONSISTENCY RELATIONS VS CONSISTENCY RULES:
        - Idea that an intermediate representation is necessary:
            - duality global/local, structural/definitions, paths/conditions
            - Between transformation networks and consistency specification
            
    METAGRAPH:
        - Of metamodel elements
        - Is a hypergraph with a labeling
    
    CREATION OF THE METAGRAPH:
        - From QVT-R files
            - Recursive construction, imitation of execution order of relations
            - Merge of relations based on common QVT-R variables (merge algorithm)
            - (?) validation of inputs
            - (-) (no details on MMT implementation)
        
    USE OF THE METAGRAPH:
        - Independent subsets of consistency relations (detection via metagraph)
            - Connected components
        - Dual (mostly for convenience in graph processing)
        - Cycle detection for redundancy test
            - Highlight on reusability (i.e. another strategy "pluggable")
            - (?) Cycle base (Paton)
            - (+) Heuristics (avoid extreme cases for cycle detection)
            
    [ILLUSTRATIONS]
        - One example of metagraph
\fi

% Detailed explanation of the decomposition procedure: different levels of decomposition (see consistency relations / consistency rules, independent connected components of graphs), metagraph creation, dual of the metagraph and, finally, idea to find redundant constraints in the remaining network. We should explicitly not discuss how finding redundant constraint is done in this section: the approach until here is the essential decomposition and in the best case only performing this decomposition already results in a set of independent trees. The strategy to find redundancies should be explicitly separated into the next section. It might be possible to plug in a different strategy to find redundancies. This should clarify the independent reusability of the decomposition and the approach to find redundancies. 

\end{copiedFrom} % SoSym MPM4CPS


\begin{copiedFrom}{SoSym MPM4CPS}

\subsection{Finding Redundancies in Transformations}
\label{chap:prevention:compatibility:redundancies}

\todo{Originally this was a section instead of a subsection, thus it does not fit into the structure anymore}

In the decomposition procedure, enumerating possibly redundant predicates and proving that such predicates are redundant are uncoupled tasks. The latter task can be regarded as a black box embedded in the former one: only the result of the redundancy test matters. As a consequence, the decomposition procedure allows the use of various strategies to prove that a predicate is redundant. There is no perfect strategy due to the limitations on the decidability of OCL expressions. In this article, we opt for a strategy that allows the decomposition procedure to be fully automated. We first discuss how predicates can be compared to prove redundancy. Then, an approach that translates OCL constraints (in predicates) to first-order logic formulae and uses an automated theorem prover is introduced. Finally, the translation rules from OCL to first-order logic are presented along with their limitations.

\subsubsection{Intensional Comparison of Predicates}

Whatever the strategy, the \textit{redundancy test} takes a couple $(E, \tupled{E_1, \dots, E_n})$ as an input and returns \textsc{true} if the predicate $E$ was proven redundant because of the sequence of predicates $\tupled{E_1, \dots, E_n}$, \textsc{false} otherwise. As stated in \autoref{def:leftequalredundancy}, a consistency relation is considered left-equal redundant if its removal from the set of consistency relations leads to an equivalent set and relates the same kinds of elements at the left side. For the relation to be redundant, there must be an alternative sequence of relations that already fulfills the role of the initial relation. This also applies to the property graph: for a predicate to be removed, there must exist another sequence of predicates relating the same properties that is strict enough not to weaken the consistency specification. Weakening the consistency specification means allowing models that would have been considered non-consistent before the removal of the predicate. 
Since we only consider predicates that relate the same properties the additional requirement of left-equal redundancy in comparison to general redundancy in \autoref{def:redundancy} is always fulfilled. In the following, we thus only discuss redundancy rather then left-equal redundancy, as it is always given by construction.
As illustrated in \autoref{fig:correctness:prevention:comparison_validinstances}, a predicate $E$ can only be removed if all instances matching the predicate also match predicates $\tupled{E_1, \dots, E_n}$.

Therefore, a redundancy test is equivalent to the comparison of two sets of instances. However, a predicate may be fulfilled by infinitely many model elements. For example, the predicate ensuring that the income of a person and the salary of an employee are equal is valid for infinitely many integer pairs. It is impossible to compare these sets element per element (i.e., extensionally). 
Since consistency relations in the decomposition procedure are defined intensionally, by means of predicates, anyway, the redundancy test compares sets in their intensional specification.
%Consequently, the redundancy test must compare sets intensionally (i.e., by comparing the way sets are defined). This is a consequence of the fact that consistency relations in the decomposition procedure are also defined intensionally, by means of predicates. 
As a result, the redundancy test uses the description of the possibly redundant predicate and the candidate alternative sequence of predicates to decide whether the predicate is redundant.  
In QVT-R, predicates are expressed as OCL constraints. As part of the construction of the property graph, these constraints are already represented as hyperedge labels. That is, comparing predicate definitions in the decomposition procedure amounts to perform a static analysis of these labels and QVT-R relation conditions (\texttt{when} and \texttt{where} clauses). In order to prove redundancy, the static analysis has to rely on a rigorous framework to reason about OCL constraints. This is provided by various formal methods in the field of software verification.

\begin{figure}
    \centering
    \begin{tikzpicture}
    \draw [kit-blue100,fill=kit-blue30] (0,0) ellipse (3cm and 1.7cm);
    \draw [kit-blue100,fill=kit-blue70] (1.5,0) ellipse (1cm and 0.7cm);
    
    \node[text width=4cm, name=e] at (-2,-2.7) {Models considered consistent according to the predicate $E$};
    \node[text width=4cm, name=ei] at (2,-2.7) {Models considered consistent according to the sequence of predicates $\tupled{E_1, \dots, E_n}$};
    
    \draw[->] (e) -- (-1, 0);
    \draw[->] (ei) -- (1.5, 0);
\end{tikzpicture}
    \caption[Redundancy of a hyperedge]{Redundancy of a hyperedge: the hyperedge $E$ is redundant, because all instances valid according to $E$ are already valid according to the alternative sequence $E_1, \dots, E_n$ of hyperedges. Taken from \owncite{klare2020compatibility-report}.}
    \label{fig:correctness:prevention:comparison_validinstances}
\end{figure}

\subsubsection{Encoding Redundancy in First-Order Formulae}

In this article, the strategy we use to prove redundancy is based on first-order logic, a well-suited and expressive mathematical language for decision procedures. The idea is to set up a first-order formula that is valid (i.e., true under every interpretation) if and only if the redundancy test is positive. To this end, the formula embeds OCL expressions (translated into first-order logic as well) contained in predicates. Then, a theorem prover evaluates the validity of the formula. 
%Theorem provers are software programs able to prove the validity of a first-order sentence.

The choice of first-order logic is motivated by the nature of OCL: there exists a general translation of OCL into first-order logic~\cite{beckert2002ocltranslation}. This result was later refined in~\cite{berardi2005umlreasoning} in order to show that OCL formulae are essentially full first-order formulae. What this means is that OCL does not form a fragment of first-order logic and needs all its expressiveness. First-order logic being undecidable in general, this also implies that not all redundant formulae can be proven valid. Therefore, the results obtained by the decomposition procedure are highly dependent on the performance of the theorem prover. Even with quantifiers and variables, the gap between programming languages and first-order sentences remains significant. OCL is composed of arithmetic operations, strings, arrays, etc. The meaning of language constructs must also be integrated into formulas. To this effect, it is possible to replace binary variables in first-order formulae with a Boolean sentence expressed in a given theory. Theories use all kinds of objects such as strings, floats, sequences, etc. With theories, the satisfiability problem equates to assign values to variables in first-order sentences such that the evaluation of sentences makes the whole formula \textsc{true}. For instance, the formula $(a \times b = 10) \wedge (a + b > 0)$ is satisfiable given the assignment $\{a = 2, b = 5\}$.
This extension is known as \textit{satisfiability modulo theories} (SMT). First-order formulae for the SMT problem are called \textit{SMT instances}. Some theorem provers come with built-in theories, they are thus called \textit{theory-based theorem provers}. In the context of the decomposition procedure, we translate constructs in OCL constraints with corresponding constraints into built-in theories of the prover. By doing so, the mapping between OCL and first-order logic is easier to achieve. % OUTLINE 2 parts

\paragraph{Modeling as a Horn Clause}

For any two models, consistency depends on condition elements in predicate-based consistency relations, which themselves depend entirely on property values for which predicate functions evaluate to \textsc{true}. As a result, redundancy can be tested by comparing descriptions of predicate functions. This information is contained in the input of the redundancy test. Let $\pi = (\propertysettuple{P}{\classtuple{C}{l}}, \propertysettuple{P}{\classtuple{C}{r}}, f_\pi)$ be a predicate for two class tuples $\classtuple{C}{l}$ and $\classtuple{C}{l}$. During the construction of the property graph, a hyperedge composed of all properties in $\propertysettuple{P}{\classtuple{C}{l}}$ and $\propertysettuple{P}{\classtuple{C}{r}}$ is labeled with the description of the predicate function $f_\pi$.

In terms of predicate functions, the predicate $E$ can be replaced by a sequence of predicates $\tupled{E_1, \dots, E_n}$ under the following condition: for any set of models, $f_E$ evaluates to \textsc{true} whenever $f_{E_1} \wedge \dots \wedge f_{E_n}$ evaluates to \textsc{true}. If this condition is met, the consistency specification is neither strengthened nor weakened after the removal of $E$. The specification is not strengthened because the removal of a predicate can only allow more sets of models to be consistent. It is not weakened either because all property values that match $\bigwedge f_{E_i}$ also match $E$. As a consequence, $E$ is redundant. That is, solutions of $\bigwedge f_{E_i}$ form a subset of solutions of $E$. This is consistent with~\autoref{fig:correctness:prevention:comparison_validinstances}. The redundancy test can be encoded as a formula in the following way:
\begin{align*}
    \formulaskip &
    (f_{E_1} \wedge \dots \wedge f_{E_n}) \Rightarrow f_{E}
\end{align*}

The formula above is called a \textit{Horn clause}. Horn clauses form an important fragment of logic in the field of automated reasoning. Terms on the left-hand side of the clause are called \textit{facts}, whereas the term on the right-hand side is called \textit{goal}. The implication represents the deduction of the goal from the facts. The assignment of values to variables in the Horn clause also models the instantiation of properties (i.e., the assignment of property values). If the Horn clause is valid, then the alternative sequence of predicates can replace the initial predicate whatever the instantiation of metamodel elements (i.e., whatever the models).

One last detail is to be taken into consideration for the translation of predicate functions. Horn clauses are usually described without quantifiers. In a Horn clause, all variables are implicitly universally quantified. Given that predicate functions are made up of OCL expressions, they contain local QVT-R variables. Consistency depends upon pattern matching, i.e., the existence of a valid assignment of variables. Therefore, QVT-R variables in the goal clause have to be existentially quantified.

\begin{example}
\autoref{fig:correctness:prevention:dual_propertygraph_re} depicts the dual of the property graph derived from the motivational example in \autoref{fig:compatibility:three_persons_example_extended}. The dual contains four connected components, including three with one predicate only. Compatibility is already proven for these three components because they are trivial trees. The other component is made up of three predicates and contains a cycle ($\{1, 2, 3\}$).
Let $3$ be the possibly redundant predicate. Then, the alternative combination of predicates is composed of $1$ and $2$. This leads to the following formula in which $3$ is the goal and $1$ and $2$ are the facts:
\begin{align*}
    &\left[(\propdisplay{\small Person::firstname} = f1) \wedge (\propdisplay{\small Person::lastname} = l1)\right.\\
    &\left.\formulaskip \wedge (\propdisplay{\small Resident::name} = f1 + ``\text{\textvisiblespace}" + l1)\right]\\
    \wedge & \left[(\propdisplay{\small Person::firstname} = f2) \wedge (\propdisplay{\small Person::lastname} = l2)\right.\\
    &\left.\formulaskip \wedge (\propdisplay{\small Employee::name} = f2 + ``\text{\textvisiblespace}" + l2)\right]\\
    \Rightarrow & \left(\exists n : (\propdisplay{\small Resident::name} = n) \wedge (\propdisplay{\small Employee::name} = n)\right)
\end{align*}
QVT-R variables have been renamed to avoid conflicts. This is necessary because they are no longer isolated as they were before in two distinct QVT-R relations. According to the SMT solver, the formula above is valid. Therefore, predicate $3$ can be removed from the property graph and its dual. There are then only two predicates left in this component. It is inherently compatible. As all independent subsets of predicates are compatible, the consistency specification is compatible.
\end{example}
    
\paragraph{Redundancy Test}

\begin{figure*}
    \centering
    \resizebox{\linewidth}{!}{\begin{tikzpicture}
    %% FRAMES
    % \draw[dotted, anchor=west] (-0.3, -1) rectangle (6, 7.5);
    %\draw[dotted, anchor=west] (7.2, -1) rectangle (12.3, 7.5);

    \node (r1) at (2.85,3.25) [draw, kit-green100, thick, dashed, minimum width=6.3cm,minimum height=8.5cm] {};
    \node (r2) at (9.75,3.25) [draw, kit-green100, thick, dashed, minimum width=5.1cm,minimum height=8.5cm] {};
    \draw[->, ultra thick, kit-green100] (r1) -- (r2);
    
    %% OCL EXPRESSIONS
    \node[name=persEmp, anchor=west, minimum width=5.5cm, text width=5.5cm, align=left] at (0, 5) {%
\begin{embeddedqvtcode}[linewidth=5.5cm, mathescape=true]
Person::firstname = fstn
$\wedge$ Person::lastname = lstn
$\wedge$ Employee::name=fstn + ' ' + lstn
\end{embeddedqvtcode}
        };
        
    \node[name=empRes, anchor=west, minimum width=5.5cm, text width=5.5cm, align=left] at (0, 1.5) {%
\begin{embeddedqvtcode}[linewidth=5.5cm, mathescape=true]
Resident::name = n
$\wedge$ Employee::name = n
\end{embeddedqvtcode}
        };
        
    \node[name=persRes, anchor=west, minimum width=5.5cm, text width=5.5cm, align=left] at (0, 0) {%
\begin{embeddedqvtcode}[linewidth=5.5cm, mathescape=true]
Person::firstname = fstn
$\wedge$ Person::lastname = lstn
$\wedge$ Resident::name=fstn + ' ' + lstn
\end{embeddedqvtcode}
        };
    
    %% LABELS    
    \node[above=0.1cm of persEmp, align=center] (possiblyredundant_label) {\large Possibly\\ \large redundant predicate};
    \node[above=0.1cm of empRes, align=center] (alternative_label) {\large Alternative sequence\\ \large of predicates};
    
    %% SMT FORMULAE
    \node[name=smtPersEmp, anchor=west, right=2cm of persEmp, minimum width=4cm, text width=4cm, align=left] {%
\begin{z3code}[linewidth=4cm]
(not
  (and
    (= firstname fstn)
    (= lastname lstn)
    (= name
      (str.++ fst
        (str.++ ' ' lstn)
      )
    )
  )
)
\end{z3code}
        };
        
    \node[name=smtEmpRes, anchor=west, right=2cm of empRes, minimum width=4cm, text width=4cm, align=left] {%
\begin{z3code}[linewidth=4cm]
...
\end{z3code}
        };
        
    \node[name=smtPersRes, anchor=west, right=2cm of persRes, minimum width=4cm, text width=4cm, align=left] {%
\begin{z3code}[linewidth=4cm]
...
\end{z3code}
        };
        
    \draw[->] (persEmp) to (smtPersEmp);
    \draw[->] (empRes) to (smtEmpRes);
    \draw[->] (persRes) to (smtPersRes);
    
    \node[above=-0.3cm of smtEmpRes] (wedge1) {$\wedge$};
    \node[above=-0.3cm of smtPersRes] (wedge2) {$\wedge$};
        
    %% SMT SOLVER
    \node[draw,rectangle, minimum size=2cm, kit-blue100, fill=kit-blue30, align=center] (smtsolver) at (14.5, 3.15){\large SMT\\\large Solver};
    \draw[->, ultra thick, kit-green100] (r2) -- (smtsolver);
    
    %% OUTCOMES
    \node[text width=3cm] (sat) at (18, 6.5){\textbf{SAT}. The initial Horn clause is not always valid so the predicate is not entirely redundant. \textit{No removal.}};
    \node[text width=3cm] (unknown) at (18, 3.25){\textbf{UNKNWON}. By conservativeness. \textit{No removal.}};
    \node[text width=3cm] (unsat) at (18, 0){\textbf{UNSAT}. The initial Horn clause is valid so the predicate is redundant. \textit{Removal.}};
    
    \draw[->] (smtsolver) -- (sat);
    \draw[->] (smtsolver) -- (unknown);
    \draw[->] (smtsolver) -- (unsat);
    
    \node[below = 0.25cm of r1] {\textbf{OCL Expressions}};
    \node[below = 0.25cm of r2, name=smtf] {\textbf{SMT Formula}};
    \node[right = 4cm of smtf] {\textbf{Redundancy Test}};
\end{tikzpicture}}
    \caption[Redundancy test overview]{Overview of the redundancy test, from OCL expressions to the SMT solver. Taken from \owncite{klare2020compatibility-report}.}
    \label{fig:correctness:prevention:redundancytest}
\end{figure*}

Redundancy can be proven by checking that the Horn clause derived from predicate functions is valid (i.e., true under every interpretation). Because the whole decomposition procedure is automated, the theorem prover, namely an SMT solver, embedded in the procedure is also automated. 
The solver takes an SMT instance as an input and answers whether it is satisfiable insofar as possible.
Proving that a Horn clause $H$ is valid is actually equivalent to proving that its negation $\neg H$ is unsatisfiable. Therefore, we prove that the SMT instance $f_{E_1} \wedge \dots \wedge f_{E_n} \wedge \neg f_{E}$ is unsatisfiable. The SMT solver can provide three possible outcomes:

\begin{description}
    \item[Satisfiable.] If $\neg H$ is satisfiable, then $H$ is not valid. This means that an interpretation exists (i.e., an instantiation of properties) that fulfills the possibly redundant predicate but not the alternative sequence of predicates. Thus, the predicate is not redundant and cannot be removed.
    \item[Unsatisfiable.] If $\neg H$ is unsatisfiable, then $H$ is valid. When the alternative sequence is fulfilled, so is the predicate. It is redundant and can be removed.
    \item[Unknown.] First-order logic being undecidable, not all formulae can be proven valid. When a theorem prover is unable to evaluate the satisfiability of a formula, it returns \textit{Unknown}. By application of the conservativeness principle, the redundancy test is considered negative. As a result, the predicate is not removed.
\end{description}

\subsubsection{Translation}

Translation refers to the process of mapping OCL expressions of QVT-R relations to SMT instances. In the context of the decomposition procedure, this is facilitated by the fact that QVT-R uses a subset of OCL called \textit{EssentialOCL}~\cite{qvt}, a side-effect free sublanguage that provides primitives data types, data structures and operations to express constraints on models. Many constructs of OCL have a direct equivalent in theories of the theorem prover. More complex constructs can often be mapped through the combination of primitive constructs. Note that there also exist constructs that cannot be translated. Language constructs of SMT solvers are described using the SMT-LIB specification, a standard that provides among others an input language for solvers~\cite{smtlib2017}. This language uses a syntax similar to that of Common Lisp. The translation is recursive: each OCL expression depends on the translation of its subexpressions. A complete reference of translated constructs has been developed in a master's thesis~\cite{pepin2019ma}.

\paragraph{Primitive Data Types}

OCL defines five primary data types: integers, reals, booleans, strings and unlimited naturals. These data types are mapped with Ecore when parsing QVT-R transformations. The mapping between Ecore data types and Z3 data types (called \textit{sorts}) is described in \autoref{tab:primitivedatatypes}. It is straightforward, except for \textit{UnlimitedNatural}, a data type to represent multiplicities. \textit{UnlimitedNatural} and \textit{Integer} are different in that the former can take an infinite value whereas the latter cannot. IntSort in Z3 cannot be infinite. In this case, a workaround is to represent an \textit{UnlimitedNatural} as a couple (IntSort, BoolSort) where the value equals $\infty$ if the Boolean is \textsc{true}. %, or equal to the integer otherwise.

\begin{table}
\renewcommand{\arraystretch}{1.2}%
\setlength\tabcolsep{2 pt}
\begin{tabular}{L{2.4cm} L{2.4cm} L{3.0cm}}
\toprule
\textbf{OCL Data Type} & \textbf{Ecore Data Type} & \textbf{SMT Data Type} \\
\midrule
    Integer & \texttt{EInt} & IntSort \\
    Real    & \texttt{EDouble} & RealSort \\
    Boolean & \texttt{EBoolean} & BoolSort \\
    String  & \texttt{Estring} & StringSort \\
    UnlimitedNatural & \texttt{EInt} & IntSort \textit{(without infinity)}\\
\bottomrule
\end{tabular}
\caption[Mapping between primitive type representations]{Mapping between primitive types representations. Taken from \owncite{klare2020compatibility-report}.} % in OCL, Ecore and Z3}
\label{tab:primitivedatatypes}
\end{table}

\paragraph{Data Structures}
Primitive data structures in OCL are called \textit{collections}. There are four types of collections based on the combination of two features, the first defining whether elements are ordered and the second whether duplicate elements are allowed. Among those four collection types, two are currently supported: sequences (ordered, duplicates allowed) and sets (non-ordered, no duplicates). In QVT-R, collections are used either as literals or as types for a special kind of properties: role names.

\subparagraph{Collection Literals}
Collection literals are OCL expressions that represent data structures with constant and predefined values (e.g., \lstinline|Sequence{1, 4, 9}| or \lstinline|Set{2, 5}|). In SMT solving, the fundamental theory to represent a collection of values is the theory of \textit{arrays}. Arrays are maps that relate a set of indexes (domain) and a set of values (codomain). They are immutable, purely functional data structures. Unlike OCL data structures, there is no notion of size in arrays. To overcome this limitation, we translate collections to algebraic data types in the SMT input language. Data types are composed of an array (collection values) and an integer storing the collection size. It is noteworthy that collection literals rarely occur in consistency relations. In general, collections are groups of objects resulting from references in metamodels.

\subparagraph{Collections from Role Names}
In QVT-R property template items, properties are either attributes (like \propdisplay{Person::name}) or role names. A role name is an alias for objects at the end of the reference owned by the class of the pattern. If the upper bound of the reference multiplicity is greater than one (e.g., \texttt{0..*}), then the role name may represent a collection of objects. The nature of the collection depends on whether the end is ordered or unique or both. Even if the content of the collection is unknown, it is possible to reason about role names by means of the theory of \textit{uninterpreted functions}~(UF). A role name $r$ of a class $c$ can be represented as a function of $c$ (e.g., $r(c)$). By abstracting the semantics of functions, uninterpreted functions help to reason about model elements without knowing all their details. For example, two role names are equal if both belong to objects that have been proven to be equal themselves.

\paragraph{Operations}

OCL also provides many operations on primitive data types and data structures, such as arithmetic operations or string operations. Following the object-oriented structure of OCL, every operation has a source and zero or more arguments. For example, the \texttt{+} operation denotes an addition when the source is an integer but a concatenation when the source is a string.
We translated operations regarding arithmetics, booleans, conversion operators, equality operators, order relations, collections and strings~\cite{pepin2019ma}.

%\paragraph{Untranslatable Operations}
Some OCL operations are said to be \textit{untranslatable}, because it is impossible to find a mapping between them and features of state-of-the-art SMT solvers. As a result, there are QVT-R relations that cannot be processed by the decomposition procedure. For instance, the string operations \lstinline{toLower} and \lstinline{toUpper} cannot be easily translated without numerous user-defined axioms in current SMT solvers. Although decision procedures for such a case exist~\cite{veanes2012transducers}, they are not yet integrated into solvers.


\subsubsection{Summary}
In this section, we have presented an approach to evaluate redundancy of a predicate in a property graph (respectively its dual) for the decomposition procedure, also depicted in \autoref{fig:correctness:prevention:redundancytest}.
The approach translates the OCL expressions of predicates into logic formulae and generates Horn clauses for a potentially redundant predicate and its alternative predicates.
If an SMT solver proves unsatisfiability of that clause, the checked predicate is redundant and can be removed.

\end{copiedFrom} % SoSym MPM4CPS