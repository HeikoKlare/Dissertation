\subsection{Redundancy as Witness for Compatibility}
\label{sec:formalapproach:redundancy}

\todoDiss{Add a definition for a \emph{compatibility-preserving consistency relation} that states that is preserved compatibility and use that for clarifying the following lemmas and theorems.}

%%
%% Problem: Not having a compatible structure, compatibility is unclear
%%
We have introduced specific structures of consistency relations that are inherently compatible.
If a given set of consistency relations does not represent one of those structures, especially because there are multiple consistency relations putting the same classes into relation, it is unclear whether such a set is compatible.

%%
%% Idea: Find and virtually remove redundant relations
%%
In the following, we present an approach to reduce a set of consistency relations to a structure of independent consistency relation trees.
The essential idea is to find relations within the set, which do not change compatibility of the consistency relation set whether or not they are contained in it.
An approach that finds such relations and---virtually---removes them from the set until the remaining relations form a set of independent consistency relation trees, proves compatibility of the given set of relations.
We first define the term of a \emph{compatibility-preserving} relation.

\begin{definition}
    \label{def:compatibilitypreserving}
    Let $\consistencyrelationset{CR}$ be a compatible set of consistency relations and let $\consistencyrelation{CR}{}$ be a consistency relation. We say that:
    \begin{align*}
        \formulaskip &
        \consistencyrelation{CR}{} \compatibilitypreservingtomath \consistencyrelationset{CR} \equivalentperdefinition \\
        & \formulaskip
        \consistencyrelationset{CR} \cup \setted{\consistencyrelation{CR}{}} \compatiblemath
    \end{align*}
\end{definition}

To be able to find such a compatibility-preserving relation, we introduce the notion of \emph{redundant} relations and prove the property of being compatibility preserving.
Informally speaking, a relation is redundant if it is expressed transitively across others, i.e., if it does not restrict or relax consistency compared to a combination of other relations.
We precisely specify a notion of redundancy in the following.

% \begin{itemize}
%     \item There may also be cycles in the fine-grained relations, such that the relation graph induced by the specification cannot witness compatibility.
%     \item If it is possible to find a set of trees that is equivalent to the given graph (\formalize{what equivalent means here!}), this serves as a witness for compatibility.
%     \item Finding such an equivalent representation can be achieved by (virtually) removing relations that have no impact on the valid instances (i.e. do not reduce the degree of consistency) (\formalize{that with the definition})
%     \item A relation is redundant if it is transitively expressed across the others, i.e. if it does not restrict the valid instances of two metamodels that are considered consistent in addition to those allowed by all other relations (\formalize{with extensional definitions of consistency}).
%     \item Virtually removing such relations leads to an equivalent representation and if that representation finally forms a tree, we have a witness for compatibility.
%     \item We explain this in detail in \autoref{sec:redundancies} and discuss how theorem proving can be used to find such redundant relations.
% \end{itemize}

\begin{definition}[Redundant Consistency Relation]
\label{def:redundancy}
    Let $\consistencyrelationset{CR}$ be a set of consistency relations for a set of metamodels $\metamodelset{M}$. %  = \setted{\metamodel{M}{1}, \dots, \metamodel{M}{k}}$.
    For a consistency relation $\consistencyrelation{CR}{} \in \consistencyrelationset{CR}$, we say that:
    \begin{align*}
        \formulaskip &
        \consistencyrelation{CR}{} \redundantinmath \consistencyrelationset{CR} \equivalentperdefinition\\
        & \formulaskip 
        \exists \consistencyrelation{CR'}{} \in \transitiveclosure{(\consistencyrelationset{CR} \setminus \setted{\consistencyrelation{CR}{}})} : 
        \forall \modelset{m} \in \metamodelinstances{\metamodelset{M}} :\\
        & \formulaskip\formulaskip
        \modelset{m} \consistenttomath \consistencyrelation{CR'}{} \Rightarrow \modelset{m} \consistenttomath \consistencyrelation{CR}{}
    \end{align*}
    % \begin{align*}
    %     \formulaskip &
    %     \consistencyrelation{CR}{} \in \consistencyrelationset{CR} \mathtext{is redundant in} \consistencyrelationset{CR} \equivalentperdefinition\\
    %     & \formulaskip 
    %     \consistencyrelationset{CR} \mathtext{equivalent to} \consistencyrelationset{CR} \setminus \setted{\consistencyrelation{CR}{}}
    % \end{align*}
    % \begin{align*}
    %     \formulaskip &
    %     \consistencyrelation{CR}{} \in \consistencyrelationset{CR} \mathtext{is redundant in} \consistencyrelationset{CR} \equivalentperdefinition\\
    %     %& \formulaskip
    %     %\forall \modelset{m} = \setted{\model{m}{1}, \dots, \model{m}{k}} \mid \model{m}{i} \in \metamodelinstances{\metamodel{M}{i}} : \\
    %     & \formulaskip 
    %     \exists \consistencyrelation{CR'}{} \in \transitiveclosure{\consistencyrelationset{CR}} : \consistencyrelation{CR'}{} \mathtext{overlapping with} \consistencyrelation{CR}{} \\
    %     & \formulaskip
    %     \land \forall \consistencyrelation{CR'}{} \in \transitiveclosure{\consistencyrelationset{CR}} \mid \consistencyrelation{CR'}{} \mathtext{overlapping with} \consistencyrelation{CR}{} :\\
    %     & \formulaskip\formulaskip
    %     \bigl(\forall \tupled{\conditionelement{c}{l}, \conditionelement{c}{r}} \in \consistencyrelation{CR}{} : \exists \tupled{\conditionelement{c'}{l}, \conditionelement{c'}{r}} \in \consistencyrelation{CR'}{} : \forall \modelset{m} \in \metamodelinstances{\metamodelset{M}} : \\
    %     & \formulaskip\formulaskip\formulaskip\modelset{m} \mathtext{contains} \tupled{\conditionelement{c}{l}, \conditionelement{c}{r}} \equivalent \modelset{m} \mathtext{contains} \tupled{\conditionelement{c'}{l}, \conditionelement{c'}{r}} \\
    %     & \formulaskip\formulaskip
    %     \land \forall \tupled{\conditionelement{c'}{l}, \conditionelement{c'}{r}} \in \consistencyrelation{CR'}{} : \exists \tupled{\conditionelement{c}{l}, \conditionelement{c}{r}} \in \consistencyrelation{CR}{} : \forall \modelset{m} \in \metamodelinstances{\metamodelset{M}} : \\
    %     & \formulaskip\formulaskip\formulaskip\modelset{m} \mathtext{contains} \tupled{\conditionelement{c}{l}, \conditionelement{c}{r}} \equivalent \modelset{m} \mathtext{contains} \tupled{\conditionelement{c'}{l}, \conditionelement{c'}{r}} \bigr)
    % \end{align*}
\end{definition}

\todo{Add examples for redundancy! How do the elements of the redundant relation have to be related to the ones in $\consistencyrelation{CR'}{}$?}
\todoDiss{Can we define an even more general notion of redundancy, not stating about the relation to a single consistency relation but the set of consistency relation, abstracting the implication to consistency to the whole set of relations?}
The definition of redundancy of a consistency relation $\consistencyrelation{CR}{}$ ensures that there is another consistency relation, possibly transitively expressed across others, such that if a model is consistent to that other relation, it is also consistent to $\consistencyrelation{CR}{}$.
This means that there are no model sets that are considered inconsistent to $\consistencyrelation{CR}{}$, but not to another relation, thus $\consistencyrelation{CR}{}$ does not restrict consistency.
Actually, the definition of redundancy implies that the set of consistency relations with and without the redundant one are equivalent according to \autoref{def:equivalence}, thus both consider the same model sets as consistent.

% The definition of redundancy ensures that the redundant relation does not provide any relaxation (first equivalence) or restriction (second equivalence) regarding existing consistency relations and that there is at least one overlapping consistency relation, as otherwise the relation will always restrict consistent models.
% Intuitively, redundancy could also be defined by requiring equivalence of the set of consistency relations with and without the redundant relation. In that case, exactly the same models would be considered consistent with and without the redundant relation.
% However, we want to use the redundancy definition to make statements about compatibility of sets of consistency relations, which requires this more restricted notion of redundancy.
% Actually, the definition of redundancy always implies equivalence.
\todoDiss{Explain that we do not require equality of elements in CR and CR' because, e.g., CR might only related names, whereas CR' related names and addresses, thus we only require that there are elements that are co-indicating consistency.}

\begin{lemma} \label{lemma:redundancyimpliesequivalence}
    Let $\consistencyrelation{CR}{} \in \consistencyrelationset{CR}$ be a redundant consistency relation in a relation set $\consistencyrelationset{CR}$. %, according to \autoref{def:redundancy}.
    Then $\consistencyrelationset{CR}$ is equivalent to $\consistencyrelationset{CR} \setminus \setted{\consistencyrelation{CR}{}}$. %, according to \autoref{def:equivalence}.
\end{lemma}

\begin{proof}
    Like discussed in \autoref{lemma:consistencytransitiveclosure}, adding a consistency relation to a set of consistency relations can never lead to a relaxation of consistency, i.e., models becoming consistent that were not considered consistent before. This is a direct consequence of \autoref{def:consistency} for consistency, which requires models be consistent to all consistency relations in a set to be considered consistent, thus restricting the set of consistent model sets by adding further consistency relations.
    In consequence, it holds that:
    \begin{align*}
        \formulaskip
        \modelset{m} \consistenttomath \consistencyrelationset{CR} \Rightarrow 
        \modelset{m} \consistenttomath \consistencyrelationset{CR} \setminus \setted{\consistencyrelation{CR}{}}
    \end{align*}
    Additionally, a direct consequence of \autoref{def:redundancy} for redundancy is that a redundant consistency relation does not restrict consistency, as it considers all models to be consistent that are also considered consistent to another consistency relation in the transitive closure of the consistency relation set. Thus, all models that are considered consistent to the transitive closure of $\consistencyrelationset{CR} \setminus \setted{\consistencyrelation{CR}{}}$ are also consistent to $\consistencyrelation{CR}{}$ and thus to all relations in $\consistencyrelationset{CR}$:
    \begin{align*}
        \formulaskip
        \modelset{m} \consistenttomath \transitiveclosure{(\consistencyrelationset{CR} \setminus \setted{\consistencyrelation{CR}{}})} \Rightarrow 
        \modelset{m} \consistenttomath \consistencyrelationset{CR}
    \end{align*}
    According to \autoref{lemma:consistencytransitiveclosure}, each set of models that is consistent to a consistency relation set is also consistent to its transitive closure an vice versa.
    In consequence, the previous implication is also true for $\consistencyrelationset{CR} \setminus \setted{\consistencyrelation{CR}{}}$ rather than $\transitiveclosure{(\consistencyrelationset{CR} \setminus \setted{\consistencyrelation{CR}{}})}$.
    Summarizing, $\consistencyrelationset{CR}$ and $\consistencyrelationset{CR} \setminus \setted{\consistencyrelation{CR}{}}$ are equivalent.
\end{proof}

\todoDiss{Possibly add that lemma}
% \begin{lemma}
%     Let $\consistencyrelationset{CR}$ be a set of consistency relation and let $\consistencyrelation{CR}{}$ be redundant in $\consistencyrelationset{CR}$.
%     Then it holds that:
%     \begin{align*}
%         \formulaskip &
%         \exists \consistencyrelation{CR'}{} \in \transitiveclosure{\consistencyrelationset{CR}} \setminus \setted{\consistencyrelation{CR}{}} : \consistencyrelation{CR}{} \mathtext{related to} \consistencyrelation{CR'}{}\\
%         &
%         \lor \forall \modelset{m} \in \metamodelinstances{\metamodelset{M}} : \modelset{m} \mathtext{consistent to} \consistencyrelation{CR}{}
%     \end{align*}
% \end{lemma}

% \begin{proof}
%     tba \todoHeiko{Add the proof}
% \end{proof}

\begin{figure}
    \centering
    \newcommand{\hdistance}{19em}
\newcommand{\vdistance}{1.5em}
\newcommand{\classwidth}{6em}

\begin{tikzpicture}

% Resident
\umlclassvarwidth{resident}{}{Resident\sameheight}{
name
}{\classwidth}

% Employee
\umlclassvarwidth[, right=\hdistance of resident.north, anchor=north]{employee}{}{Employee\sameheight}{
name
}{\classwidth}

% Location
\umlclassvarwidth[, below=\vdistance of resident.south, anchor=north]{location}{}{Location\sameheight}{
street
}{\classwidth}

% Address
\umlclassvarwidth[, below=\vdistance of employee.south, anchor=north]{address}{}{Address\sameheight}{
street
}{\classwidth}

% CONSISTENCY RELATIONS
\draw[consistency relation] ([yshift=1em]resident.east) -- node[pos=0, above right] {$r$} node[pos=0.5, above] {$\consistencyrelation{CR}{1}$} node[pos=1, above left] {$e$} ([yshift=1em]employee.west);
\draw[consistency relation, -] ([yshift=1em]$(employee.west)!0.8!(employee.west-|resident.east)$) |- node[pos=1, above right] {$l$} (location.east);
\draw[consistency relation 2] ([yshift=-1em]resident.east) -- node[pos=0, below right] {$r$} node[pos=0.5, below] {$\consistencyrelation{CR}{2}$} node[pos=1, below left] {$e$} ([yshift=-1em]employee.west);
\draw[consistency relation 2] ([yshift=-1em]$(employee.west)!0.2!(employee.west-|resident.east)$) |- node[pos=1, above left] {$a$} (address.west);

\node[consistency related element, below=2.5em of location.west, anchor=north west] (relation1) {
$\begin{aligned}
    \consistencyrelation{CR}{1} =\; & \setted{\tupled{(r,l),e} \mid r.name  \neq "" \\
    &\land (r.name = e.name \lor r.name = e.name.toLower)}
\end{aligned}$
};   
\node[consistency related element 2, below=0.3em of relation1.south west, anchor=north west] { 
$\begin{aligned}
    \consistencyrelation{CR}{2} =\; & \setted{\tupled{r,(e,a)} \mid r.name = e.name \land a.street \neq ""}
\end{aligned}$
};

\end{tikzpicture}
    %\includegraphics[width=\columnwidth]{figures/redundancy_relation_extremes.png}
    \caption{Redundant consistency relation $\consistencyrelation{CR}{1}$ in $\setted{\consistencyrelation{CR}{1}, \consistencyrelation{CR}{2}}$}
    \label{fig:correctness:formal:redundancyrelationextremes}
\end{figure}

In general, to consider a consistency relation redundant in %a set of consistency relation
$\consistencyrelationset{CR}$, it has to define equal or weaker requirements for consistency than one of the other relations in $\consistencyrelationset{CR}$.
Informally speaking, such weaker requirements mean that the redundant relation must have weaker conditions, i.e., it must require consistency for less objects and consider the same or more objects consistent to each of the left condition elements. %, i.e., it must have a weaker left condition, and consider the same or more elements consistent to those of the left condition.

\begin{example}
Such weaker consistency requirements are exemplified in \autoref{fig:correctness:formal:redundancyrelationextremes}, which shows a consistency relation $\consistencyrelation{CR}{1}$ that is redundant in $\setted{\consistencyrelation{CR}{1}, \consistencyrelation{CR}{2}}$.
A redundant consistency relation, such as $\consistencyrelation{CR}{1}$, must have weaker requirements in the left condition, such that it requires consistent elements to exist in less cases.
This means that it may have a larger set of classes that are matched and that there may be less condition elements for which consistency is required.
In case of $\consistencyrelation{CR}{1}$, the left condition contains both a resident and a location, whereas the left condition of $\consistencyrelation{CR}{2}$ only contains residents.
Thus $\consistencyrelation{CR}{1}$ requires consistent elements, i.e., employees, only if a resident and a location exists, whereas $\consistencyrelation{CR}{2}$ requires that already for an existing resident.
Furthermore, the residents for which $\consistencyrelation{CR}{1}$ defines any consistency requirements are a subset of those for which $\consistencyrelation{CR}{2}$ defines consistency requirements, as $\consistencyrelation{CR}{1}$ does not make any statements about residents having an empty name.
Thus, the left condition elements of $\consistencyrelation{CR}{1}$ are a subset of those of $\consistencyrelation{CR}{2}$.
In consequence, if $\consistencyrelation{CR}{1}$ requires consistency for a resident and a location, $\consistencyrelation{CR}{2}$ requires it anyway, because it already defines consistency for the contained resident.

Additionally, a redundant consistency relation, such as $\consistencyrelation{CR}{1}$, must have weaker requirements for the elements at the right side, such that one of the consistent right condition elements is contained anyway because another relation already required them. 
This means that the relation may have a smaller set of classes, of whom instances are required to consider the models consistent, and there may be more condition elements of the right side that are considered consistent with condition elements of the left side to not restrict the elements considered consistent.
In case of $\consistencyrelation{CR}{1}$, it only requires an emploee to exist for a resident compared to $\consistencyrelation{CR}{2}$, which also requires a non-empty address to exist. Additionally, $\consistencyrelation{CR}{1}$ does not restrict the employees that are considered consistent to employees compared to $\consistencyrelation{CR}{2}$, as it also considers employees with the same name as consistent, but additionally those having the name of the resident in lowercase.
\end{example}

\todoDiss{Add proposition about redundancy properties}
% These informal insights on the properties of a redundant consistency relation can be formalized as follows.

% \begin{proposition}
%     Let $\consistencyrelationset{CR}$ be a set of consistency relations and let $\consistencyrelation{CR}{}$ be a consistency relation. Then it holds that:
%     \begin{align*}
%         \formulaskip &
%         \consistencyrelation{CR}{} \redundantinmath \consistencyrelationset{CR} \cup \setted{\consistencyrelation{CR}{}} \equivalent \\
%         & \formulaskip
%         \exists \consistencyrelation{CR'}{} \in \consistencyrelationset{CR} : %\\
%         %& \formulaskip
%         \classtuple{C}{l,\consistencyrelation{CR'}{}} \subseteq \classtuple{C}{l,\consistencyrelation{CR}{}} \land
%         \classtuple{C}{r,\consistencyrelation{CR}{}} \subseteq \classtuple{C}{r,\consistencyrelation{CR'}{}} \\
%         & \formulaskip
%         \land \forall \conditionelement{c}{l} \in \condition{c}{l,\consistencyrelation{CR}{}} : \exists \conditionelement{c'}{l} \in \condition{c}{l,\consistencyrelation{CR'}{}} : \bigl( 
%         \conditionelement{c'}{l} \subseteq \conditionelement{c}{l} \\
%         & \formulaskip\formulaskip
%         \land \forall \tupled{\conditionelement{c'}{l},\conditionelement{c'}{r}} \in \consistencyrelation{CR'}{} : \exists \tupled{\conditionelement{c}{l},\conditionelement{c}{r}} \in \consistencyrelation{CR}{} : \conditionelement{c}{r} \subseteq \conditionelement{c'}{r} \big)
%     \end{align*}
% \end{proposition}

% \begin{proof}
%     tba
% \end{proof}


\begin{figure}
    \centering
    \newcommand{\hdistance}{19em}
\newcommand{\vdistance}{2em}
\newcommand{\classwidth}{6em}

\begin{tikzpicture}

% Employee
\umlclassvarwidth{employee}{}{Employee\sameheight}{
name
}{\classwidth}

% Person
\umlclassvarwidth[, right=\hdistance of employee.north, anchor=north]{person}{}{Person\sameheight}{
name
}{\classwidth}

% Resident
\umlclassvarwidth[, below=\vdistance of employee.south, anchor=north]{resident}{}{Resident\sameheight}{
name
}{\classwidth}


% CONSISTENCY RELATIONS
\draw[consistency relation] ([yshift=1em]employee.east) -- node[pos=0, above right] {$e$} node[pos=0.5, above] {$\consistencyrelation{CR}{1}$} node[pos=1, above left] {$p$} ([yshift=1em]person.west);
\draw[consistency relation, -] ([yshift=1em]$(person.west)!0.8!(person.west-|employee.east)$) |- node[pos=1, above right] {$l$} ([yshift=1em]resident.east);

\draw[consistency relation 2] ([yshift=-1em]employee.east) -- node[pos=0, below right] {$e$} node[pos=0.5, below] {$\consistencyrelation{CR}{2}$} node[pos=1, below left] {$p$} ([yshift=-1em]person.west);

\draw[consistency relation 2] (person.south) |- node[pos=0, below right] {$p$} node[pos=0.6, above] {$\consistencyrelation{CR}{3}$} node[pos=1, below right] {$r$} ([yshift=-1em]resident.east);

\node[consistency related element, below left=2.5em and 1em of resident.west, anchor=north west] (relation1) {
$\begin{aligned}
    \consistencyrelation{CR}{1} =\; & \setted{\tupled{(e,r),p} \mid e.name = r.name.toUpper \land e.name  = p.name}
\end{aligned}$
};   
\node[consistency related element 2, below=0.3em of relation1.south west, anchor=north west] { 
$\begin{aligned}
    \consistencyrelation{CR}{2} =\; & \setted{\tupled{e,p} \mid e.name = p.name}\\
    \consistencyrelation{CR}{3} =\; & \setted{\tupled{p,r} \mid r.name = p.name.toLower}
\end{aligned}$
};

\end{tikzpicture}
    %\includegraphics[width=\columnwidth]{figures/redundancy_compatibility_counterexample.png}
    \caption{A consistency relation $\consistencyrelation{CR}{1}$ being redundant in  $\setted{\consistencyrelation{CR}{1}, \consistencyrelation{CR}{2}, \consistencyrelation{CR}{3}}$, with $\setted{\consistencyrelation{CR}{2}, \consistencyrelation{CR}{3}}$ being compatible and $\setted{\consistencyrelation{CR}{1}, \consistencyrelation{CR}{2},\consistencyrelation{CR}{3}}$ being incompatible.}
    \label{fig:correctness:formal:redundancy_compatibility_counterexample}
\end{figure}

Our goal is to have a compatibility-preserving notion of redundancy, i.e., adding a redundant relation to a compatible relation set should preserve compatibility.
Unfortunately, our intuitive redundancy definition is not compatibility-preserving. % with our intuitive notion of redundancy, a consistency relation $\consistencyrelation{CR}{}$ that is redundant to a compatible set of consistency relations $\consistencyrelationset{CR}$ does not imply that $\consistencyrelationset{CR} \cup \setted{\consistencyrelation{CR}{}}$ is compatible.

\begin{proposition} \label{prop:redundantnotimpliescompatible}
    Let $\consistencyrelationset{CR}$ be a compatible set of consistency relations and let $\consistencyrelation{CR}{}$ be a consistency relation that is redundant in $\consistencyrelationset{CR} \cup \setted{\consistencyrelation{CR}{}}$.
    Then $\consistencyrelation{CR}{}$ is not necessarily compatibility-preserving, i.e., $\consistencyrelationset{CR} \cup \setted{\consistencyrelation{CR}{}}$ is not necessarily compatible.
    % it holds that:
    % \begin{align*}
    %     \formulaskip &
    %     \consistencyrelationset{CR} \compatiblemath \not\Rightarrow \consistencyrelationset{CR} \cup \setted{\consistencyrelation{CR}{}} \compatiblemath
    % \end{align*}
\end{proposition}

\begin{proof}
We prove the proposition by providing a counterexample. % for the implication.
Consider the example in \autoref{fig:correctness:formal:redundancy_compatibility_counterexample}. 
$\consistencyrelation{CR}{2}$ relates each employee to a person with the same name and $\consistencyrelation{CR}{3}$ relates each person to a resident with the same name in lowercase.
The consistency relation set $\setted{\consistencyrelation{CR}{2}, \consistencyrelation{CR}{3}}$ is obviously compatible, because for each employee and each person, which constitute the left condition elements of the consistency relations, a consistent model set containing the person respectively employee can be created by adding the appropriate person or employee with the same name and a resident with the name in lowercase.
Furthermore, $\consistencyrelation{CR}{1}$ is redundant in $\setted{\consistencyrelation{CR}{1}, \consistencyrelation{CR}{2}, \consistencyrelation{CR}{3}}$ according to \autoref{def:redundancy}, because if a model is consistent to $\consistencyrelation{CR}{2}$ it is also consistent to $\consistencyrelation{CR}{1}$, since $\consistencyrelation{CR}{1}$ also requires persons with the same name as an employee to be contained in a model set but in less cases, precisely only those in which the model also contains a resident such that the employee name is the one of the resident in uppercase.

However, $\setted{\consistencyrelation{CR}{1}, \consistencyrelation{CR}{2}, \consistencyrelation{CR}{3}}$ is not compatible.
Intuitively, this is due to the fact that $\consistencyrelation{CR}{1}$ and $\consistencyrelation{CR}{3}$ define an incompatible mapping between the names of residents and persons.
This is also reflected by \autoref{def:compatibility} for compatibility. Take a model with an employee and a resident named $A$. This is a condition element in $\condition{c}{l,\consistencyrelation{CR}{1}}$. 
Consequentially, $\consistencyrelation{CR}{1}$ requires a person $A$ to exist. Furthermore $\consistencyrelation{CR}{3}$ requires a resident with name $a$ to exist.
In consequence, there are two tuples of employees and residents, both with employee $A$ and one with resident $A$ respectively resident $a$ each, for which a consistent person with name $A$ is required by $\consistencyrelation{CR}{1}$.
However, $\consistencyrelation{CR}{1}$ actually forbids to have two residents, one having the lowercase name of the other, because both are condition elements in $\consistencyrelation{CR}{1}$ requiring an appropriate person to occur in a consistent model, but there is only one person that to which both can be mapped, namely the one with the uppercase name, so there is no witness structure with a unique mapping as required by \autoref{def:consistency} for consistency.
This example shows that adding a redundant consistency relation to a compatible set of consistency relations does not lead to a compatible consistency relation set.
\end{proof}

In consequence of \autoref{prop:redundantnotimpliescompatible}, we need a stronger definition of redundancy that is compatibility-preserving. 
%to be able to derive compatibility for a consistency relation set from adding a redundant relation to an already compatible consistency relation set.
In the example in \autoref{fig:correctness:formal:redundancy_compatibility_counterexample} showing \autoref{prop:redundantnotimpliescompatible}, we have seen that it is problematic if a redundant consistency relation considers more classes in its left condition than the relation it is redundant to.
Therefore, we restrict the left class tuple.

\begin{definition}[Left-equal Redundant Consistency Relation] \label{def:leftequalredundancy}
    Let $\consistencyrelationset{CR}$ be a set of consistency relations for a metamodel set $\metamodelset{M}$.
    For a consistency relation $\consistencyrelation{CR}{} \in \consistencyrelationset{CR}$, we say:
    \begin{align*}
        \formulaskip &
        \consistencyrelation{CR}{} \leftequalredundantinmath \consistencyrelationset{CR} \equivalentperdefinition \\
        & \formulaskip 
        \exists \consistencyrelation{CR'}{} \in \transitiveclosure{(\consistencyrelationset{CR} \setminus \setted{\consistencyrelation{CR}{}})} : 
        \forall \modelset{m} \in \metamodelinstances{\metamodelset{M}} :\\
        & \formulaskip\formulaskip
        \modelset{m} \consistenttomath \consistencyrelation{CR'}{} \Rightarrow \modelset{m} \consistenttomath \consistencyrelation{CR}{} \\
        & \formulaskip\formulaskip
        \land \classtuple{C}{l,\consistencyrelation{CR}{}} = \classtuple{C}{l,\consistencyrelation{CR'}{}}
        %\consistencyrelation{CR}{} \redundantinmath \consistencyrelationset{CR} \\
        %& \formulaskip
        %\land \exists \consistencyrelation{CR'}{} \in \transitiveclosure{(\consistencyrelationset{CR} \setminus \setted{\consistencyrelation{CR}{}})} : 
        %\condition{c}{l,\consistencyrelation{CR}{}} \subseteq \condition{c}{l,\consistencyrelation{CR'}{}} 
        %\forall \conditionelement{c}{l} \in \condition{c}{l,\consistencyrelation{CR}{}} :
        %\exists \conditionelement{c'}{l} \in \condition{c}{l, \consistenyrelation{CR'}{}} :
        %\forall \modelset{m} \in \metamodelinstances{\metamodelset{M}} :
    \end{align*}
\end{definition}

%The definition of left-equal redundancy restricts the notion of redundancy to cases in which the left side of the redundant consistency relation $\consistencyrelation{CR}{}$ considers instances of the same classes as another relation in the set of consistency relations.
%As discussed before, redundancy in general allows that the left side of a redundant consistency relation $\consistencyrelation{CR}{}$ considers more classes than another relation in the set of consistency relations that induces consistency of a model set to $\consistencyrelation{CR}{}$, according to the definition of redundancy.

The definition of left-equal redundancy is similar to the redundancy definition but restricts the notion of redundancy to cases in which the left condition of the redundant consistency relation $\consistencyrelation{CR}{}$ considers the same classes than the other relation in the set of consistency relations that induces consistency of a model set to $\consistencyrelation{CR}{}$.
As discussed before, redundancy in general allows that the left condition of a redundant consistency relation can consider a superset of those classes. %than another relation in the set of consistency relations that induces consistency of a model set to $\consistencyrelation{CR}{}$, according to the definition of redundancy.

\begin{lemma} \label{lemma:leftequalredundancyimpliesredundancy}
    Let $\consistencyrelation{CR}{}$ be a consistency relation that is left-equal redundant in a set of consistency relations $\consistencyrelationset{CR}$. Then $\consistencyrelation{CR}{}$ is redundant in $\consistencyrelationset{CR}$.
\end{lemma}

\begin{proof}
    Since the definition of left-equal redundancy is equal to the one for redundancy, apart from the additional restriction for the class tuples, redundancy of a left-equal redundant relation is a direct implication of the definition.
\end{proof}


Before showing that left-equal redundancy is compatibility-preserving, we introduce an auxiliary lemma that shows that if a model set contains any left condition element of a left-equal redundant relation, i.e., if that redundant relation requires the model set to contain corresponding elements for that object tuple to be consistent, there is also another relation that requires corresponding elements for that object tuple.

%In the following, we will show that the notion of left-equal redundancy, in comparison to the weaker general redundancy property, can be used to inductively prove compatibility of a set of consistency relations.

\begin{lemma} \label{lemma:leftequalredundancysubset}
    Let $\consistencyrelation{CR}{}$ be a consistency relation that is left-equal redundant in a set of consistency relations $\consistencyrelationset{CR}$ for a set of metamodels $\metamodelset{M}$. Then it holds that: 
    \begin{align*}
        \formulaskip &
        \exists \consistencyrelation{CR'}{} \in \transitiveclosure{(\consistencyrelationset{CR} \setminus \setted{\consistencyrelation{CR}{}})} : 
        \forall \conditionelement{c}{l} \in \condition{c}{l, \consistencyrelation{CR}{}} : 
        \exists \conditionelement{c'}{l} \in \condition{c}{l,\consistencyrelation{CR'}{}} : \\
        & \formulaskip
        \forall \modelset{m} \in \metamodelinstances{\metamodelset{M}} : 
        \modelset{m} \containsmath \conditionelement{c'}{l} \Rightarrow 
        \modelset{m} \containsmath \conditionelement{c}{l}
    \end{align*}
\end{lemma}

\begin{proof}
    Due to left-equal redundancy of $\consistencyrelation{CR}{}$ in $\consistencyrelationset{CR}$, we know per definition that:
    \begin{align*}
        \formulaskip &
        \exists \consistencyrelation{CR'}{} \in \transitiveclosure{(\consistencyrelationset{CR} \setminus \setted{\consistencyrelation{CR}{}})} :
        \forall \modelset{m} \in \metamodelinstances{\metamodelset{M}} : \\
        & \formulaskip
        \modelset{m} \consistenttomath \consistencyrelation{CR'}{} \Rightarrow \modelset{m} \consistenttomath \consistencyrelation{CR}{} \\
        & \formulaskip
        \land 
        \classtuple{C}{l,\consistencyrelation{CR}{}} = \classtuple{C}{l,\consistencyrelation{CR'}{}}
    \end{align*}
    This implies that:
    \begin{align*}
        \formulaskip &
        \exists \consistencyrelation{CR'}{} \in \transitiveclosure{(\consistencyrelationset{CR} \setminus \setted{\consistencyrelation{CR}{}})} :
        %\forall \modelset{m} \in \metamodelinstances{\metamodelset{M}} : \\
        %& \formulaskip
        %\condition{c}{l,\consistencyrelation{CR}{}} \subseteq \condition{c}{l,\consistencyrelation{CR'}{}} 
        \forall \conditionelement{c}{l} \in \condition{c}{l,\consistencyrelation{CR}{}} :
        \conditionelement{c}{l} \in \condition{c}{l,\consistencyrelation{CR'}{}}
    \end{align*}
    Because if there was a $\conditionelement{c}{l} \in \condition{c}{l,\consistencyrelation{CR}{}}$ so that $\conditionelement{c}{l} \not\in \condition{c}{l,\consistencyrelation{CR'}{}}$, then the model set $\modelset{m}$ only consisting of $\conditionelement{c}{l}$ would be consistent to $\consistencyrelation{CR'}{}$, because it does not require any other elements to exist for considering the model set consistent, whereas there is at least one $\tupled{\conditionelement{c}{l}, \conditionelement{c}{r}} \in \consistencyrelation{CR}{}$, so that $\modelset{m}$ needs to contain $\conditionelement{c}{r}$ for considering $\modelset{m}$ consistent to $\consistencyrelation{CR}{}$, which is not given by construction.
    This shows that $\condition{c}{l,\consistencyrelation{CR'}{}}$ contains all elements in $\condition{c}{l,\consistencyrelation{CR}{}}$, so there is always at least one element from $\condition{c}{l,\consistencyrelation{CR'}{}}$ that a model set $\modelset{m}$ contains if it contains an element from $\condition{c}{l,\consistencyrelation{CR}{}}$, %namely the same one, 
    which proves the statement in the lemma.
\end{proof}

\todoDiss{The following lemma derived the property of left-equal redundancy from redundancy, which was not correct. Maybe we can find a more general notion of redundancy from which we can derive the contains implication, reviving this lemma gain.}
% \begin{lemma} \label{lemma:redundancysubset}
%     Let $\consistencyrelation{CR}{}$ be a consistency relation that is redundant in a set of consistency relations $\consistencyrelationset{CR}$ for a set of metamodels $\metamodelset{M}$. Thus there exists a consistency relation $\consistencyrelation{CR'}{} \in \transitiveclosure{(\consistencyrelationset{CR} \setminus \setted{\consistencyrelation{CR}{}})}$ with:
%     \begin{align*}
%         \formulaskip & 
%         \forall \modelset{m} \in \metamodelinstances{\metamodelset{M}} : \modelset{m} \consistenttomath \consistencyrelation{CR'}{} \Rightarrow \modelset{m} \consistenttomath \consistencyrelation{CR}{}
%     \end{align*}
%     Then it holds that:
%     \begin{align*}
%         \formulaskip &
%         \forall \conditionelement{c}{l} \in \condition{c}{l, \consistencyrelation{CR}{}} : \exists \conditionelement{c'}{l} \in \condition{c}{l, \consistencyrelation{CR'}{}} : 
%         \forall{m} \in \metamodelinstances{\metamodelset{M}} : \\
%         & \formulaskip
%         \modelset{m} \containsmath \condition{c'}{l} \Rightarrow \modelset{m} \containsmath \condition{c}{l} %\\
%         % &
%         % \land \forall \conditionelement{c}{r} \in \condition{c}{r, \consistencyrelation{CR}{}} : \exists \conditionelement{c'}{r} \in \condition{c}{r, \consistencyrelation{CR'}{}} : 
%         % \forall{m} \in \metamodelinstances{\metamodelset{M}} : \\
%         % & \formulaskip
%         % \modelset{m} \containsmath \condition{c'}{r} \Rightarrow \modelset{m} \containsmath \condition{c}{r}
%         %\conditionelement{c}{l} \subseteq \conditionelement{c'}{l}
%     \end{align*}
% \end{lemma}

% \begin{proof}
%     % Due to symmetry of the statement for $\conditionelement{c}{l}$ and $\conditionelement{c}{r}$, the proof is also symmetric, which is why we restrict the proof to $\conditionelement{c}{l}$. 
%     We prove that
%     \begin{align*}
%         \formulaskip &
%         \forall \conditionelement{c}{l} \in \condition{c}{l, \consistencyrelation{CR}{}} : \exists \conditionelement{c'}{l} \in \condition{c}{l, \consistencyrelation{CR'}{}} : 
%         %\forall{m} \in \metamodelinstances{\metamodelset{M}} : \\
%         %& \formulaskip
%         \conditionelement{c}{l} \subseteq \conditionelement{c'}{l}
%     \end{align*}
%     which directly implies the statement according to \autoref{def:conditionelementcontainment} for the containment of condition elements.
%     Let us assume the contrary, such that:
%     \begin{align*}
%         \formulaskip &
%         \exists \conditionelement{c}{l} \in \condition{c}{l, \consistencyrelation{CR}{}} : \forall \conditionelement{c'}{l} \in \condition{c}{l, \consistencyrelation{CR'}{}} : \conditionelement{c}{l} \not\subseteq \conditionelement{c'}{l}
%     \end{align*}
%     Consider that $\conditionelement{c}{l} = \tupled{\object{o}{1}, \dots \object{o}{n}} \in \condition{c}{l, \consistencyrelation{CR}{}}$.
%     Now select a model set $\modelset{m} \in \metamodelinstances{\metamodelset{M}}$, which only contains objects $\object{o'}{1}, \dots, \object{o'}{n}$, such that $\forall i \in \setted{1, \dots, n} : \object{o}{i} \subseteq \object{o'}{i}$. In other words, we select a minimal model set that contains $\conditionelement{c}{l}$.
%     Per definition of $\consistencyrelation{CR}{}$, there must exist at least one consistency relation pair $\tupled{\conditionelement{c}{l}, \conditionelement{c}{r}} \in \consistencyrelation{CR}{}$, in which $\conditionelement{c}{l}$ occurs.
%     Since $\modelset{m}$ does not contain any $\conditionelement{c}{r}$, $\neg (\modelset{m} \consistenttomath \consistencyrelation{CR}{})$ per definition.
%     Since $\forall \conditionelement{c'}{l} \in \condition{c}{l, \consistencyrelation{CR'}{}} : \conditionelement{c}{l} \not\subseteq \conditionelement{c'}{l}$, there is no such $\conditionelement{c'}{l}$ with $\modelset{m} \containsmath \conditionelement{c'}{l}$.
%     \dots
%     \todoHeiko{Correct and finish proof}
% \end{proof}

\begin{theorem} \label{theorem:redundancycompatibility}
    Let $\consistencyrelationset{CR}$ be a compatible set of consistency relations for a set of metamodels $\metamodelset{M}$ and let $\consistencyrelation{CR}{}$ be a consistency relation that is left-equal redundant in $\consistencyrelationset{CR} \cup \setted{\consistencyrelation{CR}{}}$. Then $\consistencyrelationset{CR} \cup \setted{\consistencyrelation{CR}{}}$ is compatible. 
    % If two sets of consistency relations $\set{\consistencyrelation[1]{CR}}$ and $\set{\consistencyrelation[2]{CR}}$ are equivalent and $\set{\consistencyrelation[1]{CR}}$ is compatible, then $\set{\consistencyrelation[2]{CR}}$ is compatible as well.
\end{theorem}

\begin{proof}
    Due to left-equal redundancy of $\consistencyrelation{CR}{}$ in $\consistencyrelationset{CR} \cup \setted{\consistencyrelation{CR}{}}$, which also implies general redundancy according to \autoref{def:redundancy}, $\consistencyrelationset{CR}$ and $\consistencyrelationset{CR} \cup \setted{\consistencyrelation{CR}{}}$ are equivalent, according to \autoref{lemma:redundancyimpliesequivalence}.
    Due to that equivalence, we know that for any model set $\modelset{m} \in \metamodelinstances{\metamodelset{M}}$:
    \begin{equation} \label{eq:redundancyconsistency}
        \formulaskip 
        \modelset{m} \mathtext{consistent to} \consistencyrelationset{CR} \equivalent     \modelset{m} \mathtext{consistent to} \consistencyrelationset{CR} \cup \setted{\consistencyrelation{CR}{}}
    \end{equation}
    It follows from \autoref{def:compatibility} for compatibility and \autoref{eq:redundancyconsistency}:
    \begin{align} \label{eq:redundancycompatibleexisting}
        \formulaskip & \nonumber
        \forall \consistencyrelation{CR'}{} \in \consistencyrelationset{CR} : \forall \conditionelement{c}{l} \in \condition{c}{l, \consistencyrelation{CR'}{}} %\cup \condition{c}{r, \consistencyrelation{CR'}{}} 
        : \exists \modelset{m} \in \metamodelinstances{\metamodelset{M}} : \\
        & \formulaskip
        \modelset{m} \containsmath \conditionelement{c}{l} \land \modelset{m} \containsmath \consistencyrelationset{CR} \cup \setted{\consistencyrelation{CR}{}}
    \end{align}
    This already shows that for $\consistencyrelationset{CR}$ the compatibility definition is fulfilled, so we need to prove that the compatibility definition is fulfilled for $\consistencyrelation{CR}{}$ as well.
    % \begin{align*}
    %     \formulaskip &
    %     \forall \tupled{\conditionelement{c}{l}, \conditionelement{c}{r}} \in \consistencyrelation{CR}{} : \exists \modelset{m} \in \metamodelinstances{\metamodelset{M}}: \\
    %     & \formulaskip
    %     \modelset{m} \mathtext{contains} \tupled{\conditionelement{c}{l}, \conditionelement{c}{r}} \land \modelset{m} \mathtext{consistent to} \consistencyrelationset{CR} \cup \setted{\consistencyrelation{CR}{}}
    % \end{align*}
    Due to compatibility of $\consistencyrelationset{CR}$ and \autoref{lemma:compatibilitytransitiveclosure} showing equality of compatibility for a consistency relation set and its transitive closure, we know that:
    \begin{align} \label{eq:compatibilityclosure}
        \formulaskip & \nonumber
        \forall \consistencyrelation{CR'}{} \in \transitiveclosure{\consistencyrelationset{CR}} : \forall \conditionelement{c}{l} \in \condition{c}{l, \consistencyrelation{CR'}{}} %\cup \condition{c}{r, \consistencyrelation{CR'}{}} 
        : \exists \modelset{m} \in \metamodelinstances{\metamodelset{M}} : \\
        & \formulaskip
        \modelset{m} \containsmath \conditionelement{c}{l} \land \modelset{m} \consistenttomath \transitiveclosure{\consistencyrelationset{CR}}
    \end{align}
    Due to left-equal redundancy of $\consistencyrelation{CR}{}$ in $\consistencyrelationset{CR} \cup \setted{\consistencyrelation{CR}{}}$, we have shown in \autoref{lemma:leftequalredundancysubset} that the following is true:
    \begin{align} \label{eq:redundancycontainment}
        \formulaskip & \nonumber 
        \exists \consistencyrelation{CR'}{} \in \transitiveclosure{\consistencyrelationset{CR}} : \forall \conditionelement{c}{l} \in \condition{c}{l, \consistencyrelation{CR}{}} : \exists \conditionelement{c'}{l} \in \condition{c}{l,\consistencyrelation{CR'}{}} : \forall \modelset{m} \in \metamodelinstances{\metamodelset{M}} : \\
        & \formulaskip
        \modelset{m} \containsmath \conditionelement{c'}{l} \Rightarrow \modelset{m} \containsmath \conditionelement{c}{l}
    \end{align}
    The combination of \autoref{eq:compatibilityclosure} and \autoref{eq:redundancycontainment} gives:
    \begin{align*}
        \formulaskip & \nonumber 
        \exists \consistencyrelation{CR'}{} \in \transitiveclosure{\consistencyrelationset{CR}} : \forall \conditionelement{c}{l} \in \condition{c}{l, \consistencyrelation{CR}{}} : \exists \conditionelement{c'}{l} \in \condition{c}{l,\consistencyrelation{CR'}{}} : \\
        & \formulaskip
        (\forall \modelset{m} \in \metamodelinstances{\metamodelset{M}} : \modelset{m} \containsmath \conditionelement{c'}{l} \Rightarrow \modelset{m} \containsmath \conditionelement{c}{l}) \\
        & \formulaskip
        \land (\exists \modelset{m} \in \metamodelinstances{\metamodelset{M}} :
        \modelset{m} \containsmath \conditionelement{c'}{l} \land \modelset{m} \mathtext{consistent to} \transitiveclosure{\consistencyrelationset{CR}})
    \end{align*}
    A simplification by combining the two last lines of that statement leads to:
    \begin{align*}
        \formulaskip & \nonumber 
        \forall \conditionelement{c}{l} \in \condition{c}{l, \consistencyrelation{CR}{}} : \exists \modelset{m} \in \metamodelinstances{\metamodelset{M}} : \\
        & \formulaskip
        \modelset{m} \containsmath \conditionelement{c}{l} \land \modelset{m} \consistenttomath \transitiveclosure{\consistencyrelationset{CR}}
    \end{align*}
    Due to \autoref{eq:redundancyconsistency} and \autoref{lemma:consistencytransitiveclosure}, which shows equality of consistency for a consistency relation set and its transitive closure, this is equivalent to:
    \begin{align} \label{eq:redundancycompatiblenew}
        \formulaskip & \nonumber 
        \forall \conditionelement{c}{l} \in \condition{c}{l, \consistencyrelation{CR}{}} : \exists \modelset{m} \in \metamodelinstances{\metamodelset{M}} : \\
        & \formulaskip
        \modelset{m} \containsmath \conditionelement{c}{l} \land \modelset{m} \consistenttomath \consistencyrelationset{CR} \cup \setted{\consistencyrelation{CR}{}}
    \end{align}
    % Together with the symmetric argumentation for $\conditionelement{c}{r}$ rather than $\conditionelement{c}{l}$, we have shown that the compatibility definition holds for $\consistencyrelation{CR}{}$:
    % \begin{align} \label{eq:redundancycompatiblenew}
    %     \formulaskip & \nonumber 
    %     \forall \conditionelement{c}{} \in \condition{c}{l, \consistencyrelation{CR}{}} \cup \condition{c}{r,\consistencyrelation{CR}{}}: \exists \modelset{m} \in \metamodelinstances{\metamodelset{M}} : \\
    %     & \formulaskip
    %     \modelset{m} \mathtext{contains} \conditionelement{c}{} \land \modelset{m} \mathtext{consistent to} \consistencyrelationset{CR} \cup \setted{\consistencyrelation{CR}{}}
    % \end{align}
    The combination of \autoref{eq:redundancycompatibleexisting} and \autoref{eq:redundancycompatiblenew} shows that $\consistencyrelationset{CR} \cup \setted{\consistencyrelation{CR}{}}$ fulfills \autoref{def:compatibility} for compatibility.
    % Assume that given equivalent consistency relation sets $\set{\consistencyrelation[1]{CR}}$ and $\set{\consistencyrelation[2]{CR}}$ are equivalent and $\set{\consistencyrelation[1]{CR}}$ is compatible, whereas $\set{\consistencyrelation[2]{CR}}$ is not.
    % Then there is a consistency relation $\consistencyrelation{CR} \in \set{\consistencyrelation[2]{CR}}$ such that for all pairs of tuples $\bigtupled{\tupled{e_{l1}, \ldots, e_{ln}}, \tupled{e_{r1}, \ldots, e{rm}}} \in \consistencyrelation{CR}$ there is no set of models $\tupled{\model[1]{m}, \ldots, \model[k]{m}}$ such that for all models $\model[i]{m}, \model[j]{m}$ in that tuple either (i) $\{e_{l1}, \ldots e_{ln} \} \not\subseteq \model[i]{m} \lor \{e_{r1}, \ldots e_{rm} \not\subseteq \model[j]{m}$ or (ii) $\{ \model[1]{m}, \ldots, \model[k]{m} \} \text{not consistent according to} \set{\consistencyrelation[2]{CR}} \setminus \{ \consistencyrelation{CR} \}$. 
\end{proof}

% \begin{proof}
%     %In the proof, we always consider model sets $\modelset{m}$ to be sets of instances of the metamodels that are related by the consistency relations in $\consistencyrelationset{CR} \cup \setted{\consistencyrelation{CR}{}}$, without further mentioning that.
%     Due to the redundancy of $\consistencyrelation{CR}{}$ in $\consistencyrelationset{CR} \cup \setted{\consistencyrelation{CR}{}}$, $\consistencyrelationset{CR}$ and $\consistencyrelationset{CR} \cup \setted{\consistencyrelation{CR}{}}$ are equivalent, according to \autoref{corollary:redundancyimpliedequivalence}.
%     Due to that equivalence, it holds that for any model set $\modelset{m}$:
%     \begin{equation} \label{eq:redundancyconsistency}
%     \formulaskip 
%     \modelset{m} \mathtext{consistent to} \consistencyrelationset{CR} \equivalent \modelset{m} \mathtext{consistent to} \consistencyrelationset{CR} \cup \setted{\consistencyrelation{CR}{}}
%     \end{equation}
%     Due to \autoref{def:compatibility} for compatibility and \autoref{eq:redundancyconsistency}, it holds that:
%     \begin{align} \label{eq:redundancycompatibleexisting}
%         \formulaskip & \nonumber
%         \forall \consistencyrelation{CR'}{} \in \consistencyrelationset{CR} : \forall \tupled{\conditionelement{c}{l}, \conditionelement{c}{r}} \in \consistencyrelation{CR'}{} : \exists \modelset{m} \in \metamodelinstances{\metamodelset{M}}: \\
%         & \formulaskip
%         \modelset{m} \mathtext{contains} \tupled{\conditionelement{c}{l}, \conditionelement{c}{r}} \land \modelset{m} \mathtext{consistent to} \consistencyrelationset{CR} \cup \setted{\consistencyrelation{CR}{}}
%     \end{align}
%     This already shows that for $\consistencyrelationset{CR}$ the compatibility definition holds, so we need to prove that the compatibility definition holds for $\consistencyrelation{CR}{}$.
%     % \begin{align*}
%     %     \formulaskip &
%     %     \forall \tupled{\conditionelement{c}{l}, \conditionelement{c}{r}} \in \consistencyrelation{CR}{} : \exists \modelset{m} \in \metamodelinstances{\metamodelset{M}}: \\
%     %     & \formulaskip
%     %     \modelset{m} \mathtext{contains} \tupled{\conditionelement{c}{l}, \conditionelement{c}{r}} \land \modelset{m} \mathtext{consistent to} \consistencyrelationset{CR} \cup \setted{\consistencyrelation{CR}{}}
%     % \end{align*}
%     Due to compatibility of $\consistencyrelationset{CR}$ and \autoref{lemma:compatibilitytransitiveclosure} and \autoref{lemma:consistencytransitiveclosure}, it holds that:
%     \begin{align} \label{eq:compatibilityclosure}
%         \formulaskip & \nonumber
%         \forall \consistencyrelation{CR'}{} \in \transitiveclosure{\consistencyrelationset{CR}} : \forall \tupled{\conditionelement{c'}{l}, \conditionelement{c'}{r}} \in \consistencyrelation{CR'}{} : \exists \modelset{m} \in \metamodelinstances{\metamodelset{M}} :\\
%         & \formulaskip
%         \modelset{m} \mathtext{contains} \tupled{\conditionelement{c'}{l}, \conditionelement{c'}{r}} \land \modelset{m} \mathtext{consistent to} \transitiveclosure{\consistencyrelationset{CR}}
%     \end{align}
%     Due to the redundancy of $\consistencyrelation{CR}{}$ in $\consistencyrelationset{CR} \cup \setted{\consistencyrelation{CR}{}}$, it holds that:
%     \begin{align} \label{eq:redundancycontainment}
%         \formulaskip & \nonumber 
%         \exists \consistencyrelation{CR'}{} \in \transitiveclosure{\consistencyrelationset{CR}} : \forall \tupled{\conditionelement{c}{l}, \conditionelement{c}{r}} \in \consistencyrelation{CR}{} : \exists \tupled{\conditionelement{c'}{l}, \conditionelement{c'}{r}} \in \consistencyrelation{CR'}{} : \forall \modelset{m} \in \metamodelinstances{\metamodelset{M}} : \\
%         & \formulaskip 
%         \modelset{m} \mathtext{contains} \tupled{\conditionelement{c}{l}, \conditionelement{c}{r}} \equivalent
%         \modelset{m} \mathtext{contains} \tupled{\conditionelement{c'}{l}, \conditionelement{c'}{r}}
%     \end{align}
%     \autoref{eq:compatibilityclosure} especially holds for the selected $\consistencyrelation{CR'}{}$ and $\tupled{\conditionelement{c'}{l}, \conditionelement{c'}{r}}$ in \autoref{eq:redundancycontainment}, such that the following holds:
%     \begin{align*}
%         \formulaskip & 
%         \exists \consistencyrelation{CR'}{} \in \transitiveclosure{\consistencyrelationset{CR}} : \forall \tupled{\conditionelement{c}{l}, \conditionelement{c}{r}} \in \consistencyrelation{CR}{} : \exists \tupled{\conditionelement{c'}{l}, \conditionelement{c'}{r}} \in \consistencyrelation{CR'}{} : \\
%         & \formulaskip 
%         \forall \modelset{m} \in \metamodelinstances{\metamodelset{M}} : \modelset{m} \mathtext{contains} \tupled{\conditionelement{c}{l}, \conditionelement{c}{r}} \equivalent
%         \modelset{m} \mathtext{contains} \tupled{\conditionelement{c'}{l}, \conditionelement{c'}{r}} \\
%         & \formulaskip
%         \land \exists \modelset{m} \in \metamodelinstances{\metamodelset{M}} : \modelset{m} \mathtext{contains} \tupled{\conditionelement{c'}{l}, \conditionelement{c'}{r}} \land \modelset{m} \mathtext{consistent to} \transitiveclosure{\consistencyrelationset{CR}}
%     \end{align*}
%     Combining the last two lines leads to:
%     \begin{align*}
%         \formulaskip & 
%         \exists \consistencyrelation{CR'}{} \in \transitiveclosure{\consistencyrelationset{CR}} : \forall \tupled{\conditionelement{c}{l}, \conditionelement{c}{r}} \in \consistencyrelation{CR}{} : \exists \tupled{\conditionelement{c'}{l}, \conditionelement{c'}{r}} \in \consistencyrelation{CR'}{} : \\
%         & \formulaskip
%         \exists \modelset{m} \in \metamodelinstances{\metamodelset{M}} : \modelset{m} \mathtext{contains} \tupled{\conditionelement{c}{l}, \conditionelement{c}{r}} \land \modelset{m} \mathtext{consistent to} \transitiveclosure{\consistencyrelationset{CR}}
%     \end{align*}
%     This implies with \autoref{lemma:consistencytransitiveclosure} and \autoref{eq:redundancyconsistency} that:
%     \begin{align} \label{eq:redundancycompatiblenew}
%         \formulaskip & \nonumber
%         \forall \tupled{\conditionelement{c}{l}, \conditionelement{c}{r}} \in \consistencyrelation{CR}{} : \exists \modelset{m} \in \metamodelinstances{\metamodelset{M}} : \\
%         & \formulaskip
%         \modelset{m} \mathtext{contains} \tupled{\conditionelement{c}{l}, \conditionelement{c}{r}} \land \modelset{m} \mathtext{consistent to} \consistencyrelationset{CR} \cup \setted{\consistencyrelation{CR}{}}
%     \end{align}
%     The combination of \autoref{eq:redundancycompatibleexisting} and \autoref{eq:redundancycompatiblenew} shows that the compatibility definition is fulfilled for $\consistencyrelationset{CR} \cup \setted{\consistencyrelation{CR}{}}$.
%     % Assume that given equivalent consistency relation sets $\set{\consistencyrelation[1]{CR}}$ and $\set{\consistencyrelation[2]{CR}}$ are equivalent and $\set{\consistencyrelation[1]{CR}}$ is compatible, whereas $\set{\consistencyrelation[2]{CR}}$ is not.
%     % Then there is a consistency relation $\consistencyrelation{CR} \in \set{\consistencyrelation[2]{CR}}$ such that for all pairs of tuples $\bigtupled{\tupled{e_{l1}, \ldots, e_{ln}}, \tupled{e_{r1}, \ldots, e{rm}}} \in \consistencyrelation{CR}$ there is no set of models $\tupled{\model[1]{m}, \ldots, \model[k]{m}}$ such that for all models $\model[i]{m}, \model[j]{m}$ in that tuple either (i) $\{e_{l1}, \ldots e_{ln} \} \not\subseteq \model[i]{m} \lor \{e_{r1}, \ldots e_{rm} \not\subseteq \model[j]{m}$ or (ii) $\{ \model[1]{m}, \ldots, \model[k]{m} \} \text{not consistent according to} \set{\consistencyrelation[2]{CR}} \setminus \{ \consistencyrelation{CR} \}$. 
    
%     % Need to redefine compatibility to proceed \dots
% \end{proof}
% \todoHeiko{the proof does not require the second part of the redundancy definition, so can we omit it? Or is there a mistake in the proof?}

\begin{corollary} \label{corollary:transitiveredundancycompatibility}
    Let $\consistencyrelationset{CR}$ be a compatible set of consistency relations and let $\consistencyrelation{CR}{1}, \dots, \consistencyrelation{CR}{k}$ be consistency relations with:
    \begin{align*}
        \formulaskip &
        \forall i \in \setted{1, \dots, k} : \\
        & \formulaskip 
        \consistencyrelation{CR}{i} \leftequalredundantinmath \consistencyrelationset{CR} \cup \setted{\consistencyrelation{CR}{1}, \dots, \consistencyrelation{CR}{i}}
    \end{align*}
    Then $\consistencyrelationset{CR} \cup \setted{\consistencyrelation{CR}{1}, \dots, \consistencyrelation{CR}{k}}$ is compatible.
\end{corollary}

\begin{proof}
    This is an inductive implication of \autoref{theorem:redundancycompatibility}, because $\consistencyrelationset{CR}$ is compatible and sequentially adding $\consistencyrelation{CR}{i}$ to $\consistencyrelationset{CR} \cup \setted{\consistencyrelation{CR}{1}, \dots, \consistencyrelation{CR}{i-1}}$ ensures that $\consistencyrelationset{CR} \cup \setted{\consistencyrelation{CR}{1}, \dots, \consistencyrelation{CR}{i}}$ is compatible, because $\consistencyrelationset{CR} \cup \setted{\consistencyrelation{CR}{1}, \dots, \consistencyrelation{CR}{i-1}}$ was compatible as well.
\end{proof}

With \autoref{corollary:transitiveredundancycompatibility}, we have shown that if we have a set of consistency relations $\consistencyrelationset{CR}$ and are able to find a sequence of redundant consistency relations $\consistencyrelation{CR}{1}, \dots {\consistencyrelation{CR}{k}}$ according to \autoref{corollary:transitiveredundancycompatibility} such that we know that $\consistencyrelationset{CR} \setminus \setted{ \consistencyrelation{CR}{1}, \dots {\consistencyrelation{CR}{k}}}$ is compatible, then it is proven that $\consistencyrelationset{CR}$ is compatible.
