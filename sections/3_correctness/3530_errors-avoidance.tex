\section{Detection and Avoidance of Errors}
\label{chap:errors:avoidance}

\mnote{Error avoidance vs. detection}
Two ways to deal with the possibility of errors in transformation networks exist.
First, mistakes can be avoided (a priori), which was the major goal of the discussions and approaches presented in the previous chapters, such that no failures can occur when executing a transformation network or at least failures due to specific mistakes are avoided.
Second, mistakes can be detected (a posteriori) by identifying failures during transformation execution.
We have already discussed that how a mistakes manifests depends on the used application algorithm.
An algorithm without an artificial execution bound may fail by non-termination, one without proper consistency checks may fail by returning inconsistent models, and a conservative algorithm, such as the provenance algorithm proposed in \autoref{chap:orchestration:algorithm}, may return $\bot$.

\begin{propertable}
    \rowcolors{1}{\firstlinecolor}{\secondlinecolor}
    \begin{tabular}{C{2.4em}L{7.7em}L{6.7em}L{11.6em}}
        \toprule
        \textbf{Level} & \textbf{Name} & \textbf{Avoidance} & \textbf{Detection} \\
        \midrule
        1 & \LevelTransformation & By construction & Duplicate element creation \\
        2 & \LevelNetworkRelation & By analysis & Any network failure \\
        3 & \LevelNetworkRule & - & Any network failure \\
        \bottomrule
    \end{tabular}
    \caption[Avoidance and detection of mistakes at specification levels]{Avoidance and detection of mistakes at the different levels in the transformation network specification process.}
    \label{tab:errors:avoidance}
\end{propertable}

\mnote{Detection from categorization}
In \autoref{tab:errors:avoidance}, we depict the possibilities of avoiding and detecting mistakes at the different levels in the transformation network specification process.
Avoidability is derived from the discussions in the previous chapters, whereas the detection is a result of the preceding categorization of mistakes and resulting failures.


\subsection{Error Avoidance}

\mnote{Failure absence indicating mistake absence}
In the best case, no failures occur in a transformation network, which means that no mistakes were made at all or at least none of them leads to a failure in a specific scenario.
In fact, a network without mistakes does not mean that no failures occur, because the application algorithm can always fail because of undecidability of the orchestration problem.
Thus, the absence of failures indicates the absence of mistakes but not vice versa.

\mnote{Mistake avoidance at different levels}
To avoid mistakes, we have already discussed different approaches in the previous chapters.
Associated with the identified specification levels, we can identify at which levels mistakes can be avoided by construction, by analysis, or not at all.
At the \leveltransformation level, correctness requires transformations to be synchronizing.
As discussed in \autoref{chap:synchronization}, this property can be achieved by construction, because it is a property of a single transformation and does not depend on the other transformations to be combined with.
We have also proposed techniques, especially the matching of existing elements, to achieve this correctness by construction.
At the \levelnetworkrelation level, correctness requires consistency relations to be compatible.
As discussed in \autoref{chap:compatibility}, this property can be validated by analysis of the transformations and their consistency relations.
It can, however, not be avoided by construction.
Finally, at the \levelnetworkrule level, we do not have a precise notion of correctness, which makes it impossible to define criteria for avoidance.

\mnote{Avoidance at \leveltransformation level}
Since we assume transformations to be developed independently and reused modularly, it is especially relevant that mistakes at the \leveltransformation level, for which the required knowledge exists, can be avoided by construction.
The necessary knowledge for avoiding mistakes at the \levelnetworkrelation level does actually not exist with that assumption, thus we may not even consider them as actual mistakes.
Finally, the mistakes that cannot be avoided by construction are handled by the proposed use of a conservative application algorithm anyway.
As we have discussed before, consistency checks of transformations may be based on the assumption that consistency is achieved by construction.
Thus, it is important that correctness at the \leveltransformation level is achieved by construction, as otherwise the application algorithm may apply non-synchronizing transformation without detecting that the yielded models are inconsistent, thus returning inconsistent models.

\mnote{Avoidance by network topologies}
In \autoref{chap:classification}, we will discuss how network topologies affect how prone a transformation network is to the possibility of containing faults.
We will show that an appropriate topology excludes faults such that transformation developers cannot make mistakes at the network levels.
Thus, it is also possible to avoid such mistakes by construction, but this limits the networks we can define to specific topologies.
We also discuss in \autoref{chap:improvement} an approach to construct networks of such a topology, which mitigates the restrictions induced by the necessity of having a specific topology by introducing auxiliary models.


\subsection{Error Detection}

\mnote{Tracing between mistake and failure types}
Whenever mistakes are not avoided by construction or analysis, they can be detected by failures of the application algorithm.
The insights regarding relations between mistake and failure types may at first not sound interesting, because all mistakes at the two network levels can lead to any kind of failure.
And even if a duplication occurs, which is in particular the result of a mistake at the \leveltransformation level, this can also be a consequence of a mistake at the two network levels.
Additionally, the algorithm may not only fail because of mistakes but also because of undecidability of the orchestration problem.
Still, we can make some relevant conclusions for the detection of errors.

\mnote{Identifying failure cause by model state}
Insights about the causing mistakes can especially be derived from an inconsistent state of the models that the algorithm produced, e.g., by investigating whether this inconsistent state contains duplications of elements.
This is why we proposed the provenance algorithm in \autoref{chap:orchestration:algorithm}, which is supposed to support the identification of problems in the transformations that lead to the application algorithm not being able to find a consistent orchestration.
Thus, in case the algorithm fails for specific inputs, it is up to the transformation developer to investigate the state of the models in which the algorithm failed to identify the reason for that.

\mnote{Algorithm exchange to identify mistakes}
Whenever the application algorithm fails, it can be useful to exchange it with one with different properties.
If the algorithm does not terminate, introducing an artificial execution bound can produce an insightful inconsistent state of the models.
These inconsistent models can also be retrieved from a conservative algorithm as proposed in \autoref{chap:orchestration:algorithm}.

\mnote{Duplications indicate missing synchronization}
The occurrence of duplications is a specific indicator for missing synchronization.
They can occur in inconsistent returned models produced by the algorithm and will most likely occur because of missing synchronization.
In our evaluation in \autoref{chap:correctness_evaluation}, we will see that in the investigated case study duplications occurred because of missing synchronization in most cases or can at least be distinguished from duplications caused by other mistakes.

\mnote{Failure frequency indication}
If the algorithm fails for most inputs in any way, this may be an indicator that the algorithm is not only unable to yield consistent models because of the orchestration problem but because some essential mistakes prevent it from from finding consistent models, such that, in the worst case, no consistent orchestration exists at all.
Thus, an often failing algorithm may be an indicator for, among others, incompatibilities.

\mnote{Identifying failure cause by returning $\bot$}
It may make a difference whether a conservative algorithm fails returning $\bot$ because the maximal number of executions was reached or because a transformation could not be applied anymore.
While the inability to apply a transformation can be seen as an indicator for an actual mistake within the transformations (such as the \levelnetworkrelation level error in \autoref{fig:errors:mistake_effects_example_cases}), the abortion because of reaching the execution bound can also just result from the conservative behavior to avoid non-termination.

\mnote{Unique identification of fault existence}
Finally, in the best case errors are avoided by construction, especially potential mistakes at the \leveltransformation level.
At the network levels, mistakes cannot be avoided but, in the best case, analyzed.
Since we need a conservative application algorithm anyway, it also ensures that such mistakes do not lead to unwanted results.
In the worst case, the algorithm will only be able to yield consistent models in few or even no cases.
Then the transformation developer must investigate the state of the models with which the algorithm fails to identify the reasons.
Although there are several indicators for the existence of faults, it cannot be uniquely distinguished whether the application algorithm fails because of undecidability of the orchestration problem or because actually the transformations contain a fault.
Since we assume independent development and reuse of transformations, the focus on avoiding mistakes at the \leveltransformation level and the handling of mistakes at the network levels by a conservative algorithm fits well to that context assumption.

