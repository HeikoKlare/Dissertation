%%%
%%% AVOIDANCE PATTERNS
%%%
\section{Synchronization of Ordinary Transformation}
Goal: Avoidance strategies for interoperability mistakes, i.e. achieving synchronization of transformations (MA Torsten / Timur)
\label{chap:prevention:interoperability}

\begin{copiedFrom}{ICMT}

% FORMERLY: \subsection{Matching Elements in Operationalizations}
\subsection{Matching Elements}
\label{chap:prevention:interoperability:matching}

To avoid failures due to mistakes at the operationalization level, transformations must respect that other transformations may have already created elements.
In the binary case, this is unnecessary.
A single incremental \ac{BX} can assume that elements are either created by the user, %and then are input of the transformations
or were created by the transformation itself.
To identify corresponding elements, transformation languages usually use trace models, which are created by the transformations.
When \acp{BX} are combined to networks, %elements may also be created by other transformations.
%In consequence, 
direct trace links may be missing because a sequence of other transformations created the elements and trace links only indirectly across elements in other models.
%Thus, it is necessary to establish direct trace links between corresponding elements.´
In this scenario, corresponding elements can be matched by information at three levels:
%Such element matching can be performed on three levels:
\begin{enumerate}
    \item \emph{Explicit unique}: The information that elements correspond is unique and represented explicitly, e.g., within a trace model. %Existing transformation languages usually use this technique.
    \item \emph{Implicit unique}: The information that elements correspond is unique, but represented implicitly, e.g., in terms of key information within the models such as element names. %types and element names.
    \item \emph{Non-unique}: If no unique information exists, heuristics must be used, e.g. based on ambiguous information or transitive resolution of indirect trace links.
\end{enumerate}
\todo{Give examples for each case to show that they actually occur}

Indirect trace links, which link elements transitively across other models, usually exist for elements that correspond, because other transformations have already created them.
Nevertheless, indirect trace links cannot be used to unambiguously identify such elements.
An element can correspond to multiple elements in another model, which is why most transformation languages offer tagging of trace links with additional information to identify the correct element.
%For example, a component in an architecture description could be mapped to two classes in an object-oriented design, one providing the component implementation and one providing utilities.
%The relevant corresponding element can be retrieved if the traces are tagged with the information that one class is the implementation and one is a utility.
For example, a language may tag trace links with the transformation rule they were instantiated in.
This is helpful in the bidirectional case, but when links are resolved transitively, these tags have been created by other, independently developed transformations, and are thus unknown.
%If such tags would be considered, transformations would depend on tags of other transformations and could thus not be developed independently anymore.
Therefore, resolving indirect trace links is only a heuristic, but does not unambiguously retrieve corresponding elements.

% Explain how to match rules on three different levels, what the levels can provide etc.

% \begin{enumerate}
%     \item Direct Correspondences
%     \item Key information
%     \item Heuristics: Indirect correspondences, potentially ambiguous information
% \end{enumerate}

Finally, it is up to the transformation engine or the transformation developer %, depending on the provided abstraction level, 
to ensure that elements are correctly matched.
In contrast to the bidirectional case, direct trace links cannot be assumed in case of networks of \acp{BX}.
Therefore, key information within the models must always be considered to identify matching elements.
Whenever direct trace links or unique key information exists, relevant elements can be unambiguously matched.
In all other cases, heuristics must be used, which potentially leads to failures.

\end{copiedFrom} % ICMT

