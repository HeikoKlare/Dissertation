\subsection{Summary}
\label{sec:formalapproach:summary}

In the previous sections, we have proven the following three central insights:
\begin{enumerate}
    \item Compatibility is composable: If independent sets of consistency relations are compatible, then their union is compatible as well (\autoref{theorem:independencecompatibility}).
    \item Consistency relation trees are compatible: If there are no two concatenations of consistency relations in a consistency relation set that relate the same classes, then that set is compatible (\autoref{theorem:treecompatibility}).
    \item Left-equal redundancy is compatibility-preserving: Adding a left-equal redundant consistency relation to a compatible set of consistency relations, that set unified with the redundant relation is still compatible (\autoref{corollary:transitiveredundancycompatibility}).
\end{enumerate}

These insights enable us to define a formal approach for proving compatibility of a set of consistency relations.
Given a set of relations for which compatibility shall be proven, we search for consistency relations in that set that are left-equal redundant to it.
If iteratively removing such redundant relations---virtually---from the set leads to a set of independent consistency relation trees, it is proven that the initial set of consistency relations is compatible.

Such an approach to prove compatibility of consistency relations is \emph{conservative}.
If the approach finds redundant relations, such that a consistency relation set can be reduced to a set of independent consistency relations trees, the set is proven compatible, as we have shown by proof.
If the approach is not able to find such relations, the set may still be compatible, but the approach is not able to prove that.
Conceptually, this can be due to the fact that there may be compatibility-preserving relations that do not fulfill the definition of left-equal redundancy.
Furthermore, an actual technique to identify left-equal redundant relations may not be able to find all of them automatically, as we will see later.

\todoDiss{Formalize conservativeness}
\todoDiss{Prove that if the dual of the meta graph is a tree, it is a consistency relation tree}

In the following, we discuss how such an approach can be operationalized.
First, we discuss how actual transformations, at the example of \qvtr, can be represented in a graph-based structure, such that it conforms to our formal notion and allows to check whether the structure is an independent set of consistency relation trees.
Second, we present an approach for finding consistency relations that are left-equal redundant, by the means of an SMT solver applied to the constraints defined in \qvtr relations.
