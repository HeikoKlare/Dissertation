\section{Summary}

In this chapter, we have discussed how we can realize an application function for transformation networks.
We have motivated optimality as a desired property of such a function, which ensures that an application function always delivers consistent models if there is an order of the transformations that leads to those models. 
From that optimality notion, we have derived the central orchestration problem, for which we haven proven undecidability, even when restricting transformation networks.
Finally, we have proposed strategies to reduce the cases in which no consistent models are found and an algorithm that has a well-defined order and bound for the transformation execution and, rather than improving optimality, ensures that in cases when no consistent models can be derived at least some information can be provided that helps developers or user of transformations to identify why no consistent models were found.
We conclude this section with the following central insight.

\begin{insight}[Orchestration]
    We have proven that whether an orchestration of modular and independently developed transformations exists that restores consistency for given models and changes, denoted as the \emph{orchestration problem}, is undecidable.
    We have shown that even impractical restrictions to the individual transformations do not make the problem decidable, such that we need to accept that the problem is undecidable.
    In consequence, every algorithm that realizes an application function for transformations can only implement a conservative approximation of the orchestration problem.
    Due to this conservativeness, every algorithm will fail in cases in which actually an orchestration of the transformations exists that leads to consistent models.
    Thus, it is useful to find an algorithm that orchestrates the transformations in a way such that the state of executed transformations and the up to now delivered changes can help the transformation developer or user to identify why the algorithm failed.
    We found that this can be achieved with a strategy of iteratively restoring consistency for subsets of the transformations, such that always a subset of the transformations for which consistency could be restored and a transformation for which it could not be restored anymore can be given to ease reasoning about the cause for failing.
    We have proposed an algorithm that implements that strategy and is proven to fulfill the desired property.
\end{insight}