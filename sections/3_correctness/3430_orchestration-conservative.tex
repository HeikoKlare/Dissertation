\section{Conservative Approximation of the Orchestration Problem}

\mnote{Optimal orchestration cannot be achieved}
In the preceding section, we have proven that the orchestration problem is undecidable and we have discussed that it is unlikely to restrict the problem in a way such that it becomes decidable.
In consequence, we cannot achieve optimality of an orchestration and application function, which results in an 
algorithm that does not return optimal results and, depending on its implementation, does not terminate.
Since the algorithm cannot return optimal results anyway, termination can at least be achieved by introducing an artificial upper bound for the number of transformation applications.
This potentially prevents the algorithm even further from finding consistent orchestrations.

\mnote{Gradual optimality notion and approaches to improve it}
Based on those previous insights, we assume in this section that the orchestration problem cannot be restricted such that it becomes decidable.
We accept that any application function and any algorithm that realizes it will only realize a conservative approximation of the orchestration problem.
This means that it may only return consistent models delivered by a consistent orchestration, but it may not find a consistent orchestration although it exists.
In this case, we say that the function or algorithm, respectively, delivers \emph{false negatives} but no \emph{false positives} and call it \emph{conservative}.
We investigate how we can define optimality of the application function in gradual rather than a binary way, which is supposed to indicate how likely it is that it finds a consistent orchestration.
We then follow the goal of finding means to systematically improve optimality.
Since there are always cases in which the algorithm does not find a consistent orchestration, we propose an algorithm that is supposed to help identifying the reasons for failing in such cases in the subsequent section.

% \begin{itemize}
%     \item We conclude that we need to deal with the situation of undecidability. In consequence, orchestration must operate conservatively, i.e., we cannot assume to always find a solution, but if find it, it must be consistent.
%     \item One approach could be to reduce conservativeness. But even if we reduced conservativeness, we would not be able to completely eliminate it and thus have to deal with the situation that the strategy does not find a solution although it may exist.
%     \item We thus propose an approach that helps to identify the reason when the strategy is in the conservative case, i.e., not able to find a solution although it exists.
% \end{itemize}

%We found that we cannot achieve optimality (undecidability) for arbitrary transformations, we cannot restrict transformation such that we achieve decidability and thus optimality.
%So, we need to deal with situation can we cannot resolve all networks.
%Conclude: Conservativeness rather than correctness or achieving optimality (but improving the latter one)

% \mnote{Achieving a correct application function}
% The definition of the application function basically ensures that the function either returns $\bot$ or executes the \modellevelconsistencypreservationrules given by the orchestration function to retrieve a changes tuple of models.
% It is considered \emph{correct} if it ensures that its result is either $\bot$ or a consistent model tuple by executing the \modellevelconsistencypreservationrules given by the orchestration function.
% % A correct application function thus has to ensure that its result is either $\bot$ or a consistent model tuple by executing the \modellevelconsistencypreservationrules given by the orchestration function.
% In consequence, the application function can be realized by simply executing the result of the orchestration function and check whether the resulting model tuple is consistent or not and return an appropriate result.
% Such a realization is generic and does not depend on the actual consistency preservation rules and orchestration function but represents a generic behavior.
% Additionally, this gives an implementation of that function the ability to present a faulty result to the user, which eases finding out why no consistent state was reached.

% \mnote{Correctness is not crucial}
% Finally, correctness is not crucial, because correctness can easily be achieved by performing any execution of transformations and just ensuring that we terminate at some point in time and then decide whether the resulting models are consistent or not and appropriately deliver the result.

% \mnote{How to define an orchestration function that is as optimal as possible?}
% The remaining difficulty is how to define an orchestration function that fulfills the definition, i.e., to find a finite sequence of transformations, and also one that improves optimality, as an \emph{optimal} function can never be given.
% Although the definition of the orchestration function proposes a closed description of that function, in practice such a function will not have a closed form but will be realized as an algorithm that dynamically decides which transformation to execute next.
% Therefore the arising problem is that the length of the sequence to execute is not known a priori. Therefore, we need some abortion criterion. When a consistent result is found, this criterion is easy. But since we do not know whether a sequence exist, we need an abortion criterion that is reasonable and does not cut off the process although a consistent solution could be found, thus reducing optimality.
% A simple realization for that algorithm to deliver a finite sequence of transformations would be to define a fixed termination criterion, such as a specific number of transformation executions. However, there is no upper bound for the number of executed transformations necessary to achieve consistency. Still, a fixed number (even 0) could be defined for the number of executed transformations to fulfill the definition. Hence, optimality would be 0 then as a consistent result is never reached. We therefore discuss in the following how to define an appropriate orchestration function and how to optimize it.


\subsection{Systematic Improvement of Optimality}
%\subsection{A Gradual Notion of Optimality}
%\subsection{Reducing Conservativeness}

%Due to Turing-completeness of the network this would mean that the orchestration function can decide whether a Turing machine halts, which is proven impossible.
%Thus, our only goal can be to achieve optimality as far as possible in terms of reducing the degree of conservativeness, i.e., reduce the cases in which no sequence is found although it exists.

\mnote{Gradual notion of optimality}
Since no optimal application function can be achieved, we can at least define a gradual notion of optimality.
It indicates for how many inputs of models and changes the application function is able to return consistent models in comparison to the number of cases in which a consistent orchestration exists at all.
This can be seen as a fitness function for the optimality $\function{Opt}$ of an application function:
\begin{align*}
    &
    \function{Opt}(\appfunction{\orcfunction{\transformationset{T}}}) = \frac{
        \abs{\setted{%\modeltuple{m} \in \metamodeltupleinstanceset{M},\changetuple{\metamodeltuple{M}} \in \changeuniverse{\metamodeltuple{M}}} 
        \tupled{\modeltuple{m},\changetuple{\metamodeltuple{M}}} \mid \appfunction{\orcfunction{\transformationset{T}}}(\modeltuple{m}, \changetuple{\metamodeltuple{M}}) \mathtext{is consistent to} \transformationset{T}}}
    }{
        \abs{\setted{\tupled{\modeltuple{m},\changetuple{\metamodeltuple{M}}} \mid \mathtext{consistent orchestration of} \transformationset{T} \mathtext{exists for} \tupled{\modeltuple{m}, \changetuple{\metamodeltuple{M}}}}}
    }
\end{align*}

\mnote{Concrete optimality values are not relevant}
In fact, both the numerator as well as the denominator will usually have infinite values, as there is an infinite number of possible models and changes to them.
It does, however, not matter for us what the actual optimality value of an application function is.
The purpose of the formula is only to explicitly state the influencing factors of optimality to be able to discuss its systematic improvement.

\mnote{Probability for finding correct orchestration by general random application function}
Obviously, we may only improve the numerator to improve optimality, because the denominator, i.e., the number of cases in which consistent orchestrations exist, depends only on the transformations and not the application function.
%is, first, not easy to achieve, and, second, may also reduce the number of cases in which a consistent orchestration is found.
How to improve the numerator highly depends on the concrete implemented application and orchestration functions.
For the most general case, let us assume that we have an application function $\appfunction{\orcfunction{\transformationset{T}}}$ whose orchestration function randomly tries any orchestration, i.e., it selects one of all possible orchestrations according to an equal distribution.
So we consider the following event $E_{\modeltuple{m},\changetuple{\metamodeltuple{M}}}$:
\begin{align*}
    &
    E_{\modeltuple{m},\changetuple{\metamodeltuple{M}}}: \appfunction{\orcfunction{\transformationset{T}}}(\modeltuple{m}, \changetuple{\metamodeltuple{M}}) \mathtext{is consistent to} \transformationset{T}
\end{align*}
The probability that this event occurs is given by the ratio between the number of consistent orchestrations for that input and the number of all orchestrations:
\begin{align*}
    &
    P(E_{\modeltuple{m},\changetuple{\metamodeltuple{M}}}) = \frac{
        \abs{\setted{\tuple{\transformation{t}} \in \transformationset{T}^{< \mathbb{N}} \mid \tuple{\transformation{t}} \mathtext{is consistent orchestration for} \tupled{\modeltuple{m}, \changetuple{\metamodeltuple{M}}}}}
    }{
        \abs{\transformationset{T}^{< \mathbb{N}}}
    }
\end{align*}
Here, the denominator is the size of what we previously introduced as the problem space $P = \abs{\transformationset{T}^{< \mathbb{N}}}$ containing all possible orchestrations, and the numerator is the size of what we previously introduced as the solution space $S_{i} = \abs{\setted{\tuple{\transformation{t}} \in \transformationset{T}^{< \mathbb{N}} \mid \tuple{\transformation{t}} \mathtext{is consistent orchestration for} \tupled{\modeltuple{m}, \changetuple{\metamodeltuple{M}}}}}$, which contains all consistent orchestrations for for an input of models and changes.

\mnote{Expected value for finding a consistent orchestration}
So we can introduce a stochastic variable $AppCons_{\modeltuple{m},\changetuple{\metamodeltuple{M}}}$, which assigns the values $0$ and $1$ to the events $E_{\modeltuple{m},\changetuple{\metamodeltuple{M}}}$ and its complementary:
\begin{align*}
    AppCons_{\modeltuple{m},\changetuple{\metamodeltuple{M}}}(\omega) = \begin{cases}
        0, & \omega = \appfunction{\orcfunction{\transformationset{T}}}(\modeltuple{m}, \changetuple{\metamodeltuple{M}}) \mathtext{is not consistent to} \transformationset{T} \\
        1, & \omega = \appfunction{\orcfunction{\transformationset{T}}}(\modeltuple{m}, \changetuple{\metamodeltuple{M}}) \mathtext{is consistent to} \transformationset{T}
    \end{cases}
\end{align*}
The expected value of this variable is then equal to the event $E_{\modeltuple{m},\changetuple{\metamodeltuple{M}}}$ to occur:
\begin{align*}
    \mu(AppCons_{\modeltuple{m},\changetuple{\metamodeltuple{M}}}) = P(AppCons_{\modeltuple{m},\changetuple{\metamodeltuple{M}}} = 1) = P(E_{\modeltuple{m},\changetuple{\metamodeltuple{M}}})
\end{align*}

\mnote{Number of found consistent orchestration is sum of expected values}
For an application function that chooses a random orchestration, we can thus express the numerator of $\function{Opt}(\appfunction{\orcfunction{\transformationset{T}}})$ as the sum of expected values of the stochastic variables for all possible inputs.
\begin{align*}
    &
    \abs{\setted{\tupled{\modeltuple{m},\changetuple{\metamodeltuple{M}}} \mid \appfunction{\orcfunction{\transformationset{T}}}(\modeltuple{m}, \changetuple{\metamodeltuple{M}}) \mathtext{is consistent to} \transformationset{T}}} = \sum_{\modeltuple{m},\changetuple{\metamodeltuple{M}}} \mu(AppCons_{\modeltuple{m},\changetuple{\metamodeltuple{M}}}) \\
    & \formulaskip
    = \frac{
        \sum_{\modeltuple{m},\changetuple{\metamodeltuple{M}}} \abs{\setted{\tuple{\transformation{t}} \in \transformationset{T}^{< \mathbb{N}} \mid \tuple{\transformation{t}} \mathtext{is consistent orchestration for} \tupled{\modeltuple{m}, \changetuple{\metamodeltuple{M}}}}}
    }{
        \abs{\transformationset{T}^{< \mathbb{N}}}
    }
\end{align*}

\mnote{Probability to find consistent orchestration improvable by reducing number of possibily considered orchestrations}
Thus, if we can improve $P(E_{\modeltuple{m},\changetuple{\metamodeltuple{M}}})$, we also improve optimality, even if orchestrations are chosen randomly.
We can improve that probability by either improving the number of consistent orchestrations or by reducing the number of possibly considered orchestrations.
The number of consistent orchestrations can only be influenced by requirements to the transformations.
For example, the requirement of consistency relations to be compatible improves this values, as we have shown by example in \autoref{chap:compatibility}.
In the following we discuss how we can reduce the number of possibly considered orchestrations while not reducing the number of consistent orchestrations, thus improving the probability of the application function to find a consistent orchestration and thus improving optimality.

\begin{figure}
    \centering
    \includegraphics[width=0.8\textwidth]{figures/correctness/orchestration/simulator_screenshot.png}
    \caption{Screenshot of a network of for an architecture specification, its implementation and an API specification in OpenAPI, which requires multiple execution of the same transformation, developed in the transformation network simulator~\cite{orchestrationSimulator}.}
    \label{fig:orchestration:simulator}
\end{figure}

\mnote{No strategy that improves ability to find consistent orchestration}
The application function can, for sure, contain more intelligent logic for determining an orchestration beyond the random selection of an orchestration to improve the number of cases in which it finds a consistent orchestration.
By implementing further mechanisms to make a reasonable selection, the possibility to find a consistent orchestration may be further improved.
We investigated different orchestration strategies, such as the depth-first or breadth-first propagation of changes, and analyzed them with a simulator developed specifically for that purpose, which is available at GitHub~\cite{orchestrationSimulator}.
An example of a scenario showing the necessity to execute transformation more than once that we implemented in the simulator is depicted in \autoref{fig:orchestration:simulator}
For each strategy, however, we found categories of transformation networks in which they performed worse than another strategy.

\mnote{Backtracking can be used to explore the problem space}
Another strategy could be to try different orchestrations as soon as it turns out that one orchestration cannot yield consistent models.
This can, for example, be achieved by performing backtracking.
\autoref{algo:orchestration:application} dynamically selects transformations to execute. 
Thus as soon as the algorithm detects that no further transformations can lead to a consistent orchestration, it can revert the last transformation execution and proceed with another transformation.
This means that it resets the state of generated changes and executed transformations to the one before the current execution of the orchestration loop and proceeds again with another transformation.
If all transformations as continuations of one sequence of executed transformation are tried out, the algorithm recursively steps back the iterations of the loop.
While this approach, in theory, allows to explore the complete problem space $P = \abs{\transformationset{T}^{< \mathbb{N}}}$, it is impractical because the problem space is infinitely large.
It may, however, be used to try different options in a subset of the problem space, for example, those with a limited length.

% We need to deal with the situation that no consistent orchestration exists and that we are not able to find it, even if it exists, as the problem space is arbitrarily large (arbitrary high number of transformation executions).
% This forbids approaches such as backtracking to find an appropriate solution.
% \todo{Discuss Backtracking}

\mnote{Avoiding alternation to improve ability to find consistent orchestration}
Since it was impossible to find a strategy that is, in general, superior to other investigates strategies, we gave up that direction and focused on finding orchestrations that should be generally avoided.
To this end, we consider alternation as a possibility to reduce the number of cases in which non-termination can occur, thus improving optimality, by both its dynamic detection as well as its avoidance.
Additionally, we focused on finding a strategy that improves the ability to deal with the fact that an optimality of $1$ can never be achieved by orchestrating the transformations such that the process of finding the reason for not finding a consistent orchestration is supported.

% In fact, both these numbers usually infinite, an there is an infinite number of possible models and deltas. However, it does finally not matter for us what the actually value is, but only how to improve that value.
% \todo{We have to map that value to compatibility, which reduces the number of potential false orders.}

% \textbf{Overall Goal:} Find correct orchestration function that improves optimality.

% There are two ways to improve optimality of the orchestration function:
% \begin{enumerate}
%     \item Optimize the orchestration function, i.e., find a good order (probably this is not possible), at least find an order that helps the developer to find problems
%     \item Optimize the input, i.e., define requirements to the transformations and their relations representing the input to optimize optimality
% \end{enumerate}
% \todo{We need an example for that}

% Both goes hand in hand, because restrictions to the input can never lead to an orchestration function that always terminates without leading to unsupported relevant cases.

% This conform to two approaches:
% \begin{enumerate}
%     \item Dynamic decision about selected transformation and abortion criteria
%     \item Constructive restrictions that ensure that appropriate order is (easily) found
% \end{enumerate}

%\todo{Application function can be generically defined, orchestration maybe not? We actually want to ensure that both are generic and none of them has to be defined for a specific project.}

%\subsection{Systematic Improvement of Optimality}
%\paragraph{Avoidance of Non-Termination}

% In consequence, we propose to dynamically deal with alternation / divergence.
% To detect alternations, the execution can simply track if a state way already processed. Apart from spatial problems, this does always work.
% Finding divergence is not that easy, because it is generally not possible to define an upper bound for the number of executions of a single transformation.
% This is due to the reason that, again, this conflicts with the Halting problem.
% We can see this at the simple example in \autoref{fig:formal:noupperboundexample}.

% \begin{figure}
%     \centering
%     \includegraphics[width=\textwidth]{figures/correctness/orchestration/no_upper_bound_example_old.png}
%     \caption{Example for no upper bound}
%     \label{fig:formal:noupperboundexample}
% \end{figure}

% Depending on the value X, the transformations have to be executed X times to result in a consistent state. This value can be arbitrarily chosen, thus an arbitrary number of executions may be necessary to terminate in a consistent state.

% \todo{Moved from single execution section -> revise}
% \begin{theorem}[Orchestration with Unbounded Executions]
%     \label{theorem:unbounded_execution}
%     For any set of transformations $\transformationset{T}$, there can be models $\modeltuple{m}$ and changes $\changetuple{}$ to them for which each possible orchestration function $\orcfunction{\transformationset{T}}$ with whom $\appfunction{\orcfunction{\transformationset{T}}}(\modeltuple{m}, \changetuple{})$ is consistent, such that $\abs{\orcfunction{\transformationset{T}}(\modeltuple{m}, \changetuple{})} > \abs{\transformationset{T}}$.
% \end{theorem}
% \begin{proof}
%     We know from \autoref{lemma:minimal_executions} that $\transformationset{T}_{inc}$ requires at least $4$ executions of $\transformation{T}_{12}$ for the inputs defined in \autoref{lemma:minimal_executions} when selecting $x \geq 5$.
%     Thus, for any orchestration function, we know that $\abs{\orcfunction{\transformationset{T}}(\modeltuple{m}, \changetuple{})} > 4 > 3 = \abs{\transformationset{T}_{inc}}$.
%     This proves the theorem by example.
% \end{proof}


\subsection{Dynamic Detection of Alternation} % Divergence and alternation
\label{chap:orchestration:conservative:alternation}

\mnote{Orchestration requires abortion criterion}
The proposed algorithm, like any algorithm, is supposed to \emph{terminate} in a specific \emph{state} to be considered correct.
In our case, a correct state, as required by an application function it implements, is the return of consistent models or $\bot$, which the algorithm fulfills by construction.
In particular, the algorithm will never return models that are inconsistent, neither because it does not detect that they are inconsistent nor that it detects that they are inconsistent but still returns them.
From our previous findings regarding decidability, we know that we cannot expect the algorithm to realize an optimal application function.
Thus, we either need to implement \function{Orchestrate} such that it always returns $\bot$ after a finite number of executions to ensure termination, which results in returning $\bot$ although consistent models are expected, or we allow an arbitrary number of executions to improve the ability to find consistent results but accept that the algorithm may not terminate.

\mnote{Alternation is inifinte repeated passing of same state}
We have discussed that non-termination of the algorithm can occur because no consistent orchestration exists at all or because the algorithm is not able to find it.
A special case of non-termination is \emph{alternation}, which means that the same states are passed repeatedly. 
In case of transformation networks, alternation means that from some point in time the subsequent executions of the transformations in Line~\ref{algo:orchestration:application:line:stepcalculation} of \autoref{algo:orchestration:application} repeatedly produce the same sequence of results, i.e., of changes.
% Non-termination can, in general, manifest in terms of \emph{alternation} or \emph{divergence}, which means that either the same states are passed repeatedly or that an infinite sequence of different states is produced.
% In case of transformation networks, alternation means from some point in time the subsequent executions of the transformations in Line~\ref{algo:orchestration:application:line:stepcalculation} of \autoref{algo:orchestration:application} repeatedly produce the same sequence of results, i.e., of changes.
% Divergence means that from some point in time all results, i.e., changes, produced in Line~\ref{algo:orchestration:application:line:stepcalculation} differ.
In contrast to non-termination in general, the scenario of alternation can at least be avoided by construction.
% To this end, the history of change produced by the algorithm in Line~\ref{algo:orchestration:application:line:stepcalculation} has to be stored.
% It can either be used by the \function{Apply} function to detect alternation or by passing it to the \function{Orchestrate} function to influence the selection of transformations to avoid alternation.

\begin{definition}[Alternation of Apply Algorithmus]
    \label{def:applyalternation}
    Let there be a number $n$ of execution of the transformation execution loop in Lines~\autoref{algo:orchestration:application:line:startorchestrate}--\autoref{algo:orchestration:application:line:endorchestrate} of \autoref{algo:orchestration:application}, such that for all numbers of execution $> n$ there is a sequence of executed transformations and generated changes that occur at least two times subsequently at the end of the current states of $\mathvariable{executedTransformations}$ and $\mathvariable{generatedChanges}$.
\end{definition}

\mnote{Orchestrate function can detect alternation}
The \function{Orchestrate} function receives the history of transformations and changes and is thus able to identify the situation that the same sequence of transformations was already executed and produced equal changes in each application.
This allows it to implement the function in a way that it does not return the same sequence of transformations when it was already passed and produced the same changes.
If a concrete realization of the \function{Orchestrate} function is not implemented in a way that it can react to the detection of alternation and produce a different sequence of transformations, it can at least return $\bot$ to ensure termination of \function{Apply}, because repeated execution of the same transformations will still returns the same changes. 

\mnote{Avoiding alternation improved possibility to find consistent orchestration}
Alternation produces orchestrations that can never yield consistent models, thus they are part of the problem space $P = {\transformationset{T}^{< \mathbb{N}}}$ of finding an orchestration for a given input $i$ of models and changes but can never be part of the solution space $S_{i} = \abs{\setted{\tuple{\transformation{t}} \in \transformationset{T}^{< \mathbb{N}} \mid \tuple{\transformation{t}} \mathtext{is consistent orchestration for} \tupled{\modeltuple{m}, \changetuple{\metamodeltuple{M}}}}}$ containing the consistent orchestrations.
Avoiding such alternations thus reduces the problem space without affecting the solution space and thus improves the possibility to find a consistent orchestration, as shown in the previous subsection.

%Divergence: If it passes the same changes again, then it either does not pass those changes again 

% Derive from the previous insights that whenever the algorithm does not terminate, we can have two situations: divergence and alternation.
% Prove that no other options for occurring situation exist!

% Problemraum:
% \begin{itemize}
%     \item Ziel ist, dass ein Netzwerk von Transformationen nach einer Änderung in einem konsistenten Zustand terminiert. D.h. Korrektheit stellt Anforderungen an \emph{Terminierung}, sowie den \emph{Zustand} bei Terminierung.
%     \item Folgende Abweichungen davon können auftreten:
%     \begin{enumerate}
%         \item Nicht-Terminierung: Das Netzwerk terminiert nicht. Das bedeutet im Prinzip, dass die Ausführungsfunktion (bzw. der Laufzeit-Algorithmus, der die Funktion dynamisch emuliert) nicht \emph{sound} ist. Soundness der Ausführungsfunktion setzt voraus, dass die berechnet Aufrufsequenz endlich ist. Wenn die Ausführung nicht terminiert, bedeutet das, dass entweder die gleichen Zustände mehrfach durchlaufen werden oder eine Sequenz unendlich vieler Zustände produziert wird. Denn wenn beides nicht der Fall ist, gibt es eine endliche Sequenz unterschiedlicher Zustände, d.h. Terminierung. Das bedeutet, dass es folgende zwei Möglichkeiten gibt:
%         \begin{itemize}
%             \item Alternierung: Die gleichen Zustände werden mehrfach durchlaufen.
%             \item Divergenz: Es werden unendlich viele Zustände produziert.
%         \end{itemize}
%         \item Inkonsistente Terminierung: Die Ausführungsfunktion bzw. der Algorithmus beendet die Ausführung, aber in einem inkonsistenten Zustand. Hier lassen sich ebenfalls wieder zwei Fälle unterscheiden.
%         \begin{itemize}
%             \item Unerkannte Inkonsistenz: Der Algorithmus terminiert und denkt, der Zielzustand wäre konsistent. Dies bedeutet aber direkt, dass nicht alle Konsistenzrelationen erfüllt sind, was, zumindest in der Theorie, einfach zu prüfen wäre (entweder durch Prüfung der Relationen oder durch Ausführung der hippokratischen Transformationen, die alle nichts tun dürften)
%             \item Erkannte Inkonsistenz: Der Algorithmus terminiert, wissend dass die Lösung nicht konsistent ist. Dies kann entweder sein, weil eine Transformation für zwei Modelle in einem inkonsistenten Zustand nicht mehr anwendbar ist, oder weil irgendein anderes Abbruchkriterium erreicht ist.
%         \end{itemize}
%     \end{enumerate}
% \end{itemize}

% Assume we have an algorithm that sequentially applies transformations.
% It stops as soon as a transformation cannot be applied or the models are consistent.
% Then we need to guarantee termination.
% We need to avoid that transformations can be applied indefinitely never leading to consistent models.

% Reasons for this situation are alternation and divergence.
% Prove that if we do not pass the same model state again (alternation) and if there is no indefinite number of model states (divergence), the algorithm terminates.
% Thus, if we ensure that any execution order of transformations does never lead to alternation and divergence, we know that the algorithm terminates!!

% \begin{itemize}
%     \item Zeigen, dass es Beispiele gibt, in denen es unabhängig von der Ausführungsreihenfolge immer zu einer Alternierung kommt
%     \item Zeigen, dass es Beispiele gibt, in denen es unabhängig von der Ausführungsreihenfolge immer zu einer Divergenz kommt.
%     \item Die Beispiele sollten zeigen, dass wir keine Einschränkungen an die Transformationen machen können, was das Problem aushebelt. D.h. egal welche Einschränkungen ich an die Transformationen definiere, es lassen sich immer Beispiele konstruieren, in denen es keine Ausführungsreihenfolge gibt, in denen sie terminieren.
%     \item Mathematisch zeigen, dass Alternierung und Divergenz die einzigen Probleme sind. D.h. wenn nicht der gleiche Zustand mehrmals durchlaufen wird (Alternierung) und es nicht unendlich viele Zustände gibt (Divergenz), dann ist die Folge endlich.
%     %\item Außerdem mathematisch die Abbildung von Transformationen auf Turing-Maschinen zeigen und damit ableiten, dass allgemeine Netzwerke erstmal nicht terminieren müssen (Abbildung auf Halteproblem)
% \end{itemize}

% To avoid these problems by construction, we discussed before that we need to achieve that P = S, such that the application function can execute transformations in an arbitrary order to achieve consistency.

% Another possibility would be to allow the problems and detect them dynamically and react to them.
% We will finally discuss that in the last section.

%In the following, we discuss whether and how we may restrict synchronizing transformations, such that an arbitrary execution order can avoid divergence and alternation, such that the algorithm terminates.


\subsection{Monotony for Avoiding Alternation}

\mnote{Non-alternation by construction instead of dynamic detection}
We have discussed %in \autoref{chap:orchestration:conservative:alternation}
that alternation, as a specific kind of non-termination scenario, can be avoided by construction of the orchestration function or at least can be detected by the \function{Apply} algorithm.
Instead of detecting alternation during orchestration, we may also restrict the transformation network such that no alternation can occur by construction.
We can achieve this by defining a notion of monotony for the transformations.

\mnote{Monotony notion from synchronizing transformations is not applicable}
For the construction of synchronizing bidirectional transformations by unidirectional consistency preservation rules in \autoref{chap:synchronization:bidirectional:transformations}, we have have defined the property of \emph{partial consistency improvement}, which is a monotony notion for the two unidirectional consistency preservation rules of a synchronizing bidirectional transformation, as each execution of them improved that property.
We can, however, not define monotony in a similar way for the whole transformation network because of two reasons.
First, the notion of partial consistency is not applicable for transformation networks, because each transformation needs to restore consistency between two models completely.
Second, since each transformation is developed independently from all others, we cannot apply the notion of partial consistent improvement to the other models by restricting how far a transformation may violate consistency to the other transformations.

We thus define a different notion of monotony for transformations as follows.
\begin{definition}[Monotone Synchronizing Transformation]
    \label{def:monotonetransformation}
    Let $\metamodeltuple{M} = \metamodelsequence{M}{n}$ be metamodels and let $\transformation{t}$ be a synchronizing transformation. We call $\transformation{t}$ monotone if it does not change elements that were already changed, i.e.
    \begin{align*}
        &
        \forall \modeltuple{m} = \tupled{\model{m}{1}, \dots, \model{m}{n}} \in \metamodeltuple{M}, \changetuple{\metamodeltuple{M}} = \tupled{\change{\metamodel{M}{1}}, \dots, \change{\metamodel{M}{n}}} \in \changeuniverse{\metamodeltuple{M}} : \\
        &
        \bigl(\exists \changetuple{\metamodeltuple{M}}' = \tupled{\change{\metamodel{M}{1}}', \dots, \change{\metamodel{M}{n}}'} \in \changeuniverse{\metamodeltuple{M}} : \generalizationfunction{\metamodeltuple{M},\transformation{t}}(\modeltuple{m}, \changetuple{\metamodeltuple{M}}) = \changetuple{\metamodeltuple{M}}' \\
        & \formulaskip
        \Rightarrow
        % WE CANNOT ASSUME TRANSFORAMTION TO BE STRONG MONTONE, BECAUSE IF TRANSFORMATION IS EXECTED FOR ALREADY CONSISTENT MODELS, IT CANNOT CHANGE ANYTHING
        % (\changetuple{\metamodeltuple{M}}'(\changetuple{\metamodeltuple{M}}(\modeltuple{m})) = \changetuple{\metamodeltuple{M}}(\modeltuple{m}) \Rightarrow \modeltuple{m} \consistenttomath \transformationset{T}) \\
        % & \formulaskip
        % \land 
        \forall i \in \setted{1, \dots, n} : 
        (\change{\metamodel{M}{i}}(\model{m}{i}) \setminus \model{m}{i} \subseteq \change{\metamodel{M}{i}}'(\change{\metamodel{M}{i}}(\model{m}{i})) \\
        & \formulaskip\formulaskip
        \land
        (\model{m}{i} \setminus \change{\metamodel{M}{i}}(\model{m}{i})) \cap \change{\metamodel{M}{i}}'(\change{\metamodel{M}{i}}(\model{m}{i})) = \emptyset)
        \bigr)
        %\change{\metamodeltuple{M}}(\modeltuple{m}) \subseteq \modeltuple{m} \cup \changetuple{\metamodeltuple{M}}'(\changetuple{\metamodeltuple{M}}(\model{m}{}))
        %\land
        %\modeltuple{m} \cup \changetuple{\metamodeltuple{M}}'(\changetuple{\metamodeltuple{M}}(\model{m}{})) \subseteq \changetuple{\metamodeltuple{M}}(\model{m}{})\big)
    \end{align*}
\end{definition}

\mnote{Montone transformations do not repeatedly change the same elements}
The definition is based on the idea that transformations are only supposed to append changes but not to revert previous changes.
This means that elements that were introduced by previous changes still need to be present after applying the transformation.
Additionally, elements that were removed are not allowed to be added by the transformation again.
Thus all elements of the originally changed models were either contained in the original models or are contained in the models yielded by the transformation application, which leads to the model relations in the definition.
Additionally, 

% \begin{definition}[Strongly Montone Synchronizing Transformation]
%     Let $\metamodeltuple{M}$ be metamodels and let $\transformation{t}$ be a monotone synchronizing transformation. We call $\transformation{t}$ strongly monotone if it does not perform any changes only when all models are already consistent does not change elements that were already changed, i.e.
% \end{definition}

\mnote{Orchestrations of monotone transforamtion yield squences of pairwise different model states}
Having only monotone transformations ensures that each orchestration, which does not apply a transformation to already consistent models, yields a sequence of pairwise different model states, if the transformations are sequentially applied.
\begin{lemma}
    \label{lemma:monotonetransformationsnosamestates}
    Let $\transformationset{T}$ be a set of correct monotone synchronizing transformations for a tuple of metamodels $\metamodeltuple{M}$.
    Then for all models and changes, as well as any orchestration $\tupled{\transformation{t}_{1}, \dots, \transformation{t}_{m}}  \; (\transformation{i} \in \transformationset{T})$ that does contain a transformation when its models are already consistent, then prefixes of that orchestration only yield the same models if those prefixes are consistent orchestrations, i.e.
    \begin{align*}
        &
        \forall \modeltuple{m} \in \metamodeltupleinstanceset{M}, \changetuple{\metamodeltuple{M}} \in \changetuple{\metamodeltuple{M}} : \forall i, k \in \setted{1, \dots, m} : \\
        &
        \generalizationfunction{\metamodeltuple{M}, \transformation{t}_{i}} \concatfunction \dots \concatfunction \generalizationfunction{\metamodeltuple{M}, \transformation{t}_{1}}(\modeltuple{m}, \changetuple{\metamodeltuple{M}}) = \generalizationfunction{\metamodeltuple{M}, \transformation{t}_{k}} \concatfunction \dots \concatfunction \generalizationfunction{\metamodeltuple{M}, \transformation{t}_{1}}(\modeltuple{m}, \changetuple{\metamodeltuple{M}}) \\
        & \formulaskip 
        \Rightarrow
        \generalizationfunction{\metamodeltuple{M}, \transformation{t}_{k}} \concat \dots \concat \generalizationfunction{\metamodeltuple{M}, \transformation{t}_{1}}(\modeltuple{m}, \changetuple{\metamodeltuple{M}}) \consistenttomath \transformationset{T}
        \end{align*}
\end{lemma}
\begin{proof}
    Assume that there are two prefixes $\tupled{\transformation{t}_{1}, \dots, \transformation{t}_{i}}$ and $\tupled{\transformation{t}_{1}, \dots, \transformation{t}_{k}}$ of an orchestration, $i < k$ without loss of generality, such that they yield the same inconsistent models, i.e., $\generalizationfunction{\metamodeltuple{M}, \transformation{t}_{i}} \concatfunction \dots \concatfunction \generalizationfunction{\metamodeltuple{M}, \transformation{t}_{1}}(\modeltuple{m}, \changetuple{\metamodeltuple{M}}) = \generalizationfunction{\metamodeltuple{M}, \transformation{t}_{k}} \concatfunction \dots \concatfunction \generalizationfunction{\metamodeltuple{M}, \transformation{t}_{1}}(\modeltuple{m}, \changetuple{\metamodeltuple{M}})$ although $\generalizationfunction{\metamodeltuple{M}, \transformation{t}_{k}} \concatfunction \dots \concatfunction \generalizationfunction{\metamodeltuple{M}, \transformation{t}_{1}}(\modeltuple{m}, \changetuple{\metamodeltuple{M}})$ is not consistent to $\transformationset{T}$.
    We denote the change tuple delivered by any prefixes of length $l$ as $\changetuple{\metamodeltuple{M},l} = \tupled{\change{\metamodel{M}{1},l}, \dots, \change{\metamodel{M}{n}, l}}$ with $\tupled{\modeltuple{m}, \changetuple{\metamodeltuple{M},l}} = \generalizationfunction{\metamodeltuple{M}, \transformation{t}_{l}} \concatfunction \dots \concatfunction \generalizationfunction{\metamodeltuple{M}, \transformation{t}_{1}}(\modeltuple{m}, \changetuple{\metamodeltuple{M}})$.
    We know that the sequence of changes between the two prefixes does not perform any changes and thus acts like the identity function, i.e., $\generalizationfunction{\metamodeltuple{M}, \transformation{t}_{k}} \concatfunction \dots \concatfunction \generalizationfunction{\metamodeltuple{M}, \transformation{t}_{i+1}}(\modeltuple{m}, \changetuple{\metamodeltuple{M},i}) = \identitychange(\modeltuple{m}, \changetuple{\metamodeltuple{M},i})$
    and thus $\changetuple{\metamodeltuple{M},i}(\modeltuple{m}) = \changetuple{\metamodeltuple{M},k}(\modeltuple{m})$.
    We also know that all the transformations between the prefixes, i.e., all transformations $\transformation{t}_{l}$ for each $l$ with $i < l \leq k$, do not act like the identity function for their inputs, i.e., $\generalizationfunction{\metamodeltuple{M}, \transformation{t}_{l}}(\modeltuple{m}, \changetuple{\metamodeltuple{M},l-1}) \neq \identitychange(\modeltuple{m}, \changetuple{\metamodeltuple{M},l-1})$.
    Otherwise, the models affected by the transformation would either have been consistent before, which conflicts with the assumption that the orchestration does not contain a transformation when its models are already consistent, or they would not be consistent afterwards, which conflicts with the assumed correctness of the transformations.

    Thus, each transformation $\transformation{t}_{l} \; (i < l \leq k)$ performs modifications to the change tuple, i.e., adds or removed further elements.
    This especially applies to $\transformation{t}_{i+1}$.
    Let us assume that $\transformation{t}_{i+1}$ adds an element (analogous argumentation for the removal).
    %modifies the change tuple such that it adds or removes further elements.
    Then there is a model that contains the element after applying the change generated by the transformation, i.e., $\exists s \in \setted{1, \dots, n} : \exists \modelelement{e} : \modelelement{e} \in \change{\metamodel{M}{s},i+1}(\model{m}{s}) \setminus \change{\metamodel{M}{s},i}(\model{m}{s})$.
    Due to the transformations being monotone, we know that this element was not contained before, especially not in $\model{m}{s}$, as otherwise $\modelelement{e} \in \model{m}{s} \setminus \change{\metamodel{M}{s},i}(\model{m}{s})$ and thus $\model{m}{s} \setminus \change{\metamodel{M}{s},i}(\model{m}{s}) \cap \change{\metamodel{M}{s},i+1}(\model{m}{s}) \neq \emptyset$, which conflicts the definition of monotone transformations for $\transformation{t}_{i+1}$.

    Since $\change{\metamodel{M}{s},k}(\model{m}{s}) = \change{\metamodel{M}{s},i}(\model{m}{s})$, we know that $\modelelement{e} \not\in \change{\metamodel{M}{s},k}(\model{m}{s})$.
    Thus, there must be a transformation $\transformation{t}_{l}$ with $i+1 < l \leq k$ which, in turn, removes this element, i.e., $\modelelement{e} \in \change{\metamodel{M}{s},l-1}(\model{m}{s}) \setminus \change{\metamodel{M}{s},l}(\model{m}{s})$.
    Then $\modelelement{e} \in \change{\metamodel{M}{s},l-1}(\model{m}{s}) \setminus \model{m}{s}$ and thus $\change{\metamodel{M}{s},l-1}(\model{m}{s}) \setminus \model{m}{s} \not\subseteq \change{\metamodel{M}{s},l}(\model{m}{s})$, which conflicts the definition of monotone transformations for $\transformation{t}_{l}$.

    In consequence, each transformation $\transformation{t}_{l} \; (i < l \leq k)$ can neither add nor remove an element, thus our assumption that two prefixes that yield the same inconsistent models does not hold, which proves the lemma.
\end{proof}

\mnote{Monotone transformation prevent alternation}
With that insight, it is easy to see that given only monotone transformation, no alternation can occur in our algorithm \autoref{algo:orchestration:application}.
\begin{theorem}
    Given a set of correct, monotone synchronizing transformations $\transformationset{T}$.
    Then \autoref{algo:orchestration:application} cannot contain an alternation according to \autoref{def:applyalternation}, as long as $\function{Orchestrate}$ does not return a transformation whose models are already consistent.
\end{theorem}
\begin{proof}
    According to \autoref{lemma:monotonetransformationsnosamestates}, monotone transformations ensure that in an orchestration that does not contains transformations that need to be applied to already consistent models the application of two prefixes never yields the same changes.
    In consequence, the sequence of $\mathvariable{generatedChanges}$ in the transformation application in 
    Lines~\autoref{algo:orchestration:application:line:startorchestrate}--\autoref{algo:orchestration:application:line:endorchestrate} of \autoref{algo:orchestration:application} can never contain the same two changes.
    This would, however, be necessary to fulfill \autoref{def:applyalternation} for alternation.
\end{proof}

\mnote{Guarantee by monotone transformations is stronger than non-alternation}
In fact, the guarantee of not producing the same state twice is even stronger than non-alternation, because alternation allows to pass the same state multiple times, as long as the same sequence of states is not passed repeatedly and infinitely.
It does, however, only make sense to pass the same state twice if the orchestration algorithm that selects the next transformation to execute is able to process that situation by trying different execution orders if an alternation occurs.
Thus, the less strict requirement for alternation is suited to make statements about the orchestration strategy but not about the individual transformations, as it is unlikely to find a property for a single transformation that gives a guarantee that depends on the execution order of transformations, like alternation does.

\mnote{Monotony cannot be generally assumed}
While monotone transformation give the guarantee of non-alternation, monotony according to \autoref{def:monotonetransformation} is not a property that we cannot assume to be fulfilled by all transformations.
Although is seems intuitive that a transformation should not remove elements that were added before and vice versa, this does also mean that, for example, an attribute value may only be changed once by the transformations.
This would, however, require the transformations to always make a choice for attributes that fits for all other transformations as well.
We have seen in different examples, such as the one depicted in \autoref{fig:orchestration:no_upper_bound} and \autoref{fig:orchestration:no_orchestration}, that it may be necessary to change elements multiple times, because the transformations select values with which the models only fulfill their own consistency relation but not those of the other transformations.
It may take several executions to find a value selection with which the models are consistent to all transformations.
We might say that the transformations need to \emph{negotiate} a consistent solution.

\mnote{Ensuring monotony as often as possible improves possibility of finding consistent orchestration}
Still, the given examples were rather artificial, so they cannot be seen as an indicator for monotony to be not practically achievable.
It may, at least in some cases, be possible to specify transformations that are monotone.
Even if only some of the transformations are monotone, or if only specific rules of them are monotone, it improves the chance that an orchestration strategy finds a consistent orchestration.
Having the knowledge about the benefits of monotony gives a transformation developer the ability to implement it as often as possible.

\mnote{Combining alternation avoidance and dynamic detection}
Finally, the possibility to avoid alternation by construction can be combined with the ability of an orchestration strategy to react to alternation.
We have discussed in \autoref{chap:orchestration:conservative:alternation} that an orchestration strategy can detect alternation and adapt its strategy of selecting the next transformation in that case.
In addition, if monotony is given at least for some transformations, the orchestration strategy needs to try less execution orders and thus improves the chance of finding a consistent orchestration.

% \begin{figure}
%     \centering
%     \includegraphics[width=\textwidth]{figures/correctness/orchestration/monotony_counterexample.png}
%     \caption{Counterexample for monotony}
%     \label{fig:formal:monotonycounterexample}
% \end{figure}

% \begin{itemize}
%     \item Muss eine Transformation mit jedem beliebigen Delta umgehen können müssen? Eine Einschränkung auf Monotonie würde dies verhindern. Bzw. wir müssten zeigen, dass es Konsistenzrelationen gibt, die unter der Anforderung an Monotonie nicht wiederhergestellt werden können. Bspw. fügt eine andere Transformation 3 Elemente hinzu, wo zwei mit dem anderen entsprechend der Konsistenzrelationen korrelieren und somit keine Witness-Struktur aufgebaut werden kann, die Konsistenz beweist. Das lässt sich durch Hinzufügen weiterer Elemente potentiell nicht auflösen (siehe Beispiele im SoSym-Paper).
%     \item Refer to synchronization chapter, where we introduced a monotony notion based on transformations being partial-consistency-improving. Here, in contrast, the CPRs cannot be aligned, such that we cannot, for example, expect one transformation not to lead to a reduction of consistency regarding consistency relations of other, previously executed transformations.
% \end{itemize}
% \begin{itemize}
%     \item Im Allgemeinen könnte eine Transformation beliebige dieser Deltasequenzen modifizieren. Wir verlangen jedoch, dass eine Transformation nur Deltas anhängt, also die Sequenzen länger werden
%     \item Genauer beschränken wir auch, welche Sequenzen eine Transformation sehen und ändern darf, genau gesagt darf sie die Sequenzen von zwei Modellen sehen und eine davon verlängern.
%     \item Hier kommt bereits der Unterschied zu bisherigen Transformationen, denn die sehen nur Deltas an einem Modell und erzeugen Deltas an dem anderen. Das ist bei uns schon gänzlich anders. Bidirektionale Transformationen unterstützen das im Übrigen auch nicht, sondern sind nur Spezifikationen, aus denen sich Wiederherstellungsroutinen für beide Richtungen ableiten lassen (siehe Stevens 2010)
% \end{itemize}

% \paragraph{Idea:} Require monotony to avoid alternation

% We would have to relax the definition of transformation to be monotone, because if a transformation is monotone, it may only append information, but this is not always possible, as can be seen in the following example. A monotone transformation must be able to return bottom if it cannot make further changes to restore consistency to the relation.

% \begin{definition}[Monotone Transformation]
%     Transformation gets models M and deltas D and produces new deltas D'. Taking the union of the original models M and the new models D'(M), then D(M) must be a subset of that, because other elements would have been added and removed afterwards or elements would have been changes once by D and again in a different way by D'.

%     Generally, monotony could also mean that only the same complete model state is not passed twice. \todo{Why dont we do that?}
% \end{definition}

% This would mean that each transformation only appends changes, i.e., if an element was added/removed, the transformation may not do the inverse. The same applies to attribute/reference changes: if an attribute/reference was already changes it may not be changed again.
% This way, it is by design impossible to pass through the same state again. Actually, if a monotone transformation returns bottom, the network has to terminate with a failure.
% However, this is hard restriction to transformations. It leads to the fact that in some networks that actually have a simple solution no solution is found at all. This can be easily seen at the example in \autoref{fig:formal:monotonycounterexample}. In the example adding "aa" to the left model, any execution order of the transformations leads to the situation that a previous change must be revoked to result in a consistent state. However, it is possible to derive a consistent state for that input change.

% \begin{figure}
%     \centering
%     \includegraphics[width=\textwidth]{figures/correctness/orchestration/monotony_counterexample.png}
%     \caption{Counterexample for monotony}
%     \label{fig:formal:monotonycounterexample}
% \end{figure}

% One could now argue that there are binary relations in the example, which may never be fulfilled at all. We will later discuss how far relations that cannot be fulfilled should be restricted. However, in general, this is wanted behavior, because in general it may be necessary that transformations produce intermediate states that are not yet consistent with each other. Otherwise this would means that each transformation is always able to directly deliver a state that is consistent to all other relations, which is especially not possible, because other transformations may add further information to the models. More precisely, a relation may consider <a model consistent to all other models that contain any additional information not affected by the transformation. For example, a UML class model may be considered consistent to all Java models with any implementation of the specified methods, thus to an infinite number of models. Now saying that it should not be allowed that the transformation selects one with an empty implementation because that is not consistent to another relations induced by another transformation, such as the relationship to a component model, does not make any sense. Thus having those relation elements that may be considered locally consistent but will never occur in a globally consistent tuple of models does not make sense.
% In the example, we can see that such an inconsistent intermediate state is passed through and afterwards a consistent tuple of models is reached if not requiring monotony.
% In consequence, requiring monotony from transformations is a too strict requirement, because it is necessary to run through states that may be changed later on.

% \begin{theorem}
%     An application function for monotone transformations either returns a consistent model or produce a sequence of CPRs returning delta that return models of always growing size (i.e. it diverges).
% \end{theorem}

% \paragraph{Divergence cannot be avoided}

% There are rather equal network, one that terminates after a long time and one that never terminates. 
% Consider the example. The relations are defined in a way such that for any allocation for any of them a consistent tuple of models can be found. However, the transformations are not able to find it because they make "bad" choices from a set of choices that are conflicting. 
% This can be seen in the example that we have already given in \autoref{fig:correctness:no_execution_order}.

% Thus, systematically avoiding divergence is not possible. 

% \textbf{Central insight:} Alternation / Divergence cannot be avoided systematically (like in ordinary programming), if not restricting transformations in a way that may not be reasonable.


% \subsection{Unresolvability}

% Discuss why no execution order may exist although relations are compatible.
% If not even an order exists, the application function or the algorithm can, for sure, not find it.

% However, we found that we cannot always find an execution order if it exists and we were not able to find restrictions to transformations to ensure that it exists.
% We expect the same for the existence of an execution order at all.
% All restrictions we can make are likely to be too restrictive.
% The problem arises when there is an overlap of consistent models between some transformations, but they always decide for other elements that are not in the overlap of consistent models.
% It would, obviously, require the transformation to know about the others to ensure that this is not the case.
% This conflicts our assumption.

% Finally, it may be valid that for some changes no execution exists, because the change can not be processed on purpose \todo{Give example for that!}.
% Should this be the case if we assume compatibility?

% Although a more detailed investigation of the claim that we cannot define reasonable requirements to the transformations to ensure that they can always be ordered to restore consistency is a topic for further research, we did not investigate it in the scope of this thesis.
% Since we found it necessary to find a conservative algorithm that can deal with the case that no execution is found anyway, that algorithm covers the case that no execution order exists as well and thus is a solution for this problem as well.

% Beispiel:
% \begin{itemize}
%     \item Das ist im allgemeinen aber nicht Fall. Letztendlich trifft jede Transformation lokale Entscheidungen. Beispielsweise könnte jede einzelne Transformation gegeben eine beliebige Änderung immer dieselben Modelle (bzw. Änderungen die dazu führen) zurück liefern (im trivialsten Fall leere Modelle). Dann erfüllt jede Transformation ihre Korrektheitseigenschaft bzgl. ihrer Relation, aber das Netzwerk muss nicht korrekt sein, da bspw. T(A,B) und T(B,C) sich immer für verschiedene Instanzen von B entscheiden. Es gäbe somit nie eine konsistente Lösung für eine beliebige Ausführungsreihenfolge der Transformationen, auch wenn die Relationen das erlauben würden.
%     \item Beispiel mit Namen, wo eine Transformation immer den großen Namen zurück liefert, die andere immer den kleinen. T(A,B) bildet A auf gleiches B ab und beide auf kleine Schreibweise, obwohl beide erlaubt sind. Erzeuge A="a", dadurch B="a". T(B,C) bildet B auf C ab und beide auf große Schreibweise, obwohl beide erlaubt sind. Somit macht sie das zu B="A" und C="A". Nun wird T(A,B) wieder beide klein machen usw. Allerdings wäre eine insgesamt valide Lösung einfach alle groß oder alle klein zu machen, aber die Transformationen finden diesen Zustand nicht. 
% \end{itemize}


\section{A Conservative Application Algorithm}

\mnote{Correct algorithm that always terminates}
We have argued why it is inevitable that any algorithm realizing an application function cannot be optimal and thus will not be able to find a consistent orchestration although it exists and, in that case, either return $\bot$ or not even terminate at all.
Apart from minor improvements, such as the avoidance or detection of alternations, to improve the probability to find a consistent orchestration, or general strategies like backtracking for trying different orchestrations, we did not find systematic ways to improve optimality of the application function.
Nevertheless, we want to find an algorithm that is at least correct and does always terminate, even if it does implement a systematic way to improve optimality, thus which is conservative.

\mnote{Artificial execution bound can prevent from finding consistent orchestration}
\autoref{algo:orchestration:application} may not terminate, because it generates an infinitely long orchestrations, thus never leaving the loop in Lines~\ref{algo:orchestration:application:line:startorchestrate}--\ref{algo:orchestration:application:line:endorchestrate}.
Thus, to ensure termination we need to introduce an upper bound for the number of executed transformations.
We identified in \autoref{chap:orchestration:decidability:correctness_termination} that there is no upper bound for the number of necessary executions, thus even the shortest consistent orchestration for specific inputs can be arbitrarily long (see example in \autoref{fig:orchestration:no_upper_bound}).
Any arbitrary bound can prevent the algorithm from finding consistent orchestrations.

\mnote{High numbers of transformation executed are not wanted in practice}
From an engineers perspective, we may, however, consider the behavior that an arbitrary high number of transformations execution is required to yield consistent models as unwanted behavior.
Although the examples we have given are valid, they are rather artificial.
We claim that a transformation network that requires a rather high number of executions compared to the number of contained transformations to find consistent models does not operate as expected.
In particular, if such a high number of executions is required to find a consistent orchestration, it will be difficult to identify the reason for not finding a consistent execution in case the algorithm returns $\bot$.
Thus, we introduce an artificial upper bound for the number of transformation executions.
That bound will be well-defined, such that we can reasonably assume that no more execution are practically necessary.

\mnote{Design goals of orchestration, algorithm and termination}
In the following, we propose design goals for a conservative application algorithm and the so called \emph{provenance algorithm} itself and finally prove its correctness and termination properties.
The work was developed together with a student in a kind of scientific internship and published in \cite{gleitze2020orchestration}.

% FROM MODELS:
% Finally, one could question whether it is relevant if an executionstrategy can be guaranteed to terminate. Execution strategies will beused to tell users whether changes they made can be incorporatedinto the other models automatically. In consequence, users shouldreliably and timely get a response. We might compare this situationto merging changes in version control systems. There, users alsowant a reliable and timely response on whether their changes couldbe incorporated automatically, or whether they need to resolveconflicts manually

% From an engineering perspective, unbounded numbers of executions is still unwanted behavior. We claim that a transformation network that takes thousands of executions of the same transformation to find a consistent state works not as expected and if running into a failure would expose severe problems to find the reasons for that failures.
% Thus, we propose to simply abort the execution after some time to be sure not to run in an endless loop.

% Finally, this problem is comparable to ordinary programming, because there the same situations regarding alternation and divergence can occur that result in non-termination of a program.
% As we all know, it is impossible to systematically avoid that, but just possible to carefully develop the program and apply best practices to avoid such situations.

% \textbf{Conclusion:} Apart from minor discussed improvements, we cannot systematically improve optimality / reduce conservativeness and also not dynamically terminate when divergence is detected. Especially, there is no guarantee wo when also alternation is detected. There can be an arbitrary high number of execution before alternation is detected.
% Thus, we propose an algorithm that terminates deterministically and returns $\bot$ also when it is not able to find a solution in a fixed number of steps, but improves the ability for the transformation developer or user to find out why in specific situation no solution can be found.

% \todo{Question whether having no upper bound is only a theoretical or also a practical problem. The given example is rather artificial, so we may stop in practice without excluding relevant cases}

\subsection{Design Goals}

\mnote{Degrees of freedom in algorithm}
An adapted version of \autoref{algo:orchestration:application:line:endorchestrate} that always terminates has two degrees of freedom.
First, the execution order of transformations needs to be determined by defining the function \function{Orchestrate}.
Second, an upper bound for the number of executions of transformations, thus the loop in Lines~\ref{algo:orchestration:application:line:startorchestrate}--\ref{algo:orchestration:application:line:endorchestrate}, needs to be defined.

\mnote{Execution order is important for identfying reasons for failing}
We have already discussed that improving optimality is not an achievable goal when determining the transformation execution order by the \function{Orchestrate} function.
Since we know that the algorithm will always produce false negatives, i.e., it will not find a consistent orchestration although it exists, it is important for a transformation developer or user to be able to identify the reasons for that in practice.
The algorithm can support them in this regard by delivering the final state of the models when the orchestration aborted.
The execution order that was chosen until that state was reached is of central importance of identifying the reasons for failing.
Consider that transformations are executed in an arbitrary order and then only some of the models of the final state are actually consistent.
Apart from investigating the complete sequence of executed transformations, there is clue for the user to find the reasons for the algorithm to fail, thus about \emph{provenance} of the error.

\mnote{Orchstration principle to incrementally restore consistency}
To improve identifying provenance whenever the algorithm fails, we propose the following principle for determining an orchestration:
\begin{quote}
    \enquote{Ensure consistency among the transformations that have already been executed before executing a transformation that has not been executed yet.} \cite{gleitze2020orchestration}
\end{quote}
The principle requires that consistency is ensured iteratively for subsets of the transformations and thus the models.
As long as the models are not consistent to all already executed transformations, the orchestration may only execute those transformations instead of new ones until the models are consistent to all of them.
This ensures that consistency is preserved after each change in an incremental way, iteratively increasing improving the set of models and transformations for which consistency is restored.

\mnote{Benefits of orchestration principle}
This approach improves identifying provenance, because it restricts the potentially causal transformations to consider.
If the algorithm fails after executing a subset of the transformations $\transformationset{T}_{exec} \subseteq \transformationset{T}$.
Then there is some transformation $\transformation{t} \in \transformationset{T}_{exec}$ that was executed for its first time last.
Thus, the algorithm found an orchestration of $\transformationset{T}_{exec} \setminus \transformation{t}$ such that the models were consistent to all those transformation, but is was not able to execute $\transformation{t}$ and the transformations in $\transformationset{T}_{exec}$ afterwards, such that the models become consistent to all these transformations.
This helps the transformation developer or user to understand and find the reason for failing in different ways.
First, he or she can ignore any transformation in $\transformationset{T} \setminus \transformationset{T}_{exec}$, as the algorithm already failed to preserve consistency according to the other transformations, which can significantly reduce the number of transformations to consider.
Second, the realization of $\transformation{t}$ is somehow conflicting with the other transformations in $\transformationset{T}_{exec}$. This does not necessarily mean that there is something wrong with $\transformation{t}$, but only that also considering that transformation does either induce the situation that no consistent orchestration anymore or that it cannot be found.
Third, having a state of the models that is consistent to $\transformationset{T}_{exec} \setminus \transformation{t}$ can be used as a starting point to either identify the occurring problem or to manually restore consistency of the models.

\mnote{Principle does not improve optimality}
If the algorithm operates according to the introduced principle and is not able to preserve consistency anymore after the orchestration considers an additional transformation $\transformation{t}$, the selected execution order provides the discussed benefits for identifying the reasons for failing.
There may, however, be another orchestration that is able to ensure consistency to $\transformationset{T}_{exec}$. Executing $\transformation{t}$ earlier or also integrating further transformations in $\transformationset{T}$ before ensuring consistency to all transformations in $\transformationset{T}_{exect}$ can, for sure, lead to the situation that the algorithm finds a consistent orchestration.
This can reduce optimality of the realized orchestration function, but we claim the discussed benefits to outweigh that.

\mnote{Transformations should react to each sequence of all other transformations}
We have shown that there is no inherent upper bound for the necessary number of transformation executions.
Rather than specifying a concrete depending, whether it be fixed or depending on the network size, we derive a reasonable artificial bound for the number of execution from a property that we assume reasonable for possible orchestrations of a set of transformations.
The idea of that property is that each transformation should be allowed to react to the execution of each possible sequence of all other transformations.
It should, however, not be necessary that a transformation must be executed again after the other transformations reacted the execution of that transformation.
Thus, if a transformation was executed after applying the other transformations in any possible order, we expect the models to be consistent to that transformation.

\begin{definition}[Reactive Converging Synchronizing Transformations]
    \label{def:reactiveconverging}
    A set of synchronizing transformations $\transformationset{T}$ is \emph{reactive converging} with respect to models $\modeltuple{m}$ and changes $\changetuple{\metamodeltuple{M}}$ if in any orchestration of $\transformationset{T}$ in which the last transformation $\transformation{t} \in \transformationset{T}$ of that sequence was executed after all other duplicate-free orders of transformations in that order, the yielded models are consistent to $\transformation{t}$.
\end{definition}

\mnote{Example for reactive converging transformations}
The property does not require the other transformation were executed in any order consecutively, but only that the orchestration contains any order of those transformations, but potentially with other transformations in between.
As an example, assume a set of transformations $\transformation{t}_{1}, \transformation{t}_{2}, \transformation{t}{3}$, which is reactive converging for some input of models and changes.
After executing them for these models and changes in the order $\tupled{\transformation{t}_{1}, \transformation{t}_{2}, \transformation{t}{3}, \transformation{t}_{1}, \transformation{t}_{2}, \transformation{t}{3}}$, the models yielded by that orchestration may still be inconsistent to $\transformation{t}_{1}$, because it was not executed after the order of the transformations $\tupled{\transformation{t}_{3}, \transformation{t}_{2}}$.
After executing $\transformation{t}_{1}$ once more, the orchestration must yield consistent models, because $\transformation{t}_{1}$ was executed after the two order of the other transformations $\tupled{\transformation{t}_{2}, \transformation{t}_{3}}$ and $\tupled{\transformation{t}_{3}, \transformation{t}_{2}}$.
Likewise, $\transformation{t}_{2}$ was executed after $\tupled{\transformation{t}_{1}, \transformation{t}_{3}}$ and $\tupled{\transformation{t}_{3}, \transformation{t}_{1}}$, and $\transformation{t}_{3}$ was executed after $\tupled{\transformation{t}_{1}, \transformation{t}_{2}}$ and $\tupled{\transformation{t}_{2}, \transformation{t}_{1}}$.


\subsection{The Provenance Algorithm}

\begin{algorithm}
	\caption{The provenance algorithm in pseudocode, adapted from \cite{gleitze2020orchestration}.}
	\label{algo:orchestration:provenance}
	
	%\SetDataSty{textsf}

    %\hspace*{\algorithmicindent} \textbf{Input:}\\
    %\hspace*{\algorithmicindent} \textbf{Output:} 
    \begin{algorithmic}[1]
    \Procedure{\function{ProvenanceApply}}{$\transformationset{T}, \modeltuple{m} = \tupled{\model{m}{1}, \dots, \model{m}{n}}, \changetuple{\metamodeltuple{M}}$}
        \State $\mathvariable{isConsistent}$ $\leftarrow$ $\function{CheckConsistency}(\transformationset{t}, \modeltuple{m}$)
        \If{$\neg \mathvariable{isConsistent}$}
            \State \Return{$\bot$}
        \EndIf
        $\tupled{\change{\metamodel{M}{1}, \mathvariable{res}}, \dots, \change{\metamodel{M}{1}, \mathvariable{res}}} \gets \function{Propagate}(\transformationset{T}, \modeltuple{m}, \changetuple{\metamodeltuple{M}})$
        \State $\modeltuple{m}_{res} \leftarrow \tupled{\change{\metamodel{M}{1},\mathvariable{res}}(\model{m}{1}), \dots, \change{\metamodel{M}{n},\mathvariable{res}}(\model{m}{n})}$
        \State \Return{$\modeltuple{m}_{res}$}
    \EndProcedure
    \vspace{\baselineskip}
    \Procedure{\function{Propagate}}{$\transformationset{T}, \modeltuple{m} = \tupled{\model{m}{1}, \dots, \model{m}{n}}, \changetuple{\metamodeltuple{M}} = \tupled{\change{\metamodel{M}{1}}, \dots, \change{\metamodel{M}{n}}}$}
        \vspace{0.15\baselineskip}
        \State $\provalgexecuted \gets \emptyset$ \label{line:executed-init}
        %\State $\mathvariable{accumulatedChanges} \gets \changetuple{\metamodeltuple{M}}$         
        \vspace{0.3\baselineskip}
        
        \For{$\provalgcandidate \in \transformationset{T} \setminus \provalgexecuted \mid \changetuple{\metamodeltuple{M}}.\mathvariable{adjacentTo}(\provalgcandidate)$} \label{line:loop-start}
            %\vspace{0.15\baselineskip}
            \State $(\modeltuple{m}, \changetuple{\metamodeltuple{M},\mathvariable{candidate}}) \gets \generalizationfunction{\metamodeltuple{M}, \transformation{t}_{candidate}}(\modeltuple{m}, \changetuple{\metamodeltuple{M}})$ \label{line:first-execution}
            %\State $\transformationset{T}_{\mathvariable{subnetwork}} \gets \transformationset{T}.\mathvariable{edgeInducedSubgraph}(\transformationset{T}_\mathvariable{executed})$
            %\State $\changetuple{\metamodeltuple{M},\mathvariable{propagation}} \gets \function{Propagate}(\transformationset{T}_{\mathvariable{subnetwork}}, \modeltuple{m}, \changetuple{\metamodeltuple{M},\mathvariable{candidate}})$ \label{line:recursive-call}
            \State $\changetuple{\metamodeltuple{M},\mathvariable{propagation}} \gets \function{Propagate}(\provalgexecuted, \modeltuple{m}, \changetuple{\metamodeltuple{M},\mathvariable{candidate}})$ \label{line:recursive-call}
            %\State $\changetuple{\metamodeltuple{M},\mathvariable{candidate}} \gets \mathvariable{candidate}.\mathvariable{execute}(\mathvariable{accumulatedChanges})$ \label{line:check-execution}
            \State $\modeltuple{m}_{\mathvariable{propagation}} \gets \tupled{\change{\metamodel{M}{1},\mathvariable{propagation}}(\model{m}{1}), \dots, \change{\metamodel{M}{n},\mathvariable{propagation}}(\model{m}{n})}$
            \State $\mathvariable{isConsistent}$ $\leftarrow$ $\function{CheckConsistency}(\provalgcandidate, \modeltuple{m}_{\mathvariable{propagation}})$
            \If{$\neg \mathvariable{isConsistent}$}
                \State \Return{$\bot$} \label{line:fail}
            \EndIf
            %\vspace{0.3\baselineskip}
            \State $\changetuple{\metamodeltuple{M}} \gets \changetuple{\metamodeltuple{M},\mathvariable{propagation}}$
            \State $\provalgexecuted \gets \provalgexecuted \cup \setted{\provalgcandidate}$ \label{line:executed-update}
        \EndFor
        \State \Return{$\changetuple{\metamodeltuple{M}}$}
    \EndProcedure
    \end{algorithmic}
	% \Procedure{\(\Propagate\,\mathsf{(}\network,\changes\mathsf{)}\)}{
	% \vspace{0.15\baselineskip}		
	% \(\executed\gets\emptyset\)\;																		\label{line:executed-init}
	% \(\accumulatedChanges\gets\changes\)\;         
	% \vspace{0.3\baselineskip}
	
	% \ForEach{\(\candidate\in\network\mid\candidate\notin\executed\land\accumulatedChanges.\adjacentTo\text{\,(}\candidate\text{)}\)}{	\label{line:loop-start}
	% \vspace{0.15\baselineskip}
	% 		\(\candidateChanges\gets\candidate.\execute\,\text{(}\accumulatedChanges\text{)}\)\;			\label{line:first-execution}
	% 		\(\subnetwork\gets\network.\edgeInducedSubgraph\,\text{(}\executed\)\()\)\;
	% 		\(\propagationChanges\gets\Propagate\,\text{(}\subnetwork,\candidateChanges\text{)}\)\;			\label{line:recursive-call}
	% 		\(\candidateChanges\gets\candidate.\execute\,\text{(}\accumulatedChanges\text{)}\)\;			\label{line:check-execution}
	% 		\lIf{\(\candidateChanges.\adjacentToAny\,\text{(}\executed\text{)}\)}{\(\fail\,\text{()}\)}		\label{line:fail}				
	% \vspace{0.3\baselineskip}
	% 		\(\accumulatedChanges\cupgets\propagationChanges\cup\candidateChanges\)\;
	% 		\(\executed\cupgets\candidate\)\;																\label{line:executed-update}
	% 	}
	% \Return\(\accumulatedChanges\)\;
	% }
\end{algorithm}

\mnote{Provenance algorithm performs orchestration recursively}
We propose an algorithm that realizes the discussed design goal in \autoref{algo:orchestration:provenance}.
The algorithm is a derivation of the general algorithm implementing an application function depicted in \autoref{def:orchestration:application}.
It first checks for consistency of the given models as a prerequisite for executing the transformations.
Then the algorithm calls the recursive function \function{Propagate}, which implements the orchestration of transformations and returns a change tuple that is yielded by the determined orchestrations, which delivers consistent models if applied to the input.
While this behavior is equal to the one in \autoref{def:orchestration:application}, the orchestrate itself is implemented differently in a recursive rather than an iterative manner, which implicitly ensures termination.

\mnote{Recursive orchestration function fulfills principle of incremental consistency restoration}
The function \function{Propagate} implementing the orchestration in a recursive manner acts as follows:
It selects one of the transformations as a candidate to execute next.
This selection ensures that a transformation is selected that is adjacent to any already performed change, such that the transformation needs to perform changes.
It then applies the transformation using the generalization function $\generalizationfunction{}$.
Afterwards, it recursively executes the function \function{Propagate} with the subnetwork given by the transformations that have already been executed.
After that recursive execution it is checked whether the models yielded by the resulting changes are still consistent to the candidate transformation.
If this consistency check fails, the transformations do not fulfill the definition of being reactive converging according to \autoref{def:orchestration:reactiveconverging}, as we will prove later.
If the models are consistent to the transformation, the next candidate is picked.
In effect, the strategy realizes the defined principle in a recursive manner, because after executing a new transformations, the recursive execution ensures consistency to all yet executed transformations.

\begin{figure}
    \centering
    \resizebox{\textwidth}{!}{\usetikzlibrary{arrows.meta}

\newcommand{\modelradius}{0.1em}
\newcommand{\hmodeldistance}{0.98em}
\newcommand{\vmodeldistance}{0.9em}
\newcommand{\hnetworkdistance}{4*\hmodeldistance}
\newcommand{\vnetworkdistance}{4*\vmodeldistance}
\newcommand{\hborder}{0.4*\hmodeldistance}
\newcommand{\vborder}{0.4*\hmodeldistance}
\newcommand{\addnetworktopleft}[3]{
	\node[model,#2] (m#1left) {};
	\node[model, above right=\vmodeldistance and \hmodeldistance of m#1left.center, anchor=center] (m#1top) {};
	\draw[unconsidered, #3] (m#1left) -- (m#1top);
}
\newcommand{\addnetworkmiddle}[2]{
	\node[model, below right=\vmodeldistance and \hmodeldistance of m#1left.center, anchor=center] (m#1bottom) {};
	\draw[unconsidered, #2] (m#1top) -- (m#1bottom);
}
\newcommand{\addnetworktopright}[2]{
	\node[model, right=2*\hmodeldistance of m#1left.center, anchor=center] (m#1right) {};
	\draw[unconsidered, #2] (m#1top) -- (m#1right);
}
\newcommand{\addnetworkbottomleft}[2]{
	\draw[unconsidered, #2] (m#1left) -- (m#1bottom);
}
\newcommand{\addpseudobottom}[1]{
	\node[model, draw=none, fill=none, below right=\vmodeldistance and \hmodeldistance of m#1left.center, anchor=center] (m#1bottom) {};
}
\newcommand{\addpseudoright}[1]{
	\node[model, draw=none, fill=none, right=2*\hmodeldistance of m#1left.center, anchor=center] (m#1right) {};	
}

\newcommand{\drawnetworktopleft}[3]{
	\addnetworktopleft{#1}{#2}{#3}
	% Not necessary because no dependent networks, so save vertical space
	%\addpseudobottom{#1}
	%\addpseudoright{#1}
}
\newcommand{\drawnetworktopleftmiddle}[4]{
	\addnetworktopleft{#1}{#2}{#3}
	\addnetworkmiddle{#1}{#4}
	\addpseudoright{#1}
}
\newcommand{\drawnetworktop}[4]{
	\addnetworktopleft{#1}{#2}{#3}
	\addnetworktopright{#1}{#4}
	\addpseudobottom{#1}
}
\newcommand{\drawnetworktopmiddle}[5]{
	\addnetworktopleft{#1}{#2}{#3}
	\addnetworktopright{#1}{#4}
	\addnetworkmiddle{#1}{#5}
}
% #1: Number
% #2: Position of left model
% #3-#6: Properties of transformations (top left, top right, middle, bottom left)
\newcommand{\drawnetworkcomplete}[6]{
	\addnetworktopleft{#1}{#2}{#3}
	\addnetworktopright{#1}{#4}
	\addnetworkmiddle{#1}{#5}
	\addnetworkbottomleft{#1}{#6}
}
% #1: First network
% #2: Second network
\newcommand{\iteration}[2]{
	\draw[thick,-{>[scale=0.6]}] ($(#1right.east)+(0.5em,0)$) -- node[above,font=\scriptsize] {it} ($(#2left.west)+(-0.5em,0)$);
}
% #1: First network
% #2: Second network
\newcommand{\recursion}[2]{
	\draw[thick,-{>[scale=0.6]}] ($(#1bottom.south)+(0,-0.35em)$) -- node[right,font=\scriptsize] {rec} ($(#2top.north)+(0,0.35em)$);
}

\begin{tikzpicture}[
	model/.style={draw, circle, fill=black, inner sep=\modelradius},
	executed/.style={transformation, solid, very thin, {Triangle[open,angle=45:0.35em]}-{Triangle[open,angle=45:0.35em]}},
	candidate/.style={transformation, solid},
	unconsidered/.style={transformation, densely dotted, -}
]

\node[draw, fill=lightgray!10, above left=1*\vmodeldistance+1*\vborder and \hborder, anchor=north west, minimum width=2*\hmodeldistance+2*\hborder, minimum height=2*\vmodeldistance+2*\vborder] {};

\drawnetworkcomplete{2}{
}{candidate}{}{}{}

\node[draw, fill=lightgray!10, above right=1*\vmodeldistance+1*\vborder and \hnetworkdistance-\hborder of m2left.center, anchor=north west, minimum width=2*\hmodeldistance+2*\hborder, minimum height=\vnetworkdistance+1*\vmodeldistance+2*\vborder] {};

\drawnetworkcomplete{3}{right=\hnetworkdistance of m2left.center, anchor=center}{executed}{candidate}{}{}
\iteration{m2}{m3}
\drawnetworktopleft{3-1}{below=\vnetworkdistance of m3left.center,anchor=center}{candidate}
\recursion{m3}{m3-1}

\node[draw, fill=lightgray!10, above right=1*\vmodeldistance+1*\vborder and \hnetworkdistance-1*\hborder of m3left.center, anchor=north west, minimum width=\hnetworkdistance+2*\hmodeldistance+2*\hborder, minimum height=2*\vnetworkdistance+1*\vmodeldistance+2*\vborder] {};

\drawnetworkcomplete{4}{right=\hnetworkdistance of m3left.center, anchor=center}{executed}{executed}{candidate}{}
\iteration{m3}{m4}
\drawnetworktop{4-1}{below=\vnetworkdistance of m4left.center,anchor=center}{candidate}{}
\recursion{m4}{m4-1}
\node[draw, fill=lightgray!20, above right=1*\vmodeldistance+1*\vborder and \hnetworkdistance-1*\hborder of m4-1left.center, anchor=north west, minimum width=2*\hmodeldistance+2*\hborder, minimum height=\vnetworkdistance+1*\vmodeldistance+2*\vborder] {};
\drawnetworktop{4-2}{right=\hnetworkdistance of m4-1left.center,anchor=center}{executed}{candidate}
\iteration{m4-1}{m4-2}
\drawnetworktopleft{4-2-1}{below=\vnetworkdistance of m4-2left.center,anchor=center}{candidate}
\recursion{[yshift=0.1*\vnetworkdistance]m4-2}{[yshift=0.1*\vnetworkdistance]m4-2-1}

\node[draw, fill=lightgray!10, above right=1*\vmodeldistance+1*\vborder and 2*\hnetworkdistance-1*\hborder of m4left.center, anchor=north west, minimum width=3*\hnetworkdistance+2*\hmodeldistance+2*\hborder, minimum height=3*\vnetworkdistance+1*\vmodeldistance+2*\vborder] {};

\drawnetworkcomplete{5}{right=2*\hnetworkdistance of m4left.center, anchor=center}{executed}{executed}{executed}{candidate}
\iteration{[xshift=1*\hnetworkdistance]m4}{m5}
\drawnetworktopmiddle{5-1}{below=\vnetworkdistance of m5left.center,anchor=center}{candidate}{}{}
\recursion{m5}{m5-1}
\node[draw, fill=lightgray!20, above right=1*\vmodeldistance+1*\vborder and \hnetworkdistance-1*\hborder of m5-1left.center, anchor=north west, minimum width=2*\hmodeldistance+2*\hborder, minimum height=\vnetworkdistance+1*\vmodeldistance+2*\vborder] {};
\drawnetworktopmiddle{5-2}{right=\hnetworkdistance of m5-1left.center,anchor=center}{executed}{candidate}{}
\iteration{m5-1}{m5-2}
\drawnetworktopleft{5-2-1}{below=\vnetworkdistance of m5-2left.center,anchor=center}{candidate}
\recursion{m5-2}{m5-2-1}
\node[draw, fill=lightgray!20, above right=1*\vmodeldistance+1*\vborder and \hnetworkdistance-1*\hborder of m5-2left.center, anchor=north west, minimum width=\hnetworkdistance+2*\hmodeldistance+2*\hborder, minimum height=2*\vnetworkdistance+1*\vmodeldistance+2*\vborder] {};
\drawnetworktopmiddle{5-3}{right=\hnetworkdistance of m5-2left.center,anchor=center}{executed}{executed}{candidate}
\iteration{m5-2}{m5-3}
\drawnetworktopleftmiddle{5-3-1}{below=\vnetworkdistance of m5-3left.center,anchor=center}{candidate}{}
\recursion{m5-3}{m5-3-1}
\node[draw, fill=lightgray!30, above right=1*\vmodeldistance+1*\vborder and \hnetworkdistance-1*\hborder of m5-3-1left.center, anchor=north west, minimum width=2*\hmodeldistance+2*\hborder, minimum height=1*\vnetworkdistance+1*\vmodeldistance+2*\vborder] {};
\drawnetworktopleftmiddle{5-3-2}{right=\hnetworkdistance of m5-3-1left.center,anchor=center}{executed}{candidate}
\iteration{m5-3-1}{m5-3-2}
\drawnetworktopleft{5-3-2-1}{below=\vnetworkdistance of m5-3-2left.center,anchor=center}{candidate}
\recursion{m5-3-2}{m5-3-2-1}

% Legend
\newcommand{\vdistancelegend}{1*\vmodeldistance}
\coordinate (legend_left_column) at ([yshift=-1.6*\vnetworkdistance,xshift=0.2em]m2left.center);

\node[legendbg, minimum height=6.9*\vdistancelegend, minimum width=7.2em, above right=1.1*\vdistancelegend and -0.75em of legend_left_column, anchor=north west] {};

\draw[unconsidered] (legend_left_column) -- ([xshift=1*\hmodeldistance]legend_left_column);
\node[legend, right=1.3*\hmodeldistance of legend_left_column,anchor=west] {$\transformationset{T}$};
\draw[candidate] ([yshift=-\vdistancelegend]legend_left_column) -- ([xshift=1*\hmodeldistance,yshift=-\vdistancelegend]legend_left_column);
\node[legend, below right=\vdistancelegend and 1.3*\hmodeldistance of legend_left_column,anchor=west] {$\provalgcandidate$};
\draw[executed] ([yshift=-2*\vdistancelegend]legend_left_column) -- ([xshift=1*\hmodeldistance,yshift=-2*\vdistancelegend]legend_left_column);
\node[legend, below right=2*\vdistancelegend and 1.3*\hmodeldistance of legend_left_column,anchor=west] {$\provalgexecuted$};

\coordinate (legend_right_column) at ([yshift=-3.8*\vdistancelegend]legend_left_column);

\coordinate (legenditerationright) at ([xshift=-0.3*\hmodeldistance+\modelradius]legend_right_column);
\coordinate (legenditerationleft) at ([xshift=-0.3*\hmodeldistance-2*\hmodeldistance-\modelradius+\hnetworkdistance]legend_right_column);
\iteration{legenditeration}{legenditeration}
\node[legend, right=1.3*\hmodeldistance of legend_right_column, anchor=west] (legenditerationtext) {iteration step};

\coordinate (legendrecursionbottom) at ([yshift=-\vdistancelegend+0.5*\vnetworkdistance-\vmodeldistance-\modelradius]legend_right_column);
\coordinate (legendrecursiontop) at ([yshift=-\vdistancelegend-0.5*\vnetworkdistance+\vmodeldistance+\modelradius]legend_right_column);
\recursion{legendrecursion}{legendrecursion}
\node[legend, below=\vdistancelegend of legenditerationtext.west, anchor=west] {recursion step};

\end{tikzpicture}%
}
    \caption[Exemplary execution of the provenance algorithm]{%
    Exemplary execution of the provenance algorithm for a change in the topmost model. 
    The transformations present to the current execution of \function{Propagate}, as well as the executed and candidate transformations $\provalgexecuted$ and $\provalgcandidate$ are depicted for each iteration (horizontal) and recursion step (vertical).
}
    \label{fig:orchestration:provenance_example}
\end{figure}

\mnote{Exemplary execution of the provenance algorithm}
\autoref{fig:orchestration:provenance_example} depicts an exemplary execution of the \function{ProvenanceApply} algorithm for a set of four transformations between four metamodels.
We assume that the algorithm receives four initially consistent models and a change to the topmost one.
The example shows that in each recursion step only the subnetwork of the already executed transformations in $\provalgexecuted$ is considered.
Thus, the set of transformations gets smaller in each recursive call of \function{ProvenanceApply}.


\subsection{Correctness and Termination}



\subsection{Provenance Improvement}

Derive from the algorithm how it can improve provenance.
Currently it does not return necessary states, but the states in which it fails could just be delivered.

\todo{Discuss that is may make sense to first use transformations between already involved models rather than passing through the complete network.}


