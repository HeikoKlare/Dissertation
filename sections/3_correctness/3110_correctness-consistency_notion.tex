%%
%% CONSISTENCY NOTIONS
%%
\section{Notions of Consistency and its Preservation}
\label{chap:correctness:notions_consistency}

We begin with an informal discussion of different ways to consider consistency and its preservation. This involves \emph{intensional} and \emph{extensional}, as well \emph{monolithic} and \emph{modular} notions, and different execution strategies.

\subsection{Intensional and Extensional Consistency Notions}
\label{chap:correctness:notions_consistency:intensional_extensional}

\mnote{Intensional notion}
When we consider a tuple of models, we may intuitively assume it to be consistent if it fulfills some kind of constraints.
Defining these constraints to derive or check whether a given tuple of models is consistent constitutes an \emph{intensional specification} of consistency, because the set that contains all consistent model tuples is intensionally represented by these constraints and can be derived from it.
We can consider a set of constraints as a predicate, i.e., a Boolean-valued function $P$, which indicates whether a model tuple $\modeltuple{m} \in \metamodeltupleinstanceset{M}$ fulfills the constraints $P: \metamodeltupleinstanceset{M} \rightarrow \setted{\truemath, \falsemath}$. Then we can say that:
\begin{align*}
    \modeltuple{m} \consistenttomath P \equivalentperdefinition P(\modeltuple{m}) = \truemath
\end{align*}

\mnote{Extensional notion}
Alternatively, one can enumerate the (possibly infinite number of) consistent tuples of models.
Thus, a model tuple is considered consistent if that enumeration contains it.
This constitutes an \emph{extensional specification} of consistency.
Given such an enumeration $E = \setted{\modeltuple{m} \mid \modeltuple{m} \mathtextspacebefore{is consistent}}$, we can say that:
\begin{align*}
    \modeltuple{m} \consistenttomath E \equivalentperdefinition \modeltuple{m} \in E
\end{align*}

\mnote{Equivalence of notions}
Both kinds of specification have equal expressiveness. For each intensional specification, the extensional one can be derived by enumerating all models that fulfill the constraints:
\begin{align*}
    E = \setted{\modeltuple{m} \mid P(\modeltuple{m}) = true}
\end{align*}
An extensional specification can also be transferred to an intensional one by defining constraints that are fulfilled by exactly the enumerated instances:
\begin{align*}
    P(\modeltuple{m}) \mapsto 
    \begin{cases} 
        \truemath, & \modeltuple{m} \in E\\
        \falsemath, & \modeltuple{m} \not\in E\\
    \end{cases}
\end{align*}
For us, it will only be relevant that an intensional specification can be transformed into an extensional one.

\mnote{We use extensional specifications}
A developer who defines consistency usually wants to use an intensional specification, as tools like transformation languages allow the specification of constraints rather than enumerating consistent instances.
Since there is usually an infinite number of consistent models, he or she cannot explicitly enumerate them but only define constraints that allow to derive them.
From a theoretical perspective, however, we prefer to consider extensional specifications, because they allow to directly apply set theory in a concise way.
Due to the fact that each intensional specification can be transformed into an extensional one, we can make theoretical statements about extensional specifications that also hold for intensional ones.
In the following, we always consider extensional specifications unless otherwise stated.
So we define which models are considered consistent in terms of relations, which we also call \emph{consistency relations}.


\subsection{Monolithic and Modular Consistency Notions}
\label{chap:correctness:notions_consistency:monolithic_modular}

\mnote{Intuitive notion}
Consistency, be it specified intensionally or extensionally, can be considered in an either monolithic or modular way.
Having a single specification of consistency for an arbitrary number of models constitutes a \emph{monolithic} notion of consistency.
Like discussed for intensional and extensional consistency specifications, this can be expressed by a tuple of models fulfilling constraints or being contained in a relation.
A \emph{modular} notion of consistency considers several relations for subsets of the relevant metamodels, which together define when models are considered consistent.

\mnote{Modular specification example}
For an extensional notion of consistency between three metamodels $\metamodel{M}{1}, \metamodel{M}{2}$ and $\metamodel{M}{3}$, a modular specification could manifest in three relations $\consistencyrelation{CR}{1,2}, \consistencyrelation{CR}{1,3}$ and $\consistencyrelation{CR}{2,3}$ defining the model pairs that are considered consistent.
If two models are consistent to one of the relations, we can say that they are \emph{locally} consistent to that relation.
We are, however, interested in whether models are \emph{globally} consistent to all these relations, so we say:
\parameterizeformat{
    \begin{align*}
        & \model{m}{1}, \model{m}{2}, \model{m}{3} \mathtextspacearound{are consistent} \equivalentperdefinition 
        #2
        \tupled{\model{m}{1},\model{m}{2}} \in \consistencyrelation{CR}{1,2} \land \tupled{\model{m}{1},\model{m}{3}} \in \consistencyrelation{CR}{1,3} \land \tupled{\model{m}{2},\model{m}{3}} \in \consistencyrelation{CR}{2,3}
    \end{align*}
}{}{\\ & \formulaskip}%

\mnote{Modular specification for multiary relations}
Due to the assumptions of independent development and modular reuse, which we have defined in \autoref{chap:introduction:objective:assumptions}, we are interested in a modular notion of consistency.
In the example, we have considered a modular notion based on binary relation. Such a modular notion, however, can also be based on multiple multiary relations. 
But even with multiary relations, modularity is necessary for reasons of independent development and reuse.
For reasons of simplicity, we stick to modular notions of binary relations, although most of our considerations can be transferred to multiary ones.


\subsection{Consistency Preservation}
\label{chap:correctness:notions_consistency:preservation}

\mnote{Consistency preservation goal}
Consistency preservation is the process of ensuring that models stay consistent.
Based on a notion of consistency relations that describe when models are considered consistent, this process ensures that models stay in that relation. 
If models become changed such they that are not in the relation anymore, consistency preservation updates the models such that they are in that relation again.
In consequence, consistency preservation is always relative to relations defining consistency.

\mnote{Consistency preservation as function}
Consistency preservation can be considered as a function $\function{Cp}$ that takes (potentially inconsistent) models and returns a consistent tuple of models:
\parameterizeformat{
\begin{align*}
    & \function{Cp} : \metamodeltupleinstanceset{M} \rightarrow \metamodeltupleinstanceset{M} 
    #1 #2
    \forall \modeltuple{m} \in \metamodeltupleinstanceset{M} : \function{Cp}(\modeltuple{m}) \mathtextspacebefore{is consistent}
\end{align*}
}{\qquad\qquad}{\\ &}%
The definition of \emph{is consistent} depends on whether we rely on a monolithic or modular notion of consistency.
Thus it may require the models to be in one or multiple relations.
For example, given a monolithic relation $\consistencyrelation{CR}{}$, $\function{Cp}$ is supposed to fulfill that: 
\begin{align*}
    \forall \modeltuple{m} \in \metamodeltupleinstanceset{M} : \function{Cp}(\modeltuple{m}) \in \consistencyrelation{CR}{}
\end{align*}
Since these functions define how consistency is preserved, we also call them \emph{consistency preservation rules}.

\begin{figure}
    \centering
    \newcommand{\hdistance}{6em}

% #1: Position m1
% #2: Prefix
% #3: Suffix
% #4-#6: Model names
% #7-#8: Relation formatting
\newcommand{\network}[8][]{
    \node[model, #1] (#2_m1_#3) {#4};
    \node[model, above right=0.4*\hdistance and 0.6*\hdistance of #2_m1_#3.center, anchor=center] (#2_m2_#3) {#5};
    \node[model, right=1*\hdistance of #2_m1_#3.center, anchor=center] (#2_m3_#3) {#6};
    \draw[correspondence, #7] (#2_m1_#3) -- (#2_m2_#3);
    \draw[correspondence, #8] (#2_m1_#3) -- (#2_m3_#3);
}

\begin{tikzpicture}[
    every node/.append style={font=\footnotesize},
    model/.style={schematic model, minimum size=2em, inner sep=0.15em},
    violated/.style={densely dashed, color=darkred}
]

% INDEPENDENT
% Models
\network{ind}{1}{$\model{m}{1}$}{$\model{m}{2}$}{$\model{m}{3}$}{}{}
\network[below=1.2*\hdistance of ind_m1_1.center, anchor=center]{ind}{2}{$\model{m}{1}'$}{$\model{m}{2}$}{$\model{m}{3}$}{violated}{violated}
\network[below left=1.3*\hdistance and 0.8*\hdistance of ind_m1_2.center, anchor=center]{ind}{3}{$\model{m}{1}''$}{$\model{m}{2}'$}{$\model{m}{3}$}{}{violated}
\network[below right=1.3*\hdistance and 0.8*\hdistance of ind_m1_2.center, anchor=center]{ind}{4}{$\model{m}{1}'''$}{$\model{m}{2}$}{$\model{m}{3}'$}{violated}{}
\node[below right=0.2*\hdistance and 1.3*\hdistance of ind_m1_3.center, anchor=north, align=center] {unclear whether $\model{m}{1}'' = \model{m}{1}'''$ \\ or whether/how they can be merged};
% Execution
\draw[consistency execution] ([xshift=0.1*\hdistance, yshift=-0.1*\hdistance]ind_m1_1.south) -- node[left, align=center] {user\\ change} ([xshift=0.1*\hdistance, yshift=0.2*\hdistance]ind_m1_2.north);
\draw[consistency execution] ([xshift=0.45*\hdistance, yshift=-0.1*\hdistance]ind_m1_2.south) -- node[left=0.2em] {$\function{Cp}_{1,3}$} ([xshift=0.2*\hdistance, yshift=0*\hdistance]ind_m2_3.north);
\draw[consistency execution] ([xshift=0.55*\hdistance, yshift=-0.1*\hdistance]ind_m1_2.south) -- node[right=0.2em] {$\function{Cp}_{2,3}$} ([xshift=-0.4*\hdistance, yshift=0*\hdistance]ind_m2_4.north);

% CONSECUTIVE
% Models
\network[left=2.5*\hdistance+0.5*\difftoafiveimage of ind_m1_1.center, anchor=center]{cons}{1}{$\model{m}{1}$}{$\model{m}{2}$}{$\model{m}{3}$}{}{}
\network[below=1.2*\hdistance of cons_m1_1.center, anchor=center]{cons}{2}{$\model{m}{1}'$}{$\model{m}{2}$}{$\model{m}{3}$}{violated}{violated}
\network[below=1.2*\hdistance of cons_m1_2.center, anchor=center]{cons}{3}{$\model{m}{1}''$}{$\model{m}{2}'$}{$\model{m}{3}$}{}{violated}
\network[below=1.2*\hdistance of cons_m1_3.center, anchor=center]{cons}{4}{$\model{m}{1}'''$}{$\model{m}{2}'$}{$\model{m}{3}'$}{violated}{}
\node[below right=0.2*\hdistance and 0.5*\hdistance of cons_m1_4.center, anchor=north, align=center] {potentially $\tupled{\model{m}{1}''', \model{m}{2}'} \not\in \consistencyrelation{CR}{1,2}$};
% Execution
\draw[consistency execution] ([xshift=0.1*\hdistance, yshift=-0.1*\hdistance]cons_m1_1.south) -- node[left, align=center] {user\\ change} ([xshift=0.1*\hdistance, yshift=0.2*\hdistance]cons_m1_2.north);
\draw[consistency execution] ([xshift=0.1*\hdistance, yshift=-0.1*\hdistance]cons_m1_2.south) -- node[left] {$\function{Cp}_{1,2}$} ([xshift=0.1*\hdistance, yshift=0.2*\hdistance]cons_m1_3.north);
\draw[consistency execution] ([xshift=0.1*\hdistance, yshift=-0.1*\hdistance]cons_m1_3.south) --  node[left] {$\function{Cp}_{2,3}$} ([xshift=0.1*\hdistance, yshift=0.2*\hdistance]cons_m1_4.north);

% Labels
\node[above left=0.25*\hdistance and 0.1*\hdistance of ind_m2_1.center, anchor=south, font=\bfseries\small] {Independent Execution\sameheight};
\node[above left=0.25*\hdistance and 0.1*\hdistance of cons_m2_1.center, anchor=south, font=\bfseries\small] {Consecutive Execution\sameheight};

\end{tikzpicture}
    %\includegraphics[width=\textwidth]{figures/correctness/notion/concurrent_consecutive_execution.png}
    \caption[Execution alternatives of consistency preservation rules]{Scenarios for independently executing consistency preservation rules on input models and consecutively executing them on the results of other rules. Circles denote models, lines between models denote fulfilled (solid blue) or violated (dashed red) consistency relations, and arrows between the model states (unidirectional green) denote the conduction of user changes or consistency preservation execution.}
    \label{fig:correctness:concurrent_consecutive_execution}
\end{figure}

\mnote{Modular consistency preservation}
Like for the proposed notion of consistency, we can also consider consistency preservation in an either monolithic or modular way.
With a modular notion of consistency preservation, we may have multiple consistency preservation rules that preserve consistency, each of them for a consistency relation that defines consistency for a subset of the involved models.
Unlike for the relations defining consistency, which can be evaluated independently to identify whether models are consistent, the functions, i.e., consistency preservation rules, cannot be evaluated independently.
If each function is executed independently, each of them returns new models that may need to be merged. 
This is exemplified in the following scenario, which is also depicted in \autoref{fig:correctness:concurrent_consecutive_execution}.
Imagine two functions $\function{Cp}_{1,2}$ and $\function{Cp}_{1,3}$ that preserve consistency for relations $\consistencyrelation{CR}{1,2}$ and $\consistencyrelation{CR}{1,3}$, respectively.
Consider the input models $\tupled{\model{m}{1}, \model{m}{2}, \model{m}{3}}$ that are not consistent to $\consistencyrelation{CR}{1,2}$ and $\consistencyrelation{CR}{1,3}$, i.e., $\tupled{\model{m}{1},\model{m}{2}} \not\in \consistencyrelation{CR}{1,2}$ and $\tupled{\model{m}{1},\model{m}{3}} \not\in \consistencyrelation{CR}{1,3}$.
This can, for example, occur because $\model{m}{1}$ was changed by a user.
Now if we apply the functions independently, we have $\function{Cp}_{1,2}(\tupled{\model{m}{1}, \model{m}{2}}) = \tupled{\model{m}{1}', \model{m}{2}'} \in \consistencyrelation{CR}{1,2}$ and 
$\function{Cp}_{1,3}(\tupled{\model{m}{1}, \model{m}{3}}) = \tupled{\model{m}{1}'', \model{m}{3}'} \in \consistencyrelation{CR}{1,3}$.
It is now unclear how to unify $\model{m}{1}'$ and $\model{m}{1}''$ to $\model{m}{1}'''$, such that $\tupled{\model{m}{1}''', \model{m}{2}'} \in \consistencyrelation{CR}{1,2}$ and  $\tupled{\model{m}{1}''', \model{m}{3}'} \in \consistencyrelation{CR}{1,3}$.

\mnote{Consecutive function execution}
An intuitive approach to execute the functions is their composition, i.e., a consecutive execution that does not apply the functions for consistency preservation to the original models but to the models delivered by the previous executions of the functions, which is also exemplarily depicted in \autoref{fig:correctness:concurrent_consecutive_execution}.
If we consecutively apply the two given functions, we know that $\function{Cp}_{1,2}(\tupled{\model{m}{1}, \model{m}{2}}) = \tupled{\model{m}{1}', \model{m}{2}'} \in \consistencyrelation{CR}{1,2}$ and 
$\function{Cp}_{1,3}(\tupled{\model{m}{1}', \model{m}{3}}) = \tupled{\model{m}{1}'', \model{m}{3}'} \in \consistencyrelation{CR}{1,3}$.
It is, however, unclear whether $\tupled{\model{m}{1}'', \model{m}{2}'} \in \consistencyrelation{CR}{1,2}$, so it may be necessary to execute $\function{Cp}_{1,2}$ again.
In fact, we need some method to decide in which order and how often the consistency preservation rules are applied to result in a consistent tuple of models.
We call this an \emph{orchestration}.
The challenge to find an execution order of transformations without leading to execution cycles has also been identified by \textcite[Sec.~3.9]{kramer2017a}.

\mnote{Unification or orchestration necessary}
Even if consistency preservation rules were supposed to only modify one model instead of two, the same problems of unifying changes of their independent execution or orchestrating their consecutive execution occur as soon as there are two sequences of consistency preservation rules that change the same models.

\mnote{Benefits of consecutive execution}
In our work, we follow the approach of orchestrating and consecutively executing consistency preservation rules.
The benefits of this approach are twofold. First, there is no additional logic required for unifying the changes performed by independently executed consistency preservation rules. 
Second, the unification may deliver a model that is not consistent to any of the consistency relations anymore, whereas consecutive execution at least guarantees that the models are consistent to the last applied consistency preservation rule.
With this approach, the repeated execution of consistency preservation rules can be seen as a negotiation of a solution by reacting to the changes the other consistency preservation rules performed.

\begin{remark} 
\mnote{Monolithic notions as degraded modular ones}
Finally, every monolithic notion of consistency and its preservation can be considered a special case of a modular notion. Having only one consistency relation and one function that preserves it degrades the problem by making the necessity to perform an orchestration of functions obsolete.
\end{remark}

\mnote{Realization options for consistency preservation rules}
For now, the introduced consistency preservation rules can be any kind of functions that return consistent models. 
Their realization may, for example, be transformations that define how to react to certain changes for restoring consistency, or constraint solvers that find consistent models by solving consistency constraints. 
We do not yet need to consider how these functions are realized to derive consistent models, although we later focus on transformation-based approaches.


\subsection{Declarative and Imperative Specifications}
\label{chap:correctness:notions_consistency:declarative_imperative}

\begin{figure}
    \centering
    \begin{tikzpicture}

\node[schematic metamodel] (instances1) {$\metamodel{M}{1}$};
\node[schematic metamodel, right=22em of instances1] (instances2) {$\metamodel{M}{2}$};

\draw[consistency relation] (instances1) -- node[above] (cr) {$\consistencyrelation{CR}{1,2}$} (instances2);
\draw[transformation] (instances1) -- ++(0,-5em) -| node[pos=0.25,below] (cpr) {$\function{CP}_{1,2}$} (instances2);

\draw[-latex] ([xshift=-0.5em]cr.south) -- node[left=0.2em, align=center] {declarative specification \\ (ambiguous)} ([xshift=-0.5em]cpr.north);
\draw[-latex] ([xshift=0.5em]cpr.north) -- node[right=0.2em, align=center] {imperative specification \\ (unambiguous)}  ([xshift=0.5em]cr.south);

\end{tikzpicture}
    %\includegraphics[width=0.5\textwidth]{figures/correctness/notion/declarative_imperative}
    \caption[Declarative and imperative consistency specification]{Declarative and imperative specification of consistency relations and consistency preservation rules for two metamodels $\metamodel{M}{1}$ and $\metamodel{M}{2}$.}
    \label{fig:correctness:declarative_imperative}
\end{figure}

\mnote{Only define relation or function}
We have discussed that consistency preservation can be considered as functions, called consistency preservation rules, that preserve consistency according to some relations.
In practice, however, one will usually not specify both the consistency relation and the consistency preservation rule that preserves it.
Instead, one artifact is given and the other is implied or derived.
This leads to the two approaches of \emph{declarative} and \emph{imperative} consistency specifications, depending on whether the specification defines \emph{how} consistency is achieved.
The relation between the two approaches regarding a consistency relation and a consistency preservation rule is depicted in \autoref{fig:correctness:declarative_imperative}.

\mnote{Ambiguity in deriving rules}
As a first option, a developer may only define relations that specify consistency. Functions that preserve these relations can be derived from that.
This is called a \emph{declarative} specification, because it only declares when models are consistent but not \emph{how} consistency is achieved.
In general, there are multiple valid options for deriving a consistency preservation rule from a relation.
It can, for example, calculate the result with minimal differences to the input according to some defined metric.
Or, especially if there is an intensional specification of the relations, the approach may consider the type of input change and calculate an appropriate change according to the constraints in the intensional specification.
This approach is followed by many declarative transformation languages, such as \gls{QVTR}~\cite{qvt} or \glspl{TGG}~\cite{anjorin2014EfficientSynchronizationTGG-ECMFA}.

\mnote{Relations as fixed points}
As a second option, a developer can define consistency preservation rules without explicitly specifying the consistency relations to which they preserve consistency.
Instead, these functions imply the underlying consistency relations that they preserve, at least if we assume that a consistency preservation rule does not perform changes when the input models are already consistent.
Given a function $\function{Cp}$, the relation $\consistencyrelation{CR}{}$ it preserves is implied by its fixed points: $\consistencyrelation{CR}{} = \setted{\modeltuple{m} \mid \function{Cp}(\modeltuple{m}) = \modeltuple{m}}$.
If a function preserving consistency does not perform any changes, the models are, by definition, consistent.
Usually, we will assume that such a function returns consistent models with a single application.
Thus, if it does not perform changes when the input models are already consistent, the function is idempotent and then the consistency relation is given by its image, i.e., $\consistencyrelation{CR}{} = \setted{\modeltuple{m} \mid \exists \modeltuple{m}' : \function{Cp}(\modeltuple{m}') = \modeltuple{m}}$.
This is called an \emph{imperative} specification, because it declares \emph{how} consistency can be achieved.
Such an approach is followed by many imperative transformation languages, such as \gls{QVTO}~\cite{qvt}.


\subsection{Consistency Preservation Artifacts}

\mnote{Four artifacts}
We have discussed that consistency can be considered in a monolithic or modular way. We have, however, also mentioned that the monolithic case can be considered as a special case of the modular one.
For the general case, we thus know from the previous considerations that in a consistency preservation process at least specifications that define consistency, called \emph{consistency relations}, functions that preserve consistency, called \emph{consistency preservation rules}, and a function for orchestrating the functions, in the following called \emph{orchestration function}, are necessary. Finally, we also need a function that applies the consistency preservation rules in the order that is determined by the orchestration function, which we call the \emph{application function}.
To summarize, we consider the following four artifacts necessary to handle consistency preservation.
\begin{properdescription}
    \item[Consistency Relations:] Binary relations that specify which pairs of models shall be considered consistent.
    \item[Consistency Preservation Rules:] Functions that restore consistency for a pair of models that became inconsistent by modification.
    \item[Orchestration Function:] A function that determines the execution order of the consistency preservation rules to restore consistency.
    \item[Application Function:] A function that applies the consistency preservation rules in the order determined by the orchestration function. 
\end{properdescription}

\begin{figure}
    \centering
    \newcommand{\distance}{2em}

% #1: position left model
% #2: identifier prefix
% #3: relation style
% #4-#7: relations names 
\newcommand{\network}[7][]{
    \node[schematic metamodel, #1] (#2_m1) {};
    \node[schematic metamodel, above right=\distance and \distance of #2_m1.center, anchor=center] (#2_m2) {};
    \node[schematic metamodel, right=2*\distance of #2_m1.center, anchor=center] (#2_m3) {};
    \node[schematic metamodel, below right=\distance and \distance of #2_m1.center, anchor=center] (#2_m4) {};
    \draw[#3] (#2_m1) -- node[above left] {#4} (#2_m2);
    \draw[#3] (#2_m2) -- node[above right] {#5} (#2_m3);
    \draw[#3] (#2_m2) -- node[fill=white, opacity=0.7, text opacity=1] {#6} (#2_m4);
    \draw[#3] (#2_m1) -- node[below left] {#7} (#2_m4);    
}

\begin{tikzpicture}[
    every node/.style={font=\footnotesize},
    annotation/.style={font=\itshape\footnotesize}
]

% INPUTS
\network{input}{consistency relation}{$\consistencyrelation{CR}{1}$}{$\consistencyrelation{CR}{2}$}{$\consistencyrelation{CR}{3}$}{$\consistencyrelation{CR}{4}$}
\node[below left=0.2em and 0em of input_m1.west, anchor=east] {$\model{m}{1}$};
\node[right=1*\distance of input_m3.center, anchor=center] {+};
\node[right=2*\distance of input_m3.center, anchor=center, font=\small] (changes) {$\Delta_{\model{m}{1}}$};
\node[below=0.3em of input_m4.south, anchor=north, align=center, annotation] (text_input_models) {models consistent\\ to $\consistencyrelation{CR}{1}, \consistencyrelation{CR}{2}, \dots$};
\node[right=3*\distance of text_input_models.north, anchor=north, align=center, annotation] (text_input_changes) {changes to \\ models(s)};
\node[consistency execution element, above left=0.3*\distance and 2.5*\distance of input_m2.north, anchor=north, align=center] {Consistency\\ Relations};

% OUTPUT
\network[right=9.5*\distance of input_m1.center, anchor=center]{output}{consistency relation}{$\consistencyrelation{CR}{1}$}{$\consistencyrelation{CR}{2}$}{$\consistencyrelation{CR}{3}$}{$\consistencyrelation{CR}{4}$}
\node[below=0.3em of output_m4.south, anchor=north, align=center, annotation] (text_output_models) {changed models\\ consistent to \\ $\consistencyrelation{CR}{1}, \consistencyrelation{CR}{2}, \dots$};

\draw[consistency execution, double distance=0.15em] ([xshift=1em]changes.east) -- node[below] {Application Function} ([xshift=-1em]output_m1.west);

% ORCHESTRATION FUNCTION
\draw[consistency execution element, decorate,decoration={brace,amplitude=10pt}] 
    (text_input_changes.south east) 
    -- 
    (text_input_changes.south -| text_input_models.south west);
\coordinate (brace_midpoint) at ($(text_input_models.south west)!0.5!(text_input_changes.south east)$);
\draw[consistency execution, double distance=0.15em] 
    ([yshift=-10pt]brace_midpoint) 
    |- 
    node[pos=0.7,below] (orchestration_function) {Orchestration Function} 
    ++(3*\distance,-1*\distance);

% CONSISTENCY PRESERVATION
\network[below left=5.7*\distance and 1.7*\distance of input_m1.center, anchor=center]{cprs}{transformation}{$\consistencypreservationrule{1}$}{$\consistencypreservationrule{2}$}{$\consistencypreservationrule{3}$}{$\consistencypreservationrule{4}$}
\node[consistency execution element, above=0.3em of cprs_m2.north, anchor=south, align=center] {Consistency\\ Preservation Rules};
\draw[consistency preservation element, -latex] 
    ([xshift=0.5em]cprs_m3.east) 
    -| 
    node[pos=0.25, below, annotation] {parametrization} 
    (orchestration_function.south);

% ORCHESTRATION SEQUENCE
\coordinate (orchestration_start) at ([xshift=3*\distance, yshift=-2.7\distance]brace_midpoint);
\draw[transformation] 
    ([xshift=\distance-0.3*\distance,yshift=-0.3*\distance]orchestration_start) 
    -- 
    node[left=0.5*\distance, anchor=center] {$\langle$}
    node[below=0.35*\distance, anchor=north] {1}
    node[right=0.4*\distance, anchor=north] {,}
    ++(0.6*\distance, 0.6*\distance);
\draw[transformation] 
    ([xshift=1.8*\distance-0.3*\distance,yshift=0.3*\distance]orchestration_start) 
    -- 
    node[below=0.35*\distance, anchor=north] {2} 
    node[right=0.4*\distance, anchor=north] {,}
    ++(0.6*\distance, -0.6*\distance);
\draw[transformation] 
    ([xshift=2.6*\distance,yshift=0.3*\distance]orchestration_start) 
    -- 
    node[below=0.35*\distance, anchor=north] {3}
    node[right=0.4*\distance, anchor=north] (orchestration_middle) {,}
    ++(0, -0.6*\distance);
\draw[transformation] 
    ([xshift=3.4*\distance-0.3*\distance,yshift=-0.3*\distance]orchestration_start) 
    -- 
    node[below=0.35*\distance, anchor=north] {1}
    node[right=0.4*\distance, anchor=north] {,}
    ++(0.6*\distance, 0.6*\distance);
\draw[transformation] 
    ([xshift=4.2*\distance-0.3*\distance,yshift=0.3*\distance]orchestration_start) 
    -- 
    node[below=0.35*\distance, anchor=north] {4}
    node[right=0.4*\distance, anchor=north] {,}
    ++(0.6*\distance, -0.6*\distance);
\draw[transformation] 
    ([xshift=5.0*\distance,yshift=0.3*\distance]orchestration_start) 
    -- 
    node[below=0.35*\distance, anchor=north] {3}
    node[right=0.3*\distance, anchor=center] {$\rangle$}
    ++(0, -0.6*\distance);
\node[consistency preservation element, below=0.7em of orchestration_middle, anchor=north, annotation] {orchestrated sequence};
\draw[consistency preservation element, -latex] 
    ([xshift=\distance, yshift=0.5*\distance]orchestration_start) 
    -- 
    node[right, annotation] {parametrization} 
    ++(0,2.7*\distance);

% LABELS
\node[above right=0.5em and 1.2*\distance of input_m2.north, anchor=south, font=\bfseries\small] {Process Inputs};
\node[above=0.5em of output_m2.north, anchor=south, font=\bfseries\small] {Process Output};

\end{tikzpicture}
    %\includegraphics[width=\textwidth]{figures/correctness/notion/execution_process.png}
    \caption[Consistency specification execution process and artifacts]{Execution process and artifacts for a modular consistency specification. Central artifacts are annotated in (green) normal font.}
    \label{fig:correctness:execution_process}
\end{figure}

\mnote{Distinction between orchestration and application}
We explicitly distinguish the orchestration and the application to be able to make more fine-grained statements about the responsibilities for the orchestration and its actual execution.
This is particularly useful to determine the behavior in cases in which no orchestration of transformations that results in consistent models can be found.
The process is depicted in \autoref{fig:correctness:execution_process}. Given models that are consistent according to some consistency relations and changes to them that lead to inconsistencies, the orchestration function delivers an order of consistency preservation rules, which is used to parametrize the application function that executes these rules in the given order.
The result is, in the best case, a model tuple that is consistent to the relations again.
