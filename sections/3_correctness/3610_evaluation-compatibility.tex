\section{Compatibility}
\label{chap:correctness_evaluation:compatibility}

In \autoref{chap:compatibility}, we have presented a formal notion of compatibility, a formal approach to prove it, and a practical realization of that approach for consistency relations defined in \gls{QVTR}.
The compatibility notion is well-defined based on our formalization of transformation networks and a correctness notion for them.
The formal approach to identify consistency relations of a transformation network to be compatible is based on the insights that specific consistency relation trees are inherently compatible and that the addition and removal of consistency relations fulfilling a specific notion of redundancy preserve compatibility, thus removing redundant relations until a tree remains proves compatibility.
We have proven correctness of that formal approach by \autoref{theorem:redundancycompatibility}, \autoref{theorem:treecompatibility} and \autoref{corollary:transitiveredundancycompatibility}, such that we do not need to further evaluate its correctness.
We thus focus on correctness of the practical realization of the approach, as well as applicability of the approach.
The presented evaluation is based on the evaluation that we presented in \owncite{klare2020compatibility-report}.


\subsection{Goals and Methodology}

A tool for proving compatibility could be easily integrated into the process of developing a transformation network in order to assist transformation developers, as it operates fully automated, thus not introducing further developer effort, and improves the ability of the transformation network to find consistent models after changes.
Thus, the correctness and the applicability of the approach are of special importance.

\begin{table}
    \renewcommand{\arraystretch}{1.4}%
    \rowcolors{1}{\secondlinecolor}{\firstlinecolor}
    \begin{tabular}{p{8em} p{20em}}
        \toprule
        \rowcolor{\headinglinecolor}
        \goal{Compatibility} &
            Show that the analysis can be used by transformation developers to find incompatibilities in consistency relations of a transformation network. \\
        \question[eq:compatibility:correctness]{Correctness} & 
            \questiontext{Is compatibility always given if the analysis finds it?} \\
        \metric & 
            \metrictext{Precision: Ratio between true positives and true plus false positives} \\
        \question[eq:compatibility:applicability]{Applicability%Usefuleness/Conservativeness
        } & 
            \questiontext{How often does the analysis not prove compatibility although it is given?} \\
        \metric & 
            \metrictext{Recall: Ratio between true positives and true positives plus false negatives}\\
        \bottomrule
    \end{tabular}
    \caption[Goals, questions and metrics for compatibility]{Goals, questions and metrics for compatibility evaluation.}
    \label{tab:correctness_evaluation:gqm_compatibility}
\end{table}

\mnote{Empirical evaluation in case study}
In the subsequently presented empirical evaluation in terms of a case study, we apply the practical realization of the approach to several sets of consistency relations, which are designed to be compatible or not, according to \autoref{def:compatibility}.
We then apply the algorithm to prove compatibility to these consistency relation sets and analyze whether it properly identifies them to be compatible or not.
We denote the cases in which the algorithm proves compatibility as \emph{positives} and the ones in which is not able to prove compatibility as \emph{negatives}.
Since the algorithm operates conservatively, a negative result does not mean that incompatibility is proven, but only that compatibility could not be proven.
The goal of that evaluation together with the answered questions and evaluated metrics are summarized in \autoref{tab:correctness_evaluation:gqm_compatibility}.

\mnote{Correctness evaluation}
First, the application of the algorithm to multiple scenarios allows us to validate correctness of the practical realization of the approach according to \autoref{eq:compatibility:correctness}.
Correctness of our approach means that it is able to classify a given set of consistency relations as compatible or otherwise does not reveal a result.
This especially means that it operates conservatively and does not classify a set of consistency relations as compatible although it is not.
The algorithm is thus not allowed to produce false positives, which is why we consider the \emph{precision} metric:
\begin{align*}
    \formulaskip &
    Precision = \frac{\mathtextspacearound{true positives}}{\mathtextspacearound{true positives + false positives}}
\end{align*}
This metric needs to be $1$, as otherwise the algorithm produces false positives and would thus, per definition, be incorrect.
Thus, correctness of the algorithm directly correlates with that metric.
Analyzing this metrics serves as an indicator that the mapping of our formal approach and the underlying formalism to the practical approach realization and the used \gls{QVTR} language is correct, and especially that it operates conservatively.

\mnote{Applicability evaluation}
Second, the application of the algorithm to multiple scenarios allows us to validate its applicability according to \autoref{eq:compatibility:applicability}.
The approach uses a fully automated algorithm, thus it does not require any inputs apart from the \gls{QVTR} relations to check.
Applicability may thus be restricted if the algorithm operates too conservatively, i.e., if it often produces false negatives.
In those cases, the algorithm operates indeed correctly, but if it is not able to prove compatibility in many cases in which it is actually given, applicability is reduced as the usefulness of the results for a transformation developer are limited.
For that reason, we analyzed the \emph{recall} metric:
\begin{align*}
    &
    Recall = \frac{\mathtextspacearound{true positives}}{\mathtextspacearound{true positives + false negatives}}
\end{align*}
The higher the number of false positives, the more consistency relations could not be identified as compatible by the algorithm although they actually are, thus reducing the usefulness of the algorithm.
Applicability of the algorithm thus directly correlates with the recall metric.
For that reason, we analyzed that metric and the reasons for the cases in which the the algorithm was not able to 
prove compatibility, i.e., in which it produced false negatives.
In particular, it is relevant whether those are conceptual issues of the formal approach, such as a too restricted notion of redundancy, or a limitation of the practical approach that may be fixed by a different implementation or a different realization approach.


% We have conducted a case study to evaluate correctness and applicability of our approach.
% The evaluation focuses on the appropriate operationalization of the formal approach, which is proven correct, and its practical applicability in terms of providing an appropriate level of conservativeness. This defines our contribution \ref{contrib:evaluation}.

% \begin{itemize}
%     \item Correctness: Focused on operationalized approach, since formal approach is proven correct: Show that algorithm reveals expected results, which indicates that the mapping of the formalism to QVT-R is correct and conservative as expected and that the implementation is correct.
%     \item Applicability: Show practical applicability of the overall approach, especially regarding the degree of conservativeness. Applicability is restricted if the approach does not reveal a result although the relation are compatible in too many cases.
%     We will also consider the reasons for those non-results, i.e. whether it is a conceptual issue of the formal approach or a limitation of the operationalization.
% \end{itemize}

%\subsection{Methodology}
%\label{sec:evaluation:methodology}


% \subsection{Case Study}

% To empirically evaluate correctness and applicability of our approach, we developed a prototypical implementation and applied it to exemplary case studies.
% We give an overview of that prototype in the subsequent subsection.
% \autoref{tab:scenarios} summarizes the case studies to which we applied the approach.
% Each of those scenarios consists of three or four metamodels and especially comprises primitive data types and operations.
% They were specifically developed to evaluate our approach by defining as many kinds of relations that can be expressed with QVT-R as possible, thus also reflecting edge cases.

% \begin{table}
%     \centering
%     \renewcommand{\arraystretch}{1.2}%
%     %\rowcolors{2}{white}{gray!10}
%     \setlength\tabcolsep{4 pt}
%     \begin{tabular}{L{0.5cm} L{7.5cm} C{2cm}}
%         \toprule
%         \textbf{\#} & \textbf{Scenario Description} & \textbf{Compatible} \\
%         \midrule
%         1 & Three equal String attributes of three metamodels & \cmark\\
%         2 & Six equal String attributes of three metamodels & \cmark\\
%         3 & Concatenation of two String attributes & \cmark\\
%         4 & Double concatenation of four String attributes & \cmark\\
%         5 & Substring in a String attribute & \cmark\\
%         6 & Substring in a String attribute with precondition & \cmark\\
%         7 & Precondition with all primitive datatypes & \cmark\\
%         8 & Absolute value of Integer attribute with precondition & \cmark\\ 
%         9 & Transitive equality for three Integer attributes & \cmark\\
%         10 & Inequalities for three Integer attributes & \cmark\\
%         11 & Contradictory equalities for three Integer attributes & \xmark\\
%         12 & Contradictory inequalities for three Integer attributes & \xmark\\
%         13 & Constant property template items & \cmark\\
%         14 & Linear equations with three Integer attributes & \cmark\\ 
%         15 & Contradictory linear equations for three Int. attributes & \xmark\\
%         16 & Emptiness of various OCL sequence and set literals & \xmark\\
%         17 & Equal String attributes for four metamodels & \cmark\\
%         18 & Transitive inclusions in sequences & \cmark\\
%         19 & Comparison of role names in three metamodels & \cmark\\
%         \bottomrule
%     \end{tabular}
%     \caption[Example scenarios with compatibility classification]{Example scenarios of consistency relations and their compatibility property, from \cite{pepin2019ma}}
%     \label{tab:scenarios}
% \end{table}

% %%
% %% Description of scenarios, definition of ground truth
% %%
% We developed 14 compatible and four incompatible transformations, according to our \autoref{def:compatibility} for compatibility.
% Thus, we know the ground truth regarding compatibility of transformations for each scenario by construction.
% Applying our prototypical implementation to those scenarios classifies them as compatible (\emph{positives}) or makes no statement about compatibility (\emph{negatives}), i.e., they could either be compatible or not.
% Considering which of the results are actually correct gives us insights on correctness and applicability.

%%
%% Correctness interpretation of metrics
%%
% The approach is correct, which especially means that it operates conservatively, if it does not classify any transformation networks as compatible although they are not.
% This means that no \emph{false positives} are allowed to occur or otherwise the approach would, per definition, be incorrect.
% In other words, the \emph{precision} of the approach has to be 1:
% \begin{align*}
%     \formulaskip &
%     Precision = \frac{\mathtext{true positives}}{\mathtext{true positives + false positives}}
% \end{align*}
% %If there are any false positives, the approach is, by definition, not correct.

% %%
% %% Applicability interpretation of metrics
% %%
% Applicability of the approach depends on the degree of conservativeness, i.e., in how many cases it does not identify a transformation network as compatible although it is.
% This is reflected by the number of \emph{false negatives} and, when compared to the \emph{true positives} known as the \emph{recall}, gives insights on the degree of conservativeness:
% \begin{align*}
%     &
%     Recall = \frac{\mathtext{true positives}}{\mathtext{true positives + false negatives}}
% \end{align*}
% A high recall value indicates high applicability of the approach in terms of not being too conservative.

%\begin{itemize}
    %\item Apply a prototypical implementation of the approach to exemplary case studies.
    %\item Scenarios are summarized in \autoref{tab:scenarios}, composed of 3 or 4 metamodels, primitive data types and operations.
    %\item Scenarios only reflect supported QVT-R and OCL constructs, for which an analysis is implemented yet.
    %\item Define ground truth manually, labeling the inputs with being compatible or not, according to manually checking the definition.
    %\item We consider positives as relation sets classified as compatible and negatives as those for which no statement about compatibility is made (they may be incompatible, but could also be compatible -- conservativeness)
    %\item Evaluate results regarding precision (must be 1 if approach is correct): $\frac{true positives}{true positives + false positives}$ -- if there are any false positives, the approach is not correct
    %\item Evaluate results regarding recall: $\frac{true positives}{true positives + false negatives}$ -- false negatives are which are not classified although they are compatible, i.e. they are those which are not identified as compatible due to the approach being conservative.
    %\item High recall indicates high applicability, as there are few scenarios in which compatibility is not identified.
%\end{itemize}


\subsection{Prototypical Implementation}
\label{chap:correctness_evaluation:compatibility:implementation}

\mnote{Redundant relation detection}
The approach that we presented in \autoref{chap:compatibility:practical_approach} resulted in the implementation of a prototype, which is available in a GitHub repository~\cite{decompositionGithub}.
%The formal approach of this paper addresses a common problem in the development of cyber-physical systems: inconsistencies lead to the development of incompatible artifacts, which in turn can lead to unexpected behavior. 
% Therefore, the practicality of our approach matters. A tool for proving compatibility could be easily integrated into the development process of a transformation network in order to assist developers and domain experts.
%
%\paragraph{Implementation}
The prototypical implementation is specific to \gls{QVTR} and the \gls{OCL} constraints used in that language.
It expects a set of \gls{QVTR} transformations and returns a list of redundant \gls{QVTR} relations.
Thus, if removing the returned redundant relations from the initial set of transformations yields a set of transformations whose relations do not contain any cycles, i.e., if the form a consistency relation tree, compatibility is proven.
If cycles within the relations remain, compatibility could not be proven, either because of an actual incompatibility or because of the algorithm not being able to find redundancies to prove compatibility.

\mnote{Input validation}
Additionally, the implementation validates the given inputs.
They may be invalid because of two reasons.
First, they can contain transformations that are not well-formed, i.e., they are syntactically incorrect. In that case, the transformation cannot be processed by the compatibility analysis algorithm at al.
Second, transformations can be well-formed but invalid, e.g., because two transformations have the same name or a \gls{QVTR} domain pattern uses a nonexistent class.
Although the algorithm can still be applied to such an input, it may not produce appropriate results, thus such errors are displayed to the transformation developer when applying the algorithm in the parsing step.
Some errors, such as two transformations having the same name, could even be mitigated by automatically renaming them if such a clash occurs.
In the evaluation, we only consider valid inputs anyway.
Finally, the implementation operates completely non-intrusively, thus not altering the transformations in any way.

% The implementation of the procedure takes a set of \qvtr transformations as an input and outputs a list of redundant \qvtr relations. This list must be compared to the initial consistency specification. There are two scenarios: either the delivered specification %without redundant relations 
% forms a consistency relation tree or there are still cycles left. % after the procedure. 
% In the former case, compatibility is proven. In the latter case, remaining cycles require the developer's attention. This may be due to incompatibility or the inability of the procedure to prove redundancy.

% In addition to the features of the procedure, the prototype provides an input validation. There are two reasons why a consistency specification may cause the procedure not to operate correctly. First, specifications can be composed of not well-formed transformations, i.e., \qvtr transformations that are syntactically incorrect. In this case, the specification is not usable and the procedure immediately exits. There is another scenario: specifications that are well-formed but not valid. For example, this occurs when two transformations have the same name or when a \qvtr domain pattern uses a nonexistent class. Although this scenario is non-blocking, i.e., the decomposition procedure still produces a result, the output must be interpreted with caution. To assist the developer, the procedure displays semantic errors in the specification at the beginning of the parsing. In the end, the procedure is intended to be non-intrusive, i.e., it is does not alter any artifact and can be used at any moment during the development process, i.e., by logging its results.

%\paragraph{Implementation}

%Technical choices were mostly driven by the support of model-driven engineering technologies. One important initiative to this end is the \ac{MDA}~\cite{mda}. %, an approach of the Object Management Group. 
\mnote{Modelling tool selection}
The selection of \gls{QVTR} for the practical realization and implementation of the approach was, on the one hand, driven by the recommendation of \gls{QVTR} for defining transformation by the \gls{MDA}~\cite{mda}, and on the other hand by the fact that consistency relations are explicitly defined in \gls{QVTR}, especially in comparison to imperative languages.
We based the implementation on \gls{EMF} and its Ecore \metametamodel (see \autoref{chap:foundations:formalisms:ecore}) as one of the most common and technically mature modelling frameworks.
Within \gls{EMF}, implementations of transformation languages are provided through the \emph{Eclipse MMT}~\cite{EclipseMMT} project.
In particular, the contained \gls{QVTd}~\cite{EclipseQVTd} language provides a parser for \gls{QVTR} transformations, which, in turn, uses \emph{Eclipse OCL}~\cite{EclipseOCL} as an implementation of \gls{OCL}.

% The decomposition procedure makes use of many specifications recommended by the MDA, including \qvtr for the definition of transformations, \ac{EMOF} for metamodels~\cite{mof} and OCL for constraints over metamodels~\cite{ocl}. Eclipse provides an implementation of these languages within the \ac{EMF}~\cite{steinberg2009emf}. As a consequence, metamodels of the decomposition procedure are implemented using \textit{Ecore}, a meta-metamodel that is compliant with \ac{EMOF}. EMF supports a number of model-to-model transformation languages through the \textit{Eclipse MMT} project. In particular, the \textit{QVT Declarative} (QVTd) component provides a parser for QVT-R transformations. As QVT-R relies on OCL, QVTd makes use of \textit{Eclipse OCL}, an implementation of the OCL language.

\mnote{SMT tool selection}
For finding redundant relations, their \gls{OCL} constraints are transformed into logic formulae, whose satisfiability is then to be validated by an \gls{SMT} solver.
Most such solver are based on SMT-LIB~\cite{smtlib2017}, which is an initiative that provides a common input and output language for \gls{SMT} solvers.
Our prototype uses the Z3 theorem prover~\cite{z32008}, which is an \gls{SMT} solver, which can be used Java code and supports a large number of theories.

% Regarding the strategy for redundancy testing, the implementation of the decomposition procedure requires the use of an SMT solver. Most SMT solvers are based on SMT-LIB, an initiative that provides a common input/output language for SMT instances. The prototype relies on the Z3 theorem prover, an SMT solver with a Java binding and a large number of theories supported~\cite{z32008}.

% \todo[inline, color=kit-orange100]{footnote/ref > github URL?}
%% Conservativeness, etc.

% \begin{itemize}
%     \item \textbf{Prototype}. We developed an implementation of the decomposition procedure using:
%         \begin{itemize}
%             \item QVTd, a partial implementation of QVT-R and QVT-C in Eclipse MMT
%             \item Z3, an automated theorem prover
%         \end{itemize}
% \end{itemize}


\subsection{Case Study}
\label{chap:correctness_evaluation:compatibility:case_study}

\begin{table}
    \centering
    \small
    \renewcommand{\arraystretch}{1.2}%
    %\rowcolors{2}{white}{gray!10}
    \setlength\tabcolsep{4 pt}
    \begin{tabular}{L{0.6cm} L{7.1cm} C{2.1cm}}
        \toprule
        \textbf{\#} & \textbf{Scenario Description} & \textbf{Compatible} \\
        \midrule
        1 & Three equal String attributes of three metamodels & \cmark\\
        2 & Six equal String attributes of three metamodels & \cmark\\
        3 & Concatenation of two String attributes & \cmark\\
        4 & Double concatenation of four String attributes & \cmark\\
        5 & Substring in a String attribute & \cmark\\
        6 & Substring in a String attribute with precondition & \cmark\\
        7 & Precondition with all primitive datatypes & \cmark\\
        8 & Absolute value of Integer attribute with precondition & \cmark\\ 
        9 & Transitive equality for three Integer attributes & \cmark\\
        10 & Inequalities for three Integer attributes & \cmark\\
        11 & Contradictory equalities for three Integer attributes & \xmark\\
        12 & Contradictory inequalities for three Integer attributes & \xmark\\
        13 & Constant property template items & \cmark\\
        14 & Linear equations with three Integer attributes & \cmark\\ 
        15 & Contradictory linear equations for three Int. attributes & \xmark\\
        16 & Emptiness of various OCL sequence and set literals & \xmark\\
        17 & Equal String attributes for four metamodels & \cmark\\
        18 & Transitive inclusions in sequences & \cmark\\
        19 & Comparison of role names in three metamodels & \cmark\\
        \bottomrule
    \end{tabular}
    \caption[Example scenarios with compatibility classification]{Example scenarios of consistency relations and their compatibility property, from \cite{pepin2019ma}.}
    \label{tab:correctness_evaluation:compatibility_scenarios}
\end{table}

\mnote{Scenarios of QVT-R transformations}
We applied our just presented prototypical implementation in a case study to 19 scenarios.
Each of these scenarios consists of three or four metamodels and comprises especially primitive data types and operations.
They contain pairwise transformations between the metamodels defined in \gls{QVTR}, more specifically its implementation \gls{QVTd}.
The scenarios are depicted in \autoref{tab:correctness_evaluation:compatibility_scenarios}.

\mnote{Ground truth for scenarios}
It also depicts whether the relations of the transformations in these scenarios are compatible or not.
In total, 15 of these scenarios contain compatible consistency relations according to \autoref{def:compatibility}, whereas the other four are incompatible.
Thus, we know for each of the scenarios by construction whether its compatible or not, thus having the ground truth for our evaluations.
The application of the prototypical implementation to these scenarios yields the results \emph{positive} if it considers the relations compatible, or \emph{negative} if it was not able to prove compatibility.
Comparing these results with the ground truth in \autoref{tab:correctness_evaluation:compatibility_scenarios} allows us to identify them as true or false positives or negatives, respectively.

\mnote{Evaluation-specific scenarios}
The scenarios were specifically developed for the evaluation of the approach, thus reflecting as many kinds of relations that can be expressed with \gls{QVTR} as possible and thus also reflecting edge cases.
The implemented \gls{QVTR} relations used for the case study are also available in the GitHub repository containing the prototypical implementation~\cite{decompositionGithub}.

% To empirically evaluate correctness and applicability of our approach, we developed a prototypical implementation and applied it to exemplary case studies.
% We give an overview of that prototype in the subsequent subsection.
% \autoref{tab:scenarios} summarizes the case studies to which we applied the approach.
% Each of those scenarios consists of three or four metamodels and especially comprises primitive data types and operations.
% They were specifically developed to evaluate our approach by defining as many kinds of relations that can be expressed with QVT-R as possible, thus also reflecting edge cases.

% %%
% %% Description of scenarios, definition of ground truth
% %%
% We developed 14 compatible and four incompatible transformations, according to our \autoref{def:compatibility} for compatibility.
% Thus, we know the ground truth regarding compatibility of transformations for each scenario by construction.
% Applying our prototypical implementation to those scenarios classifies them as compatible (\emph{positives}) or makes no statement about compatibility (\emph{negatives}), i.e., they could either be compatible or not.
% Considering which of the results are actually correct gives us insights on correctness and applicability.


\subsection{Results and Interpretation}

\begin{table}
    \centering
    \small
    \renewcommand{\arraystretch}{1.2}%
    \setlength\tabcolsep{4 pt}
    \begin{tabular}{L{2cm} C{3.5cm} C{2.3cm}}
        \toprule
         & \textbf{Classified Compatible} & \textbf{Unclassified} \\
         \midrule
         \textbf{Compatible} & 12 & 4\\
         \textbf{Incompatible} & 0 & 3\\
         \bottomrule
    \end{tabular}
    \caption[Correctness of compatibility classification results]{Number of scenarios from \autoref{tab:correctness_evaluation:compatibility_scenarios} regarding actual compatibility and their classification by our approach.}
    \label{tab:correctness_evaluation:compatibility_results}
\end{table}

\mnote{Classification of case study results}
We applied the prototypical implementation of our practical approach to prove compatibility introduced in \autoref{chap:correctness_evaluation:compatibility:implementation} to the case study explained in \autoref{chap:correctness_evaluation:compatibility:case_study}.
The results of the scenario classification as compatible or not by the implementation are summarized in \autoref{tab:correctness_evaluation:compatibility_results}.


\subsubsection{Correctness}

\mnote{Requirements for correctness by construction}
We have already discussed before that correctness in terms of operating conservative is proven for the formal approach.
Since the practical approach is derived from that formal approach, correctness is also given by construction as long as the following requirements are fulfilled:
\begin{longenumerate}
    \item All relevant \gls{QVTR} relations are considered as consistency relations to be checked, i.e., all \gls{QVTR} relations are represented in the property graph.
    \item All constructs referring to expressions in \gls{QVTR} relations have to be considered. \gls{QVTR} relations are defined using variables, so all constructs referring to these variable have to be considered. In particular, all template expressions need to be considered for the construction of the property graph, namely property template items, preconditions and invariants.
\end{longenumerate}
By construction of the approach presented in \autoref{chap:compatibility:practical_approach}, we ensured that all these relevant elements are considered.
Additionally, the results of the case study further validate that we did not miss any relevant parts of \gls{QVTR} relations.

\mnote{Metric evaluation shows correctness}
The results depicted in \autoref{tab:correctness_evaluation:compatibility_results} show that the implementation did not yield any false positives.
Thus, the implementation operates conservatively as intended, not identifying consistency relations as compatible although they are not.
This results in a precision value of $1$:
\begin{align*}
    &
    Precision = \frac{\mathtextspacearound{true positives}}{\mathtextspacearound{true positives + false positives}} = \frac{12}{12+0} = 1
\end{align*}

\mnote{Answer to evaluation question}
On the one hand, this indicates that the practical approach actually conforms to the formal approach thus that the correctness proof applies as well.
On the other hand, this indicates that the implementation is correct and does not miss any relevant \gls{QVTR} constructs.
If this was the case, constraints could have been missed, which could have resulted in identifying consistency relations as compatible although they are not.
Thus, as an answer to \autoref{eq:compatibility:correctness}, the results indicate that we can expect the analysis to operate correctly.


\subsubsection{Applicability}

\mnote{Reasons for conservative behavior}
We have discussed that applicability of the approach especially depends on how often it fails in terms of not being able to prove compatibility, although the given relations are actually consistent.
In particular, conservative behavior of the approach can occur for two reasons:
\begin{properdescription}
    \item[Redundancy Notion:] Compatibility of consistency relations is proven by identifying relations that follow the definition of left-equal redundancy, as introduced in \autoref{def:leftequalredundancy}. Since this redundancy notion is not proven to be the weakest one that is compatibility-preserving, it may be a too strong requirement for identifying compatibility-preserving consistency relations.
    \item[Redundancy Undecidability:] \autoref{def:consistencyrelation} for consistency relations relies on an extension specification of consistency, which enumerates usually infinite sets of elements.
    Since such sets cannot be compared programmatically, our practical approach relies on intensional specifications in \gls{OCL} as used by \gls{QVTR}, which describe how consistent element pairs can be derived.
    Consistency relations as defined in \autoref{def:consistencyrelation} are extensional specifications and thus usually enumerate infinite sets of elements, which are impossible to compare programmatically.
    OCL is, however, in general undecidable, because it can be transformed into first-order logic~\cite{beckert2002ocltranslation}.
\end{properdescription}

\mnote{OCL limitations}
In particular, the number of quantifiers within a formula influences decidability.
Since variables in consistency relations are translated to existentially quantified formulae, the number of variables in a consistency relation is crucial for deciding its satisfiability.
Not all available \gls{OCL} constructs may be necessary to describe relevant consistency relations, still constructs involving operations on sets and strings are especially problematic, because operation on collections are transformed into quantified formulae and string provide problematic \gls{OCL} operations.
For example, \texttt{toUpper} and \texttt{toLower}, which we have also used in our running example, cannot be easily transformed into formulae for state-of-the-art \gls{SMT} solvers like Z3 and thus cannot be considered for detecting redundancies.
Additionally, \gls{SMT} solvers use heuristics, so we cannot even systematically evaluate which kinds of relations can be analyzed.

% Especially formulae that contain many quantifiers are hard to analyze.
% For that reason, the number of variables used in a consistency relation is crucial, as these variables are translated to existentially quantified formulae.
% Although not all available OCL constructs might be necessary to describe relevant consistency relations, constructs involving operations on sets and strings are problematic.
% Operation collections are transferred to quantified formulae, which are hard to analyze.
% Reasoning about strings is problematic, because some OCL operations like \texttt{toUpper} and \texttt{toLower} cannot be easily transferred to state-of-the-art SMT solvers like Z3 and thus cannot be considered for detecting redundancy.
% Furthermore, SMT solvers use heuristics, so it not possible to formally evaluate which kinds of relations can be analyzed.

%\begin{itemize}
    %\item Relations are usually infinite sets of elements, which are impossible to compare programmatically. For that reason, the operationalized approach uses intensional specifications of element instances and consistency relations (written in OCL). In general, it is impossible to ensure that an OCL expression matches a specification -- undecidability: OCL can be transformed to first-order logic~\cite{beckert2002translating}.
    %\item Limitation also concerns relations that are recurring in practice, e.g. reasoning about strings, such as OCL operations \texttt{toUpper} and \texttt{toLower}, which cannot be easily transferred to state-of-the-art SMT solvers like Z3.
    %\item SMT solvers use heuristics, so it is not possible to clearly state which kinds of relations can be analyzed and which may not.
    %\item Problematic are, among others, operations on collections (transferred to quantifiers which are hard to analyze).
    %\item Number of variables is relevant, because they are existentially quantified, so more 
%\end{itemize}

\mnote{Metric evaluation show high recall}
According to the results in \autoref{tab:correctness_evaluation:compatibility_results} from applying our prototypical implementation to the scenarios introduced in \autoref{tab:correctness_evaluation:compatibility_scenarios}, consistency relations were correctly classified as compatible in twelve out of the 15 scenarios, whereas the implementation was not able to prove compatibility in the remaining three scenarios, thus delivering three false negatives.
This leads to a recall value of \SI{80}{\percent}.
\begin{align*}
    &
    Recall = \frac{\mathtextspacearound{true positives}}{\mathtextspacearound{true positives + false negatives}} = \frac{12}{12+3} = 0.8
\end{align*}
%
% Individual scenario discussion
%
% False negatives: Scenarios 8/18/19
% \begin{itemize}
%     \item All scenarios were not identified compatible although they are
%     \item In all cases, the SMT solver returned \emph{unknown} although he should have returned \emph{unsat}, thus an actually redundant consistency relation was not removed and the consistency relation set was not considered compatible by mistake
%     \item In all cases, set operations were involved
%     \item More precisely, in scenario 8 a precondition checks that an element is included in the intersection of two set literals, which the solver was not able to check properly
%     \item In Scenario 18, transitive inclusion of sets was defined, which the solver was not able to check properly
%     \item Scenario 19 is comparable to scenario 18 but considers role names of classes with equivalent identifiers, which the solver was also not able to check properly.
% \end{itemize}
This is a first indicator for high applicability of the approach, as it was able to prove compatibility in most of the cases in which the relations were actually compatible.

\mnote{Reasons for false negatives}
The Scenarios $8$, $18$ and $19$ introduced in \autoref{tab:correctness_evaluation:compatibility_scenarios} were not identified as compatible although they actually are.
In all cases, the \gls{SMT} solver should have returned \emph{unsatisfiable} although it should have returned \emph{unknown}.
Thus, in each scenario an actually redundant consistency relation was not removed, thus not identifying the relations as compatible.
In detail, in Scenario $8$, a precondition ensures that an element is included in the intersection of two set literals, but the solver was not able to check that properly.
In Scenario $18$, the transitive inclusion of sets was defined, and in Scenario $19$, roles names of classes with equivalent identifiers were considered, which the solver was both not able to check properly as well.
In summary, all observed false negatives were caused by undecidability of satisfiability of the first-order logic formulae that were derived from the \gls{OCL} constructs.

% The three scenarios that were not classified, although they are actually compatible, are Scenarios 8, 18 and 19 from \autoref{tab:scenarios}.
% In all cases, the SMT solver returned \emph{unknown}, although it should have returned \emph{unsatisfiable}.
% In consequence, in each case an actually consistent consistency relation was not removed, thus the set of relations was not considered compatible although it is.
% More precisely, in scenario 8, a precondition ensures that an element is included in the intersection of two set literals, which the solver was not able to check properly.
% Scenarios 18 and 19 were problematic due to comparable reasons.
% While in Scenario 18 the transitive inclusion of sets was defined, Scenario 19 considers role names of classes with equivalent identifiers, which both the solver was not able to check properly.
% All observed false negatives were due to reasons of undecidability of the translation of OCL constructs to first-order logic.

\mnote{Answer to evaluation question}
In conclusion, the evaluation has shown that basic operations on primitive data types, even with non-trivial constraints involving integer equations and string operations, were treated correctly.
This led to a success rate of \SI{80}{\percent}.
As an answer to \autoref{eq:compatibility:applicability}, the approach was unable to prove compatibility in only \SI{20}{\percent} of the cases, in which more complex operations and structures requiring many quantifiers were involved and led to unprovability by the used \gls{SMT} solver.
Most importantly, however, this limitation does only concern the chosen \gls{SMT} solver approach, but neither the general concept of the formal framework and approach, nor of the practical realization itself.
In particular, we did not find a scenario in which our redundancy notion was too strict for proving compatibility.
Using a different \gls{SMT} solver or, more generally, even a different approach to validate redundancy of consistency relations can even improve the applicability results.

% %%
% %% Summary: Limitations arise from SMT solver
% %%
% To summarize, we found that basic operations on primitive data types, even with non-trivial constraints involving integer equations and string operations, were treated correctly.
% More complex operations and structures requiring many quantifiers led to unprovability by the SMT solver, especially concerning collection operations and role names.
% %In consequence, the reasons for conservative behavior in all cases were limitations of the chosen SMT solver approach, rather than the concept of either the formalized or the operationalized approach.
% Thus, the approach is especially applicable for consistency relations concerning attributes and primitive types.
% However, this limitation does only concern the chosen SMT solver approach, but neither the concept of operationalization nor of the formal framework behind it.
% We did especially not find a scenario, in which our definition of left-equal redundancy was too strict for proving compatibility.


\subsection{Discussion}

\mnote{Result summary}
The evaluation of our compatibility analysis approach has shown that in the scenarios considered in the case study it operates correctly and shows a low degree of conservativeness, i.e., it is not able to validate compatibility in only few cases.
This indicates correctness and high applicability of the approach.
Still, there are some threats to the validity of these results, which we will discuss after general conclusions on the benefits of the proposed approach.

\mnote{General benefits due to full automation}
In general, the approach is supposed to support transformation developers in designing transformation networks by checking compatibility of transformations, or more precisely their underlying consistency relations, during their individual development or when combining them to a network.
In \autoref{chap:compatibility:informal:prevention} with the example depicted in \autoref{fig:compatibility:unwanted_behavior}, we have shown incompatible consistency relations can prevent the transformations from finding consistent models.
Thus, incompatibilities will eventually lead to failing transformation network, which, in turn, require transformation developers to find the reasons for that.
Our approach provides a benefit by preventing such issues or at least by supporting the developer in finding the reasons for them when running the analysis after a failure occurs.
Due to its full automation, no further effort than running the analysis is necessary.
Additionally, a manual process of ensuring compatibility or finding incompatibilities requires manual alignment of transformations with each other or the defining of test cases, which are only able to validate compatibility, but not to verify it.
Thus, such manual techniques can only make existentially quantified statements about the existence of incompatibilities, whereas our approach can make universally quantified statements about the absence of incompatibilities.

\mnote{Benefits even with high degree of conservativeness}
Finally, even if the proposed approach had a high degree of conservativeness, i.e., if it produces a higher degree of false negatives in other scenarios than our evaluation, the approach still provides benefits.
First, the approach may still be able to prove compatibility at least in few cases.
Second, even if the approach cannot prove compatibility, it may at least detect some redundant relations and thus reduces the effort for the transformation developer to find incompatible relations.
If this is often necessary, it would be possible to define in interactive approach in which the removal of redundant relations by proof of the approach and by user decision is combined, which we will proposed in future work in \autoref{chap:futurework:correctness:compatibility:process}.
In such a process, the user could be asked to manually find and declare redundant relations if the automated approach is not able to find further ones.
Afterwards, the automated approach can proceed.


% %%
% %% Benefits of the approach
% %%
% The presented approach aims to support developers of transformation networks to independently develop individual transformations, i.e., parts of the network, without aligning the consistency relations on which the transformations are based a priori, but allows them to check their compatibility during or after development.
% For that reasons, it provides a benefit by automating a process that currently requires manual effort by either aligning consistency relations with each other or by defining test cases, which are able to validate but not to verify compatibility, i.e., which cannot make any all-quantified statements about compatibility.
% Even if the approach had a high degree of conservativeness, the approach would be beneficial for the combination of independently developed transformation.
% First, there is still a chance that the approach is able to prove compatibility for a given set of relations.
% Second, if the approach is not able to prove compatibility, it may still find some redundant relations and thus reduces the effort for the user to investigate the remaining relations for contradictions.
% It would even be possible to define an interactive approach, combining the removal of redundant relations by proof and by user decision, as we will propose in \autoref{sec:futurework:compatibilityprocess}.
% %
% In the following, we discuss threats to the validity of our evaluation and discuss limitations of both the approach and our evaluation.


\subsubsection{Threats to Validity}

\mnote{Limitations of external validity}
We have designed the evaluation carefully, such that it gives us appropriate insights regarding correctness and applicability of the approach.
Still, due to limitations in complexity of the considered scenarios there are threats especially regarding external validity of the results.
This is why we emphasized that all results only serve as an indicator for the properties to be shown.

%\paragraph{Internal Validity}
\mnote{Scenario selection}
The evaluation scenarios of the case study were developed specifically for the evaluation of the proposed approach.
Thus, they may potentially not sufficiently represent actual transformation networks.
On the other hand, the scenarios were designed to test different aspects of the approach, they represent an extensive set of consistency relations and also consider edge cases.
Scenarios not developed for the evaluation may not or only rarely cover specific and edge cases.
In fact, most meaningful results could potentially be achieved with a combination of externally developed scenarios and evaluation-specific scenarios.
However, the limited availability of scenarios, especially of scenarios developed with the tools we used for the prototype and those containing incompatibility, prevents this.

\mnote{Scenario complexity}
The defined scenarios only contain \gls{OCL} constructs that the approach currently supports.
Thus, unsupported constructs are not covered by the evaluation, which may be a bias.
The algorithm would, however, not yield a result in such scenarios anyway, thus this would not give further insights.
Additionally, this is only a limitation of the implementation rather than a conceptual limitation of the approach.
The actual threat in this is that more complex relations, which are currently not supported by the implementation, are not covered by our definition of redundancy anymore.
That would be an actual limitation not only of the implementation but also the formal approach.
In consequence, this has to be further evaluated in subsequent evaluations.

\mnote{Scenario size}
The considered scenarios only consider up to four metamodels with pairwise consistency relations.
Actual transformation networks will probably contain more and larger metamodels and consistency relations.
This is, however, not a threat to validity regarding correctness, because the inductive definition of the approach makes it independent from the number of metamodels and relations to consider.
It may only affect applicability, as increasing size may lead to logic formulae which the \gls{SMT} solver is not able to resolve anymore.
The size of scenarios may especially affect the performance and scalability of the approach, which we did not analyze in our evaluation and discuss in the subsequent limitations.

% In consequence, some of these scenarios may be rare in practice, such that 
% Scenario selection (artificial): Scenarios were specifically developed to test different aspects of the operationalized approach. Thus, they may not be representative for actual specifications used in transformation networks.
% But: Scenarios provide an extensive set of relations, concerning different types of relations and as different scenarios as possible. Some of the scenarios may be rare in practice, even giving a benefit explicitly specifying them.
% Discuss: Especially affects appropriateness of redundancy definition (relations not complex enough)

% Scenario size: Scenarios are rather small (few metamodels and relations). Actual scenarios might involve more metamodels and/or relations
% But: Inductive character of the approach makes it independent from the number of metamodels and relations.
% Size may only affect performance.

% Scenarios limited to supported constructs: May be a bias that unsupported QVT-R and OCL operations are not covered by the scenarios.
% But: Would not make sense, since the algorithm does not deliver any results for such relations.
% But: Is not a conceptual limitation but only a limitation of the current implementation.

\mnote{Conclusion of threats}
In consequence, our evaluation gives an initial indicator for the correctness and applicability of our approach based on well-selected evaluation scenarios but potentially restricted in external validity due to the limited set and complexity of scenarios.
To improve evidence in external validity, applying the approach to further and larger transformation networks would be beneficial.
However, acquiring such a networks is difficult.
Especially in existing network, transformations can be expected to be aligned with each other, thus not containing incompatibilities and limiting the evaluation to positive cases.
A possibility to reduce that problem would be the manual extension or alternation of such networks by adding transformations with redundant or incompatible consistency relations.
This would directly deliver a ground truth against which the results of the approach on these modified network can be validated.

%\paragraph{External Validity}
% Only initial results regarding correctness and applicability due to limited set of scenarios.
% But: Conservativeness is proven.
% But: Hard to acquire further transformation networks, especially those in which the transformations were not developed together and are thus aligned. Problem may be reduced by simulated redundancy, i.e., taking a set of compatible consistency relations and adding redundant one, validating whether our approach is able to remove them again and deliver a compatible network.


\subsubsection{Limitations}
\label{chap:correctness_evaluation:compatibility:discussion:limitations}

\mnote{Implementation and evaluation limitations}
We discuss two types of limitations of our approach.
First, we consider limitations of the current state of implementation.
Second, we discuss limitations of the current state of evaluation, which may have masked limitations of the current concept.

% Current limitations of our approach especially arise from its operationalization and the limitations of SMT solvers.
% Additionally, we are currently only able to argue for the benefits and performance of our approach, but further evaluation would be necessary to validate these arguments.

\mnote{Practical approch realization}
The proposed practical approach for \gls{QVTR} has fundamental as well as technical limitations.
First, \gls{SMT} solvers are limited such that they cannot analyze all kinds of formulae regarding satisfiability.
Thus, even if we can transform all kinds of \gls{QVTR} and \gls{OCL} constructs into logic formulae, they cannot necessarily be checked for satisfiability, as we have shown in the applicability evaluation.
Second, we do not yet support all kinds of \gls{QVTR} relations, as we do not yet provide a transformation for all kinds of \gls{OCL} constructs into logic formulae.
This is, however, only a technical limitation that can be solved by additional implementation effort.
We discuss this as future work in \autoref{chap:futurework:correctness:compatibility:completion}.
Additionally, we discuss the ability to use completely different realization approaches to circumvent limitations of \gls{SMT} solvers as future work in \autoref{chap:futurework:correctness:compatibility:alternatives}.

\mnote{Benefits evaluation}
We have not provided an evaluation for the benefits that we claim for our approach.
First, to the best of our knowledge, there are not competitive approaches to compare our one with.
Second, it automates a manual process without requiring additional effort, thus compared to the baseline of performing the process manually it provides any inherent and essential benefit.
Thus, further empirical evaluation in a user study could only provide a quantitative measure of the benefits rather than the qualitative one we give by argumentation.
Such an evaluation could especially consider a development process in which the approach is used and evaluate whether that whole process improves by using our approach.
We discuss such a process as future work in \autoref{chap:futurework:correctness:compatibility:process}.

\mnote{Optimality of compatibility notion}
The notion of compatibility was developed from the goal of finding contradictory consistency relations that can prevent transformations from finding consistent models after changes.
Additionally, it prevents from the specification of contradictory and thus unintended consistency relations.
Although we have shown at examples that our notion of compatibility fulfills both these notions, it is unclear whether this notion is kind of optimal in the sense that there exists no other notion that covers even more unwanted cases.
Evaluating the central purpose of the approach to improve the ability of transformations to find consistent models, i.e., to improve dealing with the orchestration problem, is part of our future work (see \autoref{chap:futurework:correctness:compatibility:orchestration}).
%\todo{future work on benefits for consistency repair}
%\todo{Not shown that compatibility notion is reasonable. We derived it theoretically, but whether its actually fits to some intuitive notion of incompatibility and covers all cases of consistency relations that do actually not fit together and could be detected automatically, is unclear.
%Maybe the notion is even to restricted because it is necessary to pass such states that are not forbidden, but as discussed, this is really unlikely, because the models can never occur in any consistent tuple of models.}
% On the other hand, the compatibility notion may even be too restrictive in two senses.
% First, it may exclude the consideration of models states that are actually necessary as transient states in a transformation network. As we have discussed in \autoref{chap:compatibility:informal:prevention} this is unlikely because the excluded models can never occur in any consistent tuple of models.
% This does, however, lead to the situation that specific user inputs are forbidden, although they could be resolved by the transformations.
% For example, consistency relations may require some string attribute to start uppercase.
% If the user enters a value starting lowercase, the transformation could easily resolve that.
% The notion of compatibility, however, forbids that, because no models with the lowercase attribute value can ever be consistent

\mnote{Optimality of redundancy notion}
We have already discussed that we defined the specific notion of left-equal redundancy (see \autoref{def:leftequalredundancy}), which has the property of being compatibility-preserving (see \autoref{theorem:redundancycompatibility}).
It is however unclear whether there may be a more relaxed notion of redundancy that is still compatibility-preserving.
Our implementation follows an even stricter notion of redundancy and still no limitations of applicability occurred.
If, however, other case studies reveal the necessity of a weaker redundancy notion to be able to prove compatibility in more cases, either the notion used in the implementation needs to be relaxed or even the formal foundation needs to be adapted.
We discuss that as future work in \autoref{chap:futurework:correctness:compatibility:redundancy}.
%\todo{Future Work: Relaxation of redundancy}

% From Slides:
% \begin{itemize}
%     \item Conservativeness: Undecidability leads to false negatives
%     \item Strictness: Some formally excluded cases possibly should be allowed
% \end{itemize}
% Both ensure that no false positives are produced
% DAS STIMMT NICHT: Die Transformationen können natürlich einen lowercase Wert zu uppercase machen. Das geht aber unabhängig von Kompatibilität, denn deswegen muss ein lowercase Wert ja noch lange nicht kompatibel sein.
% Strictness Limitation
% \begin{itemize}
%     \item Formalism cannot (easily) be adapted context-specifically
%     \item Example: A String attribute is changed, and the transformations ensure that it starts uppercase
%     \begin{itemize}
%         \item Relations may be incompatible because for lowercase values no globally consistent models exist
%         \item However, such behavior may be wanted
%     \end{itemize}
%     \item Hard to encode this into a formalism
% \end{itemize}
% Solution: Manually declare redundancies that do not follow the formalism -> Future Work

\mnote{Performance and scalability evaluation}
We have neither measured nor formally evaluated the performance and scalability of our approach and especially its practical realization.
Applicability may be affected if the approach required too much time to be executed.
\gls{SMT} solvers, such as the used Z3 solver, depend on heuristics, which makes their performance unpredictable.
Thus, it would be important to evaluate performance of the approach in a case study.
In our case study, %introduced in \autoref{sec:evaluation:methodology}, 
we did not observe any time-consuming scenarios.
However, transformation networks with more and larger transformations and especially many cycles of consistency relations need to be investigated to make generalizable statements on the performance and especially the scalability of the approach.
Since the approach is applied as an offline analysis, which does not require instant feedback, it must not fulfill real-time requirements.
Results should, however, still occur in an acceptable amount of time to achieve acceptance of the approach.
%\todo{Scalability analysis, no performance evaluation made}


% Additional:
% Although compatibility ensures that for any model element with restrained consistency a consistent model tuple can be found, it does not mean that consistency preservation rules find that model tuple. It only improved the ability to do so.

% \iffalse
% Functional correctness:
%     - Theoretical evaluation based on concepts and definitions of this paper
%     -> Functioning of the procedure, algorithms
    
% Applicability:
%     - Empirical evaluation based on an implementation of the decomposition procedure
%     - Interpretation of test results + metric achieved results against expected results
%     -> Example scenarios, execution results
% \fi

