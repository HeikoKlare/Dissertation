\section{Knowledge Levels in Transformation Specifications}

\mnote{Levels by available knowledge}
The process of specifying a transformation network can be considered at different conceptual levels depending on the knowledge a developer must have to ensure correctness at that level.
For example, at the lowest level a developer may only know that a transformation shall be used within a network without knowing the actual network, which only allows to avoid specific errors, whereas further errors are relevant and need to be considered when having knowledge about the other transformations to combine it with.
In addition, depending on the level of abstraction that a specification formalism, such as a transformation language, provides, the developer must only deal with some of these levels as the language abstracts from the others, which determines the resulting challenges a developer has to deal with.
In consequence, these levels are supposed to mean that specific kinds of mistakes can be made at each of them and that a formalism may ensure correctness with respect to one of those levels and the ones below, whereas the transformation developer is still responsible for avoiding mistakes at the levels above.

\begin{propertable}
    \rowcolors{1}{\firstlinecolor}{\secondlinecolor}
    \begin{tabular}{C{2.4em+0.05\difftoafiveimage}L{3.9em+0.83\difftoafiveimage}L{10.8em+0.1\difftoafiveimage}L{11.3em+0.02\difftoafiveimage}}
        \toprule
        \textbf{Level} & \textbf{Name} & \textbf{Correctness} & \textbf{Knowledge} \\
        \midrule
        1 & \LevelTransformation & Synchronizing transformations & Individual transformation \\
        2 & \LevelNetworkRelation & Compatible consistency relations & Consistency relations of complete network \\
        3 & \LevelNetworkRule & Interoperable consistency preservation rules & Transformations of complete network \\
        \bottomrule
    \end{tabular}
    \caption[Knowledge levels in transformation network specification]{Distinguished levels in the transformation network specification process with their correctness criteria and required knowledge.}
    \label{tab:errors:levels}
\end{propertable}

\mnote{Three distinguished levels}
We distinguish three such levels, which we summarize in \autoref{tab:errors:levels} together with their properties and discuss them in the following.
At the \emph{\leveltransformation level}, we consider the specific properties of a single transformation to be used in a, or more precisely any transformation network, especially involving synchronization.
At the \emph{\levelnetworkrelation level}, we consider the interplay of the binary consistency relations of a concrete set of transformations.
At the \emph{\levelnetworkrule level}, we consider the interplay of the consistency preservation rules of a concrete set of transformations.
These levels depend on each other, because, for example, consistency preservation rules cannot properly work together if each on its own is not at least synchronizing and thus correct at the \leveltransformation level.
Nevertheless, a transformation can be correct at the \leveltransformation level without being correct at the network relation and \levelnetworkrule level.

\mnote{Required knowledge per level}
These levels especially differ in what knowledge they require to be able to deal with and even avoid potential errors.
For the \leveltransformation level, it is sufficient to know that a transformation may be used in a transformation network without knowing the actual network.
For the \levelnetworkrelation level, at least the relations of the other transformations in the network must be known.
Finally, for the \levelnetworkrule level, the transformations of the complete network must be known.
This influences how far errors at the different levels can be avoided, first, because of the required knowledge to do so and, second, because of the possibility to ensure correctness at all.


\subsection{Knowledge-Dependent Specification Levels}

\mnote{Levels for modular consistency notion}
In the following, we introduce the three mentioned levels more precisely.
They represent a revised version of the three levels we have presented in previous work~\owncite{klare2019icmt}.
In that work, we have discussed the \emph{global} level, which considers the global knowledge in terms of the overall, multiary relation between all involved models.
We have, however, discussed different correctness notions in \autoref{chap:correctness:notions_correctness} and argued why we do not consider a \emph{monolithic} notion of consistency, which conforms to the \emph{global} specification level, as we do not assume this global knowledge to be represented explicitly, such that it would not make sense to explicitly consider correctness according to it.

\begin{properdescription}
%\mnote{Knowledge about single transformation}
\item[Level 1 (\emph{\LevelTransformation}):] \mnote{Knowledge about single transformation}
At the first level, we only consider the knowledge that a transformation shall be used within a transformation network.
According to our formalism presented in \autoref{chap:correctness:formalization}, this means that the transformation needs to be \emph{synchronizing}.
We have discussed in \autoref{chap:synchronization} how synchronization can be achieved with ordinary transformation languages.
Correctness at this level is given by the fulfillment of the synchronization property for a transformation.

%\mnote{Knowledge about consistency relations of network}
\item[Level 2 (\emph{\LevelNetworkRelation}):] \mnote{Knowledge about consistency relations of network}
At the second level, we consider the knowledge about the actual network in which the transformations shall be used, but restricted to their relations.
In consequence, it would be possible that the relations between all models are known, e.g., because there is a common understanding of the relations, which may also be documented.
We have discussed in \autoref{chap:compatibility} that \emph{compatibility} is a relevant property of the consistency relations in a transformation network to ensure that the transformations are able to find consistent models after changes.
Correctness at this level is thus given by compatibility of the consistency relations.

%\mnote{Knowledge about complete transformations of network}
\item[Level 3 (\emph{\LevelNetworkRule}):] \mnote{Knowledge about complete transformations of network}
At the third level, we consider the knowledge about the complete transformations of an actual network, thus especially also the consistency preservation rules that preserve consistency.
In \autoref{chap:orchestration}, we have discussed the problem of orchestrating these rules and also discussed several issues that may prevent an algorithm from finding a consistent orchestration, such as the selection of different, conflicting options provided by a consistency relation to restore consistency.

\end{properdescription}


\subsection{Abstraction to Specification Levels}

\mnote{Inherent correctness by formalism}
All three levels are relevant during the specification process of a transformation network, and potential mistakes that can be made at each of them need to be avoided.
As mentioned before, a specification formalism, usually a transformation language, provides a specific level of abstraction associated with one of the conceptual levels introduced above, which relieves the developer from dealing with potential problems of the lower levels.
He or she must, however, still ensure correctness with respect to all higher levels.

\mnote{Inherent correctness at \leveltransformation level}
At the lowest level, a transformation language may not ensure correctness regarding any of the levels.
For example, an imperative, unidirectional transformation language requires the developer to ensure synchronization of transformations at the \leveltransformation level, compatibility of the relations at the \levelnetworkrelation level, as well as interoperability of the consistency preservation rules at the \levelnetworkrule level.
Some declarative, bidirectional transformation languages already relieve the developer from specifying consistency preservation rules and lift the abstraction to consistency relations, from which consistency preservation rules are automatically derived.
Some languages even relieve the developer from manually ensuring synchronization, for example, by using keys for matching existing elements in \gls{QVTR}.
In this case, the transformation engine ensures correctness at the \leveltransformation level, but the developer still has to ensure it for the other levels.
Then, the developer only needs to deal with problems at the higher levels.
Integrating an analysis for compatibility, such as the one proposed in \autoref{chap:compatibility}, into \gls{QVTR} could thus also abstract from the \levelnetworkrelation level.

\mnote{Formalisms for higher-level correctness}
To the best of the author's knowledge, languages that ensure correctness at higher levels than the \leveltransformation level are currently uncommon.
This would either require the specification of multidirectional transformations, i.e., a less modular or even monolithic notion of consistency (see \autoref{chap:correctness:notions_correctness}), or at least additional analysis functionality integrated into the languages to, for example, ensure compatibility and thus correctness at the \levelnetworkrule level.
Multidirectional \gls{QVTR}~\cite{macedo2014FrameworkMultiDirectional-BX} or extensions of \glspl{TGG} to multiple models~\cite{koenigs2006MGGs-SoSym,trollmann2015TransformationTGGtoMultiModel-ICMT,trollmann2016SynchronizationTGGtoMultiModel-ICMT} provide means to define rules between multiple models, from which then consistency preservation rules between two models are derived, thus abstracting from the problems of ensuring rule compatibility and interoperability of consistency preservation rules.
The \commonalities language~\owncite{gleitze2017a}, which we present in detail in \autoref{chap:improvement}, lifts the abstraction such that the \levelnetworkrelation and \levelnetworkrule levels do not have to be considered by the transformation developer.
This is, however, achieved by a specific network topology induced by that language, which avoids several of the problems that we discussed for networks of arbitrary topologies.

\mnote{Dependencies between levels}
Correctness at the higher conceptual levels always requires correctness at the lower levels.
Especially the interoperability of transformations at the \levelnetworkrule level requires the transformations to be synchronizing, i.e., correct at the \leveltransformation level, and the relations to be compatible, i.e., to be correct at the \levelnetworkrelation level.
In fact, compatibility of the relations does not require the transformations to be synchronizing, thus the \levelnetworkrelation level does, theoretically, not require correctness at the \leveltransformation level.
From a knowledge perspective, it does, however, not make sense to ensure compatibility of relations when their transformations  are not even synchronizing, because synchronization of a transformation can already be ensured independent from the other transformations to combine it with, whereas this knowledge is required for ensuring compatibility.
