\subsection{Compatible Consistency Relation Set Structures}

%%
%% Properties for inherent compatibility
%%
We first consider two essential properties of a consistency relation set that lead to its inherent compatibility:
\begin{enumerate}
    \item Composability: We show that the union of independent, compatible sets of consistency relations is compatible.
    \item Trees: We show that relations fulfilling a special notion of \emph{consistency relation trees} are inherently compatible.
\end{enumerate}
In consequence, we know that a consistency relation set that is composed of independent subsets of consistency relation trees is inherently compatible.

We consider consistency relation sets as independent if there are no transitive consistency relations induced by relations from both sets, i.e., for each object in a model consistency is only restricted by one of those sets.

\begin{definition}[Independence of Consistency Relation Sets]
    \label{def:independence}
    Let $\consistencyrelationset{CR}_{1}$ and $\consistencyrelationset{CR}_{2}$ be two sets of consistency relations. We say that:
    \begin{align*}
        \formulaskip &
        \consistencyrelationset{CR}_{1} \mathtext{and} \consistencyrelationset{CR}_{2} \mathtext{are independent} \equivalentperdefinition \\
        & \formulaskip
        \forall \consistencyrelation{CR}{} \in \consistencyrelationset{CR}_{1} : \forall \consistencyrelation{CR'}{} \in \consistencyrelationset{CR}_{2} : \\
        & \formulaskip 
        \forall \consistencyrelation{CR}{1}, \dots, \consistencyrelation{CR}{k} \in \consistencyrelationset{CR}_{1} \cup \consistencyrelationset{CR}_{2} : \\
        & \formulaskip\formulaskip
        \consistencyrelation{CR}{} \concat \consistencyrelation{CR}{1} \concat \dots \concat \consistencyrelation{CR}{k} \concat \consistencyrelation{CR'}{} = \emptyset \\
        & \formulaskip\formulaskip
        \land \consistencyrelation{CR'}{} \concat \consistencyrelation{CR}{1} \concat \dots \concat \consistencyrelation{CR}{k} \concat \consistencyrelation{CR}{} = \emptyset
    \end{align*}
    We call $\consistencyrelationset{CR}$ \emph{connected} if there is no partition of a consistency relation set $\consistencyrelationset{CR}$ into two subsets that are independent, i.e.
    \begin{align*}
        \formulaskip &
        \forall \consistencyrelationset{CR}_{1}, \consistencyrelationset{CR}_{2} \subseteq \consistencyrelationset{CR} : \\
        & \formulaskip 
        \consistencyrelationset{CR}_{1} \cap \consistencyrelationset{CR}_{2} = \emptyset \land \consistencyrelationset{CR}_{1} \cup \consistencyrelationset{CR}_{2} = \consistencyrelationset{CR}  \\
        & \formulaskip\formulaskip
        \Rightarrow \neg (\consistencyrelationset{CR}_{1} \mathtext{and} \consistencyrelationset{CR}_{2} \mathtext{are independent),}
    \end{align*}
\end{definition}

\begin{figure}
    \centering
    \newcommand{\hdistance}{19em}
\newcommand{\vdistance}{2.5em}
\newcommand{\classwidth}{6em}

\begin{tikzpicture}

% Resident
\umlclassvarwidth{resident}{}{Resident\sameheight}{
name
}{\classwidth}

% Employee
\umlclassvarwidth[, right=\hdistance of resident.north, anchor=north]{employee}{}{Employee\sameheight}{
name
}{\classwidth}

% Location
\umlclassvarwidth[, below=\vdistance of resident.south, anchor=north]{location}{}{Location\sameheight}{
street
}{\classwidth}

% Address
\umlclassvarwidth[, below=\vdistance of employee.south, anchor=north]{address}{}{Address\sameheight}{
street
}{\classwidth}

% CONSISTENCY RELATIONS
\draw[consistency relation] (resident.east) -- node[pos=0, above right] {$e$} node[pos=0.5, below, align=center] {
    $\consistencyrelation{CR}{1}$ = \\
    $\setted{ \tupled{r,e} \mid e.name = r.name}$
} node[pos=1, above left] {$r$} (employee.west);

\draw[consistency relation] (location.east) -- node[pos=0, above right] {$l$} node[pos=0.5, below, align=center] {
    $\consistencyrelation{CR}{2}$ = \\
    $\setted{ \tupled{l,a} \mid l.street = a.street}$
} node[pos=1, above left] {$a$} (address.west);

\end{tikzpicture}
%    \includegraphics[width=\columnwidth]{figures/independence_example.png}
    \caption{Two independent (sets of) consistency relations}
    \label{fig:correctness:formal:independence_example}
\end{figure}

%We call two consistency relation sets independent if the sets of elements that they relate to each other are completely independent of each other.
\begin{example}
\autoref{fig:correctness:formal:independence_example} depicts a simple example with two consistency relations $\consistencyrelation{CR}{1}$ and $\consistencyrelation{CR}{2}$, each relating instances of two disjoint classes with each other.
Since there is no overlap in the objects that are related by the consistency relations, they are considered independent according to \autoref{def:independence}.
\end{example}

An important property of independent sets of consistency relations is that computing their union is compatibility-preserving, i.e., the union of compatible, independent consistency relation sets is compatible as well:

\begin{theorem} \label{theorem:independencecompatibility}
    Let $\consistencyrelationset{CR}_{1}$ and $\consistencyrelationset{CR}_{2}$ be two compatible sets of consistency relations. Then $\consistencyrelationset{CR}_{1} \cup \consistencyrelationset{CR}_{2}$ is compatible.
\end{theorem}

\todo{Revise proof with explicit references to independence definition}
\begin{proof}
    Since $\consistencyrelationset{CR}_{1}$ is compatible, per definition there is a model set $\modelset{m}$ for each condition element $\conditionelement{c}{}$ of the left condition of each consistency relation in $\consistencyrelationset{CR}_{1}$ that contains $\conditionelement{c}{}$ and that is consistent to $\consistencyrelationset{CR}_{1}$.
    Taking such an $\modelset{m}$, we create a new $\modelset{m'}$ by removing all elements from $\modelset{m}$, which are contained in any condition elements in any consistency relation in $\consistencyrelationset{CR}_{2}$ and thus potentially require other elements to occur to be considered consistent to that consistency relation.
    In consequence, $\modelset{m'}$ does not contain any condition elements from consistency relations in $\consistencyrelationset{CR}_{2}$ and is thus consistent to $\consistencyrelationset{CR}_{2}$ by definition. 
    Additionally, $\modelset{m'}$ is still consistent to $\consistencyrelationset{CR}_{1}$, because due to the independence of $\consistencyrelationset{CR}_{1}$ and $\consistencyrelationset{CR}_{2}$, there cannot be any consistency relations in $\consistencyrelationset{CR}_{1}$, which require the existence of the removed elements.
    In consequence, for each condition element $\conditionelement{c}{}$ of each consistency relation in $\consistencyrelationset{CR}_{1}$ there is a model set that contains $\conditionelement{c}{}$ and that is consistent to $\consistencyrelationset{CR}_{1} \cup \consistencyrelationset{CR}_{2}$.
    The analogous argumentation applies to the consistency relations in $\consistencyrelationset{CR}_{2}$, which is why the definition of compatibility is fulfilled for all condition elements of all consistency relations in $\consistencyrelationset{CR}_{1} \cup \consistencyrelationset{CR}_{2}$.
\end{proof}

The constructive proof can also be reflected exemplarily in \autoref{fig:correctness:formal:independence_example}: Take any set of models that, for example, contains a resident with an arbitrary name and is consistent to $\consistencyrelation{CR}{1}$, i.e., that also contains an employee with the same name.
If that set of models contains any addresses or locations, they can be removed %without removing the resident from the model and 
without violating consistency to $\consistencyrelation{CR}{1}$, because addresses and locations are independently related by $\consistencyrelation{CR}{2}$.

\todo{Actually, addresses may not be removable because they are referenced by persons. However, if such a reference always needs to be set, this is a restriction that we do not reflect yet in the metamodel formalism. On the other hand, if the reference was relevant for consistency in the first consistency relation, it would have to be considered there as well, thus address would be part of that consistency relation.}

\todoDiss{Maybe we can remove symmetry and define some restriction for inverse relations. It could be useful to think about implicit relations, which are induced by another one, so that the "signatures" of forward and backward direction match.}
\begin{definition}[Consistency Relation Tree] \label{def:relationtree}
    Let $\consistencyrelationset{CR}$ be a symmetric, connected set of consistency relations. 
    %Let $\alpha = \setted{\tupled{\consistencyrelation{CR}{1}, \consistencyrelation{CR}{2}} \in \consistencyrelationset{CR} \times \consistencyrelationset{CR} \mid \classtuple{C}{r,\consistencyrelation{CR}{1}} \cap \classtuple{C}{l,\consistencyrelation{CR}{2}} \neq \emptyset \land \classtuple{C}{r,\consistencyrelation{CR}{2}} \cap \classtuple{C}{l,\consistencyrelation{CR}{1}} = \emptyset}$ be a relation on the consistency relations in $\consistencyrelationset{CR}$.
    We say:
    \begin{align*}
        \formulaskip &
        \consistencyrelationset{CR} \mathtext{is a consistency relation tree} \equivalentperdefinition \\
        & \formulaskip
        %\alpha \mathtext{induces a tree on} \consistencyrelationset{CR}
        \forall \consistencyrelation{CR}{} = \consistencyrelation{CR}{1} \concat \dots \concat \consistencyrelation{CR}{m}  \in \transitiveclosure{\consistencyrelationset{CR}} : \\
        & \formulaskip
        \forall \consistencyrelation{CR'}{} = \consistencyrelation{CR'}{1} \concat \dots \concat \consistencyrelation{CR'}{n} \in \transitiveclosure{\consistencyrelationset{CR}} \setminus \consistencyrelation{CR}{} : \\
        & \formulaskip\formulaskip
        \forall s, t \mid s \neq t: 
        \consistencyrelation{CR}{s} \neq \consistencyrelation{CR^T}{t} \land \consistencyrelation{CR'}{s} \neq \consistencyrelation{CR'^T}{t} \\
        & \formulaskip\formulaskip
        \Rightarrow
        \classtuple{C}{l,\consistencyrelation{CR}{}} \cap
        \classtuple{C}{l,\consistencyrelation{CR'}{}} = \emptyset
        \lor \classtuple{C}{r,\consistencyrelation{CR}{}} \cap
        \classtuple{C}{r,\consistencyrelation{CR'}{}} = \emptyset
        %
        % \forall \tupled{\consistencyrelation{CR}{1}, \dots, \consistencyrelation{CR}{k}},  \tupled{\consistencyrelation{CR'}{1}, \dots, \consistencyrelation{CR'}{m}} \in \consistencyrelationset{CR} : \\
        % & \formulaskip\formulaskip
        % \tupled{\consistencyrelation{CR}{1}, \dots, \consistencyrelation{CR}{k}} \neq \tupled{\consistencyrelation{CR'}{1}, \dots, \consistencyrelation{CR'}{m}} \\
        % & \formulaskip\formulaskip\formulaskip
        % \land \forall i, j \mid i \neq j: 
        % \consistencyrelation{CR}{i} \neq \consistencyrelation{CR^T}{j} \land \consistencyrelation{CR'}{i} \neq \consistencyrelation{CR'^T}{j} \\
        % & \formulaskip\formulaskip
        % \Rightarrow
        % \classtuple{C}{l,\consistencyrelation{CR}{1}} \cap
        % \classtuple{C}{l,\consistencyrelation{CR'}{1}} = \emptyset
        % \lor \classtuple{C}{r,\consistencyrelation{CR}{m}} \cap
        % \classtuple{C}{r,\consistencyrelation{CR'}{m}} = \emptyset
        %
        % \forall \consistencyrelation{CR}{1}, \dots, \consistencyrelation{CR}{k} \in \consistencyrelationset{CR} \mid \consistencyrelation{CR}{i} \neq \consistencyrelation{CR}{j}, \consistencyrelation{CR}{i} \neq \consistencyrelation{CR^T}{j} : \\
        % & \formulaskip\formulaskip
        % \classtuple{C}{l,\consistencyrelation{CR}{1}\concat\dots\concat\consistencyrelation{CR}{k}} \cap
        % \classtuple{C}{r,\consistencyrelation{CR}{1}\concat\dots\concat\consistencyrelation{CR}{k}} = \emptyset
        % \forall \class{C}{l} \in
        % \classtuple{C}{l,\consistencyrelation{CR}{1}\concat\dots\concat\consistencyrelation{CR}{k}} :
        % \forall \class{C}{r} \in \classtuple{C}{r,\consistencyrelation{CR}{1}\concat\dots\concat\consistencyrelation{CR}{k}}: \\
        % & \formulaskip\formulaskip
        % \class{C}{l} \cap \class{C}{r} = \emptyset
    \end{align*}
\end{definition}
\todo{We have to assume, that no element is mapped to two elements of the same class, because then it would be possible to have an incompatible network}

The definition of a consistency relation tree requires that there are no sequences of consistency relations that put the same classes into relation, i.e. between all pairs of classes there is only one concatenation of consistency relations that puts them into relation.
Since we assume a symmetric set of consistency relations, we exclude the symmetric relations from that argument, as otherwise there would always be two such concatenations by adding a consistency relation and its transposed relation to any other concatenation.

\begin{figure}
    \centering
    \newcommand{\hdistance}{14em}
\newcommand{\classwidth}{6em}

\begin{tikzpicture}

% Person
\umlclassvarwidth{person}{}{Person\sameheight}{
firstname\\
lastname
}{\classwidth}

% Employee
\umlclassvarwidth[,above right=3.5em and \hdistance of person.center, anchor=south]{employee}{}{Employee\sameheight}{
name
}{\classwidth}

\umlclassvarwidth[,right=\hdistance of person.south, anchor=south]{resident}{}{Resident\sameheight}{
name
}{\classwidth}


% CONSISTENCY RELATIONS
\draw[consistency relation] (resident.west-|person.east) -- node[pos=0, above right] {$p$} node[pos=0.5, below] {$\consistencyrelation{CR}{1}$} node[pos=1, above left] {$r$} (resident.west);
\draw[consistency relation] (resident.north) -- node[pos=0, above left] {$r$} node[right, align=left] {$\consistencyrelation{CR}{2}$} node[pos=1, below left] {$e$} (employee.south);


\node[consistency related element, below left=1em and 2em of person.south west, anchor=north west] {
$\begin{aligned}
    \consistencyrelation{CR}{1} =\; & \setted{\tupled{p,r} \mid r.name = p.firstname + "\textnormal{\textvisiblespace}" + p.lastname}\\ %[0.3em]
    \consistencyrelation{CR}{2} =\; & \setted{\tupled{r,e} \mid r.name = e.name}
\end{aligned}$
};

\end{tikzpicture}
    %\includegraphics[width=\columnwidth]{figures/tree_example.png}
    \caption{A consistency relation tree $\setted{\consistencyrelation{CR}{1}, \consistencyrelation{CR^T}{1}, \consistencyrelation{CR}{2}, \consistencyrelation{CR^T}{2}}$}
    \label{fig:correctness:formal:tree_example}
\end{figure}

\begin{example}
\autoref{fig:correctness:formal:tree_example} depicts a rather simple consistency relation tree. 
Persons are related to residents and residents are related to employees, all having the same names respectively a concatenation of $firstname$ and $lastname$, by the relations $\consistencyrelation{CR}{1}, \consistencyrelation{CR}{2}$, as well as their transposed relations $\consistencyrelation{CR^T}{1}, \consistencyrelation{CR^T}{2}$.
There are no classes that are put into relation across different paths of consistency relations, thus the definition for a consistency relation tree is fulfilled. 
If an additional relation between persons and employees was specified, like in \autoref{fig:prologue:three_persons_example}, the tree definition would not be fulfilled.
\end{example}

The definition also covers the more complicated case in which multiple classes may be put into relation by consistency relations but there is only a subset of them that is put into relation by different consistency relations.
%
\todoDiss{Subsection with discussion about why hypertrees are not suitable here}
%
We can now prove that a set of consistency relations that is a consistency relation tree is always compatible.
We first present a lemma that shows that in a consistency relation tree you can always find an order of the relations such that the classes at the right side of a relation do not overlap with the classes at the left side of a relation that preceded in the order, i.e. there is no cycle in the relations between classes.

\begin{lemma} \label{lemma:treehassequence}
    Let $\consistencyrelationset{CR} = \setted{\consistencyrelation{CR}{1}, \consistencyrelation{CR^T}{1}, \dots, \consistencyrelation{CR}{k}, \consistencyrelation{CR^T}{k}}$ be a symmetric, connected set of consistency relations.
    $\consistencyrelationset{CR}$ is a consistency relation tree if and only if for each $\consistencyrelation{CR}{}$ there exists a sequence of consistency relations $\tupled{\consistencyrelation{CR'}{1}, \dots, \consistencyrelation{CR'}{k}}$ with $\consistencyrelation{CR'}{1} = \consistencyrelation{CR}{}$, containing for each $i$ either $\consistencyrelation{CR}{i}$ or $\consistencyrelation{CR^T}{i}$, i.e.,
    \begin{align*}
        \formulaskip &
        \forall i \in \setted{1, \dots, k} :\\
        & \formulaskip 
        \bigl( \consistencyrelation{CR}{i} \in \tupled{\consistencyrelation{CR'}{1}, \dots, \consistencyrelation{CR'}{k}}
        \land \consistencyrelation{CR^T}{i} \not\in \tupled{\consistencyrelation{CR'}{1}, \dots, \consistencyrelation{CR'}{k}} \bigl)\\
        & \formulaskip 
        \lor \bigl(\consistencyrelation{CR^T}{i} \in \tupled{\consistencyrelation{CR'}{1}, \dots, \consistencyrelation{CR'}{k}}
        \land \consistencyrelation{CR}{i} \not\in \tupled{\consistencyrelation{CR'}{1}, \dots, \consistencyrelation{CR'}{k}} \bigl)
    \end{align*}
    such that:
    \begin{align*}
        \formulaskip &
        %\forall \consistencyrelation{CR}{} \in \consistencyrelationset{CR} : 
        %\exists \consistencyrelation{CR'}{1}, \dots, \consistencyrelation{CR'}{k-1} \in \consistencyrelationset{CR} \setminus \setted{\consistencyrelation{CR}{}} :
        %\consistencyrelation{CR'}{i} \neq \consistencyrelation{CR'}{j}, i \neq j : \\
        \forall s \in \setted{1, \dots, k-1} : \forall t \in \setted{i+1, \dots, k} : \\
        & \formulaskip
        \classtuple{C}{r,\consistencyrelation{CR'}{s}} \cap \classtuple{C}{r,\consistencyrelation{CR'}{t}} = \emptyset 
        \land
        \classtuple{C}{l,\consistencyrelation{CR'}{s}} \cap 
        \classtuple{C}{r,\consistencyrelation{CR'}{t}} = \emptyset
        % \forall i \in \setted{1, \dots, k-1} : \\
        % & \formulaskip
        % (\exists j \in \setted{1, \dots, i-1} :  \classtuple{C}{l,\consistencyrelation{CR'}{j}} \subseteq \classtuple{C}{r,\consistencyrelation{CR'}{i}}) \\
        % & \formulaskip
        % \lor 
        % \classtuple{C}{l,\consistencyrelation{CR'}{}} \subseteq \classtuple{C}{r,\consistencyrelation{CR'}{i}}
    \end{align*}
\end{lemma}

\begin{proof}
    We start with the forward direction, i.e., given a consistency relation tree $\consistencyrelationset{CR}$ we show that there exists a sequence according to the requirements in \autoref{lemma:treehassequence} by constructing such a sequence $\tupled{\consistencyrelation{CR'}{1}, \dots, \consistencyrelation{CR'}{k}}$ for any $\consistencyrelation{CR}{} \in \consistencyrelationset{CR}$.
    Start with $\consistencyrelation{CR'}{1} = \consistencyrelation{CR}{}$ for any $\consistencyrelation{CR}{} \in \consistencyrelationset{CR}$.
    We now inductively add further relations to that sequence.
    Take any consistency relation $\consistencyrelation{CR}{s} = \consistencyrelation{CR}{s,1} \concat \dots \concat \consistencyrelation{CR}{s,m} \in \transitiveclosure{\consistencyrelationset{CR}}$ with $\classtuple{C}{l,\consistencyrelation{CR}{s,1}} \subseteq \classtuple{C}{r,\consistencyrelation{CR}{}}$. Such a sequence must exist because of $\consistencyrelation{CR}{}$ being connected.
    Now add all $\consistencyrelation{CR}{s,1}, \dots, \consistencyrelation{CR}{s,m}$ to the sequence, which fulfills both requirements to that sequence in \autoref{lemma:treehassequence} by definition.
    The following addition of further consistency relations can be inductively applied.
    Take any other consistency relation $\consistencyrelation{CR}{t} = \consistencyrelation{CR}{t,1} \concat \dots \concat \consistencyrelation{CR}{t,n} \in \transitiveclosure{\consistencyrelationset{CR}}$ such that:
    \begin{align*}
        \formulaskip &
        \exists \consistencyrelation{CR'}{} \in \setted{\consistencyrelation{CR}{}, \consistencyrelation{CR}{s,2}, \dots, \consistencyrelation{CR}{s,m}} :
        \classtuple{C}{l,\consistencyrelation{CR}{t,1}} \subseteq \classtuple{C}{r,\consistencyrelation{CR'}{}}\\
        & \formulaskip
        \land
        \consistencyrelation{CR}{t,1}, \consistencyrelation{CR^T}{t,1} \not\in \setted{\consistencyrelation{CR}{}, \consistencyrelation{CR}{s,2}, \dots, \consistencyrelation{CR}{s,m}}
    \end{align*}
    In other words, take any concatenation in the transitive closure of $\consistencyrelationset{CR}$ that starts with a relation with a left class tuple that is contained in a right class tuple of a relation already added to the sequence.
    Again, such a sequence must exist because of $\consistencyrelationset{CR}$ being connected and, again, add all $\consistencyrelation{CR}{t,1}, \dots, \consistencyrelation{CR}{t,n}$ to the sequence.
    Per construction, for each $\consistencyrelation{CR'}{}$ in the sequence, there is a non-empty concatenation of relations within the sequence $\consistencyrelation{CR}{} \concat \dots \concat \consistencyrelation{CR'}{}$, because relations were added in a way that such a concatenation always exists. Since all relations in the sequence are contained in $\consistencyrelationset{CR}$, such a concatenation was also contained in $\transitiveclosure{\consistencyrelationset{CR}}$.
    First, we show that the sequence still contains no duplicate elements (1.), i.e., that none of the $\consistencyrelation{CR}{t,i}$ or $\consistencyrelation{CR^T}{t,i}$ is already contained in the sequence $\tupled{\consistencyrelation{CR}{}, \consistencyrelation{CR}{s,1}, \dots, \consistencyrelation{CR}{s,m}}$. 
    Second, we show that both further conditions for the sequence defined in \autoref{lemma:treehassequence} are still fulfilled for the sequence $\tupled{\consistencyrelation{CR}{}, \consistencyrelation{CR}{s,1}, \dots, \consistencyrelation{CR}{s,m}, \consistencyrelation{CR}{t,1}, \dots, \consistencyrelation{CR}{t,n}}$ (2. ,3.).
    % We show that both conditions for the sequence in \autoref{lemma:treehassequence} are still fulfilled for our sequence $\tupled{\consistencyrelation{CR}{}, \consistencyrelation{CR}{s,2}, \dots, \consistencyrelation{CR}{s,m}, \consistencyrelation{CR}{t,2}, \dots, \consistencyrelation{CR}{t,n}}$ by assuming the contradictory:
    \begin{enumerate}
        \item
    Let us assume that the sequence $\tupled{\consistencyrelation{CR}{}, \consistencyrelation{CR}{s,1}, \dots, \consistencyrelation{CR}{s,m}}$ already contained one of the $\consistencyrelation{CR}{t,i}$ or $\consistencyrelation{CR^T}{t,i}$. If $\consistencyrelation{CR}{t,i}$ is contained in the sequence, there is a concatenation $\consistencyrelation{CR}{} \concat \dots \concat \consistencyrelation{CR}{t,i}$ with relations in $\tupled{\consistencyrelation{CR}{}, \consistencyrelation{CR}{s,1}, \dots, \consistencyrelation{CR}{s,m}}$, as well as a concatenation $\consistencyrelation{CR}{} \concat \dots \concat \consistencyrelation{CR}{t,1} \concat \dots \concat \consistencyrelation{CR}{t,i}$.
    Since $\consistencyrelation{CR}{t,1} \not\in \setted{\consistencyrelation{CR}{}, \consistencyrelation{CR}{s,2}, \dots, \consistencyrelation{CR}{s,m}}$ by construction, these two concatenations relate the same class tuples, i.e., they contradict the definition of a consistency relation tree.
    If $\consistencyrelation{CR^T}{t,i}$ was contained in the sequence $\tupled{\consistencyrelation{CR}{}, \consistencyrelation{CR}{s,2} \concat \dots \concat \consistencyrelation{CR}{s,m}}$, there is a concatenation $\consistencyrelation{CR}{} \concat \dots \concat \consistencyrelation{CR}{w} \concat \consistencyrelation{CR^T}{t,i}$ with relations in $\tupled{\consistencyrelation{CR}{}, \consistencyrelation{CR}{s,1}, \dots, \consistencyrelation{CR}{s,m}}$ and, like before, the concatenation $\consistencyrelation{CR}{} \concat \dots \concat \consistencyrelation{CR}{t,1}, \dots, \consistencyrelation{CR}{t,i}$.
    Due to $\classtuple{C}{r,\consistencyrelation{CR}{w}} \cap \classtuple{C}{l,\consistencyrelation{CR}{t,i}} \neq \emptyset$ and  $\consistencyrelation{CR^T}{t,1} \not\in \setted{\consistencyrelation{CR}{}, \consistencyrelation{CR}{s,2}, \dots, \consistencyrelation{CR}{s,m}}$ by construction, the two concatenations $\consistencyrelation{CR}{} \concat \dots \concat \consistencyrelation{CR}{w}$ and $\consistencyrelation{CR}{} \concat \dots \concat \consistencyrelation{CR}{t,1} \concat \dots \concat \consistencyrelation{CR}{t,i}$ have an overlap in both their left and right class tuples, i.e., they contradict the definition of a consistency relation tree.
    In consequence, the sequence $\tupled{\consistencyrelation{CR}{}, \consistencyrelation{CR}{s,1}, \dots, \consistencyrelation{CR}{s,m}}$ cannot have contained any $\consistencyrelation{CR}{t,i}$ or $\consistencyrelation{CR^T}{t,i}$ before.
        \item 
    Let us assume there were any $\consistencyrelation{CR'}{u}$ and $\consistencyrelation{CR'}{v}$ in the sequence $\tupled{\consistencyrelation{CR}{}, \consistencyrelation{CR}{s,1}, \dots, \consistencyrelation{CR}{s,m}, \consistencyrelation{CR}{t,1}, \dots, \consistencyrelation{CR}{t,n}}$ such that $\classtuple{C}{r,\consistencyrelation{CR'}{u}} \cap \classtuple{C}{r,\consistencyrelation{CR'}{v}} \neq \emptyset$.
    As discussed before, for each of these relations exists a concatenation of relations in the sequence $\consistencyrelation{CR}{} \concat \dots \concat \consistencyrelation{CR'}{u}$ and $\consistencyrelation{CR}{} \concat \dots \concat \consistencyrelation{CR'}{v}$, which is contained in $\transitiveclosure{\consistencyrelationset{CR}}$.
    This contradicts the definition of a consistency relation tree, so there cannot be two such relations with overlapping classes in the right class tuple.
        \item
    Let us assume there were any $\consistencyrelation{CR'}{u}$ and $\consistencyrelation{CR'}{v}\; (u < v)$ in the sequence $\tupled{\consistencyrelation{CR}{}, \consistencyrelation{CR}{s,1}, \dots, \consistencyrelation{CR}{s,m}, \consistencyrelation{CR}{t,1}, \dots, \consistencyrelation{CR}{t,n}}$ such that $\classtuple{C}{l,\consistencyrelation{CR'}{u}} \cap \classtuple{C}{r,\consistencyrelation{CR'}{v}} \neq \emptyset$.
    Again per construction, there must be a non-empty concatenation $\consistencyrelation{CR}{} \concat \dots \concat \consistencyrelation{CR'}{w} \concat \consistencyrelation{CR'}{u}$ with $w < u$. Since $\classtuple{C}{l,\consistencyrelation{CR'}{u}} \subseteq \classtuple{C}{r,\consistencyrelation{CR'}{w}}$ per definition, it holds that
    $\classtuple{C}{r,\consistencyrelation{CR'}{w}} \cap \classtuple{C}{r,\consistencyrelation{CR'}{v}} \neq \emptyset$.
    In other words, the relation $\consistencyrelation{CR'}{v}$ introduces a cycle in the relations.
    We have already shown in (2.) that this contradicts the definition of a consistency relation tree.
    \end{enumerate}
    The previous strategy for adding relations to the sequence can be continued inductively by adding relations of the transitive closure of $\consistencyrelationset{CR}$ if their relations were not already added to the sequence.
    This process can be continued until finally all relations in $\consistencyrelationset{CR}$ are added to the sequence.
    Inductively applying the same arguments as before, the final sequence still fulfills all requirements for the sequence in \autoref{lemma:treehassequence}.
    % From the relations in $\consistencyrelationset{CR} \setminus \setted{\consistencyrelation{CR}, \consistencyrelation{CR^T}{}}$, we take those $\consistencyrelation{CR'}{}$ with $\classtuple{C}{l,\consistencyrelation{CR'}{}} \subseteq \classtuple{C}{r,\consistencyrelation{CR}{}}$ and add them to the sequence in an arbitrary order.
    % We recursively proceed with this procedure in a breadth-first fashion for all those added $\consistencyrelation{CR'}{}$.
    % Due to $\consistencyrelationset{CR}$ being connected by definition, this procedure finally adds all $\consistencyrelation{CR}{i}$ or $\consistencyrelation{CR^T}{i}$ to the sequence.
    % By appending always a relation to the sequence whose left class tuple is a subset of the right class tuple of an already added element, for $\consistencyrelation{CR'}{1}$ and all $\consistencyrelation{CR'}{i}$ in the sequence there is always a concatenation of a sub-sequence $\consistencyrelation{CR'}{1} \concat \dots \concat \consistencyrelation{CR'}{i}$ with non-empty left and right class tuples.
    % We show that both conditions for the sequence in \autoref{lemma:treehassequence} are fulfilled by assuming the contradictory:
    % \begin{enumerate}
    %     \item 
    % Let us assume there were any $\consistencyrelation{CR'}{s}$ and $\consistencyrelation{CR'}{t}$ in the sequence, such that $\classtuple{C}{r,\consistencyrelation{CR'}{s}} \cap \classtuple{C}{r,\consistencyrelation{CR'}{t}} \neq \emptyset$.
    % As discussed before, for both these relations there is a concatenation with non-empty left and right class tuples $\consistencyrelation{CR''}{s} = \consistencyrelation{CR'}{1} \concat \dots \concat \consistencyrelation{CR'}{s}$ and $\consistencyrelation{CR''}{t} = \consistencyrelation{CR'}{1} \concat \dots \concat \consistencyrelation{CR'}{t}$
    % such that $\classtuple{C}{l,\consistencyrelation{CR''}{s}} \cap \classtuple{C}{l,\consistencyrelation{CR''}{t}} \neq \emptyset$ and $\classtuple{C}{r,\consistencyrelation{CR''}{s}} \cap \classtuple{C}{r,\consistencyrelation{CR''}{t}} \neq \emptyset$.
    % This contradicts the definition of a consistency relation tree.
    %     \item
    % Let us assume there were any $\consistencyrelation{CR'}{s}$ and $\consistencyrelation{CR'}{t} (s < t)$ such that $\classtuple{C}{l,\consistencyrelation{CR'}{s}} \cap \classtuple{C}{r,\consistencyrelation{CR'}{t}} \neq \emptyset$.
    % Per construction, there must be a sub-sequence $\consistencyrelation{CR'}{1} \concat \dots \consistencyrelation{CR'}{u} \concat \consistencyrelation{CR'}{s}$ with $u < s$ and $\classtuple{C}{l,\consistencyrelation{CR'}{s}} \subseteq \classtuple{C}{r,\consistencyrelation{CR'}{u}}$.
    % In consequence, $\classtuple{C}{r,\consistencyrelation{CR'}{u}} \cap \classtuple{C}{r,\consistencyrelation{CR'}{s}} \neq \emptyset$.
    % We have already shown in the first case that this contradicts the definition of a consistency relation tree.
    % In other words, the relation $\consistencyrelation{CR'}{t}$ introduces a cycle in the relations.
    % \end{enumerate}
    
    We proceed with the reverse direction, i.e., given that a sequence according to the requirements in \autoref{lemma:treehassequence} exists for all $\consistencyrelation{CR}{} \in \consistencyrelationset{CR}$, we show that the set of consistency relations fulfills the definition of a consistency relation tree.
    Let us assume that the tree definition was not fulfilled, i.e., that there were two consistency relations $\consistencyrelation{CR}{s} = \consistencyrelation{CR}{s,1} \concat \dots \concat \consistencyrelation{CR}{s,m} \in \transitiveclosure{\consistencyrelationset{CR}}$ and $\consistencyrelation{CR}{t} = \consistencyrelation{CR}{t,1} \concat \dots \concat \consistencyrelation{CR}{t,n} \in \transitiveclosure{\consistencyrelationset{CR}}$ such that $\classtuple{C}{l,\consistencyrelation{CR}{s}} \cap \classtuple{C}{l,\consistencyrelation{CR}{t}} \neq \emptyset$ and $\classtuple{C}{r,\consistencyrelation{CR}{s}} \cap \classtuple{C}{r,\consistencyrelation{CR}{t}} \neq \emptyset$.
    Without loss of generality, we assume that $\consistencyrelation{CR}{s,m} \neq \consistencyrelation{CR}{t,n}$, because otherwise we could instead consider the sequence without those last relations and still fulfill the defined requirements.
    Any sequence according to \autoref{lemma:treehassequence} containing both $\consistencyrelation{CR}{s,m}$ and $\consistencyrelation{CR}{t,n}$ would contradict the assumption, because $\classtuple{C}{r,\consistencyrelation{CR}{s,m}} \cap \classtuple{C}{r,\consistencyrelation{CR}{t,n}} \neq \emptyset$ in contradiction to the assumptions regarding the sequence.
    Thus, the sequence has to contain either $\consistencyrelation{CR^T}{s,m}$ or $\consistencyrelation{CR^T}{t,n}$.
    Let us assume that the sequence contains $\consistencyrelation{CR^T}{s,m}$.
    Then the sequence cannot contain $\consistencyrelation{CR}{s,m-1}$, because $\classtuple{C}{r,\consistencyrelation{CR^T}{s,m}} \cap \classtuple{C}{r,\consistencyrelation{CR}{s,m-1}} \neq \emptyset$, which, again, would contradict the assumptions regarding the sequence.
    This argument can be inductively applied to all $\consistencyrelation{CR}{s,i}$, such that the sequence has to contain all $\consistencyrelation{CR^T}{s,i}$.
    Since the sequence contains $\consistencyrelation{CR^T}{s,1}$, it must contain $\consistencyrelation{CR}{t,1}$, because $\classtuple{C}{r,\consistencyrelation{CR^T}{s,1}} \cap \classtuple{C}{r,\consistencyrelation{CR^T}{t,1}} \neq \emptyset$.
    In consequence of $\consistencyrelation{CR}{t,1}$ being contained in the sequence, all $\consistencyrelation{CR}{t,i}$ have to be contained as well, due to the same reasons as before.
    So we have these conditions, which introduce a cycle in the overlaps of the class tuples of the relations within the sequence:
    \begin{align*}
        \formulaskip &
        \classtuple{C}{l,\consistencyrelation{CR^T}{s,i-1}} \cap \classtuple{C}{r,\consistencyrelation{CR^T}{s,i}} \neq \emptyset %\\ 
        %&
        \land
        \classtuple{C}{l,\consistencyrelation{CR}{t,1}} \cap \classtuple{C}{r,\consistencyrelation{CR^T}{s,1}} \neq \emptyset\\
        & 
        \land 
        \classtuple{C}{l,\consistencyrelation{CR}{t,i}} \cap \classtuple{C}{r,\consistencyrelation{CR}{t,i-1}} \neq \emptyset %\\
        %&
        \land
        \classtuple{C}{l,\consistencyrelation{CR^T}{s,m}} \cap \classtuple{C}{r,\consistencyrelation{CR}{t,n}} \neq \emptyset
    \end{align*}
    %We argued why all these relations have to be contained in the sequence.
    Because of that cycle in the overlap of class tuples, there is no order of these relations $\consistencyrelation{CR''}{1}, \dots, \consistencyrelation{CR''}{m+n}$ such that for all of them it holds that $\classtuple{C}{l,\consistencyrelation{CR''}{u}} \cap \classtuple{C}{r,\consistencyrelation{CR''}{v}} \neq \emptyset\; (u < v)$, which contradicts the assumptions regarding the sequence in \autoref{lemma:treehassequence}.
    The analog argument holds when we assume that the sequence contains $\consistencyrelation{CR^T}{t,n}$ instead of $\consistencyrelation{CR^T}{s,m}$.
    In consequence, there cannot be two such concatenations $\consistencyrelation{CR}{s}$ and $\consistencyrelation{CR}{t}$ without breaking the assumptions for the sequence in \autoref{lemma:treehassequence}.
    % Take any sequence $\tupled{\consistencyrelation{CR'}{1}, \dots, \consistencyrelation{CR'}{k}}$ with $\consistencyrelation{CR'}{1} = \consistencyrelation{CR}{s,1}$, which necessarily exists per assumption.
    % Then $\tupled{\consistencyrelation{CR}{s,1}, \dots, \consistencyrelation{CR}{s,m}}$ and $\tupled{\consistencyrelation{CR}{t,1}, \dots, \consistencyrelation{CR}{t,n}}$ are contained in $\tupled{\consistencyrelation{CR'}{1}, \dots, \consistencyrelation{CR'}{k}}$, i.e.
    % \begin{align*}
    %     \formulaskip &
    %     \forall i \in \setted{1,\dots,m-1} : 
    %     \exists v, w \in \setted{1,\dots,k} \mid v < w : \\
    %     & \formulaskip
    %     \consistencyrelation{CR}{s,i} = \consistencyrelation{CR'}{v} \land \consistencyrelation{CR}{s,i+1} = \consistencyrelation{CR'}{w}
    % \end{align*}
    % and analogously for $\tupled{\consistencyrelation{CR}{t,1}, \dots, \consistencyrelation{CR}{t,n}}$.
    % If for any $\consistencyrelation{CR}{s,i}$ there was a $\consistencyrelation{CR'}{u} \in \tupled{\consistencyrelation{CR'}{1}, \dots, \consistencyrelation{CR'}{k}}$ with $\consistencyrelation{CR}{s,i} = \consistencyrelation{CR'^T}{u}$, then 
\end{proof}

% \todoHeiko{Make this proof more precise}
% \begin{proof}
%     Let us assume the contrary, i.e. for all sequences $\tupled{\consistencyrelation{CR'}{1}, \dots, \consistencyrelation{CR'}{k}}$ according to \autoref{lemma:treehassequence}, it is true that:
%     \begin{align*}
%         \formulaskip &
%         \exists i \in \setted{1, \dots, k-1} : \\
%         & \formulaskip
%         \exists j \in \setted{i+1, \dots, k-1} :  
%         \classtuple{C}{l,\consistencyrelation{CR'}{i}} \cap 
%         \classtuple{C}{r,\consistencyrelation{CR'}{j}} \neq \emptyset
%     \end{align*}
%     Now select any such sequence $\tupled{\consistencyrelation{CR'}{1}, \dots, \consistencyrelation{CR'}{k}}$. We partition the possible sequences into two disjoint subsets and consider them independently, so this sequence falls into one of the following partitions.
%     \begin{enumerate}
%     \item Consider the following subset of sequences:
%     \begin{align*}
%         \formulaskip &
%         \exists i \in \setted{1, \dots, k-1} : \\
%         & \formulaskip
%         \forall j \in \setted{1, \dots, i-1} :  
%         \classtuple{C}{l,\consistencyrelation{CR'}{i}} \not\subseteq 
%         \classtuple{C}{r,\consistencyrelation{CR'}{j}}
%     \end{align*}
%     \todoHeiko{This is not correct. there must not be such a relation.}
%     This subset includes sequences in which there is a relation whose classes of the left condition are not a subset of the classes of any of the classes of the right condition of a previous relations in that sequence.
%     Due to $\consistencyrelationset{CR}$ being connected, for each relation $\consistencyrelation{CR'}{}$ there must be a relation $\consistencyrelation{CR''}{} \in \consistencyrelationset{CR}$, such that $\classtuple{C}{l,\consistencyrelation{CR''}{}} \subseteq \classtuple{C}{r,\consistencyrelation{CR'}{}}$. Per construction, the  sequence $\tupled{\consistencyrelation{CR'}{1}, \dots, \consistencyrelation{CR'}{k}}$ either contains such a $\consistencyrelation{CR''}{}$, or $\consistencyrelation{CR''^T}{}$.
%     If the sequence contains $\consistencyrelation{CR''}{}$, then the assumption is false by construction.
%     If the sequence contains $\consistencyrelation{CR''^T}{}$ and there is no further $\consistencyrelation{CR'''}{}$ with $\classtuple{C}{l,\consistencyrelation{CR'''}{}} \subseteq \classtuple{C}{r,\consistencyrelation{CR'}{}}$, then the assumption is also false by construction.
%     If there is such another relation, then the same argumentation applies, which inductively means that the assumption is always false.
    
%     \item Consider the following, complementary subset of sequences:
%     \begin{align*}
%         \formulaskip &
%         \forall i \in \setted{1, \dots, k-1} : \\
%         & \formulaskip
%         \exists j \in \setted{1, \dots, i-1} :  
%         \classtuple{C}{l,\consistencyrelation{CR'}{i}} \subseteq 
%         \classtuple{C}{r,\consistencyrelation{CR'}{j}}
%     \end{align*}
%     %Now select any such sequence $\tupled{\consistencyrelation{CR'}{1}, \dots, \consistencyrelation{CR'}{k}}$ in which a later relation in the relation can always be concatenated to a previous relation in the sequence:
%     If the assumption held, then there is a sequence of consistency relations $\tupled{\consistencyrelation{CR'}{1}, \dots, \consistencyrelation{CR'}{s}}$, with $\classtuple{C}{l,\consistencyrelation{CR'}{i+1}} \subseteq \classtuple{C}{r,\consistencyrelation{CR'}{i}}, i \in \setted{1, \dots, s-1}$ and
%     $\classtuple{C}{r,\consistencyrelation{CR}{s}} \cap \classtuple{C}{l, \consistencyrelation{CR}{1}} \neq \emptyset$.
%     Thus, there is the concatenation $\consistencyrelation{CR^\concat}{} = \consistencyrelation{CR'}{1} \concat \dots \concat \consistencyrelation{CR'}{s-1}$ with
%     \begin{align*}
%         \formulaskip
%         \classtuple{C}{l,\consistencyrelation{CR^\concat}{}} = \classtuple{C}{l,\consistencyrelation{CR'}{1}} \land \classtuple{C}{r,\consistencyrelation{CR^\concat}{}} \subseteq  \classtuple{C}{r,\consistencyrelation{CR'}{s-1}}
%      \end{align*}
%     per \autoref{def:relationconcatenation} for concatenation.
%     There is also the consistency relation $\consistencyrelation{CR'^T}{s} \in \consistencyrelationset{CR}$ with
%     \begin{align*}
%         \formulaskip
%         \classtuple{C}{l,\consistencyrelation{CR'^T}{s}} = \classtuple{C}{r,\consistencyrelation{CR'}{s-1}} \land  \classtuple{C}{r,\consistencyrelation{CR'^T}{s}} = \classtuple{C}{l,\consistencyrelation{CR'}{1}}
%     \end{align*}
%     such that:
%     \begin{align*}
%     \formulaskip
%         \classtuple{C}{l,\consistencyrelation{CR^\concat}{}} \cap \classtuple{C}{r,\consistencyrelation{CR'^T}{s}} \neq \emptyset \land \classtuple{C}{l,\consistencyrelation{CR'^T}{s}} \cap \classtuple{C}{r,\consistencyrelation{CR^\concat}{}} \neq \emptyset.
%     \end{align*}
%     In consequence, $\consistencyrelation{CR^\concat}{}$ and $\consistencyrelation{CR'}{}$ break the definition of a consistency relation tree.
%     \end{enumerate}
%     In combination, we have disproved the contrary statement, so we know that the statement in \autoref{lemma:treehassequence} is true.
%  \end{proof}

The previous lemma shows that the definition of consistency relation trees based on unique concatenations of the same class tuples is equivalent the possibility to find sequences of the relations that do not contain cycles in the related class tuples. %for each of the relations a sequence starting with that relation and containing all other relations as well, such that there are no cycles in the classes related by these relations.
The definition is supposed to be easier to check in practice.
However, we can now show that a consistency relation tree is always compatible with a constructive proof that requires the equivalent definition from \autoref{lemma:treehassequence}.

%With the previous lemma, we can now show that a consistency relation tree is always compatible.

\begin{theorem} \label{theorem:treecompatibility}
    Let $\consistencyrelationset{CR}$ be a consistency relation tree, then $\consistencyrelationset{CR}$ is compatible.
\end{theorem}

\begin{figure}
    \centering
    \newcommand{\hdistance}{16em}
\newcommand{\objectwidth}{8em}

\begin{tikzpicture}[
    consistency process/.style={consistency relation, dashed}
]

% Person
\umlobjectvarwidth{person}{}{Person\sameheight}{
firstname = "Alice"\\
lastname = "Do"
}{\objectwidth}

% Employee
\umlobjectvarwidth[,above right=4em and \hdistance of person.center, anchor=south]{employee}{}{Employee\sameheight}{
name = "Alice Do"
}{\objectwidth}

\umlobjectvarwidth[,right=\hdistance of person.south, anchor=south]{resident}{}{Resident\sameheight}{
name = "Alice Do"
}{\objectwidth}


% CONSISTENCY RELATIONS
\draw[consistency relation] ([yshift=-1em]resident.west-|person.east) -- node[pos=0, above right] {$p$} node[pos=0.5, below] {$\consistencyrelation{CR}{1}$} node[pos=1, above left] {$r$} ([yshift=-1em]resident.west);
\draw[consistency relation] (resident.north) -- node[pos=0, above left] {$r$} node[right, align=left] {$\consistencyrelation{CR}{2}$} node[pos=1, below left] {$e$} (employee.south);

\node[consistency related element, below left=1em and 2em of person.south west, anchor=north west] {
$\begin{aligned}
    \consistencyrelation{CR}{1} =\; & \setted{\tupled{p,r} \mid r.name = p.firstname + "\textnormal{\textvisiblespace}" + p.lastname}\\[0.3em]
    \consistencyrelation{CR}{2} =\; & \setted{\tupled{r,e} \mid r.name = e.name}
\end{aligned}$
};

% CONSISTENCY PROCESS
\draw[consistency process] ([xshift=-2.5em]person.west) -- node[above] {\textbf{(1)}} (person.west);
\draw[consistency process] ([yshift=1em]resident.west-|person.east) -- node[above] {\textbf{(2)}} ([yshift=1em]resident.west);
\draw[consistency process] ([xshift=-2em]resident.north) -- node[left] {\textbf{(3)}} ([xshift=-2em]employee.south);

\end{tikzpicture}
    %\includegraphics[width=\columnwidth]{figures/tree_construction_example.png}
    \caption{An example for constructing a model with the condition element of $\consistencyrelation{CR}{1}$ containing the person named "Alice Do" for a consistency relation tree according to the consistency relations in \autoref{fig:correctness:formal:tree_example}.}
    \label{fig:correctness:formal:tree_construction_example}
\end{figure}

\begin{proof}
    We prove the statement by constructing a set of models for each condition element in the left condition of each consistency relation that contains the condition element and is consistent, i.e., that fulfills the compatibility definition.
    The basic idea is that because $\consistencyrelationset{CR}$ is a consistency relation tree, we can simply add necessary elements to get a model set that is consistent to all consistency relations, by %iterating through the tree in terms of 
    following an order of relations according to \autoref{lemma:treehassequence}.
    Thus, we explain an induction for constructing such a model set, which is also exemplified for a simple scenario in \autoref{fig:correctness:formal:tree_construction_example}, based on the relations in the consistency relation tree in \autoref{fig:correctness:formal:tree_example}.
    %First, we assume that $\consistencyrelationset{CR}$ is connected. Otherwise the following construction can be applied to the independent partitions of $\consistencyrelationset{CR}$, as their combination is compatible if each of them is compatible according to \autoref{lemma:independencecompatibility}.
    
    \paragraph{Base case:}
    Take any $\consistencyrelation{CR}{} \in \consistencyrelationset{CR}$ and any of its left side condition elements $\conditionelement{c}{l} = \tupled{\object{o}{l,1}, \dots, \object{o}{l,m}} \in \condition{c}{l, \consistencyrelation{CR}{}}$.
    %First, we construct a model set that contains $\conditionelement{c}{l}$ and is consistent to only $\consistencyrelation{CR}{}$.
    %To achieve that, 
    Select any $\conditionelement{c}{r} = \tupled{\object{o}{r,1}, \dots, \object{o}{r,n}} \in \condition{c}{r, \consistencyrelation{CR}{}}$, such that $\conditionelement{c}{l}$ and $\conditionelement{c}{r}$ constitute a consistency relation pair $\tupled{\conditionelement{c}{l}, \conditionelement{c}{r,}} \in \consistencyrelation{CR}{}$.
    %Now select any $\modelset{m} = \object{o'}{1}, \dots, \object{o'}{m+n}$, such that $\forall i \in \setted{1, \dots, n}: \object{o}{l,i} \subseteq \object{o'}{i}$ and $\forall i \in \setted{1, \dots, m}: \object{o}{r,i} \subseteq \object{o'}{n+i}$.
    Now construct the model set $\modelset{m}$ that contains only $\object{o}{l,1}, \dots, \object{o}{l,m}$ and $\object{o}{r,1}, \dots, \object{o}{r,n}$. % contains $\conditionelement{c}{l}$ and is consistent to $\consistencyrelation{CR}$ by construction.
    In consequence, we have a minimal model set $\modelset{m}$, such that $\modelset{m} \containsmath \conditionelement{c}{l}$ and $\modelset{m} \consistenttomath \consistencyrelation{CR}{}$.
    Additionally, $\modelset{m}$ is consistent to $\consistencyrelation{CR^T}{}$ due to symmetry of $\consistencyrelation{CR}{}$ and $\consistencyrelation{CR^T}{}$: It is $\conditionelement{c}{r} \in \condition{c}{l,\consistencyrelation{CR^T}{}}$ and $\tupled{\conditionelement{c}{r}, \conditionelement{c}{l}} \in \consistencyrelation{CR^T}{}$ and no other condition element of $\condition{c}{l,\consistencyrelation{CR^T}{}}$ is contained in $\modelset{m}$ by construction, thus $\modelset{m}$ is consistent to $\consistencyrelation{CR^T}{}$.
    In consequence, we know that for all $\consistencyrelation{CR}{} \in \consistencyrelationset{CR}$, $\setted{\consistencyrelation{CR}{}, \consistencyrelation{CR^T}{}}$ is compatible. 
    Considering the example in $\autoref{fig:correctness:formal:tree_construction_example}$, for the selection of any person as a condition element in $\condition{c}{l,\consistencyrelation{CR}{1}}$ (1), we select a resident in $\condition{c}{r,\consistencyrelation{CR}{1}}$ with the same name (2), such that the elements are consistent to $\consistencyrelation{CR}{1}$.
    
    \paragraph{Induction assumption:} 
    According to \autoref{lemma:treehassequence}, there is a sequence $\tupled{\consistencyrelation{CR}{1}, \dots, \consistencyrelation{CR}{k}}$ of the relations in $\consistencyrelationset{CR}$ with $\consistencyrelation{CR}{1} = \consistencyrelation{CR}{}$, such that:
    \begin{align*}
        \formulaskip &
        \forall s \in \setted{1, \dots, k-1} : \forall t \in \setted{i+1, \dots, k} : \\
        & \formulaskip
        \classtuple{C}{r,\consistencyrelation{CR'}{s}} \cap \classtuple{C}{r,\consistencyrelation{CR'}{t}} = \emptyset 
        \land
        \classtuple{C}{l,\consistencyrelation{CR'}{s}} \cap 
        \classtuple{C}{r,\consistencyrelation{CR'}{t}} = \emptyset
    \end{align*}
    Considering the example in \autoref{fig:correctness:formal:tree_construction_example}, such a sequence would be $\tupled{\consistencyrelation{CR}{1}, \consistencyrelation{CR}{2}}$, because the elements in the right condition of $\consistencyrelation{CR}{2}$ are not represented in the left condition of $\consistencyrelation{CR}{1}$.
    If, in general, we know that $\setted{\consistencyrelation{CR}{1}, \consistencyrelation{CR^T}{1} \dots, \consistencyrelation{CR}{i}, \consistencyrelation{CR^T}{i}}\; (i < k)$ is compatible, for every $\conditionelement{c}{l} \in \condition{C}{l,\consistencyrelation{CR}{}}$, we can find a model set $\modelset{m}$ that contains $\conditionelement{c}{l}$ and is consistent to $\setted{\consistencyrelation{CR}{1}, \consistencyrelation{CR^T}{1}, \dots, \consistencyrelation{CR}{i}, \consistencyrelation{CR^T}{i}}$ by definition.
    We can especially create a minimal model according to our construction for the base case and the following inductive completion.
    
    \paragraph{Induction step:}
    Consider $\consistencyrelation{CR}{i+1}$.
    There is at most one condition element $\conditionelement{c}{l} \in \condition{c}{l, \consistencyrelation{CR}{i+1}}$ with $\modelset{m} \containsmath \conditionelement{c}{l}$.
    If there were at least two condition elements $\conditionelement{c}{l}, \conditionelement{c'}{l} \in \condition{c}{l,\consistencyrelation{CR}{i+1}}$, both contained in $\modelset{m}$, then by construction there is a consistency relation $\consistencyrelation{CR}{s}\; (s < i+1)$ with $\conditionelement{c}{l},\conditionelement{c'}{l} \in \condition{c}{r,\consistencyrelation{CR}{j}}$.
    Let us assume there were two consistency relations $\consistencyrelation{CR}{s}, \consistencyrelation{CR}{t}$, each containing one of the condition elements in the right condition, then there would be non-empty concatenations $\consistencyrelation{CR}{} \concat \dots \concat \consistencyrelation{CR}{s}$ and $\consistencyrelation{CR'}{} \concat \dots \concat \consistencyrelation{CR}{t}$ with $\classtuple{C}{l,\consistencyrelation{CR}{}} \cap \classtuple{C}{l,\consistencyrelation{CR'}{}} \neq \emptyset$, because we started the construction with elements from the left condition of $\consistencyrelation{CR}{}$, so every element is contained because of a relation to those elements, and with $\classtuple{C}{r,\consistencyrelation{CR}{s}} \cap \classtuple{C}{r,\consistencyrelation{CR}{t}} \neq \emptyset$, because both condition elements $\conditionelement{c}{l}$ and $\conditionelement{c'}{l}$ instantiate the same classes, as they are both contained in $\condition{c}{l,\consistencyrelation{CR}{i+1}}$.
    This would violate \autoref{def:relationtree} for a consistency relation tree, thus there is only one such consistency relation $\consistencyrelation{CR}{s}$.
    Consequently, there must be two condition elements $\conditionelement{c}{ll}, \conditionelement{c'}{ll} \in \condition{c}{l,\consistencyrelation{CR}{s}}$ with $\tupled{\conditionelement{c}{ll},\conditionelement{c}{l}}, \tupled{\conditionelement{c'}{ll},\conditionelement{c'}{l}} \in \consistencyrelation{CR}{s}$, because per construction $\modelset{m}$ was consistent to $\consistencyrelation{CR}{s}$ and so there must be a witness structure with a unique mapping between condition elements contained in $\modelset{m}$.
    The above argument can be applied inductively until we finally find that there must be two condition elements $\conditionelement{c}{lll},\conditionelement{c'}{lll} \in \condition{c}{l,\consistencyrelation{CR}{}}$, which are contained in $\modelset{m}$.
    This is not true by construction, as we started with only one element from $\condition{c}{l,\consistencyrelation{CR}{}}$, so there is only one such condition element $\conditionelement{c}{l} \in \condition{c}{l,\consistencyrelation{CR}{i+1}}$ with $\modelset{m} \containsmath \conditionelement{c}{l}$.
    
    For this condition element $\conditionelement{c}{l} \in \condition{c}{l,\consistencyrelation{CR}{i+1}}$, select an arbitrary $\conditionelement{c}{r} = \tupled{\object{o}{1}, \dots, \object{o}{s}} \in \condition{c}{r, \consistencyrelation{CR}{i+1}}$, such that $\tupled{\conditionelement{c}{l}, \conditionelement{c}{r}} \in \consistencyrelation{CR}{i+1}$.
    %Now select any objects $\object{o'}{1}, \dots, \object{o'}{s}$, such that $\forall i \in \setted{1, \dots, s}: \object{o}{i} \subseteq \object{o'}{i}$ and create a model set $\modelset{m'}$ by adding the objects $\object{o'}{1}, \dots, \object{o'}{s}$ to $\modelset{m}$.
    Now create a model set $\modelset{m'}$ by adding the objects $\object{o}{1}, \dots, \object{o}{s}$ to $\modelset{m}$.
    Since $\conditionelement{c}{l}$ is the only of the left condition elements of $\consistencyrelation{CR}{i+1}$ that $\modelset{m}$ contains, model set $\modelset{m'}$ is consistent to $\consistencyrelation{CR}{i+1}$ per construction.
    $\modelset{m'}$ is also consistent to $\consistencyrelation{CR^T}{i+1}$, because due to the symmetry of $\consistencyrelation{CR}{i+1}$ and $\consistencyrelation{CR^T}{i+1}$, it is $\conditionelement{c}{r} \in \condition{c}{l,\consistencyrelation{CR^T}{i+1}}$ and due to $\tupled{\conditionelement{c}{r}, \conditionelement{c}{l}} \in \consistencyrelation{CR^T}{i+1}$, a consistent corresponding element exists in $\modelset{m'}$. 
    Furthermore, there cannot be any other $\conditionelement{c'}{} \in \condition{c}{l,\consistencyrelation{CR^T}{i+1}}$ with $\modelset{m'} \containsmath \conditionelement{c'}{}$, because otherwise there would have been another consistency relation $\consistencyrelation{CR'}{}$ that required the creation of $\conditionelement{c'}{}$, which means that there are two concatenations of consistency relations $\consistencyrelation{CR}{} \concat \dots \concat \consistencyrelation{CR'}{}$ and $\consistencyrelation{CR}{} \concat \dots \concat \consistencyrelation{CR}{i+1}$ that both relate instances of the same classes, which contradicts \autoref{def:relationtree} for a consistency relation tree.
    
    Additionally, due to \autoref{lemma:treehassequence}, for all  $\consistencyrelation{CR}{s}\; (s < i+1)$, we know that $\classtuple{C}{l,\consistencyrelation{CR}{s}} \cap \classtuple{C}{r,\consistencyrelation{CR}{i+1}} = \emptyset$. 
    Since the newly added elements $\conditionelement{c}{r}$ are part of $\condition{c}{r,\consistencyrelation{CR}{i+1}}$, these elements cannot match the left conditions of any of the consistency relations $\consistencyrelation{CR}{s}\; (s < i+1)$.
    So $\modelset{m'}$ is still consistent to all $\consistencyrelation{CR}{s}\; (s < i+1)$.
    Finally, due to \autoref{lemma:treehassequence}, for all  $\consistencyrelation{CR}{s}\; (s < i+1)$, we know that $\classtuple{C}{r,\consistencyrelation{CR}{s}} \cap \classtuple{C}{r,\consistencyrelation{CR}{i+1}} = \emptyset$.
    Again, since the newly added elements $\conditionelement{c}{r}$ are part of $\condition{c}{r,\consistencyrelation{CR}{i+1}}$, these elements cannot match the left conditions of any of the consistency relations $\consistencyrelation{CR^T}{s}\; (s < i+1)$.
    So $\modelset{m'}$ is still consistent to all $\consistencyrelation{CR^T}{s}\; (s < i+1)$.
    %Additionally, $\modelset{m'} \consistenttomath \consistencyrelation{CR^T}{i+1}$, because the only element of $\condition{c}{l,\consistencyrelation{CR^T}{i+1}}$ is $\conditionelement{c}{l}$. 
    %Otherwise another consistency relation would have required such an element to be updated, such there was another sequence of consistency relations next to $\consistencyrelation{CR}{}, \dots, \consistencyrelation{CR}{i+1}$ that created elements of $\classtuple{C}{r,\consistencyrelation{CR}{i+1}}$, which is not possible by \autoref{def:relationtree} for consistency relation trees.
    In consequence, we know that $\modelset{m'} \consistenttomath \setted{\consistencyrelation{CR}{1}, \consistencyrelation{CR^T}{1} \dots, \consistencyrelation{CR}{i+1}, \consistencyrelation{CR^T}{i+1}}$.
    
    Considering the example in \autoref{fig:correctness:formal:tree_construction_example}, we would select $\consistencyrelation{CR}{2}$ and add for the resident, which is in the left condition elements of $\consistencyrelation{CR}{2}$, an appropriate employee to make the model set consistent to $\consistencyrelation{CR}{2}$ (3).
    % If adding those elements could violate consistency to $\consistencyrelation{CR}{1}$, any element considered by $\consistencyrelation{CR}{1}$, in this case a resident, would have to be created, which is not possible, as the relations would not form a tree anymore.
    
    \paragraph{Conclusion}
    Taking the base case for $\consistencyrelation{CR}{}$ and the induction step for $\consistencyrelation{CR}{i+1}$, we have inductively shown that 
    \begin{align*}
        \formulaskip 
        \modelset{m'} \consistenttomath \setted{\consistencyrelation{CR}{1}, \consistencyrelation{CR^T}{1} \dots, \consistencyrelation{CR}{k}, \consistencyrelation{CR^T}{k}} = \consistencyrelationset{CR}
    \end{align*}
    Since the construction is valid for each condition element in every consistency relation in $\consistencyrelationset{CR}$, we know that a consistency relation tree $\consistencyrelationset{CR}$ is compatible.
    
    % Now select an arbitrary $\consistencyrelation{CR'}{} \in \consistencyrelationset{CR} \setminus \setted{\consistencyrelation{CR}{}}$ with $\consistencyrelation{CR}{} \concat \consistencyrelation{CR'}{} \neq \emptyset$.
    % Such a relation must exist, because otherwise $\consistencyrelationset{CR}$ would not be connected.
    % For all condition elements $\conditionelement{c}{l} \in \condition{c}{l, \consistencyrelation{CR'}{}}$ with $\modelset{m} \containsmath \conditionelement{c}{l}$, select an arbitrary  $\conditionelement{c}{r} = \tupled{\object{o}{1}, \dots, \object{o}{s}} \in \condition{c}{r, \consistencyrelation{CR'}{}}$, such that $\tupled{\condition{c}{l, \consistencyrelation{CR'}{}}, \condition{c}{r, \consistencyrelation{CR'}{}}} \in \consistencyrelation{CR'}{}$.
    % Now select any objects $\object{o'}{1}, \dots, \object{o'}{s}$, such that $\forall i \in \setted{1, \dots, s}: \object{o}{i} \subseteq \object{o'}{i}$ and create a model set $\modelset{m'}$ by adding the objects $\object{o'}{1}, \dots, \object{o'}{s}$ to $\modelset{m}$.
    % Per construction, model set $\modelset{m'}$ is consistent to $\consistencyrelation{CR'}{}$.
    % It is also still consistent to $\consistencyrelation{CR}{}$, because if any of the newly added objects $\object{o'}{1}, \dots, \object{o'}{s}$ would violate consistency to $\consistencyrelation{CR}{}$, then there must be an overlap in the classes $\classtuple{C}{r,\consistencyrelation{CR'}{}}$ of $\consistencyrelation{CR'}{}$ and the classes $\classtuple{C}{l,\consistencyrelation{CR}{}}$ of $\consistencyrelation{CR}{}$.
    % Since $\consistencyrelation{CR}{} \concat \consistencyrelation{CR'}{} \neq \emptyset$ by construction, there is also an overlap in the class tuples $\classtuple{C}{l,\consistencyrelation{CR'}{}}$ and $\classtuple{C}{r,\consistencyrelation{CR}{}}$, so that there is a cycle in the classes, violating the definition of a consistency relation tree.
    % Considering the example in \autoref{fig:correctness:formal:tree_construction_example}, we select $\consistencyrelation{CR}{2}$ and add an appropriate person to make the model set consistent to $\consistencyrelation{CR}{2}$ (3).
    % If adding those elements could violate consistency to $\consistencyrelation{CR}{1}$, any element considered by $\consistencyrelation{CR}{1}$, in this case a resident, would have to be created, which is not possible, as the relations would not form a tree anymore.
    
    % Now, having added elements to make the model set $\modelset{m'}$ consistent to $\consistencyrelation{CR}{1}, \dots, \consistencyrelation{CR}{k]}$ inductively that way, select any $\consistencyrelation{CR''}{} \in \consistencyrelationset{CR} \setminus \setted{\consistencyrelation{CR}{1}, \dots, \consistencyrelation{CR}{k}}$ with $\exists i \in \setted{1, \dots, k} : \consistencyrelation{CR}{i} \concat \consistencyrelation{CR''}{} \neq \emptyset$.
    % Like before, this relation has to exist because $\consistencyrelationset{CR}$ is connected.
    % In the same way like for $\consistencyrelation{CR'}{}$, add elements to the model set $\modelset{m'}$ to form a new model set $\modelset{m''}$ that is consistent to $\consistencyrelation{CR''}{}$.
    % If these added elements would lead to $\modelset{m''}$ not being consistent to any $\consistencyrelation{CR}{1}, \dots, \consistencyrelation{CR}{k}$, due to the same argumentation as for $\consistencyrelation{CR'}{}$, there would be a sequence of consistency relations that relates one class with itself, thus $\consistencyrelationset{CR}$ would not fulfill the definition of consistency relation tree.
    % Thus, $\modelset{m''}$ is consistent to $\setted{\consistencyrelation{CR}{1}, \dots, \consistencyrelation{CR}{k}, \consistencyrelation{CR''}{}}$.
    
    % Applying that argument inductively for all consistency relations in $\consistencyrelationset{CR}$, we are able to create a model set $\modelset{m}$ that is consistent to $\consistencyrelationset{CR}$ and that contains the initially selected condition element $\conditionelement{c}{l}$. 
    % Since the argument holds for any such condition element of any of the consistency relations, $\consistencyrelationset{CR}$ fulfills the definition of compatibility.
\end{proof}

%%
%% Summary: Independent trees are compatible
%%
Summarizing, \autoref{theorem:independencecompatibility} and \autoref{theorem:treecompatibility} have shown that consistency relation sets fulfilling a special notion of trees are compatible and that combining compatible independent sets of relations is compatibility-preserving.
In consequence, having a consistency relation set that consists of independent subsets that are consistency relation trees, this set of relations is inherently compatible.
An approach that evaluates whether a given set of consistency relations fulfills \autoref{def:independence} and \autoref{def:relationtree} for independence and trees can be used to prove compatibility of those relations.

%%
%% Transition: Actual sets may generally not be trees
%%
However, consistency relations fulfill such a structure only in specific cases.
In general, like in our motivational example in \autoref{fig:prologue:three_persons_example}, there may be different consistency relations putting the same elements into relation, such that the definition for consistency relation trees is not fulfilled.
In the following, we discuss how to find a consistency relation tree that is equivalent to a given set of consistency relations, such that this equivalence witnesses compatibility.