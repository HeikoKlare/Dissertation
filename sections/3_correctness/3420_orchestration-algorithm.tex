\section{An Algorithm for the Application Function}

%Start with defining an algorithm that realizes the application function. (different options depending on when to return $\bot$, discussed later)
%First option: we assume an oracle that returns the transformation to execute next according to the orchestration function and we stop when consistent models are achieved

\begin{algorithm}
   % \begin{algorithmic}[1]
    \Procedure{\function{FindCorresponding}}{$\consistencyrelation{CR}{}, \conditionelement{c}{l}, \model{m}{2}, \model{m}{\mathvariable{traces}}$}
        \algindentskip
        \State $\mathvariable{tracedElements} \gets \setted{\conditionelement{c}{r} \mid \tupled{\conditionelement{c}{l}, \conditionelement{c}{r}} \in \model{m}{\mathvariable{traces}}}$
        \For{$\conditionelement{c}{r} \in \mathvariable{tracedElements}$} \label{algo:synchronization:find_corresponding_elements:line:explicit}
            \If{$\tupled{\conditionelement{c}{l}, \conditionelement{c}{r}} \in \consistencyrelation{CR}{}$}
                \State \Return{$\conditionelement{c}{r}$}
            \EndIf
        \EndFor
        \algblockskip

        \For{$\conditionelement{c}{r} \in \mathcal{P}(\model{m}{2})$} \label{algo:synchronization:find_corresponding_elements:line:implicit}
            \If{$\tupled{\conditionelement{c}{l}, \conditionelement{c}{r}} \in \consistencyrelation{CR}{}$}
                \State $\model{m}{\mathvariable{traces}} \gets \model{m}{\mathvariable{traces}} \cup \setted{\tupled{\conditionelement{c}{l},\conditionelement{c}{r}}}$
                \State \Return{$\conditionelement{c}{r}$}
            \EndIf 
        \EndFor
        \algblockskip

        \State \Return{$\bot$}
        \algindentskip
    \EndProcedure
\end{algorithmic}
    \begin{algorithmic}[1]
        \Procedure{$\function{Apply}_{\transformationset{T}}$}{$\modeltuple{m}, \changetuple{\metamodeltuple{M}}$}
            \State $\mathvariable{isConsistent}$ $\leftarrow$ $\function{CheckConsistency}(\transformationset{t}, \modeltuple{m}$)
            \If{$\neq \mathvariable{isConsistent}$}
                \State \Return{$\bot$}
            \EndIf
            \State $\mathvariable{executedTransformations} \leftarrow \tupled{}$
            \State $\mathvariable{generatedChanges} \leftarrow \tupled{}$
            \State $\transformation{t}_{next}$ $\leftarrow$ $\function{Orchestrate}_{\transformationset{T}}(\modeltuple{m}, \changetuple{\metamodeltuple{M}}, \mathvariable{executedTransformations}, \mathvariable{generatedChanges})$ \label{algo:orchestration:application:line:startorchestrate}
            \While{$\transformation{t}_{next} \neq \bot$}
                \State $(\modeltuple{m}, \changetuple{\metamodeltuple{M}})$ $\leftarrow$ $\generalizationfunction{\metamodeltuple{M}, \transformation{t}_{next}}(\modeltuple{m}, \changetuple{\metamodeltuple{M}})$ \label{algo:orchestration:application:line:stepcalculation}
                \State $\mathvariable{executedTransformation}.add(\transformation{t}_{next})$
                \State $\mathvariable{generatedChanges}.add(\changetuple{\metamodeltuple{M}})$
                \State $\transformation{t}_{next}$ $\leftarrow$ $\function{Orchestrate}_{\transformationset{T}}(\modeltuple{m}, \changetuple{\metamodeltuple{M}}, \mathvariable{executedTransformations}, \mathvariable{generatedChanges})$
            \EndWhile \label{algo:orchestration:application:line:endorchestrate}
            \State $\tupled{\model{m}{1}, \dots, \model{m}{n}} \leftarrow \modeltuple{m}$
            \State $\tupled{\change{\metamodel{M}{1}}, \dots, \change{\metamodel{M}{n}}} \leftarrow \changetuple{\metamodeltuple{M}}$
            \State $\modeltuple{m}_{res} \leftarrow \tupled{\change{\metamodel{M}{1}}(\model{m}{1}), \dots, \change{\metamodel{M}{n}}(\model{m}{n})}$
            \State $\mathvariable{isConsistent}$ $\leftarrow$ $\function{CheckConsistency}(\transformationset{t}, \modeltuple{m}_{res}$) \label{algo:orchestration:application:line:startconsistencycheck}
            \If{$\neq \mathvariable{isConsistent}$}
                \State \Return{$\bot$}
            \EndIf \label{algo:orchestration:application:line:endconsistencycheck}
            % \For{$\transformation{t} \in \transformationset{T}$} \label{algo:orchestration:application:line:startcheckconsistency}
            %     %\State $(\consistencyrelation{CR}{}, \consistencypreservationrule{\consistencyrelation{CR}{}}) \leftarrow \transformation{t}$
            %     \State $\mathvariable{isConsistent}$ $\leftarrow$ $\function{CheckConsistency}_{\metamodeltuple{M}}(\modeltuple{m}_{res}, \transformation{t}$) %\consistencyrelation{CR}{})$
            %     \If{$\neq \mathvariable{isConsistent}$}
            %         \State \Return{$\bot$}
            %     \EndIf
            % \EndFor \label{algo:orchestration:application:line:endcheckconsistency}
            \State \Return{$\modeltuple{m}_{res}$} \label{algo:orchestration:application:line:returnresult}
        \EndProcedure
    \end{algorithmic}
    \caption[Application function for transformations]{Application function for transformation set.}
    \label{algo:orchestration:application}
\end{algorithm}

\mnote{Algorithm for the application function}
We have yet discussed the orchestration and application functions as purely mathematical functions.
In practice, however, they need to be implemented in terms of algorithms.
In \autoref{algo:orchestration:application}, we propose an algorithm for the application function.
It also encodes the orchestration function, because in contrast to the mathematical definition, an algorithm for the orchestration function will not determine a complete sequence of transformations for given models and changes, but dynamically select the next transformation to execute.
As soon as all transformation delivered by the orchestration are executed, it returns the resulting models if they are consistent or otherwise returns $\bot$.

\mnote{Algorithm dynamically selects next transformation}
The dynamic selection of transformations is realized by an \function{Orchestrate} function and stops as soon as no further transformations to apply are delivered.
The latter may be the case because the models are already consistent or because no further transformations can be applied.
The complete logic of the orchestration function is combined with the application of the delivered sequence in Lines~\ref{algo:orchestration:application:line:startorchestrate}--\ref{algo:orchestration:application:line:endorchestrate}.
Since, in practice, the selection of transformation has to be performed dynamically anyway, an implementation of the orchestration function always needs to apply the transformations.
Thus a separation of the orchestration function into a separate algorithm, which performs the same steps as in Lines~\ref{algo:orchestration:application:line:startorchestrate}--\ref{algo:orchestration:application:line:endorchestrate} leads to a redundancy by applying the transformations both in the separate orchestration algorithm as well as in the given algorithm.

\mnote{Orchestration needs history of changes and transformations}
The function receives the history of executed transformations and generated changes, because if the complete orchestration function was implemented in a separate method, it would also be able to use that information to determine a proper orchestration.
Otherwise, its expressiveness would be restricted with respect to the definition of an orchestration function, because that function makes a global decision for all transformations to execute base on the original input, which is not available for the \function{Orchestrate} function after its first execution anymore.
In a practical implementation of that function, the history may, however, not be considered or truncated, depending on the information necessary for the concrete implemented orchestration strategy.

\mnote{Strategies for selecting next transformation}
%The \function{Orchestrate} function is responsible for selecting the next transformation.
The \function{Orchestrate} function may implement different strategies, which we will later discuss in more detail.
The most simple strategy would be to execute the same order of transformations iteratively, thus always executing that transformation who was not executed for the longest time.
Another reasonable strategy would be to manage a queue of transformation and after executing one transformation to enqueue all transformations that are adjacent to the metamodels of the two models that were modified by the transformation if they are not yet enqueued.
This ensures that those transformation are executed next which can process changes that have just been produced by another transformation.
We will later discuss specific strategies.
Until then, the concrete strategy is not important and any of the exemplified ones can be imagined.

\mnote{Assumed further functions of algorithm}
Next to \function{Orchestrate}, the algorithm uses the external functions $\generalizationfunction{}$ and \function{CheckConsistency}.
The $\generalizationfunction{}$ function is the generalization function, which simply applies the given transformation to the appropriate models of the given tuple.
The \function{CheckConsistency} checks whether the given models are consistent to the set of transformations, according to \autoref{def:consistencytransformation}.
Due to their simplicity, we do not provide an explicit implementation of these two functions.

% Algorithm dynamically selects next transformation.
% It stops as soon as orchestration does provide further transformations.
% Then, either no transformation can be applied anymore, or the models are consistent.
% This moves some of the logic of the orchestration function into the apply function, as orchestrate only selects the next function rather than delivering a sequence in the beginning.
% Thus, Lines~\ref{algo:orchestration:application:line:startorchestrate}--\ref{algo:orchestration:application:line:endorchestrate} implement the orchestration function.
% This is reasonable for two reasons:
% First, this is how a practical algorithm will perform the selection anyway, by dynamically selecting the next transformation.
% Second, this is only a matter of implementation, as we could move the lines to a separate method, which acts as the orchestrate function by determining the sequence by dynamically applying the transformations, let it return the sequence and then let the apply function apply the transformations in that order again.
% %Second, although dynamic selection may sound more expressive than a a priori determination of the complete sequence like defined for the app and orc function in the previous definitions, this is not the case as an orc function according to that definition may be an oracle that, in a practical implementation, determines the sequence for a given input in the same way.
% Since applying the transformation in the orc function to determine a sequence and then simply applying them in the app function again is redundant, we directly implemented that dynamic selection in the app function.

% Apply is defined to be comprehensible, not efficient. For example, the consistency check can be improved by aborting as soon as one transformation to which the models are inconsistent is found.

% Gen is the generalization function, CheckConsistency is generalized consistency checker which just checks for the appropriate models in the tuple.

\todo{transformation set should be parameter, not one application/orchestration function per transformation set}


\subsection{Correctness and Termination}

\mnote{Algorithms implements correct application function}
\autoref{algo:orchestration:application} is constructed to implement an application function according to \autoref{def:applicationfunction}.
It is designed to be correct, i.e., it only returns models when they are consistent.
We show that the algorithm fulfills these properties in the following theorem.

\begin{theorem}[Apply Function Correctness]
    The \function{Apply} function in \autoref{algo:orchestration:application} fulfills the functional behavior of an application function as defined in \autoref{def:applicationfunction} and is correct according to \autoref{def:applicationfunctioncorrectness}
\end{theorem}
\begin{proof}
    The \function{Apply} function fulfills the input and output requirements of an application function according to \autoref{def:applicationfunction}.
    It only returns a model tuple in Line \ref{algo:orchestration:application:line:returnresult}, which is achieved by applying the changes delivered by the sequence of transformations delivered by the orchestration function realized as a repeated call of the \function{Orchestrate} function in Lines~\ref{algo:orchestration:application:line:startorchestrate}--\ref{algo:orchestration:application:line:endorchestrate}.
    Thus, \function{Apply} fulfills the definition of an application function.

    Correctness of an application function according to \autoref{def:applicationfunctioncorrectness} requires the output models, if not returning $\bot$, to be consistent to the consistency relations of all transformations, as long as the input models were consistent.
    The algorithm only returns models in \autoref{algo:orchestration:application:line:returnresult}.
    These models are always consistent to the consistency relations of all transformations, because Lines~\ref{algo:orchestration:application:line:startconsistencycheck}--\ref{algo:orchestration:application:line:endconsistencycheck} ensure this and otherwise return $\bot$ before.
\end{proof}

\mnote{Termination is not guaranteed}
In addition to being correct, the algorithm needs to terminate always.
Non-termination can only occur because of the loop for orchestration transformations, as there are no recursions and the other loop is finite because the set of transformations is of finite size.
According to the definition, an orchestration function is defined to return a finite sequence of transformations, which would also result in a finite number of execution of the loop for orchestrating transformations.
The implementation by a dynamic selection of the next transformation to execute can, however, lead to an infinite sequence of transformations.
The \function{Orchestrate} function receives the list of previously executed transformations, as otherwise it would never be able to identify that, for example, always the same sequence of transformations is executed and leads to the same changes, thus the algorithms is only performing an infinite alternation.
We do, however, need to ensure that the \function{Orchestrate} function returns $\bot$ after a finite number of calls.

\mnote{Options to guarantee termination}
If we assume that we can achieve optimality for the orchestration function, we would have the guarantee that if a consistent orchestration exists, the function will find it.
There is, however, no restriction to what the orchestration function may return when there is no orchestration that yields consistent models at all.
Thus, we have two options to ensure termination:
\begin{enumerate}
    \item We enable the orchestration function to identify whether a consistent orchestration exists.
    \item We find an upper bound for the number of necessary transformation executions, such that if more transformations were executed, we cannot expect the algorithm to find consistent models anymore and thus abort. 
\end{enumerate}

\mnote{Termination and optimality are conflicting}
The simplest solution would be to find an upper bound for the number of necessary transformation executions.
We will, however, prove in the following that there is no such upper bound.
Afterwards, we will show that identifying whether a consistent orchestration exists is not possible either.
This will lead to the insight that we cannot guarantee termination of the algorithm with an optimal orchestration function.


\subsection{Upper Execution Bound}

\mnote{No general upper bound for necessary number of executions}
With the example in \autoref{fig:orchestration:necessity_multiple_executions}, in which values are incremented by one upon each execution of one specific transformation until a fixed but arbitrary value $x$ is reached, we were able to show in \autoref{lemma:minimal_executions} that there can be transformation networks in which a transformation needs to be executed at least $x-1$ times for a fixed but arbitrary $x$ until consistent models are models.
Thus, any consistent orchestration contains that transformation at least $x-1$ times.
While we have used that to show that executing each transformation only once is, in general, not sufficient, we can also easily show the more general statement that we cannot find a maximal length for the orchestration of transformation networks of specific size.

\begin{theorem}[Orchestration with Fixed Number of Executions]
    \label{theorem:orchestration_fixed}
    For every $n$, there is a set of transformations $\transformationset{T}$ such that for specific models $\modeltuple{m}$ and changes $\changetuple{}$ to them for which each possible orchestration function $\orcfunction{\transformationset{T}}$ with whom $\appfunction{\orcfunction{\transformationset{T}}}(\modeltuple{m}, \changetuple{})$ is consistent, delivers a sequence as $\orcfunction{\transformationset{T}}(\modeltuple{m}, \changetuple{})$ with $\abs{\orcfunction{\transformationset{T}}(\modeltuple{m}, \changetuple{})} > n$.
\end{theorem}
\begin{proof}
    We know from \autoref{lemma:minimal_executions} that $\transformationset{T}_{inc}$ requires at least $x-1$ executions of $\transformation{T}_{12}$ for the inputs defined in \autoref{lemma:minimal_executions} and the fixed but arbitrary value $x$.
    Thus, with $x \geq n+2$ for $\transformationset{T}_{inc}$, we know that at least $x-1 = n+1$ executions of $\transformation{T}_{12}$ are necessary.
    Let $\modeltuple{m}$ and $\changetuple{}$ be the inputs defined in \autoref{lemma:minimal_executions}.
    Then for any orchestration function $\orcfunction{\transformationset{T}}$ that delivers a consistent orchestration for these inputs, we know that $\abs{\orcfunction{\transformationset{T}}(\modeltuple{m}, \changetuple{})} >= x-1 = n+1 > n$.
    This proves the theorem by example.
\end{proof}

\mnote{Upper bound cannot be used to decide whether consistent orchestration exists}
In consequence, it is not possible to find a fixed value or a value only depending on the transformation network size that defines an upper bound for the necessary number of transformation executions to yield consistent models.
Thus, even if we are able to ensure optimality of orchestration with the \function{Apply} and \function{Orchestrate} functions, there is no upper bound for the number of transformation execution that is necessary for a consistent orchestration.
We cannot abort the execution after a fixed number of loop iterations without the possibility that consistent models would have been found if the execution had proceeded and thus not ensuring optimality.

%To avoid non-termination of the algorithm, we thus need to be able to ensure that the \function{Orchestrate} function at some point returns $\bot$.
%This means that it must be able to identify when no orchestration that yields consistent models exists at all, and if it exists it must be able to find it.

%Requirement: Know whether orchestration exists, otherwise impossible to find sequence always if it exists, as we do not know whether it exists.

% This is due to the reason that the \function{Orchestrate} function does only receive the current models and changes but not the history of executed transformations.
% Thus, it cannot identify which and how many transformation have already been executed and whether the current changes have already been produced before such that 

% There are, however, still derivations of the stepwise orchestrate to the original orchestration function:
% The stepwise application of the orchestrate function cannot derive a complete sequence but only make a local decision based on the given models and changes. Thus, if the transformations produce the same changes repeatedly, i.e., there is an alternation in the produced changes, the orchestrate function cannot detect that and always determinates the same sequences of transformations to be executed next. This results in an infinite sequence of execute transformations, which is conforming to the orchestration function definition, but still is a result of the possibilities of an actual and reasonable implementation of that function.
%Additionally, the function may be able to determine beforehand whether an orchestration that yields consistent models exists and otherwise return $\bot$ in its first application.
%A practical implementation, however, will not act that way but instead determine the next transformation until no transformation can be applied anymore.
%In consequence, it is possible that several transformations are executed before detecting that no orchestration can be found.
%Since the orchestration is encoded in the apply function anyway and the function then still returns $\bot$, this is not a problem.
%This affects termination.

%Algorithm is correct if it returns only consistent models. Additionally, it must always terminate.
%ADD DEFINITION!

%Correctness is given by construction and easy to achieve.
%Termination if no transformation can be applied or consistent models are found.

%Thus: We need to find an order of transformation such that the result is consistent.

%Give example, where no execution order exists that terminates. Thus, the algorithm does not terminate.

%Approach: We want to decide whether an order exists or not.


\subsection{Undecidability of the Orchestration Problem} %Undecidability of Orchestration}

\mnote{Impossible to decide whether consistent orchestration exists}
To ensure termination of the \function{Apply} algorithm with an optimal orchestration function, we need to identify the case that no consistent orchestration exists, because that is the only situation in which otherwise an infinite number of transformation execution is possible.
Unfortunately, we will show that the problem to decide whether such an orchestration exists or not is undecidable.

%--> Undecidability

\begin{lemma}
    \label{lemma:networkfromturingmachine}
    For any Turing machine $\TM$, we can construct metamodels $\metamodeltuple{M}$ and a transformation network $\transformationnetwork{N} = \tupled{\transformationset{T}, \appfunction{\orcfunction{}}}$ with an optimal application function $\appfunction{\orcfunction{}}$, such that $\TM$ halts for input $x$ if, and only if, for models $\modeltuple{m}_{x}$ and changes $\changetuple{\metamodeltuple{M},x}$ it is $\appfunction{\orcfunction{}}(\modeltuple{m}_{x}, \changetuple{\metamodeltuple{M},x}) \consistenttomath \transformationset{T}$.
\end{lemma}
\begin{proof}
    \todo{Add proof with turing machine construction}
\end{proof}

\begin{theorem}
    \label{theorem:nooptimalapplication}
    Let $\appfunction{\orcfunction{}}$ be an application function. $\appfunction{\orcfunction{}}$ cannot be optimal.
\end{theorem}
\begin{proof}
    An optimal $\appfunction{\orcfunction{}}$ returns consistent models whenever there is a consistent orchestration.
    With such a function, we are able to decide whether such an orchestration exists or not.
    \begin{align*}
        \function{ExistsOrc}(\transformationset{T},\modeltuple{m},\changetuple{\metamodeltuple{M}}) =
            \begin{cases}
                \textsc{true}, & \appfunction{\orcfunction{}}(\transformationset{T}, \modeltuple{m},\changetuple{\metamodeltuple{M}}) \consistenttomath \transformationset{T} \\
                \textsc{false}, & otherwise
            \end{cases}
    \end{align*}
    $\function{ExistsOrc}$ returns \textsc{true} if, any only if, a consistent orchestration exists.
    $\appfunction{\orcfunction{}}$ does, per definition, only return consistent models when there is an orchestration that yields them.
    Additionally, it does always return consistent models when an orchestration that yields them exists, because it is optimal.
    It follows from \autoref{lemma:networkfromturingmachine} that we can simulate a universal Turing machine with a transformation network in the sense that we can construct a transformation network whose application function delivers consistent models for a given input if, and only if, the Turing machine halts for corresponding inputs.
    Calculating \function{ExistsOrc} would thus decide whether the Turing machine halts and thus decide the halting problem.
\end{proof}

\mnote{Algorithm cannot terminate and be optimal}
From this theorem, it directly follows that we cannot implement \function{Orchestrate} within the \function{Apply} algorithm in a way that it realizes an optimal application function and terminates always.

\begin{corollary}
    \function{Apply} according to \autoref{algo:orchestration:application} cannot terminate and return consistent models if an orchestration exists that yields them for every possible input.
\end{corollary}
\begin{proof}
    If \function{Apply} always terminated and returned consistent models whenever there is an orchestration that yields them, it would implement an optimal application function and could be used to realize \function{ExistsOrc} from the proof of \autoref{theorem:nooptimalapplication}.
    But according to \autoref{theorem:nooptimalapplication} an application function cannot be optimal.
\end{proof}

\mnote{Restrict expressiveness of transformations or accept conservativeness}
In consequence, we only have the two options to either restrict the expressiveness of the transformations such that they cannot be used to simulate a Turing machine anymore or to accept the situation that \function{Apply} may either not terminate in some cases or return $\bot$ although there is an orchestration that yields consistent models.
We call this behavior \emph{conservative}, because the algorithm does never return consistent models although there is no orchestration that yields them, but may not return consistent models in some cases in which actually such an orchestration existed.

\mnote{Discuss both options in the following}
In the following, we first introduce alternation as a special case of non-termination that can be avoided by construction.
We then discuss different options to restrict transformations to finally conclude that this is not an option for solving the above discussed problem.
Afterwards, we discuss how we can realize \function{Apply} in a way that it always terminates and produces reasonable outputs.

\begin{corollary}
    The orchestration problem (see \autoref{def:orchestrationproblem}) is undecidable.
\end{corollary}
\begin{proof}
    \todo{Add proof. Maybe this should be the theorem from the Turing machine. Then we follow insights to the algorithm from that.}
\end{proof}
%Lemma: It is undecidable whether an orchestration exists for a given input.

%Theorem: We cannot guarantee termination of Algorithm, when we allow an arbitrary number of transformation execution. (combine with upper bound, see above/below)

% Consequence: Either we need to restrict transformations to make the problem decidable or we need to restrict the application function and allow it to return $\bot$ although a sequence returning consistency exists, thus, we need to deal with conservativeness.
% \textit{Weaker version:}
% Goal: Find a solution in as many cases as possible, abort in the others (conservatively). There are two approaches to achieve that: 
% 1. Reduce the number of cases in which there is no solution by adding assumptions to the relations and transformations (restrict input of app function)
% 2. Improve the ability to find a solution if it exists (improve capabilities of app function)
% Secondary goal: In cases, in which no solution is found, support the user in understanding why no solution was found.

%Since we find that we cannot guarantee to find an orchestration, we need to deal with the conservative case anyway. Thus, we can also deal with the conservative case for the second problem instead of requiring things from the transformation which may not be (easily) fulfillable.

% Conclude that we cannot guarantee termination for an optimal orchestration function


\subsection{Alternation} % Divergence and alternation

The proposed algorithm, like any algorithm, is supposed to \emph{terminate} in specific \emph{state} to be considered correct.
In our case, a correct state, as required by an application function it implements, is the return of consistent models or $\bot$, which the algorithm fulfills by construction.
In particular, the algorithm will never return models that are inconsistent, neither because it does not detect that they are inconsistent nor that it detects that they are inconsistent but still returns them.
From our previous findings regarding decidability, we know that we cannot expect the algorithm to realize an optimal application function.
Thus, we either need to implement \function{Orchestrate} such that it always returns $\bot$ after a finite number of executions to ensure termination, which results in returning $\bot$ although consistent models are expected, or we allow an arbitrary number of executions to improve the ability to find consistent results but accept that the algorithm may not terminate.

We have discussed that non-termination of the algorithm can occur because no consistent orchestration exists at all or because the algorithm is not able to find it.
A special case of non-termination is \emph{alternation}, which means that the same states are passed repeatedly. 
In case of transformation networks, alternation means from some point in time the subsequent executions of the transformations in Line~\ref{algo:orchestration:application:line:stepcalculation} of \autoref{algo:orchestration:application} repeatedly produce the same sequence of results, i.e., of changes.
% Non-termination can, in general, manifest in terms of \emph{alternation} or \emph{divergence}, which means that either the same states are passed repeatedly or that an infinite sequence of different states is produced.
% In case of transformation networks, alternation means from some point in time the subsequent executions of the transformations in Line~\ref{algo:orchestration:application:line:stepcalculation} of \autoref{algo:orchestration:application} repeatedly produce the same sequence of results, i.e., of changes.
% Divergence means that from some point in time all results, i.e., changes, produced in Line~\ref{algo:orchestration:application:line:stepcalculation} differ.
In contrast to non-termination in general, the scenario of alternation can at least be avoided by construction.
% To this end, the history of change produced by the algorithm in Line~\ref{algo:orchestration:application:line:stepcalculation} has to be stored.
% It can either be used by the \function{Apply} function to detect alternation or by passing it to the \function{Orchestrate} function to influence the selection of transformations to avoid alternation.

The \function{Orchestrate} function receives the history of transformations and changes and is thus able to identify the situation that the same sequence of transformations was already executed and produced equal changes in each application.
This allows it to implement the function in a way that it does not return the same sequence of transformation when it was already passed and produced the same changes.
If a concrete realization of the \function{Orchestrate} function is not implemented in a way that it can react to the detection of alternation and produce a different sequence of transformations, it can at least return $\bot$ to ensure termination of \function{Apply}, because repeated execution of the same transformations will still returns the same changes. 

\todo{Maybe add reference to conservative orchestration section for the dynamic detection of alternation, if appropriate.}

%Divergence: If it passes the same changes again, then it either does not pass those changes again 

% Derive from the previous insights that whenever the algorithm does not terminate, we can have two situations: divergence and alternation.
% Prove that no other options for occurring situation exist!

% Problemraum:
% \begin{itemize}
%     \item Ziel ist, dass ein Netzwerk von Transformationen nach einer Änderung in einem konsistenten Zustand terminiert. D.h. Korrektheit stellt Anforderungen an \emph{Terminierung}, sowie den \emph{Zustand} bei Terminierung.
%     \item Folgende Abweichungen davon können auftreten:
%     \begin{enumerate}
%         \item Nicht-Terminierung: Das Netzwerk terminiert nicht. Das bedeutet im Prinzip, dass die Ausführungsfunktion (bzw. der Laufzeit-Algorithmus, der die Funktion dynamisch emuliert) nicht \emph{sound} ist. Soundness der Ausführungsfunktion setzt voraus, dass die berechnet Aufrufsequenz endlich ist. Wenn die Ausführung nicht terminiert, bedeutet das, dass entweder die gleichen Zustände mehrfach durchlaufen werden oder eine Sequenz unendlich vieler Zustände produziert wird. Denn wenn beides nicht der Fall ist, gibt es eine endliche Sequenz unterschiedlicher Zustände, d.h. Terminierung. Das bedeutet, dass es folgende zwei Möglichkeiten gibt:
%         \begin{itemize}
%             \item Alternierung: Die gleichen Zustände werden mehrfach durchlaufen.
%             \item Divergenz: Es werden unendlich viele Zustände produziert.
%         \end{itemize}
%         \item Inkonsistente Terminierung: Die Ausführungsfunktion bzw. der Algorithmus beendet die Ausführung, aber in einem inkonsistenten Zustand. Hier lassen sich ebenfalls wieder zwei Fälle unterscheiden.
%         \begin{itemize}
%             \item Unerkannte Inkonsistenz: Der Algorithmus terminiert und denkt, der Zielzustand wäre konsistent. Dies bedeutet aber direkt, dass nicht alle Konsistenzrelationen erfüllt sind, was, zumindest in der Theorie, einfach zu prüfen wäre (entweder durch Prüfung der Relationen oder durch Ausführung der hippokratischen Transformationen, die alle nichts tun dürften)
%             \item Erkannte Inkonsistenz: Der Algorithmus terminiert, wissend dass die Lösung nicht konsistent ist. Dies kann entweder sein, weil eine Transformation für zwei Modelle in einem inkonsistenten Zustand nicht mehr anwendbar ist, oder weil irgendein anderes Abbruchkriterium erreicht ist.
%         \end{itemize}
%     \end{enumerate}
% \end{itemize}

% Assume we have an algorithm that sequentially applies transformations.
% It stops as soon as a transformation cannot be applied or the models are consistent.
% Then we need to guarantee termination.
% We need to avoid that transformations can be applied indefinitely never leading to consistent models.

% Reasons for this situation are alternation and divergence.
% Prove that if we do not pass the same model state again (alternation) and if there is no indefinite number of model states (divergence), the algorithm terminates.
% Thus, if we ensure that any execution order of transformations does never lead to alternation and divergence, we know that the algorithm terminates!!

% \begin{itemize}
%     \item Zeigen, dass es Beispiele gibt, in denen es unabhängig von der Ausführungsreihenfolge immer zu einer Alternierung kommt
%     \item Zeigen, dass es Beispiele gibt, in denen es unabhängig von der Ausführungsreihenfolge immer zu einer Divergenz kommt.
%     \item Die Beispiele sollten zeigen, dass wir keine Einschränkungen an die Transformationen machen können, was das Problem aushebelt. D.h. egal welche Einschränkungen ich an die Transformationen definiere, es lassen sich immer Beispiele konstruieren, in denen es keine Ausführungsreihenfolge gibt, in denen sie terminieren.
%     \item Mathematisch zeigen, dass Alternierung und Divergenz die einzigen Probleme sind. D.h. wenn nicht der gleiche Zustand mehrmals durchlaufen wird (Alternierung) und es nicht unendlich viele Zustände gibt (Divergenz), dann ist die Folge endlich.
%     %\item Außerdem mathematisch die Abbildung von Transformationen auf Turing-Maschinen zeigen und damit ableiten, dass allgemeine Netzwerke erstmal nicht terminieren müssen (Abbildung auf Halteproblem)
% \end{itemize}

% To avoid these problems by construction, we discussed before that we need to achieve that P = S, such that the application function can execute transformations in an arbitrary order to achieve consistency.

% Another possibility would be to allow the problems and detect them dynamically and react to them.
% We will finally discuss that in the last section.

%In the following, we discuss whether and how we may restrict synchronizing transformations, such that an arbitrary execution order can avoid divergence and alternation, such that the algorithm terminates.

