\section{Summary}

In this chapter, we have discussed the separation of the transformation network specification process into three different levels, we have categorized the possible mistakes, faults and failures that can occur in such a network and discussed which of them can be avoided or detected.
We have discussed the avoidance and detection of errors at a rather conceptual level, emphasizing what a transformation developer has to do to achieve correctness by construction and what he or she has to do if a transformation is failing.
We did, however, not discuss or propose a concrete process for the resolution of errors when they occur in a productive environment.
This involves a system developer, who uses the transformations to keep his or her models consistent and detects failing transformations, as well as the transformation developer, who is responsible for correcting the potential faults in the implementation.
Such a process discussion is our of the scope of this thesis and thus referred to as future work (see~\autoref{chap:futurework:correctness:orchestration:process}).

%Fitting that transformation level can be avoided by construction, other level are handled by conservative algorithm.

%\todo{Maybe discuss replacement of algorithm with less-restrictive one as in failures section and problem of consistency check by transformation execution again or here instead}

\begin{insight}[Errors]
    Errors in transformation networks can be classified regarding mistakes made by the transformation developers when thinking about consistency and its preservation, faults made during their implementation in terms of transformations, and failures, which are the manifestation of faults when executing the transformations.
    We found that we can assign different kinds of mistakes to three different conceptual levels in the specification process, depending on the necessary knowledge about the transformation network.
    We were able to derive that mistakes regarding a single transformation cover missing synchronization, which can and has to be avoided by construction.
    This is especially necessary if transformations assume consistency to be achieved by construction, because then non-synchronizing transformations produce faulty results that they assume to be consistent, because they have generated them.
    All other types of mistakes concern the network of transformations, either restricted to the relations or also concerning the consistency preservation rules.
    While consistency relations can at least be analyzed for compatibility, further mistakes cannot be avoided but only be detected by the application algorithm failing in specific scenarios.
    Due to the assumption of independent transformation development and reuse, it fits well that a conservative application algorithm is necessary anyway and also covers mistakes concerned with the network of transformations.
    Only if the transformation network fails in many cases, because of non-fitting transformations, such as having incompatible consistency relations, the transformation developers need to investigate the reasons for the algorithm to fail.
\end{insight}