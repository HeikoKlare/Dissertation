\section{Categorization of Errors in Transformation Networks}

\todo{Discuss problem of default values, or more general \enquote{special cases}?}

\mnote{Categorization of mistakes, faults and failures}
In this section, we identify and categorize potential \emph{failures} that can occur when executing transformation networks, which are derived from the failure cases of the application algorithms already discussed in \autoref{chap:orchestration}.
We then consider the \emph{mistakes} and the resulting \emph{faults} in the transformation specifications, which a transformation developer can make.
The mistakes are specific for the yet introduced conceptual specification levels, thus we derive them from those levels.
We finally relate the mistakes to the failures that can occur when transformation networks containing faults that are caused by those mistakes are executed nevertheless.

% In this section, we %first identify and 
% categorize potential \emph{failures} that can occur when executing \acp{BX} in a network to preserve consistency.
% We then consider \emph{mistakes} that a developer can make and that lead to \emph{faults} in the specifications of consistency and its preservation.
% \todo{Heißt der folgende Satz nicht, dass auf jeder Ebene genau ein Typ von Fehler auftreten kann?}
% We derive them from the specification levels introduced in \autoref{chap:properties:levels}, as each kind of mistake is specific for one of those levels.
% We finally relate the mistakes to the failures that can occur while executing the operationalization of a faulty consistency specification.
% That categorization forms our contribution \ref{contrib:issues}.
% In the following, we only discuss failures and their causing mistakes, but no strategies to solve or avoid them.
% Such strategies are discussed in \autoref{chap:prevention}.
%The identification and categorization in this section is based on argumentation. To show the correctness of identified mistakes, failures and their dependencies, we provide an appropriate evaluation in \autoref{sec:evaluation}.

%\todoHeiko{Introduce mistake, fault, failure} 

% \begin{itemize}
%     \item Define three essential abstraction levels in the development process
%     \item Levels depend on each other, so \emph{fulfillment} on one level is mandatory to investigate the next level
%     \item Mistakes on all levels may introduce failures in the execution of the operationalization of consistency constraint preservation
%     \item We summarize potential failures, identify their causes (mistakes and faults) and then categorize and relate them
% \end{itemize}

\subsection{Mistakes, Faults and Failures}

\mnote{Distinction of error types}
Errors in transformation networks can occur in different contexts, for example in terms of the transformation networks, more precisely the application algorithm, producing an incorrect result, or in terms of a transformation developer defining an erroneous transformation.
To be able to distinguish these contexts, we have already used the terms \emph{mistakes}, \emph{fault} and \emph{failure} with a short introduction of their distinction, as specializations of the general term \emph{error}.
They are supposed to describe erroneous or inappropriate knowledge of a developer (mistakes), erroneous implementations (faults) and erroneous execution results (failures).
These different types of errors depend on each other, as a mistake can lead to a fault, which can finally lead to a failure.

\begin{properdescription}
    \item[Mistake:] \mnote{Mistake results from erroneous knowledge of transformation developer}
    A mistake is made by a transformation developer. It is based on missing or erroneous knowledge about either the concrete transformation or the necessity to ensure certain properties. For example, the missing knowledge that transformations need to be synchronizing leads to a mistake in the conceptualization of transformations as they do not ensure this required property. The missing knowledge that compatibility is required as well as the missing knowledge about the other transformations of the transformation network can lead to the mistake that incompatible transformations are realized.
    If a transformation language abstracts from a specific conceptual level and thus relieves the developer from ensuring that no mistakes at that level are made, a faulty implementation of the language can, for sure, also have faulty behavior because of those mistakes, if they were made by the transformation language developer. We do, however, not consider that case explicitly. 
    \item[Fault:] \mnote{Fault is manifestation of mistake in transformation implementation}
    A fault is the manifestation of a mistake in the implementation of transformations. For example, the missing knowledge about the necessity to have synchronizing transformations can lead to the fault that the implementation does not properly match existing elements instead of creating new ones. A fault is, thus, always the consequence of a mistake. It is also made by a transformation developer, but can be seen explicitly in the implementation, whereas a mistake can only be detected by the fault in the implementation to which it led.
    \item[Failure:] \mnote{Failure is manifestation of fault when executing transformations}
    A failure occurs at execution time of the transformation and is the manifestation of a fault when executing a faulty transformation network. A failure is the incorrect result of the execution of transformations. Whenever the transformations in a network have a faulty implementation, failures such as the termination in inconsistent states or non-termination of the application algorithm can occur. Since the occurrence of a failure depends on the scenario in which the transformations are executed, not every fault must lead to a failure. On the other hand, a fault can also lead to several failures, for example, because a transformation is executed multiple times.
\end{properdescription}

\mnote{Relation to other notions of error terms}
Several similar terms like errors, mistakes, faults, bugs, defects and so on are used in software engineering and especially in software testing.
They are sometimes used interchangeably and sometimes with specific meanings.
One common notion is the distinction of faults, errors and failures in software testing, however also with different meanings, of which at least one is comparable to ours using the term \emph{error} for what we call \emph{mistake}.
We decided to avoid the overloaded term \emph{error} and make the human \emph{mistake} explicit.


\subsection{Possible Failure Types}
\label{chap:errors:categorization:failures}

\mnote{Essential failures are non-termination of inconsistent models}
Failures are the manifestation of faults during transformation execution and thus the final result of mistakes made by a transformation developer.
A failure means that the transformation network or more precisely the execution of the application algorithm reached an unwanted state.
We have already discussed in \autoref{chap:orchestration:decidability:correctness_termination} that the application algorithm can fail by not implementing a correct application function, thus either returning models that are inconsistent or by not terminating at all.

\mnote{Specialization options for general failure types}
Termination in an inconsistent state and non-termination already form the two general failure types that can occur when executing faulty transformations.
They can be further specialized in different dimensions, e.g., regarding determinism of inconsistent termination or regarding whether too many or too few elements (or combination of them) consist for being consistent, i.e., whether corresponding condition elements are missing or whether there are too many condition elements for which no consistent models can be found by adding further ones.
We did, however, find in \cite[Table~5.7]{saglam2020ma} that the distinction regarding elements does not bring any benefits when tracing the failures back to the causal mistakes.
We do, however, consider \emph{duplications} as one specific additional failure type, which can lead to both the return of inconsistent models or non-termination, depending on whether the application algorithm aborts or not.
Duplications of elements are of special importance, because they are the essential manifestation of missing synchronization in transformations, as we have discussed in \autoref{chap:synchronization:achieving}.

\begin{figure}
    \centering
    \newcommand{\labelcolumnwidth}{4.5em}
\newcommand{\contentcolumnwidth}{7em}
\newcommand{\failurecolumnfactor}{1.65}
\newcommand{\rowdistance}{6em}

\begin{tikzpicture}[
    entry/.style={font=\small}
]

    
    \node[text depth=18em, minimum width=\contentcolumnwidth+\labelcolumnwidth, minimum height=20em] %, fill=lightgray!50]
    (errors) {\textbf{Mistakes}};
    \node[right=\contentcolumnwidth+0.5*\labelcolumnwidth of errors.center, anchor=center, text depth=18em, minimum width=\contentcolumnwidth, minimum height=20em] %, fill=lightgray!25] 
    (faults) {\textbf{Faults}};
    \node[right=(0.5+0.5*\failurecolumnfactor)*\contentcolumnwidth of faults.center, anchor=center, text depth=18em, minimum width=\failurecolumnfactor*\contentcolumnwidth, minimum height=20em] %, fill=lightgray!50] 
    (failures) {\textbf{Failures}};
    
    \node[below=5em of errors.north west, anchor=west, align=left, inner sep=0.7em] (transformation) {Level 1:\\ \emph{Transfor-}\\\emph{mation}};
    \node[below=\rowdistance of transformation.west, anchor=west, align=left, inner sep=0.7em] (relation) {Level 2:\\ \emph{Network}\\\emph{Relation}};
    \node[below=\rowdistance of relation.west, anchor=west, align=left, inner sep=0.7em] (rule) {Level 3:\\ \emph{Network}\\\emph{Rule}};
    
    \draw[thick] (errors.north west) -- ++(\labelcolumnwidth+2*\contentcolumnwidth+\failurecolumnfactor*\contentcolumnwidth, 0);
    \draw[very thin] ([yshift=0.5*\rowdistance, xshift=0.05*\contentcolumnwidth]transformation.west) -- ++(\labelcolumnwidth+0.9*\contentcolumnwidth,0);
    \draw[very thin] ([yshift=0.5*\rowdistance, xshift=\labelcolumnwidth+1.05*\contentcolumnwidth]transformation.west) -- ++(0.9*\contentcolumnwidth,0);
    \draw[very thin] ([yshift=0.5*\rowdistance, xshift=\labelcolumnwidth+2.05*\contentcolumnwidth]transformation.west) -- ++(\failurecolumnfactor*\contentcolumnwidth-0.1*\contentcolumnwidth,0);
    \draw[dashed, very thin] ([yshift=0.5*\rowdistance, xshift=0.05*\contentcolumnwidth]relation.west) -- ++(\labelcolumnwidth+0.9*\contentcolumnwidth,0);
    \draw[dashed, very thin] ([yshift=0.5*\rowdistance, xshift=0.05*\contentcolumnwidth]rule.west) -- ++(\labelcolumnwidth+0.9*\contentcolumnwidth,0);
    \draw[thick] (errors.south west) -- ++(\labelcolumnwidth+2*\contentcolumnwidth+\failurecolumnfactor*\contentcolumnwidth, 0);
    
    % ERRORS
    \node[entry, right=\labelcolumnwidth+0.5*\contentcolumnwidth of transformation.west, anchor=center, align=center] (error_transformation) {missing\\ synchronization};
    
    \node[entry, right=\labelcolumnwidth+0.5*\contentcolumnwidth of relation.west, anchor=center, align=center] (error_relation) {incompatible \\ constraint\\ knowledge};
    
    \node[entry, right=\labelcolumnwidth+0.5*\contentcolumnwidth of rule.west, anchor=center, align=center] (error_rule) {contradicting \\ options \\ selection};
    
    % FAULTS
    \node[entry, right=\contentcolumnwidth of error_transformation.center, anchor=center, align=center] (fault_matching) {missing \\ element\\ matching};
    \node[entry, below=5em of fault_matching.south, anchor=north, align=center] (fault_contradicting) {contradicting \\ element\\ generation / \\ change};
    
    % FAILURES
    \node[entry, right=(0.5+0.5*\failurecolumnfactor)*\contentcolumnwidth of fault_matching.north, anchor=north, align=center] (failure_duplications) {
        duplications\\
        \textbullet\ multiple instantiations\\
        \textbullet\ multiple insertions
    };
    
    \node[entry, below=1em of failure_duplications.south, anchor=north, align=center] (failure_termination) {
        inconsistent termination\\
        \textbullet\ deterministic\\
        \textbullet\ non-deterministic
    };
    
    \node[entry, below=1em of failure_termination.south, anchor=north, align=center] (failure_non_termination) {
        non-termination\\
        \textbullet\ alternation\\
        \textbullet\ divergence
    };
    
    \node[entry, below=1em of failure_non_termination.south, anchor=north, align=center] (failure_bot) {
        returning $\bot$
    };

    % ERROR -> FAULT
    \draw[-latex] ([xshift=-1.2em,yshift=0.6em]error_transformation.east) -- ([yshift=0.6em]fault_matching.west);
    
    \draw[-latex] ([xshift=-0.9em]error_relation.east) .. controls ++ (2.2em,0em) and ($(fault_contradicting.west)-(1.5em,-0.8em)$) .. ([yshift=0.8em]fault_contradicting.west);
    \draw[-latex] ([xshift=-0.5em]error_rule.east) .. controls ++ (2em,0em) and ($(fault_contradicting.west)-(1.5em,-0.3em)$) .. ([yshift=0.3em]fault_contradicting.west);
    
   
    % % FAULT -> FAILURE
    \draw[-latex] ([xshift=-0.3em]fault_matching.east) .. controls ++ (3em,0em) and ($(failure_duplications.west)-(1em,-1.1em)$) .. ([yshift=1.1em,xshift=2em]failure_duplications.west);
    
    \draw[-latex] ([xshift=-0.7em,yshift=0.8em]fault_contradicting.east) .. controls ++ (2.2em,0em) and ($(failure_termination.west)-(2.2em,-1.1em)$) .. ([yshift=1.1em,xshift=0em]failure_termination.west);
    \draw[-latex] ([xshift=-0.7em,yshift=0.5em]fault_contradicting.east) .. controls ++ (2em,0em) and ($(failure_non_termination.west)-(2em,-1.1em)$) .. ([yshift=1.1em,xshift=0em]failure_non_termination.west);
    \draw[-latex] ([xshift=-0.7em,yshift=0.2em]fault_contradicting.east) .. controls ++ (3em,0em) and ($(failure_bot.west)-(3em,0em)$) .. (failure_bot.west);

    \draw[-latex] ([xshift=-2em,yshift=1.1em]failure_duplications.east) -- ++(3em,0) |- (failure_bot.east);
    \draw[-latex] ($(failure_duplications.east)+(1em,-3.6em)$) -- ++(-0.9em,0);
    \draw[-latex] ($(failure_duplications.east)+(1em,-8.1em)$) -- ++(-2.2em,0);
    
\end{tikzpicture}
    \caption{Categorization and Dependencies of Mistakes, Faults and Failures}
    \label{fig:errors:categorization}
\end{figure}

\mnote{Failures only occur with specific application algorithms}
In \autoref{fig:error:categorization}, we depict the different failures types with their specializations, which we discuss in the following.
Please note that we do not assume a specific application algorithm when discussing failures.
Whether a potential failure occurs or not highly depends on the used algorithm.
For example, using the provenance algorithm proposed in \autoref{chap:orchestration:algorithm} will neither lead to non-termination, nor to inconsistent models, at least of the consistency check is implemented properly.
Having an artificial upper bound for the number of transformation executions, for sure, always prevents from non-termination.
Only if transformation are executed without checking consistency afterwards or without defining an execution bound, the discussed failures can actually occur.
In other cases, the actual failures are masked by the algorithm simply aborting and returning $\bot$.

% Mistakes in the specification of consistency, no matter on which of the specification levels, % (\autoref{sec:process:levels}),
% can lead to failures when executing the preservation of consistency according to that specification. % on an actual system. 
% Before identifying the causal mistakes, we first categorize the types of potential failures into three categories. We depict them in \autoref{fig:errors:categorization}.

We further distinguish the already discussed failure types as follows.
\begin{properdescription}
    \item[Inconsistent termination:] \mnote{Deterministic or non-determinstic termination in inconsistent states}
    Inconsistent termination means that the application algorithm terminates and the models it returns are inconsistent.
    This can only occur if the algorithm does not check the models yielded by the application of transformations in the order determined by the orchestration function for consistency.
    Furthermore it can terminate \emph{deterministically} or \emph{non-deterministically}, depending on whether each execution delivers the same inconsistent models or different ones, because different execution orders of the transformations are selected, which yield different inconsistent models.

    \item[Non-termination:] \mnote{Non-termination in terms of alternation or divergence}
    Non-termination means that the application algorithm does not terminate, but executes transformations indefinitely without achieving a consistent state of the models.
    We can further distinguish between \emph{alternation} and \emph{divergence} as defined in \autoref{def:applyalternation}.
    Alternation means that the same model states are produced repeatedly, which can for example be because a feature, such as an attribute or reference, alternates between two or more values.
    In other cases, we have divergence, which means that some feature values are changed indefinitely, such as a number counting up or a string being repeatedly appended, or an infinite number of elements is created.
    While an alternating algorithm can easily run endlessly, a diverging algorithm with abort at some point in time in many cases, because endless element creation or string concatenation can lead to an overflow of available memory.
    
    \item[Duplications:] \mnote{Element duplications as a special case leading to non-termiation or inconsistent termination}
    As a more specific failure case, we have introduced element duplications, which can especially arise if transformations are not synchronizing and thus do not match existing elements rather than creating new ones.
    We can further separate this into \emph{multiple instantiation}, which can occur because different consistency preservation rules instantiate an element multiple times, although all of them represent the same element, and into \emph{multiple referencing}, which can occur because an element is inserted into a reference or attribute list several times, although it should be inserted only once. 
    In fact, such duplications ultimately can ultimately also lead to inconsistent termination or non-termination, because either the algorithm returns after a finite number of transformation executions without checking consistency or the transformations are not able to restore consistency for the models and the algorithm does thus not terminate.
    Duplications, however, represent a special case, which, as we will see in the evaluation in \autoref{chap:correctness_evaluation}, is one of the most important error case for transformation networks.
    Thus, identifying such duplications in the generated models can ease finding the causal mistake in terms of missing synchronization.
\end{properdescription}

\mnote{Consistency checks are prone to not work properly}
Although we have discussed that such failures will not occur in an application algorithm that checks consistency and has an artificial execution bound, knowing them and their relation to the causal mistakes is still important.
First, when a transformation network with such an application algorithm yields $\bot$ in most cases, there will likely be a fault in the transformation implementations.
Temporarily replacing the algorithm with a less restrictive one can help to find the reasons, because then, for example, duplications may be detectable that help to identify missing synchronization.
Second, in many transformation languages, consistency relations are not represented explicitly, thus consistency checks are performed by executing the transformation and checking whether changes were performed.
Then, if transformations are non-synchronizing, they return an actually inconsistent state, which may, however, not be identified by the transformation as such, because it does not assume the synchronization scenario and thus assumes that changes must only be processed for one model and thus the models are consistent after executing the consistency preservation rules.

% If mistakes are made during the specification of consistency, no matter on which of the levels introduced in \autoref{sec:process:levels}, this can finally lead to failures in the consistency preservation executed on an actual system. Before identifying the causing mistakes, we first give an overview on the types of failures that may occur and separate them into three categories.\\[-0.7em]

% \compactsubsection{Termination in inconsistent states}
% \begin{enumerate}[topsep=4pt]
%     \item \emph{Deterministic:} The consistency preservation process can deterministically terminate in a state that is not consistent. % wrt. the defined consistency specification.
%     \item \emph{Non-deterministic:} Consistency preservation can non-deterministically terminate in an inconsistent state, depending on the execution order of the binary consistency preservation specifications. %in which the partial consistency preservation rules are executed.
% \end{enumerate}

% \compactsubsection{Non-termination}
% \begin{enumerate}[resume, topsep=4pt]
%     \item \emph{Alternating loops:} Consistency preservation can be non-terminating, alternating between two or more values in at least one feature (e.g. a number or a String alternating between two values).
%     \item \emph{Diverging loops:} Consistency preservation can be non-terminating, having at least one feature with a diverging value (e.g. a number counting up or down, a String being always appended).
% \end{enumerate}

% \compactsubsection{Duplications}
% \begin{enumerate}[resume, topsep=4pt]
%     \item \emph{Multiple instantiation:} An element can be instantiated multiple times by different consistency preservation specifications, although all of them represent the same element and thus should be the same. E.g. an element is created by transformations $\mathcal{M}_1 \rightarrow \mathcal{M}_2 \rightarrow \mathcal{M}_3$ and another is created by transformations $\mathcal{M}_1 \rightarrow \mathcal{M}_3$, although the same element is meant.
%     \item \emph{Multiple referencing:} An element may also be inserted into a non-containment reference or an attribute list several times, although the same element is meant, within the same situations as multiple instantiation can occur.
% \end{enumerate}


\subsection{Mistake and Fault Types}
\label{chap:errors:categorization:mistakes}

Failures can occur because of different conceptual and technical mistakes.
We do not consider technical mistakes, e.g., missing handling of null or comparable things.
We do not consider conceptual mistakes regarding correctness of a single transformation (unidirectional). I.e., if a single transformation does not ensure consistency in one direction appropriately, we do not consider that a relevant mistake. Mistakes regarding synchronization, however (occurring with concurrent changes but not changes in one of the models), are relevant.
We assume that transformations are correct, i.e., CPRs correct w.r.t. their relations

Weitere Fehlertypen diskutieren / explizit ausschließen, insb. Implementierungsfehler, technische Fehler. Diese führen i.d.R. aber auch dazu, dass eine Transformation nicht korrekt ist.
Wir nehmen wir Korrektheit einer Transformation an!
In der Evaluation werden wir das noch entsprechend unterscheiden.

Inkompatible Constraints und inkompatible Optionsauswahl lässt sich oft gar nicht unterscheiden, insb. wenn Constraints nicht explizit angegeben sind. Dann sind Constraints implizit durch CPRs definiert. Es könnte aber auch sein, dass die (vom Entwickler gedachten) Constraints relaxierter sind und dann das Problem die Optionsauswahl ist und nicht die Inkompatibilität von Constraints.



%\todoErik{Ich dachte immer, \enquote{Mistakes} machen nur Menschen}
Developers or the transformation engine can make different kinds of mistakes on each of the specification levels, which lead to faults in the specification and finally to different kinds of failures during consistency preservation.
In the following, we derive mistakes and faults from the specification levels, depicted in \autoref{fig:errors:categorization}.

\subsubsection{Global Level}
Regarding global consistency specifications for a set of model types, two basic mistakes can be made. 
These mistakes concern compliance of the defined consistency specification with the actual notion of consistency between the involved model types.
First, a specification can be incomplete (\emph{underspecified}), which means that some consistency constraints are missed. 
As a result, the consistency specification according to \autoref{def:consistency_specification} would contain more tuples of models than are actually consistent to each other. 
%%As a result, if one would define the consistency specification according to \autoref{def:consistency_specification}, more tuples of models would be in the relation than are actually consistent to each other. 
%Incomplete consistency specifications can lead to \emph{false positives}, when investigating whether a given tuple of models is consistent or not.
Another potential mistake are too restricted (\emph{overspecified}) consistency specifications, which means that additional, faulty consistency constraints are considered. 
As a result, actually consistent tuples of models would be missing in the consistency specification according to \autoref{def:consistency_specification}. 
%As a result, if one would define the consistency specification according to \autoref{def:consistency_specification}, actually consistent tuples of models would not be in the relation. 
%This %, in contrast, 
%can lead to \emph{false negatives}, because actually consistent models are identified as inconsistent.

\begin{figure}[bt]
    \centering
%    \includegraphics[angle=-90, width=\textwidth]{figures/levels_overview.pdf}
    \newcommand{\modeltypesize}{3.6em}
\newcommand{\modelsize}{0.3em}

\begin{tikzpicture}[
    model type/.style={draw, circle, minimum height=\modeltypesize},
    model/.style={draw, fill, circle, minimum height=\modelsize, inner sep=0},
    label distance=-0.2em,
    every label/.append style={font=\small},
    legend/.style={font=\footnotesize}
]

\foreach \level in {1,2,3} {
    \node[model type, label=265:{$\mathcal{M}_1$}] at (\level*3.45*\modeltypesize, 0) (L\level_MM1) {};
    \node[model, above left=0.25*\modeltypesize and 0.2*\modeltypesize of L\level_MM1.center, anchor=center] (L\level_MM1_M1) {};
    \node[model, right=0.3*\modeltypesize of L\level_MM1.center, anchor=center] (L\level_MM1_M2) {};
    \node[model, below left=0.25*\modeltypesize and 0.2*\modeltypesize of L\level_MM1.center, anchor=center] (L\level_MM1_M3) {};
    
    \node[model type, label=280:{$\mathcal{M}_2$}] at (\level*3.45*\modeltypesize+1.7*\modeltypesize, 0) (L\level_MM2) {};
    \node[model, above right=0.25*\modeltypesize and 0.2*\modeltypesize of L\level_MM2.center, anchor=center] (L\level_MM2_M1) {};
    \node[model, left=0.3*\modeltypesize of L\level_MM2.center, anchor=center] (L\level_MM2_M2) {};
    \node[model, below right=0.25*\modeltypesize and 0.2*\modeltypesize of L\level_MM2.center, anchor=center] (L\level_MM2_M3) {};
    
    \node[model type, label=190:{$\mathcal{M}_3$}] at (\level*3.45*\modeltypesize+0.85*\modeltypesize, -1.35*\modeltypesize) (L\level_MM3) {};
    \node[model, above=0.25*\modeltypesize of L\level_MM3.center, anchor=center] (L\level_MM3_M1) {};
    \node[model, below left=0.25*\modeltypesize and 0.1*\modeltypesize of L\level_MM3.center, anchor=center] (L\level_MM3_M2) {};
    \node[model, below right=0.05*\modeltypesize and 0.25*\modeltypesize of L\level_MM3.center, anchor=center] (L\level_MM3_M3) {};
}

% CONSISTENCY L1
\draw[consistency relation] (L1_MM1_M1) -- (L1_MM2_M1);
\draw[consistency relation] ($(L1_MM1_M1)!0.13!(L1_MM2_M1)$) -- (L1_MM3_M2);
\draw[consistency relation] (L1_MM1_M2) -- (L1_MM2_M2);
\draw[consistency relation] ($(L1_MM1_M2)!0.5!(L1_MM2_M2)$) -- (L1_MM3_M1);

% CONSISTENCY L2
\draw[consistency relation] (L2_MM1_M1) -- (L2_MM2_M1);
\draw[consistency relation] (L2_MM1_M1) -- (L2_MM3_M2);
\draw[consistency relation, dashed] (L2_MM2_M1) -- (L2_MM3_M2);
\draw[consistency relation] (L2_MM1_M2) -- (L2_MM2_M2);
\draw[consistency relation] (L2_MM1_M2) -- (L2_MM3_M1);
\draw[consistency relation] (L2_MM2_M2) -- (L2_MM3_M1);
%\draw[consistency relation, dash dot] (L2_MM2_M1) -- (L2_MM3_M3);

% CONSISTENCY L3
%\draw[consistency relation] (L3_MM1_M1) -- (L3_MM2_M1);
%\draw[consistency relation] (L3_MM1_M1) -- (L3_MM3_M2);
%\draw[consistency relation] (L3_MM2_M1) -- (L3_MM3_M2);
%\draw[consistency relation] (L3_MM1_M2) -- (L3_MM2_M2);
%\draw[consistency relation] (L3_MM1_M2) -- (L3_MM3_M1);
%\draw[consistency relation] (L3_MM2_M2) -- (L3_MM3_M1);

% USER CHANGE L3
\node[model, user changed element, above left=0.25*\modeltypesize and 0.2*\modeltypesize of L3_MM1.center, anchor=center] (L3_MM1_M1) {};
\node[model, user changed element, above right=0.25*\modeltypesize and 0.2*\modeltypesize of L3_MM2.center, anchor=center] (L3_MM2_M1) {};
\node[model, user changed element, below left=0.25*\modeltypesize and 0.1*\modeltypesize of L3_MM3.center, anchor=center] (L3_MM3_M2) {};
\draw[-latex, user changed element] (L3_MM1_M1) -- node[legend, above=0.2em] {$\Delta$} (L3_MM1_M2);

% CONSISTENCY PRESERVATION L3
\node[model, consistency changed element, right=0.3*\modeltypesize of L3_MM1.center, anchor=center] (L3_MM1_M2) {};
\node[model, consistency changed element, left=0.3*\modeltypesize of L3_MM2.center, anchor=center] (L3_MM2_M2) {};
%\node[model, consistency changed element, above=0.25*\modeltypesize of L3_MM3.center, anchor=center] (L3_MM3_M1) {};
\node[model, consistency changed element, below right=0.05*\modeltypesize and 0.25*\modeltypesize of L3_MM3.center, anchor=center] (L3MM3_M3) {};
\draw[-latex, consistency changed element] (L3_MM2_M1) -- node[legend, below right=-0.3em] {$\Delta$} (L3_MM2_M2);
%\draw[-latex, consistency changed element] (L3_MM3_M2) -- node[legend, below right=0 and -0.3em] {$\Delta$} (L3_MM3_M1);
\draw[-latex, consistency changed element] (L3_MM3_M2) -- node[legend, below] {$\Delta$} (L3_MM3_M3);
\draw[-latex, dashed, consistency changed element, thin] ($(L3_MM1_M1)!0.4!(L3_MM1_M2)$) to[bend left=15] node[legend, above] {$\mathit{CPS}_{\mathit{CS}_{1,2}}$} ($(L3_MM2_M1)!0.4!(L3_MM2_M2)$);
%\draw[-latex, dashed, consistency changed element, very thin] ($(L3_MM1_M1)!0.4!(L3_MM1_M2)$) to[bend right=15] ($(L3_MM3_M2)!0.4!(L3_MM3_M1)$);
\draw[-latex, dashed, consistency changed element, very thin] ($(L3_MM1_M1)!0.4!(L3_MM1_M2)$) to[bend right=15] node[legend, below left=0.5em and -0.5em] {$\mathit{CPS}_{\mathit{CS}_{1,3}}$} ($(L3_MM3_M2)!0.4!(L3_MM3_M3)$);

% CS LABELS
\node[consistency related element, above left=1.5em and -0.1em of L1_MM3.north, anchor=east, font=\footnotesize] {$\mathit{CS}$};
\node[consistency related element, above left=0.1em and 1.6em of L2_MM3, anchor=center, font=\footnotesize] {$\mathit{CS}_{1,3}$};
\node[consistency related element, above right=0.4em and 1.2em of L2_MM3, anchor=center, font=\footnotesize] {$\mathit{CS}_{2,3}$};
\node[consistency related element, above right=0.8em and 0.25*\modeltypesize of L2_MM1.east, anchor=south, font=\footnotesize] {$\mathit{CS}_{1,2}$};
%\node[consistency related element, above left=0.4em and 1.8em of L3_MM3, anchor=center, font=\footnotefont] {$CS_{1,3}$};
%\node[consistency related element, above right=0.4em and 1.4em of L3_MM3, anchor=center, font=\footnotefont] {$CS_{2,3}$};
%\node[consistency related element, above right=0.8em and 0.25*\modeltypesize of L2_MM1.east, anchor=south, font=\footnotefont] {$CS_{1,2}$};

% LEVEL LABELS
\node[anchor=north] (level1_label) at ([yshift=1.15*\modeltypesize]$(L1_MM1)!0.5!(L1_MM2)$) {Level 1: \emph{Global}};
\node[anchor=north] (level2_label) at ([yshift=1.15*\modeltypesize]$(L2_MM1)!0.5!(L2_MM2)$) {Level 2: \emph{Modularization}};
\node[anchor=north] (level3_label) at ([yshift=1.15*\modeltypesize]$(L3_MM1)!0.5!(L3_MM2)$) {Level 3: \emph{Operationalization}};

\coordinate (legend_anchor) at ([yshift=-0.2*\modeltypesize]L2_MM3.south);

\node[matrix, left=-0.2em of legend_anchor, anchor=north east, outer sep=0em, column sep=0.15em, row sep=-0.4em] (legend_left) {
    \node[model, anchor=center] (model_legend) {}; &
    \node[legend, anchor=west] {model of model type $\mathcal{M}_x$}; \\
    %
    \node[model, user changed element, anchor=center] (user_changed_model_legend)  {}; &
    \node[legend, anchor=west] {consistent models before user change}; \\
    %
    \node[model, consistency changed element, anchor=center] (consistency_preserved_model_legend)  {}; &
    \node[legend, anchor=west] {consistent models after consistency preservation}; \\
};

\node[matrix, left=0.2em of legend_anchor, anchor=north west, column sep=0.15em, row sep=-0.4em] (legend_right) {
    \draw[consistency relation] (0,0.25em) -- (1em,0.25em);
    \draw[consistency relation] (0.5em,0.25em) -- (0.5em,-0.2em); &
    \node[legend, consistency related element, anchor=west] {element of consistency specification}; \\% $\mathit{CS}$}; \\
    % 
    \draw[-latex, user changed element] (0,0) -- (1em,0); &
    \node[legend, user changed element, anchor=west] {user change introducing inconsistency}; \\
    %
    \draw[-latex, consistency changed element] (0,0) -- (1em,0); &
    \node[legend, consistency changed element, anchor=west, align=left] {execution of consistency preservation specification};\\ % $\mathit{CPS}_{\mathit{CS}}$}; \\
};
\coordinate (legend_left_dummy) at ([xshift=-0.3em]legend_left.west);
\node[draw=darkgray, inner sep=0em, fit=(legend_left_dummy)(legend_left)(legend_right)] {};

\end{tikzpicture}

    \caption{Examples for Mistakes on Different Specification Levels}
    \label{fig:errors:mistakes_specification_levels}
\end{figure}

\subsubsection{Modularization Level}
When developers modularize the global consistency specification by defining binary consistency specifications, these modular specifications can be non-compliant with the global one. 
Two kinds of mistakes, similar to those at the global level, can be distinguished, regarding compliance of modular and global specifications. %, but regarding compliance of modular and global specifications rather than between the global specification and the actual notion of consistency.
First, modular consistency specifications can be incomplete (\emph{underspecified}), so that there are global constraints which are not covered by them. 
The modular consistency specifications $\mathit{CS}_{1,2}$, $\mathit{CS}_{2,3}$ and $\mathit{CS}_{1,3}$ in \autoref{fig:errors:mistakes_specification_levels} are incomplete iff
%For three model types $\mathcal{M}_1, \mathcal{M}_2$ and $\mathcal{M}_3$ with a global consistency specification $\mathit{CS}$, the binary specifications $\mathit{CS}_{1,2}$, $\mathit{CS}_{2,3}$ and $\mathit{CS}_{1,3}$, as depicted in \autoref{fig:levels_overview} are underspecified iff 
%\todoHeiko{Hier die Grafik aus dem Level-Kapitel übernehmen}
\begin{align*}
    & \exists M_1, M_2, M_3 : \\
    & \hspace{1em} (M_1, M_2) \in \mathit{CS}_{1,2} \land (M_2, M_3) \in \mathit{CS}_{2,3} \land (M_1, M_3) \in \mathit{CS}_{1,3} \land (M_1, M_2, M_3) \not\in \mathit{CS}
\end{align*}
This finally leads to \emph{false positives} when investigating whether a given tuple of models is consistent regarding the global specification. %as actually inconsistent models (regarding the global specification) are identified as consistent. %investigating whether a given set of models is consistent regarding the global specification or not, because actually inconsistent models regarding $\mathit{CS}$ are consistent according to all modular relations $\mathit{CS}_{1,2}$, $\mathit{CS}_{2,3}$ and $\mathit{CS}_{1,3}$.
%This finally leads to \emph{false positives} when investigating whether a given set of models is consistent regarding the global specification or not, because actually inconsistent models regarding $CS$ are consistent according to all modular relations $\mathit{CS}_{1,2}$, $\mathit{CS}_{2,3}$ and $\mathit{CS}_{1,3}$.
%\todoHeiko{Das folgende eher zu Avoidance Strategy? Überlapp vorhanden?}
%Such incomplete specifications can especially occur if constraints between two types of models are not expressed at all (so the consistency specification covers all model pairs), but are only transitively defined over two or more other relations. 
%For example, if $\mathit{CS}_{1,3}$ shall be omitted and transitively expressed across $\mathit{CS}_{1,2}$ and $\mathit{CS}_{2_3}$, the following must hold:
% \begin{align*}
%     & \forall M_1 \in \mathcal{M}_1 : \forall M_2 \in \mathcal{M}_2 : \forall M_3 \in \mathcal{M}_3 : \\
%     & \hspace{1em} \mathit{CS}(M_1, M_2, M_3) \iff \mathit{CS}_{1,2}(M_1, M_2) \land \mathit{CS}_{2,3}(M_2, M_3)
% \end{align*}
%\begin{align*}
    %& \forall M_1, M_2, M_3 : (M_1, M_2, M_3) \in \mathit{CS} \Leftrightarrow (M_1, M_2) \in \mathit{CS}_{1,2} \land (M_2, M_3) \in \mathit{CS}_{2,3}
%\end{align*}
%If this transitive relation misses or is even unable to express certain direct constraints, inconsistent models would be idenitified as consistent. %\todoHeiko{Das transitive muss man wohl an einem Beispiel erklären, am besten Ref. zu Intro}
Modular consistency specifications cannot only be incomplete because of an actual specification mistake, but also because of $n$-ary relations on the global level that cannot be expressed by a set of binary relations.
We excluded that case by our assumption made in \autoref{chap:properties:levels}, as otherwise a modularization into binary relations would not be possible at all.
If such cases have to be supported, the modularization would have to be extended to also consider $n$-ary relations.

Second, a modular specification can be too restricted (\emph{overspecified}) regarding the global consistency specification if additional constraints are added. 
The modular consistency specifications in \autoref{fig:errors:mistakes_specification_levels} are overspecified iff
\begin{align*}
    & \exists M_1, M_2, M_3 : \\
    & \hspace{1em} (M_1, M_2, M_3) \in \mathit{CS} \land \big[ (M_1, M_2) \not\in \mathit{CS}_{1,2} \lor (M_2, M_3) \not\in \mathit{CS}_{2,3} \lor (M_1, M_3) \not\in \mathit{CS}_{1,3} \big]
\end{align*}
In \autoref{fig:errors:mistakes_specification_levels}, omitting the dashed relation in $\mathit{CS}_{2,3}$ would lead to such an overspecifiation.
Overspecifications lead to additional constraints regarding the global specification, but also, and more severe, to contradicting constraints regarding other modular specifications.
In case of contradictions, the modular consistency specifications cannot be fulfilled at the same time.
In such a case, the graph of consistency relations %, as shown in \autoref{fig:mistakes_specification_levels}, 
would contain no cylces, i.e. sets of models that are consistent to each other.
We have discussed an example for such contradicting specifications %in the motivating example 
in \autoref{chap:properties:levels}, where constraints for transferring an employee name contradicted. % contains contradicting constraints for transferring the name. % to other types of models.
%In this case, when several binary specifications are combined to keep multiple models consistency, the resulting fault are incompatible binary specifications in the sense that different relations cannot hold at the same time because they are based on different global consistency specifications.
Such mistakes lead to \emph{false negatives} as actually consistent models (regarding the global specification) are identified as inconsistent. %, when investigating whether a given set of models is consistent regarding the global specification or not, because actually consistent models regarding $\mathit{CS}$ are not consistent according to the modular relations $\mathit{CS}_{1,2}$, $\mathit{CS}_{2,3}$ and $\mathit{CS}_{1,3}$.

%\begin{itemize}
    %\item inadequate structure
    %\item missing knowledge about other modular relation
%\end{itemize}

\subsubsection{Operationalization Level}
The types of mistakes that can be made at the operationalization level are different from those at the other levels, because this level does not concern the definition of consistency specifications (\autoref{def:consistency_specification}), but of consistency \emph{preservation} specifications (\autoref{def:consistency_preservation_specification}).
Such specifications are faulty if no composition of them exists that returns a consistent tuple of models for each possible change. % it does not lead to a consistent state after making modifications to a consistent tuple of models.
In \autoref{fig:errors:mistakes_specification_levels}, an exemplary application of a single consistency preservation specification is depicted that leads to models that are not consistent according to the (global and modular) consistency specifications.
%If no concatenation of CPSs exists that finally returns a consistent set of models for each possible change, the specifications are faulty.
Let $\mathcal{CPS}$ be a set of consistency preservation specifications  %, e.g. $\mathcal{CPS} := \{\mathit{CPS}_{1}, \ldots, \mathit{CPS}_{m}\}$
for the binary consistency specifications $\mathcal{CS}$ % := \{\mathit{CS}_{1,2}, \mathit{CS}_{2,3}, \mathit{CS}_{1,3}\}$ %(where there can be more than one consistency preservation specifications for each consistency specification) 
%in \autoref{fig:mistakes_specification_levels}. %, metamodels $\mathcal{M}_0, \ldots, \mathcal{M}_n$, 
and
let $\mathfrak{M}_{\mathcal{CS}}$ be the set of model tuples that are consistent regarding $\mathcal{CS}$ (cf. \autoref{chap:properties:terminology}). 
The consistency preservation specifications are faulty iff
% \begin{align*}
%     & \exists M_0, M'_0 \in \mathcal{M}_0, M_1, M'_1 \in \mathcal{M}_1, M_2, M'_2 \in \mathcal{M}_2 : \forall \mathit{CPS}_0, \ldots, \mathit{CPS}_k \in \mathcal{CPS} : \\
%     & \hspace{1em} ((M_0, M''_0), (M_1, M''_1), (M_2, M''_2)) = \mathit{CPS}_0 \circ \dots \circ \mathit{CPS}_k((M_0, M'_0), (M_1, M'_1), (M_2, M'_2)) \\
%     & \hspace{1em} \Rightarrow \exists \mathit{CS}_{i,j} \in \mathcal{CS} : \neg \mathit{CS}_{i,j}(M''_i, M''_j)
% %    & \exists M_0, M'_0 \in \mathcal{M}_0, \ldots M_n, M'_n \in \mathcal{M}_n : \nexists \mathit{CPS}_0, \ldots, \mathit{CPS}_k \in \mathcal{CPS} : \\
% %    & \hspace{1em} ((M_0, M''_0), \dots, (M_n, M''n)) := \mathit{CPS}_0 \circ \dots \circ \mathit{CPS}_k((M_0, M'_0), \dots, (M_n, M'_n)) \\
% %    & \hspace{1em} \land \exists \mathit{CS}_{i,j} \in \mathcal{CS} : \neg \mathit{CS}_{i,j}(M''_i, M''_j)
% \end{align*}
\begin{align*}
    & \exists (M_1, \dots, M_n) \in \mathfrak{M}_{\mathcal{CS}}, (M'_1, \dots, M'_n) \in \mathcal{M}_1 \times \dots \times \mathcal{M}_n: \forall \mathit{CPS}_1, \ldots, \mathit{CPS}_k \in \mathcal{CPS} : \\
    %& \exists M_1, M'_1 \in \mathcal{M}_1, \dots, M_n, M'_n \in \mathcal{M}_n : \forall \mathit{CPS}_1, \ldots, \mathit{CPS}_k \in \mathcal{CPS} : \\
    & \hspace{1em} \mathit{CPS}_1 \circ \dots \circ \mathit{CPS}_k \big((M_1, M'_1), \dots, (M_n, M'_n) \big) = \big( (M_1, M''_1), \dots, (M_n, M''_n) \big)\\
    & \hspace{2em} \land \exists \mathit{CS}_{i,j} \in \mathcal{CS} : (M''_i, M''_j) \notin \mathit{CS}_{i,j}
%    & \exists M_0, M'_0 \in \mathcal{M}_0, \ldots M_n, M'_n \in \mathcal{M}_n : \nexists \mathit{CPS}_0, \ldots, \mathit{CPS}_k \in \mathcal{CPS} : \\
%    & \hspace{1em} ((M_0, M''_0), \dots, (M_n, M''n)) := \mathit{CPS}_0 \circ \dots \circ \mathit{CPS}_k((M_0, M'_0), \dots, (M_n, M'_n)) \\
%    & \hspace{1em} \land \exists \mathit{CS}_{i,j} \in \mathcal{CS} : \neg \mathit{CS}_{i,j}(M''_i, M''_j)
\end{align*}
%\todoHeiko{Das ist nicht so schön, weil nicht klar ist, welche CS to welcher CPS gehört und so}
%This means that there can be changes for which no execution of consistency preservation specifications is able to properly restore consistency. 

In practice, mistakes at the operationalization level occur due to missing identification of equal elements in different consistency preservation specifications. 
In our motivational example (\autoref{fig:properties:motivational_example}), %consider that an employee is created in the scheduling system for an employee created in the task management system after introducing one in the personnel management system.
consider that an employee is created in the personnel management system, transformed to the task management system and from that to the scheduling system.
The additional direct specification between personnel management and scheduling system has to consider the already created employee rather than instantiating a new one.
%We consider %this case and 
%options to avoid such problems in \autoref{sec:avoiding:matching}.
%In our example, if a class is created in Java after creating a UML class for a ADL component through appropriate consistency preservation specifications and the consistency preservation specification between ADL and Java also defines the creation of a class in Java, it is necessary that the already existing class is considered rather than creating a new class. We will consider this case and options to avoid such problems in \autoref{sec:avoiding:matching}.

% \begin{itemize}
%     \item unknown connection of elements in consistency specifications
%     \item Really, really make an example here, to distinguish from modularization level!!
% \end{itemize}

%In the following, we call all mistakes on modularization and operationalization level \emph{interoperability issues}, as they are all concerned with modularized specifications that have to interoperate. 


\subsection{Causal Chains}

\begin{figure}
    \centering
    \newcommand{\classdistance}{5em}
\newcommand{\classwidth}{5em}
\newcommand{\objectwidth}{6.3em}

% #1: height
% #2: width
% #3: positioning options
% #4: name
\newcommand{\modeltypebgtextabove}[4]{\node[fill=lightgray!20, text depth=#1, minimum width=#2, #3] {\small \textit{#4}};}
\newcommand{\modeltypebgtextbelow}[4]{\node[fill=lightgray!20, text height=#1, minimum width=#2, #3] {\small \textit{#4}};}

\usetikzlibrary{positioning,arrows.meta,shapes.misc,matrix}

\begin{tikzpicture}[
    consistency preservation/.style={-latex, consistency changed element, font=\small},
    user change/.style={-latex, user changed element, font=\small},
    legend/.style={font=\footnotesize},
    mininode/.style={inner sep=.25em},
]
% requires styles: consistency related element, user changed element, consistency changed element
% requires tikzuml

\pgfdeclarelayer{bg}
\pgfsetlayers{bg,main}

% METAMODEL

\umlclassvarwidth{mm_task_employee}{}{Employee}{
name\\
\dots
}{\classwidth}    

\umlclassvarwidth[, below left=0.4*\classdistance and 2.5*\classdistance of mm_task_employee.north, anchor=north]{mm_personnel_employee}{}{
Employee}{
firstName\\
lastName\\
\dots
}{\classwidth}

\umlclassvarwidth[, right=5*\classdistance of mm_personnel_employee.north, anchor=north]{mm_scheduling_employee}{}{
Employee}{
name\\
\dots
}{\classwidth}

\draw[consistency relation] (mm_personnel_employee.north) |- node[above, pos=0.6] {name = firstName + "\textvisiblespace" + lastName} ([yshift=-0.7em]mm_task_employee.north west);
\draw[consistency relation]([yshift=-0.7em]mm_task_employee.north east) -| node[above, pos=0.25] {name = name} (mm_scheduling_employee.north);
\draw[consistency relation] ([yshift=-3em]mm_personnel_employee.north east) -- node[below, pos=0.5, align=left] {
\textit{Opt. 1}: name = lastName + ",\textvisiblespace" + firstname\\
\textit{Opt. 2}: name = firstName + "\textvisiblespace" + lastname
} ([yshift=-3em]mm_scheduling_employee.north west);

% LEGEND
\coordinate (legend_anchor) at ([yshift=-1.3*\classdistance]mm_task_employee);

\node[draw=darkgray, matrix, legend, nodes=mininode, below=0em of legend_anchor, anchor=north, outer sep=0, inner sep=0.4em, column sep=0.2em, row sep=-0.2em] (legend) {
    \draw[consistency relation] (0,0) -- (1.4em,0); &
    \node[consistency related element, anchor=west] {consistency constraint}; \\
    %
    \draw[-latex, user changed element] (0,0) -- (1.4em,0); &
    \node[user changed element, anchor=west] {user change}; \\
    %
    \draw[-latex, consistency changed element] (0,0) -- (1.4em, 0); &
    \node[consistency changed element, anchor=west, align=left] (legend_cpr_label) {consistency preservation};\\
};

% \draw[consistency relation] (legend_anchor) -- ([xshift=1.5em]legend_anchor);
% \node[legend, consistency related element, anchor=west] at ([xshift=1.5em]legend_anchor) {consistency constraint};

% \draw[-latex, user changed element] ([yshift=-1em]legend_anchor) -- ([yshift=-1em, xshift=1.5em]legend_anchor);
% \node[legend, user changed element, anchor=west] at ([yshift=-1em, xshift=1.5em]legend_anchor) {user change};

% \draw[-latex, consistency changed element] ([yshift=-2em]legend_anchor) -- ([yshift=-2em, xshift=1.5em]legend_anchor);
% \node[legend, consistency changed element, anchor=north west, align=left] (legend_cpr_label) at ([yshift=-1.3em, xshift=1.5em]legend_anchor) {consistency preservation};

%\coordinate (legend_upper_left) at ([xshift=-0.7em, yshift=0.7em]legend_anchor);
%\begin{pgfonlayer}{bg}
%    \node[fill=lightgray!30, fit=(legend_upper_left)(legend_cpr_label)] {};
%\end{pgfonlayer}


% LEVEL 2 ERROR
\coordinate (failure_l2_anchor) at ([xshift=-2.15*\classdistance, yshift=-2.65*\classdistance]mm_task_employee.north); %([xshift=4*\classdistance]mm_task_employee.north);

\umlobjectvarwidth[, consistency changed element, fill=white, below=0 of failure_l2_anchor, anchor=north]{l2_task_employee}{}{
: Employee}{
name="Alice Do"
}{\objectwidth}

\umlobjectvarwidth[, user changed element, fill=white, below left=1.2*\classdistance and 0.75*\classdistance of l2_task_employee.north, anchor=north] {l2_personnel_employee}{}{
: Employee}{
firstName="Alice"
lastName="Do"
}{\objectwidth}

\umlobjectvarwidth[, consistency changed element, fill=white, right=1.9*\classdistance of l2_personnel_employee.north, anchor=north] {l2_scheduling_employee}{}{
: Employee}{
name="Do, Alice"
}{\objectwidth}

\umlhuman{l2_human}{at ([xshift=1em, yshift=4.5em]l2_personnel_employee.north west)}{user changed element}{}{0.5}
\draw[user change] ([xshift=1em,yshift=3.3em]l2_personnel_employee.north west) -- node[above right=-0.7em and -0.2em, align=center] {1.\\ \tiny «create»} ([xshift=1em]l2_personnel_employee.north west);
\draw[consistency preservation] ([xshift=1em]l2_personnel_employee.north) -- node[left=0.2em] {2.} ([xshift=2.5em]l2_task_employee.south west);
\draw[consistency preservation] ([yshift=-2.5em]l2_personnel_employee.north east) -- node[below] {3.} ([yshift=-2.5em]l2_scheduling_employee.north west);

\draw[consistency preservation, dashed] ([xshift=-2em]l2_task_employee.south east) -- node[above right=0.3em and 0.2em, anchor=west] {4. \large \Lightning} ([xshift=-1em]l2_scheduling_employee.north);
\draw[consistency preservation, dashed] ([yshift=-1.5em]l2_scheduling_employee.north west) -- node[above] {5.}  ([yshift=-1.5em]l2_personnel_employee.north east);
\draw[consistency preservation, dashed] ([xshift=2em]l2_personnel_employee.north) -- node[right=0.2em] {6.} ([xshift=3.5em]l2_task_employee.south west);


% LEVEL 3 ERROR
\coordinate (failure_l3_anchor) at ([xshift=4.3*\classdistance]failure_l2_anchor);

\umlobjectvarwidth[, consistency changed element, fill=white, below=0 of failure_l3_anchor, anchor=north]{l3_task_employee}{}{
: Employee}{
name="Alice Do"
}{\objectwidth}

\umlobjectvarwidth[, user changed element, fill=white, below left=1.2*\classdistance and 1.15*\classdistance of l3_task_employee.north, anchor=north] {l3_personnel_employee}{}{
: Employee}{
firstName="Alice"
lastName="Do"
}{\objectwidth}

\umlobjectvarwidth[, consistency changed element, fill=white, below right=0.5em and 1.9*\classdistance of l3_personnel_employee.center, anchor=south] {l3_scheduling_employee}{}{
: Employee}{
name="Alice Do"
}{\objectwidth}

\umlobjectvarwidth[, consistency changed element, fill=white, below right=1em and 1.9*\classdistance of l3_personnel_employee.center, anchor=north] {l3_scheduling_employee_duplicate}{}{
: Employee}{
name="Alice Do"
}{\objectwidth}

\umlhuman{l3_human}{at ([xshift=1em, yshift=4.5em]l3_personnel_employee.north west)}{user changed element}{}{0.5}
\draw[user change] ([xshift=1em,yshift=3.3em]l3_personnel_employee.north west) -- node[above right=-0.7em and -0.2em, align=center] {1.\\ \tiny «create»} ([xshift=1em]l3_personnel_employee.north west);
\draw[consistency preservation] ([xshift=1em]l3_personnel_employee.north) -- node[left=0.3em] {2.} ([xshift=2em]l3_task_employee.south west);
\draw[consistency preservation] ([xshift=-2em]l3_task_employee.south east) -- node[above right=-0.5em and 0.2em] {3.} ([xshift=-1em]l3_scheduling_employee.north);
\draw[consistency preservation] ([yshift=1em]l3_personnel_employee.south east) -- node[below=0.2em] {4.} ([yshift=-1em]l3_scheduling_employee_duplicate.north west);

\begin{pgfonlayer}{bg}
    \modeltypebgtextabove{4.8em}{1.5*\classwidth}{above=1.7em of mm_task_employee.north, anchor=north}{Task Management}
    \modeltypebgtextbelow{6.4em}{1.5*\classwidth}{below=1.7em of mm_personnel_employee.south, anchor=south}{Personnel Data\vphantom{g}}
    \modeltypebgtextbelow{5.2em}{1.5*\classwidth}{below=1.7em of mm_scheduling_employee.south, anchor=south}{Scheduling}

    \modeltypebgtextabove{4.1em}{1.27*\objectwidth}{above=1.7em of l2_task_employee.north, anchor=north}{Task Management}
    \modeltypebgtextbelow{5.6em}{1.27*\objectwidth}{below=1.7em of l2_personnel_employee.south, anchor=south}{Personnel Data\vphantom{g}}
    \modeltypebgtextbelow{4.8em}{1.27*\objectwidth}{below=1.7em of l2_scheduling_employee.south, anchor=south}{Scheduling}
    
    \modeltypebgtextabove{4.1em}{1.27*\objectwidth}{above=1.7em of l3_task_employee.north, anchor=north}{Task Management}
    \modeltypebgtextbelow{5.6em}{1.27*\objectwidth}{below=1.7em of l3_personnel_employee.south, anchor=south}{Personnel Data\vphantom{g}}
    \modeltypebgtextbelow{8.3em}{1.27*\objectwidth}{below=1.7em of l3_scheduling_employee_duplicate.south, anchor=south}{Scheduling}
\end{pgfonlayer}


% LABELS AND LINES

\coordinate (y_verticalline) at ([yshift=2.4*\classdistance]l2_personnel_employee.west);
\draw[draw=gray, very thin] ([xshift=-0.05*\objectwidth]y_verticalline-|l2_personnel_employee.west) -- node[below=0.2em] (l2_label) {Level 2 Mistake \textit{(Opt. 1)}} ([xshift=-1em]y_verticalline-|legend.west);

%\node[below=0.4em, anchor=south west, align=left, font=\mediumfont] (l2_label) {Level 2 Mistake \textit{(Opt. 1)}}; 

%{Consistency Preservation with
%\node[above=2em of failure_l2_anchor, align=center, font=\smallerfont {Consistency Preservation with \textit{Opt. 1}\\ \textit{(Error on Level 2)}};

\draw[draw=gray, very thin] ([xshift=0.05*\objectwidth]y_verticalline-|l3_scheduling_employee.east) -- node[below=0.2em] (l3_label) {Level 3 Mistake \textit{(Opt. 2)}} ([xshift=1em]y_verticalline-|legend.east);
%\draw[dashed] ([yshift=2.4*\classdistance]l2_personnel_employee.west) -- node[below=0.2em, font=\mediumfont] (l2_label) {Level 2 Mistake \textit{(Opt. 1)}} ([xshift=-1em, yshift=2.4*\classdistance]l2_personnel_employee.west-|legend.west);

%\node[anchor=south east, align=left, font=\mediumfont] (l3_label) at (l2_label.south-|l3_scheduling_employee.east) {Level 3 Mistake \textit{(Opt. 2)}};
%\node[above=2em of failure_l3_anchor, align=center, font=\smallerfont] {Consistency Preservation with \textit{Opt. 2}\\ \textit{(Error on Level 3)}};
%\draw[dashed] ([yshift=0.4em]l3_scheduling_employee.east|-l3_label.north) -- ([xshift=1em, yshift=0.4em]l3_label.north-|legend.east);

\draw[draw=gray, very thin] ([yshift=-1em]legend.south) -- ([yshift=-2.8*\classdistance]legend.south);

\end{tikzpicture}


%    \includegraphics[angle=270, width=\textwidth]{figures/mistakes_examples_employee.pdf}
    \caption{Consistency Constraints on Metamodel Extract (top), Failure due to Mistake on Modularization Level (left), Failure due to Mistake on Operationalization Level (right)}
    \label{fig:errors:mistake_effects_example}
\end{figure}

%We associated the mistakes presented in the previous section with the specification level they can occur on. Additionally, we summarized potential failures that can occur when executing final consistency preservation specification in the section before.
Although all failures occur during operationalization, the mistakes that lead to them can also be made at a higher specification level, such as the modularization or global level.
More importantly, each type of failure can be traced back to specific types of mistakes, or, vice versa, specific mistakes lead to specific kinds of failures.
\autoref{fig:errors:mistake_effects_example} shows extracts of the three metamodels from our motivation, as well as consistency constraints between them.
There are two options for a constraint between personnel data and scheduling system.
The first option is contradictory to the one defined between personnel data and task management system, as already discussed in \autoref{chap:properties:levels}.
This demonstrates that contradictory constraints are a typical fault that can result from contradicting modular knowledge, when different persons define such constraints independently.
If, nevertheless, such a contradictory consistency specification is operationalized to a consistency preservation specification, the propagation of changes may never terminate.
This is shown in the left scenario in \autoref{fig:errors:mistake_effects_example}, where
%Due to the contradicting constraints, 
the name is replaced repeatedly in an \emph{alternating loop} as indicated by the dashed arrows.

If no mistakes are made on the modularization level, so that no contradictions exist, %which especially means that the consistency specifications are free of contradictions, 
missing matching of equal elements in the consistency preservation specifications can still lead to duplicate element instantiations.
With the second option for the constraint in \autoref{fig:errors:mistake_effects_example}, %no contradicting constraints and thus 
no mistakes on modularization level exist.
However, a missing matching of elements %in the consistency preservation specification 
can lead to the situation shown in the right scenario of \autoref{fig:errors:mistake_effects_example}, in which two employees are instantiated across different transformation paths.
%We also demonstrated in the example that missing matching of equal elements in the consistency preservation specifications can lead to duplicate instantiations of elements.

These were two of several causal chains for mistakes and faults to resulting failures.
We give a full overview of those dependencies in \autoref{fig:errors:categorization}.
Missing constraints lead to deterministic inconsistencies, because such inconsistencies are not modelled and thus resolved.
Additional consistency constraints do not lead to any actual failures, but reduce the set of consistent models. 
The only consequence is that consistency preservation does not consider models that would actually be consistent.
Contradicting constraints, which can arise from a faulty modularization, are more severe, as we have seen in the example:
They can either lead to non-deterministic inconsistencies, e.g., depending on the execution order of consistency preservation specifications, or to loops that alternate or diverge values.
Finally, the missing element matching at the operationalization level can lead to multiple instantiations, as we have seen in the example, or multiple insertions. %, if elements are added to a multi-valued reference multiple times.

%\todoHeiko{Tun wir das wirklich? Oder nur ein Level?}
%In the following, we discuss strategies to avoid mistakes at the different levels.
%Afterwards, we evaluate whether our identified categorizes of mistakes, faults and failures and their dependencies are actually correct.

%Categorize the detected Causes/Mistakes into three categories, which map to the steps identified in the first subsection. Two categories have to be resolved by user and, especially, are only resolvable in the moment when concrete transformations are combined (explain why!). One category can be solved by applying appropriate pattern, explained in the next section. Each category resolution is an assumption of the next (e.g. pattern matching does not make any sense when transformations are incompatible or at least the failures than can occur may differ).