\section{A Formal Notion of Compatibility}
\label{chap:compatibility:formal_notion}

\mnote{Formal notion of fine-grained consistency and compatibility}
In this section, we precisely define our up to now informally introduced notion of \emph{compatibility}.
For that, we use the fine-grained notion of consistency and defining relations as proposed in \autoref{chap:correctness:finegrained}.
We discuss implicit relations, which are induced by a set of consistency relations, such as transitive relations, and,
finally, derive a compatibility notion from the consistency formalization and its pursued perception.
The contents of this and the remaining sections of this chapter are mostly, in parts literally, taken from our published article on proving compatibility~\owncite{klare2020compatibility-report}.


\subsection{Implicit Consistency Relations}

\mnote{Concatenation of consistency relations}
Considering sets of consistency relations, as they are implicitly defined by the set of transformations in a transformation network, their combination is of especial interest.
Each set of consistency relations defines relations between two sets of classes, but also implies further \emph{transitive} consistency relations.
Having one relation between classes $\class{A}{}$ and $\class{B}{}$ and one between $\class{B}{}$ and $\class{C}{}$ implies an additional relation between $A$ and $C$.
We define a notion for the concatenation of relations that implies such transitive relations, which are supposed to reflect the consistency constraints introduced by the concatenated relations.
This especially means that models should always be consistent to a concatenation of consistency relations if they are consistent to each of the concatenated relations, as otherwise the concatenation would introduce additional constraints.
To achieve this, the following definition makes appropriate restrictions to the derived consistency relation pairs.

\todoLater{Actually, a concatenation may also consider that two or more relations are concatenated to a single one. I.e., CR1 could map something to A and B, and CR2 could map A to something and CR3 could map B to something. Then there could be a combination of all of them. In fact, each pair of consistency relations between the same metamodels can be combined to "larger" relation that then may be concatenated to other relations. Such a pair could even be a pair of a relation with itself, like if a relation maps on element to two of the same class and another relation then maps one element of the class to another. 
In summary, our notion of transitivity has to consider that concatenation may not only be sequences, but acyclic graphs.
Update: The interesting case is that CR1 requires elements for which at least two CR2 and CR3 require further elements. Then CR4 only requires elements for the ones required by CR2 and CR3. In our definition, on concatenation would consider CR4. }

\begin{definition}[Consistency Relations Concatenation] \label{def:relationconcatenation}
    Let $\consistencyrelation{CR}{1}$ and $\consistencyrelation{CR}{2}$ be two consistency relations. Their concatenation $ \consistencyrelation{CR}{1} \concat \consistencyrelation{CR}{2}$ is defined as follows:
    \begin{align*}
        &
        \consistencyrelation{CR}{1} \concat \consistencyrelation{CR}{2} \equalsperdefinition \setted{\tupled{\conditionelement{c}{l}, \conditionelement{c}{r}} \mid \\
        & \formulaskip 
        \exists
        \tupled{\conditionelement{c}{l}, \conditionelement{c}{r,1}} \in \consistencyrelation{CR}{1} : \exists
        \tupled{\conditionelement{c}{l,2}, \conditionelement{c}{r}} \in \consistencyrelation{CR}{2} : 
        \conditionelement{c}{l,2} \subseteq \conditionelement{c}{r,1}\\
        & \formulaskip
        \land \forall \tupled{\conditionelement{c}{l}, \conditionelement{c}{r,1}'} \in \consistencyrelation{CR}{1} : \exists \tupled{\conditionelement{c}{l,2}', \conditionelement{c}{r,2}'} \in \consistencyrelation{CR}{2} : \conditionelement{c}{l,2}' \subseteq \conditionelement{c}{r,1}'
        }
    \end{align*}
    with $\classtuple{C}{l,\consistencyrelation{CR}{}} = \classtuple{C}{l,\consistencyrelation{CR}{1}}$ and $\classtuple{C}{r,\consistencyrelation{CR}{}} = \classtuple{C}{r,\consistencyrelation{CR}{2}}$
\end{definition}

\mnote{Requirements for concatenation}
The concatenation of two consistency relations contains pairs of object tuples that are related across common elements in the right respectively left side of the consistency relation pairs.
Such a concatenation may be empty.
Two requirements ensure that all models considered consistent to the concatenation are also consistent to the single relations:
First, two consistency relation pairs of $\consistencyrelation{CR}{1}$ and $\consistencyrelation{CR}{2}$ are only combined if the left condition element of the consistency relation pair of $\consistencyrelation{CR}{2}$ is a subset of the right condition element of the consistency relation pair of $\consistencyrelation{CR}{1}$.
Only in that case the existence of the right condition element of the pair of $\consistencyrelation{CR}{1}$ in a model requires the existence of an according condition element in $\consistencyrelation{CR}{2}$.
Second, it is necessary that for all elements $\conditionelement{c}{r,1}'$ in the right side of $\consistencyrelation{CR}{1}$, which are considered consistent to a condition element $\conditionelement{c}{l}$, there must be a matching condition element, i.e., a subset of $\conditionelement{c}{r,1}'$, in the left condition of $\consistencyrelation{CR}{2}$.
If there was an element $\conditionelement{c}{r,1}'$ in the right side of $\consistencyrelation{CR}{1}$ for which the left side condition of $\consistencyrelation{CR}{2}$ does not contain a subset, the concatenation does not constrain consistency for the existence of $\conditionelement{c}{l}$.
Thus, without these restrictions the occurrence of $\conditionelement{c}{l}$ in a model tuple would not necessarily impose any consistency requirement by $\consistencyrelation{CR}{2}$.
We explain these two restrictions at an example.

\begin{figure}
    \centering
    \begin{subfigure}{\textwidth}
        \centering
        \newcommand{\hdistance}{11.3em}
\newcommand{\classwidth}{4.5em}
\newcommand{\internalvdistance}{1.7em}

\begin{tikzpicture}

% Person
\umlclassvarwidth{person}{}{Person\sameheight}{
name
}{\classwidth}

% Employee
\umlclassvarwidth[, right=\hdistance of person.north, anchor=north]{resident}{}{Resident\sameheight}{
name\\
street
}{\classwidth}

%Resident and Address
\umlclassvarwidth[, right=\hdistance of resident.north, anchor=north]{employee}{}{Employee\sameheight}{
name
}{\classwidth}

\umlclassvarwidth[, below=\internalvdistance of employee.south, anchor=north]{address}{}{Address\sameheight}{
street
}{\classwidth}

\umlassociationfromto{(employee) -- node[uml role end, pos=1, above left] {address} (address)}

% CONSISTENCY RELATIONS
\draw[directed consistency relation] (person.east) -- node[pos=0, above right] {$p$} node[pos=0.5, below] {$\consistencyrelation{CR}{1}$} node[pos=1, above left] {$r$} (person.east-|resident.west);
\draw[directed consistency relation] (employee.west-|resident.east) -- node[pos=0, above right] {$r$} node[pos=0.5, below, align=left] {$\consistencyrelation{CR}{2}$ / \\ $\consistencyrelation{CR}{2}'$} node[pos=1, above left] {$e$} (employee.west);
\draw[directed consistency relation] ($(employee.west)!0.2!(employee.west-|resident.east)$) |- node[pos=1, above left] {$a$} (address.west);

\node[consistency related element, below=5em of person.west, anchor=north west] {
$\begin{aligned}
    \consistencyrelation{CR}{1} =\; & \setted{\tupled{p,r} \mid \mathvariable{p.name} = \mathvariable{r.name}}\\[0.3em]
    \consistencyrelation{CR}{2} =\; & \setted{\tupled{r,(e,a)} \mid \mathvariable{r.name} = \mathvariable{e.name} \land \mathvariable{r.street} = \mathvariable{a.street}}\\
    \consistencyrelation{CR}{2}' =\; & \setted{\tupled{r,(e,a)} \mid \tupled{r,(e,a)} \in \consistencyrelation{CR}{2} \land \mathvariable{r.street} \neq \textnormal{\enquote{}}}
\end{aligned}$
};

\end{tikzpicture}
    \end{subfigure}

    \vspace{1em}
    \begin{subfigure}{\textwidth}
        \centering
        \newcommand{\hdistance}{(12.5em+0.45*\difftoafiveimage)}
\newcommand{\classwidth}{4.5em}
\newcommand{\internalvdistance}{1.7em}

\begin{tikzpicture}

% Person
\umlclassvarwidth{person}{}{Person\sameheight}{
name
}{\classwidth}

% Employee and Location
\umlclassvarwidth[, right=\hdistance of person.north, anchor=north]{resident}{}{Resident\sameheight}{
name
}{\classwidth}

\umlclassvarwidth[, below=\internalvdistance of resident.south, anchor=north]{location}{}{Location\sameheight}{
street
}{\classwidth}

\umlassociationfromto{(resident) -- node[uml role end, pos=1, above left] {address} (location)}

%Resident and Address
\umlclassvarwidth[, right=\hdistance of resident.north, anchor=north]{employee}{}{Employee\sameheight}{
name
}{\classwidth}

\umlclassvarwidth[, below=\internalvdistance of employee.south, anchor=north]{address}{}{Address\sameheight}{
street
}{\classwidth}

\umlassociationfromto{(employee) -- node[uml role end, pos=1, above left] {address} (address)}

% CONSISTENCY RELATIONS
\draw[directed consistency relation] (person.east) -- node[pos=0, above right] {$p$} node[pos=0.5, below] {$\consistencyrelation{CR}{3}$} node[pos=1, above left] {$r$} (person.east-|resident.west);
\draw[directed consistency relation] (resident.east) -- node[pos=0, above right] {$r$} node[pos=0.5, below, align=left] {$\consistencyrelation{CR}{4}$} node[pos=1, above left] {$e$} (employee.west);
\draw[directed consistency relation, -] ($(employee.west)!0.8!(employee.west-|resident.east)$) |- node[pos=1, above right] {$l$} (location.east);
\draw[directed consistency relation] ($(employee.west)!0.2!(employee.west-|resident.east)$) |- node[pos=1, above left] {$a$} (address.west);
\filldraw[consistency related element] ($(employee.west)!0.2!(employee.west-|resident.east)$) circle (0.15em);
\filldraw[consistency related element] ($(employee.west)!0.2!(employee.west-|resident.east)$) circle (0.15em);

\node[consistency related element, below=6.3em-\isafour*1.3em of person.west, anchor=north west] {
$\begin{aligned}
    \consistencyrelation{CR}{3} =\; & \setted{\tupled{p,r} \mid p.name = r.name}\\[0.3em]
    \consistencyrelation{CR}{4} =\; & \setted{\tupled{(r,l),(e,a)} \mid r.name = e.name \land l.street = a.street}
\end{aligned}$
};

\end{tikzpicture}
    \end{subfigure}
    %\includegraphics[width=\columnwidth]{figures/consistency_concatenation_example.png} 
    %\includegraphics[width=\columnwidth]{figures/concatenation_subset.png}
    \caption[Examples for consistency relation concatenation]{Two scenarios, each with two consistency relations: 
    Consistency relations $\consistencyrelation{CR}{1}$ and two options $\consistencyrelation{CR}{2}, \consistencyrelation{CR}{2}'$ with $\consistencyrelation{CR}{1} \concat \consistencyrelation{CR}{2} \neq \emptyset$ and $\consistencyrelation{CR}{1} \concat \consistencyrelation{CR}{2}' = \emptyset$, and consistency relations $\consistencyrelation{CR}{3}$ and $\consistencyrelation{CR}{4}$ with $\consistencyrelation{CR}{3} \concat \consistencyrelation{CR}{4} = \emptyset$ and $\consistencyrelation{CR}{4}^T \concat \consistencyrelation{CR}{3}^T \neq \emptyset$. Taken from~\owncite[Fig.~3]{klare2020compatibility-report}.}
    \label{fig:compatibility:concatenation_example}
\end{figure}

\begin{example}
\autoref{fig:compatibility:concatenation_example} extends the initial example (\autoref{fig:compatibility:three_persons_example_extended} on page \pageref{fig:compatibility:three_persons_example_extended}) with further classes in the consistency relations, such that they do not only relate single classes to each other.
It defines an address for employees and, in the second example, also a location for the addresses of residents, which are represented in additional classes.
Both examples contain consistency relations $\consistencyrelation{CR}{1}$ and $\consistencyrelation{CR}{3}$, respectively, between persons and residents, which define that for each person a resident with the same name has to exist.
The examples provide different options for the consistency relation between residents (with locations) and employees with addresses ($\consistencyrelation{CR}{2}, \consistencyrelation{CR}{2}', \consistencyrelation{CR}{4}$), which exemplify the necessity for the restrictions in \autoref{def:relationconcatenation}:
\begin{enumerate}
    \item $\consistencyrelation{CR}{1} \concat \consistencyrelation{CR}{2}$: 
$\consistencyrelation{CR}{2}$ requires for each resident an employee with the same name and an address with an arbitrary street name.
In consequence, $\consistencyrelation{CR}{1} \concat \consistencyrelation{CR}{2}$ relates each person to an employee having the same name and addresses with all possible street names.
All models that are consistent to the concatenation are also consistent to the single relations.
    \item $\consistencyrelation{CR}{1} \concat \consistencyrelation{CR}{2}'$: 
$\consistencyrelation{CR}{2}'$ is similar to $\consistencyrelation{CR}{2}$ but additionally requires that the street of a resident must not be empty. 
In consequence, for a resident with an empty address it is not required that an employee exists.
This results in $\consistencyrelation{CR}{1} \concat \consistencyrelation{CR}{2}' = \emptyset$, because every person can be consistent to a resident with an empty street name, thus not requiring a corresponding employee.
This shows the necessity of the second restriction in the definition. 
    \item $\consistencyrelation{CR}{3} \concat \consistencyrelation{CR}{4}$: 
The concatenation $\consistencyrelation{CR}{3} \concat \consistencyrelation{CR}{4}$ is obviously empty, because $\consistencyrelation{CR}{3}$ requires a resident for each person, but $\consistencyrelation{CR}{4}$ only requires an employee if there is also a location.
Such a location does not necessarily exist if a person exists, thus if the models are consistent to $\consistencyrelation{CR}{3}$ and $\consistencyrelation{CR}{4}$ there must not necessarily be an employee for any contained person.
This shows the necessity for the first restriction in \autoref{def:relationconcatenation}, which would require a left condition element from $\consistencyrelation{CR}{4}$ (resident and location) to be a subset of a right condition element in $\consistencyrelation{CR}{3}$ (resident).
    \item $\consistencyrelation{CR}{4}^T \concat \consistencyrelation{CR}{3}^T$: 
The concatenation of the transposed relations $\consistencyrelation{CR}{4}^T \concat \consistencyrelation{CR}{3}^T$ is not empty, but actually contains all combinations of each possible employee with all addresses and relates them to a person with the same name.
This is reasonable, because $\consistencyrelation{CR}{4}^T$ requires for all existing employees and addresses that an appropriate resident with the same name %(and also a location)
has to exist, which then requires a person with that name to exist due to $\consistencyrelation{CR}{3}^T$.
The definition does only cover that case due to its first restriction, because $\conditionelement{c}{l,2}$, i.e., the resident related to a person by $\consistencyrelation{CR}{3}^T$ is a subset of $\conditionelement{c}{r,1}$, i.e., a tuple of resident and location.
\end{enumerate}
\end{example}

\mnote{Concatenation not restricting consistency}
We can formally show that the defined notion of concatenation does not lead to any restriction of consistency regarding the single relations:

\begin{lemma}[Concatenation Consistency] \label{lemma:concatenationimpliesconsistency}
    Let $\consistencyrelation{CR}{1}$ and $\consistencyrelation{CR}{2}$ be two consistency relations for a metamodel tuple $\metamodeltuple{M}$ and let $\consistencyrelation{CR}{} = \consistencyrelation{CR}{1} \concat \consistencyrelation{CR}{2}$ be their concatenation. Then it holds that:
    \begin{align*}
        &
        \forall \modeltuple{m} \in \metamodeltupleinstanceset{M}: \modeltuple{m} \consistenttomath \setted{\consistencyrelation{CR}{1}, \consistencyrelation{CR}{2}} \Rightarrow \modeltuple{m} \consistenttomath \consistencyrelation{CR}{}
    \end{align*}
\end{lemma}

\begin{proof}
    For any tuple of models $\modeltuple{m}$ that is consistent to $\consistencyrelation{CR}{1}$ and $\consistencyrelation{CR}{2}$, take a witness structure $\consistencyrelation{W}{1}$ that witnesses consistency of $\modeltuple{m}$ to $\consistencyrelation{CR}{1}$ and $\consistencyrelation{W}{2}$ that witnesses consistency of $\modeltuple{m}$ to $\consistencyrelation{CR}{2}$.
    Now consider the composed witness structure $\consistencyrelation{W}{} = \consistencyrelation{W}{1} \concat \consistencyrelation{W}{2}$.
    We show that $\consistencyrelation{W}{}$ is a valid witness structure for $\consistencyrelation{CR}{}$.

    Let us assume there were $\tupled{\conditionelement{c}{l}, \conditionelement{c}{r}}, \tupled{\conditionelement{c}{l}', \conditionelement{c}{r}'} \in \consistencyrelation{W}{}$ with $\conditionelement{c}{l} = \conditionelement{c}{l}'$ and $\conditionelement{c}{r} \neq \conditionelement{c}{r}'$, such that $\consistencyrelation{W}{}$ is not a witness structure for $\consistencyrelation{CR}{}$.
    Per definition, $\conditionelement{c}{l}$ only occurs in one $\tupled{\conditionelement{c}{l}, \conditionelement{c}{r,1}} \in \consistencyrelation{W}{1}$.
    So there must be two consistency relation pairs $\tupled{\conditionelement{c}{l,2}, \conditionelement{c}{r}}, \tupled{\conditionelement{c}{l,2}', \conditionelement{c}{r}'} \in \consistencyrelation{CR}{2}$ with $\conditionelement{c}{l,2} \subseteq \conditionelement{c}{r,1}$ and $\conditionelement{c}{l,2}' \subseteq \conditionelement{c}{r,1}$.
    However, since $\conditionelement{c}{l,2}$ and $\conditionelement{c}{l,2}'$ contain instances of the same classes and are both subsets of the same object tuple $\conditionelement{c}{r,1}$, we have $\conditionelement{c}{l,2} = \conditionelement{c}{l,2}'$.
    So we know that $\consistencyrelation{W}{}$ fulfills the first condition of a witness structure according to \autoref{def:consistency} for consistency:
    \begin{align*}
        &
        \forall \tupled{\conditionelement{c}{l,1}, \conditionelement{c}{r,1}}, \tupled{\conditionelement{c}{l,2}, \conditionelement{c}{r,2}} \in \consistencyrelation{W}{} : 
        \tupled{\conditionelement{c}{l,1}, \conditionelement{c}{r,1}} = \tupled{\conditionelement{c}{l,2}, \conditionelement{c}{r,2}} \lor \conditionelement{c}{l,1} \neq \conditionelement{c}{l,2} \land \conditionelement{c}{r,1} \neq \conditionelement{c}{r,2}
    \end{align*}
    Additionally, since $\consistencyrelation{W}{1}$ and $\consistencyrelation{W}{2}$ are witness structures for consistency of $\modeltuple{m}$ to $\consistencyrelation{CR}{1}$ and $\consistencyrelation{CR}{2}$, the model tuple contains all condition elements in $\consistencyrelation{W}{1}$ and $\consistencyrelation{W}{2}$.
    Consequentially, $\modeltuple{m}$ also contains the condition elements in $\consistencyrelation{W}{}$, as those in $\consistencyrelation{W}{}$ are composed of the ones in $\consistencyrelation{W}{1}$ and $\consistencyrelation{W}{2}$. This implies that the second condition of \autoref{def:consistency} is fulfilled:
    \begin{align*}
        &
        \forall \tupled{\conditionelement{c}{l}, \conditionelement{c}{r}} \in \consistencyrelation{W}{} : \modeltuple{m} \containsmath \conditionelement{c}{l} \land \modeltuple{m} \containsmath \conditionelement{c}{r}
    \end{align*}
    Finally, we assume the third condition of \autoref{def:consistency} was unfulfilled, i.e.: 
    \begin{align*}
        &
        \exists \conditionelement{c}{l}' \in \condition{c}{l,\consistencyrelation{CR}{}} : \modeltuple{m} \containsmath \conditionelement{c}{l}' \land \conditionelement{c}{l}' \not\in \condition{c}{l,\consistencyrelation{W}{}}
    \end{align*}
    We know that $\condition{c}{l,\consistencyrelation{CR}{}} \subseteq \condition{c}{l,\consistencyrelation{CR}{1}}$, because the left condition elements in $\consistencyrelation{CR}{}$ are, per definition, taken from the left condition elements in $\consistencyrelation{CR}{1}$ and thus also contained in $\consistencyrelation{CR}{1}$.
    Since $\modeltuple{m} \containsmath \conditionelement{c}{l}'$, there must be a consistency relation pair $\tupled{\conditionelement{c}{l}', \conditionelement{c}{r,1}'} \in \consistencyrelation{W}{1}$ that witnesses consistency of $\conditionelement{c}{l}'$ according to $\consistencyrelation{CR}{1}$.
    There must be at least one consistency relation pair $\tupled{\conditionelement{c}{l,2}', \conditionelement{c}{r,2}'} \in \consistencyrelation{CR}{2}$ with $\conditionelement{c}{l,2}' \subseteq \conditionelement{c}{r,1}'$, because otherwise $\conditionelement{c}{l}'$ would, per definition, not occur in the left condition of $\consistencyrelation{CR}{}$.
    For all such tuples $\tupled{\conditionelement{c}{l,2}', \conditionelement{c}{r,2}'}$, we know that $\modeltuple{m} \containsmath \conditionelement{c}{l,2}'$, because $\modeltuple{m} \containsmath \conditionelement{c}{r,1}'$ due to its containment in $\consistencyrelation{W}{1}$ and due to $\conditionelement{c}{l,2}' \subseteq \conditionelement{c}{r,1}'$.
    In consequence, consistency to $\consistencyrelation{CR}{2}$ requires that for one of those $\conditionelement{c}{r,2}'$ it holds that $\modeltuple{m} \containsmath \conditionelement{c}{r,2}'$ and that there is $\tupled{\conditionelement{c}{l,2}', \conditionelement{c}{r,2}'} \in \consistencyrelation{W}{2}$ that witnesses this consistency.
    Summarizing, due to $\tupled{\conditionelement{c}{l}', \conditionelement{c}{r,1}'} \in \consistencyrelation{W}{1}$ and $\tupled{\conditionelement{c}{l,2}', \conditionelement{c}{r,2}'} \in \consistencyrelation{W}{2}$ with $\conditionelement{c}{l,2}' \subseteq \conditionelement{c}{r,1}'$ and due to the definition of $\consistencyrelation{W}{}$ as $\consistencyrelation{W}{1} \concat \consistencyrelation{W}{2}$, we know that $\tupled{\conditionelement{c}{l}', \conditionelement{c}{r,2}'} \in \consistencyrelation{W}{}$, which breaks our assumption.
    So we have shown that:
    \begin{align*}
        &
        \forall \conditionelement{c}{l}' \in \condition{c}{l,\consistencyrelation{CR}{}} : \modeltuple{m} \containsmath \conditionelement{c}{l}' \Rightarrow \conditionelement{c}{l}' \in \condition{c}{l,\consistencyrelation{W}{}}
    \end{align*}
    Summarizing, we have shown that $\consistencyrelation{W}{}$ fulfills all three requirements for a witness structure for consistency according to \autoref{def:consistency}, so we know that $\modeltuple{m} \consistenttomath \consistencyrelation{CR}{}$.
\end{proof}


\subsection{Transitive Closure of Consistency Relations}

\mnote{Transitive closure of implicit relations}
Based on the introduced notion of concatenation, we can define a transitive closure for sets of consistency relations, which contains all relations in that set complemented by all possible concatenations of them, i.e., \emph{implicit relations} of that set.
Having shown that our definition of consistency relations concatenation is well-defined in the sense that it does not introduce further restrictions for consistency, we can show that the transitive closure does not restrict consistency in comparison to the set of consistency relations itself.

\begin{definition}[Consistency Relations Transitive Closure] \label{def:transitiveclosure}
    Let $\consistencyrelationset{CR}$ be a set of consistency relation. We define its transitive closure $\transitiveclosure{\consistencyrelationset{CR}}$ as:
    \begin{align*}
        \transitiveclosure{\consistencyrelationset{CR}{}} \equalsperdefinition \setted{\consistencyrelation{CR}{} \mid & \exists \consistencyrelation{CR}{1}, \dots, \consistencyrelation{CR}{k} \in \consistencyrelationset{CR}{} :
        \consistencyrelation{CR}{} = \consistencyrelation{CR}{1} \concat \dots \concat \consistencyrelation{CR}{k} }
    \end{align*}
\end{definition}

\mnote{Direct and direct relations}
The transitive closure of a set of consistency relations $\consistencyrelationset{CR}$ contains all consistency relations of $\consistencyrelationset{CR}$ and all their concatenations. That means, the transitive closure contains consistency relations that relate all elements that are directly or indirectly related due to $\consistencyrelationset{CR}$.
Due to cycles in the concatenation of relations, this closure can, in general, be of infinite size.

\mnote{Multiple concatenation not restricting consistency}
The transitive closure of a consistency relation set does not further restrict consistency in comparison to the original set by construction of concatenation, i.e., if a model tuple is consistent to a set of consistency relations, it is also consistent to their transitive closure.
We show that in the following by first extending the argument of \autoref{lemma:concatenationimpliesconsistency}, which shows that concatenation does not further restrict consistency, to the transitive closure, which is only a set of concatenations of consistency relations.

\begin{lemma}[Relation Set Consistency]
    Let $\consistencyrelationset{CR}$ be a set of consistency relations for a tuple of metamodels $\metamodeltuple{M}$. 
    Then it holds that:
    \begin{align*}
        &
        \forall \consistencyrelation{CR}{} \in \transitiveclosure{\consistencyrelationset{CR}{}} \setminus \consistencyrelationset{CR} :
        \exists \consistencyrelation{CR}{1}, \dots, \consistencyrelation{CR}{k} \in \consistencyrelationset{CR} : \forall \modeltuple{m} \in \metamodeltupleinstanceset{M} : \\
        & \formulaskip
        \modeltuple{m} \consistenttomath \setted{\consistencyrelation{CR}{1}, \dots \consistencyrelation{CR}{k}} \Rightarrow \modeltuple{m} \consistenttomath \consistencyrelation{CR}{} 
    \end{align*}
\end{lemma}

\begin{proof}
    Per definition, every $\consistencyrelation{CR}{} \in \transitiveclosure{\consistencyrelationset{CR}}$ is a concatenation of consistency relations in $\consistencyrelationset{CR}$, i.e.:
    \begin{align*}
        &
        \forall \consistencyrelation{CR}{} \in \transitiveclosure{\consistencyrelationset{CR}} : \exists \consistencyrelation{CR}{1}, \dots, \consistencyrelation{CR}{k} \in \consistencyrelationset{CR} :
        \consistencyrelation{CR}{} = \consistencyrelation{CR}{1} \concat \dots \concat \consistencyrelation{CR}{k}
    \end{align*}
    We already know for every two consistency relations $\consistencyrelation{CR}{1}$ and $\consistencyrelation{CR}{2}$ and all model tuples $\modeltuple{m}$ that if $\modeltuple{m} \consistenttomath \setted{\consistencyrelation{CR}{1}, \consistencyrelation{CR}{2}}$ then $\modeltuple{m} \consistenttomath \consistencyrelation{CR}{1} \concat \consistencyrelation{CR}{2}$ (\autoref{lemma:concatenationimpliesconsistency}).
    Inductively applying that argument to $\consistencyrelation{CR}{1}, \dots, \consistencyrelation{CR}{k}$ shows that $\modeltuple{m} \consistenttomath \consistencyrelation{CR}{}$ whenever $\modeltuple{m} \consistenttomath \setted{\consistencyrelation{CR}{1}, \dots, \consistencyrelation{CR}{k}}$.
\end{proof}

\mnote{Transitive closure not restricting consistency}
As a direct result of the previous lemma, we can show that the transitive closure of a consistency relation set considers the same tuples of models consistent as the consistency relation set itself.

\begin{lemma}[Transitive Closure Consistency] \label{lemma:consistencytransitiveclosure}
    Let $\consistencyrelationset{CR}$ be a consistency relation set for a metamodel tuple $\metamodeltuple{M}$.
    Then:
    \begin{align*}
        \forall \modeltuple{m} \in \metamodeltupleinstanceset{M}: \modeltuple{m} \consistenttomath \consistencyrelationset{CR} \equivalent
        \modeltuple{m} \consistenttomath \transitiveclosure{\consistencyrelationset{CR}}
    \end{align*}
\end{lemma}

\begin{proof}
    Adding a consistency relation to a set of consistency relations can never relax consistency, i.e., lead to models becoming consistent that were not considered consistent before. This is a direct consequence of \autoref{def:consistency} for consistency, which defines models as consistent when they are consistent to all consistency relations in a set, thus only restricting the set of consistent model tuples by adding further relations.
    In consequence, it holds that:
    \begin{align*}
        \modeltuple{m} \consistenttomath \transitiveclosure{\consistencyrelationset{CR}} \Rightarrow \modeltuple{m} \consistenttomath \consistencyrelationset{CR}
    \end{align*}
    According to \autoref{lemma:consistencytransitiveclosure}, a tuple of models that is consistent to $\consistencyrelationset{CR}$ is always consistent to all transitive relations in $\transitiveclosure{\consistencyrelationset{CR}}$ as well. Thus, we know that:
    \begin{align*}
        \modeltuple{m} \consistenttomath \consistencyrelationset{CR} \Rightarrow
        \modeltuple{m} \consistenttomath \transitiveclosure{\consistencyrelationset{CR}}
    \end{align*}
    In consequence, models are considered consistent equally for $\consistencyrelationset{CR}$ and its transitive closure $\transitiveclosure{\consistencyrelationset{CR}}$.
\end{proof}


\subsection{Compatibility of Consistency Relations}

\mnote{Formalization of compatibility}
Based on the notion of fine-grained consistency relations and their concatenation, we can precisely formulate our initially informal notion of \emph{compatibility} of consistency relations.
We have stated that we consider consistency relation incompatible if they are contradictory, like the relation between names in our initial example in \autoref{fig:compatibility:three_persons_example_extended}.
In that example, for residents with non-lowercase names no consistent tuple of models could be derived.
We formalize this notion of \emph{non-contradictory} relations by requiring that relations may not restrict that an object tuple, for which consistency is defined in any consistency relation, cannot occur in a consistent model tuple anymore.

\begin{definition}[Compatibility] \label{def:compatibility}
    Let $\consistencyrelationset{CR}$ be a set of consistency relations for a tuple of metamodels $\metamodeltuple{M}$.
    We say that:
    \begin{align*}
        &
        \consistencyrelationset{CR} \compatiblemath \equivalentperdefinition
        \forall \consistencyrelation{CR}{} \in \consistencyrelationset{CR} : \forall \conditionelement{c}{} \in \condition{c}{l, \consistencyrelation{CR}{}}: \exists \modeltuple{m} \in \metamodeltupleinstanceset{M} : \\
        & \formulaskip
        \modeltuple{m} \containsmath \conditionelement{c}{} \land \modeltuple{m} \consistenttomath \consistencyrelationset{CR}
    \end{align*}
    We call a set of consistency relation $\consistencyrelationset{CR}$ \emph{incompatible} if it does not fulfill the definition of compatibility.
\end{definition}

\mnote{Compatibility at running example}
We exemplify this notion of compatibility at an extract of the initial example with different consistency relations.

\begin{example}
\autoref{fig:compatibility:incompatibility_example} shows an extract of the three metamodels from \autoref{fig:compatibility:three_persons_example_extended} and several consistency relations, of which different combinations are compatible or incompatible according to the previous definition.
We always consider the actual relations together with their transposed ones to have a symmetric set of consistency relations.

\begin{figure}
    \centering
    \newcommand{\hdistance}{14em}
\newcommand{\classwidth}{6em}

\begin{tikzpicture}

% Person
\umlclassvarwidth{person}{}{Person\sameheight}{
firstname\\
lastname
}{\classwidth}

% Employee
\umlclassvarwidth[,above right=4em and \hdistance of person.center, anchor=south]{employee}{}{Employee\sameheight}{
name
}{\classwidth}

\umlclassvarwidth[,right=\hdistance of person.south, anchor=south]{resident}{}{Resident\sameheight}{
name
}{\classwidth}


% CONSISTENCY RELATIONS
\draw[directed consistency relation] (person.north) |- node[pos=0, above right] {$p$} node[pos=0.5, above right] {$\consistencyrelation{CR}{2}$ / $\consistencyrelation{CR}{2}'$ / $\consistencyrelation{CR}{2}''$} node[pos=1, below left] {$e$} (employee.west);
\draw[directed consistency relation] (employee.south) -- node[pos=0, below left] {$e$} node[right, align=left] {$\consistencyrelation{CR}{3}$  / $\consistencyrelation{CR}{3}'$} node[pos=1, above left] {$r$} (resident.north);
\draw[directed consistency relation] (resident.west-|person.east) -- node[pos=0, above right] {$p$} node[pos=0.5, below] {$\consistencyrelation{CR}{1}$} node[pos=1, above left] {$r$} (resident.west);

\node[consistency related element, below left=1.5em and 1em of person.south west, anchor=north west] {
$\begin{aligned}
    &
    \consistencyrelation{CR}{1} = \setted{\tupled{p,r} \mid \mathvariable{r.name} = \mathvariable{p.firstname} + \textnormal{\enquote{\textvisiblespace}} + \mathvariable{p.lastname}}\\[0.5em]
    &
    \consistencyrelation{CR}{2} = \setted{\tupled{p,e} \mid \mathvariable{e.name} = \mathvariable{p.firstname} + \textnormal{\enquote{\textvisiblespace}} + \mathvariable{p.lastname}}\\
    &
    \consistencyrelation{CR}{2}' = \setted{\tupled{p,e} \mid \mathvariable{e.name} = \mathvariable{p.firstname} + \textnormal{\enquote{,\textvisiblespace}} + \mathvariable{p.lastname}}\\
    &
    \consistencyrelation{CR}{2}'' = \setted{\tupled{p,e} \mid \mathvariable{e.name} = \mathvariable{p.lastname} + \textnormal{\enquote{\textvisiblespace}} + \mathvariable{p.firstname}}\\[0.5em]
    &
    \consistencyrelation{CR}{3} = \setted{\tupled{r,e} \mid \mathvariable{r.name} = \mathvariable{e.name}}\\
    &
    \consistencyrelation{CR}{3}' = \setted{\tupled{r,e} \mid \mathvariable{r.name} = \mathvariable{e.name.toLower}}
\end{aligned}$
};

\end{tikzpicture}
    %\includegraphics[width=\columnwidth]{figures/incompatibility_example.png}
    \caption[Different incompatibility scenarios]{Three metamodels with different options for consistency relations. The relation sets $\setted{\consistencyrelation{CR}{1}, \consistencyrelation{CR}{1}^T,\consistencyrelation{CR}{2}, \consistencyrelation{CR}{2}^T, \consistencyrelation{CR}{3}, \consistencyrelation{CR}{3}^T}$ and $\setted{\consistencyrelation{CR}{1}, \consistencyrelation{CR}{1}^T, \consistencyrelation{CR}{2}'', \consistencyrelation{CR}{2}''^T, \consistencyrelation{CR}{3}, \consistencyrelation{CR}{3}^T}$ are compatible, whereas the sets $\setted{\consistencyrelation{CR}{1}, \consistencyrelation{CR}{1}^T, \consistencyrelation{CR}{2}', \consistencyrelation{CR}{2}'^T, \consistencyrelation{CR}{3}, \consistencyrelation{CR}{3}^T}$ and $\setted{\consistencyrelation{CR}{1}, \consistencyrelation{CR}{1}^T, \consistencyrelation{CR}{2}, \consistencyrelation{CR}{2}^T, \consistencyrelation{CR}{3}', \consistencyrelation{CR}{3}'^T}$ are not. Taken from~\owncite[Fig.~4]{klare2020compatibility-report}.}
    \label{fig:compatibility:incompatibility_example}
\end{figure}

\begin{properdescription}
\item[$\setted{\consistencyrelation{CR}{1}, \consistencyrelation{CR}{1}^T,\consistencyrelation{CR}{2}, \consistencyrelation{CR}{2}^T, \consistencyrelation{CR}{3}, \consistencyrelation{CR}{3}^T}$:]
These relations are obviously compatible, because they relate $\mathvariable{firstname}$, $\mathvariable{lastname}$ and $\mathvariable{name}$ in the same way. Thus, for each object with any name, and thus any condition element in all of the consistency relations, a consistent model tuple can be found by adding instances of the other classes with appropriate names.

\item[$\setted{\consistencyrelation{CR}{1}, \consistencyrelation{CR}{1}^T,\consistencyrelation{CR}{2}', \consistencyrelation{CR}{2}'^T, \consistencyrelation{CR}{3}, \consistencyrelation{CR}{3}^T}$:]
These relations are incompatible, because for each person $\mathvariable{p1}$, another person $\mathvariable{p2}$ must exist with $\mathvariable{p2.firstname} = \mathvariable{p1.firstname}\; +\; ","$ and $\mathvariable{p2.lastname} = \mathvariable{p1.lastname}$ due to $\consistencyrelation{CR}{2}'$ and the transitive relations requiring the addition of a comma. Thus, each person would require an infinite number of further persons to exist in a consistent tuple of models. Models are, however, finite, so there is no such model tuple and the relations are incompatible.

\item[$\setted{\consistencyrelation{CR}{1}, \consistencyrelation{CR}{1}^T, \consistencyrelation{CR}{2}', \consistencyrelation{CR}{2}'^T, \consistencyrelation{CR}{3}, \consistencyrelation{CR}{3}^T}$:]
These relations are compatible. The relations define that for a person $\mathvariable{p1}$, another person $\mathvariable{p2}$ with $\mathvariable{p2.firstname} = \mathvariable{p1.lastname}$ and $\mathvariable{p2.lastname} = \mathvariable{p1.firstname}$ has to exist, so that the tuple of models is consistent.
Although that behavior may not be desired, it does not violate the definition of compatibility, because for every object in the relations, a consistent model tuple can be constructed.
In general, it can even be necessary that consistency relations require the same elements with swapped attribute values to exist, such that this behavior can and should not be forbidden.
Finally, such a relation does also not prevent a consistency preservation rule from finding a consistent model tuple.
In consequence, such behavior may be undesired due to the specific semantics a the domain, but it can neither be detected automatically, not does it lead to problems when executing transformations.

\item[$\setted{\consistencyrelation{CR}{1}, \consistencyrelation{CR}{1}^T, \consistencyrelation{CR}{2}, \consistencyrelation{CR}{2}^T, \consistencyrelation{CR}{3}', \consistencyrelation{CR}{3}'^T}$:]
These consistency relations reflect the ones of our motivational example for an intuitive notion of incompatibility, discussed at \autoref{fig:compatibility:intuitive_incompatibility}.
The formal definition of compatibility also considers these relations as incompatible, because it is not possible to create a resident with an uppercase name, such that the containing tuple of models is consistent.
For a resident with $\mathvariable{name} = "A\text{\textvisiblespace}B"$, a person with $\mathvariable{firstname} = "A"$ and $\mathvariable{lastname} = "B"$ has to exist, which requires the existence of an employee with $\mathvariable{name} = "A\text{\textvisiblespace}B"$. Now $\consistencyrelation{CR}{3}'$ requires a resident with $\mathvariable{name} = "a\text{\textvisiblespace}b"$ to exist, which in turn requires a resident with $\mathvariable{firstname} = "a"$ and $\mathvariable{lastname} = "b"$ and an employee with $\mathvariable{name} = "a\text{\textvisiblespace}b"$ to exist.
In consequence, there are two employees, one with the uppercase and one with the lowercase name, for which a resident with the lowercase name has to exist according to the relation $\consistencyrelation{CR}{3}'$. So there is no witness structure with a unique mapping between the elements that is required to fulfill \autoref{def:consistency} for consistency.
\end{properdescription}
\end{example}

\mnote{Goal of compatibility}
To summarize, compatibility is supposed to ensure that consistency relations do not impose restrictions on other relations such that their condition elements, for which consistency is defined, can never occur in consistent models.
The goal of ensuring compatibility is especially to prevent the execution of consistency preservation rules in transformation networks from non-termination, as it may occur especially in the second scenario, in which an infinitely large model would be required to fulfill the consistency relations.

\mnote{Equivalence of transitive closure}
Finally, analogously to the equivalence of a set of consistency relations $\consistencyrelationset{CR}$ and its transitive closure $\transitiveclosure{\consistencyrelationset{CR}}$ in regards to consistency of a tuple of models, we can show that a set of consistency relations and its transitive closure are always equal with regards to compatibility.

\begin{lemma}[Transitive Closure Compatibility] \label{lemma:compatibilitytransitiveclosure}
    Let $\consistencyrelationset{CR}$ be a set of consistency relations.
    It holds that:
    \begin{align*}
        \consistencyrelationset{CR} \compatiblemath \equivalent
        \transitiveclosure{\consistencyrelationset{CR}} \compatiblemath
    \end{align*}
\end{lemma}

\begin{proof}
    The reverse direction of the equivalence is given by definition, since compatibility of a set of consistency relations implies compatibility of every subset by definition.
    So we have to show the forward direction by considering the compatibility definition for all $\consistencyrelation{CR}{} \in \transitiveclosure{\consistencyrelationset{CR}}$.
    We partition $\transitiveclosure{\consistencyrelationset{CR}}$ into $\consistencyrelationset{CR}$ and $\transitiveclosure{\consistencyrelationset{CR}} \setminus \consistencyrelationset{CR}$ and consider their consistency relations independently.
    
    First, we consider $\consistencyrelation{CR}{} \in \transitiveclosure{\consistencyrelationset{CR}} \setminus \consistencyrelationset{CR}$.
    According to \autoref{def:transitiveclosure} for the transitive closure, each $\consistencyrelation{CR}{} \in \transitiveclosure{\consistencyrelationset{CR}} \setminus \consistencyrelationset{CR}$ is a concatenation of consistency relations $\consistencyrelation{CR}{1}, \dots, \consistencyrelation{CR}{k} \in \consistencyrelationset{CR}$.
    In consequence of that definition, we know that $\condition{c}{l,\consistencyrelation{CR}{}} \subseteq \condition{c}{l,\consistencyrelation{CR}{1}}$, so it is given that:
    \begin{align}
        & \nonumber \label{eq:transitiverelationcontainment}
        \forall \conditionelement{c}{l} \in \condition{c}{l,\consistencyrelation{CR}{}} : \exists \conditionelement{c}{l}' \in \condition{c}{l,\consistencyrelation{CR}{1}} : \forall \modeltuple{m} \in \metamodeltupleinstanceset{M} : \\ 
        & \formulaskip
        \modeltuple{m} \containsmath \conditionelement{c}{l} \Rightarrow \modeltuple{m} \containsmath \conditionelement{c}{l}'
    \end{align}
    Since $\consistencyrelationset{CR}$ is compatible, we know from \autoref{def:compatibility} for compatibility that:
    \begin{align}
        & 
        \label{eq:compatibilitysingleelement}
        \forall \conditionelement{c}{l}' \in \condition{c}{l, \consistencyrelation{CR}{1}}
        : \exists \modeltuple{m} \in \metamodeltupleinstanceset{M} :
        \modeltuple{m} \containsmath \conditionelement{c}{l}' \land \modeltuple{m} \consistenttomath \consistencyrelationset{CR}
    \end{align}
    Because of \autoref{eq:transitiverelationcontainment} and \autoref{eq:compatibilitysingleelement}, we know that:
    \begin{align}
        &
        \label{eq:compatibilitysinglelementtransitive}
        \forall \conditionelement{c}{l} \in \condition{c}{l, \consistencyrelation{CR}{}}
        : \exists \modeltuple{m} \in \metamodeltupleinstanceset{M} :
        \modeltuple{m} \containsmath \conditionelement{c}{l} \land \modeltuple{m} \consistenttomath \consistencyrelationset{CR}
    \end{align}  
    Furthermore, \autoref{lemma:consistencytransitiveclosure} states that:
    \begin{align}
        & \label{eq:consistencytransitiveequal}
        \forall \modeltuple{m} \in \metamodeltupleinstanceset{M}: \modeltuple{m} \consistenttomath \consistencyrelationset{CR} \equivalent \modeltuple{m} \consistenttomath \transitiveclosure{\consistencyrelationset{CR}}
    \end{align}
    In consequence of Equations \ref{eq:compatibilitysinglelementtransitive} and \ref{eq:consistencytransitiveequal}, we know that:
    \begin{align}
        & \nonumber \label{eq:compatibilityonlyclosure}
        \forall \consistencyrelation{CR}{} \in \transitiveclosure{\consistencyrelationset{CR}} \setminus \consistencyrelationset{CR} : \forall \conditionelement{c}{} \in \condition{c}{l, \consistencyrelation{CR}{}}
        : \exists \modeltuple{m} \in \metamodeltupleinstanceset{M} : \\
        & \formulaskip
        \modeltuple{m} \containsmath \conditionelement{c}{} \land \modeltuple{m} \consistenttomath \transitiveclosure{\consistencyrelationset{CR}}
    \end{align}
    %
    Second, we consider $\consistencyrelation{CR}{} \in \consistencyrelationset{CR}$.
    Due to compatibility of $\consistencyrelationset{CR}$ and \autoref{lemma:consistencytransitiveclosure} showing equality of consistency of $\modeltuple{m}$ regarding $\consistencyrelationset{CR}$ and $\transitiveclosure{\consistencyrelationset{CR}}$, it is true that:
    \begin{align}
        & \nonumber \label{eq:compatibilitynonclosure}
        \forall \consistencyrelation{CR}{} \in \consistencyrelationset{CR} : \forall \conditionelement{c}{} \in \condition{c}{l, \consistencyrelation{CR}{}}
        : \exists \modeltuple{m} \in \metamodeltupleinstanceset{M} : \\
        & \formulaskip
        \modeltuple{m} \containsmath \conditionelement{c}{} \land \modeltuple{m} \consistenttomath \transitiveclosure{\consistencyrelationset{CR}}
    \end{align}
    %
    Equations \ref{eq:compatibilityonlyclosure} and \ref{eq:compatibilitynonclosure} show compatibility of $\transitiveclosure{\consistencyrelationset{CR}}$ if $\consistencyrelationset{CR}$ is compatible.
\end{proof}
