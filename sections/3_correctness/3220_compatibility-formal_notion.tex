\section{A Formal Notion of Compatibility}
\label{chap:compatibility:formal_notion}

\begin{copiedFrom}{SoSym MPM4CPS}

\mnote{Formal notion of fine-grained consistency and compatibility}
In this section, we precisely define our, yet informally introduced notion of \emph{compatibility}.
%We first discuss properties of transformation networks %in comparison to a single transformation 
%with an intuitive notion of compatibility, based on considerations in existing work.
We define consistency more precisely based on fine-grained consistency relations, proceed with implicit relations, which are induced by a set of consistency relations, such as transitive relations, and,
finally, derive a compatibility notion from the consistency formalization and its pursued perception.
To distinguish the previously given coarse-grained and the coming fine-grained definition of consistency and consistency relations, we used the \enquote{model-level} prefix in the coarse-grained notion.
The contents of this and the remaining sections of this chapter are mostly, even literally, taken from our published article on proving compatibility~\owncite{klare2020compatibility-report}.

%This serves as our contribution \ref{contrib:formalization}.
%An introduction to properties of transformation networks and especially compatibility of transformations, what compatibility means and how the network influences the possibility to give guarantees regarding compatibility.

%\todoHeiko{Discuss extensional definition of relations / constraints and its relation to intensional definitions}
% \subsection{Properties of Transformation Network}
% \label{sec:compatibility:networkproperties}

% %% Correctness as a property of single, bidirectional transformations
% Keeping pairs of models consistent by means of incremental, bidirectional transformations has been well researched in recent years~\cite{stevens2010sosym, etzlstorfer2013a, cleve2019dagstuhl}.
% A bidirectional transformation consists of a \emph{relation} that specifies which pairs of models are considered consistent and a pair of directional transformations, denoted as \emph{consistency repair routines}, that take one modified and one originally consistent model and deliver a new model that is consistent to the modified one~\cite{stevens2010sosym}.
% Several well-defined properties of such transformations have been identified.
% The essential \emph{correctness} property states that a consistency repair routine delivers a result such that the models are actually consistent according to the defined relation~\cite{stevens2010sosym}.
% Another important property is \emph{hippocraticness}, which states that a consistency repair routine returns the input model if it was already consistent to the modified one~\cite{stevens2010sosym}.

%%
%% Correctness property of transformation networks, not induced by correctness of single transformations
%%
% When we combine several transformations to a network to achieve consistency between multiple models, those properties of the single transformations are still relevant, as each transformations on its own has to be at least correct to work properly in a network of transformations.
% However, correctness of the single transformations does not induce correctness of the transformation network.
% Taking an arbitrary set of correct transformations and executing them one after another does not necessarily constitute a terminating approach that delivers a result, in which all models are consistent according to the relations of the transformations, because the result of one transformation may violate the relation of another.
% It is possible that the approach does either not terminate, because there is a divergence or alternation in values changed or elements created, or terminates in a state that is not consistent regarding the relations of all transformations~\cite{klare2019icmt}.
% %Consider the example of $R_{PE}, R_{PR}, R'_{ER}$ in \autoref{fig:motivational_example}. A transformation network in which each transformation adds the corresponding element for an added element, such as a resident for an added person by the transformation of $R_{PR}$.
% \todoDiss{Readd}
% %Since the execution of one transformation may lead to a violation of the relation of another transformation, executing each transformation only once can easily lead to termination in an inconsistent state.
% %In consequence, transformations have to be executed in a fixed point iteration manner, until all relations are fulfilled.
% The property of a network to always result in a state in which the models are consistent to all relations of the transformations if they are executed in a specific order, can be seen as a \emph{correctness} property for transformation networks.
% In this work, we focus on that correctness property and do not discuss further quality properties of transformation networks, such as \emph{modularity}, \emph{evolvability} and \emph{comprehensibility}~\cite{klare2018docsym}.


%%
%% Correctness considerable at different levels.
%%
% A single transformation can only be incorrect in terms of its repair routines, because there are no restrictions regarding its relation that may prevent the repair routine from being able to produce a correct result.
% In previous work, however, we identified that transformation networks can be incorrect at different levels~\cite{klare2019icmt}.
% Networks can also be incorrect at the level of relations rather than repair routines, because multiple relations can be contradictory, i.e., they can relate elements in different ways such that the relations cannot be fulfilled at the same time.
% \todoDiss{Readd}
%This especially concerns the \emph{modularization level}, which considers correctness of the binary relations of a set of bidirectional transformations describing when a set of models is consistent, and the \emph{operationalization level}, which considers correctness of the consistency repair routines in terms of producing models that are correct regarding the relations of all bidirectional transformations.
%For a single transformation, there are no restrictions regarding its relation that may prevent the repair routines from being able to produce a correct result, thus correctness is only considered at the operationalization level. %The correctness definition of \textcite{stevens2010sosym} for a single transformation only concerns the operationalization level, because a single relation is correct by construction. 
% Considering only a single transformation, its relation is correct by definition, at least if there is no external specification against which the relations have to be correct, thus the definition of \textcite{stevens2010sosym} for correctness only considers the consistency repair routines.
%However, in case of a transformation network, the different relations may be contradictory, i.e., elements are related in a different way such that the relations cannot be fulfilled at the same time.
% In such a case, the consistency repair routines cannot be result in a state that is consistent according to the relations anymore, thus they may not terminate anymore.
%This is due to the fact that several relations can relate the same elements in a different way, such that they cannot be fulfilled at the same time, i.e. that they are contradictory.
%In such a case, the iteration will not terminate.
% We call the relations of such transformations \emph{incompatible}.
%We define this informal notion of compatibility more precisely in the remainder of this section.

%%
%% Compatibility not achievable by construction -> we focus on checking it
%%
% In consequence, compatibility of relations is a necessary prerequisite for consistency repair routines to produce correct results in transformation networks.
% We also found that correctness of the consistency repair routines can already be achieved by construction, whereas compatibility of the relations cannot be achieved by construction but in the best case be checked for a set of relations.
% In this work, we focus on the possibility to check compatibility of the relations of a set of transformations.
% In the following, we therefore precisely define the notion of compatibility of relations, which excludes contradictions in relations that can prevent consistency repair routines from fulfilling the relation.

% \todo{Example!}

%%
%% Tree topology excludes compatibility
%%
%\subsection{Network Topology Impacts}
% Finally, the topology of a transformation network directly influences how prone it is to incompatibilities of its relations.
% Contradictions of consistency relations, as exemplified with the relations $R_{PE}, R_{PR}, R'_{ER}$ in \autoref{fig:prologue:three_persons_example}, can only occur if the same classes are related to each other by different (sequences of) transformations in a different way.
% For example, in \autoref{fig:prologue:three_persons_example}, each combination of two relations puts the same classes into relation as the third one.
%This can also be a the case if a sequence of transformations introduces this relation, like each combination of two relations in \autoref{fig:motivational_example} relates the same classes as the third relation.
% This means that a transformation network, in which each pair of classes is only related by one sequence of transformations, cannot have contradictory relations and is thus inherently compatible.



% \subsection{Consistency by Transformation Networks}

% \begin{itemize}
%     \item Incremental, bidirectional transformations serve as a means for preserving consistency of models by updating one if the other is modified
%     \item Transformations specify the conditions for consistency (consistency relations), as well as how consistency can be restored after modifications (consistency repair)~\cite{stevens2010sosym}
%     \item Consistency between more than two models can be achieved by coupling bidirectional transformations to networks
%     \item Even if the single transformations are correct, i.e. consistency repair produces results that conform to the consistency relations~\cite{stevens2010sosym}, this may not be the case when \emph{independently developed} bidirectional transformations are combined.
%     \item Interoperability problems can occur at the level of consistency relations, i.e. that the relations of the different transformations cannot be fulfilled at the same time, or at the level of consistency repair, i.e. that the repair routines do not terminate in a consistent state or even not at all~\cite{klare2019icmt}
%     \item While \textcite{klare2019icmt} focused on techniques to avoid interoperability issues at the level of consistency repair, in this work we focus on finding interoperability issues of the underlying consistency relations, which we previously defined as \emph{compatibility} issues~\cite{klare2018docsym}.
%     The avoidance of interoperability issues at the relations level is a necessary assumption to avoid issues at the operationalization level.
% \end{itemize}


% \subsection{Network Topology Impacts}

% \begin{itemize}
%     \item Idea: Couple independently developed transformations (refer to running example) to network
%     \item Problem: Trade-off between different properties, depending on the chosen or induced network topology
%     \item Introduce contrary properties \emph{compatibility} and \emph{modularity} (provide only a naive explanation) and shortly refer to evolvability/comprehensibility \cite{klare2018docsym}.
%     \item Explain problems of building tree structures (inherent compatibility) and motivate arbitrary graph structures, providing high modularity, but not giving compatibility guarantees
% \end{itemize}

\subsection{Fine-grained Consistency Relations}

We have yet given a common definition of consistency~\cite{stevens2010sosym} by enumerating consistent pairs of models in a relation (see \autoref{chap:correctness:notions_consistency}).
However, for our studies on compatibility, we need a more fine-grained notion of consistency.
Considering transformations languages, such as \qvtr, first, relations are defined at the level of classes and their properties, i.e., how properties of instances of some classes are related to properties of instances of other classes.
Second, they are defined in an \emph{intensional} way, i.e., constraints specify which elements shall be considered consistent, rather than enumerating all consistent instances in an an \emph{extensional} specification.
We have already discussed that both ways have equal expressiveness and that we stick to extensional specifications for reasons of simplicity, which can be transformed into extensional ones by enumerating all instances that fulfill the constraints.
However, we reuse the concept of specifying relations at the level of classes and their properties. % rather than complete models.
This makes it easier to make statements about dependencies between consistency relations.
For example, two fine-grained consistency relations considering completely independent sets of classes cannot interfere, and thus especially do not introduce any compatibility problems, which is not easy to express when considering relations at the level of complete models.
Finally, from such a fine-grained specification, a holistic relation at the level of models can always be derived by enumerating all models that fulfill all the fine-grained specifications, thus it does not restrict expressiveness in any way and can be seen as a \emph{compositional approach} for defining consistency.

% In the following, we start with introducing a fine-grained notion of consistency relations. % based on such fine-grained consistency relations.
% We proceed with considerations on implicit relations, which are induced by a set of consistency relations, such as transitive relations, to finally precisely define a notion of compatibility.

% General notion of consistency consider relations of models. We use a more fine-grained notion based on consistency relations, which requires those relations to be unidirectional, and the notion of consistency to be more complex. The simple notion (model left requires related model right and vice versa) is not applicable, but induced by our fine-grained specification.

% \begin{itemize}
%     \item Instead of considering relations on the metamodel level, we can consider relations on the element level
%     \item Idea: Create a graph of meta elements (meta-classes, attributes, references) containing edges for all relations between elements defined by consistency relations (if there is a rule relating two or more elements, there is an edge between them)
%     \item If this graph of fine-grained relations forms a set of trees, the transformations are compatible (apply definition from previous section), because a change to one element may only be propagated across one path to each of all other elements
%     \item This is what we call \emph{decomposition} that witnesses compatibility.
%     \item We explain this in detail in \autoref{sec:decomposition}.
% \end{itemize}


%\subsection{Compatibility in Transformation Networks}

% As introduced in the previous section, simply put, compatibility in a transformation network means that there are no contradictions in the relations defined by the transformations. That means, there are not restrictions in the relations that prevent another relation from being fulfilled.
% To precisely specify that notion of \emph{compatibility}, the problems associated with it and prove correctness of a strategy to verify compatibility of given relations, we define our notion of consistency in terms of conditions and consistency relations and define the term \emph{compatibility} based on that.


%\subsubsection{Consistency}

The first definitions on conditions and consistency relations are based on the work of \textcite[sec. 2.3.2, 4.1.1]{kramer2017a}.
The central idea of the consistency notion is to have consistency relations, which contain pairs of objects and, broadly speaking, requires that if the objects in one side of the pair occur in a model, the others have to occur in another model as well.

\begin{definition}[Condition]
    A condition $\condition{c}{}$ for a class tuple $\classtuple{C}{\condition{c}{}} = \tupled{\class{C}{\condition{c}{},1}, \dots, \class{C}{\condition{c}{},n}}$ is a set of object tuples %$\condition{c}{}$ = \setted{\tupled{\object{o}{1}, \dots, \object{o}{n}}}$ 
    with: 
    \begin{align*}
    \formulaskip &
    \forall \tupled{\object{o}{1}, \dots, \object{o}{n}} \in \condition{c}{}: \forall i \in \setted{1, \dots, n} : \object{o}{i} \in \metamodelinstances{\class{C}{\condition{c}, i}}
    \end{align*}
    An element $\conditionelement{c}{} \in \condition{c}{}$ is called a \emph{condition element}.
    %
    For a set of models $\modelset{m} \in \metamodelinstances{\metamodelset{M}}$ of a metamodel set $\metamodelset{M}$ and %= \setted{\model{m}{1}, \dots, \model{m}{k}}$ 
    a condition element $\conditionelement{c}{}$, % = \tupled{\object{o}{1}, \dots, \object{o}{n}}$ 
    %be a condition element.
    we say that: 
    \begin{align*}
        \formulaskip &
        \modelset{m} \mathtext{contains} \conditionelement{c}{} \equivalentperdefinition
        \exists \model{m}{} \in \modelset{m} : \conditionelement{c}{} \subseteq \model{m}{}
    \end{align*}
\end{definition}

\emph{Conditions} represent object tuples that instantiate the same tuple of classes. They are supposed to occur in models that fulfill a certain condition regarding consistency, i.e., they define the objects that can occur in the previously mentioned pairs of consistency relations, which we specify later.
% \begin{definition}[Condition Element Containment] \label{def:conditionelementcontainment}
%     For a set of metamodels $\metamodelset{M}$, let $\modelset{m} \in \metamodelinstances{\metamodelset{M}}$ %= \setted{\model{m}{1}, \dots, \model{m}{k}}$ 
%     be a set of models and let $\conditionelement{c}{} = \tupled{\object{o}{1}, \dots, \object{o}{n}}$ be a condition element.
%     We say that:
%     \begin{align*}
%         \formulaskip &
%         \modelset{m} \mathtext{contains} c \equivalentperdefinition \\
%         & \formulaskip
%         \exists \model{m}{} \in \modelset{m} : \exists \object{o'}{1}, \dots, \object{o'}{n} \in \model{m}{} : \forall i \in \setted{1, \dots, n} : \object{o}{i} \subseteq \object{o'}{i}
%     \end{align*}
% \end{definition}
We say that a set of models contains a condition element if any of the models contains all the objects within the condition element. %a set of objects, in which the condition is represented. 
%This does not necessarily mean that the objects have to be exactly those of the condition, but may be a superset of them.
%That definition allows to a condition may only specify partial information of an object to match, as, for example, a consistency relation may only relate specific properties of classes instead of the whole classes.
This implies that the metamodel of such a model has to contain all the classes in the class tuple of the condition.

\todo{Relations also support the semi-automatic case: Whenever there is a choice to map a condition element to another, but no necessity, it is fine that a consistency relation does not require anything. The requirement only comes to the consistency preservation rule, which is supposed to update elements that have a corresponding element. Thus, if there is a corresponding element, then the CPR should update it upon change. This is, however, no necessity of the pure consistency relations, as the models would even be consistent if the there is no corresponding element. Thus, we see that our notion of consistency is purely induced by the models but cannot be influenced by information given from outside, such as user decisions.}
\todoLater{Call class tuple the "signature" of a relation.}
\begin{definition}[Consistency Relation]
\label{def:consistencyrelation}
    %Let $\metamodel{M}{1}, \metamodel{M}{2}$ be two metamodels. 
    %A consistency relation $\consistencyrelation{CR}{}$ is defined for two class tuples $\classtuple{C}{l,\consistencyrelation{CR}{}}$ and $\classtuple{C}{r,\consistencyrelation{CR}{}}$ 
    %with $\forall \class{C}{l,i} \in \classtuple{C}{l,\consistencyrelation{CR}{}} : \exists \class{C}{} \in \metamodel{M}{1} : \class{C}{l,i} \subseteq \class{C}{}$ and $\forall \class{C}{r,i} \in \classtuple{C}{r,\consistencyrelation{CR}{}} : \exists \class{C}{} \in \metamodel{M}{2} : \class{C}{r,i} \subseteq \class{C}{}$.
    % $\consistencyrelation{CR}{}$ is a set of pairs of condition elements
    % \begin{align*}
    %     \formulaskip
    %     \consistencyrelation{CR} = \setted{\tuple{c}{l} = \tupled{\object{o}{l,1}, \dots, \object{o}{l,n}}, \tuple{c}{r} = \tupled{\object{o}{r,1}, \dots, \object{o}{r,m}}}
    % \end{align*}
    % such that there are induced conditions $\condition{c}{l,\consistencyrelation{CR}{}}, \condition{c}{r,\consistencyrelation{CR}{}}$ 
    % \begin{align*}
    %     \exists 
    % \end{align*}
    Let $\classtuple{C}{l,\consistencyrelation{CR}{}}$ and $\classtuple{C}{r,\consistencyrelation{CR}{}}$ be two class tuples.
    A consistency relation $\consistencyrelation{CR}{}$ is a subset of pairs of condition elements in conditions $\condition{c}{l,\consistencyrelation{CR}{}}, \condition{c}{r,\consistencyrelation{CR}{}}$ with
    $\classtuple{C}{l,\consistencyrelation{CR}{}} = \classtuple{C}{\condition{c}{l,\consistencyrelation{CR}{}}}$ and $\classtuple{C}{r,\consistencyrelation{CR}{}} = \classtuple{C}{\condition{c}{r,\consistencyrelation{CR}{}}}$ :
    \begin{align*}
        \formulaskip & 
        \consistencyrelation{CR}{} \subseteq \condition{c}{l,\consistencyrelation{CR}{}} \times \condition{c}{r,\consistencyrelation{CR}{}}
    \end{align*}
    We call a pair of condition elements %$\consistencyrelationpair{cr}{} 
    $\tupled{\conditionelement{c}{l}, \conditionelement{c}{r}} \in \consistencyrelation{CR}{}$ a \emph{consistency relation pair}. 
    For a set of models $\modelset{m}$ and a consistency relation pair $\tupled{\conditionelement{c}{l}, \conditionelement{c}{r}}$, we say that:
    \begin{align*}
        \formulaskip & 
        \modelset{m} \mathtext{contains} \tupled{\conditionelement{c}{l}, \conditionelement{c}{r}} \equivalentperdefinition \modelset{m} \mathtext{contains} \conditionelement{c}{l} \land \modelset{m} \mathtext{contains} \conditionelement{c}{r}
    \end{align*}
    Without loss of generality, we assume that each condition element of both conditions occurs in at least one consistency relation pair, i.e.,
    \begin{align*}
        \formulaskip & 
        \forall \conditionelement{c}{} \in \condition{c}{l} : \exists \tupled{\conditionelement{c}{l}, \conditionelement{c}{r}} \in \consistencyrelation{CR}{} : \conditionelement{c}{} = \conditionelement{c}{l}\\
        &  
        \land \forall \conditionelement{c}{} \in \condition{c}{r} : \exists \tupled{\conditionelement{c}{l}, \conditionelement{c}{r}} \in \consistencyrelation{CR}{} : \conditionelement{c}{} = \conditionelement{c}{r}
    \end{align*}
    % these metamodels is a (possibly infinite) set of pairs of model element tuples $\consistencyrelation{CR} = \{ \bigtupled{\tupled{e_{l1}, \ldots, e_{ln}}, \tupled{e_{r1}, \ldots, e_{rm}}}, \bigtupled{\tupled{f_{l1}, \ldots, f_{ln}}, \tupled{f_{r1}, \ldots, f_{rm}}}, \newline \dots \}$ with the elements of each left tuple $e_{li}, f_{li}, \ldots \in \metamodelinstances{\metamodel{M}}$ and the elements of each right tuple $e_{ri}, f_{ri}, \ldots \in \metamodelinstances{\metamodel{N}}$.
    % A consistency relation denotes that the element tuples have to co-occur or to not occur at all in instances of $\metamodel{M}$ and $\metamodel{N}$ to consider these instances consistent.
\end{definition}

A consistency relation according to \autoref{def:consistencyrelation} is a set of pairs of object tuples, which are supposed to indicate the tuples of objects that are considered consistent with each other, i.e., if a model contains one of the left object tuples occurs in the relation one of the related right object tuples has to occur in a model as well.
It is based on two conditions that define relevant object tuples in each of the two metamodels and defines the ones that are related to each other.
%\todo{Metamodels in definition may not be different, so consistency relations within a model can be defined. May be a problem if the class tuples are overlapping, so exclude and discuss that?}

We can now define a notion of consistency for a set of models based on the definition of consistency relations.

\begin{definition}[Consistency] \label{def:consistency}
    %Let $\metamodelset{M} = \setted{ \metamodel{M}{1}, \dots, \metamodel{M}{k} }$ be a set of metamodels and 
    Let $\consistencyrelation{CR}{}$ be a consistency relation % for two of the metamodels in $\metamodelset{M}$.
    %Let 
    and let $\modelset{m} \in \metamodelinstances{\metamodelset{M}}$ %= \setted{ \model{m}{1}, \dots, \model{m}{k} }, \model{m}{i} \in \metamodelinstances{\metamodel{M}{i}}$ 
    be a set of models of the metamodels in $\metamodelset{M}$.
    We say that:
    % \begin{align*}
    %     \formulaskip
    %     & 
    %     \modelset{m} \mathtext{consistent to} \consistencyrelation{CR}{} \equivalentperdefinition \\
    %     & \formulaskip
    %     \forall \tuple{c}{l} \in \condition{c}{l, \consistencyrelation{CR}{}} \mid \modelset{m} \mathtext{contains} \conditionelement{c}{l} : \exists \conditionelement{c}{r} \in \condition{c}{r, \consistencyrelation{CR}{}} : \\
    %     & \formulaskip\formulaskip 
    %     \tupled{\conditionelement{c}{l}, \conditionelement{c}{r}} \in \consistencyrelation{CR}{} \land \modelset{m} \mathtext{contains} \conditionelement{c}{r} \\
    %     & \formulaskip
    %     \land \forall \conditionelement{c}{r} \in \condition{c}{r, \consistencyrelation{CR}{}} \mid \modelset{m} \mathtext{contains} \conditionelement{c}{r} : \exists \conditionelement{c}{l} \in \condition{c}{l, \consistencyrelation{CR}{}} : \\
    %     & \formulaskip\formulaskip 
    %     \tupled{\conditionelement{c}{l}, \conditionelement{c}{r}} \in \consistencyrelation{CR}{} \land \modelset{m} \mathtext{contains} \conditionelement{c}{l}
    % \end{align*}
     \begin{align*}
        \formulaskip
        & 
        \modelset{m} \mathtext{consistent to} \consistencyrelation{CR}{} \equivalentperdefinition \\
        & \formulaskip
        \exists \consistencyrelation{W}{} \subseteq \consistencyrelation{CR}{} : 
        \bigl( \forall \tupled{\conditionelement{c}{l,1}, \conditionelement{c}{r,1}}, \tupled{\conditionelement{c}{l,2}, \conditionelement{c}{r,2}} \in \consistencyrelation{W}{} : \\
        & \formulaskip\formulaskip\formulaskip
        \tupled{\conditionelement{c}{l,1}, \conditionelement{c}{r,1}} = \tupled{\conditionelement{c}{l,2}, \conditionelement{c}{r,2}} \lor 
        ( \conditionelement{c}{l,1} \neq \conditionelement{c}{l,2} \land \conditionelement{c}{r,1} \neq \conditionelement{c}{l,2}) \bigr) \\
        & \formulaskip\formulaskip
        \land \forall \tupled{\conditionelement{c}{l}, \conditionelement{c}{r}} \in  \consistencyrelation{W}{} : \modelset{m} \containsmath \conditionelement{c}{l} \land \modelset{m} \containsmath \conditionelement{c}{r} \\
        & \formulaskip\formulaskip
        \land \forall \conditionelement{c'}{l} \in \condition{c}{l,\consistencyrelation{CR}{}} : \modelset{m} \containsmath \conditionelement{c'}{l} \Rightarrow \conditionelement{c'}{l} \in \condition{c}{l,\consistencyrelation{W}{}}
        % & \formulaskip\formulaskip
        % \land 
        % \forall \conditionelement{c'}{r} \in \condition{c}{r,\consistencyrelation{CR}{}} \mid \modelset{m} \containsmath \conditionelement{c'}{r} : \conditionelement{c'}{r} \in \condition{c}{r,\consistencyrelation{W}{}}
    \end{align*}
    We call such a $\consistencyrelation{W}{}$ a \emph{witness structure} for consistency of $\modelset{m}$ to $\consistencyrelation{CR}{}$, and for all elements $\tupled{\conditionelement{w}{l}, \conditionelement{w}{r}} \in \consistencyrelation{W}{}$, we call $\conditionelement{w}{l}$ and $\conditionelement{w}{r}$ \emph{corresponding to} each other.
    
    For a set of consistency relations $\consistencyrelationset{CR} = \setted{\consistencyrelation{CR}{1}, \consistencyrelation{CR}{2}, \dots}$, we say that:
    \begin{align*}
        \formulaskip &
        \modelset{m} \mathtext{consistent to} \consistencyrelationset{CR} \equivalentperdefinition %\\
        %& \formulaskip
        \forall \consistencyrelation{CR}{} \in \consistencyrelationset{CR} : \modelset{m} \mathtext{consistent to} \consistencyrelation{CR}{}
    \end{align*}
\end{definition}

\begin{figure}
    \centering
    \newcommand{\hdistance}{20em}
\newcommand{\vdistance}{0.8em}
\newcommand{\internalvdistance}{0.3em}
\newcommand{\classwidth}{6em}
\newcommand{\objectwidth}{6.7em}
\newcommand{\leftshift}{5em}

\begin{tikzpicture}[
    witness/.style={consistency relation, latex-latex},
    witness fault/.style={witness, color=darkred, dashed}
]

% Employee
\umlclassvarwidth{employee}{}{Employee\sameheight}{
name
}{\classwidth}

% Resident
\umlclassvarwidth[, right=\hdistance of employee.north, anchor=north]{resident}{}{Resident\sameheight}{
name
}{\classwidth}

% CONSISTENCY RELATIONS
\draw[consistency relation] (employee.east) -- node[pos=0, above right] {$e$} node[pos=0.5, below, align=center] {
$\begin{aligned}
    \consistencyrelation{CR}{} = \setted{ \tupled{e,r} \mid e.name = r.name \\[-0.4em]
	\lor e.name.toLower = r.name }
\end{aligned}$} node[pos=1, above left] {$r$} (resident.west);

% EXAMPLE 1
% Employee
\umlobjectvarwidth[, below right=2.2*\vdistance and \leftshift of employee.south west, anchor=north west]{instance1_employee}{}{Employee\sameheight}{
	name = "Alice"
}{\objectwidth}
% Resident
\umlobjectvarwidth[, below=2.2*\vdistance of resident.south east, anchor=north east]{instance1_resident}{}{Resident\sameheight}{
	name = "Alice"
}{\objectwidth}

% EXAMPLE 2
% Employee
\umlobjectvarwidth[, below=\vdistance of instance1_employee.south, anchor=north]{instance2_employee}{}{Employee\sameheight}{
	name = "alice"
}{\objectwidth}
% Resident
\umlobjectvarwidth[, below=\vdistance of instance1_resident.south, anchor=north]{instance2_resident}{}{Resident\sameheight}{
	name = "Alice"
}{\objectwidth}


% EXAMPLE 3
% Employee
\umlobjectvarwidth[, below=\vdistance of instance2_employee.south, anchor=north]{instance3_employee1}{}{Employee\sameheight}{
	name = "alice"
}{\objectwidth}
\umlobjectvarwidth[, below=\internalvdistance of instance3_employee1.south, anchor=north]{instance3_employee2}{}{Employee\sameheight}{
	name = "Alice"
}{\objectwidth}
% Resident
\umlobjectvarwidth[, below=\vdistance of instance2_resident.south, anchor=north]{instance3_resident1}{}{Resident\sameheight}{
	name = "alice"
}{\objectwidth}
\umlobjectvarwidth[, below=\internalvdistance of instance3_resident1.south, anchor=north]{instance3_resident2}{}{Resident\sameheight}{
	name = "Alice"
}{\objectwidth}

% EXAMPLE 4
% Employee
\umlobjectvarwidth[, below=\vdistance of instance3_employee2.south, anchor=north]{instance4_employee1}{}{Employee\sameheight}{
	name = "Alice"
}{\objectwidth}
% Resident
\umlobjectvarwidth[, below=\vdistance of instance3_resident2.south, anchor=north]{instance4_resident1}{}{Resident\sameheight}{
	name = "Alice"
}{\objectwidth}
\umlobjectvarwidth[, below=\internalvdistance of instance4_resident1.south, anchor=north]{instance4_resident2}{}{Resident\sameheight}{
	name = "John"
}{\objectwidth}

% EXAMPLE 5
% Resident
\umlobjectvarwidth[, below=\vdistance of instance4_resident2.south, anchor=north]{instance5_resident1}{}{Resident\sameheight}{
	name = "Alice"
}{\objectwidth}
\umlobjectvarwidth[, below=\internalvdistance of instance5_resident1.south, anchor=north]{instance5_resident2}{}{Resident\sameheight}{
	name = "alice"
}{\objectwidth}
% Employee
\umlobjectvarwidth[, left=\hdistance-\leftshift+(\classwidth-\objectwidth) of instance5_resident1.north, anchor=north]{instance5_employee1}{}{Employee\sameheight}{
	name = "alice"
}{\objectwidth}

% EXAMPLE 6
% Resident
\umlobjectvarwidth[, below=\vdistance of instance5_resident2.south, anchor=north]{instance6_resident1}{}{Resident\sameheight}{
	name = "alice"
}{\objectwidth}
% Employee
\umlobjectvarwidth[, left=\hdistance-\leftshift+(\classwidth-\objectwidth) of instance6_resident1.north, anchor=north]{instance6_employee1}{}{Employee\sameheight}{
	name = "alice"
}{\objectwidth}
\umlobjectvarwidth[, below=\internalvdistance of instance6_employee1.south, anchor=north]{instance6_employee2}{}{Employee\sameheight}{
	name = "Alice"
}{\objectwidth}

\node[left=0.5*\leftshift of instance1_employee.west] {\textbf{1.}};
\node[left=0.5*\leftshift of instance2_employee.west] {\textbf{2.}};
\node[left=0.5*\leftshift of instance3_employee1.west] {\textbf{3.}};
\node[left=0.5*\leftshift of instance4_employee1.west] {\textbf{4.}};
\node[left=0.5*\leftshift of instance5_employee1.west] {\textbf{5.}};
\node[left=0.5*\leftshift of instance6_employee1.west] {\textbf{6.}};

\draw ([xshift=-\leftshift,yshift=0.5*\vdistance]instance1_employee.north west) -- ([yshift=0.5*\vdistance]instance1_resident.north east); 
\draw ([xshift=-\leftshift,yshift=0.5*\vdistance]instance2_employee.north west) -- ([yshift=0.5*\vdistance]instance2_resident.north east); 	 
\draw ([xshift=-\leftshift,yshift=0.5*\vdistance]instance3_employee1.north west) -- ([yshift=0.5*\vdistance]instance3_resident1.north east); 	 
\draw ([xshift=-\leftshift,yshift=0.5*\vdistance]instance4_employee1.north west) -- ([yshift=0.5*\vdistance]instance4_resident1.north east); 	 
\draw ([xshift=-\leftshift,yshift=0.5*\vdistance]instance5_employee1.north west) -- ([yshift=0.5*\vdistance]instance5_resident1.north east); 
\draw ([xshift=-\leftshift,yshift=0.5*\vdistance]instance6_employee1.north west) -- ([yshift=0.5*\vdistance]instance6_resident1.north east); 	


% WITNESS STRUCTURE
\draw[witness] (instance1_employee) -- (instance1_resident);
\draw[witness] (instance2_employee) -- (instance2_resident);
\draw[witness] (instance3_employee1) -- (instance3_resident1);
\draw[witness] (instance3_employee2) -- (instance3_resident2);
\draw[witness] (instance4_employee1) -- (instance4_resident1);
\draw[witness fault] (instance5_employee1) -- (instance5_resident1);
\draw[witness fault] (instance5_employee1) -- (instance5_resident2);
\draw[witness fault] (instance6_employee1) -- (instance6_resident1);
\draw[witness fault] (instance6_employee2) -- (instance6_resident1);

\end{tikzpicture}
    \caption[Examples for fine-grained consistency relations]{A consistency relation between employee and resident and six example model pairs: model pairs 1--4 being consistent with an appropriate witness structure $\consistencyrelation{W}{}$ shown in blue and model pair 5 and 6 being inconsistent with an inappropriate mapping structure shown in red and dashed. Taken from \owncite{klare2020compatibility-report}.}
    \label{fig:compatibility:consistency_example}
\end{figure}

A consistency relation $\consistencyrelation{CR}{}$ relates one condition element at the left side to one or more other condition elements at the right side of the relation.
The definition of consistency ensures that if one condition element $\conditionelement{c}{} \in \condition{c}{l,\consistencyrelation{CR}{}}$ in the left side of the relation occurs in a set of models, exactly one of the condition elements related to it by a consistency relation $\consistencyrelation{CR}{}$ occurs in another model to consider the set of models consistent.
%\todo{Give example why this is reasonable, i.e. a counterexample for the more simple notion of Max.}
If another element that is related to $\conditionelement{c}{}$ occurs in the models, this one has to be, in turn, related to another condition element $\conditionelement{c'}{} \in \condition{c}{l,\consistencyrelation{CR}{}}$ of the left side of condition elements by $\consistencyrelation{CR}{}$, which also occurs in the models.
Without that restriction, using the modified relation $\consistencyrelation{R}{ER}'$ in the running example defined in \autoref{fig:compatibility:three_persons_example_extended}, there may exist two employees having the same name only with different capitalizations, which are considered consistent to a single resident having the same in lower case.
The given restriction ensures that if there are several alternatives for a corresponding element to be consistent, only one is actually present.
This is usually expected, as we will see in examples in the following.
To achieve that, the definition uses an auxiliary structure $\consistencyrelation{W}{}$, which serves as a witness structure for those pairs of condition elements that co-occur in the models.

\begin{example}
The definition of consistency is exemplified in \autoref{fig:compatibility:consistency_example}, which is an alternation of an extract of \autoref{fig:compatibility:three_persons_example_extended} only considering employees and residents. Models with employees and residents are considered consistent if for each employee exactly one resident with the same name or the same name in lowercase exists.
The model pairs $1$--$3$ are obviously consistent according to the definition, because there is always a pair of objects that fulfills the consistency relation.
In model pair $4$, there is a consistent resident for each employee, but there is no appropriate employee for the resident with $name = "Alice"$. However, our definition of consistency only requires that for each condition element of the left side of the relation that appears in the models an appropriate right element occurs, but not vice versa. Thus, a relation is interpreted unidirectionally, which we will discuss in more detail in the following.
In model pair $5$, there are two residents with names in different capitalizations, which would both be considered consistent to the employee according to the consistency relation.
Comparably, in model pair $6$, there is a resident that fulfills the consistency relations for both employees, each having a different but matching capitalization. 
However, the consistency definition requires that each element in a model for which consistency is defined by a consistency relation may only have one corresponding element %, which is defined in the consistency relation, 
in the model. 
In this case, there are two residents respectively two employees that could be considered consistent to the employee respectively resident, thus there is no appropriate witness structure with a unique mapping between the elements as required by the consistency definition.
\end{example}

%For example, if the relation specifies pairs for all instances of two classes that have the same name, then if one class instance occurs in one model, the other with the same name has to occur in another model.
%Additionally, the relation could ignore the capitalization of the first letter of a class, thus relating all classes with the same name ignoring the first letter capitalization. 
%In consequence, one class would be considered consistent with two others, but only one of them would be allowed to occur to consider the models consistent.

% This means a set of models is considered consistent to a consistency relation $\consistencyrelation{CR}{}$, if the models contain one of the condition elements in one side of the consistency relations, then one of the condition elements the one is related to according to the consistency relation must occur in the models as well.
% For example, if the relation specifies pairs for all instances of two classes that have the same name, then if one class instance occurs in one model, the other with the same name has to occur in another model.
% Additionally, this definition allows that one condition elements can be considered consistent to multiple other, thus the condition element occurs in several pairs of the consistency relation, and in that case only one of them has to be fulfilled.
% Example: do not consider whether name start upper or lower case

As mentioned before, we define the notion of consistency in a unidirectional way, which means that a consistency relation may define that some elements $\conditionelement{c}{r}$ are required to occur in a set of models if some elements $\conditionelement{c}{l}$ occur, but not vice versa.
Such a unidirectional notion can also be reasonable in our example, as it could make sense to require a resident for each employee, but not every resident might also be an employee.
To achieve a bijective consistency definition, for each consistency relation $\consistencyrelation{CR}{}$ its transposed relation $\consistencyrelation{CR^T}{} = \setted{\tupled{\conditionelement{c}{l}, \conditionelement{c}{r}} \mid \tupled{\conditionelement{c}{r}, \conditionelement{c}{l}} \in \consistencyrelation{CR}{}}$ can be considered as well.
Regarding \autoref{fig:compatibility:consistency_example}, if we consider the relation between employees and residents as well as its transposed, the model pair $4$ would also be considered inconsistent, because an appropriate employee for each resident would be required by the transposed relation.
We call sets of consistency relations that contain only bijective definitions of consistency \emph{symmetric}.

\begin{definition}[Symmetric Consistency Relation Set]
    Let $\consistencyrelationset{CR}$ be a set of consistency relations.
    We say that:
    \begin{align*}
        \formulaskip &
        \consistencyrelationset{CR} \mathtext{is symmetric} \equivalentperdefinition \\
        & \formulaskip
        \forall \consistencyrelation{CR}{} \in \consistencyrelationset{CR} :
        \exists \consistencyrelation{CR'}{} \in \consistencyrelationset{CR} :
        \consistencyrelation{CR'}{} = \consistencyrelation{CR^T}{}
    \end{align*}
\end{definition}

Any description of bijective consistency relations can be achieved by defining a symmetric set of consistency relations.
We chose to define consistency in a unidirectional way due to two reasons:
\begin{properenumerate}
    \item Some relevant consistency relations are actually not bijective. 
    Apart from the simple example concerning residents and employees, this situation always occurs when objects at different levels of abstraction are related.
    Consider a relation between components and classes, requiring for each component an implementation class but not vice versa, or a relation between UML models and object-oriented code, requiring for each UML class an appropriate class in code but not vice versa.
    %
    %It may be desired that %keeping an UML model consistent with object-oriented code, 
    %for each UML class an appropriate class in object-oriented code exists, but not vice versa. 
    These relations could not be expressed if consistency relations were always considered bidirectional for determining consistency.
    \item We consider networks of consistency relations, in which, as we will see later, a combination of multiple bijective consistency relations does not necessarily imply a bijective consistency relation again. 
    Thus, we need a unidirectional notion of consistency relations anyway.
\end{properenumerate}


\end{copiedFrom} % SoSym MPM4CPS

\subsection{Equivalence to \ModelLevelConsistencyRelations}

\mnote{Necessity to validate expressiveness of fine-grained consistency notion}
The model-level consistency notion introduced in \autoref{chap:correctness:notions_consistency} is established and based on notions used by several researchers.
The given fine-grained definition of consistency is based on the insight that practical approaches to describe consistency and its preservation use fine-grained rules rather than enumerating consistent model pairs.
We did, however, only provide examples that justify specific decisions in the definitions, such as the witness structure for correspondences, but we did not argue if and why this notion has equal expressiveness.

\mnote{Fine-grained relations can always be expressed by model-level one}
To show that every set of fine-grained consistency relations can be expressed by a single \modellevelconsistencyrelation, we can use the same constructive approach that we use to define consistency according to multiple consistency relations, be they at model-level or fine-grained.
Given fine-grained consistency relations $\setted{\consistencyrelation{CR}{1}, \dots, \consistencyrelation{CR}{k}}$, we can construct an equivalent \modellevelconsistencyrelation as follows:
\begin{align*}
    \consistencyrelation{CR}{} = \setted{\modeltuple{m} \mid \modeltuple{m} \consistenttomath \setted{\consistencyrelation{CR}{1}, \dots, \consistencyrelation{CR}{k}}}
\end{align*}

\mnote{Model-level relation can, in general, not be expressed by fine-grained ones}
A \modellevelconsistencyrelation can, however, not necessarily be expressed by fine-grained consistency relations.
The most simple construction approach would define a single fine-grained consistency relation to express a \modellevelconsistencyrelation, which contains the complete models instead of extracts of them.
The definition of consistency is, however, different for the two types of relations.
While at the model-level consistency is defined as two (or more) models being in a relation (see \autoref{def:modellevelconsistency}), fine-grained consistency relations do only describe that if a left-hand side element occurs in a model, then the right-hand side has to occur in another.
If two models are considered consistent by a \modellevelconsistencyrelation, they will also be consistent to the accordingly constructed fine-grained relation, because there is a witness structure that contains exactly the two consistent models.
If there is, however, a model that is not considered consistent to any other model in the \modellevelconsistencyrelation, thus the \modellevelconsistencyrelation does not contain any pair with that model, then there will also be no such pair in the fine-grained consistency relation.
According to the definition of consistency for fine-grained relations (see \autoref{def:consistency}), if there is no condition element in the relation, then consistency is not constrained for the contained model elements.
In consequence, such a model would be considered consistent to every other model.

\mnote{Consistency relations should not restrict valid models}
While, at first, this seems inappropriate, it has a specific reason.
The specification of consistency relations does usually only constraint what has to exist in one model if something exists in another.
It does, however, not constraint which elements are allowed to occur at all.
For example, a consistency relation may define that for a component in an architecture model there has to be a corresponding class in the object-oriented design model, but it may not restrict that only components of specific names are allowed.
Such restrictions should and actually are separate specifications not related to consistency between models but restricting a model on its own.
Thus, the metamodel or some additional specification for it should provide such additional restrictions of valid models.
In our formalism, this would be a restriction of $\metamodelinstanceset{M}{}$ for a metamodel $\metamodel{M}{}$.

\mnote{Fine-grained and model-level relations are equivalent if valid models are restricted externally}
If we assume such a restriction of $\metamodelinstanceset{M}{}$, it would be the union of all instances of that metamodel at one side of the \modellevelconsistencyrelation.
Then, each \modellevelconsistencyrelation can also be defined by a fined-grained one.
A model that does not occur in the \modellevelconsistencyrelation is also not contained in $\metamodelinstanceset{M}{}$, thus is is no valid model at all.
If the model occurs in $\metamodelinstanceset{M}{}$, there is also a pair in the fine-grained consistency relation that defines that for this model an according other model has to occur.
\todo{This is only valid if there are no two models with one being contained in the other and two corresponding models with one not being contained in the other. E.g. component is mapped to class, but if additionally component has an interface then the class it is mapepd to has a different name. Then both consistency relation pairs for the component models apply but only one of the corresponding element tuples is contained in the other model.}

\mnote{Insights to any kind of relation are valid for the other as well}
Summarizing, we know that \modellevelconsistencyrelations have equal expressiveness to fine-grained consistency relations, if the restriction that specific models are not allowed to occur at all are encoded in a separate specification, such as a restriction of $\metamodelinstanceset{M}{}$ for each metamodel $\metamodel{M}{}$.
In consequence, insights based on any of the two kinds of consistency relations also apply to the other.

%\todo{Derive that element-level consistency relations induce model-level consistency relations to be conforming with previous formalism}

%\begin{itemize}
    %\item Is compatibility a subclass of correctness? Is every correct set of relations compatible as well?
    %\item Problematisch: unser Konsistenzbegriff für Relationen (feingranulare Relationen) schließt keine Modelle aus, der Konsistenzbegriff hier aber schon. Wie realisiere ich die feingranularen Relationen, die dafür sorgen, dass nur genau ein Tupel von Modellen konsistent ist?
    %\item Wir müssen bei der Ableitung unseres Kompatibilitätsbegriffes erklären, dass bei uns der vollständige Ausschluss bestimmter Modelle nicht Teil einer feingranularen Konsistenzrelation sein darf, sondern Teil einer weiteren Spezifikation, die angibt, welche Modelle überhaupt valide sind. Denn so ist es in Transformationssprachen tatsächlich auch.
%\end{itemize}



\begin{copiedFrom}

\subsection{Implicit Consistency Relations}

Each set of consistency relations defines binary consistency relations, each between two sets of classes.
However, such consistency relations imply further \emph{transitive} consistency relations.
Having one relation between classes $A$ and $B$ and one between $B$ and $C$ implies an additional relation between $A$ and $C$, for which we define a notion for the concatenation of relations.
The goal of this notion is to provide a relation that is induced by the concatenated ones. This means, if a model is consistent to the concatenation, it should also be consistent to the single relations, as otherwise the concatenation would introduce additional consistency constraints.
To achieve this, the following definition makes appropriate restrictions to the derived consistency relation pairs.

\todo{Actually, a concatenation may also consider that two or more relations are concatenated to a single one. I.e. CR1 could map something to A and B, and CR2 could map A to something and CR3 could map B to something. Then there could be a combination of all of them. In fact, each pair of consistency relations between the same metamodels can be combined to "larger" relation that then may be concatenated to other relations. Such a pair could even be a pair of a relation with itself, like if a relation maps on element to two of the same class and another relation then maps one element of the class to another. 
In summary, our notion of transitivity has to consider that concatenation may not only be sequences, but acyclic graphs.}

\begin{definition}[Consistency Relations Concatenation] \label{def:relationconcatenation}
    Let $\consistencyrelation{CR}{1}, \consistencyrelation{CR}{2}$ be two consistency relations. The concatenation is defined as follows:
    \begin{align*}
        \formulaskip &
        \consistencyrelation{CR}{} = \consistencyrelation{CR}{1} \concat \consistencyrelation{CR}{2} := \setted{\tupled{\conditionelement{c}{l}, \conditionelement{c}{r}} \mid \\
        & \formulaskip 
        \exists %\consistencyrelationpair{cr}{1} = 
        \tupled{\conditionelement{c}{l}, \conditionelement{c}{r,1}} \in \consistencyrelation{CR}{1} : \exists %\consistencyrelationpair{cr}{2} = 
        \tupled{\conditionelement{c}{l,2}, \conditionelement{c}{r}} \in \consistencyrelation{CR}{2} : %\conditionelement{c}{l,1} = \conditionelement{c}{l} \land \conditionelement{c}{r,2} = \conditionelement{c}{r} \\
        %& \formulaskip\formulaskip
        %\land 
        \conditionelement{c}{r,1} \subseteq \conditionelement{c}{l,2}\\
        & \formulaskip
        \land \forall \tupled{\conditionelement{c}{l}, \conditionelement{c'}{r,1}} \in \consistencyrelation{CR}{1} : \exists \tupled{\conditionelement{c'}{l,2}, \conditionelement{c'}{r,2}} \in \consistencyrelation{CR}{2} : \conditionelement{c'}{l,2} \subseteq \conditionelement{c'}{r,1}
        }
    \end{align*}
    with $\classtuple{C}{l,\consistencyrelation{CR}{}} = \classtuple{C}{l,\consistencyrelation{CR}{1}}$ and $\classtuple{C}{r,\consistencyrelation{CR}{}} = \classtuple{C}{r,\consistencyrelation{CR}{2}}$
\end{definition}

The concatenation of two consistency relations contains all pairs of object tuples that are related across common elements in the right respectively left side of the consistency relation pairs.
Such a concatenation may be empty.
Two requirements ensure that all models considered consistent to the concatenated relation are also consistent to the single relations:
First, it is important that a pair of consistency relations $\consistencyrelation{CR}{1}, \consistencyrelation{CR}{2}$ is only combined if the left condition element of the consistency relation pair from $\consistencyrelation{CR}{2}$ is a subset of the right condition element of the consistency relation pair $\tupled{\conditionelement{c}{l}, \conditionelement{c}{r}}$ of $\consistencyrelation{CR}{1}$.
%For example, if $\consistencyrelation{CR}{1}$ requires for an element $a$ the elements $b$ and $c$ to exist, then $\consistencyrelation{CR}{2}$ must define a relation for a subset of $b$ and $c$, such that it transitively requires the existence of further elements.
Second, it is necessary that for all elements $\conditionelement{c}{r}$ in the right side of $\consistencyrelation{CR}{1}$, to which a condition element $\conditionelement{c}{l}$ is considered consistent, there must be a matching condition element, i.e. a subset of $\conditionelement{c}{r}$, in the left condition of $\consistencyrelation{CR}{2}$.
Otherwise, in both cases the occurrence of $\conditionelement{c}{l}$ in a model set would not necessarily impose any consistency requirement by $\consistencyrelation{CR}{2}$.
In the following, we explain these two requirements at an example.

% \begin{figure}
%     \centering
%     \includegraphics[width=\columnwidth]{figures/concatenation_subset.png}
%     \caption{Two consistency relations with $\consistencyrelation{CR}{1} \concat \consistencyrelation{CR}{2} = \emptyset$ and $\consistencyrelation{CR^T}{2} \concat \consistencyrelation{CR^T}{1} \neq \emptyset$}
%     \label{fig:concatenation_subset}
% \end{figure}

% \begin{figure}
%     \centering
%     \includegraphics[width=\columnwidth]{figures/combined_concatenation_example.png}    \caption{Consistency relations $\consistencyrelation{CR}{1}$ and options $\consistencyrelation{CR}{2}, \consistencyrelation{CR'}{2}, \consistencyrelation{CR''}{2}$ with $\consistencyrelation{CR}{1} \concat \consistencyrelation{CR}{2} = \neq \emptyset$, $\consistencyrelation{CR}{1} \concat \consistencyrelation{CR'}{2} = \emptyset$, $\consistencyrelation{CR}{1} \concat \consistencyrelation{CR''}{2} = \emptyset$ and $\consistencyrelation{CR''^T}{2} \concat \consistencyrelation{CR^T}{1} \neq \emptyset$}
%     \label{fig:concatenation_example}
% \end{figure}


\begin{figure}
    \centering
    \begin{subfigure}{\textwidth}
        \centering
        \newcommand{\hdistance}{11.3em}
\newcommand{\classwidth}{4.5em}
\newcommand{\internalvdistance}{1.7em}

\begin{tikzpicture}

% Person
\umlclassvarwidth{person}{}{Person\sameheight}{
name
}{\classwidth}

% Employee
\umlclassvarwidth[, right=\hdistance of person.north, anchor=north]{resident}{}{Resident\sameheight}{
name\\
street
}{\classwidth}

%Resident and Address
\umlclassvarwidth[, right=\hdistance of resident.north, anchor=north]{employee}{}{Employee\sameheight}{
name
}{\classwidth}

\umlclassvarwidth[, below=\internalvdistance of employee.south, anchor=north]{address}{}{Address\sameheight}{
street
}{\classwidth}

\umlassociationfromto{(employee) -- node[uml role end, pos=1, above left] {address} (address)}

% CONSISTENCY RELATIONS
\draw[directed consistency relation] (person.east) -- node[pos=0, above right] {$p$} node[pos=0.5, below] {$\consistencyrelation{CR}{1}$} node[pos=1, above left] {$r$} (person.east-|resident.west);
\draw[directed consistency relation] (employee.west-|resident.east) -- node[pos=0, above right] {$r$} node[pos=0.5, below, align=left] {$\consistencyrelation{CR}{2}$ / \\ $\consistencyrelation{CR}{2}'$} node[pos=1, above left] {$e$} (employee.west);
\draw[directed consistency relation] ($(employee.west)!0.2!(employee.west-|resident.east)$) |- node[pos=1, above left] {$a$} (address.west);

\node[consistency related element, below=5em of person.west, anchor=north west] {
$\begin{aligned}
    \consistencyrelation{CR}{1} =\; & \setted{\tupled{p,r} \mid \mathvariable{p.name} = \mathvariable{r.name}}\\[0.3em]
    \consistencyrelation{CR}{2} =\; & \setted{\tupled{r,(e,a)} \mid \mathvariable{r.name} = \mathvariable{e.name} \land \mathvariable{r.street} = \mathvariable{a.street}}\\
    \consistencyrelation{CR}{2}' =\; & \setted{\tupled{r,(e,a)} \mid \tupled{r,(e,a)} \in \consistencyrelation{CR}{2} \land \mathvariable{r.street} \neq \textnormal{\enquote{}}}
\end{aligned}$
};

\end{tikzpicture}
    \end{subfigure}

    \vspace{1em}
    \begin{subfigure}{\textwidth}
        \centering
        \newcommand{\hdistance}{(12.5em+0.45*\difftoafiveimage)}
\newcommand{\classwidth}{4.5em}
\newcommand{\internalvdistance}{1.7em}

\begin{tikzpicture}

% Person
\umlclassvarwidth{person}{}{Person\sameheight}{
name
}{\classwidth}

% Employee and Location
\umlclassvarwidth[, right=\hdistance of person.north, anchor=north]{resident}{}{Resident\sameheight}{
name
}{\classwidth}

\umlclassvarwidth[, below=\internalvdistance of resident.south, anchor=north]{location}{}{Location\sameheight}{
street
}{\classwidth}

\umlassociationfromto{(resident) -- node[uml role end, pos=1, above left] {address} (location)}

%Resident and Address
\umlclassvarwidth[, right=\hdistance of resident.north, anchor=north]{employee}{}{Employee\sameheight}{
name
}{\classwidth}

\umlclassvarwidth[, below=\internalvdistance of employee.south, anchor=north]{address}{}{Address\sameheight}{
street
}{\classwidth}

\umlassociationfromto{(employee) -- node[uml role end, pos=1, above left] {address} (address)}

% CONSISTENCY RELATIONS
\draw[directed consistency relation] (person.east) -- node[pos=0, above right] {$p$} node[pos=0.5, below] {$\consistencyrelation{CR}{3}$} node[pos=1, above left] {$r$} (person.east-|resident.west);
\draw[directed consistency relation] (resident.east) -- node[pos=0, above right] {$r$} node[pos=0.5, below, align=left] {$\consistencyrelation{CR}{4}$} node[pos=1, above left] {$e$} (employee.west);
\draw[directed consistency relation, -] ($(employee.west)!0.8!(employee.west-|resident.east)$) |- node[pos=1, above right] {$l$} (location.east);
\draw[directed consistency relation] ($(employee.west)!0.2!(employee.west-|resident.east)$) |- node[pos=1, above left] {$a$} (address.west);
\filldraw[consistency related element] ($(employee.west)!0.2!(employee.west-|resident.east)$) circle (0.15em);
\filldraw[consistency related element] ($(employee.west)!0.2!(employee.west-|resident.east)$) circle (0.15em);

\node[consistency related element, below=6.3em-\isafour*1.3em of person.west, anchor=north west] {
$\begin{aligned}
    \consistencyrelation{CR}{3} =\; & \setted{\tupled{p,r} \mid p.name = r.name}\\[0.3em]
    \consistencyrelation{CR}{4} =\; & \setted{\tupled{(r,l),(e,a)} \mid r.name = e.name \land l.street = a.street}
\end{aligned}$
};

\end{tikzpicture}
    \end{subfigure}
%    \includegraphics[width=\columnwidth]{figures/consistency_concatenation_example.png} 
    %\includegraphics[width=\columnwidth]{figures/concatenation_subset.png}
    \caption[Examples for consistency relation concatenation]{Two scenarios, each with two consistency relations: 
    Consistency relations $\consistencyrelation{CR}{1}$ and two options $\consistencyrelation{CR}{2}, \consistencyrelation{CR'}{2}$ with $\consistencyrelation{CR}{1} \concat \consistencyrelation{CR}{2} \neq \emptyset$ and $\consistencyrelation{CR}{1} \concat \consistencyrelation{CR'}{2} = \emptyset$, and consistency relations $\consistencyrelation{CR}{3}$ and $\consistencyrelation{CR}{4}$ with $\consistencyrelation{CR}{3} \concat \consistencyrelation{CR}{4} = \emptyset$ and $\consistencyrelation{CR^T}{4} \concat \consistencyrelation{CR^T}{3} \neq \emptyset$. Taken from \owncite{klare2020compatibility-report}.}
    \label{fig:compatibility:concatenation_example}
\end{figure}

\begin{example}
\autoref{fig:compatibility:concatenation_example} extends the initial example (\autoref{fig:compatibility:three_persons_example_extended}) with further classes in the consistency relations, such that they do not only relate single classes to each other.
It defines an address for employees and in the second example also a location for the address of residents, which are represented in additional classes.
Both examples contains a consistency relation $\consistencyrelation{CR}{1}$ respectively $\consistencyrelation{CR}{3}$ between persons and residents, which define that for each person a resident with the same name has to exist.
The examples provide different options for consistency relation between residents (with locations) and employees with addresses ($\consistencyrelation{CR}{2}, \consistencyrelation{CR'}{2}, \consistencyrelation{CR}{4}$), which exemplify the necessity for the restrictions in \autoref{def:relationconcatenation}:
\begin{enumerate}
    \item $\consistencyrelation{CR}{1} \concat \consistencyrelation{CR}{2}$: 
$\consistencyrelation{CR}{2}$ requires for each resident an employee with the same name and an address with an arbitrary street name.
In consequence, $\consistencyrelation{CR}{1} \concat \consistencyrelation{CR}{2}$ defines a relation for each person with an employee having the same name and all addresses with possible street names.
All models that are consistent to the concatenation are also consistent to the single relations.
    \item $\consistencyrelation{CR}{1} \concat \consistencyrelation{CR'}{2}$: 
$\consistencyrelation{CR'}{2}$ is similar to $\consistencyrelation{CR}{2}$ but additionally requires that the street of a resident must not be empty. 
In consequence, for a resident with an empty address it is not required that an employee exists.
This results in $\consistencyrelation{CR}{1} \concat \consistencyrelation{CR'}{2} = \emptyset$, because for any person, there must not be an employee, as the person can be consistent to a resident with an empty street name.
This shows the necessity of the second restriction in the definition. 
    \item $\consistencyrelation{CR}{3} \concat \consistencyrelation{CR}{4}$: 
The concatenation $\consistencyrelation{CR}{3} \concat \consistencyrelation{CR}{4}$ is obviously empty, because $\consistencyrelation{CR}{3}$ requires a resident for each person, but $\consistencyrelation{CR}{4}$ only requires an employee if there is also a location.
Such a location does not necessarily exist if a person %and thus a resident
exists, thus if the models are consistent to $\consistencyrelation{CR}{3}$ and $\consistencyrelation{CR}{4}$ there must not necessarily be an employee for any contained person.
This shows the necessity for the first restriction in \autoref{def:relationconcatenation}, which would require a left condition element from $\consistencyrelation{CR}{4}$ (resident and location) to be a subset of a right condition element in $\consistencyrelation{CR}{3}$ (resident). %, which is never the case. % because of $\consistencyrelation{CR}{4}$ requiring more elements than $\consistencyrelation{CR}{3}$ ensures to exist or a person.
%This shows the necessity of the first restriction in the definition, which requires for all persons that for each consistent resident according to $\consistencyrelation{CR}{3}$, there is also a condition in $\consistencyrelation{CR}{4}$ that requires an employee to exist.
%However, generally speaking, in this case the left condition elements of the second relation are a subset of those of the right side of the first relation, which means that the first relation does never require all elements to exist that are necessary for the second relation to require existence of any further elements.
    \item $\consistencyrelation{CR^T}{4} \concat \consistencyrelation{CR^T}{3}$: 
The concatenation of the transposed relations $\consistencyrelation{CR^T}{4} \concat \consistencyrelation{CR^T}{3}$ is not empty, but actually contains all combinations of each possible employee with all addresses and relates them to a person with the same name.
This is reasonable, because $\consistencyrelation{CR^T}{4}$ requires for all existing employees and addresses that an appropriate resident with the same name %(and also a location)
has to exist, which then requires a person with that name to exist due to $\consistencyrelation{CR^T}{3}$.
The definition does only cover that due to its first restriction, because $\conditionelement{c}{l,2}$, i.e., the resident related to a person by $\consistencyrelation{CR^T}{3}$ is a subset of $\conditionelement{c}{r,1}$, i.e., a tuple of resident and location.
\end{enumerate}
\end{example}

% The exemplary consistency relation show why it is necessary that the left condition element of the second consistency relation pair needs to be a subset of the right condition element of the first consistency relation pair when concatenating them.
% The concatenation $\consistencyrelation{CR}{1} \concat \consistencyrelation{CR}{2}$ is obviously empty, because for each resident a person is required, but for $\consistencyrelation{CR}{2}$ to require an employee, there must always be a location.
% Such an address does not necessarily exist if a resident and thus a person exists, thus if the models are consistent to $\consistencyrelation{CR}{1}$.
% In consequence, there must not always be an employee if a resident exists.
% The concatenation $\consistencyrelation{CR^T}{2} \concat \consistencyrelation{CR^T}{1}$ is not empty, but actually contains all combinations of each possible employee with all addresses and relates them to a resident with the same name.
% This is reasonable, because $\consistencyrelation{CR^T}{2}$ requires for all existing employees and addresses that an appropriate person with the same name (and also a location) have to exist, which then requires a resident to exist due to $\consistencyrelation{CR^T}{1}$.
% The definition does only cover that, because $\conditionelement{c}{l,2}$, i.e. the person related to a resident by $\consistencyrelation{CR^T}{1}$ is a subset of $\conditionelement{c}{r,1}$, i.e. a tuple of person and location.
 
% \begin{figure}
%     \centering
%     \includegraphics[width=\columnwidth]{figures/consistency_concatenation_example.png}
%     \caption{Two consistency relations with two alternatives for $\consistencyrelation{CR}{2}$ with $\consistencyrelation{CR}{1} \concat \consistencyrelation{CR}{2} \neq \emptyset$ and $\consistencyrelation{CR}{1} \concat \consistencyrelation{CR'}{2} = \emptyset$}
%     \label{fig:concatenation_example}
% \end{figure}

% \autoref{fig:concatenation_example} exemplifies the definition of concatenation and the necessity for its restrictions.
% Considering $\consistencyrelation{CR}{1} \concat \consistencyrelation{CR}{2}$, this concatenation relates all residents to employees with the same name and all addresses with any names. 
% So each resident is considered consistent to an employee with the same name and the existence of an address with any street name.
% This is reasonable, because $\consistencyrelation{CR}{1}$ requires a person with an arbitrary address to exist for each resident with the same name. Additionally $\consistencyrelation{CR}{2}$ requires an employee with the same name and an appropriate address to exist.
% For any person to which a resident is considered consistent, an appropriate employee and address exist, which are considered consistent, so the concatenation contains those elements, so $\consistencyrelation{CR}{1} \concat \consistencyrelation{CR}{2}$ considers all models consistent that are also consistent to the single relations.
% In contrast, $\consistencyrelation{CR'}{2}$ further restricts $\consistencyrelation{CR}{2}$ by requiring that the street name must not be empty. 
% In consequence, persons with an empty street name do not need to have an appropriate employee and address to be considered consistent.
% In consequence, the concatenation $\consistencyrelation{CR}{1} \concat \consistencyrelation{CR'}{2}$ is empty, because a resident does not necessarily require an employee to exist, because if a person with the same name and an empty street name exist, $\consistencyrelation{CR}{1}$ and $\consistencyrelation{CR'}{2}$ do not require an employee exists.
% This motivates the necessity for the last restriction in the definition of concatenation, which requires for all residents that for each consistent person according to $\consistencyrelation{CR}{1}$, there is also a condition in $\consistencyrelation{CR'}{2}$ that requires an employee to exist.

%Requiring that there must only be a partial overlap in the related elements, i.e. $\conditionelement{c}{r,1} \cap \conditionelement{c}{l,2} \neq \emptyset$ would lead to a combined consistency relation $\consistencyrelation{CR}{}$ that restricts consistency in comparison to the combined relations $\consistencyrelation{CR}{1}$ and $\consistencyrelation{CR}{2}$.

%If there is no overlap between two relations, i.e. they have no elements in common that they put into relation, then the concatenation of them is empty by definition.

\todoLater{Maybe readd overlapping definition}
% To state when consistency relations are overlapping with each other in the sense that they potentially have elements in common, we define when we denote consistency relations as \emph{overlapping}.

% \begin{definition}[Overlapping Consistency Relations]
%     Let $\consistencyrelation{CR}{1}$ and $\consistencyrelation{CR}{2}$ be two consistency relations. We say that:
%     \begin{align*}
%         \formulaskip &
%         \consistencyrelation{CR}{1} \mathtext{is overlapping with} \consistencyrelation{CR}{2} \equivalentperdefinition \\
%         & \formulaskip
%         \exists \class{C}{} \in \classtuple{C}{l,\consistencyrelation{CR}{1}} : \exists \class{C'}{} \in \classtuple{C}{l,\consistencyrelation{CR}{2}} : \class{C}{} \cap \class{C'}{} \neq \emptyset \\
%         & \formulaskip 
%         \land \exists \class{C}{} \in \classtuple{C}{r,\consistencyrelation{CR}{1}} \exists \class{C'}{} \in \classtuple{C}{r,\consistencyrelation{CR}{2}} : \class{C}{} \cap \class{C'}{} \neq \emptyset
%     \end{align*}
% \end{definition}

% \todoHeiko{Overlapping is not needed right now}

% Consistency relation are considered \emph{overlapping}, if they relate classes that have an overlap in their properties in both sides of the relations. 
% \todoHeiko{Add an example for non-trivial overlap here!}

We can formally show that the defined notion of concatenation does not lead to any restriction of consistency regarding the single relations:

\begin{lemma} \label{lemma:concatenationimpliesconsistency}
    Let $\consistencyrelation{CR}{1}, \consistencyrelation{CR}{2}$ be two consistency relations and let $\consistencyrelation{CR}{} = \consistencyrelation{CR}{1} \concat \consistencyrelation{CR}{2}$ be their concatenation. For all model sets $\modelset{m} \in \metamodelinstances{\metamodelset{M}}$ the following statement holds:
    \begin{align*}
        \formulaskip &
        \modelset{m} \consistenttomath \setted{\consistencyrelation{CR}{1}, \consistencyrelation{CR}{2}} \Rightarrow \modelset{m} \consistenttomath \consistencyrelation{CR}{}
    \end{align*}
\end{lemma}

\begin{proof}
    For any set of models $\modelset{m}$ that is consistent to $\consistencyrelation{CR}{1}$ and $\consistencyrelation{CR}{2}$, take the witness structure $\consistencyrelation{W}{1}$ that witnesses consistency of $\modelset{m}$ to $\consistencyrelation{CR}{1}$ and $\consistencyrelation{W}{2}$ that witnesses consistency of $\modelset{m}$ to $\consistencyrelation{CR}{2}$.
    Now consider the composed witness structure $\consistencyrelation{W}{} = \consistencyrelation{W}{1} \concat \consistencyrelation{W}{2}$.
    Let us assume there were $\tupled{\conditionelement{c}{l}, \conditionelement{c}{r}}, \tupled{\conditionelement{c'}{l}, \conditionelement{c'}{r}} \in \consistencyrelation{W}{}$ with $\conditionelement{c}{l} = \conditionelement{c'}{l}$ and $\conditionelement{c}{r} \neq \conditionelement{c'}{r}$.
    Per definition $\conditionelement{c}{l}$ only occurs in one $\tupled{\conditionelement{c}{l}, \conditionelement{c}{r,1}} \in \consistencyrelation{W}{1}$.
    So there must be two $\tupled{\conditionelement{c}{l,2}, \conditionelement{c}{r}}, \tupled{\conditionelement{c'}{l,2}, \conditionelement{c'}{r}} \in \consistencyrelation{CR}{2}$ with $\conditionelement{c}{l,2} \subseteq \conditionelement{c}{r,1}$ and $\conditionelement{c'}{l,2} \subseteq \conditionelement{c}{r,1}$.
    However, since $\conditionelement{c}{l,2}$ and $\conditionelement{c'}{l,2}$ contain instances of the same classes and are both subsets of the same other object tuples $\conditionelement{c}{r,1}$, we have $\conditionelement{c}{l,2} = \conditionelement{c'}{l,2}$.
    So we know that:
    \begin{align*}
        \formulaskip &
        \forall \tupled{\conditionelement{c}{l,1}, \conditionelement{c}{r,1}}, \tupled{\conditionelement{c}{l,2}, \conditionelement{c}{r,2}} \in \consistencyrelation{W}{} : \\
        & \formulaskip
        \tupled{\conditionelement{c}{l,1}, \conditionelement{c}{r,1}} = \tupled{\conditionelement{c}{l,2}, \conditionelement{c}{r,2}} \lor \conditionelement{c}{l,1} \neq \conditionelement{c}{l,2} \land \conditionelement{c}{r,1} \neq \conditionelement{c}{l,2}
    \end{align*}
    Additionally, since $\consistencyrelation{W}{1}$ and $\consistencyrelation{W}{2}$ are witness structures for consistency of $\modelset{m}$ to $\consistencyrelation{CR}{1}$ and $\consistencyrelation{CR}{2}$, the model set contains all condition elements in $\consistencyrelation{W}{1}$ and $\consistencyrelation{W}{2}$.
    Consequentially, $\modelset{m}$ also contains the condition elements in $\consistencyrelation{W}{}$, as those in $\consistencyrelation{W}{}$ are composed of the ones in $\consistencyrelation{W}{1}$ and $\consistencyrelation{W}{2}$. This implies that:
    \begin{align*}
        \formulaskip &
        \forall \tupled{\conditionelement{c}{l}, \conditionelement{c}{r}} \in  \consistencyrelation{W}{} : \modelset{m} \containsmath \conditionelement{c}{l} \land \modelset{m} \containsmath \conditionelement{c}{r}
    \end{align*}
    Finally, let us assume that: 
    \begin{align*}
        \formulaskip &
        \exists \conditionelement{c'}{l} \in \condition{c}{l,\consistencyrelation{CR}{}} : \modelset{m} \containsmath \conditionelement{c'}{l} \land \conditionelement{c'}{l} \not\in \condition{c}{l,\consistencyrelation{W}{}}
    \end{align*}
    We know that $\condition{c}{l,\consistencyrelation{CR}{}} \subseteq \condition{c}{l,\consistencyrelation{CR}{1}}$, because the left condition elements in $\consistencyrelation{CR}{}$ are taken from the left condition elements in $\consistencyrelation{CR}{1}$ per definition and thus also contained $\consistencyrelation{CR}{1}$.
    Since $\modelset{m} \containsmath \conditionelement{c'}{l}$, there must be a consistency relation pair $\tupled{\conditionelement{c'}{l}, \conditionelement{c'}{r,1}} \in \consistencyrelation{W}{1}$, which witnesses consistency of $\conditionelement{c'}{l}$ according to $\consistencyrelation{CR}{1}$.
    There must be at least one consistency relation pair $\tupled{\conditionelement{c'}{l,2}, \conditionelement{c'}{r,2}} \in \consistencyrelation{CR}{2}$ with $\conditionelement{c'}{l,2} \subseteq \conditionelement{c'}{r,1}$, because otherwise $\conditionelement{c'}{l}$ would per definition not occur in the left condition of $\consistencyrelation{CR}{}$.
    For all such tuples $\tupled{\conditionelement{c'}{l,2}, \conditionelement{c'}{r,2}}$, we know that $\modelset{m} \containsmath \conditionelement{c'}{l,2}$, because $\modelset{m} \containsmath \conditionelement{c'}{r,1}$ due to its containment in $\consistencyrelation{W}{1}$ and due to $\conditionelement{c'}{l,2} \subseteq \conditionelement{c'}{r,1}$.
    In consequence, consistency to $\consistencyrelation{CR}{2}$ requires that for one of those $\conditionelement{c'}{r,2}$ it holds that $\modelset{m} \containsmath \conditionelement{c'}{r,2}$ and that there is $\tupled{\conditionelement{c'}{l,2}, \conditionelement{c'}{r,2}} \in \consistencyrelation{W}{2}$ that witnesses this consistency.
    Summarizing, due to $\tupled{\conditionelement{c'}{l}, \conditionelement{c'}{r,1}} \in \consistencyrelation{W}{1}$ and $\tupled{\conditionelement{c'}{l,2}, \conditionelement{c'}{r,2}} \in \consistencyrelation{W}{2}$ with $\conditionelement{c'}{l,2} \subseteq \conditionelement{c'}{r,1}$ and due to the definition of $\consistencyrelation{W}{}$ as the concatenation of $\consistencyrelation{W}{1}$ and $\consistencyrelation{W}{2}$, we know that $\tupled{\conditionelement{c'}{l}, \conditionelement{c'}{r,2}} \in \consistencyrelation{W}{}$, which breaks our assumption.
    So we have shown that:
    \begin{align*}
        \formulaskip &
        \forall \conditionelement{c'}{l} \in \condition{c}{l,\consistencyrelation{CR}{}} \mid \modelset{m} \containsmath \conditionelement{c'}{l} : \conditionelement{c'}{l} \in \condition{c}{l,\consistencyrelation{W}{}}
    \end{align*}
    Summarizing, we have shown that $\consistencyrelation{W}{}$ fulfills all requirements to a witness structure according to \autoref{def:consistency} for $\modelset{m}$ being consistent to $\consistencyrelation{CR}{}$, so we know that $\modelset{m} \consistenttomath \consistencyrelation{CR}{}$.
    % If any set of models $\modelset{m}$ is inconsistent to $\consistencyrelation{CR}{}$, then
    % \begin{align*}
    %     \formulaskip &
    %     \exists \tuple{c}{l} \in \condition{c}{l, \consistencyrelation{CR'}{}} \mid \modelset{m} \mathtext{contains} \conditionelement{c}{l} : \\
    %     & \formulaskip %\label{eq:consistencytransitivenoncontainment}
    %     \forall \conditionelement{c}{r} \in \condition{c}{r, \consistencyrelation{CR'}{}} \mid \tupled{\conditionelement{c}{l}, \conditionelement{c}{r}} \in \consistencyrelation{CR}{} : \neg (\modelset{m} \mathtext{contains} \conditionelement{c}{r}) \\
    %     & \formulaskip %\label{eq:consistencytransitiveduplicatecontainment}
    %     \lor \exists \conditionelement{c'}{r} \in \condition{c}{r, \consistencyrelation{CR'}{}} \setminus \setted{\conditionelement{c}{r}} \mid \tupled{\conditionelement{c}{l}, \conditionelement{c'}{r}} \in \consistencyrelation{CR'}{}: \modelset{m} \mathtext{contains} \conditionelement{c'}{r}
    % \end{align*}
    % \begin{enumerate}
    %     \item Assume that there is no condition element $\conditionelement{c}{r}$, such that $\modelset{m} \containsmath \conditionelement{c}{r}$.
    %     Due to $\consistencyrelation{CR}{}$ being a concatenation of $\consistencyrelation{CR}{1}, \dots, \consistencyrelation{CR}{k}$, for every consistency relation pair $\tupled{\conditionelement{c}{l}, \conditionelement{c}{r}} \in \consistencyrelation{CR}{}$, there is a sequence of consistency relation pairs $\tupled{\conditionelement{c}{l,1}, \conditionelement{c}{r,1}} \in \consistencyrelation{CR}{1}, \dots, \tupled{\conditionelement{c}{l,k}, \conditionelement{c}{r,k}} \in \consistencyrelation{CR}{k}$, such that there is an overlap in the pairs of each sequential consistency relation pair, i.e. $\exists \object{o}{1} \in \conditionelement{c}{r,i}, \object{o}{2} \in \conditionelement{c}{l,i+1} : \object{o}{1} \cap \object{o}{2} \neq \emptyset$.
        
       
    %     \item Assume that there are at least two condition elements $\conditionelement{c}{r}, \conditionelement{c'}{r}$, such that $\modelset{m} \containsmath \conditionelement{c}{r}$ and $\modelset{m} \containsmath \conditionelement{c'}{r}$.
        
    % \end{enumerate}
    
    
    % If any set of models $\modelset{m}$ is consistent to $\setted{\consistencyrelation{CR}{1}, \dots, \consistencyrelation{CR}{k}}$, then
    % \begin{align*}
    %     \formulaskip &
    %     \forall \consistencyrelation{CR'}{} \in \setted{\consistencyrelation{CR}{1}, \dots, \consistencyrelation{CR}{k}} : \\
    %     & \formulaskip 
    %     \forall \tuple{c}{l} \in \condition{c}{l, \consistencyrelation{CR'}{}} \mid \modelset{m} \mathtext{contains} \conditionelement{c}{l} : \exists \conditionelement{c}{r} \in \condition{c}{r, \consistencyrelation{CR'}{}} \mid \tupled{\conditionelement{c}{l}, \conditionelement{c}{r}} \in \consistencyrelation{CR}{} : \\
    %     & \formulaskip\formulaskip 
    %     \modelset{m} \mathtext{contains} \conditionelement{c}{r} \\
    %     & \formulaskip\formulaskip
    %     \land \forall \conditionelement{c'}{r} \in \condition{c}{r, \consistencyrelation{CR'}{}} \setminus \setted{\conditionelement{c}{r}} \mid \tupled{\conditionelement{c}{l}, \conditionelement{c'}{r}} \in \consistencyrelation{CR'}{}: \neg \modelset{m} \mathtext{contains} \conditionelement{c'}{r}
    % \end{align*}
    % This especially holds for all $\conditionelement{c}{l} \in \condition{c}{l,\consistencyrelation{CR}{1}} \mid \modelset{m} \containsmath \conditionelement{c}{l}$.
    % Consider the respective condition elements $\conditionelement{c}{r}$, which $\modelset{m}$ contains as well. If there is a $\conditionelement{c'}{l} \in \condition{c}{l,\consistencyrelation{CR}{2}}$ with $\exists \object{o}{1} \in \conditionelement{c}{r} : \exists \object{o}{2} \in \conditionelement{c'}{l}$ such that $\object{o}{1}$
    % tba \todoHeiko{Add proof}
\end{proof}

% Having shown that our definition of consistency relation concatenation is well-defined in the sense that it does not introduce further restrictions for consistency, we are able to show that the transitive closure of a consistency relation set does also not restrict consistency in comparison to the set of consistency relations itself.

We can use this notion of concatenation to define a transitive closure for sets of consistency relations, which contains all relations in that set complemented by all possible concatenations of them, i.e., \emph{implicit relations} of that set.
Having shown that our definition of consistency relations concatenation is well-defined in the sense that it does not introduce further restrictions for consistency, we are also able to show that the transitive closure does not restrict consistency in comparison to the set of consistency relation itself.


\begin{definition}[Transitive Closure of Consistency Relations] \label{def:transitiveclosure}
    Let $\consistencyrelationset{CR}$ be a set of consistency relations.
    We define its transitive closure $\transitiveclosure{\consistencyrelationset{CR}}$ as:
    \begin{align*}
        \formulaskip
        \transitiveclosure{\consistencyrelationset{CR}{}} = \setted{\consistencyrelation{CR}{} \mid & \exists \consistencyrelation{CR}{1}, \dots, \consistencyrelation{CR}{k} \in \consistencyrelationset{CR}{} : \\
        &
        \consistencyrelation{CR}{} = \consistencyrelation{CR}{1} \concat \dots \concat \consistencyrelation{CR}{k} }
    \end{align*}
\end{definition}

The transitive closure of a set of consistency relations $\consistencyrelationset{CR}$ contains all consistency relations of $\consistencyrelationset{CR}$ and all concatenations of relations in $\consistencyrelationset{CR}$. That means, the transitive closure contains consistency relations that relate all elements that are directly or indirectly related due to $\consistencyrelationset{CR}$.

The transitive closure of a consistency relation set does not further restrict consistency in comparison to the original set by construction of concatenation, i.e., if a model set is consistent to a set of consistency relations, it is also consistent to their transitive closure.
We show that in the following by first extending the argument of \autoref{lemma:concatenationimpliesconsistency}, which shows that concatenation does not further restrict consistency, to the transitive closure, which is only a set of concatenations of consistency relations.

\begin{lemma}
    Let $\consistencyrelationset{CR}$ be a set of consistency relations for a set of metamodels $\metamodelset{M}$. Then:
    \begin{align*}
        \formulaskip &
        \forall \consistencyrelation{CR}{} \in \transitiveclosure{\consistencyrelationset{CR}{}} \setminus \consistencyrelationset{CR} :
        \exists \consistencyrelation{CR}{1}, \dots, \consistencyrelation{CR}{k} \in \consistencyrelationset{CR} : \forall \modelset{m} \in \metamodelinstances{\metamodelset{M}} : \\
        & \formulaskip
        \modelset{m} \consistenttomath \setted{\consistencyrelation{CR}{1}, \dots \consistencyrelation{CR}{k}} \Rightarrow \modelset{m} \consistenttomath \consistencyrelation{CR}{} 
    \end{align*}
\end{lemma}

\begin{proof}
    Per definition, any $\consistencyrelation{CR}{} \in \transitiveclosure{\consistencyrelationset{CR}}$ is a concatenation of consistency relations in $\consistencyrelationset{CR}$, i.e.
    \begin{align*}
        \formulaskip &
        \forall \consistencyrelation{CR}{} \in \transitiveclosure{\consistencyrelationset{CR}} : \exists \consistencyrelation{CR}{1}, \dots, \consistencyrelation{CR}{k} \in \consistencyrelationset{CR} : \\
        & \formulaskip 
        \consistencyrelation{CR}{} = \consistencyrelation{CR}{1} \concat \dots \concat \consistencyrelation{CR}{k}
    \end{align*}
    We already know for any two consistency relations $\consistencyrelation{CR}{1}, \consistencyrelation{CR}{2}$ and all model sets $\modelset{m}$ that if $\modelset{m} \consistenttomath \setted{\consistencyrelation{CR}{1}, \consistencyrelation{CR}{2}}$, then $\modelset{m} \consistenttomath \consistencyrelation{CR}{1} \concat \consistencyrelation{CR}{2}$ due to \autoref{lemma:concatenationimpliesconsistency}.
    Inductively applying that argument to $\consistencyrelation{CR}{1}, \dots, \consistencyrelation{CR}{k}$ shows that for all models $\modelset{m}$ with $\modelset{m} \consistenttomath \setted{\consistencyrelation{CR}{1}, \dots, \consistencyrelation{CR}{k}}$ we know that $\modelset{m} \consistenttomath \consistencyrelation{CR}{}$.
\end{proof}

As a direct result of the previous lemma, we can now show that the transitive closure of a consistency relation set considers the same sets of models consistent as the consistency relation set itself.

\begin{lemma} \label{lemma:consistencytransitiveclosure}
    Let $\consistencyrelationset{CR}$ be a set of consistency relations for a set of metamodels $\metamodelset{M}$.
    Then for all sets of models $\modelset{m} \in \metamodelinstances{\metamodelset{M}}$ it is true that:
    \begin{align*}
        \formulaskip
        \modelset{m} \mathtext{consistent to} \consistencyrelationset{CR} \equivalent
        \modelset{m} \mathtext{consistent to} \transitiveclosure{\consistencyrelationset{CR}}
    \end{align*}
\end{lemma}

\begin{proof}
    Adding a consistency relation to a set of consistency relations can never lead to a relaxation of consistency, i.e., models becoming consistent that were not considered consistent before. This is a direct consequence of \autoref{def:consistency} for consistency, which requires models be consistent to all consistency relations in a set to be considered consistent, thus only restricting the set of consistent model sets by adding further consistency relations.
    In consequence, it holds that:
    \begin{align*}
        \formulaskip
        \modelset{m} \mathtext{consistent to} \transitiveclosure{\consistencyrelationset{CR}} \Rightarrow \modelset{m} \mathtext{consistent to} \consistencyrelationset{CR}
    \end{align*}
    Due to \autoref{lemma:consistencytransitiveclosure}, we know that a set of models that is consistent to $\consistencyrelationset{CR}$ is always consistent to all transitive relations in $\transitiveclosure{\consistencyrelationset{CR}}$ as well. Thus, we know that:
    \begin{align*}
        \formulaskip
        \modelset{m} \mathtext{consistent to} \consistencyrelationset{CR} \Rightarrow
        \modelset{m} \mathtext{consistent to} \transitiveclosure{\consistencyrelationset{CR}}
    \end{align*}
    In consequence, models are considered consistent equally for $\consistencyrelationset{CR}$ and its transitive closure $\transitiveclosure{\consistencyrelationset{CR}}$.
\end{proof}


\subsection{Compatibility}

Based on the fine-grained notion of consistency in terms of consistency relations, we can know precisely formulate our initially informal notion of \emph{compatibility} of consistency relations.
We stated that we consider consistency relation incompatible if they are somehow contradictory, like the relation between names in our initial example in \autoref{fig:compatibility:three_persons_example_extended}.
In that example, for residents with non-lowercase names no consistent set of models could be derived.
To capture that in a definition, we consider relations compatible if for all condition elements in the consistency relations, i.e., for every tuple of objects for which consistency is somehow constrained by requiring further elements to exist in a set of models to consider it consistent, a consistent model containing those objects can be found. In consequence, a consistency relation is not allowed to prevent objects for which other relations specify consistency from existing in consistent models.

\begin{definition}[Compatibility] \label{def:compatibility}
    Let $\consistencyrelationset{CR}$ be a set of consistency relations for a set of metamodels $\metamodelset{M}$. % = \setted{\metamodel{M}{1}, \dots \metamodel{M}{k}}$.
    We say that:
    \begin{align*}
        \formulaskip &
        \consistencyrelationset{CR} \mathtext{compatible} \equivalentperdefinition \\
        & \formulaskip
        \forall \consistencyrelation{CR}{} \in \consistencyrelationset{CR} : \forall \conditionelement{c}{} \in \condition{c}{l, \consistencyrelation{CR}{}} %\cup \condition{c}{r, \consistencyrelation{CR}{}} 
        : \exists \modelset{m} \in \metamodelinstances{\metamodelset{M}} : \\
        & \formulaskip \formulaskip
        \modelset{m} \mathtext{contains} \conditionelement{c}{} \land \modelset{m} \mathtext{consistent to} \consistencyrelationset{CR}
        % \forall \consistencyrelation{CR}{} \in \consistencyrelationset{CR} : \forall %\consistencyrelationpair{cr}{} = 
        % \tupled{\conditionelement{c}{l}, \conditionelement{c}{r}} \in \consistencyrelation{CR}{} : \exists \modelset{m} \in \metamodelinstances{\metamodelset{M}} : \\
        % & \formulaskip \formulaskip
        % \modelset{m} \mathtext{contains} \tupled{\conditionelement{c}{l}, \conditionelement{c}{r}} \land \modelset{m} \mathtext{consistent to} \consistencyrelationset{CR}
    \end{align*}
    We call a set of consistency relation $\consistencyrelationset{CR}$ \emph{incompatible} if it does not fulfill the definition of compatibility.
\end{definition}

%According to that definition, selecting any condition element (i.e. an object tuple) that occurs in the left side of a consistency relation pair, thus requiring another condition element to occur in a set of model to consider it consistent, it must be possible to derive a set of models that contains that object tuple and is considered consistent. 
\autoref{def:compatibility} formalizes the notion of \emph{non-contradictory} relations by requiring that a relation may not restrict that an object tuple, for which consistency is defined in any consistency relation, cannot occur in a model set anymore.
We exemplify this notion of compatibility on an extract of the initial example with different consistency relations.

% According to that definition, selecting any pair of object tuples from any consistency relation, it must be possible to derive a set of models that contains those tuples and is considered consistent. This formalizes the notion of \emph{non-contradictory} relations, as no relation restricts that an element of another relation cannot be fulfilled anymore.

\begin{figure}
    \centering
    \newcommand{\hdistance}{14em}
\newcommand{\classwidth}{6em}

\begin{tikzpicture}

% Person
\umlclassvarwidth{person}{}{Person\sameheight}{
firstname\\
lastname
}{\classwidth}

% Employee
\umlclassvarwidth[,above right=4em and \hdistance of person.center, anchor=south]{employee}{}{Employee\sameheight}{
name
}{\classwidth}

\umlclassvarwidth[,right=\hdistance of person.south, anchor=south]{resident}{}{Resident\sameheight}{
name
}{\classwidth}


% CONSISTENCY RELATIONS
\draw[directed consistency relation] (person.north) |- node[pos=0, above right] {$p$} node[pos=0.5, above right] {$\consistencyrelation{CR}{2}$ / $\consistencyrelation{CR}{2}'$ / $\consistencyrelation{CR}{2}''$} node[pos=1, below left] {$e$} (employee.west);
\draw[directed consistency relation] (employee.south) -- node[pos=0, below left] {$e$} node[right, align=left] {$\consistencyrelation{CR}{3}$  / $\consistencyrelation{CR}{3}'$} node[pos=1, above left] {$r$} (resident.north);
\draw[directed consistency relation] (resident.west-|person.east) -- node[pos=0, above right] {$p$} node[pos=0.5, below] {$\consistencyrelation{CR}{1}$} node[pos=1, above left] {$r$} (resident.west);

\node[consistency related element, below left=1.5em and 1em of person.south west, anchor=north west] {
$\begin{aligned}
    &
    \consistencyrelation{CR}{1} = \setted{\tupled{p,r} \mid \mathvariable{r.name} = \mathvariable{p.firstname} + \textnormal{\enquote{\textvisiblespace}} + \mathvariable{p.lastname}}\\[0.5em]
    &
    \consistencyrelation{CR}{2} = \setted{\tupled{p,e} \mid \mathvariable{e.name} = \mathvariable{p.firstname} + \textnormal{\enquote{\textvisiblespace}} + \mathvariable{p.lastname}}\\
    &
    \consistencyrelation{CR}{2}' = \setted{\tupled{p,e} \mid \mathvariable{e.name} = \mathvariable{p.firstname} + \textnormal{\enquote{,\textvisiblespace}} + \mathvariable{p.lastname}}\\
    &
    \consistencyrelation{CR}{2}'' = \setted{\tupled{p,e} \mid \mathvariable{e.name} = \mathvariable{p.lastname} + \textnormal{\enquote{\textvisiblespace}} + \mathvariable{p.firstname}}\\[0.5em]
    &
    \consistencyrelation{CR}{3} = \setted{\tupled{r,e} \mid \mathvariable{r.name} = \mathvariable{e.name}}\\
    &
    \consistencyrelation{CR}{3}' = \setted{\tupled{r,e} \mid \mathvariable{r.name} = \mathvariable{e.name.toLower}}
\end{aligned}$
};

\end{tikzpicture}
    %\includegraphics[width=\columnwidth]{figures/incompatibility_example.png}
    \caption[Different incompatibility scenarios]{Three metamodels with different consistency relations. The sets $\setted{\consistencyrelation{CR}{1}, \consistencyrelation{CR^T}{1},\consistencyrelation{CR}{2}, \consistencyrelation{CR^T}{2}, \consistencyrelation{CR}{3}, \consistencyrelation{CR^T}{3}}$ and $\setted{\consistencyrelation{CR}{1}, \consistencyrelation{CR^T}{1}, \consistencyrelation{CR''}{2}, \consistencyrelation{CR''^T}{2}, \consistencyrelation{CR}{3}, \consistencyrelation{CR^T}{3}}$ are compatible, whereas the sets $\setted{\consistencyrelation{CR}{1}, \consistencyrelation{CR^T}{1}, \consistencyrelation{CR'}{2}, \consistencyrelation{CR'^T}{2}, \consistencyrelation{CR}{3}, \consistencyrelation{CR^T}{3}}$ and $\setted{\consistencyrelation{CR}{1}, \consistencyrelation{CR^T}{1}, \consistencyrelation{CR}{2}, \consistencyrelation{CR^T}{2}, \consistencyrelation{CR'}{3}, \consistencyrelation{CR'^T}{3}}$ are not. Taken from \owncite{klare2020compatibility-report}.}
    \label{fig:compatibility:incompatibility_example}
\end{figure}

\begin{example}
\autoref{fig:compatibility:incompatibility_example} shows an extract of the three metamodels from \autoref{fig:compatibility:three_persons_example_extended} and several consistency relations, of which different combinations are compatible or incompatible according to the previous definition.
We always consider the actual relations together with their transposed ones to have a symmetric set of consistency relations.
% \begin{enumerate}
% \item $\setted{\consistencyrelation{CR}{1}, \consistencyrelation{CR^T}{1},\consistencyrelation{CR}{2}, \consistencyrelation{CR^T}{2}, \consistencyrelation{CR}{3}}$:
% These consistency relations are obviously compatible, because they relate $name$ and $firstname$ respectively $lastname$ in the same way. Thus, for any element model element with any name, a consistent model can be found by adding the instances of the other classes with equal names.

% \item $\setted{\consistencyrelation{CR}{1}, \consistencyrelation{CR^T}{1},\consistencyrelation{CR'}{2}, \consistencyrelation{CR'^T}{2}, \consistencyrelation{CR}{3}, \consistencyrelation{CR^T}{3}}$:
% These consistency relations are obviously not compatible, because for each person with $firstname$ and $lastname$, another person with $firstname,$ and $lastname$ has to exist due to the transitive relations requiring the addition of a comma. Thus, for each person an infinite number of further persons would have to exist to achieve a consistent set of models. However, models are assumed to be finite, so there is not such set of models and the relations are considered incompatible.

% \item $\setted{\consistencyrelation{CR}{1}, \consistencyrelation{CR^T}{1}, \consistencyrelation{CR'}{2}, \consistencyrelation{CR'^T}{2}, \consistencyrelation{CR}{3}, \consistencyrelation{CR^T}{3}}$:
% These consistency relations are compatible, although one might not expect that. The relations define that for a resident with $firstname = f$ and $lastname = l$ another resident with $firstname = l$ and $lastname = f$ has to exist, so that the set of models is consistent.
% Although that behavior may not be intuitive, it does not violate the definition of compatibility, because for any element of the relations, a consistent model can be constructed.
% In general, such a behavior cannot be forbidden, because comparable behavior might be expected, such as that for a software component an implementation class as well a utility class with different names are created due to different relations, which leads to comparable behavior as in the example.
% To detect such a problem, further semantics of properties would have to be considered, as it is necessary to know that a first name should never be mapped to a last name in our example.

% \item $\setted{\consistencyrelation{CR}{1}, \consistencyrelation{CR^T}{1}, \consistencyrelation{CR}{2}, \consistencyrelation{CR^T}{2}, \consistencyrelation{CR'}{3}, \consistencyrelation{CR'^T}{3}}$:
% These consistency relations reflect the ones of our motivational example in \autoref{fig:motivational_example}.
% According to the informal notion of incompatibility that we motivated in the introduction with that example, our formal definition of compatibility also considers these relations as incompatible, because it is not possible to create a resident with an uppercase name, so that the containing set of models is consistent.
% For a resident with $name = \mathtext{"A\textvisiblespace B"}$, a person with $firstname = \mathtext{"A"}$ and $lastname = \mathtext{"B"}$ has to exist, which requires existence of an employee with $name = \mathtext{"A\textvisiblespace B"}$. Now $\consistencyrelation{CR'}{3}$ requires a resident with $name = \mathtext{"a\textvisiblespace b"}$ to exist, which in turn requires a resident with $firstname = \mathtext{"a"}$ and $lastname = \mathtext{"b"}$ and an employee with $name = \mathtext{"a\textvisiblespace b"}$ to exist.
% In consequence, there are two employees, one with the uppercase and one with the lowercase name, for which a person with name the lowercase name has to exist according to the relation $\consistencyrelation{CR'}{3}$. So there is no witness structure with a unique mapping between the elements that is required to fulfill the consistency definition.
% \end{enumerate}
\begin{description}
\item[$\setted{\consistencyrelation{CR}{1}, \consistencyrelation{CR^T}{1},\consistencyrelation{CR}{2}, \consistencyrelation{CR^T}{2}, \consistencyrelation{CR}{3}}$:]
These relations are obviously compatible, because they relate $firstname$ respectively $lastname$ and $name$ in the same way. Thus, for each object with any name, and thus any condition element in all of the consistency relations, a consistent model set can be found by adding instances of the other classes with appropriate names.

\item[$\setted{\consistencyrelation{CR}{1}, \consistencyrelation{CR^T}{1},\consistencyrelation{CR'}{2}, \consistencyrelation{CR'^T}{2}, \consistencyrelation{CR}{3}, \consistencyrelation{CR^T}{3}}$:]
These relations are obviously not compatible, because for each person with $firstname = f$ and $lastname = l$, another person with $firstname = f + ","$ and $lastname = l$ has to exist due to $\consistencyrelation{CR'}{2}$ and the transitive relations requiring the addition of a comma. Thus, each person would require an infinite number of further persons to exist in a consistent set of models. However, models are assumed to be finite, so there is no such model set and the relations are incompatible.

\item[$\setted{\consistencyrelation{CR}{1}, \consistencyrelation{CR^T}{1}, \consistencyrelation{CR'}{2}, \consistencyrelation{CR'^T}{2}, \consistencyrelation{CR}{3}, \consistencyrelation{CR^T}{3}}$:]
These relations are compatible, although one might not expect that. The relations define that for a resident with $firstname = f$ and $lastname = l$ another resident with $firstname = l$ and $lastname = f$ has to exist, so that the set of models is consistent.
Although that behavior may not be intuitive, it does not violate the definition of compatibility, because for any object in the relations, a consistent model can be constructed.
In general, such a behavior cannot be forbidden, because comparable behavior might be expected, such as that for a software component an implementation class as well a utility class with different names are created due to different relations, which leads to comparable behavior as in the example.
Finally, such a relation would not prevent a consistency repair routine from finding a consistent set of models.
So this can be seen as a semantic problem that requires further relation-specific knowledge, as it is necessary to know that a first name should never be mapped to a last name in our example.

\item[$\setted{\consistencyrelation{CR}{1}, \consistencyrelation{CR^T}{1}, \consistencyrelation{CR}{2}, \consistencyrelation{CR^T}{2}, \consistencyrelation{CR'}{3}, \consistencyrelation{CR'^T}{3}}$:]
These consistency relations reflect the ones of our motivational example in \autoref{fig:compatibility:three_persons_example_extended}.
According to the informal notion of incompatibility that we motivated in the introduction with that example, our formal definition of compatibility also considers these relations as incompatible, because it is not possible to create a resident with an uppercase name, such that the containing set of models is consistent.
For a resident with $name = \mathtext{"A\textvisiblespace B"}$, a person with $firstname = \mathtext{"A"}$ and $lastname = \mathtext{"B"}$ has to exist, which requires existence of an employee with $name = \mathtext{"A\textvisiblespace B"}$. Now $\consistencyrelation{CR'}{3}$ requires a resident with $name = \mathtext{"a\textvisiblespace b"}$ to exist, which in turn requires a resident with $firstname = \mathtext{"a"}$ and $lastname = \mathtext{"b"}$ and an employee with $name = \mathtext{"a\textvisiblespace b"}$ to exist.
In consequence, there are two employees, one with the uppercase and one with the lowercase name, for which a resident with the lowercase name has to exist according to the relation $\consistencyrelation{CR'}{3}$. So there is no witness structure with a unique mapping between the elements that is required to fulfill \autoref{def:consistency} for consistency.
\end{description}
\end{example}

To summarize, compatibility is supposed to ensure that consistency relations do not impose restrictions on other relations such that their condition elements, for which consistency is defined, can never occur in consistent models.
The goal of ensuring compatibility of consistency relations is especially to prevent consistency repair routines of model transformation from non-termination, as may occur especially in the second scenario, where an infinitely large model would be required to fulfill the consistency relations.

Finally, analogously to the equivalence of a set of consistency relations $\consistencyrelationset{CR}$ and its transitive closure $\transitiveclosure{\consistencyrelationset{CR}}$ in regards to consistency of a set of models, we can show that a set of consistency relations and its transitive closure are always equal with regards to compatibility.

\begin{lemma} \label{lemma:compatibilitytransitiveclosure}
    Let $\consistencyrelationset{CR}$ be a set of consistency relations for a set of metamodels $\metamodelset{M}$.
    It holds that:
    \begin{align*}
        \formulaskip
        \consistencyrelationset{CR} \mathtext{compatible} \equivalent
        \transitiveclosure{\consistencyrelationset{CR}} \mathtext{compatible}
    \end{align*}
\end{lemma}

\begin{proof}
    The reverse direction of the equivalence is given by definition, since compatibility of a sub of consistency relations implies compatibility of any subset by definition.
    So we have to show the forward direction by considering the compatibility definition for all $\consistencyrelation{CR}{} \in \transitiveclosure{\consistencyrelationset{CR}}$.
    We partition $\transitiveclosure{\consistencyrelationset{CR}}$ into $\consistencyrelationset{CR}$ and $\transitiveclosure{\consistencyrelationset{CR}} \setminus \consistencyrelationset{CR}$ and consider their consistency relations independently.
    
    First, we consider $\consistencyrelation{CR}{} \in \transitiveclosure{\consistencyrelationset{CR}} \setminus \consistencyrelationset{CR}$.
    According to \autoref{def:transitiveclosure} for the transitive closure, each $\consistencyrelation{CR}{} \in \transitiveclosure{\consistencyrelationset{CR}} \setminus \consistencyrelationset{CR}$ is a concatenation of consistency relations $\consistencyrelation{CR}{1}, \dots, \consistencyrelation{CR}{k} \in \consistencyrelationset{CR}$.
    In consequence of that definition we know that $\condition{c}{l,\consistencyrelation{CR}{}} \subseteq \condition{c}{l,\consistencyrelation{CR}{1}}$, so it is given that:
    \begin{align}
        \formulaskip & \nonumber \label{eq:transitiverelationcontainment}
        \forall \conditionelement{c}{l} \in \condition{c}{l,\consistencyrelation{CR}{}} : \exists \conditionelement{c'}{l} \in \condition{c}{l,\consistencyrelation{CR}{1}} : \forall \modelset{m} \in \metamodelinstances{\metamodelset{M}} : \\ 
        & \formulaskip
        \modelset{m} \containsmath \conditionelement{c}{l} \Rightarrow \modelset{m} \containsmath \conditionelement{c'}{l}
    \end{align}
    Since $\consistencyrelationset{CR}$ is compatible, we especially know from \autoref{def:compatibility} for compatibility that:
    \begin{align}
        \formulaskip & \nonumber \label{eq:compatibilitysingleelement}
        \forall \conditionelement{c'}{l} \in \condition{c}{l, \consistencyrelation{CR}{1}} %\cup \condition{c}{r, \consistencyrelation{CR}{}} 
        : \exists \modelset{m} \in \metamodelinstances{\metamodelset{M}} : \\
        & \formulaskip \formulaskip
        \modelset{m} \mathtext{contains} \conditionelement{c'}{l} \land \modelset{m} \mathtext{consistent to} \consistencyrelationset{CR}
    \end{align}
    Because of \autoref{eq:transitiverelationcontainment} and \autoref{eq:compatibilitysingleelement}, we know that:
    \begin{align}
        \formulaskip & \nonumber \label{eq:compatibilitysinglelementtransitive}
        \forall \conditionelement{c}{l} \in \condition{c}{l, \consistencyrelation{CR}{}} %\cup \condition{c}{r, \consistencyrelation{CR}{}} 
        : \exists \modelset{m} \in \metamodelinstances{\metamodelset{M}} : \\
        & \formulaskip \formulaskip
        \modelset{m} \mathtext{contains} \conditionelement{c}{l} \land \modelset{m} \mathtext{consistent to} \consistencyrelationset{CR}
    \end{align}  
    %Due to \autoref{eq:transitiverelationcontainment}, this statement is also true for all $\conditionelement{c}{l} \in \condition{c}{l, \consistencyrelation{CR}{}}$.
    Furthermore, \autoref{lemma:consistencytransitiveclosure} states that for all model sets $\modelset{m} \in \metamodelinstances{\metamodelset{M}}$ it is true that:
    \begin{align}
        \formulaskip & \label{eq:consistencytransitiveequal}
        \modelset{m} \consistenttomath \consistencyrelationset{CR} \equivalent \modelset{m} \consistenttomath \transitiveclosure{\consistencyrelationset{CR}}
    \end{align}
    In consequence of equations \ref{eq:compatibilitysinglelementtransitive} and \ref{eq:consistencytransitiveequal}, we know that:
    \begin{align}
        \formulaskip & \nonumber \label{eq:compatibilityonlyclosure}
        \forall \consistencyrelation{CR}{} \in \transitiveclosure{\consistencyrelationset{CR}} \setminus \consistencyrelationset{CR} : \forall \conditionelement{c'}{} \in \condition{c}{l, \consistencyrelation{CR}{}} %\cup \condition{c}{r, \consistencyrelation{CR}{}} 
        : \exists \modelset{m} \in \metamodelinstances{\metamodelset{M}} : \\
        & \formulaskip \formulaskip
        \modelset{m} \mathtext{contains} \conditionelement{c'}{} \land \modelset{m} \mathtext{consistent to} \transitiveclosure{\consistencyrelationset{CR}}
    \end{align}
    
    Second, we consider $\consistencyrelation{CR}{} \in \consistencyrelationset{CR}$.
    Due to the definition of compatibility of $\consistencyrelationset{CR}$ and \autoref{lemma:consistencytransitiveclosure} showing equality of consistency of $\modelset{m}$ regarding $\consistencyrelationset{CR}$ and $\transitiveclosure{\consistencyrelationset{CR}}$ it is true that:
    \begin{align}
        \formulaskip & \nonumber \label{eq:compatibilitynonclosure}
        \forall \consistencyrelation{CR}{} \in \consistencyrelationset{CR} : \forall \conditionelement{c'}{} \in \condition{c}{l, \consistencyrelation{CR}{}} %\cup \condition{c}{r, \consistencyrelation{CR}{}} 
        : \exists \modelset{m} \in \metamodelinstances{\metamodelset{M}} : \\
        & \formulaskip \formulaskip
        \modelset{m} \mathtext{contains} \conditionelement{c'}{} \land \modelset{m} \mathtext{consistent to} \transitiveclosure{\consistencyrelationset{CR}}
    \end{align}
    
    With \autoref{eq:compatibilityonlyclosure} and \autoref{eq:compatibilitynonclosure}, we have shown compatibility of $\transitiveclosure{\consistencyrelationset{CR}}$ if $\consistencyrelationset{CR}$ is compatible.
\end{proof}

% \begin{figure}
%     \centering
%     \includegraphics[width=0.7\columnwidth]{figures/incompatible_constraints.png}
%     \caption{Three metamodels and three consistency relations relating those elements that have the same valid $i$ ($C1$ and $C2$ resp. $C1$ and $C3$) or a $i$ value differing by $1$ ($C2$ and $C3$)}
%     \label{fig:incompatible_constraints}
% \end{figure}

% \begin{example}
%     Consider the three metamodels in \autoref{fig:incompatible_constraints}, each containing one class with one attribute $i$. They are related by three consistency relations specifying that in two cases objects having the same $i$ are in the consistency relations, whereas in one case two objects with and $i$ differing by $1$ are in the consistency relation. For any of the consistency relation pairs there are not finite models that are consistency to the consistency relations. Such a model would have to be infinite, as it needed to contain the infinite number of objects with all values of $i$.
%     In consequence, those consistency relations would be considered \emph{incompatible} according to \ref{def:compatibility}.
% \end{example}

% \begin{definition}[Strong Compatibility]
%     Let $\set{\consistencyrelation{CR}}$ be a set of consistency relations for metamodels $\metamodel[1]{M}, \ldots, \metamodel[k]{M}$.
%     A consistency relation $\consistencyrelation{CR}{}$ is considered \emph{strongly compatible with} $\set{\consistencyrelation{CR}{}}$ iff
%     \begin{align*}
%         \formulaskip
%         & 
%         \forall \tupled{\model[1]{m}, \ldots, \model[k]{m}} \in \metamodelinstances{\metamodel[1]{M}} \times \dots \times \metamodelinstances{\metamodel[k]{M}} : \\
%         & 
%     \end{align*}
% \end{definition}


% \begin{definition}[Weak Compatibility]
%     Let $\set{\consistencyrelation{CR}}$ be a set of consistency relations for metamodels $\metamodel[1]{M}, \ldots, \metamodel[k]{M}$.
%     A consistency relation $\consistencyrelation{CR}$ is considered \emph{compatible with} $\set{\consistencyrelation{CR}}$ iff
%     \begin{align*}
%         \formulaskip
%         & 
%         \exists \tupled{\tupled{e_{l1}, \ldots, e_{ln}}, \tupled{e_{r1}, \ldots, e_{rm}}} \in \consistencyrelation{CR} : \\
%         & 
%         \exists \tupled{\model[1]{m}, \ldots, \model[k]{m}} \in \metamodelinstances{\metamodel[1]{M}} \times \dots \times \metamodelinstances{\metamodel[k]{M}} : \exists i, j \in \{1, \ldots, k \} : \\
%         & \formulaskip 
%         \{ e_{l1}, \ldots, e_{ln} \} \subseteq \model[i]{m} \land \{ e_{r1}, \ldots, e_{rm} \} \subseteq \model[j]{m} \\
%         & \formulaskip\formulaskip
%         \land \{ \model[1]{m}, \ldots, \model[k]{m} \} \; \text{consistent according to} \; \set{\consistencyrelation{CR}}
%     \end{align*}
% \end{definition}

% This means that a consistency relation is considered compatible with a set of other consistency relations if there is at least one entry of the consistency relation, i.e. a pair of element tuples, which co-occurs in any set of models such that the models are consistent according to the other consistency relations as well.
% If there is no such set of models, which contains one entry of the consistency relation and is consistent according to the other consistency relations, the consistency relations can never be fulfilled altogether, so the consistency relation is considered \emph{incompatible}.

% \begin{definition}[Possibly well-defined Consistency Relations]
%     A set of consistency relations $\set{\consistencyrelation{CR}}$ is considered \emph{possibly well-defined} iff 
%     \begin{align*}
%         \formulaskip
%         & 
%         \forall \consistencyrelation{CR} \in \set{\consistencyrelation{CR}}: \consistencyrelation{CR} \; \text{is compatible with} \; \set{\consistencyrelation{CR}} \setminus \{ \consistencyrelation{CR} \}
%     \end{align*}
% \end{definition}

%\begin{itemize}
%    \item Insight: Compatibility is a mandatory requirement for interoperability of transformations. If they are not compatible, consistency repair will not be able to find a set of consistent models after certain modifications, because the consistency relations do not do not specify appropriate sets of consistent models.
    %\item TODO: Discuss valid models, why we do not consider them and prove that invariants + consistency relations can express any consistency relation.
%\end{itemize}

\end{copiedFrom} % SoSym MPM4CPS

