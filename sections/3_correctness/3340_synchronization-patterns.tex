%%%
%%% AVOIDANCE PATTERNS
%%%
\section{Synchronization of Bidirectional Transformations}
% Das Praktische Problem
Während wir es dem Entwickler überlassen eine unidirektionale synchronisierende Transformation zu entwickeln, die die entsprechend notwendigen Eigenschaften hat, widmen wir uns noch einem praktischen Problem.
Konsistenz herstellen tun CPR sowieso. Sie reagieren auf Änderungen und erzeugen Änderungen im Zielmodell, sodass die partielle Konsistenz erhöht wird.
Bleibt also zu klären, wie das Einführen von Inkonsistenzen durch das Hinzufügen von Elementen verhindert werden kann.
Dies können wir auf Basis der konkreten Änderungen an Modellen machen oder auf Basis der möglichen Änderungen von Condition Elements.

\subsection{Case Distinction for Changes}
% Erkenntnis: Fallunterscheidung über Changes bringt nichts
Es ist leicht einzusehen, dass eine Fallunterscheidung über Changes nichts bringt.
Beispielsweise könnte ein externer Change (d2) und ein Change per CPRr (d2') völlig unabhängige Elemente betreffen (z.B. einmal ein Attribut einer Klasse, einmal eine Referenz einer anderen Klasse), insofern können wir sie nacheinander ausführen. Allerdings können sie zusammen ein neues Condition Element induzieren. Die Rücktransformation CPRl müsste hierfür ein entsprechendes Condition Element in d1(m1) hinzufügen, was induktiv wieder ein neues Condition Element provozieren kann.
Und das kann passieren, obwohl jede CPR für sich in der Lage ist Konsistenz in einem Schritt herzustellen.
Somit ist es nicht sinnvoll über Änderungen zu unterscheiden, sondern über Condition Elements, da relevant ist, ob eine Kombination von Änderungen dazu führt, dass sich ein Condition Element ändert, ein neues entsteht oder ein altes entfernt wird.

Die Kombination zweier Changes kann niemals dazu führen, dass ein Condition Element entfernt wird, was nicht eh entfernt worden wäre, da bereits beim Nicht-Vorhandensein eines Elementes aus dem Condition Element das Condition Element schon nicht mehr im Modell ist.
Gleiches gilt für die Änderung eines Condition Element. Ein Condition Element ist genau eine Menge von Modellelementen. Entweder ein Change ändert dieses oder nicht, aber nicht erst eine Kombination von Änderungen kann dazu führen.
Interessant ist die Erzeugung eines Condition Elementes, denn diese kann durch die Kombination mehrere Änderungen entstehen. Erst wenn alle Modellelemente innerhalb eines Condition Elementes erzeugt wurde induziert dies Konsistenzanforderungen.

\subsection{Case Distinction for Condition Elements}
Die Unterscheidung über Condition Elements ist sinnvoller, denn genau dann wenn ein solches Element betroffen ist hat das Auswirkungen auf die Konsistenz. Durch welche Art von Änderung das genau entsteht, ist zweitrangig. Man könnte lediglich noch untersuchen, welche Modelländerung zu welcher Änderung eines Condition Elementes führen kann, um dem Entwickler bei der Einschätzung zu helfen.
\todo{Das könnten wir wirklich noch tun!}

Inkonsistenzen werden dadurch eingeführt, dass neue Condition Elements hinzugefügt werden, für die es keine eindeutigen korrespondierenden Elemente gibt, d.h. keine Witness-Struktur aufgebaut werden kann.
Dies ist der Fall sein, wenn Elemente erzeugt werden, um notwendige Konsistenz herzustellen, dadurch aber neue Condition Elements induziert werden, die wieder Konsistenzhaltung bedürfen.
Entsteht das neue Condition Element aus Modellelementen, die alle im partiell konsistenten Teilmodell liegen, dann ist die CPR falsch.
Für dieses partielle Modell, welches vorher konsistent war, und die zugehörige Änderung müsste die CPR gem. Definition Korrektheit ein konsistentes Ergebnis produzieren, kann also keine Condition Elements in m2 induzieren, die keine korrespondierenden Elemente in m2 haben, da die Modelle dann nicht konsistent sind.
Entsteht das neue Condition Element mit Modellelementen, die nicht im partiell konsistenten Teilmodell liegen, ist das okay, da hierfür keine Konsistenz verlangt ist.

Probleme mit denen CPR durch nebenläufige Änderung umgehen können muss:
1. Neue Condition Elements sind bereits vorhanden
2. Condition Elements wurden auf Zielseite modifiziert
3. Condition Elements wurden auf Zielseite gelöscht

1. Das müssen wir herausfinden, siehe Trace-Modell
2. CPR muss wissen, welche korrespondierenden Elemente es vorher gab und diese auf der Zielseite entsprechend anpassen. Im Prinzip muss die CPR nur betrachten, welche Bedingungen die Elemente erfüllen müssen und diese wiederherstellen. Hier müssen aber nur die Änderungen in beide Richtungen entsprechend übertragen werden.
3. CPR muss nichts tun, da durch das Löschen auf der Zielseite auch ein Löschen auf der Quellseite nötig ist, um wieder eine passende Witness-Struktur zu induzieren. Das kann nur die gegenläufige CPR sicherstellen.

Konsequenz: 1. ist der Fall, den wir uns noch genauer anschauen müssen.

% ZUSAMMENFASSUNG DER SZENARIEN BEI BEARBEITUNG VON MODIFIZIERTEM M2
Mögliche Situationen wenn wir an 1->2 statt m2 direkt d2(m2) übergeben:
1. d2(m2) ändert Elemente in einem Condition Element, die 1->2 auch ändern muss. Dann ändert 1->2 einfach partiell die Elemente, und 2->1 macht dann nachher den Rest (nachweisen, dass das geht)
2. d2(m2) fügt neue Condition Elemente hinzu: Unproblematisch, wenn sie nicht mit Elemente in d1(m1) zusammenhängen (d.h. keine Witness-Struktur zwischen d1(m1) und d2(m2) die Elemente verbindet), dann werden sie von 2->1 bearbeitet. Wenn sie mit Element in d1(m1) zusammenhängen (also es eine Witness-Struktur gibt, die Elemente verbindet, auch wenn das noch nicht im Trace-Modell steht), muss hier das Matching passieren! Hier ist natürlich von einer passenden Granularität der Konsistenzrelationen auszugehen. Bspw. kann es reichen, dass eine eine Person/Resident/... mit einem passenden Namen vorhanden ist, ohne dass irgendwelche anderen Werte übereinstimmen (das könnte dann in weiteren Konsistenzrelationen stehen). Dann gäbe es z.B. eine Konsistenzrelation, die angibt, dass für jede Person ein Resident mit dem gleichen Namen vorhanden sein muss, und dann noch eine die beschreibt, dass für Person/Resident-Paare mit dem gleichen Namen auch andere Attribute passend abgebildet werden müssen. Dies ist aber in Transformationssprachen eh implizit so realisiert und lässt sich mit unserem Formalismus für feingranulare Konsistenzrelationen problemlos abbilden.
3. d2(m2) entfernt Condition Elemente: Unproblematisch, wenn die Verarbeitung von d1 nicht auf die entsprechenden Condition Elemente zugreifen muss, also nichts in m1 geändert wurde was zu gelöschten Elementen in d2(m2) korrespondiert. Ansonsten können keine Informationen übertragen werden, da die Witness-Struktur nicht mehr gilt, also macht 1->2 an der Stelle nicht weiter und 2->1 übernimmt den Abbau der Elemente in m1.




Goal: Avoidance strategies for interoperability mistakes, i.e. achieving synchronization of transformations (MA Torsten / Timur)
\label{chap:prevention:interoperability}

\begin{copiedFrom}{ICMT}

% FORMERLY: \subsection{Matching Elements in Operationalizations}
\subsection{Matching Elements}
\label{chap:prevention:interoperability:matching}

\todo{Compare to QVT-R description in \cite[7.10.2]{qvt}, which specifies that elements are created if no elements match the defined key property values specified in the object template}

To avoid failures due to mistakes at the operationalization level, transformations must respect that other transformations may have already created elements.
In the binary case, this is unnecessary.
A single incremental \ac{BX} can assume that elements are either created by the user, %and then are input of the transformations
or were created by the transformation itself.
To identify corresponding elements, transformation languages usually use trace models, which are created by the transformations.
When \acp{BX} are combined to networks, %elements may also be created by other transformations.
%In consequence, 
direct trace links may be missing because a sequence of other transformations created the elements and trace links only indirectly across elements in other models.
%Thus, it is necessary to establish direct trace links between corresponding elements.´
In this scenario, corresponding elements can be matched by information at three levels:
%Such element matching can be performed on three levels:
\begin{enumerate}
    \item \emph{Explicit unique}: The information that elements correspond is unique and represented explicitly, e.g., within a trace model. %Existing transformation languages usually use this technique.
    \item \emph{Implicit unique}: The information that elements correspond is unique, but represented implicitly, e.g., in terms of key information within the models such as element names. %types and element names.
    \item \emph{Non-unique}: If no unique information exists, heuristics must be used, e.g. based on ambiguous information or transitive resolution of indirect trace links.
\end{enumerate}
\todo{Give examples for each case to show that they actually occur}

Indirect trace links, which link elements transitively across other models, usually exist for elements that correspond, because other transformations have already created them.
Nevertheless, indirect trace links cannot be used to unambiguously identify such elements.
An element can correspond to multiple elements in another model, which is why most transformation languages offer tagging of trace links with additional information to identify the correct element.
%For example, a component in an architecture description could be mapped to two classes in an object-oriented design, one providing the component implementation and one providing utilities.
%The relevant corresponding element can be retrieved if the traces are tagged with the information that one class is the implementation and one is a utility.
For example, a language may tag trace links with the transformation rule they were instantiated in.
This is helpful in the bidirectional case, but when links are resolved transitively, these tags have been created by other, independently developed transformations, and are thus unknown.
%If such tags would be considered, transformations would depend on tags of other transformations and could thus not be developed independently anymore.
Therefore, resolving indirect trace links is only a heuristic, but does not unambiguously retrieve corresponding elements.

% Explain how to match rules on three different levels, what the levels can provide etc.

% \begin{enumerate}
%     \item Direct Correspondences
%     \item Key information
%     \item Heuristics: Indirect correspondences, potentially ambiguous information
% \end{enumerate}

Finally, it is up to the transformation engine or the transformation developer %, depending on the provided abstraction level, 
to ensure that elements are correctly matched.
In contrast to the bidirectional case, direct trace links cannot be assumed in case of networks of \acp{BX}.
Therefore, key information within the models must always be considered to identify matching elements.
Whenever direct trace links or unique key information exists, relevant elements can be unambiguously matched.
In all other cases, heuristics must be used, which potentially leads to failures.

\end{copiedFrom} % ICMT



\subsection{From Model Changes to Condition Element Changes}

% TAKEN FROM SYNCHRONIZATION SCENARIOS SECTION
% We map model changes to potential changes of condition elements, so we find out when a developer has to consider what may happen
Since we consider the practical realization of such preservation rules with ordinary transformation languages, we also specifically consider the changes that can be processed by those transformation languages.
Thus, we focus on the types of changes that can be performed in EMOF-based and conforming Ecore-based models.

\begin{figure}
    \centering
    \begin{forest}%
for tree={parent anchor=south,
         child anchor=north,
%          l+=1cm,
%          fill=black!10,
         draw,
         delay={content={\strut #1}},
%          node distance=2ex and 1ex,
         },
featuremandatory/.style={tikz={\node[draw,fill=black!60,inner sep=2pt,circle,anchor=south,yshift=-3pt]at(.north){};}},
featureoptional/.style={tikz={\node[draw,fill=white,inner sep=2pt,circle,anchor=south,yshift=-3pt]at(.north){};}},
[Change
   [Atomic
      [Content,featuremandatory
         [Additive]
         [Subtractive]
      ]
      [Target,featuremandatory
         [Root]
         [Feature
            [Type,featuremandatory
               [Attribute]
               [Reference]
            ]
            [Multiplicity,featuremandatory
               [Single]
               [Multi]
            ]
         ]
      ]
      [Existential,featureoptional
         [Create]
         [Delete]
      ]
   ]
   [Compound
      [\dots]
   ]
]      
% fill angles 
\foreach \i/\j/\k in {!1/!/!2,!21/!2/!21,!121/!12/!122,!12211/!1221/!12212,!12221/!1222/!12222,!131/!13/!132}
{
\coordinate (A) at (\i.north);
\coordinate (O) at (\j.south);
\coordinate (B) at (\k.north);
\featurexor{A}{O}{B}
}
\foreach \i/\j/\k in {!111/!11/!112}
{
\coordinate (A) at (\i.north);
\coordinate (O) at (\j.south);
\coordinate (B) at (\k.north);
\featureor{A}{O}{B}
}
\node [yshift=-6ex,xshift=-22ex,anchor=north west] (leg) at (!12211) {\textbf{Constraints:}};
%\node [yshift=-9ex,xshift=-30ex,anchor=north west] at (!12211) {1.\ Permute $\Rightarrow$ Multi};
\node [below=3ex of leg.north west, anchor=north west] (leg1) {1.\ Single $\Rightarrow$ (Additive $\wedge$ Subtractive)};
\node [below=3ex of leg1.north west, anchor=north west] (leg2) {2.\ Multi $\Rightarrow$ (Additive $\oplus$ Subtractive)};
\node [below=3ex of leg2.north west, anchor=north west] (leg3) {3.\ Root $\Rightarrow$ (Additive $\oplus$ Subtractive)};
\node [right=38ex of leg1.north west, anchor=north west] (leg4) {4.\ Existential $\Rightarrow$ (Root $\oplus$ Reference)};
\node [below=3ex of leg4.north west, anchor=north west] (leg5) {5.\ Create $\Rightarrow$ Additive};
\node [below=3ex of leg5.north west, anchor=north west] (leg6) {6.\ Delete $\Rightarrow$ Subtractive};
% \begin{enumerate}[leftmargin=*]
%  \item Permute $\shortimplies$ Multi
%  \item (Multi $\wedge$ Content) $\shortimplies$ (Additive XOR Subtractive)
%  \item Single $\shortimplies$ (Additive $\wedge$ Subtractive)
% \end{enumerate}
\end{forest}
    \caption{Feature model for all changes in Ecore-based models, adapted from \cite[Fig. 5.3]{kramer2017a}.}
    \label{fig:synchronization:change_feature_model}
\end{figure}

\textcite{kramer2017a} proposes feature models for changes in EMOF- and Ecore-based models, which are supposed to be able to express all kinds of possible changes in them.
The feature model for EMOF-based models is given at \cite[Fig. 5.2]{kramer2017a} and the change model for Ecore-based models is given at \cite[Fig. 5.3]{kramer2017a}.
We depict the latter one in \autoref{fig:synchronization:change_feature_model}.

Make distinction of EMOF- and Ecore-based model, point out differences and explain why we focus on Ecore-based models.


Changes to Max:
\begin{itemize}
    \item No compound changes
    \item No permutation, as it is compound change of subtractive and additive multi-valued feature change. Whether or not permutation rather than removal and addition is relevant is up to the one interpreting the change sequence. It is the same as a move of an element from one reference to another, which is also modelled as a compound change.
    \item Content is mandatory, because we do not have permutation changes as atomic ones, thus every change is either additive or subtractive
    \item Constraints are limited to those still relevant for the remaining scenarios
    \item Fixed error: original model did require a create change of a root element to be subtractive, which does not make any sense, leading to simplification of constraints
\end{itemize}

Arising changes, i.e., all possible configurations of the feature model:
\begin{itemize}
    \item Additive root change (create or not)
    \item Subtractive root change (create or not)
    \item Single-valued attribute change
    \item Additive multi-valued attribute change
    \item Subtractive multi-valued attribute change
    \item Single-valued reference change (with create and/or delete and/or nothing)
    \item Additive multi-valued reference change (create or not)
    \item Subtractive multi-valued reference change (delete or not)
\end{itemize}



FROM MAX:
Different information and case distinctions are necessary to describe all possible model changes for modelling languages that follow the Essential Meta Object Facility (EMOF) standard or the Ecore variant. 
Both meta-modelling languages and the differences between them are described in \autoref{chap:foundations:modeling:models}.
Only two differences have a major effect on our change modelling language and the specifications language that use them:
\begin{enumerate}%[1.]
\item In EMOF, properties can be typed using metaclasses or using other data types, but in Ecore these are distinguished as references and attributes.
\item Ecore requires that all elements except for a root element are contained in exactly one container and EMOF only requires that all elements have at most one container~\cite[pp.\ 31-32]{mof}.
\end{enumerate}
If we only consider these two differences, then Ecore can be seen as a refinement of EMOF, which only adds a more fine-grained distinction of properties and further containment restrictions.
Because of this refinement relation, we will first describe which information is necessary to represent model changes of EMOF-based models and then add further information and distinctions for Ecore-based models.
Finally, we briefly explain how we made all this information available in practice using a change modelling language.

