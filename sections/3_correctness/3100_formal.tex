\chapter{A Formal Proof of Correctness Requirements
    \pgsize{25 p.}
}

\todo{Restructure into formalization of correctness (containing compatibility, interoperability etc.) and formal proof of proving correctness. Or maybe move that to prevention section?}

Allgemeine Definition Transformationsnetzwerke:
\begin{itemize}
    \item Definition Transformation aus Relation und Wiederherstellungsroutinen; Routinen nehmen n Modelle und n Deltasequenzen (eine pro Modell) und liefern n Deltasequenzen zurück.
    \item Im Allgemeinen könnte eine Transformation beliebige dieser Deltasequenzen modifizieren. Wir verlangen jedoch, dass eine Transformation nur Deltas anhängt, also die Sequenzen länger werden
    \item Genauer beschränken wir auch, welche Sequenzen eine Transformation sehen und ändern darf, genau gesagt darf sie die Sequenzen von zwei Modellen sehen und eine davon verlängern.
    \item Hier kommt bereits der Unterschied zu bisherigen Transformationen, denn die sehen nur Deltas an einem Modell und erzeugen Deltas an dem anderen. Das ist bei uns schon gänzlich anders. Bidirektionale Transformationen unterstützen das im Übrigen auch nicht, sondern sind nur Spezifikationen, aus denen sich Wiederherstellungsroutinen für beide Richtungen ableiten lassen (siehe Stevens 2010)
    \item Relationen in erster Instanz auf Modellebene (also bzgl. ganzer Modelle, nicht einzelner Modellelemente) definieren
    \item Direkt als multidirektionale Transformation definieren, also beliebig viele geändert Ein- und Ausgabemodelle (oder jeweils nur eins?)
    \item Korrektheit einer Transformation (nach Stevens) definieren!
    \item Versuchen den Konkatenationsoperator zu definieren ohne dass er alle Metamodelle referenzieren muss (also Transformation wählt aus einer großen Eingabemenge relevanten Modelle aus, ändert relevante und dann fügt der Operator sie in die große Menge ein)
    \item Definition Transformationsnetzwerk als Tupel aus Metamodellen, Transformationen und einer Ausführungsfunktion. 
    \item Die Ausführungsfunktion führt für eine gegebene Änderung eine Auswahl der Transformationen nacheinander aus.
    \item Korrektheit eines Netzwerkes definieren: Die Ausführungsfunktion erzeugt eine Transformationssequenz, die angewendet auf eine Änderung für alle Änderungen ein korrektes Ergebnisse produziert, d.h. die Modelle sind konsistent bzgl. allen Konsistenzrelationen.
\end{itemize}

An intuitive way of describing consistency is to specify all tuples of models that are considered consistent to each other, i.e., to specify a relation between the models. 

\begin{definition}[\ModelLevelConsistencyRelation]
    Given metamodel $\metamodelsequence{M}{n}$, a \emph{\modellevelconsistencyrelation} $\consistencyrelation{CR}{}$ is a relation for instances of the metamodels $\consistencyrelation{CR}{} \subseteq \metamodelinstanceset{M}{1} \times \dots \times \metamodelinstanceset{M}{n}$.
    We consider a tuple of models $\tupled{\model{m}{1}, \dots, \model{m}{n}}, \model{m}{i} \in \metamodelinstanceset{M}{i}$ \emph{consistent to $\consistencyrelation{CR}{}$} if and only if $\tupled{\model{m}{1}, \dots, \model{m}{n}} \in \consistencyrelation{CR}{}$.
    Otherwise, we call $\tupled{\model{m}{1}, \dots, \model{m}{n}}$ \emph{inconsistent to $\consistencyrelation{CR}{}$}.
\end{definition}

We explicitly denote this kind of consistency relation as \emph{model-level}, because we will later need a more fine-grained notion of consistency relations at the level of \metaclasses and need to distinguish between the two.

In the following, we only consider binary consistency relations. Having several consistency relations to define how several metamodels are related, we need to define a notion of consistency based on several consistency relationsl

\begin{definition}[Consistency]
    Let $\metamodelsequence{M}{n}$ be metamodels and let $\consistencyrelation{CR}{} \subseteq \metamodelinstanceset{M}{i} \times \metamodelinstanceset{M}{j}$ be a binary \modellevelconsistencyrelation for any two metamodels $\metamodel{M}{i}, \metamodel{M}{j} \in \setted{\metamodelsequence{M}{n}}$.
    For a given set of models $\modelset{m} \in \metamodelsetinstanceset{M}{}$, we say that this model set is \emph{consistent to} $\consistencyrelation{CR}{}$ if and only if the instances of $\metamodel{M}{i}$ and $\metamodel{M}{j}$ are in that relation:
    \begin{align*} 
        &
        \modelset{m} \consistenttomath \consistencyrelation{CR}{} \equivalentperdefinition \\
        & \formulaskip
        \exists \model{m}{i} \in \metamodelinstanceset{M}{i}, \model{m}{j} \in \metamodelinstanceset{M}{j} : \model{m}{i} \in \modelset{m} \land \model{m}{j} \in \modelset{m} \land \tupled{\model{m}{i}, \model{m}{j}} \in \consistencyrelation{CR}{}
    \end{align*}
    For a set of binary \modellevelconsistencyrelations $\consistencyrelationset{CR}$ for metamodels $\metamodelsequence{M}{n}$ and a given set of models $\modelset{m} \in \metamodelsetinstanceset{M}{}$, we say that this model set is \emph{consistent to} $\consistencyrelationset{CR}$ if and only if the it is consistent to each consistency relation in that set:
    \begin{align*} 
        &
        \modelset{m} \consistenttomath \consistencyrelationset{CR} \equivalentperdefinition \\
        & \formulaskip
        \forall \consistencyrelation{CR}{} \in \consistencyrelationset{CR} : \modelset{m} \consistenttomath \consistencyrelationset{CR}
    \end{align*}
\end{definition}


\begin{definition}[Change]
    Given a metamodel $\metamodel{M}{}$, a change $\change{\metamodel{M}{}}$ is a function that takes an instance of that metamodel and returns another instances:
    \begin{align*}
        &
        \change{\metamodel{M}{}}: \metamodelinstanceset{M}{} \rightarrow \metamodelinstanceset{M}{}
    \end{align*}
    It encodes any kind of change, which may be just an element addition, or removal, an attribute change and so on, or any composition of changes.
    For us, it does not matter how the function behaves in cases, in which the encoded change cannot be applied, e.g., because the changed or removed element does not exist. The function may do nothing for those models, i.e. return the identical model, or even be undefined for those model, i.e., be partial.
    \todoLater{Check whether this behavior is correct.}
    We denote the universe of all changes in $\metamodel{M}{}$, i.e. all subsets of $\metamodelinstanceset{M}{} \times \metamodelinstanceset{M}{}$ that are functional, as 
    \begin{align*}
        &
        \changeuniverse{\metamodel{M}{}} = \setted{\change{\metamodel{M}{}} \mid \change{\metamodel{M}{}} \subseteq \metamodelinstanceset{M}{} \times \metamodelinstanceset{M}{} \land
        (\tupled{\model{m}{1}, \model{m}{2}}, \tupled{\model{m}{1}, \model{m}{3}} \in \change{\metamodel{M}{}} \Rightarrow \model{m}{2} = \model{m}{3}})
    \end{align*}
\end{definition}

\begin{definition}[Consistency Preservation Rule]
    Given two metamodels $\metamodel{M}{1}, \metamodel{M}{2}$ and a binary consistency relation between them $\consistencyrelation{CR}{} \subseteq \metamodelinstanceset{M}{1} \times \metamodelinstanceset{M}{2}$.
    A consistency preservation rule $\consistencypreservationrule{\consistencyrelation{CR}{}} : (\metamodelinstanceset{M}{1}, \metamodelinstanceset{M}{2}, \change{\metamodel{M}{1}}) \rightarrow \change{\metamodel{M}{2}}$ for that consistency relation is a function that takes two consistent models and a change in the first one and returns a change in the second one, such that the resulting models when applying both changes are consistent again:
    \begin{align*}
        &
        \forall \model{m}{1} \in \metamodelinstanceset{M}{1}, \model{m}{2} \in \metamodelinstanceset{M}{2} :
        \tupled{\model{m}{1}, \model{m}{2}} \in \consistencyrelation{CR}{} \Rightarrow\\
        & \formulaskip
        \forall \change{\metamodel{M}{1}} \in \changeuniverse{\metamodel{M}{1}} :
        \exists \change{\metamodel{M}{2}} \in \changeuniverse{\metamodel{M}{2}} :\\
        & \formulaskip
        \consistencypreservationrule{\consistencyrelation{CR}{}}(\model{m}{1}, \model{m}{2}, \change{\metamodel{M}{1}}) = \change{\metamodel{M}{2}} 
        \land \tupled{\change{\metamodel{M}{1}}(\model{m}{1}), \change{\metamodel{M}{2}}(\model{m}{2})} \in \consistencyrelation{CR}{}
    \end{align*}
\end{definition}

This is equivalent to the definition of \emph{consistency restorers} in \cite{stevens2010sosym}. In contrast, they are only defined on models states rather than an explicit definition of changes, which gives us later the possibility to consider the composition of changes.

A consistency preservation rule is defined to restore consistency after a modification in a left model of the underlying consistency relation by creating a change for the right model. To consider consistency preservation rules that preserve consistency in the other direction, we regard the inverse of the consistency relation as well, denoted as $\inverseconsistencyrelation{CR}{} = \setted{\tupled{\model{m}{1}, \model{m}{2}} \mid \tupled{\model{m}{2}, \model{m}{1}} \in \consistencyrelation{CR}{}}$.

A consistency relation together with two consistency restorers, or consistency preservation rules in our terminology, forms a \emph{bidirectional transformation}.

\begin{definition}[Bidirectional Transformation]
    Let $\consistencyrelation{CR}{}$ be a consistency relation, and $\consistencypreservationrule{\consistencyrelation{CR}{}}$ and $\consistencypreservationrule{\inverseconsistencyrelation{CR}{}}$ two consistency preservation rules to restore consistency according to that relation in both directions, i.e., after changes in any of the models.
    A bidirectional transformation is a triple $\tupled{\consistencyrelation{CR}{}, \consistencypreservationrule{\consistencyrelation{CR}{}}, \consistencypreservationrule{\inverseconsistencyrelation{CR}{}}}$.
\end{definition}

The definition could also be given for an arbitrary number of metamodels, but we restrict ourselves to binary specifications, as explained in \todoLater{ref}.

\begin{definition}[Consistency Preservation Application Function]
    \todo{Define for transformations instead?}
    Let $\consistencypreservationruleset{}$ be a set of consistency preservation rules for a set of consistency relations $\consistencyrelationset{CR}$ on metamodels $\metamodelsetinstanceset{M} = \setted{\metamodelsequence{M}{n}}$.
    A consistency preservation application function $\consistencyappfunction{\consistencypreservationruleset{}}$ for these rules is function:
    \begin{align*}
        &
        \consistencyappfunction{\consistencypreservationruleset{}} : (\metamodelsetinstanceset{M}, \change{\metamodel{M}{1}, \dots, \change{\metamodel{M}{n}}}) \rightarrow (\metamodelsetinstanceset{M})
    \end{align*}
    The function takes a consistent set of models and returns a consistent set of models by acquiring changes from the consistency preservation rules in $\consistencypreservationruleset{}$. Thus, it has to fulfill the following conditions:
    \begin{align*}
        &
        \forall \modelset{m} \in \metamodelsetinstanceset{M} : \forall \change{\metamodel{M}{1}} \in \changeuniverse{\metamodel{M}{1}}, \dots, \change{\metamodel{M}{n}} \in \changeuniverse{\metamodel{M}{n}}:
        \modelset{m} \consistenttomath \consistencyrelationset{CR} \Rightarrow \\
        & \formulaskip
        \exists \modelset{m'} \in \metamodelsetinstanceset{M} :
        \modelset{m'} = \consistencyappfunction{\consistencypreservationruleset{}}(\modelset{m}, \change{\metamodel{M}{1}}, \dots, \change{\metamodel{M}{n}}) \land \modelset{m'}\consistenttomath \consistencyrelationset{CR} \\
        & \formulaskip
        \land \exists \consistencypreservationrule{1}, \dots, \consistencypreservationrule{m} \in \consistencypreservationruleset{} : 
        \exists \change{\metamodel{M}{1}}' \in \changeuniverse{\metamodel{M}{1}}, \dots, \change{\metamodel{M}{n}}' \in \changeuniverse{\metamodel{M}{n}}:\\
        & \formulaskip \formulaskip 
        \consistencypreservationrule{1} \concat \dots \concat \consistencypreservationrule{m}(\modelset{m}, \change{\metamodel{M}{1}}, \dots, \change{\metamodel{M}{n}}) = \change{\metamodel{M}{1}}', \dots, \change{\metamodel{M}{n}}\\
        & \formulaskip \formulaskip
        \land \setted{\change{\metamodel{M}{1}}'(\model{m}{1}), \dots, \change{\metamodel{M}{n}}(\model{m}{n})} = \modelset{m'}
    \end{align*}
    \todo{This is not yet correct, there has to be some definition of how to apply tuples of changes. Additionally, there should be model set and delta sets equally defined.}
\end{definition}

Now it is obvious that the consistency preservation rules can actually do anything to achieve consistency, including returning always the same set of models that is consistent, although that may not be expected. We will discuss later which reasonable assumptions can be made to the behavior to on the hand not restrict the possibilities of the transformation developer and on the other hand be able to ensure some properties of the transformations and their execution.


Two levels of correctness:
\begin{enumerate}
    \item Local correctness: a consistency relation is correct to the global relation and the CPR is to the relation, i.e. given two models and changes in them, the transformation can produce a change that restores consistency regarding the global consistency relation of these two models (i.e. there are some other models with which these two models would be consistent regarding the global specification) --> a network is locally correct, if this property is fulfilled
    \item Global correctness: the binary relations together are equal to the global one and the execution function is able to find consistency models after a change to initially consistent models --> network is globally correct, if this property is fulfilled
\end{enumerate}
Potentiell ist lokale Korrektheit (zumindest einer CPR zu ihrer CR per Konstruktion) herstellbar -- das war auch das Ergebnis bisheriger Studien --, eventuell auch von einer CR zu einer globalen CR, obwohl die ja eigentlich meist nicht existiert, daher nehmen wir das als gegeben an.
Dann zeigen, dass die globale Beziehung der Relationen nicht äquivalent ist zu den einzelnen lokalen, daher kommt hier zusätzliche Komplexität rein (Kompatibilitätsbegriff).
Final muss noch die Ausführungsfunktion korrekt sein, hier aber Problem der Turing-Vollständigkeit. 
Daher Einschränkungen an Transformationen finden bzw. ingenieurmäßige Ausführungsreihenfolge festlegen, die möglichst oft richtige Lösungen findet und sonst konservativ mit einem Fehler terminiert.


\textbf{On top of ordinary bx correctness:}
\begin{itemize}
    \item Transformations need to be synchronizing
    \item Consistency relations need to fulfill a notion of correctness
    \item Exkurs:
    \begin{itemize}
    \item Is compatibility a subclass of correctness? Is every correct set of relations compatible as well?
    \item Problematisch: unser Konsistenzbegriff für Relationen (feingranulare Relationen) schließt keine Modelle aus, der Konsistenzbegriff hier aber schon. Wie realisiere ich die feingranularen Relationen, die dafür sorgen, dass nur genau ein Tupel von Modellen konsistent ist?
    \item Wir müssen bei der Ableitung unseres Kompatibilitätsbegriffes erklären, dass bei uns der vollständige Ausschluss bestimmter Modelle nicht Teil einer feingranularen Konsistenzrelation sein darf, sondern Teil einer weiteren Spezifikation, die angibt, welche Modelle überhaupt valide sind. Denn so ist es in Transformationssprachen tatsächlich auch.
    \end{itemize}
    \item Execution function needs to be defined, which potentially induces requirements to the transformations.
\end{itemize}


Trivialisierung des Problems:
\begin{itemize}
    \item Ohne weitere Annahmen ist das immer dadurch erreichbar, dass die Transformationen einen beliebigen anderen Zustand der Modelle produzieren. Im einfachsten Fall liefert jede Transformation immer die gleichen konsistenten Modelle zurück, unabhängig von der Änderung. Dann ist der Endzustand der Modelle nach der Ausführung des Netzwerks immer der gleiche.
    \item Das ist im allgemeinen aber nicht Fall. Letztendlich trifft jede Transformation lokale Entscheidungen. Beispielsweise könnte jede einzelne Transformation gegeben eine beliebige Änderung immer dieselben Modelle (bzw. Änderungen die dazu führen) zurückliefern (im trivialsten Fall leere Modelle). Dann erfüllt jede Transformation ihre Korrektheitseigenschaft bzgl. ihrer Relation, aber das Netzwerk muss nicht korrekt sein, da bspw. T(A,B) und T(B,C) sich immer für verschiedene Instanzen von B entscheiden. Es gäbe somit nie eine konsistente Lösung für eine beliebige Ausführungsreihenfolge der Transformationen, auch wenn die Relationen das erlauben würden.
    \item Beispiel mit Namen, wo eine Transformation immer den großen Namen zurückliefert, die andere immer den kleinen. T(A,B) bildet A auf gleiches B ab und beide auf kleine Schreibweise, obwohl beide erlaubt sind. Erzeuge A="a", dadurch B="a". T(B,C) bildet B auf C ab und beide auf große Schreibweise, obwohl beide erlaubt sind. Somit macht sie das zu B="A" und C="A". Nun wird T(A,B) wieder beide klein machen usw. Allerdings wäre eine insgesamt valide Lösung einfach alle groß oder alle klein zu machen, aber die Transformationen finden diesen Zustand nicht. 
    \item Allgemeiner ist zu sagen, dass ein Transformationsnetzwerk eine Turing-Maschine emulieren kann. \todo{Nachweisen!}
    Im allgemeinen terminiert das Netzwerk somit nicht, schlimmer noch, es ist unentscheidbar, ob das Netzwerk hält (siehe Halteproblem).
    \item Dies zeigt bereits, dass keine Ausführungsfunktion definiert werden kann, die immer ein konsistentes Ergebnis liefert.
    \item Wir versuchen daher Annahmen an Transformationen zu finden, um diese Fälle auszuschließen bzw. systematisch zu verringern. 
    \item Außerdem möchten wir eine Ausführungsfunktion haben, die ein konsistentes Ergebnis liefert oder einen Fehler, denn es muss nicht immer eine korrekte Lösung geben. Ziel ist es dann die Anzahl der Fälle, in denen sie einen Fehler zurückgibt, zu reduzieren.
\end{itemize}

Zielsetzung:
\begin{itemize}
    \item Korrekte Anwendungsfunktion finden (in bestehenden Arbeiten~\cite{stevens2017a}) auch "Resolution" genannt (formal definieren!):
    \item Welche Anforderungen müssen wir dafür an die Transformationen stellen, damit solch eine Funktion definiert werden kann?
    \item Wir bezeichnen das Transformationsnetzwerk, in dem eine Transformation eingesetzt wird, als "Kontext"
    \item Welche dieser Eigenschaften kann die einzelne Transformation (ohne Kenntnis der anderen) erfüllen und für welche muss der Kontext (d.h. die anderen Transformationen) bekannt sein?
    \item $\Rightarrow$ Interesse an "kontextfreien" Eigenschaften (lassen sich ohne Kenntnis der anderen Transformationen sicherstellen -> Wiederverwendbarkeit) und "kontextsensitiven" Eigenschaften (Erfüllung der Eigenschaft nur durch Kenntnis über das Transformationsnetzwerk möglich)
    \item Kontextfreie Eigenschaften involvieren solche, die wir eh schon von Transformationen kennen (Korrektheit einer Transformation, Hippokratie etc.) und solche, die dadurch zustande kommen, dass man weiß, dass diese Transformation in einem Netzwerk eingesetzt werden soll.
    \item Zielsetzungsoptionen:
    \begin{itemize}
        \item Wir schränken die Transformationen so ein, dass es immer mindestens eine Ausführungsreihenfolge der Transformationen gibt, sodass für jede beliebige Änderung ein konsistentes Ergebnis durch Anwenden der Transformationen gefunden werden kann
        \item Wir akzeptieren, dass es Änderungen gibt, für die das Netzwerk kein konsistentes Ergebnis produzieren kann. Dann muss das Netzwerk (mindestens) in diesen Fällen mit einer Fehlermeldung terminieren.
        \item Eine Option ist, dass das Netzwerk dieses Verhalten nur approximiert bzw. approximieren kann, dann muss es sich konservativ verhalten, d.h. im Fall, dass es keine Lösung gibt, auf jeden Fall eine Fehlermeldung geben, und im Fall, in dem es eine Lösung gibt, diese bestenfalls finden oder ausgeben, dass es keine finden kann (d.h. keine False Positives bzw. Nicht-Terminierung). Ziel ist es dann den Grad der Konservativität zu minimieren.
    \end{itemize}
    \item Lösungsoptionen (Grad der Einschränkung an die Transformationen):
    \begin{itemize}
        \item Hohe Einschränkung: Jede beliebige Reihenfolge von ausgeführten Transformationen führt letztendlich zu einem korrekten Ergebnis (Fixpunktiteration -- Allquantifizierung) -- Hippokratie-Eigenschaft sorgt dafür, dass keine Transformation wieder etwas ändert, wenn Konsistenz bereits hergestellt ist.
        Diese Eigenschaft ist in der Praxis möglicherweise zu strikt, da sie sehr starke Anforderungen an die Transformationen stellen müsste. Dafür wäre aber die Anwendungsfunktion trivial.
        \item Mittlere Einschränkung: Es gibt eine Reihenfolge von ausgeführten Transformationen für jede Änderung die terminiert (Existenzquantifizierung) und die Ausführungsfunktion findet diese Reihenfolge.
        Utopisch, dass die Anwendungsfunktion aus (potentiell sehr mächtigen) Transformationen die richtige Reihenfolge errechnen kann. Dafür aber (möglicherweise) weniger Anforderungen an die Transformationen (zumindest nicht mehr Anforderungen, denn die Allquantifzierung induziert die Existenzquantifizierung). Eine Funktion könnte dann zumindest nach best-effort versuchen, die richtige Reihenfolge zu finden und konservativ abbrechen, wenn sie diese nicht finden kann (also entweder konsistent terminieren oder terminieren mit der Aussage, dass es entweder keine solche Reihenfolge gibt -- bei relaxierten Anforderungen -- oder dass es sie nicht finden kann).  
        \item Geringe Einschränkung: Es gibt potentiell keine Reihenfolge der Transformationen, die bei einer Änderung zu einer konsistenten Lösung kommt. Hier müsste die Ausführungsfunktion entsprechend einen Fehler ausgeben.
        \item Bestehende Arbeiten (\cite{stevens2017a}) schlagen auch vor eine Baumstruktur zu berechnen (Spannbaum), in dem nur entlang der Baumkanten die Transformationen ausgeführt werden. Dies ist jedoch eine starke Einschränkung daran, was die Transformationen ausdrücken können. Betrachtet man beispielsweise PCM, UML und Java, und hat eine Änderung in PCM. Dann könnte der Spannbaum entweder PCM -> UML -> Java sein, oder PCM -> UML + PCM -> Java. In ersterem Fall würde Verhaltensbeschreibung, die von PCM nach Java übertragen, aber in UML nicht dargestellt wird, nicht übertragen. Im zweiten Fall würde zusätzliche Information zwischen UML und Java nicht propagiert (Beispiel?) --> Hier sollte auf das Properties-Kapitel verwiesen werden, wo diese "Bottlenecks" erklärt sein sollten, inklusive einem Beispiel, die allgemein Baumstrukturen für Transformationsnetwerke ausschließen.
    \end{itemize}
    \item Dies setzt voraus, dass die Transformationen und die Anwendungsfunktion mit jeder beliebigen Nutzer-Änderung umgehen kann. Man kann jedoch auch verlangen, dass die Anwendungsfunktion genau dann, wenn es überhaupt eine Ausführungsreihenfolge gibt, diese findet, und sonst einen Fehler ausgibt.
    \item \textbf{Wichtig:} Im Allgemeinen kann eine Ausführungsfunktion keine terminierende Reihenfolge berechnen, da die Transformationen Turing-vollständig sind und deshalb die Frage, welche Reihenfolge zu einer Terminierung führt, unentscheidbar ist (Halteproblem). Daher können wir nur einen konservativen Algorithmus angeben, der ein sinnvolles Abbruchkriterium definiert, mit dem die Ausführung beendet wird, auch wenn potentiell eine Lösung hätte gefunden werden können. Die Fragestellung ist also, wie die Ausführungsfunktion aussehen muss, damit sie in möglichst vielen Fällen, in denen es eine terminierenden Reihenfolge gibt, diese auch findet. Insbesondere lässt sich somit keine geschlossene Form für die Ausführungsfunktion angeben, sondern nur ein Algorithmus, der zur Laufzeit eine Reihenfolge (dynamisch) festlegt.
\end{itemize}

Problemraum:
\begin{itemize}
    \item Ziel ist, dass ein Netzwerk von Transformationen nach einer Änderung in einem konsistenten Zustand terminiert. D.h. Korrektheit stellt Anforderungen an \emph{Terminierung}, sowie den \emph{Zustand} bei Terminierung.
    \item Folgende Abweichungen davon können auftreten:
    \begin{enumerate}
        \item Nicht-Terminierung: Das Netzwerk terminiert nicht. Das bedeutet im Prinzip, dass die Ausführungsfunktion (bzw. der Laufzeit-Algorithmus, der die Funktion dynamisch emuliert) nicht \emph{sound} ist. Soundness der Ausführungsfunktion setzt voraus, dass die berechnet Aufrufsequenz endlich ist. Wenn die Ausführung nicht terminiert, bedeutet das, dass entweder die gleichen Zustände mehrfach durchlaufen werden oder eine Sequenz unendlich vieler Zustände produziert wird. Denn wenn beides nicht der Fall ist, gibt es eine endliche Sequenz unterschiedlicher Zustände, d.h. Terminierung. Das bedeutet, dass es folgende zwei Möglichkeiten gibt:
        \begin{itemize}
            \item Alternierung: Die gleichen Zustände werden mehrfach durchlaufen.
            \item Divergenz: Es werden unendlich viele Zustände produziert.
        \end{itemize}
        \item Inkonsistente Terminierung: Die Ausführungsfunktion bzw. der Algorithmus beendet die Ausführung, aber in einem inkonsistenten Zustand. Hier lassen sich ebenfalls wieder zwei Fälle unterscheiden.
        \begin{itemize}
            \item Unerkannte Inkonsistenz: Der Algorithmus terminiert und denkt, der Zielzustand wäre konsistent. Dies bedeutet aber direkt, dass nicht alle Konsistenzrelationen erfüllt sind, was, zumindest in der Theorie, einfach zu prüfen wäre (entweder durch Prüfung der Relationen oder durch Ausführung der hippokratischen Transformationen, die alle nichts tun dürften)
            \item Erkannte Inkonsistenz: Der Algorithmus terminiert, wissend dass die Lösung nicht konsistent ist. Dies kann entweder sein, weil eine Transformation für zwei Modelle in einem inkonsistenten Zustand nicht mehr anwendbar ist, oder weil irgendein anderes Abbruchkriterium erreicht ist.
        \end{itemize}
    \end{enumerate}
\end{itemize}

Annahmen:
\begin{itemize}
    \item Nutzeränderungen dürfen nicht rückgängig gemacht werden.
    \item Nutzeränderungen lassen sich so feingranular zerlegen, dass, falls durch die Erzeugung/Änderung eine Konsistenzrelation verletzt wird, es in jeder unabhängigen Teilmenge von Konsistenzrelationen eine verletzte Konsistenzrelation gibt, für die die geänderten Elemente einem Condition Elemente entsprechen, es also insbesondere keine Teilmenge der geänderten Element gibt, die bereits dieses Condition Element sind. Ansonsten ist durch unsere Kompatibilitäts-Definition nicht sichergestellt, dass eine konsistente Modellmenge gefunden werden kann.
\end{itemize}

Voraussetzungen:
\begin{itemize}
    \item Relationen müssen korrekt sein, d.h. sie müssen bzgl. einer globalen (meist eher implizit bekannten) n-ären Relation zwischen allen Modellen identisch sein. Eine n-äre Relation lässt sich nicht immer zerlegen (siehe Stevens), aber wir nehmen das an.
    \item Die einzelne Transformation muss bzgl. ihrer Relation korrekt sein, d.h. sie muss bei Änderungen in beiden Modellen ein zur Relation konsistentes Modell liefern.
\end{itemize}

Ebenen der Korrektheit:
\begin{itemize}
    \item Relationen müssen korrekt sein, d.h. gegeben eine Nutzeränderung muss es überhaupt möglich sein eine konsistente Menge an Modellen zu finden. Wenn Transformationen etwas beliebigen tun dürfen geht das immer. Wir nehmen an, dass eine Nutzeränderung nicht rückgängig gemacht werden soll (bzw. wenn sie rückgängig gemacht werden würde eigentlich die Änderung invalide war, d.h. keine Konsistenz im Netzwerk hergestellt werden kann). Daher sind Relationen nur korrekt, wenn für fixierte Elemente, die durch eine Nutzeränderung entstehen können, eine Modellmenge abgeleitet werden kann, die bzgl. der Relationen konsistent ist. D.h. gegeben einige Elemente muss es eine Modellmenge geben, die in allen Relationen liegt und die diese Elemente enthält (-> Kompatibilitätsbegriff). Wir betrachten in Kapitel ?, wie man Kompatibilität präzise definieren und feststellen/garantieren kann.\\
    Resultat: Gegeben eine Änderung ist es möglich eine Transformation anzugeben, die aus der Änderung ein konsistentes Modell produziert.
    \item Einzelne Transformationen müssen korrekt sein: Wir fordern Korrektheit der Transformation sowieso. Allerdings machen in einem Netzwerk verschiedene Transformationen Änderungen an allen Modellen, d.h. wir müssen nicht den "normalen" Transformationsfall unterstützen, dass Deltas in einem Modell ins andere übertragen werden, um Konsistenz herzustellen, sondern die Transformationen müssen \emph{synchronisierend} sein, also Deltas in beiden Modelle annehmen und dann Konsistenz herstellen. Wir definieren diese Synchronisationseigenschaft und betrachten in Kapitel ?, welcher zusätzlichen Anforderungen sich dadurch bzgl. EMOF-Modellen ergeben. Der Input sind Deltas in zwei Modellen, und einzelne Deltas sind potentiell als "authoritative" definiert, was bedeutet, dass die erzeugten/geänderten Elemente nicht noch einmal geändert/gelöscht werden dürfen. Das realisiert die Anforderung, dass Nutzeränderungen nicht rückgängig gemacht werden dürfen. \\
    Resultat: Gegeben Änderungen in zwei Modellen (mit potentiell authoritativen Änderungen) gibt die Transformation ein konsistentes (bzgl. der Konsistenzrelation) Modellpaar zurück. 
    \item Korrektheit der Anwendungsfunktion: Die Anwendungsfunktion muss die Transformationen in einer 
\end{itemize}

Annahme an Transformationen:
\begin{itemize}
    \item Muss eine Transformation mit jedem beliebigen Delta umgehen können müssen? Eine Einschränkung auf Monotonie würde dies verhindern. Bzw. wir müssten zeigen, dass es Konsistenzrelationen gibt, die unter der Anforderung an Monotonie nicht wiederhergestellt werden können. Bspw. fügt eine andere Transformation 3 Elemente hinzu, wo zwei mit dem anderen entsprechend der Konsistenzrelationen korrelieren und somit keine Witness-Struktur aufgebaut werden kann, die Konsistenz beweist. Das lässt sich durch Hinzufügen weiterer Elemente potentiell nicht auflösen (siehe Beispiele im SoSym-Paper).
\end{itemize}

Notwendigkeit Transformationen oder Anwendungsfunktion einzuschränken:
\begin{itemize}
    \item Zeigen, dass es Beispiele gibt, in denen es keine einzige Ausführungsreihenfolge gibt (All-Quantifizierung), die zu einem konsistenten Ergebnis führt:
    \item Zeigen, dass es Beispiele gibt, in denen es unabhängig von der Ausführungsreihenfolge immer zu einer Alternierung kommt
    \item Zeigen, dass es Beispiele gibt, in denen es unabhängig von der Ausführungsreihenfolge immer zu einer Divergenz kommt.
    \item Die Beispiele sollten zeigen, dass wir keine Einschränkungen an die Transformationen machen können, was das Problem aushebelt. D.h. egal welche Einschränkungen ich an die Transformationen definiere, es lassen sich immer Beispiele konstruieren, in denen es keine Ausführungsreihenfolge gibt, in denen sie terminieren.
    \item Mathematisch zeigen, dass Alternierung und Divergenz die einzigen Probleme sind. D.h. wenn nicht der gleiche Zustand mehrmals durchlaufen wird (Alternierung) und es nicht unendlich viele Zustände gibt (Divergenz), dann ist die Folge endlich.
    \item Außerdem mathematisch die Abbildung von Transformationen auf Turing-Maschinen zeigen und damit ableiten, dass allgemeine Netzwerke erstmal nicht terminieren müssen (Abbildung auf Halteproblem)
\end{itemize}

Zielsetzung die Zweite:
\begin{itemize}
    \item Wir definieren möglichst minimale Beschränkungen, die dazu führen, dass das Netzwerk terminiert. D.h. es terminiert entweder konsistent oder es terminiert mit einem Fehler, der sagt, dass entweder keine Konsistenz hergestellt werden kann (es gibt keine Ausführungsreihenfolge der Transformationen, die zu Konsistenz führt) oder dass die Anwendungsfunktion nicht in der Lage war eine passende Ausführungsreihenfolge zu finden (Konservativität)
    \item Zwei Arten von Beschränkungen
    \begin{itemize}
        \item Beschränkungen an die Transformationen, die dazu führen, dass es in mehr Fällen mindestens eine Ausführungsreihenfolge gibt, in der das Netzwerk konsistent terminiert
        \item Beschränkungen an die Ausführungsfunktion, sodass die Ausführung auf jeden Fall terminiert, wenn auch konservativ, d.h. mit Fehler, obwohl es eine korrekte Lösung gegeben hätte.
    \end{itemize}
\end{itemize}



Exkurs: Menge (konsistenter) Modelle bildet keinen topologischen Raum
\begin{itemize}
    \item Topologischer Raum besteht aus Grundmenge und Mengensystem von Teilmengen mit den Eigenschaften, dass die Grundmenge offen ist, der Schnitt endlich vieler Mengen offen und die Vereinigung beliebig vieler Mengen offen ist. 
    \item Die Grundmenge wäre die Menge aller Modellelemente
    \item Diese Menge ist normalweise offen, da z.B. für ein Element mit einem String-Attributwert immer noch das Element mit dem gleichen String-Attributwert plus einem weiteren Symbol in der Menge liegt (und man die Ordnung in der Menge entsprechend definiert). Dass ein Metamodell möglicherweise Einschränkungen definiert und dann im schlimmsten Fall nur ein einziges Modell valide ist, lassen wir hier außen vor.
    \item Betrachten wir nun eine Topologie auf dieser Menge, also ein Mengensystem aus konsistenten Modellen. Leider ist jedoch der Schnitt zweier konsistenter Modelle nicht zwangsläufig konsistent. Insbesondere sind diese Mengen auch nicht offen, da sie die abgeschlossene Menge darstellen, die genau ein Modell beschreiben. 
    \item Somit lässt sich die Definition von Topologien hier nicht anwenden.
\end{itemize}

\todo{Überlegen, wo hier die Definition von (undirektionalen Relationen) rein muss.}
Präzisere Eigenschaften:
\begin{itemize} 
    \item Synchronisationseigenschaft: Eine Transformation kann mit Änderungen an mehreren Modellen umgehen, d.h. gegeben zwei konsistente Modelle + Änderungen an beiden resultiert in zwei Modellen, die konsistent bzgl. der Relation(en) zwischen den Metamodellen sind
    \item 
\end{itemize}  




\begin{itemize}
    \item Kompatibilität entsprechend Modularisierungsebene
    \item Synchronisation auf Operationalisierungsebene: Abwägen, dass eine Transformation verschiedene Zustände sehen könnte, auf denen sie ausgeführt wird. Aber letztendlich muss sie damit klarkommen, dass zwei Modelle geändert wurden. 
\end{itemize}

TODO:
\begin{itemize}
    \item Authoritative Modelle (bzw. eher authoritative Regionen) diskutieren (Verweis Stevens)
\end{itemize}

\input{sections/3_correctness/3110_formal_compatibility}
\section{A Formal Approach to Prove Compatibility}

\begin{copiedFrom}{SoSym MPM4CPS}

In this section, we use the definition of compatibility to derive a formal approach for proving compatibility of consistency relations.
The approach bases on two ideas:
\begin{enumerate}
    \item A set of consistency relations in which each pair of classes is only related across one concatenation of relations is inherently compatible, because there cannot be any contradictory relations. We precisely define this in a specific notion of \emph{consistency relation trees}.
    \item A consistency relation that is redundant in a set of relations, i.e., a relation that does not alter the notion of consistency for models regarding the other relations in that set, does not affect compatibility and can thus be removed from that set of relations. % in which it is redundant.
\end{enumerate}
Given a set of consistency relations, compatibility can be proven inductively if a consistency relation tree that is equivalent to the set of relations can be found by only removing redundant relations from that set.
Finding such an equivalent consistency relation tree serves as a \emph{witness} for compatibility of a set of relations.
In the following, we formalize and prove this inductive approach to check compatibility of a set of consistency relations.
This constitutes our contribution \ref{contrib:formalapproach}.

The sketched approach for witnessing compatibility is based on a definition of equivalence for sets of consistency relations.
We consider two sets of consistency relations equivalent if they consider the same sets of models as consistent:

\begin{definition}[Equivalence of Consistency Relations]
\label{def:equivalence}
    Let $\consistencyrelationset{CR}_{1}, \consistencyrelationset{CR}_{2}$ be two sets of consistency relations defined for a set of metamodels $\metamodelset{M}$. % = \setted{\metamodel{M}{1}, \ldots, \metamodel{M}{k}}$.
    We say that:
    \begin{align*}
        \formulaskip
        &
        \consistencyrelationset{CR}_{1} \equivalenttomath \consistencyrelationset{CR}_{2} \equivalentperdefinition \forall \modelset{m} \in \metamodelinstances{\metamodelset{M}} : \\
        %& \formulaskip
        %\forall \modelset{m} = \setted{\model{m}{1}, \dots \model{m}{k}}, \model{m}{i} \in \metamodelinstances{\metamodel{M}{i}} : \\ 
        & \formulaskip%\formulaskip
        \modelset{m} \consistenttomath \consistencyrelationset{CR}_{1} \equivalent \modelset{m} \consistenttomath \consistencyrelationset{CR}_{2}
    \end{align*}
\end{definition}

%Two sets of consistency relations are considered equivalent if any set of models is either considered consistent or not by both of them in the same way.

The goal of our approach is to find a set of consistency relations that is compatible and equivalent to a given consistency relation set.
We will later use equivalence to introduce a specific notion of redundancy that is compatibility-preserving.
In the following, we first consider structures of consistency relation sets that are inherently compatible and afterwards consider redundancy as a means to find an equivalent representation of a relation set that has such a structure.

\input{sections/3_correctness/3121_formal_approach_trees}
\subsection{Redundancy as Witness for Compatibility}
\label{sec:formalapproach:redundancy}

\todoDiss{Add a definition for a \emph{compatibility-preserving consistency relation} that states that is preserved compatibility and use that for clarifying the following lemmas and theorems.}

%%
%% Problem: Not having a compatible structure, compatibility is unclear
%%
We have introduced specific structures of consistency relations that are inherently compatible.
If a given set of consistency relations does not represent one of those structures, especially because there are multiple consistency relations putting the same classes into relation, it is unclear whether such a set is compatible.

%%
%% Idea: Find and virtually remove redundant relations
%%
In the following, we present an approach to reduce a set of consistency relations to a structure of independent consistency relation trees.
The essential idea is to find relations within the set, which do not change compatibility of the consistency relation set whether or not they are contained in it.
An approach that finds such relations and---virtually---removes them from the set until the remaining relations form a set of independent consistency relation trees, proves compatibility of the given set of relations.
We first define the term of a \emph{compatibility-preserving} relation.

\begin{definition}
    \label{def:compatibilitypreserving}
    Let $\consistencyrelationset{CR}$ be a compatible set of consistency relations and let $\consistencyrelation{CR}{}$ be a consistency relation. We say that:
    \begin{align*}
        \formulaskip &
        \consistencyrelation{CR}{} \compatibilitypreservingtomath \consistencyrelationset{CR} \equivalentperdefinition \\
        & \formulaskip
        \consistencyrelationset{CR} \cup \setted{\consistencyrelation{CR}{}} \compatiblemath
    \end{align*}
\end{definition}

To be able to find such a compatibility-preserving relation, we introduce the notion of \emph{redundant} relations and prove the property of being compatibility preserving.
Informally speaking, a relation is redundant if it is expressed transitively across others, i.e., if it does not restrict or relax consistency compared to a combination of other relations.
We precisely specify a notion of redundancy in the following.

% \begin{itemize}
%     \item There may also be cycles in the fine-grained relations, such that the relation graph induced by the specification cannot witness compatibility.
%     \item If it is possible to find a set of trees that is equivalent to the given graph (\formalize{what equivalent means here!}), this serves as a witness for compatibility.
%     \item Finding such an equivalent representation can be achieved by (virtually) removing relations that have no impact on the valid instances (i.e. do not reduce the degree of consistency) (\formalize{that with the definition})
%     \item A relation is redundant if it is transitively expressed across the others, i.e. if it does not restrict the valid instances of two metamodels that are considered consistent in addition to those allowed by all other relations (\formalize{with extensional definitions of consistency}).
%     \item Virtually removing such relations leads to an equivalent representation and if that representation finally forms a tree, we have a witness for compatibility.
%     \item We explain this in detail in \autoref{sec:redundancies} and discuss how theorem proving can be used to find such redundant relations.
% \end{itemize}

\begin{definition}[Redundant Consistency Relation]
\label{def:redundancy}
    Let $\consistencyrelationset{CR}$ be a set of consistency relations for a set of metamodels $\metamodelset{M}$. %  = \setted{\metamodel{M}{1}, \dots, \metamodel{M}{k}}$.
    For a consistency relation $\consistencyrelation{CR}{} \in \consistencyrelationset{CR}$, we say that:
    \begin{align*}
        \formulaskip &
        \consistencyrelation{CR}{} \redundantinmath \consistencyrelationset{CR} \equivalentperdefinition\\
        & \formulaskip 
        \exists \consistencyrelation{CR'}{} \in \transitiveclosure{(\consistencyrelationset{CR} \setminus \setted{\consistencyrelation{CR}{}})} : 
        \forall \modelset{m} \in \metamodelinstances{\metamodelset{M}} :\\
        & \formulaskip\formulaskip
        \modelset{m} \consistenttomath \consistencyrelation{CR'}{} \Rightarrow \modelset{m} \consistenttomath \consistencyrelation{CR}{}
    \end{align*}
    % \begin{align*}
    %     \formulaskip &
    %     \consistencyrelation{CR}{} \in \consistencyrelationset{CR} \mathtext{is redundant in} \consistencyrelationset{CR} \equivalentperdefinition\\
    %     & \formulaskip 
    %     \consistencyrelationset{CR} \mathtext{equivalent to} \consistencyrelationset{CR} \setminus \setted{\consistencyrelation{CR}{}}
    % \end{align*}
    % \begin{align*}
    %     \formulaskip &
    %     \consistencyrelation{CR}{} \in \consistencyrelationset{CR} \mathtext{is redundant in} \consistencyrelationset{CR} \equivalentperdefinition\\
    %     %& \formulaskip
    %     %\forall \modelset{m} = \setted{\model{m}{1}, \dots, \model{m}{k}} \mid \model{m}{i} \in \metamodelinstances{\metamodel{M}{i}} : \\
    %     & \formulaskip 
    %     \exists \consistencyrelation{CR'}{} \in \transitiveclosure{\consistencyrelationset{CR}} : \consistencyrelation{CR'}{} \mathtext{overlapping with} \consistencyrelation{CR}{} \\
    %     & \formulaskip
    %     \land \forall \consistencyrelation{CR'}{} \in \transitiveclosure{\consistencyrelationset{CR}} \mid \consistencyrelation{CR'}{} \mathtext{overlapping with} \consistencyrelation{CR}{} :\\
    %     & \formulaskip\formulaskip
    %     \bigl(\forall \tupled{\conditionelement{c}{l}, \conditionelement{c}{r}} \in \consistencyrelation{CR}{} : \exists \tupled{\conditionelement{c'}{l}, \conditionelement{c'}{r}} \in \consistencyrelation{CR'}{} : \forall \modelset{m} \in \metamodelinstances{\metamodelset{M}} : \\
    %     & \formulaskip\formulaskip\formulaskip\modelset{m} \mathtext{contains} \tupled{\conditionelement{c}{l}, \conditionelement{c}{r}} \equivalent \modelset{m} \mathtext{contains} \tupled{\conditionelement{c'}{l}, \conditionelement{c'}{r}} \\
    %     & \formulaskip\formulaskip
    %     \land \forall \tupled{\conditionelement{c'}{l}, \conditionelement{c'}{r}} \in \consistencyrelation{CR'}{} : \exists \tupled{\conditionelement{c}{l}, \conditionelement{c}{r}} \in \consistencyrelation{CR}{} : \forall \modelset{m} \in \metamodelinstances{\metamodelset{M}} : \\
    %     & \formulaskip\formulaskip\formulaskip\modelset{m} \mathtext{contains} \tupled{\conditionelement{c}{l}, \conditionelement{c}{r}} \equivalent \modelset{m} \mathtext{contains} \tupled{\conditionelement{c'}{l}, \conditionelement{c'}{r}} \bigr)
    % \end{align*}
\end{definition}

\todo{Add examples for redundancy! How do the elements of the redundant relation have to be related to the ones in $\consistencyrelation{CR'}{}$?}
\todoDiss{Can we define an even more general notion of redundancy, not stating about the relation to a single consistency relation but the set of consistency relation, abstracting the implication to consistency to the whole set of relations?}
The definition of redundancy of a consistency relation $\consistencyrelation{CR}{}$ ensures that there is another consistency relation, possibly transitively expressed across others, such that if a model is consistent to that other relation, it is also consistent to $\consistencyrelation{CR}{}$.
This means that there are no model sets that are considered inconsistent to $\consistencyrelation{CR}{}$, but not to another relation, thus $\consistencyrelation{CR}{}$ does not restrict consistency.
Actually, the definition of redundancy implies that the set of consistency relations with and without the redundant one are equivalent according to \autoref{def:equivalence}, thus both consider the same model sets as consistent.

% The definition of redundancy ensures that the redundant relation does not provide any relaxation (first equivalence) or restriction (second equivalence) regarding existing consistency relations and that there is at least one overlapping consistency relation, as otherwise the relation will always restrict consistent models.
% Intuitively, redundancy could also be defined by requiring equivalence of the set of consistency relations with and without the redundant relation. In that case, exactly the same models would be considered consistent with and without the redundant relation.
% However, we want to use the redundancy definition to make statements about compatibility of sets of consistency relations, which requires this more restricted notion of redundancy.
% Actually, the definition of redundancy always implies equivalence.
\todoDiss{Explain that we do not require equality of elements in CR and CR' because, e.g., CR might only related names, whereas CR' related names and addresses, thus we only require that there are elements that are co-indicating consistency.}

\begin{lemma} \label{lemma:redundancyimpliesequivalence}
    Let $\consistencyrelation{CR}{} \in \consistencyrelationset{CR}$ be a redundant consistency relation in a relation set $\consistencyrelationset{CR}$. %, according to \autoref{def:redundancy}.
    Then $\consistencyrelationset{CR}$ is equivalent to $\consistencyrelationset{CR} \setminus \setted{\consistencyrelation{CR}{}}$. %, according to \autoref{def:equivalence}.
\end{lemma}

\begin{proof}
    Like discussed in \autoref{lemma:consistencytransitiveclosure}, adding a consistency relation to a set of consistency relations can never lead to a relaxation of consistency, i.e., models becoming consistent that were not considered consistent before. This is a direct consequence of \autoref{def:consistency} for consistency, which requires models be consistent to all consistency relations in a set to be considered consistent, thus restricting the set of consistent model sets by adding further consistency relations.
    In consequence, it holds that:
    \begin{align*}
        \formulaskip
        \modelset{m} \consistenttomath \consistencyrelationset{CR} \Rightarrow 
        \modelset{m} \consistenttomath \consistencyrelationset{CR} \setminus \setted{\consistencyrelation{CR}{}}
    \end{align*}
    Additionally, a direct consequence of \autoref{def:redundancy} for redundancy is that a redundant consistency relation does not restrict consistency, as it considers all models to be consistent that are also considered consistent to another consistency relation in the transitive closure of the consistency relation set. Thus, all models that are considered consistent to the transitive closure of $\consistencyrelationset{CR} \setminus \setted{\consistencyrelation{CR}{}}$ are also consistent to $\consistencyrelation{CR}{}$ and thus to all relations in $\consistencyrelationset{CR}$:
    \begin{align*}
        \formulaskip
        \modelset{m} \consistenttomath \transitiveclosure{(\consistencyrelationset{CR} \setminus \setted{\consistencyrelation{CR}{}})} \Rightarrow 
        \modelset{m} \consistenttomath \consistencyrelationset{CR}
    \end{align*}
    According to \autoref{lemma:consistencytransitiveclosure}, each set of models that is consistent to a consistency relation set is also consistent to its transitive closure an vice versa.
    In consequence, the previous implication is also true for $\consistencyrelationset{CR} \setminus \setted{\consistencyrelation{CR}{}}$ rather than $\transitiveclosure{(\consistencyrelationset{CR} \setminus \setted{\consistencyrelation{CR}{}})}$.
    Summarizing, $\consistencyrelationset{CR}$ and $\consistencyrelationset{CR} \setminus \setted{\consistencyrelation{CR}{}}$ are equivalent.
\end{proof}

\todoDiss{Possibly add that lemma}
% \begin{lemma}
%     Let $\consistencyrelationset{CR}$ be a set of consistency relation and let $\consistencyrelation{CR}{}$ be redundant in $\consistencyrelationset{CR}$.
%     Then it holds that:
%     \begin{align*}
%         \formulaskip &
%         \exists \consistencyrelation{CR'}{} \in \transitiveclosure{\consistencyrelationset{CR}} \setminus \setted{\consistencyrelation{CR}{}} : \consistencyrelation{CR}{} \mathtext{related to} \consistencyrelation{CR'}{}\\
%         &
%         \lor \forall \modelset{m} \in \metamodelinstances{\metamodelset{M}} : \modelset{m} \mathtext{consistent to} \consistencyrelation{CR}{}
%     \end{align*}
% \end{lemma}

% \begin{proof}
%     tba \todoHeiko{Add the proof}
% \end{proof}

\begin{figure}
    \centering
    \newcommand{\hdistance}{19em}
\newcommand{\vdistance}{1.5em}
\newcommand{\classwidth}{6em}

\begin{tikzpicture}

% Resident
\umlclassvarwidth{resident}{}{Resident\sameheight}{
name
}{\classwidth}

% Employee
\umlclassvarwidth[, right=\hdistance of resident.north, anchor=north]{employee}{}{Employee\sameheight}{
name
}{\classwidth}

% Location
\umlclassvarwidth[, below=\vdistance of resident.south, anchor=north]{location}{}{Location\sameheight}{
street
}{\classwidth}

% Address
\umlclassvarwidth[, below=\vdistance of employee.south, anchor=north]{address}{}{Address\sameheight}{
street
}{\classwidth}

% CONSISTENCY RELATIONS
\draw[consistency relation] ([yshift=1em]resident.east) -- node[pos=0, above right] {$r$} node[pos=0.5, above] {$\consistencyrelation{CR}{1}$} node[pos=1, above left] {$e$} ([yshift=1em]employee.west);
\draw[consistency relation, -] ([yshift=1em]$(employee.west)!0.8!(employee.west-|resident.east)$) |- node[pos=1, above right] {$l$} (location.east);
\draw[consistency relation 2] ([yshift=-1em]resident.east) -- node[pos=0, below right] {$r$} node[pos=0.5, below] {$\consistencyrelation{CR}{2}$} node[pos=1, below left] {$e$} ([yshift=-1em]employee.west);
\draw[consistency relation 2] ([yshift=-1em]$(employee.west)!0.2!(employee.west-|resident.east)$) |- node[pos=1, above left] {$a$} (address.west);

\node[consistency related element, below=2.5em of location.west, anchor=north west] (relation1) {
$\begin{aligned}
    \consistencyrelation{CR}{1} =\; & \setted{\tupled{(r,l),e} \mid r.name  \neq "" \\
    &\land (r.name = e.name \lor r.name = e.name.toLower)}
\end{aligned}$
};   
\node[consistency related element 2, below=0.3em of relation1.south west, anchor=north west] { 
$\begin{aligned}
    \consistencyrelation{CR}{2} =\; & \setted{\tupled{r,(e,a)} \mid r.name = e.name \land a.street \neq ""}
\end{aligned}$
};

\end{tikzpicture}
    %\includegraphics[width=\columnwidth]{figures/redundancy_relation_extremes.png}
    \caption{Redundant consistency relation $\consistencyrelation{CR}{1}$ in $\setted{\consistencyrelation{CR}{1}, \consistencyrelation{CR}{2}}$}
    \label{fig:correctness:formal:redundancyrelationextremes}
\end{figure}

In general, to consider a consistency relation redundant in %a set of consistency relation
$\consistencyrelationset{CR}$, it has to define equal or weaker requirements for consistency than one of the other relations in $\consistencyrelationset{CR}$.
Informally speaking, such weaker requirements mean that the redundant relation must have weaker conditions, i.e., it must require consistency for less objects and consider the same or more objects consistent to each of the left condition elements. %, i.e., it must have a weaker left condition, and consider the same or more elements consistent to those of the left condition.

\begin{example}
Such weaker consistency requirements are exemplified in \autoref{fig:correctness:formal:redundancyrelationextremes}, which shows a consistency relation $\consistencyrelation{CR}{1}$ that is redundant in $\setted{\consistencyrelation{CR}{1}, \consistencyrelation{CR}{2}}$.
A redundant consistency relation, such as $\consistencyrelation{CR}{1}$, must have weaker requirements in the left condition, such that it requires consistent elements to exist in less cases.
This means that it may have a larger set of classes that are matched and that there may be less condition elements for which consistency is required.
In case of $\consistencyrelation{CR}{1}$, the left condition contains both a resident and a location, whereas the left condition of $\consistencyrelation{CR}{2}$ only contains residents.
Thus $\consistencyrelation{CR}{1}$ requires consistent elements, i.e., employees, only if a resident and a location exists, whereas $\consistencyrelation{CR}{2}$ requires that already for an existing resident.
Furthermore, the residents for which $\consistencyrelation{CR}{1}$ defines any consistency requirements are a subset of those for which $\consistencyrelation{CR}{2}$ defines consistency requirements, as $\consistencyrelation{CR}{1}$ does not make any statements about residents having an empty name.
Thus, the left condition elements of $\consistencyrelation{CR}{1}$ are a subset of those of $\consistencyrelation{CR}{2}$.
In consequence, if $\consistencyrelation{CR}{1}$ requires consistency for a resident and a location, $\consistencyrelation{CR}{2}$ requires it anyway, because it already defines consistency for the contained resident.

Additionally, a redundant consistency relation, such as $\consistencyrelation{CR}{1}$, must have weaker requirements for the elements at the right side, such that one of the consistent right condition elements is contained anyway because another relation already required them. 
This means that the relation may have a smaller set of classes, of whom instances are required to consider the models consistent, and there may be more condition elements of the right side that are considered consistent with condition elements of the left side to not restrict the elements considered consistent.
In case of $\consistencyrelation{CR}{1}$, it only requires an emploee to exist for a resident compared to $\consistencyrelation{CR}{2}$, which also requires a non-empty address to exist. Additionally, $\consistencyrelation{CR}{1}$ does not restrict the employees that are considered consistent to employees compared to $\consistencyrelation{CR}{2}$, as it also considers employees with the same name as consistent, but additionally those having the name of the resident in lowercase.
\end{example}

\todoDiss{Add proposition about redundancy properties}
% These informal insights on the properties of a redundant consistency relation can be formalized as follows.

% \begin{proposition}
%     Let $\consistencyrelationset{CR}$ be a set of consistency relations and let $\consistencyrelation{CR}{}$ be a consistency relation. Then it holds that:
%     \begin{align*}
%         \formulaskip &
%         \consistencyrelation{CR}{} \redundantinmath \consistencyrelationset{CR} \cup \setted{\consistencyrelation{CR}{}} \equivalent \\
%         & \formulaskip
%         \exists \consistencyrelation{CR'}{} \in \consistencyrelationset{CR} : %\\
%         %& \formulaskip
%         \classtuple{C}{l,\consistencyrelation{CR'}{}} \subseteq \classtuple{C}{l,\consistencyrelation{CR}{}} \land
%         \classtuple{C}{r,\consistencyrelation{CR}{}} \subseteq \classtuple{C}{r,\consistencyrelation{CR'}{}} \\
%         & \formulaskip
%         \land \forall \conditionelement{c}{l} \in \condition{c}{l,\consistencyrelation{CR}{}} : \exists \conditionelement{c'}{l} \in \condition{c}{l,\consistencyrelation{CR'}{}} : \bigl( 
%         \conditionelement{c'}{l} \subseteq \conditionelement{c}{l} \\
%         & \formulaskip\formulaskip
%         \land \forall \tupled{\conditionelement{c'}{l},\conditionelement{c'}{r}} \in \consistencyrelation{CR'}{} : \exists \tupled{\conditionelement{c}{l},\conditionelement{c}{r}} \in \consistencyrelation{CR}{} : \conditionelement{c}{r} \subseteq \conditionelement{c'}{r} \big)
%     \end{align*}
% \end{proposition}

% \begin{proof}
%     tba
% \end{proof}


\begin{figure}
    \centering
    \newcommand{\hdistance}{19em}
\newcommand{\vdistance}{2em}
\newcommand{\classwidth}{6em}

\begin{tikzpicture}

% Employee
\umlclassvarwidth{employee}{}{Employee\sameheight}{
name
}{\classwidth}

% Person
\umlclassvarwidth[, right=\hdistance of employee.north, anchor=north]{person}{}{Person\sameheight}{
name
}{\classwidth}

% Resident
\umlclassvarwidth[, below=\vdistance of employee.south, anchor=north]{resident}{}{Resident\sameheight}{
name
}{\classwidth}


% CONSISTENCY RELATIONS
\draw[consistency relation] ([yshift=1em]employee.east) -- node[pos=0, above right] {$e$} node[pos=0.5, above] {$\consistencyrelation{CR}{1}$} node[pos=1, above left] {$p$} ([yshift=1em]person.west);
\draw[consistency relation, -] ([yshift=1em]$(person.west)!0.8!(person.west-|employee.east)$) |- node[pos=1, above right] {$l$} ([yshift=1em]resident.east);

\draw[consistency relation 2] ([yshift=-1em]employee.east) -- node[pos=0, below right] {$e$} node[pos=0.5, below] {$\consistencyrelation{CR}{2}$} node[pos=1, below left] {$p$} ([yshift=-1em]person.west);

\draw[consistency relation 2] (person.south) |- node[pos=0, below right] {$p$} node[pos=0.6, above] {$\consistencyrelation{CR}{3}$} node[pos=1, below right] {$r$} ([yshift=-1em]resident.east);

\node[consistency related element, below left=2.5em and 1em of resident.west, anchor=north west] (relation1) {
$\begin{aligned}
    \consistencyrelation{CR}{1} =\; & \setted{\tupled{(e,r),p} \mid e.name = r.name.toUpper \land e.name  = p.name}
\end{aligned}$
};   
\node[consistency related element 2, below=0.3em of relation1.south west, anchor=north west] { 
$\begin{aligned}
    \consistencyrelation{CR}{2} =\; & \setted{\tupled{e,p} \mid e.name = p.name}\\
    \consistencyrelation{CR}{3} =\; & \setted{\tupled{p,r} \mid r.name = p.name.toLower}
\end{aligned}$
};

\end{tikzpicture}
    %\includegraphics[width=\columnwidth]{figures/redundancy_compatibility_counterexample.png}
    \caption{A consistency relation $\consistencyrelation{CR}{1}$ being redundant in  $\setted{\consistencyrelation{CR}{1}, \consistencyrelation{CR}{2}, \consistencyrelation{CR}{3}}$, with $\setted{\consistencyrelation{CR}{2}, \consistencyrelation{CR}{3}}$ being compatible and $\setted{\consistencyrelation{CR}{1}, \consistencyrelation{CR}{2},\consistencyrelation{CR}{3}}$ being incompatible.}
    \label{fig:correctness:formal:redundancy_compatibility_counterexample}
\end{figure}

Our goal is to have a compatibility-preserving notion of redundancy, i.e., adding a redundant relation to a compatible relation set should preserve compatibility.
Unfortunately, our intuitive redundancy definition is not compatibility-preserving. % with our intuitive notion of redundancy, a consistency relation $\consistencyrelation{CR}{}$ that is redundant to a compatible set of consistency relations $\consistencyrelationset{CR}$ does not imply that $\consistencyrelationset{CR} \cup \setted{\consistencyrelation{CR}{}}$ is compatible.

\begin{proposition} \label{prop:redundantnotimpliescompatible}
    Let $\consistencyrelationset{CR}$ be a compatible set of consistency relations and let $\consistencyrelation{CR}{}$ be a consistency relation that is redundant in $\consistencyrelationset{CR} \cup \setted{\consistencyrelation{CR}{}}$.
    Then $\consistencyrelation{CR}{}$ is not necessarily compatibility-preserving, i.e., $\consistencyrelationset{CR} \cup \setted{\consistencyrelation{CR}{}}$ is not necessarily compatible.
    % it holds that:
    % \begin{align*}
    %     \formulaskip &
    %     \consistencyrelationset{CR} \compatiblemath \not\Rightarrow \consistencyrelationset{CR} \cup \setted{\consistencyrelation{CR}{}} \compatiblemath
    % \end{align*}
\end{proposition}

\begin{proof}
We prove the proposition by providing a counterexample. % for the implication.
Consider the example in \autoref{fig:correctness:formal:redundancy_compatibility_counterexample}. 
$\consistencyrelation{CR}{2}$ relates each employee to a person with the same name and $\consistencyrelation{CR}{3}$ relates each person to a resident with the same name in lowercase.
The consistency relation set $\setted{\consistencyrelation{CR}{2}, \consistencyrelation{CR}{3}}$ is obviously compatible, because for each employee and each person, which constitute the left condition elements of the consistency relations, a consistent model set containing the person respectively employee can be created by adding the appropriate person or employee with the same name and a resident with the name in lowercase.
Furthermore, $\consistencyrelation{CR}{1}$ is redundant in $\setted{\consistencyrelation{CR}{1}, \consistencyrelation{CR}{2}, \consistencyrelation{CR}{3}}$ according to \autoref{def:redundancy}, because if a model is consistent to $\consistencyrelation{CR}{2}$ it is also consistent to $\consistencyrelation{CR}{1}$, since $\consistencyrelation{CR}{1}$ also requires persons with the same name as an employee to be contained in a model set but in less cases, precisely only those in which the model also contains a resident such that the employee name is the one of the resident in uppercase.

However, $\setted{\consistencyrelation{CR}{1}, \consistencyrelation{CR}{2}, \consistencyrelation{CR}{3}}$ is not compatible.
Intuitively, this is due to the fact that $\consistencyrelation{CR}{1}$ and $\consistencyrelation{CR}{3}$ define an incompatible mapping between the names of residents and persons.
This is also reflected by \autoref{def:compatibility} for compatibility. Take a model with an employee and a resident named $A$. This is a condition element in $\condition{c}{l,\consistencyrelation{CR}{1}}$. 
Consequentially, $\consistencyrelation{CR}{1}$ requires a person $A$ to exist. Furthermore $\consistencyrelation{CR}{3}$ requires a resident with name $a$ to exist.
In consequence, there are two tuples of employees and residents, both with employee $A$ and one with resident $A$ respectively resident $a$ each, for which a consistent person with name $A$ is required by $\consistencyrelation{CR}{1}$.
However, $\consistencyrelation{CR}{1}$ actually forbids to have two residents, one having the lowercase name of the other, because both are condition elements in $\consistencyrelation{CR}{1}$ requiring an appropriate person to occur in a consistent model, but there is only one person that to which both can be mapped, namely the one with the uppercase name, so there is no witness structure with a unique mapping as required by \autoref{def:consistency} for consistency.
This example shows that adding a redundant consistency relation to a compatible set of consistency relations does not lead to a compatible consistency relation set.
\end{proof}

In consequence of \autoref{prop:redundantnotimpliescompatible}, we need a stronger definition of redundancy that is compatibility-preserving. 
%to be able to derive compatibility for a consistency relation set from adding a redundant relation to an already compatible consistency relation set.
In the example in \autoref{fig:correctness:formal:redundancy_compatibility_counterexample} showing \autoref{prop:redundantnotimpliescompatible}, we have seen that it is problematic if a redundant consistency relation considers more classes in its left condition than the relation it is redundant to.
Therefore, we restrict the left class tuple.

\begin{definition}[Left-equal Redundant Consistency Relation] \label{def:leftequalredundancy}
    Let $\consistencyrelationset{CR}$ be a set of consistency relations for a metamodel set $\metamodelset{M}$.
    For a consistency relation $\consistencyrelation{CR}{} \in \consistencyrelationset{CR}$, we say:
    \begin{align*}
        \formulaskip &
        \consistencyrelation{CR}{} \leftequalredundantinmath \consistencyrelationset{CR} \equivalentperdefinition \\
        & \formulaskip 
        \exists \consistencyrelation{CR'}{} \in \transitiveclosure{(\consistencyrelationset{CR} \setminus \setted{\consistencyrelation{CR}{}})} : 
        \forall \modelset{m} \in \metamodelinstances{\metamodelset{M}} :\\
        & \formulaskip\formulaskip
        \modelset{m} \consistenttomath \consistencyrelation{CR'}{} \Rightarrow \modelset{m} \consistenttomath \consistencyrelation{CR}{} \\
        & \formulaskip\formulaskip
        \land \classtuple{C}{l,\consistencyrelation{CR}{}} = \classtuple{C}{l,\consistencyrelation{CR'}{}}
        %\consistencyrelation{CR}{} \redundantinmath \consistencyrelationset{CR} \\
        %& \formulaskip
        %\land \exists \consistencyrelation{CR'}{} \in \transitiveclosure{(\consistencyrelationset{CR} \setminus \setted{\consistencyrelation{CR}{}})} : 
        %\condition{c}{l,\consistencyrelation{CR}{}} \subseteq \condition{c}{l,\consistencyrelation{CR'}{}} 
        %\forall \conditionelement{c}{l} \in \condition{c}{l,\consistencyrelation{CR}{}} :
        %\exists \conditionelement{c'}{l} \in \condition{c}{l, \consistenyrelation{CR'}{}} :
        %\forall \modelset{m} \in \metamodelinstances{\metamodelset{M}} :
    \end{align*}
\end{definition}

%The definition of left-equal redundancy restricts the notion of redundancy to cases in which the left side of the redundant consistency relation $\consistencyrelation{CR}{}$ considers instances of the same classes as another relation in the set of consistency relations.
%As discussed before, redundancy in general allows that the left side of a redundant consistency relation $\consistencyrelation{CR}{}$ considers more classes than another relation in the set of consistency relations that induces consistency of a model set to $\consistencyrelation{CR}{}$, according to the definition of redundancy.

The definition of left-equal redundancy is similar to the redundancy definition but restricts the notion of redundancy to cases in which the left condition of the redundant consistency relation $\consistencyrelation{CR}{}$ considers the same classes than the other relation in the set of consistency relations that induces consistency of a model set to $\consistencyrelation{CR}{}$.
As discussed before, redundancy in general allows that the left condition of a redundant consistency relation can consider a superset of those classes. %than another relation in the set of consistency relations that induces consistency of a model set to $\consistencyrelation{CR}{}$, according to the definition of redundancy.

\begin{lemma} \label{lemma:leftequalredundancyimpliesredundancy}
    Let $\consistencyrelation{CR}{}$ be a consistency relation that is left-equal redundant in a set of consistency relations $\consistencyrelationset{CR}$. Then $\consistencyrelation{CR}{}$ is redundant in $\consistencyrelationset{CR}$.
\end{lemma}

\begin{proof}
    Since the definition of left-equal redundancy is equal to the one for redundancy, apart from the additional restriction for the class tuples, redundancy of a left-equal redundant relation is a direct implication of the definition.
\end{proof}


Before showing that left-equal redundancy is compatibility-preserving, we introduce an auxiliary lemma that shows that if a model set contains any left condition element of a left-equal redundant relation, i.e., if that redundant relation requires the model set to contain corresponding elements for that object tuple to be consistent, there is also another relation that requires corresponding elements for that object tuple.

%In the following, we will show that the notion of left-equal redundancy, in comparison to the weaker general redundancy property, can be used to inductively prove compatibility of a set of consistency relations.

\begin{lemma} \label{lemma:leftequalredundancysubset}
    Let $\consistencyrelation{CR}{}$ be a consistency relation that is left-equal redundant in a set of consistency relations $\consistencyrelationset{CR}$ for a set of metamodels $\metamodelset{M}$. Then it holds that: 
    \begin{align*}
        \formulaskip &
        \exists \consistencyrelation{CR'}{} \in \transitiveclosure{(\consistencyrelationset{CR} \setminus \setted{\consistencyrelation{CR}{}})} : 
        \forall \conditionelement{c}{l} \in \condition{c}{l, \consistencyrelation{CR}{}} : 
        \exists \conditionelement{c'}{l} \in \condition{c}{l,\consistencyrelation{CR'}{}} : \\
        & \formulaskip
        \forall \modelset{m} \in \metamodelinstances{\metamodelset{M}} : 
        \modelset{m} \containsmath \conditionelement{c'}{l} \Rightarrow 
        \modelset{m} \containsmath \conditionelement{c}{l}
    \end{align*}
\end{lemma}

\begin{proof}
    Due to left-equal redundancy of $\consistencyrelation{CR}{}$ in $\consistencyrelationset{CR}$, we know per definition that:
    \begin{align*}
        \formulaskip &
        \exists \consistencyrelation{CR'}{} \in \transitiveclosure{(\consistencyrelationset{CR} \setminus \setted{\consistencyrelation{CR}{}})} :
        \forall \modelset{m} \in \metamodelinstances{\metamodelset{M}} : \\
        & \formulaskip
        \modelset{m} \consistenttomath \consistencyrelation{CR'}{} \Rightarrow \modelset{m} \consistenttomath \consistencyrelation{CR}{} \\
        & \formulaskip
        \land 
        \classtuple{C}{l,\consistencyrelation{CR}{}} = \classtuple{C}{l,\consistencyrelation{CR'}{}}
    \end{align*}
    This implies that:
    \begin{align*}
        \formulaskip &
        \exists \consistencyrelation{CR'}{} \in \transitiveclosure{(\consistencyrelationset{CR} \setminus \setted{\consistencyrelation{CR}{}})} :
        %\forall \modelset{m} \in \metamodelinstances{\metamodelset{M}} : \\
        %& \formulaskip
        %\condition{c}{l,\consistencyrelation{CR}{}} \subseteq \condition{c}{l,\consistencyrelation{CR'}{}} 
        \forall \conditionelement{c}{l} \in \condition{c}{l,\consistencyrelation{CR}{}} :
        \conditionelement{c}{l} \in \condition{c}{l,\consistencyrelation{CR'}{}}
    \end{align*}
    Because if there was a $\conditionelement{c}{l} \in \condition{c}{l,\consistencyrelation{CR}{}}$ so that $\conditionelement{c}{l} \not\in \condition{c}{l,\consistencyrelation{CR'}{}}$, then the model set $\modelset{m}$ only consisting of $\conditionelement{c}{l}$ would be consistent to $\consistencyrelation{CR'}{}$, because it does not require any other elements to exist for considering the model set consistent, whereas there is at least one $\tupled{\conditionelement{c}{l}, \conditionelement{c}{r}} \in \consistencyrelation{CR}{}$, so that $\modelset{m}$ needs to contain $\conditionelement{c}{r}$ for considering $\modelset{m}$ consistent to $\consistencyrelation{CR}{}$, which is not given by construction.
    This shows that $\condition{c}{l,\consistencyrelation{CR'}{}}$ contains all elements in $\condition{c}{l,\consistencyrelation{CR}{}}$, so there is always at least one element from $\condition{c}{l,\consistencyrelation{CR'}{}}$ that a model set $\modelset{m}$ contains if it contains an element from $\condition{c}{l,\consistencyrelation{CR}{}}$, %namely the same one, 
    which proves the statement in the lemma.
\end{proof}

\todoDiss{The following lemma derived the property of left-equal redundancy from redundancy, which was not correct. Maybe we can find a more general notion of redundancy from which we can derive the contains implication, reviving this lemma gain.}
% \begin{lemma} \label{lemma:redundancysubset}
%     Let $\consistencyrelation{CR}{}$ be a consistency relation that is redundant in a set of consistency relations $\consistencyrelationset{CR}$ for a set of metamodels $\metamodelset{M}$. Thus there exists a consistency relation $\consistencyrelation{CR'}{} \in \transitiveclosure{(\consistencyrelationset{CR} \setminus \setted{\consistencyrelation{CR}{}})}$ with:
%     \begin{align*}
%         \formulaskip & 
%         \forall \modelset{m} \in \metamodelinstances{\metamodelset{M}} : \modelset{m} \consistenttomath \consistencyrelation{CR'}{} \Rightarrow \modelset{m} \consistenttomath \consistencyrelation{CR}{}
%     \end{align*}
%     Then it holds that:
%     \begin{align*}
%         \formulaskip &
%         \forall \conditionelement{c}{l} \in \condition{c}{l, \consistencyrelation{CR}{}} : \exists \conditionelement{c'}{l} \in \condition{c}{l, \consistencyrelation{CR'}{}} : 
%         \forall{m} \in \metamodelinstances{\metamodelset{M}} : \\
%         & \formulaskip
%         \modelset{m} \containsmath \condition{c'}{l} \Rightarrow \modelset{m} \containsmath \condition{c}{l} %\\
%         % &
%         % \land \forall \conditionelement{c}{r} \in \condition{c}{r, \consistencyrelation{CR}{}} : \exists \conditionelement{c'}{r} \in \condition{c}{r, \consistencyrelation{CR'}{}} : 
%         % \forall{m} \in \metamodelinstances{\metamodelset{M}} : \\
%         % & \formulaskip
%         % \modelset{m} \containsmath \condition{c'}{r} \Rightarrow \modelset{m} \containsmath \condition{c}{r}
%         %\conditionelement{c}{l} \subseteq \conditionelement{c'}{l}
%     \end{align*}
% \end{lemma}

% \begin{proof}
%     % Due to symmetry of the statement for $\conditionelement{c}{l}$ and $\conditionelement{c}{r}$, the proof is also symmetric, which is why we restrict the proof to $\conditionelement{c}{l}$. 
%     We prove that
%     \begin{align*}
%         \formulaskip &
%         \forall \conditionelement{c}{l} \in \condition{c}{l, \consistencyrelation{CR}{}} : \exists \conditionelement{c'}{l} \in \condition{c}{l, \consistencyrelation{CR'}{}} : 
%         %\forall{m} \in \metamodelinstances{\metamodelset{M}} : \\
%         %& \formulaskip
%         \conditionelement{c}{l} \subseteq \conditionelement{c'}{l}
%     \end{align*}
%     which directly implies the statement according to \autoref{def:conditionelementcontainment} for the containment of condition elements.
%     Let us assume the contrary, such that:
%     \begin{align*}
%         \formulaskip &
%         \exists \conditionelement{c}{l} \in \condition{c}{l, \consistencyrelation{CR}{}} : \forall \conditionelement{c'}{l} \in \condition{c}{l, \consistencyrelation{CR'}{}} : \conditionelement{c}{l} \not\subseteq \conditionelement{c'}{l}
%     \end{align*}
%     Consider that $\conditionelement{c}{l} = \tupled{\object{o}{1}, \dots \object{o}{n}} \in \condition{c}{l, \consistencyrelation{CR}{}}$.
%     Now select a model set $\modelset{m} \in \metamodelinstances{\metamodelset{M}}$, which only contains objects $\object{o'}{1}, \dots, \object{o'}{n}$, such that $\forall i \in \setted{1, \dots, n} : \object{o}{i} \subseteq \object{o'}{i}$. In other words, we select a minimal model set that contains $\conditionelement{c}{l}$.
%     Per definition of $\consistencyrelation{CR}{}$, there must exist at least one consistency relation pair $\tupled{\conditionelement{c}{l}, \conditionelement{c}{r}} \in \consistencyrelation{CR}{}$, in which $\conditionelement{c}{l}$ occurs.
%     Since $\modelset{m}$ does not contain any $\conditionelement{c}{r}$, $\neg (\modelset{m} \consistenttomath \consistencyrelation{CR}{})$ per definition.
%     Since $\forall \conditionelement{c'}{l} \in \condition{c}{l, \consistencyrelation{CR'}{}} : \conditionelement{c}{l} \not\subseteq \conditionelement{c'}{l}$, there is no such $\conditionelement{c'}{l}$ with $\modelset{m} \containsmath \conditionelement{c'}{l}$.
%     \dots
%     \todoHeiko{Correct and finish proof}
% \end{proof}

\begin{theorem} \label{theorem:redundancycompatibility}
    Let $\consistencyrelationset{CR}$ be a compatible set of consistency relations for a set of metamodels $\metamodelset{M}$ and let $\consistencyrelation{CR}{}$ be a consistency relation that is left-equal redundant in $\consistencyrelationset{CR} \cup \setted{\consistencyrelation{CR}{}}$. Then $\consistencyrelationset{CR} \cup \setted{\consistencyrelation{CR}{}}$ is compatible. 
    % If two sets of consistency relations $\set{\consistencyrelation[1]{CR}}$ and $\set{\consistencyrelation[2]{CR}}$ are equivalent and $\set{\consistencyrelation[1]{CR}}$ is compatible, then $\set{\consistencyrelation[2]{CR}}$ is compatible as well.
\end{theorem}

\begin{proof}
    Due to left-equal redundancy of $\consistencyrelation{CR}{}$ in $\consistencyrelationset{CR} \cup \setted{\consistencyrelation{CR}{}}$, which also implies general redundancy according to \autoref{def:redundancy}, $\consistencyrelationset{CR}$ and $\consistencyrelationset{CR} \cup \setted{\consistencyrelation{CR}{}}$ are equivalent, according to \autoref{lemma:redundancyimpliesequivalence}.
    Due to that equivalence, we know that for any model set $\modelset{m} \in \metamodelinstances{\metamodelset{M}}$:
    \begin{equation} \label{eq:redundancyconsistency}
        \formulaskip 
        \modelset{m} \mathtext{consistent to} \consistencyrelationset{CR} \equivalent     \modelset{m} \mathtext{consistent to} \consistencyrelationset{CR} \cup \setted{\consistencyrelation{CR}{}}
    \end{equation}
    It follows from \autoref{def:compatibility} for compatibility and \autoref{eq:redundancyconsistency}:
    \begin{align} \label{eq:redundancycompatibleexisting}
        \formulaskip & \nonumber
        \forall \consistencyrelation{CR'}{} \in \consistencyrelationset{CR} : \forall \conditionelement{c}{l} \in \condition{c}{l, \consistencyrelation{CR'}{}} %\cup \condition{c}{r, \consistencyrelation{CR'}{}} 
        : \exists \modelset{m} \in \metamodelinstances{\metamodelset{M}} : \\
        & \formulaskip
        \modelset{m} \containsmath \conditionelement{c}{l} \land \modelset{m} \containsmath \consistencyrelationset{CR} \cup \setted{\consistencyrelation{CR}{}}
    \end{align}
    This already shows that for $\consistencyrelationset{CR}$ the compatibility definition is fulfilled, so we need to prove that the compatibility definition is fulfilled for $\consistencyrelation{CR}{}$ as well.
    % \begin{align*}
    %     \formulaskip &
    %     \forall \tupled{\conditionelement{c}{l}, \conditionelement{c}{r}} \in \consistencyrelation{CR}{} : \exists \modelset{m} \in \metamodelinstances{\metamodelset{M}}: \\
    %     & \formulaskip
    %     \modelset{m} \mathtext{contains} \tupled{\conditionelement{c}{l}, \conditionelement{c}{r}} \land \modelset{m} \mathtext{consistent to} \consistencyrelationset{CR} \cup \setted{\consistencyrelation{CR}{}}
    % \end{align*}
    Due to compatibility of $\consistencyrelationset{CR}$ and \autoref{lemma:compatibilitytransitiveclosure} showing equality of compatibility for a consistency relation set and its transitive closure, we know that:
    \begin{align} \label{eq:compatibilityclosure}
        \formulaskip & \nonumber
        \forall \consistencyrelation{CR'}{} \in \transitiveclosure{\consistencyrelationset{CR}} : \forall \conditionelement{c}{l} \in \condition{c}{l, \consistencyrelation{CR'}{}} %\cup \condition{c}{r, \consistencyrelation{CR'}{}} 
        : \exists \modelset{m} \in \metamodelinstances{\metamodelset{M}} : \\
        & \formulaskip
        \modelset{m} \containsmath \conditionelement{c}{l} \land \modelset{m} \consistenttomath \transitiveclosure{\consistencyrelationset{CR}}
    \end{align}
    Due to left-equal redundancy of $\consistencyrelation{CR}{}$ in $\consistencyrelationset{CR} \cup \setted{\consistencyrelation{CR}{}}$, we have shown in \autoref{lemma:leftequalredundancysubset} that the following is true:
    \begin{align} \label{eq:redundancycontainment}
        \formulaskip & \nonumber 
        \exists \consistencyrelation{CR'}{} \in \transitiveclosure{\consistencyrelationset{CR}} : \forall \conditionelement{c}{l} \in \condition{c}{l, \consistencyrelation{CR}{}} : \exists \conditionelement{c'}{l} \in \condition{c}{l,\consistencyrelation{CR'}{}} : \forall \modelset{m} \in \metamodelinstances{\metamodelset{M}} : \\
        & \formulaskip
        \modelset{m} \containsmath \conditionelement{c'}{l} \Rightarrow \modelset{m} \containsmath \conditionelement{c}{l}
    \end{align}
    The combination of \autoref{eq:compatibilityclosure} and \autoref{eq:redundancycontainment} gives:
    \begin{align*}
        \formulaskip & \nonumber 
        \exists \consistencyrelation{CR'}{} \in \transitiveclosure{\consistencyrelationset{CR}} : \forall \conditionelement{c}{l} \in \condition{c}{l, \consistencyrelation{CR}{}} : \exists \conditionelement{c'}{l} \in \condition{c}{l,\consistencyrelation{CR'}{}} : \\
        & \formulaskip
        (\forall \modelset{m} \in \metamodelinstances{\metamodelset{M}} : \modelset{m} \containsmath \conditionelement{c'}{l} \Rightarrow \modelset{m} \containsmath \conditionelement{c}{l}) \\
        & \formulaskip
        \land (\exists \modelset{m} \in \metamodelinstances{\metamodelset{M}} :
        \modelset{m} \containsmath \conditionelement{c'}{l} \land \modelset{m} \mathtext{consistent to} \transitiveclosure{\consistencyrelationset{CR}})
    \end{align*}
    A simplification by combining the two last lines of that statement leads to:
    \begin{align*}
        \formulaskip & \nonumber 
        \forall \conditionelement{c}{l} \in \condition{c}{l, \consistencyrelation{CR}{}} : \exists \modelset{m} \in \metamodelinstances{\metamodelset{M}} : \\
        & \formulaskip
        \modelset{m} \containsmath \conditionelement{c}{l} \land \modelset{m} \consistenttomath \transitiveclosure{\consistencyrelationset{CR}}
    \end{align*}
    Due to \autoref{eq:redundancyconsistency} and \autoref{lemma:consistencytransitiveclosure}, which shows equality of consistency for a consistency relation set and its transitive closure, this is equivalent to:
    \begin{align} \label{eq:redundancycompatiblenew}
        \formulaskip & \nonumber 
        \forall \conditionelement{c}{l} \in \condition{c}{l, \consistencyrelation{CR}{}} : \exists \modelset{m} \in \metamodelinstances{\metamodelset{M}} : \\
        & \formulaskip
        \modelset{m} \containsmath \conditionelement{c}{l} \land \modelset{m} \consistenttomath \consistencyrelationset{CR} \cup \setted{\consistencyrelation{CR}{}}
    \end{align}
    % Together with the symmetric argumentation for $\conditionelement{c}{r}$ rather than $\conditionelement{c}{l}$, we have shown that the compatibility definition holds for $\consistencyrelation{CR}{}$:
    % \begin{align} \label{eq:redundancycompatiblenew}
    %     \formulaskip & \nonumber 
    %     \forall \conditionelement{c}{} \in \condition{c}{l, \consistencyrelation{CR}{}} \cup \condition{c}{r,\consistencyrelation{CR}{}}: \exists \modelset{m} \in \metamodelinstances{\metamodelset{M}} : \\
    %     & \formulaskip
    %     \modelset{m} \mathtext{contains} \conditionelement{c}{} \land \modelset{m} \mathtext{consistent to} \consistencyrelationset{CR} \cup \setted{\consistencyrelation{CR}{}}
    % \end{align}
    The combination of \autoref{eq:redundancycompatibleexisting} and \autoref{eq:redundancycompatiblenew} shows that $\consistencyrelationset{CR} \cup \setted{\consistencyrelation{CR}{}}$ fulfills \autoref{def:compatibility} for compatibility.
    % Assume that given equivalent consistency relation sets $\set{\consistencyrelation[1]{CR}}$ and $\set{\consistencyrelation[2]{CR}}$ are equivalent and $\set{\consistencyrelation[1]{CR}}$ is compatible, whereas $\set{\consistencyrelation[2]{CR}}$ is not.
    % Then there is a consistency relation $\consistencyrelation{CR} \in \set{\consistencyrelation[2]{CR}}$ such that for all pairs of tuples $\bigtupled{\tupled{e_{l1}, \ldots, e_{ln}}, \tupled{e_{r1}, \ldots, e{rm}}} \in \consistencyrelation{CR}$ there is no set of models $\tupled{\model[1]{m}, \ldots, \model[k]{m}}$ such that for all models $\model[i]{m}, \model[j]{m}$ in that tuple either (i) $\{e_{l1}, \ldots e_{ln} \} \not\subseteq \model[i]{m} \lor \{e_{r1}, \ldots e_{rm} \not\subseteq \model[j]{m}$ or (ii) $\{ \model[1]{m}, \ldots, \model[k]{m} \} \text{not consistent according to} \set{\consistencyrelation[2]{CR}} \setminus \{ \consistencyrelation{CR} \}$. 
\end{proof}

% \begin{proof}
%     %In the proof, we always consider model sets $\modelset{m}$ to be sets of instances of the metamodels that are related by the consistency relations in $\consistencyrelationset{CR} \cup \setted{\consistencyrelation{CR}{}}$, without further mentioning that.
%     Due to the redundancy of $\consistencyrelation{CR}{}$ in $\consistencyrelationset{CR} \cup \setted{\consistencyrelation{CR}{}}$, $\consistencyrelationset{CR}$ and $\consistencyrelationset{CR} \cup \setted{\consistencyrelation{CR}{}}$ are equivalent, according to \autoref{corollary:redundancyimpliedequivalence}.
%     Due to that equivalence, it holds that for any model set $\modelset{m}$:
%     \begin{equation} \label{eq:redundancyconsistency}
%     \formulaskip 
%     \modelset{m} \mathtext{consistent to} \consistencyrelationset{CR} \equivalent \modelset{m} \mathtext{consistent to} \consistencyrelationset{CR} \cup \setted{\consistencyrelation{CR}{}}
%     \end{equation}
%     Due to \autoref{def:compatibility} for compatibility and \autoref{eq:redundancyconsistency}, it holds that:
%     \begin{align} \label{eq:redundancycompatibleexisting}
%         \formulaskip & \nonumber
%         \forall \consistencyrelation{CR'}{} \in \consistencyrelationset{CR} : \forall \tupled{\conditionelement{c}{l}, \conditionelement{c}{r}} \in \consistencyrelation{CR'}{} : \exists \modelset{m} \in \metamodelinstances{\metamodelset{M}}: \\
%         & \formulaskip
%         \modelset{m} \mathtext{contains} \tupled{\conditionelement{c}{l}, \conditionelement{c}{r}} \land \modelset{m} \mathtext{consistent to} \consistencyrelationset{CR} \cup \setted{\consistencyrelation{CR}{}}
%     \end{align}
%     This already shows that for $\consistencyrelationset{CR}$ the compatibility definition holds, so we need to prove that the compatibility definition holds for $\consistencyrelation{CR}{}$.
%     % \begin{align*}
%     %     \formulaskip &
%     %     \forall \tupled{\conditionelement{c}{l}, \conditionelement{c}{r}} \in \consistencyrelation{CR}{} : \exists \modelset{m} \in \metamodelinstances{\metamodelset{M}}: \\
%     %     & \formulaskip
%     %     \modelset{m} \mathtext{contains} \tupled{\conditionelement{c}{l}, \conditionelement{c}{r}} \land \modelset{m} \mathtext{consistent to} \consistencyrelationset{CR} \cup \setted{\consistencyrelation{CR}{}}
%     % \end{align*}
%     Due to compatibility of $\consistencyrelationset{CR}$ and \autoref{lemma:compatibilitytransitiveclosure} and \autoref{lemma:consistencytransitiveclosure}, it holds that:
%     \begin{align} \label{eq:compatibilityclosure}
%         \formulaskip & \nonumber
%         \forall \consistencyrelation{CR'}{} \in \transitiveclosure{\consistencyrelationset{CR}} : \forall \tupled{\conditionelement{c'}{l}, \conditionelement{c'}{r}} \in \consistencyrelation{CR'}{} : \exists \modelset{m} \in \metamodelinstances{\metamodelset{M}} :\\
%         & \formulaskip
%         \modelset{m} \mathtext{contains} \tupled{\conditionelement{c'}{l}, \conditionelement{c'}{r}} \land \modelset{m} \mathtext{consistent to} \transitiveclosure{\consistencyrelationset{CR}}
%     \end{align}
%     Due to the redundancy of $\consistencyrelation{CR}{}$ in $\consistencyrelationset{CR} \cup \setted{\consistencyrelation{CR}{}}$, it holds that:
%     \begin{align} \label{eq:redundancycontainment}
%         \formulaskip & \nonumber 
%         \exists \consistencyrelation{CR'}{} \in \transitiveclosure{\consistencyrelationset{CR}} : \forall \tupled{\conditionelement{c}{l}, \conditionelement{c}{r}} \in \consistencyrelation{CR}{} : \exists \tupled{\conditionelement{c'}{l}, \conditionelement{c'}{r}} \in \consistencyrelation{CR'}{} : \forall \modelset{m} \in \metamodelinstances{\metamodelset{M}} : \\
%         & \formulaskip 
%         \modelset{m} \mathtext{contains} \tupled{\conditionelement{c}{l}, \conditionelement{c}{r}} \equivalent
%         \modelset{m} \mathtext{contains} \tupled{\conditionelement{c'}{l}, \conditionelement{c'}{r}}
%     \end{align}
%     \autoref{eq:compatibilityclosure} especially holds for the selected $\consistencyrelation{CR'}{}$ and $\tupled{\conditionelement{c'}{l}, \conditionelement{c'}{r}}$ in \autoref{eq:redundancycontainment}, such that the following holds:
%     \begin{align*}
%         \formulaskip & 
%         \exists \consistencyrelation{CR'}{} \in \transitiveclosure{\consistencyrelationset{CR}} : \forall \tupled{\conditionelement{c}{l}, \conditionelement{c}{r}} \in \consistencyrelation{CR}{} : \exists \tupled{\conditionelement{c'}{l}, \conditionelement{c'}{r}} \in \consistencyrelation{CR'}{} : \\
%         & \formulaskip 
%         \forall \modelset{m} \in \metamodelinstances{\metamodelset{M}} : \modelset{m} \mathtext{contains} \tupled{\conditionelement{c}{l}, \conditionelement{c}{r}} \equivalent
%         \modelset{m} \mathtext{contains} \tupled{\conditionelement{c'}{l}, \conditionelement{c'}{r}} \\
%         & \formulaskip
%         \land \exists \modelset{m} \in \metamodelinstances{\metamodelset{M}} : \modelset{m} \mathtext{contains} \tupled{\conditionelement{c'}{l}, \conditionelement{c'}{r}} \land \modelset{m} \mathtext{consistent to} \transitiveclosure{\consistencyrelationset{CR}}
%     \end{align*}
%     Combining the last two lines leads to:
%     \begin{align*}
%         \formulaskip & 
%         \exists \consistencyrelation{CR'}{} \in \transitiveclosure{\consistencyrelationset{CR}} : \forall \tupled{\conditionelement{c}{l}, \conditionelement{c}{r}} \in \consistencyrelation{CR}{} : \exists \tupled{\conditionelement{c'}{l}, \conditionelement{c'}{r}} \in \consistencyrelation{CR'}{} : \\
%         & \formulaskip
%         \exists \modelset{m} \in \metamodelinstances{\metamodelset{M}} : \modelset{m} \mathtext{contains} \tupled{\conditionelement{c}{l}, \conditionelement{c}{r}} \land \modelset{m} \mathtext{consistent to} \transitiveclosure{\consistencyrelationset{CR}}
%     \end{align*}
%     This implies with \autoref{lemma:consistencytransitiveclosure} and \autoref{eq:redundancyconsistency} that:
%     \begin{align} \label{eq:redundancycompatiblenew}
%         \formulaskip & \nonumber
%         \forall \tupled{\conditionelement{c}{l}, \conditionelement{c}{r}} \in \consistencyrelation{CR}{} : \exists \modelset{m} \in \metamodelinstances{\metamodelset{M}} : \\
%         & \formulaskip
%         \modelset{m} \mathtext{contains} \tupled{\conditionelement{c}{l}, \conditionelement{c}{r}} \land \modelset{m} \mathtext{consistent to} \consistencyrelationset{CR} \cup \setted{\consistencyrelation{CR}{}}
%     \end{align}
%     The combination of \autoref{eq:redundancycompatibleexisting} and \autoref{eq:redundancycompatiblenew} shows that the compatibility definition is fulfilled for $\consistencyrelationset{CR} \cup \setted{\consistencyrelation{CR}{}}$.
%     % Assume that given equivalent consistency relation sets $\set{\consistencyrelation[1]{CR}}$ and $\set{\consistencyrelation[2]{CR}}$ are equivalent and $\set{\consistencyrelation[1]{CR}}$ is compatible, whereas $\set{\consistencyrelation[2]{CR}}$ is not.
%     % Then there is a consistency relation $\consistencyrelation{CR} \in \set{\consistencyrelation[2]{CR}}$ such that for all pairs of tuples $\bigtupled{\tupled{e_{l1}, \ldots, e_{ln}}, \tupled{e_{r1}, \ldots, e{rm}}} \in \consistencyrelation{CR}$ there is no set of models $\tupled{\model[1]{m}, \ldots, \model[k]{m}}$ such that for all models $\model[i]{m}, \model[j]{m}$ in that tuple either (i) $\{e_{l1}, \ldots e_{ln} \} \not\subseteq \model[i]{m} \lor \{e_{r1}, \ldots e_{rm} \not\subseteq \model[j]{m}$ or (ii) $\{ \model[1]{m}, \ldots, \model[k]{m} \} \text{not consistent according to} \set{\consistencyrelation[2]{CR}} \setminus \{ \consistencyrelation{CR} \}$. 
    
%     % Need to redefine compatibility to proceed \dots
% \end{proof}
% \todoHeiko{the proof does not require the second part of the redundancy definition, so can we omit it? Or is there a mistake in the proof?}

\begin{corollary} \label{corollary:transitiveredundancycompatibility}
    Let $\consistencyrelationset{CR}$ be a compatible set of consistency relations and let $\consistencyrelation{CR}{1}, \dots, \consistencyrelation{CR}{k}$ be consistency relations with:
    \begin{align*}
        \formulaskip &
        \forall i \in \setted{1, \dots, k} : \\
        & \formulaskip 
        \consistencyrelation{CR}{i} \leftequalredundantinmath \consistencyrelationset{CR} \cup \setted{\consistencyrelation{CR}{1}, \dots, \consistencyrelation{CR}{i}}
    \end{align*}
    Then $\consistencyrelationset{CR} \cup \setted{\consistencyrelation{CR}{1}, \dots, \consistencyrelation{CR}{k}}$ is compatible.
\end{corollary}

\begin{proof}
    This is an inductive implication of \autoref{theorem:redundancycompatibility}, because $\consistencyrelationset{CR}$ is compatible and sequentially adding $\consistencyrelation{CR}{i}$ to $\consistencyrelationset{CR} \cup \setted{\consistencyrelation{CR}{1}, \dots, \consistencyrelation{CR}{i-1}}$ ensures that $\consistencyrelationset{CR} \cup \setted{\consistencyrelation{CR}{1}, \dots, \consistencyrelation{CR}{i}}$ is compatible, because $\consistencyrelationset{CR} \cup \setted{\consistencyrelation{CR}{1}, \dots, \consistencyrelation{CR}{i-1}}$ was compatible as well.
\end{proof}

With \autoref{corollary:transitiveredundancycompatibility}, we have shown that if we have a set of consistency relations $\consistencyrelationset{CR}$ and are able to find a sequence of redundant consistency relations $\consistencyrelation{CR}{1}, \dots {\consistencyrelation{CR}{k}}$ according to \autoref{corollary:transitiveredundancycompatibility} such that we know that $\consistencyrelationset{CR} \setminus \setted{ \consistencyrelation{CR}{1}, \dots {\consistencyrelation{CR}{k}}}$ is compatible, then it is proven that $\consistencyrelationset{CR}$ is compatible.

\input{sections/3_correctness/3123_formal_approach_summary}

\todo{Construction of valid models. Valid models may restrict the usable instances of a metamodel. Discuss impact on definitions and theorems and especially the constructive discussions within the proofs. Especially consider the meaning of references in models.}

%\subsection{TODO}
%\begin{itemize}
%    \item Afterthought: Construction of valid models. Valid models may restrict the usable instances of a metamodel. Discuss impact on definitions and theorems and especially the constructive discussions within the proofs. Especially consider the meaning of references in models.
%    \item This does not yet consider if the same meta element is used in different contexts. Thus there may be two paths between elements on the meta element level, although there will never be two paths in the instantiated models, because they occur in different contexts.
%    \item Explain conservativeness
%\end{itemize}

\end{copiedFrom} % SoSym MPM4CPS