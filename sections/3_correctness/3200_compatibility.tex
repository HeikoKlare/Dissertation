\chapter{Proving Compatibility of Consistency Relations
    \pgsize{65 p.}
}
\label{chap:compatibility}

\mnote{Consistency relations correct by construction}
Transformations, from which we construct transformation networks, are composed of consistency relations and consistency preservation rules that preserve them, as we have defined in \autoref{chap:correctness}.
We focus on binary relations and according preservation rules, which relate two metamodels.
While we have precisely defined correctness of transformations and their orchestration in a network, we found that the underlying consistency relations themselves can, from a theoretical perspective, be considered correct by construction, as there is no other artifact (be it explicit or only given implicitly) with respect to which it has to be correct.
Since we assume transformations to be developed independently and reused in a modular way, we can especially not assume a monolithic consistency relation to which the modular consistency relations must be correct (see \autoref{chap:correctness:notions_correctness:dimensions}).

\mnote{Contradictions in relations}
We have, however, already given examples for cases in which binary consistency relations are somehow contradictory.
This is the case if the developers of individual transformations have different, conflicting notions of consistency between the metamodels.
In the worst case, this can lead to the situation that no single tuple of models would be considered consistent to a set of binary consistency relations, which is obviously unwanted behavior.
We have discussed an abstract example for that case already in \autoref{chap:correctness:notions_correctness:relations}.

\begin{figure}
    \centering
    %% From motivational_example in MPM4CPS paper

\newcommand{\hdistance}{14em}
\newcommand{\vdistance}{1.6em}
\newcommand{\classwidth}{6em}
\renewcommand{\sameheight}{\vphantom{yR}}

\begin{tikzpicture}

% Person
\umlclassvarwidth{person}{}{Person\sameheight}{
firstname\\
lastname\\
address\\
income
}{\classwidth}

% Employee
\umlclassvarwidth[,above right=\vdistance and \hdistance of person.east, anchor=south]{employee}{}{Employee\sameheight}{
name\\
socsecnumber\\
salary
}{\classwidth}

\umlclassvarwidth[,below=2*\vdistance of employee.south, anchor=north]{resident}{}{Resident\sameheight}{
name\\
address\\
socsecnumber
}{\classwidth}


% CONSISTENCY RELATIONS
\draw[consistency relation] (person.north) |- node[pos=0, above left] {$p$} node[pos=0.75, above] {$\consistencyrelation{CR}{PE}$} node[pos=1, above left] {$e$} (employee.west);
\draw[consistency relation] (employee.south) -- node[pos=0, below left] {$e$} node[right, align=left] {$\consistencyrelation{CR}{ER}$ / $ \consistencyrelation{CR}{ER}'$}% /\\ $R'_{ER}$} 
node[pos=1, above left] {$r$} (resident.north);
\draw[consistency relation] (resident.west) -| node[pos=0, below left] {$r$} node[pos=0.25, below] {$\consistencyrelation{CR}{PR}$ / $\consistencyrelation{CR}{PR}'$} node[pos=1, below left] {$p$} (person.south);

\node[consistency related element, below left=5em and 2em of person.south west, anchor=north west, inner sep=0em] {
$\begin{aligned}
    \consistencyrelation{CR}{PE} =\; &
        \setted{\tupled{p,e} \mid %\\
        %&
        \mathvariable{p.firstname} + "\text{\textvisiblespace}" + \mathvariable{p.lastname} = \mathvariable{e.name}%\\
        %&
        \land \mathvariable{p.income} = \mathvariable{e.salary}
    }\\[0.4em]
    \consistencyrelation{CR}{PR} =\; &
        \setted{\tupled{p,r} \mid %\\
        %& 
        \mathvariable{p.firstname} + "\text{\textvisiblespace}" + \mathvariable{p.lastname} = \mathvariable{r.name}%\\
        %& 
        \land \mathvariable{p.address} = \mathvariable{r.address}
    }\\
    \consistencyrelation{CR}{PR}' =\; &
        \setted{\tupled{p,r} \mid %\\
        %& 
        \mathvariable{p.lastname} + ",\text{\textvisiblespace}" + \mathvariable{p.firstname} = \mathvariable{r.name}%\\
        %& 
        \land \mathvariable{p.address} = \mathvariable{r.address}
    }\\[0.4em]
    \consistencyrelation{CR}{ER} =\; &
        \setted{\tupled{e,r} \mid %\\
        %& 
        \mathvariable{e.name} = \mathvariable{r.name} %\\
        %& 
        \land \mathvariable{e.socsecnumber} = \mathvariable{r.socsecnumber}
    }\\
    \consistencyrelation{CR}{ER}' =\; &
        \setted{\tupled{e,r} \mid %\\
        %& 
        \mathvariable{e.name.toLower} = \mathvariable{r.name} %\\
        %& 
        \land \mathvariable{e.socsecnumber} = \mathvariable{r.socsecnumber}
    }
\end{aligned}$
};

\end{tikzpicture}
    \caption[Three metamodels with (in-)compatible consistency relations]{Derivation of \autoref{fig:networks:three_persons_example}: Three simple metamodels for persons, employees and residents, and three binary relations $\consistencyrelation{CR}{PE}, \consistencyrelation{CR}{PR}, \consistencyrelation{CR}{ER}$ for each pair of them, with $\consistencyrelation{CR}{PR}'$ as an alternative for $\consistencyrelation{CR}{PR}$ and $\consistencyrelation{CR}{ER}'$ as an alternative for $\consistencyrelation{CR}{ER}$. Adapted from~\owncite[Fig. 1]{klare2020compatibility-report}.}
    \label{fig:compatibility:three_persons_example_extended}
\end{figure}

\mnote{Intuitive compatibility in running example}
We recapture the running example defined in \autoref{fig:networks:three_persons_example} and extend it with alternatives for two of the binary consistency relations in \autoref{fig:compatibility:three_persons_example_extended}.
The example contains three pairwise consistency relations between persons, employees and residents.
They are defined in a way such that none of them can be omitted, because each pair shares a unique overlap in their attributes.
In the example, the consistency relations $\consistencyrelation{CR}{PE}, \consistencyrelation{CR}{PR}$ and $\consistencyrelation{CR}{ER}$ (as well as their transposed ones) are fulfilled if for each person (and each employee and resident analogously) in the models exactly one employee and one resident exist that fulfill the defined relations for names and other attributes.
According to our notion of consistency relations (\autoref{def:consistencyrelation}), it is essential that always only one such corresponding element exists.
Intuitively, these consistency relations are \emph{compatible}, as they lead to a reasonable set of model tuples that are considered consistent.

\mnote{Intuitive incompatibility in modified running example}
In contrast, considering $\consistencyrelation{CR}{PR}'$ instead of $\consistencyrelation{CR}{PR}$, the relations can never be fulfilled, because the concatenation of $\mathvariable{firstname}$ and $\mathvariable{lastname}$ from person to employee conflicts with the one from person to resident.
The relation between employees and persons assumes $\mathvariable{firstname}$ and $\mathvariable{lastname}$ in that order, whereas the relation between residents and persons assumes them to be concatenated vice versa and to be separated by a comma.
Fulfilling these relations would require an infinitely large model, as the cycle of the relations requires for each person, employee, and resident the existence of the others with $\mathvariable{firstname}$ and $\mathvariable{lastname}$ swapped and extended with a comma.
As finite models cannot fulfill this, the set of consistent model tuples would be empty.

\mnote{Further incompatibility in modified running example}
In addition, considering consistency relation $\consistencyrelation{CR}{ER}'$ instead of $\consistencyrelation{CR}{ER}$, no models containing residents with a $\mathvariable{name}$ not written in lowercase can be consistent to all relations, as depicted in the example in \autoref{fig:compatibility:intuitive_incompatibility}, which, for reasons of simplicity, omits all other attributes than the names.
A resident with a non-lowercase $\mathvariable{name}$ requires a person with equally capitalized $\mathvariable{firstname}$ and $\mathvariable{lastname}$ to exist.
This requires an employee with an equally capitalized $\mathvariable{name}$ to exist.
The relation $\consistencyrelation{CR}{ER}'$ now requires a resident with the $\mathvariable{name}$ written in lowercase to exist, which, again, requires a person with the lowercase $\mathvariable{name}$ to exist.
This, in turn, requires an employee with the lowercase $\mathvariable{name}$ to exist as well.
In consequence, the resident with the lowercase $\mathvariable{name}$ would correspond to both the employee with the original and the lowercase $\mathvariable{name}$, whereas the resident with the original $\mathvariable{name}$ does not correspond to any employee.
Since there is no witness structure with a unique mapping of corresponding elements, as also reflected in \autoref{fig:correctness:consistency_example}, such models cannot be consistent to the consistency relations.
More intuitively speaking, it is impossible to find an employee that fulfills the consistency relation $\consistencyrelation{CR}{ER}'$ for a resident with a non-lowercase $\mathvariable{name}$.
This is what we call and later precisely define as an \emph{incompatibility} of the consistency relations, as they define constraints that cannot be fulfilled at the same time.
This can always occur if there is a cycle in the graph induced by the consistency relations.

\begin{figure}
    \centering
    \newcommand{\classdistance}{5.5em}
\newcommand{\objectwidth}{7.1em}
\newcommand{\hdistance}{(2.8*\classdistance+\objectwidth+0.5*\difftoafiveimage)}

\begin{tikzpicture}

\umlobjectvarwidth[, fill=white]{person1}{}{
: Person\sameheight}{
firstname = "Alice"
lastname = "Avid"
}{\objectwidth}

\umlobjectvarwidth[, fill=white, below right=0.4*\classdistance and 0.3*\classdistance+\objectwidth of person1.north, anchor=north]{person2}{}{
: Person\sameheight}{
firstname = "alice"
lastname = "avid"
}{\objectwidth}

\umlobjectvarwidth[, fill=white, below=1.6*\classdistance of person1.north, anchor=north] {resident1}{}{
: Resident\sameheight}{
name = "Alice Avid"
}{\objectwidth}

\umlobjectvarwidth[, fill=white, right=0.3*\classdistance+\objectwidth of resident1.north, anchor=north] {resident2}{}{
: Resident\sameheight}{
name = "alice avid"
}{\objectwidth}

\umlobjectvarwidth[, fill=white, right=\hdistance of person1.north, anchor=north] {employee1}{}{
: Employee\sameheight}{
name = "Alice Avid"
}{\objectwidth}

\umlobjectvarwidth[, fill=white, below left=0.7*\classdistance and 0.3*\objectwidth of employee1.north, anchor=north]{employee2}{}{
: Employee\sameheight}{
name = "alice avid"
}{\objectwidth}

\draw[correspondence] (person1.south) -- (resident1.north);
\draw[correspondence] (person1.east|-employee1.west) -- (employee1);
\draw[correspondence] ([xshift=-0.15*\objectwidth]employee1.south east) |- (resident2);
\draw[correspondence] (resident2) -- (person2);
\draw[correspondence] (person2) -- (employee2);
\draw[correspondence] (employee2) -- (resident2);

% \coordinate[right=0.85*\objectwidth of employee2.south east] (legend);
% \node[draw, fill=lightlightgray, minimum width=1.15*\objectwidth, minimum height=0.9*\objectwidth, above=0.07*\objectwidth of legend, anchor=north] {};
% \draw[correspondence] ($(legend)-(0.2*\objectwidth,0.6em)$) -- ++(0.4*\objectwidth,0);
% \node[consistency related element, below=1em of legend, text width=1*\objectwidth, anchor=north, align=center, font=\footnotesize] (correspondence_text) {corresponding elements \\ (not forming a valid witness structure)};
% %\node[draw, fit=(legend)(correspondence_text)] {};

\end{tikzpicture}
    %\includegraphics[width=0.85\textwidth]{figures/correctness/compatibility/intuitive_incompatibility.png}
    \caption[Example for an intuitive notion of incompatibility]{Elements required by the consistency relations $\consistencyrelation{CR}{PE}$, $\consistencyrelation{CR}{PR}$ and $\consistencyrelation{CR}{ER}'$ (as well as their transposed) in \autoref{fig:compatibility:three_persons_example_extended} for a resident with the name \enquote{Alice Avid}. Solid blue lines connect corresponding elements, which do not form a valid witness structure.}
    \label{fig:compatibility:intuitive_incompatibility}
\end{figure}

\mnote{Incompatibilities affect consistency preservation}
Such incompatibilities are unwanted, as they indicate that developers have different, contradictory notions of consistency.
Additionally, they can easily result in transformations that do not yield consistent models or for which finding an orchestration that yields consistent models becomes unnecessarily difficult.
For that reason, we first discuss scenarios to identify an intuitive notion of \emph{compatibility}, which we then transfer into a formal notion.
Afterwards, we develop a formal, inductive approach to prove compatibility of relations, for which we prove correctness.
It is based on the insight that consistency relations forming a specific kind of tree structure are compatible and that removing a specific kind of redundant relations preserves compatibility.
We then derive a practical approach for the transformation language \gls{QVTR}.
This chapter thus constitutes our contribution \autoref{contrib:correctness:compatibility}, which consists of four subordinate contributions: a discussion of compatibility notions; a formal definition of one such notion; a formal approach to prove compatibility; and finally a practical realization of that approach.
It answers the following research question:

\researchquestionrepeat{rq:correctness:compatibility}

\mnote{Provability of compatibility}
We will see that it is, in general, not possible to prove that transformations are incompatible if the language, in which the relations are described, is undecidable, such as \gls{QVTR}.
We can, however, at least conservatively validate compatibility of transformations.
Thus, if our approach proves compatibility, the transformations are actually compatible, but not vice versa.
This enables transformation developers to validate compatibility of their transformations on-the-fly during transformation development, if developed for a specific scenario, or a posteriori during their combination, according to the scenarios introduced in \autoref{chap:networks:specification_process}.
In particular, in the first scenario developers can immediately react to the introduction of incompatibilities during transformation development.

\mnote{Publication of contributions}
We have published central contributions of this chapter, in particular the formal and the practical approach for validating compatibility, in previous work~\owncite{klare2020compatibility-report}.
Parts of some sections of this chapter are also literally taken from that publication, which we further indicate in the respective sections.
The practical approach has been developed in the Master's thesis of \textowncite{pepin2019ma}, which was supervised by the author of this thesis.

%%%
%%% COMPATIBILITY NOTION
%%%
\section{Towards a Notion of Compatibility}

\mnote{Modular relations induce monolithic ones}
A set of binary consistency relations induces a monolithic, $n$-ary relation, also called \emph{global relation}, as discussed in \autoref{chap:correctness:notions_correctness:relations}.
A monolithic relation $\consistencyrelation{R}{}$ for metamodels $\metamodelsequence{M}{n}$ and pairwise consistency relations $\consistencyrelation{R}{i,j}$ is defined by:
\begin{align*}
    \consistencyrelation{R}{} = \setted{\tupled{\model{m}{1}, \dots, \model{m}{n}} \mid \bigwedge\limits_{1 \le i < j \le n} \tupled{\model{m}{i}, \model{m}{j}} \in \consistencyrelation{R}{i,j}}
\end{align*}
As discussed before, the consistency relations are correct by definition and so is the induced global relation, even if it is empty.
It is, however, unclear whether the relations are \enquote{reasonable} in combination.
%However, just because the induced monolithic relation does, for example, only contain one consistent tuple of models, this may not imply that it is not as intended.

\begin{figure}
    \centering
    \includegraphics[width=\textwidth]{figures/correctness/compatibility/empty_global_relation.png}
    \caption[Consistency relations that imply an empty global relation]{Example for consistency relations that imply an empty global relation}
    \label{fig:compatibility:empty_global_relation}
\end{figure}

\mnote{Empty induced global relations indicate incompatibility}
In fact, if the relations induce an empty global relation, these relations do actually not properly fit to each other, because no single tuple of models would be considered consistent, thus no system could be consistently described.
We would thus consider such relations incompatible.
\autoref{fig:compatibility:empty_global_relation} shows an extended version of the example already given in \autoref{chap:correctness:notions_correctness:relations}, inducing an empty global relation.
This is an abstraction of the concrete examples, which we have already discussed for our our running example, in which modified consistency relations lead to an empty set of consistent model tuples, due to conflicting conversions and concatenations of names between persons, residents and employees.

\mnote{Goal of identifying incompatible relations}
There may, however, be more cases than empty induced global relations that we want to exclude by considering the relations incompatible.
In general, the goal of finding incompatibilities and excluding them is twofold:
First, we may want to identify that different developers of modular relations have an incompatible notion of consistent, such that the results would never be as expected.
This is what we have seen in the examples with the name relations.
We want to exclude these cases, because developers will not want to combine transformation based on relations that are contradicting.
Second, incompatibilities may lead to transformations not being able to find consistent models, so the orchestration would not be able to execute transformations in an order that achieves a consistent state.
If we, for example, encoded the relations from the running with the inverse concatenation of $firstname$ and $lastname$ ($R'_{PR}$) into transformations, each cycle in which the transformation are executed would produce one new person, employee, and resident, with swapped $firstname$ and $lastname$ and a comma appended to $lastname$.
In consequence, transformations would not be able to find a consistent state and, if not stopped preemptively, be executed endlessly.
Thus we also want to exclude such cases, because it can prevent a transformation network from termination.

%\begin{itemize}
    %\item We discussed that we are not interested in correctness of modular relations w.r.t. one monolithic relations, as that relations is usually unknown and thus correctness cannot be checked (cf ~\autoref{chap:correctness:notions:dimensions})
    %\item This would mean that modular relations are correct by construction
    %\item Although relations are, theoretically, correct by construction, we have seen that there are relations, for which we may say that they do not properly work together, i.e., they are \emph{incompatible}, as discussed in \autoref{chap:correctness:notions:relations}
    %\item There, we had three binary relations, which, taken together, do not impose an overall relation containing any consistent set of models, see \autoref{fig:compatibility:empty_global_relation}
    %\item We have already given a more concrete example for that case from our running example before.
    %\item It is obvious that we may not want to have an induced global relation that is empty, however, there may be more cases that we want to exclude.
    %%\item Exclude somehow defined \emph{incompatibilites} may be wanted for two reasons: First, we may want to identify that different developers of modular relations have an incompatible notion of consistency, such that the result would never be as expected (this is what we have seen for the names). Second, incompatibilites may lead to transformations not being able to find a consistent solution, so the orchestration would not be able to execute transformations to achieve a consistent state. To show that, we extend the relations example to a transformation example, showing that no consistent state can be reached, i.e., for an added person no consistent employees etc. can be found
%\end{itemize}


%%
%% OBSOLETE RELATIONS
%%
\subsection{Necessity of Obsolete Relation Elements}

\begin{figure}
    \centering
    \includegraphics[width=\textwidth]{figures/correctness/compatibility/obsolete_relations.png}
    \caption[Example for obsolete elements in consistency relations]{Example for \enquote{obsolete} models pairs in consistency relation $R_1$, which can never occur in a globally consistent set of models}
    \label{fig:compatibility:obsolete_relations}
\end{figure}

\mnote{Locally consistent model pairs which are never globally consistent are \enquote{obsolete}}
A first intuitive option to define incompatibility is the presence of model pairs in the consistency relations, for which no globally consistent model tuple containing them can be found.
This canonically covers the case, in which the modular relations induce an empty global relation, because for none of the models pairs in each relation a globally consistent model tuple containing them can be found.
An example for that case is depicted in \autoref{fig:compatibility:obsolete_relations}, in which the relation $\consistencyrelation{R}{1}$ contains the pairs $\tupled{\model{m}{1}, \model{m}{2}'}$ and $\tupled{\model{m}{1}', \model{m}{2}}$, for which neither $\model{m}{3}$ nor $\model{m}{3}'$ is consistent to both other consistency relations, as the induced global relation is $\consistencyrelation{R}{} = \setted{\tupled{\model{m}{1},\model{m}{2},\model{m}{3}},\tupled{\model{m}{1}',\model{m}{2}',\model{m}{3}'}}$.
Thus, these model pairs may be denoted \enquote{obsolete} as they cannot occur in any globally consistent model tuple.

\mnote{Forbidding obsolete model pairs breaks assumptions}
While this point of view may be reasonable for the consistency relations only, as we are finally only interested in results that are globally consistent, it induces problems to the process of achieving such a result by means of transformation networks.
In fact, transformation networks need to allow intermediate states of models, which are only locally consistent, although they can never occur in a globally consistent state.
This is necessary, because otherwise each transformation would have to consider which model pairs are not only locally consistent but can be globally consistent as well.
We, however, excluded that by assumption of independent development and modular reuse and let the orchestration of transformations kind of negotiate a consistent result.

\begin{figure}
    \centering
    \includegraphics[width=\textwidth]{figures/correctness/compatibility/obsolete_relations_scenario.jpg}
    \caption[Concrete scenario with obsolete relation elements]{Example for an \enquote{obsolete} models pair in consistency relations between \gls{PCM}, UML and Java: The empty Java method realization cannot be globally consistent to the UML class model that defines the method in the class interface, because it realizes a \gls{PCM} component, for which the consistency relation requires at least a default implementation.}
    \label{fig:compatibility:obsolete_relations_scenario}
\end{figure}

\mnote{Example for obsolete relation necessity}
Consider the following example, which is also exemplarily depicted in \autoref{fig:compatibility:obsolete_relations_scenario}:
A UML class model and Java code are considered consistent when the same classes and interfaces with the same (in Java potentially empty) methods are contained.
In fact, for each UML model a usually infinite number of consistent Java models exists, containing arbitrary implementations of the methods.
In addition, \gls{PCM} models and UML class models are consistent when components are realized as classes implementing the provided interfaces of the components and thus their methods.
Analogously, each component is represented by a Java class implementing the provided interface.
The consistency relation between \gls{PCM} and Java may, however, require that a method within a class that realizes a method of a provided interface of a component has at least some default implementation, be it logging or something more component-specific.
%The transformation between \gls{UML} and Java may initialize a method added to a class with an empty body or dummy return statement.
%The transformation between \gls{PCM} and Java may, however, initializes a method within a class that realizes a component with a default implementation, be it logging or something more component-specific.
If we would now consider model pairs that can never occur in globally consistent model tuples as incompatible and thus forbid them, a UML model could not be considered consistent to a Java model if any method in a class that realizes a component and that is defined by one of its interfaces is realized by a Java method with an empty body.
The transformation between UML and Java would thus not be allowed to create an empty Java method upon creation of a UML method.
This would, however, enforce the relation between UML and Java to encode information about components, which both breaks our assumption of independent development, as the UML-Java transformation developer would need to know about that, and of modular reuse, because the transformation is then tied to the scenario in which \gls{PCM} is used as well.

\mnote{Obsolete relations do not induce a proper notion of incompatibility}
In consequence of the given scenario and the according insight that transformations may need to produce transient states that are only locally consistent to ensure independence of the transformations and their reusability in different contexts, such obsolete consistency relations do not induce a proper notion of incompatibility.

%First option: Remove "obsolete" relation elements

% \begin{itemize}
%     \item A first option might be to say that there should not be model pairs in the relations for which no globally consistent set of models can be found (give an abstract example for that)
%     \item However, in consequence a transformation could not produce such models, which might be necessary as transient (intermediate) states: E.g. a UML and Java code model are consistent when the same classes and interface with the same (in Java potentially empty) methods are present. In fact, a UML model is consistent with all (usually infinite) Java models with the same classes and methods, but with arbitrary method implementation. PCM and UML are consistent when components are realized as classes implementing the provided interfaces and thus their methods. Analogously, each component is represented by a Java class implementing the provided interfaces, but having an implementation that has some default functionality (be it logging or something more component-specific). 
%     Forbidding relations elements that can never be globally consistent would forbid that the relation between UML and Java allows Java models with empty method implementations for UML classes representing components. This would, however, require the relation/transformation between UML and Java know about components, which should not be the case due to modularity and independent development. Additionally, even in the concrete setup in which PCM is used, if a method is added to an interface of a class in UML which is a provided interface of its component in UML, the transformations may first create the Java method with an empty body, then propagate the method to the PCM interface and then propagate the default implementation of that method from PCM to Java. In this case, the transient state with the empty method body in Java is passed. Forbidding that would require an appropriate orchestration, i.e., first propagating the information across PCM, which could be defined for this specific case but not automatically decided in general, or the UML-Java transformation would have to consider that the Java implementation may not be empty, which, as discussed, contradicts modularity.
%     \item In general, this case is reflected in \autoref{fig:compatibility:obsolete_relations}, which shows two pairs in $R_1$, which can never occur in a globally consistent model, because the implied global relation is .... and does not contain those model pairs
% \end{itemize}


%%
%% UNWANTED BEHAVIOR
%%
\subsection{Prevention from Finding Consistent Solutions}

% Second attempt: Derive incompatibility notion from unwanted behavior
\begin{figure}
    \centering
    \includegraphics[width=\textwidth]{figures/correctness/compatibility/unwanted_behavior.png}
    \caption[Example for the unwanted rejection of a user change]{Example for the rejection of a user change because of consistency relations containing model pairs that are never globally consistent}
    \label{fig:compatibility:unwanted_behavior}
\end{figure}

\mnote{Example for transformation execution that reverts a user change}
To identify a proper notion of incompatibility, we now consider an exemplary transformation scenario from which we can derive such a notion.
In the example depicted in \autoref{fig:compatibility:unwanted_behavior}, we start with the models $\model{m}{1}$, $\model{m}{2}$ and $\model{m}{3}$, which are consistent to all three consistency relations.
If a user performs a change of $\model{m}{2}$ to $\model{m}{2}'$, we may assume the following transformation executions:
The transformation for $\consistencyrelation{R}{2}$ changes $\model{m}{3}$ to $\model{m}{3}'$, the one for $\consistencyrelation{R}{3}$ changes $\model{m}{1}$ to $\model{m}{1}'$ and then the one for $\consistencyrelation{R}{1}$ changes $\model{m}{2}'$ back to $\model{m}{2}$, as that is the only model consistent to $\model{m}{1}'$.
Now the transformation for $\consistencyrelation{R}{2}$ changes $\model{m}{3}'$ back to $\model{m}{3}$ and finally the one or $\consistencyrelation{R}{1}$ restores $\model{m}{1}$.
As a result, the transformation network executed the transformations in a way such that the original models are returned, which are actually consistent but reject the user change.

\mnote{Unnecessarily bad local selections can prevent network from finding proper solution}
Apart from the three given models, only $\model{m}{1}''$, $\model{m}{2}'$ and $\model{m}{3}'$ are consistent.
Upon the user change of $\model{m}{2}$ to $\model{m}{2}'$, we would actually expect the transformations to find the latter triple of models as a consistent result, as otherwise, like in the above example, the original models are returned, which actually rejects the user change.
The problem results from the model $\model{m}{1}'$ being present in the consistency relations, but not being consistent in any globally consistent model tuple.
For each of the transformations the local selection of $\model{m}{1}'$ is fine, as there are models to which it is locally consistent according to each consistency relation on its own.

\mnote{Scenario is different from obsolete relation elements}
Note that this scenario is different from the case discussed for obsolete relation elements.
In the scenarios discussed for obsolete relation elements, each model in such an obsolete pair occurs in a globally consistent model tuple, but not both models in that pair together do.
For example, the Java class with an empty method body actually occurs in a globally consistent model tuple, but not together with the UML class model in which the method is defined in the class interface, although they are locally consistent.

\mnote{Models that do not occur in any globally consistent model tuple are problematic}
We have seen that it is problematic, when consistency relations define consistency of models that do not occur in any globally consistent model tuple.
This can easily lead to transformations that do find the expected solutions and unnecessarily reject user changes.
We did not define a requirement that user changes may not be reverted on purpose, as that behavior may also be expected to express that certain changes are not allowed to be made.
However, if there was a reasonable series of transformations that returns a consistent tuple of models with reflects the user changes, it should be preferred over one that reverts the user change.

% \begin{itemize}
%     \item Considering the example in \autoref{fig:compatibility:unwanted_behavior}, we start with $A_1, B_1, C_1$ and have a user change of $B_1$ to $B_2$. Apart from the input models, only $A_3, B_2, C_2$ are globally consistent. So we would expect the transformations to find the latter models. Because of also having consistency relations to $A_2$ and transformation that locally select $A_2$ (because they do not know that there is no globally consistent state with $A_2$), passing across that model can lead to resulting in $A_1, B_1, C_1$ again. We did not define that requirement yet because it is not essentially necessary, but usually user may not want their changes be undone by a transformation, at least if there was another set of consistent models reflecting their change.
%     \item Thus, we would call this behavior unwanted.
%     \item Note that this scenario is different from the described scenario with empty Java method bodies, because in that case there are models for which an empty Java method body occurs in a globally consistent states (those in which the method does not realize a method of a provided interface of a component). Thus if the corresponding class does not realize a component but an ordinary class, it would be fine that it is consistent to the UML class with an empty body. In contrast, in the case described here, there is no single set of models in which $A_2$ was consistent.
% \end{itemize



%%
%% COMPATIBILITY NOTION
%%
\subsection{An Informal Notion of Compatibility}

\mnote{Models that are never globally consistent are problematic}
The discussed case that models do not occur in any globally consistent model tuple can be seen as a special case of obsolete relation elements, because it actually means that for each pair in a consistency relation in which a model occurs, the model pair cannot occur in a globally consistent model.
We found that in a combination of relations a model is problematic if
\begin{enumerate}
    \item it is locally consistent to another model, i.e., it occurs in a pair of a binary consistency relation and
    \item it can never be globally consistent, i.e., it is not contained in any model tuple that is consistent to all consistency relations.
\end{enumerate}
The model $\model{m}{1}'$ in \autoref{fig:compatibility:unwanted_behavior} was such a model, as it was locally consistent to $\model{m}{2}$ and $\model{m}{3}'$, but those two are inconsistent.

\mnote{Problematic models can be introduced by users or transformations}
We can distinguish two cases that lead to occurence of such a model like $\model{m}{1}'$:
\begin{properdescription}
    \item[User:] The model was created by the user, thus adapting the model is unwanted as the user introduced it. Such a change should be rejected as the model cannot be globally consistent.
    \item[Transformation:] The model was created by a transformation. In our example, this can either be the case because $\model{m}{2}$ or $\model{m}{3}'$ was created. There is, however, at least $\model{m}{1}''$ to which $\model{m}{2}$ and $\model{m}{3}'$ are consistent, so the transformation should better select that one. If there was not such $\model{m}{1}''$, then $\model{m}{2}$ and $\model{m}{3}'$ would be in the same situation not occurring in any globally consistent model tuple, thus the argumentation could be applied inductively.
\end{properdescription}
In consequence, allowing such models during the process of describing a system and preserving consistency between the system models does not provide any benefits and thus should in the best case not occur.
There is no reason to create it, but it may prevent transformations from finding consistent states.
In fact, disallowing the adaptation of the user change is even more reasonable when not concerning the complete model, like proposed with authoritative models by \textcite{stevens2020BidirectionalTransformationLarge-SoSym}, but only the part considered by a specific rule that describes consistency, such as a rule specifying the relation between classes and components, rather than between \gls{PCM} and UML.
This does also fit the way in which transformation are usually defined, as they consist of rules that relate only limited sets of meta-classes and their properties.
This is why our coming formalization of compatibility and its analysis will base on a more fine-grained notion of consistency, i.e., relations that relate extracts of models rather than the complete models.

\mnote{A notion of compatibility}
From these insights, we formulate the following, for now informal and still on the level of models, notion of compatibilility:
\begin{quote}
    For every model occurring in a pair of a consistency relation, a globally consistent model tuple containing it must exist.
\end{quote}

% 'This notion of compatibility reflects the intuitive requirements that we implicitly discussed.
% 'First, if the relation imply an empty global consistency relation, they are not considered compatible.
% Second, ...
% Informal notion reflects intuitive requirements:
% \begin{itemize}
%     \item We intuitively started to think about incompatibility in terms of relations implying an empty global relation. This is directly covered by the defined notion, because it that case for no single model a globally consistent tuple is found, thus the relations would be considered incompatible
%     \item We had the example of incompatible name mappings: Having a model with a single person, there is no globally consistent tuple of models, because the (faulty) inversion of firstname and lastname in one the relations leads to the necessity of having the person with inverted firstname and lastname in the model as well. This conflicts the introduced informal notion of compatibility.
%     As teasered before, we will precisely define the notion in a more fine-grained way at the level of elements rather than complete models. 
% \end{itemize}

\mnote{Compatibility notion is reasonable for preservation process}
This notion is especially reasonable when we consider the process of preserving consistency after user changes.
If a user performs a change, we want to ensure that if consistency of the modified elements is restricted by a consistency relation, there should be at least one consistent tuple of models that reflects the user change.
If this is not the case, the transformations will not be able to produce a reasonable result, apart from reverting or adapting the user change.

\mnote{Compatibility notion does not reflect specific semantic contradictions}
Note that this notion of compatibility does only exclude combinations of relations according to the above made argumentation of being generally useless and potentially preventing transformations from finding consistency result.
This does, however, not exclude further useless or unintended combinations of relations, for which the semantics of the relations would have to be known and analyzed.
The already discussed example of the necessity to infinitely swap $firstname$ and $lastname$ and append a comma induced by the alternative consistency relation $\consistencyrelation{R}{PR}'$ in \autoref{fig:compatibility:three_persons_example_extended} let to the situation that no tuple of models can fulfill those constraints, thus the global induced consistency relation is empty.
If we, however, relax $\consistencyrelation{R}{PR}'$ such that only $firstname$ and $lastname$ are swapped, but no comma is appended, the relations can be fulfilled by models which contain each person twice, once with its proper name and once with swapped first and last name.
Although we might say that the relations are not intended that way, it is impossible for a generic approach to validate that without knowing about the semantics of the attributes $firstname$, $lastname$ and their combination in $name$.
In a different context, it may be desired that two attributes are concatenated in both orders, thus it is generally necessary to not disallow that case.

\mnote{Compatibility is context-dependent and not achievable by construction}
Obviously, the given notion of compatibility is a property of a set of consistency relations and not of a single consistency relation on its own.
We may also say that compatibility of a single relation is \emph{context-dependent}.
In consequence, that property can neither be analyzed nor systematically achieved for a single consistency relation.
We can, by definition, not provide a construction approach for consistency relations to be compatible in each context.
Compatibility can only be achieved by construction if all consistency relations to be used together are known and developed together, such that compatibility can be analyzed on-the-fly.

%Derived (informal) notion of compatibility:
%\begin{itemize}
    % \item Models are problematic if
    % \begin{itemize}
    %     \item They can be locally consistent (according to one relation) to another model
    %     \item They are never globally consistent (not contained in any set of models that is consistent)
    % \end{itemize}
    % \item $A_2$ was such a model only being locally consistent with $B_1$ and $C_2$, but those are inconsistent.
    % \item Case Analysis:
    % \begin{itemize}
    %     \item $A_2$ is created by a user: Adapting this model is unwanted, as the user introduced it
    %     \item $A_2$ is created by a transformation: Either created because $B_1$ or $C_2$ was introduced. However, there are other $A_X$ to which $B_1$ and $C_2$ are consistent (otherwise they would be in the same situation as $A_2$), so the transformations should better select such a different $A_X$, as $A_2$ will never be consistent
    % \end{itemize}
    %\item As we can see from the example, considering such models does not provide any benefits (because they do never occur in a globally consistent state), but they can easily prevent transformations from finding a consistent state.
    %\item We can derive the following informal notion of incompatibility: For each model in a relation, a globally consistent model tuple must exist
    %\item More precisely, we will define this notion at the level of model elements rather than complete models, as transformations are usually defined in terms of rules that affect only some of the elements of models, such as a limited set of meta-classes and their properties.
%\end{itemize}


% Kompatibilität wird insbesondere interessant, wenn man annimmt, dass eine Nutzeränderung nicht rückgängig gemacht werden können soll. Wenn wir sie dann feingranular zerlegen, ist immer nur ein Element einer Relation betroffen, sodass dann relevant ist, ob wir hierfür Modelle finden könnten, die konsistent sind (wir diskutieren später bzgl. Orchestrierung, dass die Transformationen sie dann tatsächlich auch finden):
% Relationen müssen korrekt sein, d.h. gegeben eine Nutzeränderung muss es überhaupt möglich sein eine konsistente Menge an Modellen zu finden. Wenn Transformationen etwas beliebigen tun dürfen geht das immer. Wir nehmen an, dass eine Nutzeränderung nicht rückgängig gemacht werden soll (bzw. wenn sie rückgängig gemacht werden würde eigentlich die Änderung invalide war, d.h. keine Konsistenz im Netzwerk hergestellt werden kann). Daher sind Relationen nur korrekt, wenn für fixierte Elemente, die durch eine Nutzeränderung entstehen können, eine Modellmenge abgeleitet werden kann, die bzgl. der Relationen konsistent ist. D.h. gegeben einige Elemente muss es eine Modellmenge geben, die in allen Relationen liegt und die diese Elemente enthält (-> Kompatibilitätsbegriff). Wir betrachten in Kapitel ?, wie man Kompatibilität präzise definieren und feststellen/garantieren kann.\\
% Resultat: Gegeben eine Änderung ist es möglich eine Transformation anzugeben, die aus der Änderung ein konsistentes Modell produziert.

% \todo{In general, an empty set of consistent models can be intended. Without knowing about "semantics" it is impossible to validate whether restrictions are intended or not. However, we want to identify unintended situations, so we may define situations that we actually do not want to support, because there is a high possibility that the situation is unintended.}


\subsection{An Analysis for Compatibility of Relations}

\mnote{Overview of compatibility analysis approach}
In the following sections, we define a formal notion of compatibility and derive a formal, as well as practical approach for analyzing, or more precisely, proving it.
To give an overview of that approach, we first briefly introduce the central idea based on the given informal notion of compatibility, which we first introduced in \owncite{klare2018docsym} and \owncite{klare2019icmt}.

\mnote{Tree are compatible, but in general not achievable}
We have seen that incompatibilities can arise whenever there are cycles in the graph induced by consistency relations.
This means that the same models are related across two paths of relations, which may be contradictory.
Thus, to avoid incompatibilities by construction, one could define a network of transformations and thus underlying consistency relations that does not contain any cycles.
This situation is given when the network forms a tree.
As we have already discussed, it is, however, in general not possible to define such a tree.
First, it contradicts our assumption of independent development, as transformations would need to be aligned such that the missing direct relations between metamodels are expressed across other paths.
Second, like we have seen in the running example in \autoref{fig:compatibility:three_persons_example_extended}, if three metamodels all share specific information only pairwise, there needs to be a cycle of transformations to keep that information consistent.

\mnote{Redundancy and independence are compatibility-preserving}
Even if we cannot always construct tree, we can use the insight that trees of transformations consist of inherently compatible consistency relations to analyze arbitrary topologies for compatibility.
This is based on two techniques:
\begin{properdescription}
    \item[Redundancy:] If a consistency relation is redundant in a network, i.e., the same model tuples are considered consistent with or without that specific relation we can, virtually, remove it without affecting compatibility of the relations. More precisely, $\consistencyrelation{R}{1}$ is redundant in $\setted{\consistencyrelation{R}{1}, \consistencyrelation{R}{2}, \consistencyrelation{R}{3}}$ if and only if a model tuple $\tupled{\model{m}{1}, \model{m}{2}, \model{m}{3}}$ is consistent to $\setted{\consistencyrelation{R}{1}, \consistencyrelation{R}{2}, \consistencyrelation{R}{3}}$ exact when it is consistent to $\setted{\consistencyrelation{R}{2}, \consistencyrelation{R}{3}}$.
    Inductively applying the virtual removal of redundant relations until the remaining network is a tree, which is inherently compatible, we inductively know that the network with the redundant relations is compatible as well.
    \item[Independence:] When we consider consistency relations at the level of elements rather than models, the independence of relations is a second compatibility-preserving property.
    For example, if consistency between components and classes between \gls{PCM}, UML and Java is expressed in one set of relations and consistency between different interface representations in another, they can be considered independent, because modifications in components and classes do never affect interfaces and vice versa.
    Proving compatibility for each independent set of consistency relations inductively proves compatibility of the union of all these sets.
\end{properdescription}
Finding independent subsets of relations and removing their redundancies until only trees remain proves compatibility.
We call this approach \emph{decomposition}, as we decompose the original relations into independent, essential relations, and we say that the resulting trees \emph{witness} compatibility.

\begin{figure}
    \centering
    \newcommand{\hdistance}{(22em+0.3*\difftoafiveimage)}
\newcommand{\vdistance}{1.5em}
\newcommand{\classwidth}{6em}

\begin{tikzpicture}

% Resident
\umlclassvarwidth{resident}{}{Resident\sameheight}{
name
}{\classwidth}

% Employee
\umlclassvarwidth[, right=\hdistance of resident.north, anchor=north]{employee}{}{Employee\sameheight}{
name
}{\classwidth}

% Location
\umlclassvarwidth[, below=\vdistance of resident.south, anchor=north]{location}{}{Location\sameheight}{
street
}{\classwidth}

% Address
\umlclassvarwidth[, below=\vdistance of employee.south, anchor=north]{address}{}{Address\sameheight}{
street
}{\classwidth}

% CONSISTENCY RELATIONS
\draw[directed consistency relation] (resident.east) -- node[pos=0, above right] {$e$} node[pos=0.5, below, align=center] {
    $\consistencyrelation{CR}{1} = \setted{ \tupled{r,e} \mid \mathvariable{e.name} = \mathvariable{r.name}}$
} node[pos=1, above left] {$r$} (employee.west);

\draw[directed consistency relation] (location.east) -- node[pos=0, above right] {$l$} node[pos=0.5, below, align=center] {
    $\consistencyrelation{CR}{2} = \setted{ \tupled{l,a} \mid \mathvariable{l.street} = \mathvariable{a.street}}$
} node[pos=1, above left] {$a$} (address.west);

\end{tikzpicture}
    \caption[Exemplary overview of compatibility analysis idea]{Example for the decomposition of independent and removal of redundant consistency relations for analyzing compatibility. Adapted from \owncite{klare2018docsym}.}
    \label{fig:compatibility:decomposition}
\end{figure}

\mnote{Example for compatibility analysis}
Although we did not yet define how relations at the level of elements can be formally expressed, \autoref{fig:compatibility:decomposition} sketches the ideas for proving compatibility based on the given informal notion.
We have consistency relations $\consistencyrelation{R}{1}, \consistencyrelation{R}{2}, \consistencyrelation{R}{3}$ between three metamodels.
We know that $\consistencyrelation{R}{3}$ can be separated into disjoint $\consistencyrelation{R}{4}$ and $\consistencyrelation{R}{5}$, i.e., the relations are independent, thus one relation may related components and classes and the other may relate different interface representations, as exemplarily explained before.
Additionally, we know that the combination of $\consistencyrelation{R}{1}$ and $\consistencyrelation{R}{2}$ is a subset of $\consistencyrelation{R}{4}$, thus $\consistencyrelation{R}{4}$ is redundant as models are only considered consistent if they are consistent to $\consistencyrelation{R}{1}$ and $\consistencyrelation{R}{2}$ anyway.
In other words, $\consistencyrelation{R}{1}$ and $\consistencyrelation{R}{2}$ is more restrictive regarding consistency than $\consistencyrelation{R}{4}$.
In consequence, we can virtually remove $\consistencyrelation{R}{4}$ and consider $\consistencyrelation{R}{1}$ and $\consistencyrelation{R}{2}$ independently from $\consistencyrelation{R}{5}$, inductively due to its independence from $\consistencyrelation{R}{4}$.
The result are two independent trees of relations, which are inherently compatible.
Since redundancy and independence are compatibility-preserving, this proves that the original relations are compatible.

% \begin{copiedFrom}{ICMT}
% %FORMERLY: \subsection{Contradiction-free Modularizations}

% Contradictions in binary consistency specifications cannot be avoided by design.
% %They can, in the best case, be found by analyzing a combination of modular specifications.
% However, the structure of the network of specifications influences how prone to mistakes it is~\cite{klare2018docsym}.
% Two extremes of networks are depicted in \autoref{fig:correctness:modularization_strategies}:
% One is to have a specification for each pair of model types, inducing a dense graph. % of specifications.
% This extreme is prone to contradictions, because all relations are redundantly specified across several paths.
% Another extreme is to define each relation, potentially indirectly, only once, so that only one path of consistency specifications exists between each pair of model types.
% This leads to a tree of specifications, which is inherently free of contradictions and %, as there is only one path of specifications between each pair of model types.
% %Thus, it 
% avoids modularization mistakes by design.
% %Nevertheless, developers must understand the complete network to know which relations are represented, which is contradictory to our assumption of independent development.
% %Nevertheless, it breaks our assumption that specifications are developed independently, because relations are expressed indirectly, so developers must understand the complete network to know which relations are represented.
% However, it requires that such a network structure exists at all, because between three model types there must always be one relation that can be expressed transitively across the other two (cf.~\cite{klare2018docsym}).
% For example, if $\mathit{CS}_{1,3}$ in \autoref{fig:correctness:mistakes_specification_levels} shall be omitted and transitively expressed across $\mathit{CS}_{1,2}$ and $\mathit{CS}_{2,3}$, it must hold that:
% % \begin{align*}
% %     & \forall M_1 \in \mathcal{M}_1 : \forall M_2 \in \mathcal{M}_2 : \forall M_3 \in \mathcal{M}_3 : \\
% %     & \hspace{1em} \mathit{CS}(M_1, M_2, M_3) \iff \mathit{CS}_{1,2}(M_1, M_2) \land \mathit{CS}_{2,3}(M_2, M_3)
% % \end{align*}
% \begin{align*}
%     & \forall M_1, M_2, M_3 : (M_1, M_2, M_3) \in \mathit{CS} \Leftrightarrow (M_1, M_2) \in \mathit{CS}_{1,2} \land (M_2, M_3) \in \mathit{CS}_{2,3}
% \end{align*}
% %If this transitive relation misses or is even unable to express certain direct constraints, inconsistent models would be identified as consistent. %\todoHeiko{Das transitive muss man wohl an einem Beispiel erklären, am besten Ref. zu Intro}

% \begin{figure}[tb]
%     \centering
%     \newcommand{\hmmdistance}{3em}
\newcommand{\vmmdistance}{2em}

\begin{tikzpicture}[
    mm/.style={draw, circle, fill=gray},
    consistency relation/.style={latex-latex,consistency related element},
    mininode/.style={inner sep=.25em}, 
    legend/.style={font=\small}
]

% fully connected graph

\node[mm] (full_left) {};
\node[mm, above right=\vmmdistance and \hmmdistance of full_left.center, anchor=center] (full_top) {};
\node[mm, below right=\vmmdistance and \hmmdistance of full_left.center, anchor=center] (full_bottom) {};
\node[mm, right=2*\hmmdistance of full_left.center, anchor=center] (full_right) {};
\node[mm, below left=\vmmdistance and \hmmdistance of full_left.center, anchor=center] (full_bottomleft) {};

\draw[consistency relation] (full_left) -- (full_top);
\draw[consistency relation] (full_left) -- (full_right);
\draw[consistency relation] (full_left) -- (full_bottom);
\draw[consistency relation] (full_top) -- (full_right);
\draw[consistency relation] (full_top) -- (full_bottom);
%\draw[consistency relation] (full_top) to[bend left=50] ([xshift=1em]full_right) to [bend left=50] (full_bottom);
\draw[consistency relation] (full_right) -- (full_bottom);
\draw[consistency relation] (full_left) -- (full_bottomleft);
\draw[consistency relation] (full_bottom) -- (full_bottomleft);
\draw[consistency relation] (full_top) to[bend right=30] (full_bottomleft);
\draw[consistency relation] (full_bottomleft) .. controls ++(7em, -1em) and ([yshift=-3em, xshift=-0.2em]full_right.south) .. (full_right);
%\draw[consistency relation] (full_bottomleft) to[bend right=35] ([xshift=0.6*\mmdistance]full_bottom) to[bend right=20] (full_right);


%\node[below left=0.4*\vmmdistance and 0.5*\hmmdistance of full_bottom.south, anchor=north] {(a)};


% tree graph

\node[mm, right=4*\hmmdistance of full_left.center, anchor=center] (tree_left) {};
\node[mm, above right=\vmmdistance and \hmmdistance of tree_left.center, anchor=center] (tree_top) {};
\node[mm, below right=\vmmdistance and \hmmdistance of tree_left.center, anchor=center] (tree_bottom) {};
\node[mm, right=2*\hmmdistance of tree_left.center, anchor=center] (tree_right) {};
\node[mm, below left=\vmmdistance and \hmmdistance of tree_left.center, anchor=center] (tree_bottomleft) {};

\draw[consistency relation] (tree_left) -- (tree_top);
\draw[consistency relation] (tree_left) -- (tree_bottom);
\draw[consistency relation] (tree_left) -- (tree_bottomleft);
\draw[consistency relation] (tree_top) -- (tree_right);

%\node[below left=0.4*\vmmdistance and 0.5*\hmmdistance of tree_bottom.south, anchor=north] {(b)};

\node[draw=darkgray, matrix, legend, nodes=mininode, below right=0em and 2*\hmmdistance of tree_top, anchor=north west, outer sep=0, inner sep=0.4em, column sep=0.4em, row sep=0.2em] (legend) {
    \node[mm, anchor=center] (legend_mm) {}; &
    \node[anchor=west] (legend_mm_text) {model type}; \\
    %
    \draw[consistency relation] (-0.7em,0) -- (0.7em,0); &
    \node[anchor=west, align=left] {consistency\\ specification}; \\
};

% \node[mm, below right=0.5em and 2.5*\hmmdistance of tree_top] (legend_mm) {};
% \node[right=0.2em of legend_mm] (legend_mm_text) {\smallerfont model type};
% \draw[consistency relation] ([yshift=-2em, xshift=-0.7em]legend_mm.center) -- ([yshift=-2em, xshift=0.7em]legend_mm.center);
% \node[below=2em of legend_mm_text.north west, anchor=north west, align=left] {\smallerfont consistency\\ \smallerfont specification};

\end{tikzpicture}
%     \caption{Extremes of Strategies for Modularizing Consistency Specifications}
% %    \todoHeiko{Ergänzen um Angaben Metamodelle und evtl. Relationsnamen. Angleichen zur Relationen-Grafik}
%     \label{fig:correctness:modularization_strategies}
% \end{figure}

% In consequence, if a network of transformation can be built that is a tree, mistakes at the modularization level are avoided by design.
% If a tree cannot be achieved, it is necessary to find and fix mistakes when transformations are combined to a network.
% In this case, the consistency specifications must be revised whenever non-termination or non-deterministic termination of consistency preservation is observed (see \autoref{fig:correctness:categorization}). %, or, if possible, analyzed for potential contradictions before execution~\cite{klare2018docsym}.
%This can potentially be supported by model checking techniques~\cite{klare2018docsym}.

% \subsection{Modularization Options}
% %ADD: We consider level 2 and 3, and assume level 1.
% %ADD: Transformations languages on level 2 or 3, but could also be on level 1
% \todoHeiko{ADD: Distinction on level 2 regarding one path / several paths -> redundancy or bottleneck -- Do that here?}
% \todoHeiko{Eigentlich brauchen wir der Argumentation nach diesen Abschnitt nicht. Er ist aber extrem wichtig. Irgendwo anders einbauen?}

% For modularizing consistency specifications, two extremes of approaches can be applied, as discussed in \cite{klare2018docsym} and depicted in \autoref{fig:modularization_strategies}. One of these extremes is to have each existing binary consistency relation represented in a consistency specification, which leads to a dense graph of specifications as shown in the left of \autoref{fig:modularization_strategies}. Another extreme is to define each consistency relation only across one path in the graph of consistency specifications, leading to a tree of consistency specifications as shown in the right of \autoref{fig:modularization_strategies}.

% \begin{figure}[tb]
%     \centering
%     \newcommand{\hmmdistance}{3em}
\newcommand{\vmmdistance}{2em}

\begin{tikzpicture}[
    mm/.style={draw, circle, fill=gray},
    consistency relation/.style={latex-latex,consistency related element},
    mininode/.style={inner sep=.25em}, 
    legend/.style={font=\small}
]

% fully connected graph

\node[mm] (full_left) {};
\node[mm, above right=\vmmdistance and \hmmdistance of full_left.center, anchor=center] (full_top) {};
\node[mm, below right=\vmmdistance and \hmmdistance of full_left.center, anchor=center] (full_bottom) {};
\node[mm, right=2*\hmmdistance of full_left.center, anchor=center] (full_right) {};
\node[mm, below left=\vmmdistance and \hmmdistance of full_left.center, anchor=center] (full_bottomleft) {};

\draw[consistency relation] (full_left) -- (full_top);
\draw[consistency relation] (full_left) -- (full_right);
\draw[consistency relation] (full_left) -- (full_bottom);
\draw[consistency relation] (full_top) -- (full_right);
\draw[consistency relation] (full_top) -- (full_bottom);
%\draw[consistency relation] (full_top) to[bend left=50] ([xshift=1em]full_right) to [bend left=50] (full_bottom);
\draw[consistency relation] (full_right) -- (full_bottom);
\draw[consistency relation] (full_left) -- (full_bottomleft);
\draw[consistency relation] (full_bottom) -- (full_bottomleft);
\draw[consistency relation] (full_top) to[bend right=30] (full_bottomleft);
\draw[consistency relation] (full_bottomleft) .. controls ++(7em, -1em) and ([yshift=-3em, xshift=-0.2em]full_right.south) .. (full_right);
%\draw[consistency relation] (full_bottomleft) to[bend right=35] ([xshift=0.6*\mmdistance]full_bottom) to[bend right=20] (full_right);


%\node[below left=0.4*\vmmdistance and 0.5*\hmmdistance of full_bottom.south, anchor=north] {(a)};


% tree graph

\node[mm, right=4*\hmmdistance of full_left.center, anchor=center] (tree_left) {};
\node[mm, above right=\vmmdistance and \hmmdistance of tree_left.center, anchor=center] (tree_top) {};
\node[mm, below right=\vmmdistance and \hmmdistance of tree_left.center, anchor=center] (tree_bottom) {};
\node[mm, right=2*\hmmdistance of tree_left.center, anchor=center] (tree_right) {};
\node[mm, below left=\vmmdistance and \hmmdistance of tree_left.center, anchor=center] (tree_bottomleft) {};

\draw[consistency relation] (tree_left) -- (tree_top);
\draw[consistency relation] (tree_left) -- (tree_bottom);
\draw[consistency relation] (tree_left) -- (tree_bottomleft);
\draw[consistency relation] (tree_top) -- (tree_right);

%\node[below left=0.4*\vmmdistance and 0.5*\hmmdistance of tree_bottom.south, anchor=north] {(b)};

\node[draw=darkgray, matrix, legend, nodes=mininode, below right=0em and 2*\hmmdistance of tree_top, anchor=north west, outer sep=0, inner sep=0.4em, column sep=0.4em, row sep=0.2em] (legend) {
    \node[mm, anchor=center] (legend_mm) {}; &
    \node[anchor=west] (legend_mm_text) {model type}; \\
    %
    \draw[consistency relation] (-0.7em,0) -- (0.7em,0); &
    \node[anchor=west, align=left] {consistency\\ specification}; \\
};

% \node[mm, below right=0.5em and 2.5*\hmmdistance of tree_top] (legend_mm) {};
% \node[right=0.2em of legend_mm] (legend_mm_text) {\smallerfont model type};
% \draw[consistency relation] ([yshift=-2em, xshift=-0.7em]legend_mm.center) -- ([yshift=-2em, xshift=0.7em]legend_mm.center);
% \node[below=2em of legend_mm_text.north west, anchor=north west, align=left] {\smallerfont consistency\\ \smallerfont specification};

\end{tikzpicture}
%     \caption{Strategies for Modularizing Consistency Specifications}
%     \todoHeiko{Ergänzen um Angaben Metamodelle und evtl. Relationsnamen. Angleichen zur Relationen-Grafik}
%     \label{fig:modularization_strategies}
% \end{figure}

% To discuss the implications of the different strategies, we consider the simplified example in \autoref{fig:relations_example}. The relation $R$ is defined for three metamodels $\mathcal{M}_1, \mathcal{M}_2$ and $\mathcal{M}_3$ on system level (Level 1). This relation can be modularized into three binary relations $R_1, R_2 and R_3$ on modularization level (Level 2).

% \begin{figure}[tb]
%     \centering
%     \includegraphics[angle=270, width=0.7\textwidth]{figures/relations_example.pdf}
%     \caption{Example Relations on System and Modularization Level}
%     \label{fig:relations_example}
% \end{figure}

% The first extreme, to specify every existing consistency relation, has the advantage that modularity is very high in the sense that any subset of the metamodels between which consistency is specified can be selected and kept consistent when instantiating them. In the example, it is possible to use only instances of $\mathcal{M}_1$ and $\mathcal{M}_2$ for example, omitting $\mathcal{M}_3$ without losing the specification of certain consistency relations. 
% The drawback of this approach is the high degree of redundancy, as each consistency relations is specified multiple times across different paths. Those redundant specifications can easily contradict and lead to mistakes.
% The cause for contradicting consistency specifications is that at least one of them is not conform to the underlying consistency relation of the system. In the example, $R_1$ would be erroneous if the following condition holds:
% \begin{align*}
%     \exists M_1 \in \mathcal{M}_1 : \exists M_2 \in \mathcal{M}_2 : \exists M_3 \in \mathcal{M}_3 : [R(M_1, M_2, M_3) \land \neg R_1(M_1, M_2)] \lor [\neg R(M_1, M_2, M_3] \land R_1(M_1, M_2)]
% \end{align*}
% Since the relation $R$ is usually not specified explicitly, but implicitly through the partial, binary consistency specifications, such mistakes lead to contradictions between different binary consistency specifications.

% The second extreme, if each consistency relations is only represented across one path of consistency specifications, no redundancies exist, avoiding potential contradictions and mistakes induced by that. Nevertheless, it requires to find a network structure that is capable of representing all consistency relations without having multiple paths between two metamodels in the graph of consistency specifications. This requires that of three metamodels there is always one representing the overlapping information of the two others. 
% Therefore it must be possible to remove one of the relations without losing the representation of certain consistency relations. In the example, $R_3$ can be omitted, so that only $R_1$ and $R_2$ are used, if the following condition holds:
% \begin{align*}
%     \forall M_1 \in \mathcal{M}_1 : \forall M_2 \in \mathcal{M}_2 : \forall M_3 \in \mathcal{M}_3 : R(M_1, M_2, M_3) \Leftrightarrow R_1(M_1, M_2) \land R_2(M_2, M_3)
% \end{align*}
% If a developer accidentally leaves one a binary relation although the condition above is not fulfilled, some global consistency relations are not covered any more, which can lead to inconsistencies.

% An essential drawback of the second approach is that the network structure prescribes which parts of the consistency relations a binary consistency specification has to contain. Therefore, the consistency specifications cannot be developed independently, as their contents depend on those of the others. 
% \todoHeiko{Das letzte stimmt eigentlich nicht, denn auch bei dem anderen Verfahren muss man eigentlich die anderen Transformationen kennen, damit man Widersprüche vermeidet}
% \todoHeiko{Was wollen wir mit diesem Abschnitt jetzt eigentlich genau sagen?}

% \todoHeiko{MÖGLICHERWEISE: Diese Abschnitt hinter die Issues stellen als Begründung für die Issues und deren Diskussion auf Level 2? Dann steht das ungefähr auf einer Ebene mit Level 3! Könnte man auch in eigenes Kapitel packen, dann werden Level 2 und Level 3 explizit behandelt, Level 1 schließen wir ja eh aus. Gefällt mir auf den ersten Blick VIEL BESSER.}

% \end{copiedFrom} % ICMT


% \begin{copiedFrom}{DocSym}

% Defining a tree of transformations to preserve consistency between multiple models is difficult, because of three metamodels there must always be one containing all overlapping information of the two others.
% It is easy to see that this is hardly applicable in practice.
% %Even if all metamodels basically have a refinement relation, there may be some additional information in more abstract metamodels not represented in the more fine-grained ones.
% Nevertheless, it may be sufficient to find subsets of independent consistency relations, of which each induces such a tree, instead of considering the whole set of relations between two metamodels.
% %For example, defining the consistency relations between \ac{ADL} components and classes and those between service functionality specifications in an \ac{ADL} and class methods may require a fully connected graph of consistency relations between \ac{ADL}, \ac{UML} and Java, but treating them independent of each other, each of them may be representable by a tree of relations.
% For example, consider the relations in \autoref{fig:correctness:decomposition}.
% \ref{fig:correctness:independent_relations:R3} represents a subset of the transitive relation $\mbox{\ref{fig:correctness:independent_relations:R1}} \concat \mbox{\ref{fig:correctness:independent_relations:R2}}$.
% In consequence, \ref{fig:correctness:independent_relations:R3} can be omitted, but the remaining consistency relations still form a graph.
% However, since \ref{fig:correctness:independent_relations:R4} is independent from the other relations, it can be treated separately from the others, resulting in two trees of independent consistency relations.

% \begin{figure}
%     \centering
%     \newcommand{\mmdistance}{5em}

\begin{tikzpicture}[
    mm/.style={draw, circle, fill=lightgray, inner sep=0.25em},
]

% graph

\node[mm] (graph_middle) {$\metamodel{M}{1}$};
\node[mm, below left=0.5*\mmdistance and sqrt(3)/2*\mmdistance of graph_middle.center, anchor=center] (graph_bottomleft) {$\metamodel{M}{2}$};
\node[mm, below right=0.5*\mmdistance and sqrt(3)/2*\mmdistance of graph_middle.center, anchor=center] (graph_bottomright) {$\metamodel{M}{3}$};

\draw[consistency relation] (graph_middle) -- node[pos=0.4, above left] {\mylabel{fig:correctness:independent_relations:R1}{$\consistencyrelation{R}{1}$}} (graph_bottomleft);
\draw[consistency relation] (graph_middle) -- node[pos=0.4, above right] {\mylabel{fig:correctness:independent_relations:R2}{$\consistencyrelation{R}{2}$}} (graph_bottomright);
\draw[consistency relation] (graph_bottomleft) -- node[below] {\mylabel{fig:correctness:independent_relations:R3}{$\consistencyrelation{R}{3}$} = \mylabel{fig:correctness:independent_relations:R4}{$\consistencyrelation{R}{4}$} $\cup$ \mylabel{fig:correctness:independent_relations:R5}{$\consistencyrelation{R}{5}$}} (graph_bottomright);

\node[below right=0.4*\mmdistance and \mmdistance of graph_bottomleft.center, anchor=north, align=center] 
{ 
$\consistencyrelation{R}{5} \cap (\consistencyrelation{R}{1} \concat \consistencyrelation{R}{2}) = \emptyset$\\
$\consistencyrelation{R}{5} \cap \consistencyrelation{R}{4} = \emptyset$\\[0.3em]
$\consistencyrelation{R}{1} \concat \consistencyrelation{R}{2} \subseteq \consistencyrelation{R}{4}$ %(R_1 \cup R_2)^*\setminus (R_1 \cup R_2)$\\
};


% trees

\node[mm, right=2.5*\mmdistance of graph_middle] (tree1_middle) {$\metamodel{M}{1}$};
\node[mm, below left=0.5*\mmdistance and sqrt(3)/2*\mmdistance of tree1_middle.center, anchor=center] (tree1_bottomleft) {$\metamodel{M}{2}$};
\node[mm, below right=0.5*\mmdistance and sqrt(3)/2*\mmdistance of tree1_middle.center, anchor=center] (tree1_bottomright) {$\metamodel{M}{3}$};

\draw[consistency relation] (tree1_middle) -- node[pos=0.4, above left] {$\consistencyrelation{R}{1}$} (tree1_bottomleft);
\draw[consistency relation] (tree1_middle) -- node[pos=0.4, above right] {$\consistencyrelation{R}{2}$} (tree1_bottomright);


\node[mm, below=0.9*\mmdistance of tree1_middle.center, anchor=center] (tree2_middle) {$\metamodel{M}{1}$};
\node[mm, below left=0.5*\mmdistance and sqrt(3)/2*\mmdistance of tree2_middle.center, anchor=center] (tree2_bottomleft) {$\metamodel{M}{2}$};
\node[mm, below right=0.5*\mmdistance and sqrt(3)/2*\mmdistance of tree2_middle.center, anchor=center] (tree2_bottomright) {$\metamodel{M}{3}$};

\draw[consistency relation] (tree2_bottomleft) -- node[below] {$\consistencyrelation{R}{5}$} (tree2_bottomright);


% combinators

\node[above=0.4*\mmdistance of tree2_middle.center, anchor=center] {\Large\textbf{+}};

\draw[-latex] ([xshift=0.2*\mmdistance]graph_bottomright.east) --
    ([xshift=-0.2*\mmdistance]tree1_bottomleft.west);


\end{tikzpicture}
%     \caption{Exemplary decomposition of consistency relations}
%     \label{fig:correctness:decomposition}
% \end{figure}

% In our thesis, we will therefore analyze different case studies for the existence of such independent consistency relations, which allows to decompose them to independent trees.
% We will develop an appropriate representation of consistency relations, which is more sophisticated than the set notation used in this paper, and an analysis for independence of those relations.
% We will then investigate how that can be integrated into existing, declarative transformation languages.
% In consequence of that, we improve the \emph{applicability} of a tree-based specification of transformations.

% \todo{Hier fehlt noch was}
% %The notion of such relations to be sets of element tuples that is used here here is only used for illustration, as is hard to reduce actual relations of \metaclasses and their elements to such tuple sets.
% \todo{Erik: Ich finde, daß bis hier im Text noch zu wenig klar wird, daß es hier um Konsistenzrelationen geht. Man könnte auch den Eindruck bekommen, es geht generell um Transformations-Strukturen, die z.B. weder bidirektional noch inkrementell sein müssen. Natürlich haben wir auch wieder das Grundproblem, daß wir \emph{Konsistenz} nicht sauber definiert haben, aber das Faß will man hier vielleicht nicht aufmachen \ldots}

% \end{copiedFrom} % DocSym


\section{A Formal Notion of Compatibility}
\label{chap:compatibility:formal_notion}

%\begin{copiedFrom}{SoSym MPM4CPS}

\mnote{Formal notion of fine-grained consistency and compatibility}
In this section, we precisely define our yet informally introduced notion of \emph{compatibility}.
%We first discuss properties of transformation networks %in comparison to a single transformation 
%with an intuitive notion of compatibility, based on considerations in existing work.
For that, we use the fine-grained notion of consistency and defining relations as proposed in \autoref{chap:correctness:finegrained}.
We discuss implicit relations, which are induced by a set of consistency relations, such as transitive relations, and,
finally, derive a compatibility notion from the consistency formalization and its pursued perception.
%To distinguish the previously given coarse-grained and the coming fine-grained definition of consistency and consistency relations, we used the \enquote{model-level} prefix in the coarse-grained notion.
The contents of this and the remaining sections of this chapter are mostly, even literally, taken from our published article on proving compatibility~\owncite{klare2020compatibility-report}.

%This serves as our contribution \ref{contrib:formalization}.
%An introduction to properties of transformation networks and especially compatibility of transformations, what compatibility means and how the network influences the possibility to give guarantees regarding compatibility.

%\todoHeiko{Discuss extensional definition of relations / constraints and its relation to intensional definitions}
% \subsection{Properties of Transformation Network}
% \label{sec:compatibility:networkproperties}

% %% Correctness as a property of single, bidirectional transformations
% Keeping pairs of models consistent by means of incremental, bidirectional transformations has been well researched in recent years~\cite{stevens2010sosym, etzlstorfer2013a, cleve2019dagstuhl}.
% A bidirectional transformation consists of a \emph{relation} that specifies which pairs of models are considered consistent and a pair of directional transformations, denoted as \emph{consistency repair routines}, that take one modified and one originally consistent model and deliver a new model that is consistent to the modified one~\cite{stevens2010sosym}.
% Several well-defined properties of such transformations have been identified.
% The essential \emph{correctness} property states that a consistency repair routine delivers a result such that the models are actually consistent according to the defined relation~\cite{stevens2010sosym}.
% Another important property is \emph{hippocraticness}, which states that a consistency repair routine returns the input model if it was already consistent to the modified one~\cite{stevens2010sosym}.

%%
%% Correctness property of transformation networks, not induced by correctness of single transformations
%%
% When we combine several transformations to a network to achieve consistency between multiple models, those properties of the single transformations are still relevant, as each transformations on its own has to be at least correct to work properly in a network of transformations.
% However, correctness of the single transformations does not induce correctness of the transformation network.
% Taking an arbitrary set of correct transformations and executing them one after another does not necessarily constitute a terminating approach that delivers a result, in which all models are consistent according to the relations of the transformations, because the result of one transformation may violate the relation of another.
% It is possible that the approach does either not terminate, because there is a divergence or alternation in values changed or elements created, or terminates in a state that is not consistent regarding the relations of all transformations~\cite{klare2019icmt}.
% %Consider the example of $R_{PE}, R_{PR}, R'_{ER}$ in \autoref{fig:motivational_example}. A transformation network in which each transformation adds the corresponding element for an added element, such as a resident for an added person by the transformation of $R_{PR}$.
% \todoDiss{Readd}
% %Since the execution of one transformation may lead to a violation of the relation of another transformation, executing each transformation only once can easily lead to termination in an inconsistent state.
% %In consequence, transformations have to be executed in a fixed point iteration manner, until all relations are fulfilled.
% The property of a network to always result in a state in which the models are consistent to all relations of the transformations if they are executed in a specific order, can be seen as a \emph{correctness} property for transformation networks.
% In this work, we focus on that correctness property and do not discuss further quality properties of transformation networks, such as \emph{modularity}, \emph{evolvability} and \emph{comprehensibility}~\cite{klare2018docsym}.


%%
%% Correctness considerable at different levels.
%%
% A single transformation can only be incorrect in terms of its repair routines, because there are no restrictions regarding its relation that may prevent the repair routine from being able to produce a correct result.
% In previous work, however, we identified that transformation networks can be incorrect at different levels~\cite{klare2019icmt}.
% Networks can also be incorrect at the level of relations rather than repair routines, because multiple relations can be contradictory, i.e., they can relate elements in different ways such that the relations cannot be fulfilled at the same time.
% \todoDiss{Readd}
%This especially concerns the \emph{modularization level}, which considers correctness of the binary relations of a set of bidirectional transformations describing when a set of models is consistent, and the \emph{operationalization level}, which considers correctness of the consistency repair routines in terms of producing models that are correct regarding the relations of all bidirectional transformations.
%For a single transformation, there are no restrictions regarding its relation that may prevent the repair routines from being able to produce a correct result, thus correctness is only considered at the operationalization level. %The correctness definition of \textcite{stevens2010sosym} for a single transformation only concerns the operationalization level, because a single relation is correct by construction. 
% Considering only a single transformation, its relation is correct by definition, at least if there is no external specification against which the relations have to be correct, thus the definition of \textcite{stevens2010sosym} for correctness only considers the consistency repair routines.
%However, in case of a transformation network, the different relations may be contradictory, i.e., elements are related in a different way such that the relations cannot be fulfilled at the same time.
% In such a case, the consistency repair routines cannot be result in a state that is consistent according to the relations anymore, thus they may not terminate anymore.
%This is due to the fact that several relations can relate the same elements in a different way, such that they cannot be fulfilled at the same time, i.e. that they are contradictory.
%In such a case, the iteration will not terminate.
% We call the relations of such transformations \emph{incompatible}.
%We define this informal notion of compatibility more precisely in the remainder of this section.

%%
%% Compatibility not achievable by construction -> we focus on checking it
%%
% In consequence, compatibility of relations is a necessary prerequisite for consistency repair routines to produce correct results in transformation networks.
% We also found that correctness of the consistency repair routines can already be achieved by construction, whereas compatibility of the relations cannot be achieved by construction but in the best case be checked for a set of relations.
% In this work, we focus on the possibility to check compatibility of the relations of a set of transformations.
% In the following, we therefore precisely define the notion of compatibility of relations, which excludes contradictions in relations that can prevent consistency repair routines from fulfilling the relation.

% \todo{Example!}

%%
%% Tree topology excludes compatibility
%%
%\subsection{Network Topology Impacts}
% Finally, the topology of a transformation network directly influences how prone it is to incompatibilities of its relations.
% Contradictions of consistency relations, as exemplified with the relations $R_{PE}, R_{PR}, R'_{ER}$ in \autoref{fig:prologue:three_persons_example}, can only occur if the same classes are related to each other by different (sequences of) transformations in a different way.
% For example, in \autoref{fig:prologue:three_persons_example}, each combination of two relations puts the same classes into relation as the third one.
%This can also be a the case if a sequence of transformations introduces this relation, like each combination of two relations in \autoref{fig:motivational_example} relates the same classes as the third relation.
% This means that a transformation network, in which each pair of classes is only related by one sequence of transformations, cannot have contradictory relations and is thus inherently compatible.



% \subsection{Consistency by Transformation Networks}

% \begin{itemize}
%     \item Incremental, bidirectional transformations serve as a means for preserving consistency of models by updating one if the other is modified
%     \item Transformations specify the conditions for consistency (consistency relations), as well as how consistency can be restored after modifications (consistency repair)~\cite{stevens2010sosym}
%     \item Consistency between more than two models can be achieved by coupling bidirectional transformations to networks
%     \item Even if the single transformations are correct, i.e. consistency repair produces results that conform to the consistency relations~\cite{stevens2010sosym}, this may not be the case when \emph{independently developed} bidirectional transformations are combined.
%     \item Interoperability problems can occur at the level of consistency relations, i.e. that the relations of the different transformations cannot be fulfilled at the same time, or at the level of consistency repair, i.e. that the repair routines do not terminate in a consistent state or even not at all~\cite{klare2019icmt}
%     \item While \textcite{klare2019icmt} focused on techniques to avoid interoperability issues at the level of consistency repair, in this work we focus on finding interoperability issues of the underlying consistency relations, which we previously defined as \emph{compatibility} issues~\cite{klare2018docsym}.
%     The avoidance of interoperability issues at the relations level is a necessary assumption to avoid issues at the operationalization level.
% \end{itemize}


% \subsection{Network Topology Impacts}

% \begin{itemize}
%     \item Idea: Couple independently developed transformations (refer to running example) to network
%     \item Problem: Trade-off between different properties, depending on the chosen or induced network topology
%     \item Introduce contrary properties \emph{compatibility} and \emph{modularity} (provide only a naive explanation) and shortly refer to evolvability/comprehensibility \cite{klare2018docsym}.
%     \item Explain problems of building tree structures (inherent compatibility) and motivate arbitrary graph structures, providing high modularity, but not giving compatibility guarantees
% \end{itemize}

%\end{copiedFrom} % SoSym MPM4CPS

%%%%%%% MOVED FINE-GRAINED RELATION INTRODUCTION TO CORRECTNESS CHAPTER


%\begin{copiedFrom}{SoSym MPM4CPS}

\subsection{Implicit Consistency Relations}

\mnote{Concatenation of consistency relations}
Considering sets of consistency relations, as they are implicitly defined by the set of transformations in a transformation network, their combination is of especial interest.
Each set of consistency relations defines relations between two sets of classes, but also implies further \emph{transitive} consistency relations.
Having one relation between classes $\class{A}{}$ and $\class{B}{}$ and one between $\class{B}{}$ and $\class{C}{}$ implies an additional relation between $A$ and $C$.
We define a notion for the concatenation of relations that implies such transitive relations, which are supposed to reflect the consistency constraints introduced by the concatenated relations.
This especially means that models should always be consistent to a concatenation of consistency relations if they are consistent to each of the concatenated relations, as otherwise the concatenation would introduce additional consistency constraints.
To achieve this, the following definition makes appropriate restrictions to the derived consistency relation pairs.

\todoLater{Actually, a concatenation may also consider that two or more relations are concatenated to a single one. I.e. CR1 could map something to A and B, and CR2 could map A to something and CR3 could map B to something. Then there could be a combination of all of them. In fact, each pair of consistency relations between the same metamodels can be combined to "larger" relation that then may be concatenated to other relations. Such a pair could even be a pair of a relation with itself, like if a relation maps on element to two of the same class and another relation then maps one element of the class to another. 
In summary, our notion of transitivity has to consider that concatenation may not only be sequences, but acyclic graphs.}

\begin{definition}[Consistency Relations Concatenation] \label{def:relationconcatenation}
    Let $\consistencyrelation{CR}{1}, \consistencyrelation{CR}{2}$ be two consistency relations. Their concatenation $ \consistencyrelation{CR}{1} \concat \consistencyrelation{CR}{2}$ is defined as follows:
    \begin{align*}
        &
        %\consistencyrelation{CR}{} = 
        \consistencyrelation{CR}{1} \concat \consistencyrelation{CR}{2} \equalsperdefinition \setted{\tupled{\conditionelement{c}{l}, \conditionelement{c}{r}} \mid \\
        & \formulaskip 
        \exists %\consistencyrelationpair{cr}{1} = 
        \tupled{\conditionelement{c}{l}, \conditionelement{c}{r,1}} \in \consistencyrelation{CR}{1} : \exists %\consistencyrelationpair{cr}{2} = 
        \tupled{\conditionelement{c}{l,2}, \conditionelement{c}{r}} \in \consistencyrelation{CR}{2} : %\conditionelement{c}{l,1} = \conditionelement{c}{l} \land \conditionelement{c}{r,2} = \conditionelement{c}{r} \\
        %& \formulaskip\formulaskip
        %\land 
        \conditionelement{c}{l,2} \subseteq \conditionelement{c}{r,1}\\
        & \formulaskip
        \land \forall \tupled{\conditionelement{c}{l}, \conditionelement{c}{r,1}'} \in \consistencyrelation{CR}{1} : \exists \tupled{\conditionelement{c}{l,2}', \conditionelement{c}{r,2}'} \in \consistencyrelation{CR}{2} : \conditionelement{c}{l,2}' \subseteq \conditionelement{c}{r,1}'
        }
    \end{align*}
    with $\classtuple{C}{l,\consistencyrelation{CR}{}} = \classtuple{C}{l,\consistencyrelation{CR}{1}}$ and $\classtuple{C}{r,\consistencyrelation{CR}{}} = \classtuple{C}{r,\consistencyrelation{CR}{2}}$
\end{definition}

\mnote{Requirements for concatenation}
The concatenation of two consistency relations contains pairs of object tuples that are related across common elements in the right respectively left side of the consistency relation pairs.
Such a concatenation may be empty.
Two requirements ensure that all models considered consistent to the concatenation are also consistent to the single relations:
First, two consistency relation pairs of $\consistencyrelation{CR}{1}, \consistencyrelation{CR}{2}$ are only combined if the left condition element of the consistency relation pair from $\consistencyrelation{CR}{2}$ is a subset of the right condition element of the consistency relation pair $\tupled{\conditionelement{c}{l}, \conditionelement{c}{r,1}}$ of $\consistencyrelation{CR}{1}$.
Only in that case the existence of the right condition element of the pair of $\consistencyrelation{CR}{1}$ in a model requires the existence of an according condition element in $\consistencyrelation{CR}{2}$.
%For example, if $\consistencyrelation{CR}{1}$ requires for an element $a$ the elements $b$ and $c$ to exist, then $\consistencyrelation{CR}{2}$ must define a relation for a subset of $b$ and $c$, such that it transitively requires the existence of further elements.
Second, it is necessary that for all elements $\conditionelement{c}{r,1}'$ in the right side of $\consistencyrelation{CR}{1}$, which is considered consistent to a condition element $\conditionelement{c}{l}$, there must be a matching condition element, i.e. a subset of $\conditionelement{c}{r,1}'$, in the left condition of $\consistencyrelation{CR}{2}$.
If there was an element $\conditionelement{c}{r,1}'$ in the right side of $\consistencyrelation{CR}{1}$ for which the left side condition of $\consistencyrelation{CR}{2}$ does not contain a subset, the concatenation does not constraint consistency for the existence of $\conditionelement{c}{l}$.
Thus, without these two restrictions the occurrence of $\conditionelement{c}{l}$ in a model tuple would not necessarily impose any consistency requirement by $\consistencyrelation{CR}{2}$.
In the following, we explain these two requirements at an example.

% \begin{figure}
%     \centering
%     \includegraphics[width=\columnwidth]{figures/concatenation_subset.png}
%     \caption{Two consistency relations with $\consistencyrelation{CR}{1} \concat \consistencyrelation{CR}{2} = \emptyset$ and $\consistencyrelation{CR^T}{2} \concat \consistencyrelation{CR^T}{1} \neq \emptyset$}
%     \label{fig:concatenation_subset}
% \end{figure}

% \begin{figure}
%     \centering
%     \includegraphics[width=\columnwidth]{figures/combined_concatenation_example.png}    \caption{Consistency relations $\consistencyrelation{CR}{1}$ and options $\consistencyrelation{CR}{2}, \consistencyrelation{CR'}{2}, \consistencyrelation{CR''}{2}$ with $\consistencyrelation{CR}{1} \concat \consistencyrelation{CR}{2} = \neq \emptyset$, $\consistencyrelation{CR}{1} \concat \consistencyrelation{CR'}{2} = \emptyset$, $\consistencyrelation{CR}{1} \concat \consistencyrelation{CR''}{2} = \emptyset$ and $\consistencyrelation{CR''^T}{2} \concat \consistencyrelation{CR^T}{1} \neq \emptyset$}
%     \label{fig:concatenation_example}
% \end{figure}


\begin{figure}
    \centering
    \begin{subfigure}{\textwidth}
        \centering
        \newcommand{\hdistance}{11.3em}
\newcommand{\classwidth}{4.5em}
\newcommand{\internalvdistance}{1.7em}

\begin{tikzpicture}

% Person
\umlclassvarwidth{person}{}{Person\sameheight}{
name
}{\classwidth}

% Employee
\umlclassvarwidth[, right=\hdistance of person.north, anchor=north]{resident}{}{Resident\sameheight}{
name\\
street
}{\classwidth}

%Resident and Address
\umlclassvarwidth[, right=\hdistance of resident.north, anchor=north]{employee}{}{Employee\sameheight}{
name
}{\classwidth}

\umlclassvarwidth[, below=\internalvdistance of employee.south, anchor=north]{address}{}{Address\sameheight}{
street
}{\classwidth}

\umlassociationfromto{(employee) -- node[uml role end, pos=1, above left] {address} (address)}

% CONSISTENCY RELATIONS
\draw[directed consistency relation] (person.east) -- node[pos=0, above right] {$p$} node[pos=0.5, below] {$\consistencyrelation{CR}{1}$} node[pos=1, above left] {$r$} (person.east-|resident.west);
\draw[directed consistency relation] (employee.west-|resident.east) -- node[pos=0, above right] {$r$} node[pos=0.5, below, align=left] {$\consistencyrelation{CR}{2}$ / \\ $\consistencyrelation{CR}{2}'$} node[pos=1, above left] {$e$} (employee.west);
\draw[directed consistency relation] ($(employee.west)!0.2!(employee.west-|resident.east)$) |- node[pos=1, above left] {$a$} (address.west);

\node[consistency related element, below=5em of person.west, anchor=north west] {
$\begin{aligned}
    \consistencyrelation{CR}{1} =\; & \setted{\tupled{p,r} \mid \mathvariable{p.name} = \mathvariable{r.name}}\\[0.3em]
    \consistencyrelation{CR}{2} =\; & \setted{\tupled{r,(e,a)} \mid \mathvariable{r.name} = \mathvariable{e.name} \land \mathvariable{r.street} = \mathvariable{a.street}}\\
    \consistencyrelation{CR}{2}' =\; & \setted{\tupled{r,(e,a)} \mid \tupled{r,(e,a)} \in \consistencyrelation{CR}{2} \land \mathvariable{r.street} \neq \textnormal{\enquote{}}}
\end{aligned}$
};

\end{tikzpicture}
    \end{subfigure}

    \vspace{1em}
    \begin{subfigure}{\textwidth}
        \centering
        \newcommand{\hdistance}{11.3em}
\newcommand{\classwidth}{4.5em}
\newcommand{\internalvdistance}{1.7em}

\begin{tikzpicture}

% Person
\umlclassvarwidth{person}{}{Person\sameheight}{
name
}{\classwidth}

% Employee and Location
\umlclassvarwidth[, right=\hdistance of person.north, anchor=north]{resident}{}{Resident\sameheight}{
name
}{\classwidth}

\umlclassvarwidth[, below=\internalvdistance of resident.south, anchor=north]{location}{}{Location\sameheight}{
street
}{\classwidth}

\umlassociationfromto{(resident) -- node[uml role end, pos=1, above left] {address} (location)}

%Resident and Address
\umlclassvarwidth[, right=\hdistance of resident.north, anchor=north]{employee}{}{Employee\sameheight}{
name
}{\classwidth}

\umlclassvarwidth[, below=\internalvdistance of employee.south, anchor=north]{address}{}{Address\sameheight}{
street
}{\classwidth}

\umlassociationfromto{(employee) -- node[uml role end, pos=1, above left] {address} (address)}

% CONSISTENCY RELATIONS
\draw[consistency relation] (person.east) -- node[pos=0, above right] {$p$} node[pos=0.5, below] {$\consistencyrelation{CR}{3}$} node[pos=1, above left] {$r$} (person.east-|resident.west);
\draw[consistency relation] (resident.east) -- node[pos=0, above right] {$r$} node[pos=0.5, below, align=left] {$\consistencyrelation{CR}{4}$} node[pos=1, above left] {$e$} (employee.west);
\draw[consistency relation, -] ($(employee.west)!0.8!(employee.west-|resident.east)$) |- node[pos=1, above right] {$l$} (location.east);
\draw[consistency relation] ($(employee.west)!0.2!(employee.west-|resident.east)$) |- node[pos=1, above left] {$a$} (address.west);

\node[consistency related element, below=7em of person.west, anchor=north west] {
$\begin{aligned}
    \consistencyrelation{CR}{3} =\; & \setted{\tupled{p,r} \mid p.name = r.name}\\[0.3em]
    \consistencyrelation{CR}{4} =\; & \setted{\tupled{(r,l),(e,a)} \mid r.name = e.name \land l.street = a.street}
\end{aligned}$
};

\end{tikzpicture}
    \end{subfigure}
%    \includegraphics[width=\columnwidth]{figures/consistency_concatenation_example.png} 
    %\includegraphics[width=\columnwidth]{figures/concatenation_subset.png}
    \caption[Examples for consistency relation concatenation]{Two scenarios, each with two consistency relations: 
    Consistency relations $\consistencyrelation{CR}{1}$ and two options $\consistencyrelation{CR}{2}, \consistencyrelation{CR}{2}'$ with $\consistencyrelation{CR}{1} \concat \consistencyrelation{CR}{2} \neq \emptyset$ and $\consistencyrelation{CR}{1} \concat \consistencyrelation{CR}{2}' = \emptyset$, and consistency relations $\consistencyrelation{CR}{3}$ and $\consistencyrelation{CR}{4}$ with $\consistencyrelation{CR}{3} \concat \consistencyrelation{CR}{4} = \emptyset$ and $\consistencyrelation{CR}{4}^T \concat \consistencyrelation{CR}{3}^T \neq \emptyset$. Taken from \owncite{klare2020compatibility-report}.}
    \label{fig:compatibility:concatenation_example}
\end{figure}

\begin{example}
\autoref{fig:compatibility:concatenation_example} extends the initial example (\autoref{fig:compatibility:three_persons_example_extended} on page \pageref{fig:compatibility:three_persons_example_extended}) with further classes in the consistency relations, such that they do not only relate single classes to each other.
It defines an address for employees and, in the second example, also a location for the addresses of residents, which are represented in additional classes.
Both examples contain a consistency relation $\consistencyrelation{CR}{1}$ and $\consistencyrelation{CR}{3}$, respectively, between persons and residents, which define that for each person a resident with the same name has to exist.
The examples provide different options for consistency relation between residents (with locations) and employees with addresses ($\consistencyrelation{CR}{2}, \consistencyrelation{CR}{2}', \consistencyrelation{CR}{4}$), which exemplify the necessity for the restrictions in \autoref{def:relationconcatenation}:
\begin{enumerate}
    \item $\consistencyrelation{CR}{1} \concat \consistencyrelation{CR}{2}$: 
$\consistencyrelation{CR}{2}$ requires for each resident an employee with the same name and an address with an arbitrary street name.
In consequence, $\consistencyrelation{CR}{1} \concat \consistencyrelation{CR}{2}$ defines a relation for each person with an employee having the same name and all addresses with possible street names.
All models that are consistent to the concatenation are also consistent to the single relations.
    \item $\consistencyrelation{CR}{1} \concat \consistencyrelation{CR}{2}'$: 
$\consistencyrelation{CR}{2}'$ is similar to $\consistencyrelation{CR}{2}$ but additionally requires that the street of a resident must not be empty. 
In consequence, for a resident with an empty address it is not required that an employee exists.
This results in $\consistencyrelation{CR}{1} \concat \consistencyrelation{CR}{2}' = \emptyset$, because for any person there must not be an employee, as the person can be consistent to a resident with an empty street name.
This shows the necessity of the second restriction in the definition. 
    \item $\consistencyrelation{CR}{3} \concat \consistencyrelation{CR}{4}$: 
The concatenation $\consistencyrelation{CR}{3} \concat \consistencyrelation{CR}{4}$ is obviously empty, because $\consistencyrelation{CR}{3}$ requires a resident for each person, but $\consistencyrelation{CR}{4}$ only requires an employee if there is also a location.
Such a location does not necessarily exist if a person %and thus a resident
exists, thus if the models are consistent to $\consistencyrelation{CR}{3}$ and $\consistencyrelation{CR}{4}$ there must not necessarily be an employee for any contained person.
This shows the necessity for the first restriction in \autoref{def:relationconcatenation}, which would require a left condition element from $\consistencyrelation{CR}{4}$ (resident and location) to be a subset of a right condition element in $\consistencyrelation{CR}{3}$ (resident). %, which is never the case. % because of $\consistencyrelation{CR}{4}$ requiring more elements than $\consistencyrelation{CR}{3}$ ensures to exist or a person.
%This shows the necessity of the first restriction in the definition, which requires for all persons that for each consistent resident according to $\consistencyrelation{CR}{3}$, there is also a condition in $\consistencyrelation{CR}{4}$ that requires an employee to exist.
%However, generally speaking, in this case the left condition elements of the second relation are a subset of those of the right side of the first relation, which means that the first relation does never require all elements to exist that are necessary for the second relation to require existence of any further elements.
    \item $\consistencyrelation{CR}{4}^T \concat \consistencyrelation{CR}{3}^T$: 
The concatenation of the transposed relations $\consistencyrelation{CR}{4}^T \concat \consistencyrelation{CR}{3}^T$ is not empty, but actually contains all combinations of each possible employee with all addresses and relates them to a person with the same name.
This is reasonable, because $\consistencyrelation{CR}{4}^T$ requires for all existing employees and addresses that an appropriate resident with the same name %(and also a location)
has to exist, which then requires a person with that name to exist due to $\consistencyrelation{CR}{3}^T$.
The definition does only cover that case due to its first restriction, because $\conditionelement{c}{l,2}$, i.e., the resident related to a person by $\consistencyrelation{CR}{3}^T$ is a subset of $\conditionelement{c}{r,1}$, i.e., a tuple of resident and location.
\end{enumerate}
\end{example}

% The exemplary consistency relation show why it is necessary that the left condition element of the second consistency relation pair needs to be a subset of the right condition element of the first consistency relation pair when concatenating them.
% The concatenation $\consistencyrelation{CR}{1} \concat \consistencyrelation{CR}{2}$ is obviously empty, because for each resident a person is required, but for $\consistencyrelation{CR}{2}$ to require an employee, there must always be a location.
% Such an address does not necessarily exist if a resident and thus a person exists, thus if the models are consistent to $\consistencyrelation{CR}{1}$.
% In consequence, there must not always be an employee if a resident exists.
% The concatenation $\consistencyrelation{CR^T}{2} \concat \consistencyrelation{CR^T}{1}$ is not empty, but actually contains all combinations of each possible employee with all addresses and relates them to a resident with the same name.
% This is reasonable, because $\consistencyrelation{CR^T}{2}$ requires for all existing employees and addresses that an appropriate person with the same name (and also a location) have to exist, which then requires a resident to exist due to $\consistencyrelation{CR^T}{1}$.
% The definition does only cover that, because $\conditionelement{c}{l,2}$, i.e. the person related to a resident by $\consistencyrelation{CR^T}{1}$ is a subset of $\conditionelement{c}{r,1}$, i.e. a tuple of person and location.
 
% \begin{figure}
%     \centering
%     \includegraphics[width=\columnwidth]{figures/consistency_concatenation_example.png}
%     \caption{Two consistency relations with two alternatives for $\consistencyrelation{CR}{2}$ with $\consistencyrelation{CR}{1} \concat \consistencyrelation{CR}{2} \neq \emptyset$ and $\consistencyrelation{CR}{1} \concat \consistencyrelation{CR'}{2} = \emptyset$}
%     \label{fig:concatenation_example}
% \end{figure}

% \autoref{fig:concatenation_example} exemplifies the definition of concatenation and the necessity for its restrictions.
% Considering $\consistencyrelation{CR}{1} \concat \consistencyrelation{CR}{2}$, this concatenation relates all residents to employees with the same name and all addresses with any names. 
% So each resident is considered consistent to an employee with the same name and the existence of an address with any street name.
% This is reasonable, because $\consistencyrelation{CR}{1}$ requires a person with an arbitrary address to exist for each resident with the same name. Additionally $\consistencyrelation{CR}{2}$ requires an employee with the same name and an appropriate address to exist.
% For any person to which a resident is considered consistent, an appropriate employee and address exist, which are considered consistent, so the concatenation contains those elements, so $\consistencyrelation{CR}{1} \concat \consistencyrelation{CR}{2}$ considers all models consistent that are also consistent to the single relations.
% In contrast, $\consistencyrelation{CR'}{2}$ further restricts $\consistencyrelation{CR}{2}$ by requiring that the street name must not be empty. 
% In consequence, persons with an empty street name do not need to have an appropriate employee and address to be considered consistent.
% In consequence, the concatenation $\consistencyrelation{CR}{1} \concat \consistencyrelation{CR'}{2}$ is empty, because a resident does not necessarily require an employee to exist, because if a person with the same name and an empty street name exist, $\consistencyrelation{CR}{1}$ and $\consistencyrelation{CR'}{2}$ do not require an employee exists.
% This motivates the necessity for the last restriction in the definition of concatenation, which requires for all residents that for each consistent person according to $\consistencyrelation{CR}{1}$, there is also a condition in $\consistencyrelation{CR'}{2}$ that requires an employee to exist.

%Requiring that there must only be a partial overlap in the related elements, i.e. $\conditionelement{c}{r,1} \cap \conditionelement{c}{l,2} \neq \emptyset$ would lead to a combined consistency relation $\consistencyrelation{CR}{}$ that restricts consistency in comparison to the combined relations $\consistencyrelation{CR}{1}$ and $\consistencyrelation{CR}{2}$.

%If there is no overlap between two relations, i.e. they have no elements in common that they put into relation, then the concatenation of them is empty by definition.

%\todoLater{Maybe readd overlapping definition}
% To state when consistency relations are overlapping with each other in the sense that they potentially have elements in common, we define when we denote consistency relations as \emph{overlapping}.

% \begin{definition}[Overlapping Consistency Relations]
%     Let $\consistencyrelation{CR}{1}$ and $\consistencyrelation{CR}{2}$ be two consistency relations. We say that:
%     \begin{align*}
%         \formulaskip &
%         \consistencyrelation{CR}{1} \mathtext{is overlapping with} \consistencyrelation{CR}{2} \equivalentperdefinition \\
%         & \formulaskip
%         \exists \class{C}{} \in \classtuple{C}{l,\consistencyrelation{CR}{1}} : \exists \class{C'}{} \in \classtuple{C}{l,\consistencyrelation{CR}{2}} : \class{C}{} \cap \class{C'}{} \neq \emptyset \\
%         & \formulaskip 
%         \land \exists \class{C}{} \in \classtuple{C}{r,\consistencyrelation{CR}{1}} \exists \class{C'}{} \in \classtuple{C}{r,\consistencyrelation{CR}{2}} : \class{C}{} \cap \class{C'}{} \neq \emptyset
%     \end{align*}
% \end{definition}

% \todoHeiko{Overlapping is not needed right now}

% Consistency relation are considered \emph{overlapping}, if they relate classes that have an overlap in their properties in both sides of the relations. 
% \todoHeiko{Add an example for non-trivial overlap here!}

\mnote{Concatenation not restricting consistency}
We can formally show that the defined notion of concatenation does not lead to any restriction of consistency regarding the single relations:

\begin{lemma}[Concatenation Consistency] \label{lemma:concatenationimpliesconsistency}
    Let $\consistencyrelation{CR}{1}, \consistencyrelation{CR}{2}$ be two consistency relations and let $\consistencyrelation{CR}{} = \consistencyrelation{CR}{1} \concat \consistencyrelation{CR}{2}$ be their concatenation. For all model tuples $\modeltuple{m} \in \metamodeltupleinstanceset{M}$ the following statement holds:
    \begin{align*}
        &
        \modeltuple{m} \consistenttomath \setted{\consistencyrelation{CR}{1}, \consistencyrelation{CR}{2}} \Rightarrow \modeltuple{m} \consistenttomath \consistencyrelation{CR}{}
    \end{align*}
\end{lemma}

\begin{proof}
    For any tuple of models $\modeltuple{m}$ that is consistent to $\consistencyrelation{CR}{1}$ and $\consistencyrelation{CR}{2}$, take a witness structure $\consistencyrelation{W}{1}$ that witnesses consistency of $\modeltuple{m}$ to $\consistencyrelation{CR}{1}$ and $\consistencyrelation{W}{2}$ that witnesses consistency of $\modeltuple{m}$ to $\consistencyrelation{CR}{2}$.
    Now consider the composed witness structure $\consistencyrelation{W}{} = \consistencyrelation{W}{1} \concat \consistencyrelation{W}{2}$.
    We show that $\consistencyrelation{W}{}$ is a valid witness structure for $\consistencyrelation{CR}{}$.

    Let us assume there were $\tupled{\conditionelement{c}{l}, \conditionelement{c}{r}}, \tupled{\conditionelement{c}{l}', \conditionelement{c}{r}'} \in \consistencyrelation{W}{}$ with $\conditionelement{c}{l} = \conditionelement{c}{l}'$ and $\conditionelement{c}{r} \neq \conditionelement{c}{r}'$, such $\consistencyrelation{W}{}$ is not a witness structure for $\consistencyrelation{CR}{}$.
    Per definition, $\conditionelement{c}{l}$ only occurs in one $\tupled{\conditionelement{c}{l}, \conditionelement{c}{r,1}} \in \consistencyrelation{W}{1}$.
    So there must be two consistency relation pairs $\tupled{\conditionelement{c}{l,2}, \conditionelement{c}{r}}, \tupled{\conditionelement{c}{l,2}', \conditionelement{c}{r}'} \in \consistencyrelation{CR}{2}$ with $\conditionelement{c}{l,2} \subseteq \conditionelement{c}{r,1}$ and $\conditionelement{c}{l,2}' \subseteq \conditionelement{c}{r,1}$.
    However, since $\conditionelement{c}{l,2}$ and $\conditionelement{c}{l,2}'$ contain instances of the same classes and are both subsets of the same object tuple $\conditionelement{c}{r,1}$, we have $\conditionelement{c}{l,2} = \conditionelement{c}{l,2}'$.
    So we know that $\consistencyrelation{W}{}$ fulfills the first condition of a witness structure according to \autoref{def:consistency} for consistency:
    \begin{align*}
        &
        \forall \tupled{\conditionelement{c}{l,1}, \conditionelement{c}{r,1}}, \tupled{\conditionelement{c}{l,2}, \conditionelement{c}{r,2}} \in \consistencyrelation{W}{} : %\\
        %& \formulaskip
        \tupled{\conditionelement{c}{l,1}, \conditionelement{c}{r,1}} = \tupled{\conditionelement{c}{l,2}, \conditionelement{c}{r,2}} \lor \conditionelement{c}{l,1} \neq \conditionelement{c}{l,2} \land \conditionelement{c}{r,1} \neq \conditionelement{c}{r,2}
    \end{align*}
    Additionally, since $\consistencyrelation{W}{1}$ and $\consistencyrelation{W}{2}$ are witness structures for consistency of $\modeltuple{m}$ to $\consistencyrelation{CR}{1}$ and $\consistencyrelation{CR}{2}$, the model tuple contains all condition elements in $\consistencyrelation{W}{1}$ and $\consistencyrelation{W}{2}$.
    Consequentially, $\modeltuple{m}$ also contains the condition elements in $\consistencyrelation{W}{}$, as those in $\consistencyrelation{W}{}$ are composed of the ones in $\consistencyrelation{W}{1}$ and $\consistencyrelation{W}{2}$. This implies that the second condition of \autoref{def:consistency} is fulfilled:
    \begin{align*}
        &
        \forall \tupled{\conditionelement{c}{l}, \conditionelement{c}{r}} \in  \consistencyrelation{W}{} : \modeltuple{m} \containsmath \conditionelement{c}{l} \land \modeltuple{m} \containsmath \conditionelement{c}{r}
    \end{align*}
    Finally, let us assume that the third condition of \autoref{def:consistency} is not fulfilled, i.e.: 
    \begin{align*}
        &
        \exists \conditionelement{c}{l}' \in \condition{c}{l,\consistencyrelation{CR}{}} : \modeltuple{m} \containsmath \conditionelement{c}{l}' \land \conditionelement{c}{l}' \not\in \condition{c}{l,\consistencyrelation{W}{}}
    \end{align*}
    We know that $\condition{c}{l,\consistencyrelation{CR}{}} \subseteq \condition{c}{l,\consistencyrelation{CR}{1}}$, because the left condition elements in $\consistencyrelation{CR}{}$ are, per definition, taken from the left condition elements in $\consistencyrelation{CR}{1}$ and thus also contained in $\consistencyrelation{CR}{1}$.
    Since $\modeltuple{m} \containsmath \conditionelement{c}{l}'$, there must be a consistency relation pair $\tupled{\conditionelement{c}{l}', \conditionelement{c}{r,1}'} \in \consistencyrelation{W}{1}$, which witnesses consistency of $\conditionelement{c}{l}'$ according to $\consistencyrelation{CR}{1}$.
    There must be at least one consistency relation pair $\tupled{\conditionelement{c}{l,2}', \conditionelement{c}{r,2}'} \in \consistencyrelation{CR}{2}$ with $\conditionelement{c}{l,2}' \subseteq \conditionelement{c}{r,1}'$, because otherwise $\conditionelement{c}{l}'$ would per definition not occur in the left condition of $\consistencyrelation{CR}{}$.
    For all such tuples $\tupled{\conditionelement{c}{l,2}', \conditionelement{c}{r,2}'}$, we know that $\modeltuple{m} \containsmath \conditionelement{c}{l,2}'$, because $\modeltuple{m} \containsmath \conditionelement{c}{r,1}'$ due to its containment in $\consistencyrelation{W}{1}$ and due to $\conditionelement{c}{l,2}' \subseteq \conditionelement{c}{r,1}'$.
    In consequence, consistency to $\consistencyrelation{CR}{2}$ requires that for one of those $\conditionelement{c}{r,2}'$ it holds that $\modeltuple{m} \containsmath \conditionelement{c}{r,2}'$ and that there is $\tupled{\conditionelement{c}{l,2}', \conditionelement{c}{r,2}'} \in \consistencyrelation{W}{2}$ that witnesses this consistency.
    Summarizing, due to $\tupled{\conditionelement{c}{l}', \conditionelement{c}{r,1}'} \in \consistencyrelation{W}{1}$ and $\tupled{\conditionelement{c}{l,2}', \conditionelement{c}{r,2}'} \in \consistencyrelation{W}{2}$ with $\conditionelement{c}{l,2}' \subseteq \conditionelement{c}{r,1}'$ and due to the definition of $\consistencyrelation{W}{}$ as the concatenation of $\consistencyrelation{W}{1}$ and $\consistencyrelation{W}{2}$, we know that $\tupled{\conditionelement{c}{l}', \conditionelement{c}{r,2}'} \in \consistencyrelation{W}{}$, which breaks our assumption.
    So we have shown that:
    \begin{align*}
        &
        \forall \conditionelement{c}{l}' \in \condition{c}{l,\consistencyrelation{CR}{}} : \modeltuple{m} \containsmath \conditionelement{c}{l}' \Rightarrow \conditionelement{c}{l}' \in \condition{c}{l,\consistencyrelation{W}{}}
    \end{align*}
    Summarizing, we have shown that $\consistencyrelation{W}{}$ fulfills all three requirements for a witness structure according to \autoref{def:consistency} for $\modeltuple{m}$ being consistent to $\consistencyrelation{CR}{}$, so we know that $\modeltuple{m} \consistenttomath \consistencyrelation{CR}{}$.
    % If any set of models $\modelset{m}$ is inconsistent to $\consistencyrelation{CR}{}$, then
    % \begin{align*}
    %     \formulaskip &
    %     \exists \tuple{c}{l} \in \condition{c}{l, \consistencyrelation{CR'}{}} \mid \modelset{m} \mathtext{contains} \conditionelement{c}{l} : \\
    %     & \formulaskip %\label{eq:consistencytransitivenoncontainment}
    %     \forall \conditionelement{c}{r} \in \condition{c}{r, \consistencyrelation{CR'}{}} \mid \tupled{\conditionelement{c}{l}, \conditionelement{c}{r}} \in \consistencyrelation{CR}{} : \neg (\modelset{m} \mathtext{contains} \conditionelement{c}{r}) \\
    %     & \formulaskip %\label{eq:consistencytransitiveduplicatecontainment}
    %     \lor \exists \conditionelement{c'}{r} \in \condition{c}{r, \consistencyrelation{CR'}{}} \setminus \setted{\conditionelement{c}{r}} \mid \tupled{\conditionelement{c}{l}, \conditionelement{c'}{r}} \in \consistencyrelation{CR'}{}: \modelset{m} \mathtext{contains} \conditionelement{c'}{r}
    % \end{align*}
    % \begin{enumerate}
    %     \item Assume that there is no condition element $\conditionelement{c}{r}$, such that $\modelset{m} \containsmath \conditionelement{c}{r}$.
    %     Due to $\consistencyrelation{CR}{}$ being a concatenation of $\consistencyrelation{CR}{1}, \dots, \consistencyrelation{CR}{k}$, for every consistency relation pair $\tupled{\conditionelement{c}{l}, \conditionelement{c}{r}} \in \consistencyrelation{CR}{}$, there is a sequence of consistency relation pairs $\tupled{\conditionelement{c}{l,1}, \conditionelement{c}{r,1}} \in \consistencyrelation{CR}{1}, \dots, \tupled{\conditionelement{c}{l,k}, \conditionelement{c}{r,k}} \in \consistencyrelation{CR}{k}$, such that there is an overlap in the pairs of each sequential consistency relation pair, i.e. $\exists \object{o}{1} \in \conditionelement{c}{r,i}, \object{o}{2} \in \conditionelement{c}{l,i+1} : \object{o}{1} \cap \object{o}{2} \neq \emptyset$.
    %
    %
    %     \item Assume that there are at least two condition elements $\conditionelement{c}{r}, \conditionelement{c'}{r}$, such that $\modelset{m} \containsmath \conditionelement{c}{r}$ and $\modelset{m} \containsmath \conditionelement{c'}{r}$.
    %  
    % \end{enumerate}
    %
    %
    % If any set of models $\modelset{m}$ is consistent to $\setted{\consistencyrelation{CR}{1}, \dots, \consistencyrelation{CR}{k}}$, then
    % \begin{align*}
    %     \formulaskip &
    %     \forall \consistencyrelation{CR'}{} \in \setted{\consistencyrelation{CR}{1}, \dots, \consistencyrelation{CR}{k}} : \\
    %     & \formulaskip 
    %     \forall \tuple{c}{l} \in \condition{c}{l, \consistencyrelation{CR'}{}} \mid \modelset{m} \mathtext{contains} \conditionelement{c}{l} : \exists \conditionelement{c}{r} \in \condition{c}{r, \consistencyrelation{CR'}{}} \mid \tupled{\conditionelement{c}{l}, \conditionelement{c}{r}} \in \consistencyrelation{CR}{} : \\
    %     & \formulaskip\formulaskip 
    %     \modelset{m} \mathtext{contains} \conditionelement{c}{r} \\
    %     & \formulaskip\formulaskip
    %     \land \forall \conditionelement{c'}{r} \in \condition{c}{r, \consistencyrelation{CR'}{}} \setminus \setted{\conditionelement{c}{r}} \mid \tupled{\conditionelement{c}{l}, \conditionelement{c'}{r}} \in \consistencyrelation{CR'}{}: \neg \modelset{m} \mathtext{contains} \conditionelement{c'}{r}
    % \end{align*}
    % This especially holds for all $\conditionelement{c}{l} \in \condition{c}{l,\consistencyrelation{CR}{1}} \mid \modelset{m} \containsmath \conditionelement{c}{l}$.
    % Consider the respective condition elements $\conditionelement{c}{r}$, which $\modelset{m}$ contains as well. If there is a $\conditionelement{c'}{l} \in \condition{c}{l,\consistencyrelation{CR}{2}}$ with $\exists \object{o}{1} \in \conditionelement{c}{r} : \exists \object{o}{2} \in \conditionelement{c'}{l}$ such that $\object{o}{1}$
    % tba \todoHeiko{Add proof}
\end{proof}

% Having shown that our definition of consistency relation concatenation is well-defined in the sense that it does not introduce further restrictions for consistency, we are able to show that the transitive closure of a consistency relation set does also not restrict consistency in comparison to the set of consistency relations itself.


\subsection{Transitive Closure of Consistency Relations}

\mnote{Transitive closure of implicit relations}
Based on the introduced notion of concatenation, we can define a transitive closure for sets of consistency relations, which contains all relations in that set complemented by all possible concatenations of them, i.e., \emph{implicit relations} of that set.
Having shown that our definition of consistency relations concatenation is well-defined in the sense that it does not introduce further restrictions for consistency, we can show that the transitive closure does not restrict consistency in comparison to the set of consistency relation itself.

\begin{definition}[Consistency Relations Transitive Closure] \label{def:transitiveclosure}
    Let $\consistencyrelationset{CR}$ be a set of consistency relations.
    We define its transitive closure $\transitiveclosure{\consistencyrelationset{CR}}$ as:
    \begin{align*}
        \transitiveclosure{\consistencyrelationset{CR}{}} = \setted{\consistencyrelation{CR}{} \mid & \exists \consistencyrelation{CR}{1}, \dots, \consistencyrelation{CR}{k} \in \consistencyrelationset{CR}{} : %\\
        %&
        \consistencyrelation{CR}{} = \consistencyrelation{CR}{1} \concat \dots \concat \consistencyrelation{CR}{k} }
    \end{align*}
\end{definition}

\mnote{Direct and direct relations}
The transitive closure of a set of consistency relations $\consistencyrelationset{CR}$ contains all consistency relations of $\consistencyrelationset{CR}$ and all concatenations of relations in $\consistencyrelationset{CR}$. That means, the transitive closure contains consistency relations that relate all elements that are directly or indirectly related due to $\consistencyrelationset{CR}$.
Due to cycles in the concatenation of relations, this closure can, in general, be of infinite size.

\mnote{Multiple concatenation not restricting consistency}
The transitive closure of a consistency relation set does not further restrict consistency in comparison to the original set by construction of concatenation, i.e., if a model tuple is consistent to a set of consistency relations, it is also consistent to their transitive closure.
We show that in the following by first extending the argument of \autoref{lemma:concatenationimpliesconsistency}, which shows that concatenation does not further restrict consistency, to the transitive closure, which is only a set of concatenations of consistency relations.

\begin{lemma}[Relation Set Consistency]
    Let $\consistencyrelationset{CR}$ be a set of consistency relations for a tuple of metamodels $\metamodeltuple{M}$. Then:
    \begin{align*}
        &
        \forall \consistencyrelation{CR}{} \in \transitiveclosure{\consistencyrelationset{CR}{}} \setminus \consistencyrelationset{CR} :
        \exists \consistencyrelation{CR}{1}, \dots, \consistencyrelation{CR}{k} \in \consistencyrelationset{CR} : \forall \modeltuple{m} \in \metamodeltupleinstanceset{M} : \\
        & \formulaskip
        \modeltuple{m} \consistenttomath \setted{\consistencyrelation{CR}{1}, \dots \consistencyrelation{CR}{k}} \Rightarrow \modeltuple{m} \consistenttomath \consistencyrelation{CR}{} 
    \end{align*}
\end{lemma}

\begin{proof}
    Per definition, any $\consistencyrelation{CR}{} \in \transitiveclosure{\consistencyrelationset{CR}}$ is a concatenation of consistency relations in $\consistencyrelationset{CR}$, i.e.
    \begin{align*}
        &
        \forall \consistencyrelation{CR}{} \in \transitiveclosure{\consistencyrelationset{CR}} : \exists \consistencyrelation{CR}{1}, \dots, \consistencyrelation{CR}{k} \in \consistencyrelationset{CR} : %\\
        %& \formulaskip 
        \consistencyrelation{CR}{} = \consistencyrelation{CR}{1} \concat \dots \concat \consistencyrelation{CR}{k}
    \end{align*}
    We already know for any two consistency relations $\consistencyrelation{CR}{1}, \consistencyrelation{CR}{2}$ and all model tuples $\modeltuple{m}$ that if $\modeltuple{m} \consistenttomath \setted{\consistencyrelation{CR}{1}, \consistencyrelation{CR}{2}}$ then $\modeltuple{m} \consistenttomath \consistencyrelation{CR}{1} \concat \consistencyrelation{CR}{2}$ due to \autoref{lemma:concatenationimpliesconsistency}.
    Inductively applying that argument to $\consistencyrelation{CR}{1}, \dots, \consistencyrelation{CR}{k}$ shows that for all models $\modeltuple{m}$ with $\modeltuple{m} \consistenttomath \setted{\consistencyrelation{CR}{1}, \dots, \consistencyrelation{CR}{k}}$ we know that $\modeltuple{m} \consistenttomath \consistencyrelation{CR}{}$.
\end{proof}

\mnote{Transitive closure not restricting consistency}
As a direct result of the previous lemma, we can now show that the transitive closure of a consistency relation set considers the same tuples of models consistent as the consistency relation set itself.

\begin{lemma}[Transitive Closure Consistency] \label{lemma:consistencytransitiveclosure}
    Let $\consistencyrelationset{CR}$ be a consistency relation set for a metamodel tuple $\metamodeltuple{M}$.
    Then:
    \begin{align*}
        \forall \modeltuple{m} \in \metamodeltupleinstanceset{M}: \modeltuple{m} \consistenttomath \consistencyrelationset{CR} \equivalent
        \modeltuple{m} \consistenttomath \transitiveclosure{\consistencyrelationset{CR}}
    \end{align*}
\end{lemma}

\begin{proof}
    Adding a consistency relation to a set of consistency relations can never lead to a relaxation of consistency, i.e., models becoming consistent that were not considered consistent before. This is a direct consequence of \autoref{def:consistency} for consistency, which requires models be consistent to all consistency relations in a set to be considered consistent, thus only restricting the set of consistent model tuples by adding further consistency relations.
    In consequence, it holds that:
    \begin{align*}
        \modeltuple{m} \consistenttomath \transitiveclosure{\consistencyrelationset{CR}} \Rightarrow \modeltuple{m} \consistenttomath \consistencyrelationset{CR}
    \end{align*}
    According to \autoref{lemma:consistencytransitiveclosure}, a tuple of models that is consistent to $\consistencyrelationset{CR}$ is always consistent to all transitive relations in $\transitiveclosure{\consistencyrelationset{CR}}$ as well. Thus, we know that:
    \begin{align*}
        \modeltuple{m} \consistenttomath \consistencyrelationset{CR} \Rightarrow
        \modeltuple{m} \consistenttomath \transitiveclosure{\consistencyrelationset{CR}}
    \end{align*}
    In consequence, models are considered consistent equally for $\consistencyrelationset{CR}$ and its transitive closure $\transitiveclosure{\consistencyrelationset{CR}}$.
\end{proof}


\subsection{Compatibility of Consistency Relations}

\mnote{Formalization of compatibility}
Based on the notion of fine-grained consistency relations and their concatenation, we can know precisely formulate our initially informal notion of \emph{compatibility} of consistency relations.
We have stated that we consider consistency relation incompatible if they are somehow contradictory, like the relation between names in our initial example in \autoref{fig:compatibility:three_persons_example_extended}.
In that example, for residents with non-lowercase names no consistent tuple of models could be derived.
We formalize this notion of \emph{non-contradictory} relations by requiring that a relation may not restrict that an object tuple, for which consistency is defined in any consistency relation, cannot occur in a model tuple anymore.
%More precisely, we consider relations compatible if for all condition elements in the consistency relations, i.e., for every tuple of objects for which consistency is somehow constrained by requiring further elements to exist in a tuple of models to consider it consistent, a consistent model tuple containing those objects can be found. %In consequence, a consistency relation is not allowed to forbid existence of objects in consistency models for which other relations restrain consistency.

\begin{definition}[Compatibility] \label{def:compatibility}
    Let $\consistencyrelationset{CR}$ be a set of consistency relations for a tuple of metamodels $\metamodeltuple{M}$. % = \setted{\metamodel{M}{1}, \dots \metamodel{M}{k}}$.
    We say that:
    \begin{align*}
        &
        \consistencyrelationset{CR} \compatiblemath \equivalentperdefinition %\\
        %& \formulaskip
        \forall \consistencyrelation{CR}{} \in \consistencyrelationset{CR} : \forall \conditionelement{c}{} \in \condition{c}{l, \consistencyrelation{CR}{}} %\cup \condition{c}{r, \consistencyrelation{CR}{}} 
        : \exists \modeltuple{m} \in \metamodeltupleinstanceset{M} : \\
        & \formulaskip %\formulaskip
        \modeltuple{m} \containsmath \conditionelement{c}{} \land \modeltuple{m} \consistenttomath \consistencyrelationset{CR}
        % \forall \consistencyrelation{CR}{} \in \consistencyrelationset{CR} : \forall %\consistencyrelationpair{cr}{} = 
        % \tupled{\conditionelement{c}{l}, \conditionelement{c}{r}} \in \consistencyrelation{CR}{} : \exists \modelset{m} \in \metamodelinstances{\metamodelset{M}} : \\
        % & \formulaskip \formulaskip
        % \modelset{m} \mathtext{contains} \tupled{\conditionelement{c}{l}, \conditionelement{c}{r}} \land \modelset{m} \mathtext{consistent to} \consistencyrelationset{CR}
    \end{align*}
    We call a set of consistency relation $\consistencyrelationset{CR}$ \emph{incompatible} if it does not fulfill the definition of compatibility.
\end{definition}

%According to that definition, selecting any condition element (i.e. an object tuple) that occurs in the left side of a consistency relation pair, thus requiring another condition element to occur in a set of model to consider it consistent, it must be possible to derive a set of models that contains that object tuple and is considered consistent. 
%\autoref{def:compatibility} formalizes the notion of \emph{non-contradictory} relations by requiring that a relation may not restrict that an object tuple, for which consistency is defined in any consistency relation, cannot occur in a model tuple anymore.
\mnote{Compatibility at running example}
We exemplify this notion of compatibility at an extract of the initial example with different consistency relations.

% According to that definition, selecting any pair of object tuples from any consistency relation, it must be possible to derive a set of models that contains those tuples and is considered consistent. This formalizes the notion of \emph{non-contradictory} relations, as no relation restricts that an element of another relation cannot be fulfilled anymore.

\begin{figure}
    \centering
    \newcommand{\hdistance}{14em}
\newcommand{\classwidth}{6em}

\begin{tikzpicture}

% Person
\umlclassvarwidth{person}{}{Person\sameheight}{
firstname\\
lastname
}{\classwidth}

% Employee
\umlclassvarwidth[,above right=4em and \hdistance of person.center, anchor=south]{employee}{}{Employee\sameheight}{
name
}{\classwidth}

\umlclassvarwidth[,right=\hdistance of person.south, anchor=south]{resident}{}{Resident\sameheight}{
name
}{\classwidth}


% CONSISTENCY RELATIONS
\draw[directed consistency relation] (person.north) |- node[pos=0, above right] {$p$} node[pos=0.5, above right] {$\consistencyrelation{CR}{2}$ / $\consistencyrelation{CR}{2}'$ / $\consistencyrelation{CR}{2}''$} node[pos=1, below left] {$e$} (employee.west);
\draw[directed consistency relation] (employee.south) -- node[pos=0, below left] {$e$} node[right, align=left] {$\consistencyrelation{CR}{3}$  / $\consistencyrelation{CR}{3}'$} node[pos=1, above left] {$r$} (resident.north);
\draw[directed consistency relation] (resident.west-|person.east) -- node[pos=0, above right] {$p$} node[pos=0.5, below] {$\consistencyrelation{CR}{1}$} node[pos=1, above left] {$r$} (resident.west);

\node[consistency related element, below=1em of person.south west, anchor=north west, inner sep=0em] {
$\begin{aligned}
    &
    \consistencyrelation{CR}{1} = \setted{\tupled{p,r} \mid \mathvariable{r.name} = \mathvariable{p.firstname} + "\text{\textvisiblespace}" + \mathvariable{p.lastname}}\\[0.5em]
    &
    \consistencyrelation{CR}{2} = \setted{\tupled{p,e} \mid \mathvariable{e.name} = \mathvariable{p.firstname} + "\text{\textvisiblespace}" + \mathvariable{p.lastname}}\\
    &
    \consistencyrelation{CR}{2}' = \setted{\tupled{p,e} \mid \mathvariable{e.name} = \mathvariable{p.firstname} + ",\text{\textvisiblespace}" + \mathvariable{p.lastname}}\\
    &
    \consistencyrelation{CR}{2}'' = \setted{\tupled{p,e} \mid \mathvariable{e.name} = \mathvariable{p.lastname} + "\text{\textvisiblespace}" + \mathvariable{p.firstname}}\\[0.5em]
    &
    \consistencyrelation{CR}{3} = \setted{\tupled{r,e} \mid \mathvariable{r.name} = \mathvariable{e.name}}\\
    &
    \consistencyrelation{CR}{3}' = \setted{\tupled{r,e} \mid \mathvariable{r.name} = \mathvariable{e.name.toLower}}
\end{aligned}$
};

\end{tikzpicture}
    %\includegraphics[width=\columnwidth]{figures/incompatibility_example.png}
    \caption[Different incompatibility scenarios]{Three metamodels with different options of consistency relations. The sets $\setted{\consistencyrelation{CR}{1}, \consistencyrelation{CR}{1}^T,\consistencyrelation{CR}{2}, \consistencyrelation{CR}{2}^T, \consistencyrelation{CR}{3}, \consistencyrelation{CR}{3}^T}$ and $\setted{\consistencyrelation{CR}{1}, \consistencyrelation{CR}{1}^T, \consistencyrelation{CR}{2}'', \consistencyrelation{CR}{2}''^T, \consistencyrelation{CR}{3}, \consistencyrelation{CR}{3}^T}$ are compatible, whereas the sets $\setted{\consistencyrelation{CR}{1}, \consistencyrelation{CR}{1}^T, \consistencyrelation{CR}{2}', \consistencyrelation{CR}{2}'^T, \consistencyrelation{CR}{3}, \consistencyrelation{CR}{3}^T}$ and $\setted{\consistencyrelation{CR}{1}, \consistencyrelation{CR}{1}^T, \consistencyrelation{CR}{2}, \consistencyrelation{CR}{2}^T, \consistencyrelation{CR}{3}', \consistencyrelation{CR}{3}'^T}$ are not. Taken from \owncite{klare2020compatibility-report}.}
    \label{fig:compatibility:incompatibility_example}
\end{figure}

\begin{example}
\autoref{fig:compatibility:incompatibility_example} shows an extract of the three metamodels from \autoref{fig:compatibility:three_persons_example_extended} and several consistency relations, of which different combinations are compatible or incompatible according to the previous definition.
We always consider the actual relations together with their transposed ones to have a symmetric set of consistency relations.
% \begin{enumerate}
% \item $\setted{\consistencyrelation{CR}{1}, \consistencyrelation{CR^T}{1},\consistencyrelation{CR}{2}, \consistencyrelation{CR^T}{2}, \consistencyrelation{CR}{3}}$:
% These consistency relations are obviously compatible, because they relate $name$ and $firstname$ respectively $lastname$ in the same way. Thus, for any element model element with any name, a consistent model can be found by adding the instances of the other classes with equal names.

% \item $\setted{\consistencyrelation{CR}{1}, \consistencyrelation{CR^T}{1},\consistencyrelation{CR'}{2}, \consistencyrelation{CR'^T}{2}, \consistencyrelation{CR}{3}, \consistencyrelation{CR^T}{3}}$:
% These consistency relations are obviously not compatible, because for each person with $firstname$ and $lastname$, another person with $firstname,$ and $lastname$ has to exist due to the transitive relations requiring the addition of a comma. Thus, for each person an infinite number of further persons would have to exist to achieve a consistent set of models. However, models are assumed to be finite, so there is not such set of models and the relations are considered incompatible.

% \item $\setted{\consistencyrelation{CR}{1}, \consistencyrelation{CR^T}{1}, \consistencyrelation{CR'}{2}, \consistencyrelation{CR'^T}{2}, \consistencyrelation{CR}{3}, \consistencyrelation{CR^T}{3}}$:
% These consistency relations are compatible, although one might not expect that. The relations define that for a resident with $firstname = f$ and $lastname = l$ another resident with $firstname = l$ and $lastname = f$ has to exist, so that the set of models is consistent.
% Although that behavior may not be intuitive, it does not violate the definition of compatibility, because for any element of the relations, a consistent model can be constructed.
% In general, such a behavior cannot be forbidden, because comparable behavior might be expected, such as that for a software component an implementation class as well a utility class with different names are created due to different relations, which leads to comparable behavior as in the example.
% To detect such a problem, further semantics of properties would have to be considered, as it is necessary to know that a first name should never be mapped to a last name in our example.

% \item $\setted{\consistencyrelation{CR}{1}, \consistencyrelation{CR^T}{1}, \consistencyrelation{CR}{2}, \consistencyrelation{CR^T}{2}, \consistencyrelation{CR'}{3}, \consistencyrelation{CR'^T}{3}}$:
% These consistency relations reflect the ones of our motivational example in \autoref{fig:motivational_example}.
% According to the informal notion of incompatibility that we motivated in the introduction with that example, our formal definition of compatibility also considers these relations as incompatible, because it is not possible to create a resident with an uppercase name, so that the containing set of models is consistent.
% For a resident with $name = \mathtext{"A\textvisiblespace B"}$, a person with $firstname = \mathtext{"A"}$ and $lastname = \mathtext{"B"}$ has to exist, which requires existence of an employee with $name = \mathtext{"A\textvisiblespace B"}$. Now $\consistencyrelation{CR'}{3}$ requires a resident with $name = \mathtext{"a\textvisiblespace b"}$ to exist, which in turn requires a resident with $firstname = \mathtext{"a"}$ and $lastname = \mathtext{"b"}$ and an employee with $name = \mathtext{"a\textvisiblespace b"}$ to exist.
% In consequence, there are two employees, one with the uppercase and one with the lowercase name, for which a person with name the lowercase name has to exist according to the relation $\consistencyrelation{CR'}{3}$. So there is no witness structure with a unique mapping between the elements that is required to fulfill the consistency definition.
% \end{enumerate}
\begin{properdescription}
\item[$\setted{\consistencyrelation{CR}{1}, \consistencyrelation{CR}{1}^T,\consistencyrelation{CR}{2}, \consistencyrelation{CR}{2}^T, \consistencyrelation{CR}{3}, \consistencyrelation{CR}{3}^T}$:]
These relations are obviously compatible, because they relate $\mathvariable{firstname}$, $\mathvariable{lastname}$ and $\mathvariable{name}$ in the same way. Thus, for each object with any name, and thus any condition element in all of the consistency relations, a consistent model tuple can be found by adding instances of the other classes with appropriate names.

\item[$\setted{\consistencyrelation{CR}{1}, \consistencyrelation{CR}{1}^T,\consistencyrelation{CR}{2}', \consistencyrelation{CR}{2}'^T, \consistencyrelation{CR}{3}, \consistencyrelation{CR}{3}^T}$:]
These relations are obviously incompatible, because for each person $\mathvariable{p1}$ with $\mathvariable{p1.firstname}$ and $\mathvariable{p1.lastname}$, another person $\mathvariable{p2}$ with $\mathvariable{p2.firstname} = \mathvariable{p1.firstname} + \textnormal{\enquote{,}}$ and $\mathvariable{p2.lastname} = \mathvariable{p1.lastname}$ has to exist due to $\consistencyrelation{CR}{2}'$ and the transitive relations requiring the addition of a comma. Thus, each person would require an infinite number of further persons to exist in a consistent tuple of models. However, models are assumed to be finite, so there is no such model tuple and the relations are incompatible.

\item[$\setted{\consistencyrelation{CR}{1}, \consistencyrelation{CR}{1}^T, \consistencyrelation{CR}{2}', \consistencyrelation{CR}{2}'^T, \consistencyrelation{CR}{3}, \consistencyrelation{CR}{3}^T}$:]
These relations are compatible. The relations define that for a person $\mathvariable{p1}$ with $\mathvariable{p1.firstname}$ and $\mathvariable{p1.lastname}$ another person $\mathvariable{p2}$ with $\mathvariable{p2.firstname} = \mathvariable{p1.lastname}$ and $\mathvariable{p2.lastname} = \mathvariable{p1.firstname}$ has to exist, so that the tuple of models is consistent.
Although that behavior may not be intuitive, it does not violate the definition of compatibility, because for any object in the relations, a consistent model can be constructed.
In general, it can even be necessary that consistency relations require the same elements with swapped attribute values to exist, such that this behavior can and should not be forbidden.
Finally, such a relation would also not prevent a consistency preservation rule from finding a consistent tuple of models.
In consequence, such a behavior may be unwanted in specific situations due to the semantics of the specific domain, like in the example, but it can neither be detected automatically, not does it lead to problems when executing transformations.

\item[$\setted{\consistencyrelation{CR}{1}, \consistencyrelation{CR}{1}^T, \consistencyrelation{CR}{2}, \consistencyrelation{CR}{2}^T, \consistencyrelation{CR}{3}', \consistencyrelation{CR}{3}'^T}$:]
These consistency relations reflect the ones of our motivational example in \autoref{fig:compatibility:three_persons_example_extended}.
According to the informal notion of incompatibility that we have motivated in the introduction with that example, our formal definition of compatibility also considers these relations as incompatible, because it is not possible to create a resident with an uppercase name, such that the containing tuple of models is consistent.
For a resident with $\mathvariable{name} = \textnormal{\enquote{A\textvisiblespace B}}$, a person with $\mathvariable{firstname} = \textnormal{\enquote{A}}$ and $\mathvariable{lastname} = \textnormal{\enquote{B}}$ has to exist, which requires the existence of an employee with $\mathvariable{name} = \textnormal{\enquote{A\textvisiblespace B}}$. Now $\consistencyrelation{CR}{3}'$ requires a resident with $\mathvariable{name} = \textnormal{\enquote{a\textvisiblespace b}}$ to exist, which in turn requires a resident with $\mathvariable{firstname} = \textnormal{\enquote{a}}$ and $\mathvariable{lastname} = \textnormal{\enquote{b}}$ and an employee with $\mathvariable{name} = \textnormal{\enquote{a\textvisiblespace b}}$ to exist.
In consequence, there are two employees, one with the uppercase and one with the lowercase name, for which a resident with the lowercase name has to exist according to the relation $\consistencyrelation{CR}{3}'$. So there is no witness structure with a unique mapping between the elements that is required to fulfill \autoref{def:consistency} for consistency.
\end{properdescription}
\end{example}

\mnote{Goal of compatibility}
To summarize, compatibility is supposed to ensure that consistency relations do not impose restrictions on other relations such that their condition elements, for which consistency is defined, can never occur in consistent models.
The goal of ensuring compatibility of consistency relations is especially to prevent the execution of consistency preservation rules in transformation networks from non-termination, as may occur especially in the second scenario, in which an infinitely large model would be required to fulfill the consistency relations.

\mnote{Equivalence of transitive closure}
Finally, analogously to the equivalence of a set of consistency relations $\consistencyrelationset{CR}$ and its transitive closure $\transitiveclosure{\consistencyrelationset{CR}}$ in regards to consistency of a tuple of models, we can show that a set of consistency relations and its transitive closure are always equal with regards to compatibility.

\begin{lemma}[Transitive Closure Compatibility] \label{lemma:compatibilitytransitiveclosure}
    Let $\consistencyrelationset{CR}$ be a set of consistency relations for a tuple of metamodels $\metamodeltuple{M}$.
    It holds that:
    \begin{align*}
        \consistencyrelationset{CR} \compatiblemath \equivalent
        \transitiveclosure{\consistencyrelationset{CR}} \compatiblemath
    \end{align*}
\end{lemma}

\begin{proof}
    The reverse direction of the equivalence is given by definition, since compatibility of a set of consistency relations implies compatibility of any subset by definition.
    So we have to show the forward direction by considering the compatibility definition for all $\consistencyrelation{CR}{} \in \transitiveclosure{\consistencyrelationset{CR}}$.
    We partition $\transitiveclosure{\consistencyrelationset{CR}}$ into $\consistencyrelationset{CR}$ and $\transitiveclosure{\consistencyrelationset{CR}} \setminus \consistencyrelationset{CR}$ and consider their consistency relations independently.
    
    First, we consider $\consistencyrelation{CR}{} \in \transitiveclosure{\consistencyrelationset{CR}} \setminus \consistencyrelationset{CR}$.
    According to \autoref{def:transitiveclosure} for the transitive closure, each $\consistencyrelation{CR}{} \in \transitiveclosure{\consistencyrelationset{CR}} \setminus \consistencyrelationset{CR}$ is a concatenation of consistency relations $\consistencyrelation{CR}{1}, \dots, \consistencyrelation{CR}{k} \in \consistencyrelationset{CR}$.
    In consequence of that definition, we know that $\condition{c}{l,\consistencyrelation{CR}{}} \subseteq \condition{c}{l,\consistencyrelation{CR}{1}}$, so it is given that:
    \begin{align}
        & \nonumber \label{eq:transitiverelationcontainment}
        \forall \conditionelement{c}{l} \in \condition{c}{l,\consistencyrelation{CR}{}} : \exists \conditionelement{c}{l}' \in \condition{c}{l,\consistencyrelation{CR}{1}} : \forall \modeltuple{m} \in \metamodeltupleinstanceset{M} : \\ 
        & \formulaskip
        \modeltuple{m} \containsmath \conditionelement{c}{l} \Rightarrow \modeltuple{m} \containsmath \conditionelement{c}{l}'
    \end{align}
    Since $\consistencyrelationset{CR}$ is compatible, we especially know from \autoref{def:compatibility} for compatibility that:
    \begin{align}
        & \nonumber \label{eq:compatibilitysingleelement}
        \forall \conditionelement{c}{l}' \in \condition{c}{l, \consistencyrelation{CR}{1}} %\cup \condition{c}{r, \consistencyrelation{CR}{}} 
        : \exists \modeltuple{m} \in \metamodeltupleinstanceset{M} : \\
        & \formulaskip
        \modeltuple{m} \containsmath \conditionelement{c}{l}' \land \modeltuple{m} \consistenttomath \consistencyrelationset{CR}
    \end{align}
    Because of \autoref{eq:transitiverelationcontainment} and \autoref{eq:compatibilitysingleelement}, we know that:
    \begin{align}
        & \nonumber \label{eq:compatibilitysinglelementtransitive}
        \forall \conditionelement{c}{l} \in \condition{c}{l, \consistencyrelation{CR}{}} %\cup \condition{c}{r, \consistencyrelation{CR}{}} 
        : \exists \modeltuple{m} \in \metamodeltupleinstanceset{M} : \\
        & \formulaskip
        \modeltuple{m} \containsmath \conditionelement{c}{l} \land \modeltuple{m} \consistenttomath \consistencyrelationset{CR}
    \end{align}  
    %Due to \autoref{eq:transitiverelationcontainment}, this statement is also true for all $\conditionelement{c}{l} \in \condition{c}{l, \consistencyrelation{CR}{}}$.
    Furthermore, \autoref{lemma:consistencytransitiveclosure} states that:
    \begin{align}
        & \label{eq:consistencytransitiveequal}
        \forall \modeltuple{m} \in \metamodeltupleinstanceset{M}: \modeltuple{m} \consistenttomath \consistencyrelationset{CR} \equivalent \modeltuple{m} \consistenttomath \transitiveclosure{\consistencyrelationset{CR}}
    \end{align}
    In consequence of Equations \ref{eq:compatibilitysinglelementtransitive} and \ref{eq:consistencytransitiveequal}, we know that:
    \begin{align}
        & \nonumber \label{eq:compatibilityonlyclosure}
        \forall \consistencyrelation{CR}{} \in \transitiveclosure{\consistencyrelationset{CR}} \setminus \consistencyrelationset{CR} : \forall \conditionelement{c}{} \in \condition{c}{l, \consistencyrelation{CR}{}} %\cup \condition{c}{r, \consistencyrelation{CR}{}} 
        : \exists \modeltuple{m} \in \metamodeltupleinstanceset{M} : \\
        & \formulaskip
        \modeltuple{m} \containsmath \conditionelement{c}{} \land \modeltuple{m} \consistenttomath \transitiveclosure{\consistencyrelationset{CR}}
    \end{align}
    %
    Second, we consider $\consistencyrelation{CR}{} \in \consistencyrelationset{CR}$.
    Due to compatibility of $\consistencyrelationset{CR}$ and \autoref{lemma:consistencytransitiveclosure} showing equality of consistency of $\modeltuple{m}$ regarding $\consistencyrelationset{CR}$ and $\transitiveclosure{\consistencyrelationset{CR}}$, it is true that:
    \begin{align}
        & \nonumber \label{eq:compatibilitynonclosure}
        \forall \consistencyrelation{CR}{} \in \consistencyrelationset{CR} : \forall \conditionelement{c}{} \in \condition{c}{l, \consistencyrelation{CR}{}} %\cup \condition{c}{r, \consistencyrelation{CR}{}} 
        : \exists \modeltuple{m} \in \metamodeltupleinstanceset{M} : \\
        & \formulaskip
        \modeltuple{m} \containsmath \conditionelement{c}{} \land \modeltuple{m} \consistenttomath \transitiveclosure{\consistencyrelationset{CR}}
    \end{align}
    %
    With \autoref{eq:compatibilityonlyclosure} and \autoref{eq:compatibilitynonclosure}, we have shown compatibility of $\transitiveclosure{\consistencyrelationset{CR}}$ if $\consistencyrelationset{CR}$ is compatible.
\end{proof}

% \begin{figure}
%     \centering
%     \includegraphics[width=0.7\columnwidth]{figures/incompatible_constraints.png}
%     \caption{Three metamodels and three consistency relations relating those elements that have the same valid $i$ ($C1$ and $C2$ resp. $C1$ and $C3$) or a $i$ value differing by $1$ ($C2$ and $C3$)}
%     \label{fig:incompatible_constraints}
% \end{figure}

% \begin{example}
%     Consider the three metamodels in \autoref{fig:incompatible_constraints}, each containing one class with one attribute $i$. They are related by three consistency relations specifying that in two cases objects having the same $i$ are in the consistency relations, whereas in one case two objects with and $i$ differing by $1$ are in the consistency relation. For any of the consistency relation pairs there are not finite models that are consistency to the consistency relations. Such a model would have to be infinite, as it needed to contain the infinite number of objects with all values of $i$.
%     In consequence, those consistency relations would be considered \emph{incompatible} according to \ref{def:compatibility}.
% \end{example}

% \begin{definition}[Strong Compatibility]
%     Let $\set{\consistencyrelation{CR}}$ be a set of consistency relations for metamodels $\metamodel[1]{M}, \ldots, \metamodel[k]{M}$.
%     A consistency relation $\consistencyrelation{CR}{}$ is considered \emph{strongly compatible with} $\set{\consistencyrelation{CR}{}}$ iff
%     \begin{align*}
%         \formulaskip
%         & 
%         \forall \tupled{\model[1]{m}, \ldots, \model[k]{m}} \in \metamodelinstances{\metamodel[1]{M}} \times \dots \times \metamodelinstances{\metamodel[k]{M}} : \\
%         & 
%     \end{align*}
% \end{definition}


% \begin{definition}[Weak Compatibility]
%     Let $\set{\consistencyrelation{CR}}$ be a set of consistency relations for metamodels $\metamodel[1]{M}, \ldots, \metamodel[k]{M}$.
%     A consistency relation $\consistencyrelation{CR}$ is considered \emph{compatible with} $\set{\consistencyrelation{CR}}$ iff
%     \begin{align*}
%         \formulaskip
%         & 
%         \exists \tupled{\tupled{e_{l1}, \ldots, e_{ln}}, \tupled{e_{r1}, \ldots, e_{rm}}} \in \consistencyrelation{CR} : \\
%         & 
%         \exists \tupled{\model[1]{m}, \ldots, \model[k]{m}} \in \metamodelinstances{\metamodel[1]{M}} \times \dots \times \metamodelinstances{\metamodel[k]{M}} : \exists i, j \in \{1, \ldots, k \} : \\
%         & \formulaskip 
%         \{ e_{l1}, \ldots, e_{ln} \} \subseteq \model[i]{m} \land \{ e_{r1}, \ldots, e_{rm} \} \subseteq \model[j]{m} \\
%         & \formulaskip\formulaskip
%         \land \{ \model[1]{m}, \ldots, \model[k]{m} \} \; \text{consistent according to} \; \set{\consistencyrelation{CR}}
%     \end{align*}
% \end{definition}

% This means that a consistency relation is considered compatible with a set of other consistency relations if there is at least one entry of the consistency relation, i.e. a pair of element tuples, which co-occurs in any set of models such that the models are consistent according to the other consistency relations as well.
% If there is no such set of models, which contains one entry of the consistency relation and is consistent according to the other consistency relations, the consistency relations can never be fulfilled altogether, so the consistency relation is considered \emph{incompatible}.

% \begin{definition}[Possibly well-defined Consistency Relations]
%     A set of consistency relations $\set{\consistencyrelation{CR}}$ is considered \emph{possibly well-defined} iff 
%     \begin{align*}
%         \formulaskip
%         & 
%         \forall \consistencyrelation{CR} \in \set{\consistencyrelation{CR}}: \consistencyrelation{CR} \; \text{is compatible with} \; \set{\consistencyrelation{CR}} \setminus \{ \consistencyrelation{CR} \}
%     \end{align*}
% \end{definition}

%\begin{itemize}
%    \item Insight: Compatibility is a mandatory requirement for interoperability of transformations. If they are not compatible, consistency repair will not be able to find a set of consistent models after certain modifications, because the consistency relations do not do not specify appropriate sets of consistent models.
    %\item TODO: Discuss valid models, why we do not consider them and prove that invariants + consistency relations can express any consistency relation.
%\end{itemize}

%\end{copiedFrom} % SoSym MPM4CPS


\section{A Formal Approach to Prove Compatibility}
\label{chap:formal:approach}

\begin{copiedFrom}{SoSym MPM4CPS}

In this section, we use the definition of compatibility to derive a formal approach for proving compatibility of consistency relations.
The approach bases on two ideas:
\begin{longenumerate}
    \item A set of consistency relations in which each pair of classes is only related across one concatenation of relations is inherently compatible, because there cannot be any contradictory relations. We precisely define this in a specific notion of \emph{consistency relation trees}.
    \item A consistency relation that is redundant in a set of relations, i.e., a relation that does not alter the notion of consistency for models regarding the other relations in that set, does not affect compatibility and can thus be removed from that set of relations. % in which it is redundant.
\end{longenumerate}
Given a set of consistency relations, compatibility can be proven inductively if a consistency relation tree that is equivalent to the set of relations can be found by only removing redundant relations from that set.
Finding such an equivalent consistency relation tree serves as a \emph{witness} for compatibility of a set of relations.
In the following, we formalize and prove this inductive approach to check compatibility of a set of consistency relations.
%This constitutes our contribution \ref{contrib:formalapproach}.

The sketched approach for witnessing compatibility is based on a definition of equivalence for sets of consistency relations.
We consider two sets of consistency relations equivalent if they consider the same sets of models as consistent:

\begin{definition}[Equivalence of Consistency Relations]
\label{def:equivalence}
    Let $\consistencyrelationset{CR}_{1}, \consistencyrelationset{CR}_{2}$ be two sets of consistency relations defined for a tuple of metamodels $\metamodeltuple{M}$. % = \setted{\metamodel{M}{1}, \ldots, \metamodel{M}{k}}$.
    We say that:
    \begin{align*}
        \formulaskip
        &
        \consistencyrelationset{CR}_{1} \equivalenttomath \consistencyrelationset{CR}_{2} \equivalentperdefinition \forall \modeltuple{m} \in \metamodeltupleinstanceset{M} : \\
        %& \formulaskip
        %\forall \modelset{m} = \setted{\model{m}{1}, \dots \model{m}{k}}, \model{m}{i} \in \metamodelinstances{\metamodel{M}{i}} : \\ 
        & \formulaskip%\formulaskip
        \modeltuple{m} \consistenttomath \consistencyrelationset{CR}_{1} \equivalent \modeltuple{m} \consistenttomath \consistencyrelationset{CR}_{2}
    \end{align*}
\end{definition}

%Two sets of consistency relations are considered equivalent if any set of models is either considered consistent or not by both of them in the same way.

The goal of our approach is to find a set of consistency relations that is compatible and equivalent to a given consistency relation set.
We will later use equivalence to introduce a specific notion of redundancy that is compatibility-preserving.
In the following, we first consider structures of consistency relation sets that are inherently compatible and afterwards consider redundancy as a means to find an equivalent representation of a relation set that has such a structure.

%%
%% Properties for inherent compatibility
%%
We first consider two essential properties of a consistency relation set that lead to its inherent compatibility:
\begin{properdescription}
    \item[Composability:] We show that the union of independent, compatible sets of consistency relations is compatible.
    \item[Trees:] We show that relations fulfilling a special notion of \emph{consistency relation trees} are inherently compatible.
\end{properdescription}
In consequence, we know that a consistency relation set that is composed of independent subsets of consistency relation trees is inherently compatible.
Afterwards, we discuss how to find redundant consistency relations to be able to reduce and decompose sets of relations into compositions of independent consistency relation trees.


\subsection{Independence of Consistency Relations}

We consider consistency relation sets as independent if there are no transitive consistency relations induced by relations from both sets, i.e., for each object in a model consistency is only restricted by one of those sets.

\begin{definition}[Independence of Consistency Relation Sets]
    \label{def:independence}
    Let $\consistencyrelationset{CR}_{1}$ and $\consistencyrelationset{CR}_{2}$ be two sets of consistency relations. We say that:
    \begin{align*}
        \formulaskip &
        \consistencyrelationset{CR}_{1} \mathtext{and} \consistencyrelationset{CR}_{2} \mathtext{are independent} \equivalentperdefinition \\
        & \formulaskip
        \forall \consistencyrelation{CR}{} \in \consistencyrelationset{CR}_{1} : \forall \consistencyrelation{CR'}{} \in \consistencyrelationset{CR}_{2} : \\
        & \formulaskip 
        \forall \consistencyrelation{CR}{1}, \dots, \consistencyrelation{CR}{k} \in \consistencyrelationset{CR}_{1} \cup \consistencyrelationset{CR}_{2} : \\
        & \formulaskip\formulaskip
        \consistencyrelation{CR}{} \concat \consistencyrelation{CR}{1} \concat \dots \concat \consistencyrelation{CR}{k} \concat \consistencyrelation{CR'}{} = \emptyset \\
        & \formulaskip\formulaskip
        \land \consistencyrelation{CR'}{} \concat \consistencyrelation{CR}{1} \concat \dots \concat \consistencyrelation{CR}{k} \concat \consistencyrelation{CR}{} = \emptyset
    \end{align*}
    We call $\consistencyrelationset{CR}$ \emph{connected} if there is no partition of a consistency relation set $\consistencyrelationset{CR}$ into two subsets that are independent, i.e.
    \begin{align*}
        \formulaskip &
        \forall \consistencyrelationset{CR}_{1}, \consistencyrelationset{CR}_{2} \subseteq \consistencyrelationset{CR} : \\
        & \formulaskip 
        \consistencyrelationset{CR}_{1} \cap \consistencyrelationset{CR}_{2} = \emptyset \land \consistencyrelationset{CR}_{1} \cup \consistencyrelationset{CR}_{2} = \consistencyrelationset{CR}  \\
        & \formulaskip\formulaskip
        \Rightarrow \neg (\consistencyrelationset{CR}_{1} \mathtext{and} \consistencyrelationset{CR}_{2} \mathtext{are independent),}
    \end{align*}
\end{definition}

\begin{figure}
    \centering
    \newcommand{\hdistance}{(22em+0.3*\difftoafiveimage)}
\newcommand{\vdistance}{1.5em}
\newcommand{\classwidth}{6em}

\begin{tikzpicture}

% Resident
\umlclassvarwidth{resident}{}{Resident\sameheight}{
name
}{\classwidth}

% Employee
\umlclassvarwidth[, right=\hdistance of resident.north, anchor=north]{employee}{}{Employee\sameheight}{
name
}{\classwidth}

% Location
\umlclassvarwidth[, below=\vdistance of resident.south, anchor=north]{location}{}{Location\sameheight}{
street
}{\classwidth}

% Address
\umlclassvarwidth[, below=\vdistance of employee.south, anchor=north]{address}{}{Address\sameheight}{
street
}{\classwidth}

% CONSISTENCY RELATIONS
\draw[directed consistency relation] (resident.east) -- node[pos=0, above right] {$e$} node[pos=0.5, below, align=center] {
    $\consistencyrelation{CR}{1} = \setted{ \tupled{r,e} \mid \mathvariable{e.name} = \mathvariable{r.name}}$
} node[pos=1, above left] {$r$} (employee.west);

\draw[directed consistency relation] (location.east) -- node[pos=0, above right] {$l$} node[pos=0.5, below, align=center] {
    $\consistencyrelation{CR}{2} = \setted{ \tupled{l,a} \mid \mathvariable{l.street} = \mathvariable{a.street}}$
} node[pos=1, above left] {$a$} (address.west);

\end{tikzpicture}
%    \includegraphics[width=\columnwidth]{figures/independence_example.png}
    \caption[Two independent sets of consistency relations]{Two independent (sets of) consistency relations. Taken from \owncite{klare2020compatibility-report}.}
    \label{fig:compatibility:independence_example}
\end{figure}

%We call two consistency relation sets independent if the sets of elements that they relate to each other are completely independent of each other.
\begin{example}
\autoref{fig:compatibility:independence_example} depicts a simple example with two consistency relations $\consistencyrelation{CR}{1}$ and $\consistencyrelation{CR}{2}$, each relating instances of two disjoint classes with each other.
Since there is no overlap in the objects that are related by the consistency relations, they are considered independent according to \autoref{def:independence}.
\end{example}

An important property of independent sets of consistency relations is that computing their union is compatibility-preserving, i.e., the union of compatible, independent consistency relation sets is compatible as well:

\begin{theorem} \label{theorem:independencecompatibility}
    Let $\consistencyrelationset{CR}_{1}$ and $\consistencyrelationset{CR}_{2}$ be two compatible sets of consistency relations. Then $\consistencyrelationset{CR}_{1} \cup \consistencyrelationset{CR}_{2}$ is compatible.
\end{theorem}

\todoLater{Revise proof with explicit references to independence definition}
\begin{proof}
    Since $\consistencyrelationset{CR}_{1}$ is compatible, per definition there is a model tuple $\modeltuple{m}$ for each condition element $\conditionelement{c}{}$ of the left condition of each consistency relation in $\consistencyrelationset{CR}_{1}$ that contains $\conditionelement{c}{}$ and that is consistent to $\consistencyrelationset{CR}_{1}$.
    Taking such an $\modeltuple{m}$, we create a new $\modeltuple{m'}$ by removing all elements from $\modeltuple{m}$, which are contained in any condition elements in any consistency relation in $\consistencyrelationset{CR}_{2}$ and thus potentially require other elements to occur to be considered consistent to that consistency relation.
    In consequence, $\modeltuple{m'}$ does not contain any condition elements from consistency relations in $\consistencyrelationset{CR}_{2}$ and is thus consistent to $\consistencyrelationset{CR}_{2}$ by definition. 
    Additionally, $\modeltuple{m'}$ is still consistent to $\consistencyrelationset{CR}_{1}$, because due to the independence of $\consistencyrelationset{CR}_{1}$ and $\consistencyrelationset{CR}_{2}$, there cannot be any consistency relations in $\consistencyrelationset{CR}_{1}$, which require the existence of the removed elements.
    In consequence, for each condition element $\conditionelement{c}{}$ of each consistency relation in $\consistencyrelationset{CR}_{1}$ there is a model tuple that contains $\conditionelement{c}{}$ and that is consistent to $\consistencyrelationset{CR}_{1} \cup \consistencyrelationset{CR}_{2}$.
    The analogous argumentation applies to the consistency relations in $\consistencyrelationset{CR}_{2}$, which is why the definition of compatibility is fulfilled for all condition elements of all consistency relations in $\consistencyrelationset{CR}_{1} \cup \consistencyrelationset{CR}_{2}$.
\end{proof}

The constructive proof can also be reflected exemplarily in \autoref{fig:compatibility:independence_example}: Take any tuple of models that, for example, contains a resident with an arbitrary name and is consistent to $\consistencyrelation{CR}{1}$, i.e., that also contains an employee with the same name.
If that tuple of models contains any addresses or locations, they can be removed %without removing the resident from the model and 
without violating consistency to $\consistencyrelation{CR}{1}$, because addresses and locations are independently related by $\consistencyrelation{CR}{2}$.

\todo{Actually, addresses may not be removable because they are referenced by persons. However, if such a reference always needs to be set, this is a restriction that we do not reflect yet in the metamodel formalism. On the other hand, if the reference was relevant for consistency in the first consistency relation, it would have to be considered there as well, thus address would be part of that consistency relation.}


\subsection{Consistency Relation Trees}

\mnote{Intuitively, trees are compatible}
We want to find a property or specific structure of consistency relation sets, which leads to inherent consistency of the contained relations.
If we can then prove for a set of relations that we reduce it to such a structure in a compatibility-preserving way, we know that the relations are compatible.
Intuitively, such a structure is given by a kind of trees, because then there are no two concatenations of consistency relations which can relate elements in a contradictory way.

\begin{figure}
    \centering
    \includegraphics[width=0.5\textwidth]{figures/correctness/compatibility/hypertree.jpg}
    \caption[A hypertree with its host graph]{A hypertree (blue edges) with its host graph (green edges).}
    \label{fig:compatibility:hypertree}
\end{figure}

\mnote{Relations may be compatible if their induced hypergraphs are hypertrees}
Since consistency relations put tuple of classes into relation, it may seem natural to use a hypergraph, consisting of the classes as vertices and the class tuples of the consistency relations as hyperedges, to describe them, and expect that if the hypergraph induced by the consistency relations forms a hypertree\footnote{A hypergraph is a hypertree if there is a tree, such that every edge of the hypergraph is a set of vertices of a connected subtree of the tree~\cite{brandstaedt1998hypertrees}. Such a tree is called a \emph{host graph}.}, it is compatible.
An example for a hypertree is depicted in \autoref{fig:compatibility:hypertree}.
It consists of two hyperedges, one only relating persons and residents and one relating persons with residents and their address locations.
This is a scenario that may occur when one consistency relation puts persons and residents with their names into relation, and another describes the relation of their addresses.
These relations form a hypertree, as there is a tree, depicted in the figure, of which the vertices of all hyperedges are connected subtrees.

\mnote{Relations inducing hypertrees are not necessarily compatible}
The relations in the example may, however, not be necessarily compatible.
They can, for example, define a contradictory relation between the names of persons and residents.
In general, hypertrees induced by consistency relations are not necessarily compatible, because hyperedges can be subsets of other hyperedges and the relations inducing these hyperedges may contradict each other.
Additionally, the hyperedges to do only put sets of classes into relation and are not able distinguish between the sets belonging to different metamodels.
Thus, we have to exclude that the same classes are put into relation by multiple consistency relations in a different way.
This leads to our definition consistency relation trees.

\todoLater{Maybe we can remove symmetry and define some restriction for inverse relations. It could be useful to think about implicit relations, which are induced by another one, so that the "signatures" of forward and backward direction match.}
\begin{definition}[Consistency Relation Tree] \label{def:relationtree}
    Let $\consistencyrelationset{CR}$ be a symmetric, connected set of consistency relations. 
    %Let $\alpha = \setted{\tupled{\consistencyrelation{CR}{1}, \consistencyrelation{CR}{2}} \in \consistencyrelationset{CR} \times \consistencyrelationset{CR} \mid \classtuple{C}{r,\consistencyrelation{CR}{1}} \cap \classtuple{C}{l,\consistencyrelation{CR}{2}} \neq \emptyset \land \classtuple{C}{r,\consistencyrelation{CR}{2}} \cap \classtuple{C}{l,\consistencyrelation{CR}{1}} = \emptyset}$ be a relation on the consistency relations in $\consistencyrelationset{CR}$.
    We say:
    \begin{align*}
        \formulaskip &
        \consistencyrelationset{CR} \mathtext{is a consistency relation tree} \equivalentperdefinition \\
        & \formulaskip
        %\alpha \mathtext{induces a tree on} \consistencyrelationset{CR}
        \forall \consistencyrelation{CR}{} = \consistencyrelation{CR}{1} \concat \dots \concat \consistencyrelation{CR}{m}  \in \transitiveclosure{\consistencyrelationset{CR}} : \\
        & \formulaskip
        \forall \consistencyrelation{CR'}{} = \consistencyrelation{CR'}{1} \concat \dots \concat \consistencyrelation{CR'}{n} \in \transitiveclosure{\consistencyrelationset{CR}} \setminus \consistencyrelation{CR}{} : \\
        & \formulaskip\formulaskip
        \forall s, t \mid s \neq t: 
        \consistencyrelation{CR}{s} \neq \consistencyrelation{CR^T}{t} \land \consistencyrelation{CR'}{s} \neq \consistencyrelation{CR'^T}{t} \\
        & \formulaskip\formulaskip
        \Rightarrow
        \classtuple{C}{l,\consistencyrelation{CR}{}} \cap
        \classtuple{C}{l,\consistencyrelation{CR'}{}} = \emptyset
        \lor \classtuple{C}{r,\consistencyrelation{CR}{}} \cap
        \classtuple{C}{r,\consistencyrelation{CR'}{}} = \emptyset
        %
        % \forall \tupled{\consistencyrelation{CR}{1}, \dots, \consistencyrelation{CR}{k}},  \tupled{\consistencyrelation{CR'}{1}, \dots, \consistencyrelation{CR'}{m}} \in \consistencyrelationset{CR} : \\
        % & \formulaskip\formulaskip
        % \tupled{\consistencyrelation{CR}{1}, \dots, \consistencyrelation{CR}{k}} \neq \tupled{\consistencyrelation{CR'}{1}, \dots, \consistencyrelation{CR'}{m}} \\
        % & \formulaskip\formulaskip\formulaskip
        % \land \forall i, j \mid i \neq j: 
        % \consistencyrelation{CR}{i} \neq \consistencyrelation{CR^T}{j} \land \consistencyrelation{CR'}{i} \neq \consistencyrelation{CR'^T}{j} \\
        % & \formulaskip\formulaskip
        % \Rightarrow
        % \classtuple{C}{l,\consistencyrelation{CR}{1}} \cap
        % \classtuple{C}{l,\consistencyrelation{CR'}{1}} = \emptyset
        % \lor \classtuple{C}{r,\consistencyrelation{CR}{m}} \cap
        % \classtuple{C}{r,\consistencyrelation{CR'}{m}} = \emptyset
        %
        % \forall \consistencyrelation{CR}{1}, \dots, \consistencyrelation{CR}{k} \in \consistencyrelationset{CR} \mid \consistencyrelation{CR}{i} \neq \consistencyrelation{CR}{j}, \consistencyrelation{CR}{i} \neq \consistencyrelation{CR^T}{j} : \\
        % & \formulaskip\formulaskip
        % \classtuple{C}{l,\consistencyrelation{CR}{1}\concat\dots\concat\consistencyrelation{CR}{k}} \cap
        % \classtuple{C}{r,\consistencyrelation{CR}{1}\concat\dots\concat\consistencyrelation{CR}{k}} = \emptyset
        % \forall \class{C}{l} \in
        % \classtuple{C}{l,\consistencyrelation{CR}{1}\concat\dots\concat\consistencyrelation{CR}{k}} :
        % \forall \class{C}{r} \in \classtuple{C}{r,\consistencyrelation{CR}{1}\concat\dots\concat\consistencyrelation{CR}{k}}: \\
        % & \formulaskip\formulaskip
        % \class{C}{l} \cap \class{C}{r} = \emptyset
    \end{align*}
\end{definition}
\todo{We have to assume, that no element is mapped to two elements of the same class, because then it would be possible to have an incompatible network}

The definition of a consistency relation tree requires that there are no sequences of consistency relations that put the same classes into relation, i.e. between all pairs of classes there is only one concatenation of consistency relations that puts them into relation.
Since we assume a symmetric set of consistency relations, we exclude the symmetric relations from that argument, as otherwise there would always be two such concatenations by adding a consistency relation and its transposed relation to any other concatenation.

\begin{figure}
    \centering
    \newcommand{\hdistance}{11.5em}
\newcommand{\classwidth}{6em}

\begin{tikzpicture}

% Person
\umlclassvarwidth{person}{}{Person\sameheight}{
firstname\\
lastname
}{\classwidth}


\umlclassvarwidth[,right=\hdistance of person.center, anchor=center]{resident}{}{Resident\sameheight}{
name
}{\classwidth}

% Employee
\umlclassvarwidth[,right=\hdistance of resident.center, anchor=center]{employee}{}{Employee\sameheight}{
name
}{\classwidth}


% CONSISTENCY RELATIONS
\draw[directed consistency relation] (resident.west-|person.east) -- node[pos=0, above right] {$p$} node[pos=0.5, below] {$\consistencyrelation{CR}{1}$} node[pos=1, above left] {$r$} (resident.west);
\draw[directed consistency relation] (resident) -- node[pos=0, above right] {$r$} node[below] {$\consistencyrelation{CR}{2}$} node[pos=1, above left] {$e$} (employee);


\node[consistency related element, below left=1em and 0em of resident.south, anchor=north, inner sep=0em] {
$\begin{aligned}
    \consistencyrelation{CR}{1} =\; & \setted{\tupled{p,r} \mid r.name = \mathvariable{p.firstname} + "\text{\textvisiblespace}" + \mathvariable{p.lastname}}\\ %[0.3em]
    \consistencyrelation{CR}{2} =\; & \setted{\tupled{r,e} \mid \mathvariable{r.name} = \mathvariable{e.name}}
\end{aligned}$
};

\end{tikzpicture}
    %\includegraphics[width=\columnwidth]{figures/tree_example.png}
    \caption[A consistency relation tree]{A consistency relation tree $\setted{\consistencyrelation{CR}{1}, \consistencyrelation{CR^T}{1}, \consistencyrelation{CR}{2}, \consistencyrelation{CR^T}{2}}$. Taken from \owncite{klare2020compatibility-report}.}
    \label{fig:compatibility:tree_example}
\end{figure}

\begin{example}
\autoref{fig:compatibility:tree_example} depicts a rather simple consistency relation tree. 
Persons are related to residents and residents are related to employees, all having the same names respectively a concatenation of $firstname$ and $lastname$, by the relations $\consistencyrelation{CR}{1}, \consistencyrelation{CR}{2}$, as well as their transposed relations $\consistencyrelation{CR^T}{1}, \consistencyrelation{CR^T}{2}$.
There are no classes that are put into relation across different paths of consistency relations, thus the definition for a consistency relation tree is fulfilled. 
If an additional relation between persons and employees was specified, like in \autoref{fig:compatibility:three_persons_example_extended}, the tree definition would not be fulfilled.
\end{example}

The definition also covers the more complicated case in which multiple classes may be put into relation by consistency relations but there is only a subset of them that is put into relation by different consistency relations.
%
%\todoDiss{Subsection with discussion about why hypertrees are not suitable here}
%
We can now prove that a set of consistency relations that is a consistency relation tree is always compatible.
We first present a lemma that shows that in a consistency relation tree you can always find an order of the relations such that the classes at the right side of a relation do not overlap with the classes at the left side of a relation that preceded in the order, i.e. there is no cycle in the relations between classes.

\begin{lemma} \label{lemma:treehassequence}
    Let $\consistencyrelationset{CR} = \setted{\consistencyrelation{CR}{1}, \consistencyrelation{CR^T}{1}, \dots, \consistencyrelation{CR}{k}, \consistencyrelation{CR^T}{k}}$ be a symmetric, connected set of consistency relations.
    $\consistencyrelationset{CR}$ is a consistency relation tree if and only if for each $\consistencyrelation{CR}{}$ there exists a sequence of consistency relations $\tupled{\consistencyrelation{CR'}{1}, \dots, \consistencyrelation{CR'}{k}}$ with $\consistencyrelation{CR'}{1} = \consistencyrelation{CR}{}$, containing for each $i$ either $\consistencyrelation{CR}{i}$ or $\consistencyrelation{CR^T}{i}$, i.e.,
    \begin{align*}
        \formulaskip &
        \forall i \in \setted{1, \dots, k} :\\
        & \formulaskip 
        \bigl( \consistencyrelation{CR}{i} \in \tupled{\consistencyrelation{CR'}{1}, \dots, \consistencyrelation{CR'}{k}}
        \land \consistencyrelation{CR^T}{i} \not\in \tupled{\consistencyrelation{CR'}{1}, \dots, \consistencyrelation{CR'}{k}} \bigl)\\
        & \formulaskip 
        \lor \bigl(\consistencyrelation{CR^T}{i} \in \tupled{\consistencyrelation{CR'}{1}, \dots, \consistencyrelation{CR'}{k}}
        \land \consistencyrelation{CR}{i} \not\in \tupled{\consistencyrelation{CR'}{1}, \dots, \consistencyrelation{CR'}{k}} \bigl)
    \end{align*}
    such that:
    \begin{align*}
        \formulaskip &
        %\forall \consistencyrelation{CR}{} \in \consistencyrelationset{CR} : 
        %\exists \consistencyrelation{CR'}{1}, \dots, \consistencyrelation{CR'}{k-1} \in \consistencyrelationset{CR} \setminus \setted{\consistencyrelation{CR}{}} :
        %\consistencyrelation{CR'}{i} \neq \consistencyrelation{CR'}{j}, i \neq j : \\
        \forall s \in \setted{1, \dots, k-1} : \forall t \in \setted{i+1, \dots, k} : \\
        & \formulaskip
        \classtuple{C}{r,\consistencyrelation{CR'}{s}} \cap \classtuple{C}{r,\consistencyrelation{CR'}{t}} = \emptyset 
        \land
        \classtuple{C}{l,\consistencyrelation{CR'}{s}} \cap 
        \classtuple{C}{r,\consistencyrelation{CR'}{t}} = \emptyset
        % \forall i \in \setted{1, \dots, k-1} : \\
        % & \formulaskip
        % (\exists j \in \setted{1, \dots, i-1} :  \classtuple{C}{l,\consistencyrelation{CR'}{j}} \subseteq \classtuple{C}{r,\consistencyrelation{CR'}{i}}) \\
        % & \formulaskip
        % \lor 
        % \classtuple{C}{l,\consistencyrelation{CR'}{}} \subseteq \classtuple{C}{r,\consistencyrelation{CR'}{i}}
    \end{align*}
\end{lemma}

\begin{proof}
    We start with the forward direction, i.e., given a consistency relation tree $\consistencyrelationset{CR}$ we show that there exists a sequence according to the requirements in \autoref{lemma:treehassequence} by constructing such a sequence $\tupled{\consistencyrelation{CR'}{1}, \dots, \consistencyrelation{CR'}{k}}$ for any $\consistencyrelation{CR}{} \in \consistencyrelationset{CR}$.
    Start with $\consistencyrelation{CR'}{1} = \consistencyrelation{CR}{}$ for any $\consistencyrelation{CR}{} \in \consistencyrelationset{CR}$.
    We now inductively add further relations to that sequence.
    Take any consistency relation $\consistencyrelation{CR}{s} = \consistencyrelation{CR}{s,1} \concat \dots \concat \consistencyrelation{CR}{s,m} \in \transitiveclosure{\consistencyrelationset{CR}}$ with $\classtuple{C}{l,\consistencyrelation{CR}{s,1}} \subseteq \classtuple{C}{r,\consistencyrelation{CR}{}}$. Such a sequence must exist because of $\consistencyrelation{CR}{}$ being connected.
    Now add all $\consistencyrelation{CR}{s,1}, \dots, \consistencyrelation{CR}{s,m}$ to the sequence, which fulfills both requirements to that sequence in \autoref{lemma:treehassequence} by definition.
    The following addition of further consistency relations can be inductively applied.
    Take any other consistency relation $\consistencyrelation{CR}{t} = \consistencyrelation{CR}{t,1} \concat \dots \concat \consistencyrelation{CR}{t,n} \in \transitiveclosure{\consistencyrelationset{CR}}$ such that:
    \begin{align*}
        \formulaskip &
        \exists \consistencyrelation{CR'}{} \in \setted{\consistencyrelation{CR}{}, \consistencyrelation{CR}{s,2}, \dots, \consistencyrelation{CR}{s,m}} :
        \classtuple{C}{l,\consistencyrelation{CR}{t,1}} \subseteq \classtuple{C}{r,\consistencyrelation{CR'}{}}\\
        & \formulaskip
        \land
        \consistencyrelation{CR}{t,1}, \consistencyrelation{CR^T}{t,1} \not\in \setted{\consistencyrelation{CR}{}, \consistencyrelation{CR}{s,2}, \dots, \consistencyrelation{CR}{s,m}}
    \end{align*}
    In other words, take any concatenation in the transitive closure of $\consistencyrelationset{CR}$ that starts with a relation with a left class tuple that is contained in a right class tuple of a relation already added to the sequence.
    Again, such a sequence must exist because of $\consistencyrelationset{CR}$ being connected and, again, add all $\consistencyrelation{CR}{t,1}, \dots, \consistencyrelation{CR}{t,n}$ to the sequence.
    Per construction, for each $\consistencyrelation{CR'}{}$ in the sequence, there is a non-empty concatenation of relations within the sequence $\consistencyrelation{CR}{} \concat \dots \concat \consistencyrelation{CR'}{}$, because relations were added in a way that such a concatenation always exists. Since all relations in the sequence are contained in $\consistencyrelationset{CR}$, such a concatenation was also contained in $\transitiveclosure{\consistencyrelationset{CR}}$.
    First, we show that the sequence still contains no duplicate elements (1.), i.e., that none of the $\consistencyrelation{CR}{t,i}$ or $\consistencyrelation{CR^T}{t,i}$ is already contained in the sequence $\tupled{\consistencyrelation{CR}{}, \consistencyrelation{CR}{s,1}, \dots, \consistencyrelation{CR}{s,m}}$. 
    Second, we show that both further conditions for the sequence defined in \autoref{lemma:treehassequence} are still fulfilled for the sequence $\tupled{\consistencyrelation{CR}{}, \consistencyrelation{CR}{s,1}, \dots, \consistencyrelation{CR}{s,m}, \consistencyrelation{CR}{t,1}, \dots, \consistencyrelation{CR}{t,n}}$ (2. ,3.).
    % We show that both conditions for the sequence in \autoref{lemma:treehassequence} are still fulfilled for our sequence $\tupled{\consistencyrelation{CR}{}, \consistencyrelation{CR}{s,2}, \dots, \consistencyrelation{CR}{s,m}, \consistencyrelation{CR}{t,2}, \dots, \consistencyrelation{CR}{t,n}}$ by assuming the contradictory:
    \begin{longenumerate}
        \item
    Let us assume that the sequence $\tupled{\consistencyrelation{CR}{}, \consistencyrelation{CR}{s,1}, \dots, \consistencyrelation{CR}{s,m}}$ already contained one of the $\consistencyrelation{CR}{t,i}$ or $\consistencyrelation{CR^T}{t,i}$. If $\consistencyrelation{CR}{t,i}$ is contained in the sequence, there is a concatenation $\consistencyrelation{CR}{} \concat \dots \concat \consistencyrelation{CR}{t,i}$ with relations in $\tupled{\consistencyrelation{CR}{}, \consistencyrelation{CR}{s,1}, \dots, \consistencyrelation{CR}{s,m}}$, as well as a concatenation $\consistencyrelation{CR}{} \concat \dots \concat \consistencyrelation{CR}{t,1} \concat \dots \concat \consistencyrelation{CR}{t,i}$.
    Since $\consistencyrelation{CR}{t,1} \not\in \setted{\consistencyrelation{CR}{}, \consistencyrelation{CR}{s,2}, \dots, \consistencyrelation{CR}{s,m}}$ by construction, these two concatenations relate the same class tuples, i.e., they contradict the definition of a consistency relation tree.
    If $\consistencyrelation{CR^T}{t,i}$ was contained in the sequence $\tupled{\consistencyrelation{CR}{}, \consistencyrelation{CR}{s,2} \concat \dots \concat \consistencyrelation{CR}{s,m}}$, there is a concatenation $\consistencyrelation{CR}{} \concat \dots \concat \consistencyrelation{CR}{w} \concat \consistencyrelation{CR^T}{t,i}$ with relations in $\tupled{\consistencyrelation{CR}{}, \consistencyrelation{CR}{s,1}, \dots, \consistencyrelation{CR}{s,m}}$ and, like before, the concatenation $\consistencyrelation{CR}{} \concat \dots \concat \consistencyrelation{CR}{t,1}, \dots, \consistencyrelation{CR}{t,i}$.
    Due to $\classtuple{C}{r,\consistencyrelation{CR}{w}} \cap \classtuple{C}{l,\consistencyrelation{CR}{t,i}} \neq \emptyset$ and  $\consistencyrelation{CR^T}{t,1} \not\in \setted{\consistencyrelation{CR}{}, \consistencyrelation{CR}{s,2}, \dots, \consistencyrelation{CR}{s,m}}$ by construction, the two concatenations $\consistencyrelation{CR}{} \concat \dots \concat \consistencyrelation{CR}{w}$ and $\consistencyrelation{CR}{} \concat \dots \concat \consistencyrelation{CR}{t,1} \concat \dots \concat \consistencyrelation{CR}{t,i}$ have an overlap in both their left and right class tuples, i.e., they contradict the definition of a consistency relation tree.
    In consequence, the sequence $\tupled{\consistencyrelation{CR}{}, \consistencyrelation{CR}{s,1}, \dots, \consistencyrelation{CR}{s,m}}$ cannot have contained any $\consistencyrelation{CR}{t,i}$ or $\consistencyrelation{CR^T}{t,i}$ before.
        \item 
    Let us assume there were any $\consistencyrelation{CR'}{u}$ and $\consistencyrelation{CR'}{v}$ in the sequence $\tupled{\consistencyrelation{CR}{}, \consistencyrelation{CR}{s,1}, \dots, \consistencyrelation{CR}{s,m}, \consistencyrelation{CR}{t,1}, \dots, \consistencyrelation{CR}{t,n}}$ such that $\classtuple{C}{r,\consistencyrelation{CR'}{u}} \cap \classtuple{C}{r,\consistencyrelation{CR'}{v}} \neq \emptyset$.
    As discussed before, for each of these relations exists a concatenation of relations in the sequence $\consistencyrelation{CR}{} \concat \dots \concat \consistencyrelation{CR'}{u}$ and $\consistencyrelation{CR}{} \concat \dots \concat \consistencyrelation{CR'}{v}$, which is contained in $\transitiveclosure{\consistencyrelationset{CR}}$.
    This contradicts the definition of a consistency relation tree, so there cannot be two such relations with overlapping classes in the right class tuple.
        \item
    Let us assume there were any $\consistencyrelation{CR'}{u}$ and $\consistencyrelation{CR'}{v}\; (u < v)$ in the sequence $\tupled{\consistencyrelation{CR}{}, \consistencyrelation{CR}{s,1}, \dots, \consistencyrelation{CR}{s,m}, \consistencyrelation{CR}{t,1}, \dots, \consistencyrelation{CR}{t,n}}$ such that $\classtuple{C}{l,\consistencyrelation{CR'}{u}} \cap \classtuple{C}{r,\consistencyrelation{CR'}{v}} \neq \emptyset$.
    Again per construction, there must be a non-empty concatenation $\consistencyrelation{CR}{} \concat \dots \concat \consistencyrelation{CR'}{w} \concat \consistencyrelation{CR'}{u}$ with $w < u$. Since $\classtuple{C}{l,\consistencyrelation{CR'}{u}} \subseteq \classtuple{C}{r,\consistencyrelation{CR'}{w}}$ per definition, it holds that
    $\classtuple{C}{r,\consistencyrelation{CR'}{w}} \cap \classtuple{C}{r,\consistencyrelation{CR'}{v}} \neq \emptyset$.
    In other words, the relation $\consistencyrelation{CR'}{v}$ introduces a cycle in the relations.
    We have already shown in (2.) that this contradicts the definition of a consistency relation tree.
    \end{longenumerate}

    The previous strategy for adding relations to the sequence can be continued inductively by adding relations of the transitive closure of $\consistencyrelationset{CR}$ if their relations were not already added to the sequence.
    This process can be continued until finally all relations in $\consistencyrelationset{CR}$ are added to the sequence.
    Inductively applying the same arguments as before, the final sequence still fulfills all requirements for the sequence in \autoref{lemma:treehassequence}.
    % From the relations in $\consistencyrelationset{CR} \setminus \setted{\consistencyrelation{CR}, \consistencyrelation{CR^T}{}}$, we take those $\consistencyrelation{CR'}{}$ with $\classtuple{C}{l,\consistencyrelation{CR'}{}} \subseteq \classtuple{C}{r,\consistencyrelation{CR}{}}$ and add them to the sequence in an arbitrary order.
    % We recursively proceed with this procedure in a breadth-first fashion for all those added $\consistencyrelation{CR'}{}$.
    % Due to $\consistencyrelationset{CR}$ being connected by definition, this procedure finally adds all $\consistencyrelation{CR}{i}$ or $\consistencyrelation{CR^T}{i}$ to the sequence.
    % By appending always a relation to the sequence whose left class tuple is a subset of the right class tuple of an already added element, for $\consistencyrelation{CR'}{1}$ and all $\consistencyrelation{CR'}{i}$ in the sequence there is always a concatenation of a sub-sequence $\consistencyrelation{CR'}{1} \concat \dots \concat \consistencyrelation{CR'}{i}$ with non-empty left and right class tuples.
    % We show that both conditions for the sequence in \autoref{lemma:treehassequence} are fulfilled by assuming the contradictory:
    % \begin{enumerate}
    %     \item 
    % Let us assume there were any $\consistencyrelation{CR'}{s}$ and $\consistencyrelation{CR'}{t}$ in the sequence, such that $\classtuple{C}{r,\consistencyrelation{CR'}{s}} \cap \classtuple{C}{r,\consistencyrelation{CR'}{t}} \neq \emptyset$.
    % As discussed before, for both these relations there is a concatenation with non-empty left and right class tuples $\consistencyrelation{CR''}{s} = \consistencyrelation{CR'}{1} \concat \dots \concat \consistencyrelation{CR'}{s}$ and $\consistencyrelation{CR''}{t} = \consistencyrelation{CR'}{1} \concat \dots \concat \consistencyrelation{CR'}{t}$
    % such that $\classtuple{C}{l,\consistencyrelation{CR''}{s}} \cap \classtuple{C}{l,\consistencyrelation{CR''}{t}} \neq \emptyset$ and $\classtuple{C}{r,\consistencyrelation{CR''}{s}} \cap \classtuple{C}{r,\consistencyrelation{CR''}{t}} \neq \emptyset$.
    % This contradicts the definition of a consistency relation tree.
    %     \item
    % Let us assume there were any $\consistencyrelation{CR'}{s}$ and $\consistencyrelation{CR'}{t} (s < t)$ such that $\classtuple{C}{l,\consistencyrelation{CR'}{s}} \cap \classtuple{C}{r,\consistencyrelation{CR'}{t}} \neq \emptyset$.
    % Per construction, there must be a sub-sequence $\consistencyrelation{CR'}{1} \concat \dots \consistencyrelation{CR'}{u} \concat \consistencyrelation{CR'}{s}$ with $u < s$ and $\classtuple{C}{l,\consistencyrelation{CR'}{s}} \subseteq \classtuple{C}{r,\consistencyrelation{CR'}{u}}$.
    % In consequence, $\classtuple{C}{r,\consistencyrelation{CR'}{u}} \cap \classtuple{C}{r,\consistencyrelation{CR'}{s}} \neq \emptyset$.
    % We have already shown in the first case that this contradicts the definition of a consistency relation tree.
    % In other words, the relation $\consistencyrelation{CR'}{t}$ introduces a cycle in the relations.
    % \end{enumerate}
    
    We proceed with the reverse direction, i.e., given that a sequence according to the requirements in \autoref{lemma:treehassequence} exists for all $\consistencyrelation{CR}{} \in \consistencyrelationset{CR}$, we show that the set of consistency relations fulfills the definition of a consistency relation tree.
    Let us assume that the tree definition was not fulfilled, i.e., that there were two consistency relations $\consistencyrelation{CR}{s} = \consistencyrelation{CR}{s,1} \concat \dots \concat \consistencyrelation{CR}{s,m} \in \transitiveclosure{\consistencyrelationset{CR}}$ and $\consistencyrelation{CR}{t} = \consistencyrelation{CR}{t,1} \concat \dots \concat \consistencyrelation{CR}{t,n} \in \transitiveclosure{\consistencyrelationset{CR}}$ such that $\classtuple{C}{l,\consistencyrelation{CR}{s}} \cap \classtuple{C}{l,\consistencyrelation{CR}{t}} \neq \emptyset$ and $\classtuple{C}{r,\consistencyrelation{CR}{s}} \cap \classtuple{C}{r,\consistencyrelation{CR}{t}} \neq \emptyset$.
    Without loss of generality, we assume that $\consistencyrelation{CR}{s,m} \neq \consistencyrelation{CR}{t,n}$, because otherwise we could instead consider the sequence without those last relations and still fulfill the defined requirements.
    Any sequence according to \autoref{lemma:treehassequence} containing both $\consistencyrelation{CR}{s,m}$ and $\consistencyrelation{CR}{t,n}$ would contradict the assumption, because $\classtuple{C}{r,\consistencyrelation{CR}{s,m}} \cap \classtuple{C}{r,\consistencyrelation{CR}{t,n}} \neq \emptyset$ in contradiction to the assumptions regarding the sequence.
    Thus, the sequence has to contain either $\consistencyrelation{CR^T}{s,m}$ or $\consistencyrelation{CR^T}{t,n}$.
    Let us assume that the sequence contains $\consistencyrelation{CR^T}{s,m}$.
    Then the sequence cannot contain $\consistencyrelation{CR}{s,m-1}$, because $\classtuple{C}{r,\consistencyrelation{CR^T}{s,m}} \cap \classtuple{C}{r,\consistencyrelation{CR}{s,m-1}} \neq \emptyset$, which, again, would contradict the assumptions regarding the sequence.
    This argument can be inductively applied to all $\consistencyrelation{CR}{s,i}$, such that the sequence has to contain all $\consistencyrelation{CR^T}{s,i}$.
    Since the sequence contains $\consistencyrelation{CR^T}{s,1}$, it must contain $\consistencyrelation{CR}{t,1}$, because $\classtuple{C}{r,\consistencyrelation{CR^T}{s,1}} \cap \classtuple{C}{r,\consistencyrelation{CR^T}{t,1}} \neq \emptyset$.
    In consequence of $\consistencyrelation{CR}{t,1}$ being contained in the sequence, all $\consistencyrelation{CR}{t,i}$ have to be contained as well, due to the same reasons as before.
    So we have these conditions, which introduce a cycle in the overlaps of the class tuples of the relations within the sequence:
    \begin{align*}
        \formulaskip &
        \classtuple{C}{l,\consistencyrelation{CR^T}{s,i-1}} \cap \classtuple{C}{r,\consistencyrelation{CR^T}{s,i}} \neq \emptyset %\\ 
        %&
        \land
        \classtuple{C}{l,\consistencyrelation{CR}{t,1}} \cap \classtuple{C}{r,\consistencyrelation{CR^T}{s,1}} \neq \emptyset\\
        & 
        \land 
        \classtuple{C}{l,\consistencyrelation{CR}{t,i}} \cap \classtuple{C}{r,\consistencyrelation{CR}{t,i-1}} \neq \emptyset %\\
        %&
        \land
        \classtuple{C}{l,\consistencyrelation{CR^T}{s,m}} \cap \classtuple{C}{r,\consistencyrelation{CR}{t,n}} \neq \emptyset
    \end{align*}
    %We argued why all these relations have to be contained in the sequence.
    Because of that cycle in the overlap of class tuples, there is no order of these relations $\consistencyrelation{CR''}{1}, \dots, \consistencyrelation{CR''}{m+n}$ such that for all of them it holds that $\classtuple{C}{l,\consistencyrelation{CR''}{u}} \cap \classtuple{C}{r,\consistencyrelation{CR''}{v}} \neq \emptyset\; (u < v)$, which contradicts the assumptions regarding the sequence in \autoref{lemma:treehassequence}.
    The analog argument holds when we assume that the sequence contains $\consistencyrelation{CR^T}{t,n}$ instead of $\consistencyrelation{CR^T}{s,m}$.
    In consequence, there cannot be two such concatenations $\consistencyrelation{CR}{s}$ and $\consistencyrelation{CR}{t}$ without breaking the assumptions for the sequence in \autoref{lemma:treehassequence}.
    % Take any sequence $\tupled{\consistencyrelation{CR'}{1}, \dots, \consistencyrelation{CR'}{k}}$ with $\consistencyrelation{CR'}{1} = \consistencyrelation{CR}{s,1}$, which necessarily exists per assumption.
    % Then $\tupled{\consistencyrelation{CR}{s,1}, \dots, \consistencyrelation{CR}{s,m}}$ and $\tupled{\consistencyrelation{CR}{t,1}, \dots, \consistencyrelation{CR}{t,n}}$ are contained in $\tupled{\consistencyrelation{CR'}{1}, \dots, \consistencyrelation{CR'}{k}}$, i.e.
    % \begin{align*}
    %     \formulaskip &
    %     \forall i \in \setted{1,\dots,m-1} : 
    %     \exists v, w \in \setted{1,\dots,k} \mid v < w : \\
    %     & \formulaskip
    %     \consistencyrelation{CR}{s,i} = \consistencyrelation{CR'}{v} \land \consistencyrelation{CR}{s,i+1} = \consistencyrelation{CR'}{w}
    % \end{align*}
    % and analogously for $\tupled{\consistencyrelation{CR}{t,1}, \dots, \consistencyrelation{CR}{t,n}}$.
    % If for any $\consistencyrelation{CR}{s,i}$ there was a $\consistencyrelation{CR'}{u} \in \tupled{\consistencyrelation{CR'}{1}, \dots, \consistencyrelation{CR'}{k}}$ with $\consistencyrelation{CR}{s,i} = \consistencyrelation{CR'^T}{u}$, then 
\end{proof}

% \todoHeiko{Make this proof more precise}
% \begin{proof}
%     Let us assume the contrary, i.e. for all sequences $\tupled{\consistencyrelation{CR'}{1}, \dots, \consistencyrelation{CR'}{k}}$ according to \autoref{lemma:treehassequence}, it is true that:
%     \begin{align*}
%         \formulaskip &
%         \exists i \in \setted{1, \dots, k-1} : \\
%         & \formulaskip
%         \exists j \in \setted{i+1, \dots, k-1} :  
%         \classtuple{C}{l,\consistencyrelation{CR'}{i}} \cap 
%         \classtuple{C}{r,\consistencyrelation{CR'}{j}} \neq \emptyset
%     \end{align*}
%     Now select any such sequence $\tupled{\consistencyrelation{CR'}{1}, \dots, \consistencyrelation{CR'}{k}}$. We partition the possible sequences into two disjoint subsets and consider them independently, so this sequence falls into one of the following partitions.
%     \begin{enumerate}
%     \item Consider the following subset of sequences:
%     \begin{align*}
%         \formulaskip &
%         \exists i \in \setted{1, \dots, k-1} : \\
%         & \formulaskip
%         \forall j \in \setted{1, \dots, i-1} :  
%         \classtuple{C}{l,\consistencyrelation{CR'}{i}} \not\subseteq 
%         \classtuple{C}{r,\consistencyrelation{CR'}{j}}
%     \end{align*}
%     \todoHeiko{This is not correct. there must not be such a relation.}
%     This subset includes sequences in which there is a relation whose classes of the left condition are not a subset of the classes of any of the classes of the right condition of a previous relations in that sequence.
%     Due to $\consistencyrelationset{CR}$ being connected, for each relation $\consistencyrelation{CR'}{}$ there must be a relation $\consistencyrelation{CR''}{} \in \consistencyrelationset{CR}$, such that $\classtuple{C}{l,\consistencyrelation{CR''}{}} \subseteq \classtuple{C}{r,\consistencyrelation{CR'}{}}$. Per construction, the  sequence $\tupled{\consistencyrelation{CR'}{1}, \dots, \consistencyrelation{CR'}{k}}$ either contains such a $\consistencyrelation{CR''}{}$, or $\consistencyrelation{CR''^T}{}$.
%     If the sequence contains $\consistencyrelation{CR''}{}$, then the assumption is false by construction.
%     If the sequence contains $\consistencyrelation{CR''^T}{}$ and there is no further $\consistencyrelation{CR'''}{}$ with $\classtuple{C}{l,\consistencyrelation{CR'''}{}} \subseteq \classtuple{C}{r,\consistencyrelation{CR'}{}}$, then the assumption is also false by construction.
%     If there is such another relation, then the same argumentation applies, which inductively means that the assumption is always false.
    
%     \item Consider the following, complementary subset of sequences:
%     \begin{align*}
%         \formulaskip &
%         \forall i \in \setted{1, \dots, k-1} : \\
%         & \formulaskip
%         \exists j \in \setted{1, \dots, i-1} :  
%         \classtuple{C}{l,\consistencyrelation{CR'}{i}} \subseteq 
%         \classtuple{C}{r,\consistencyrelation{CR'}{j}}
%     \end{align*}
%     %Now select any such sequence $\tupled{\consistencyrelation{CR'}{1}, \dots, \consistencyrelation{CR'}{k}}$ in which a later relation in the relation can always be concatenated to a previous relation in the sequence:
%     If the assumption held, then there is a sequence of consistency relations $\tupled{\consistencyrelation{CR'}{1}, \dots, \consistencyrelation{CR'}{s}}$, with $\classtuple{C}{l,\consistencyrelation{CR'}{i+1}} \subseteq \classtuple{C}{r,\consistencyrelation{CR'}{i}}, i \in \setted{1, \dots, s-1}$ and
%     $\classtuple{C}{r,\consistencyrelation{CR}{s}} \cap \classtuple{C}{l, \consistencyrelation{CR}{1}} \neq \emptyset$.
%     Thus, there is the concatenation $\consistencyrelation{CR^\concat}{} = \consistencyrelation{CR'}{1} \concat \dots \concat \consistencyrelation{CR'}{s-1}$ with
%     \begin{align*}
%         \formulaskip
%         \classtuple{C}{l,\consistencyrelation{CR^\concat}{}} = \classtuple{C}{l,\consistencyrelation{CR'}{1}} \land \classtuple{C}{r,\consistencyrelation{CR^\concat}{}} \subseteq  \classtuple{C}{r,\consistencyrelation{CR'}{s-1}}
%      \end{align*}
%     per \autoref{def:relationconcatenation} for concatenation.
%     There is also the consistency relation $\consistencyrelation{CR'^T}{s} \in \consistencyrelationset{CR}$ with
%     \begin{align*}
%         \formulaskip
%         \classtuple{C}{l,\consistencyrelation{CR'^T}{s}} = \classtuple{C}{r,\consistencyrelation{CR'}{s-1}} \land  \classtuple{C}{r,\consistencyrelation{CR'^T}{s}} = \classtuple{C}{l,\consistencyrelation{CR'}{1}}
%     \end{align*}
%     such that:
%     \begin{align*}
%     \formulaskip
%         \classtuple{C}{l,\consistencyrelation{CR^\concat}{}} \cap \classtuple{C}{r,\consistencyrelation{CR'^T}{s}} \neq \emptyset \land \classtuple{C}{l,\consistencyrelation{CR'^T}{s}} \cap \classtuple{C}{r,\consistencyrelation{CR^\concat}{}} \neq \emptyset.
%     \end{align*}
%     In consequence, $\consistencyrelation{CR^\concat}{}$ and $\consistencyrelation{CR'}{}$ break the definition of a consistency relation tree.
%     \end{enumerate}
%     In combination, we have disproved the contrary statement, so we know that the statement in \autoref{lemma:treehassequence} is true.
%  \end{proof}

The previous lemma shows that the definition of consistency relation trees based on unique concatenations of the same class tuples is equivalent the possibility to find sequences of the relations that do not contain cycles in the related class tuples. %for each of the relations a sequence starting with that relation and containing all other relations as well, such that there are no cycles in the classes related by these relations.
The definition is supposed to be easier to check in practice.
However, we can now show that a consistency relation tree is always compatible with a constructive proof that requires the equivalent definition from \autoref{lemma:treehassequence}.

%With the previous lemma, we can now show that a consistency relation tree is always compatible.

\begin{theorem} \label{theorem:treecompatibility}
    Let $\consistencyrelationset{CR}$ be a consistency relation tree, then $\consistencyrelationset{CR}$ is compatible.
\end{theorem}

\begin{figure}
    \centering
    \newcommand{\hdistance}{(10.6em+0.4*\difftoafiveimage)}
\newcommand{\objectwidth}{(6.8em+0.05\difftoafiveimage)}
\renewcommand{\sameheight}{\vphantom{yR}}

\begin{tikzpicture}[
    consistency process/.style={consistency execution}
]

% Person
\umlobjectvarwidth{person}{}{: Person\sameheight}{
firstname = "Alice"\\
lastname = "Avid"
}{\objectwidth-0.3em}

% Resident
\umlobjectvarwidth[,right=\hdistance-0.3em of person.center, anchor=center]{resident}{}{: Resident\sameheight}{
name = "Alice Avid"
}{\objectwidth}

% Employee
\umlobjectvarwidth[,right=\hdistance of resident.center, anchor=center]{employee}{}{: Employee\sameheight}{
name = "Alice Avid"
}{\objectwidth}


% CONSISTENCY RELATIONS
\draw[directed consistency relation] ([yshift=-1em]person.east) -- node[pos=0, above right] {$p$} node[below] {$\consistencyrelation{CR}{1}$} node[pos=1, above left] {$r$} ([yshift=-1em]resident.west);
\draw[directed consistency relation] ([yshift=-1em]resident.east) -- node[pos=0, above right] {$r$} node[below] {$\consistencyrelation{CR}{2}$} node[pos=1, above left] {$e$} ([yshift=-1em]employee.west);

\node[consistency related element, below=1.5em of resident.south, anchor=north, inner sep=0em] {
$\begin{aligned}
    \consistencyrelation{CR}{1} =\; & \setted{\tupled{p,r} \mid r.name = \mathvariable{p.firstname} + "\text{\textvisiblespace}" + \mathvariable{p.lastname}}\\
    \consistencyrelation{CR}{2} =\; & \setted{\tupled{r,e} \mid \mathvariable{r.name} = \mathvariable{e.name}}
\end{aligned}$
};

% CONSISTENCY PROCESS
\draw[consistency process] ([yshift=1em,xshift=-2em]person.west) -- node[above] {1.} ([yshift=1em]person.west);
\draw[consistency process] ([yshift=1em]person.east) -- node[above] {2.} ([yshift=1em]resident.west);
\draw[consistency process] ([yshift=1em]resident.east) -- node[above] {3.} ([yshift=1em]employee.west);

\end{tikzpicture}
    %\includegraphics[width=\columnwidth]{figures/tree_construction_example.png}
    \caption[Construction of a model tuple for a consistency relation tree]{An example for constructing a model with the condition element of $\consistencyrelation{CR}{1}$ containing the person named "Alice Do" for a consistency relation tree according to the consistency relations in \autoref{fig:compatibility:tree_example}. Taken from \owncite{klare2020compatibility-report}.}
    \label{fig:compatibility:tree_construction_example}
\end{figure}

\begin{proof}
    We prove the statement by constructing a tuple of models for each condition element in the left condition of each consistency relation that contains the condition element and is consistent, i.e., that fulfills the compatibility definition.
    The basic idea is that because $\consistencyrelationset{CR}$ is a consistency relation tree, we can simply add necessary elements to get a model tuple that is consistent to all consistency relations, by %iterating through the tree in terms of 
    following an order of relations according to \autoref{lemma:treehassequence}.
    Thus, we explain an induction for constructing such a model tuple, which is also exemplified for a simple scenario in \autoref{fig:compatibility:tree_construction_example}, based on the relations in the consistency relation tree in \autoref{fig:compatibility:tree_example}.
    %First, we assume that $\consistencyrelationset{CR}$ is connected. Otherwise the following construction can be applied to the independent partitions of $\consistencyrelationset{CR}$, as their combination is compatible if each of them is compatible according to \autoref{lemma:independencecompatibility}.
    
    \paragraph{Base case:}
    Take any $\consistencyrelation{CR}{} \in \consistencyrelationset{CR}$ and any of its left side condition elements $\conditionelement{c}{l} = \tupled{\object{o}{l,1}, \dots, \object{o}{l,m}} \in \condition{c}{l, \consistencyrelation{CR}{}}$.
    %First, we construct a model set that contains $\conditionelement{c}{l}$ and is consistent to only $\consistencyrelation{CR}{}$.
    %To achieve that, 
    Select any $\conditionelement{c}{r} = \tupled{\object{o}{r,1}, \dots, \object{o}{r,n}} \in \condition{c}{r, \consistencyrelation{CR}{}}$, such that $\conditionelement{c}{l}$ and $\conditionelement{c}{r}$ constitute a consistency relation pair $\tupled{\conditionelement{c}{l}, \conditionelement{c}{r,}} \in \consistencyrelation{CR}{}$.
    %Now select any $\modelset{m} = \object{o'}{1}, \dots, \object{o'}{m+n}$, such that $\forall i \in \setted{1, \dots, n}: \object{o}{l,i} \subseteq \object{o'}{i}$ and $\forall i \in \setted{1, \dots, m}: \object{o}{r,i} \subseteq \object{o'}{n+i}$.
    Now construct the model tuple $\modeltuple{m}$ that contains only $\object{o}{l,1}, \dots, \object{o}{l,m}$ and $\object{o}{r,1}, \dots, \object{o}{r,n}$. % contains $\conditionelement{c}{l}$ and is consistent to $\consistencyrelation{CR}$ by construction.
    In consequence, we have a minimal model tuple $\modeltuple{m}$, such that $\modeltuple{m} \containsmath \conditionelement{c}{l}$ and $\modeltuple{m} \consistenttomath \consistencyrelation{CR}{}$.
    Additionally, $\modeltuple{m}$ is consistent to $\consistencyrelation{CR^T}{}$ due to symmetry of $\consistencyrelation{CR}{}$ and $\consistencyrelation{CR^T}{}$: It is $\conditionelement{c}{r} \in \condition{c}{l,\consistencyrelation{CR^T}{}}$ and $\tupled{\conditionelement{c}{r}, \conditionelement{c}{l}} \in \consistencyrelation{CR^T}{}$ and no other condition element of $\condition{c}{l,\consistencyrelation{CR^T}{}}$ is contained in $\modeltuple{m}$ by construction, thus $\modeltuple{m}$ is consistent to $\consistencyrelation{CR^T}{}$.
    In consequence, we know that for all $\consistencyrelation{CR}{} \in \consistencyrelationset{CR}$, $\setted{\consistencyrelation{CR}{}, \consistencyrelation{CR^T}{}}$ is compatible. 
    Considering the example in $\autoref{fig:compatibility:tree_construction_example}$, for the selection of any person as a condition element in $\condition{c}{l,\consistencyrelation{CR}{1}}$ (1), we select a resident in $\condition{c}{r,\consistencyrelation{CR}{1}}$ with the same name (2), such that the elements are consistent to $\consistencyrelation{CR}{1}$.
    
    \paragraph{Induction assumption:} 
    According to \autoref{lemma:treehassequence}, there is a sequence $\tupled{\consistencyrelation{CR}{1}, \dots, \consistencyrelation{CR}{k}}$ of the relations in $\consistencyrelationset{CR}$ with $\consistencyrelation{CR}{1} = \consistencyrelation{CR}{}$, such that:
    \begin{align*}
        \formulaskip &
        \forall s \in \setted{1, \dots, k-1} : \forall t \in \setted{i+1, \dots, k} : \\
        & \formulaskip
        \classtuple{C}{r,\consistencyrelation{CR'}{s}} \cap \classtuple{C}{r,\consistencyrelation{CR'}{t}} = \emptyset 
        \land
        \classtuple{C}{l,\consistencyrelation{CR'}{s}} \cap 
        \classtuple{C}{r,\consistencyrelation{CR'}{t}} = \emptyset
    \end{align*}
    Considering the example in \autoref{fig:compatibility:tree_construction_example}, such a sequence would be $\tupled{\consistencyrelation{CR}{1}, \consistencyrelation{CR}{2}}$, because the elements in the right condition of $\consistencyrelation{CR}{2}$ are not represented in the left condition of $\consistencyrelation{CR}{1}$.
    If, in general, we know that $\setted{\consistencyrelation{CR}{1}, \consistencyrelation{CR^T}{1} \dots, \consistencyrelation{CR}{i}, \consistencyrelation{CR^T}{i}}\; (i < k)$ is compatible, for every $\conditionelement{c}{l} \in \condition{C}{l,\consistencyrelation{CR}{}}$, we can find a model tuple $\modeltuple{m}$ that contains $\conditionelement{c}{l}$ and is consistent to $\setted{\consistencyrelation{CR}{1}, \consistencyrelation{CR^T}{1}, \dots, \consistencyrelation{CR}{i}, \consistencyrelation{CR^T}{i}}$ by definition.
    We can especially create a minimal model according to our construction for the base case and the following inductive completion.
    
    \paragraph{Induction step:}
    Consider $\consistencyrelation{CR}{i+1}$.
    There is at most one condition element $\conditionelement{c}{l} \in \condition{c}{l, \consistencyrelation{CR}{i+1}}$ with $\modeltuple{m} \containsmath \conditionelement{c}{l}$.
    If there were at least two condition elements $\conditionelement{c}{l}, \conditionelement{c'}{l} \in \condition{c}{l,\consistencyrelation{CR}{i+1}}$, both contained in $\modeltuple{m}$, then by construction there is a consistency relation $\consistencyrelation{CR}{s}\; (s < i+1)$ with $\conditionelement{c}{l},\conditionelement{c'}{l} \in \condition{c}{r,\consistencyrelation{CR}{j}}$.
    Let us assume there were two consistency relations $\consistencyrelation{CR}{s}, \consistencyrelation{CR}{t}$, each containing one of the condition elements in the right condition, then there would be non-empty concatenations $\consistencyrelation{CR}{} \concat \dots \concat \consistencyrelation{CR}{s}$ and $\consistencyrelation{CR'}{} \concat \dots \concat \consistencyrelation{CR}{t}$ with $\classtuple{C}{l,\consistencyrelation{CR}{}} \cap \classtuple{C}{l,\consistencyrelation{CR'}{}} \neq \emptyset$, because we started the construction with elements from the left condition of $\consistencyrelation{CR}{}$, so every element is contained because of a relation to those elements, and with $\classtuple{C}{r,\consistencyrelation{CR}{s}} \cap \classtuple{C}{r,\consistencyrelation{CR}{t}} \neq \emptyset$, because both condition elements $\conditionelement{c}{l}$ and $\conditionelement{c'}{l}$ instantiate the same classes, as they are both contained in $\condition{c}{l,\consistencyrelation{CR}{i+1}}$.
    This would violate \autoref{def:relationtree} for a consistency relation tree, thus there is only one such consistency relation $\consistencyrelation{CR}{s}$.
    Consequently, there must be two condition elements $\conditionelement{c}{ll}, \conditionelement{c'}{ll} \in \condition{c}{l,\consistencyrelation{CR}{s}}$ with $\tupled{\conditionelement{c}{ll},\conditionelement{c}{l}}, \tupled{\conditionelement{c'}{ll},\conditionelement{c'}{l}} \in \consistencyrelation{CR}{s}$, because per construction $\modeltuple{m}$ was consistent to $\consistencyrelation{CR}{s}$ and so there must be a witness structure with a unique mapping between condition elements contained in $\modeltuple{m}$.
    The above argument can be applied inductively until we finally find that there must be two condition elements $\conditionelement{c}{lll},\conditionelement{c'}{lll} \in \condition{c}{l,\consistencyrelation{CR}{}}$, which are contained in $\modeltuple{m}$.
    This is not true by construction, as we started with only one element from $\condition{c}{l,\consistencyrelation{CR}{}}$, so there is only one such condition element $\conditionelement{c}{l} \in \condition{c}{l,\consistencyrelation{CR}{i+1}}$ with $\modeltuple{m} \containsmath \conditionelement{c}{l}$.
    
    For this condition element $\conditionelement{c}{l} \in \condition{c}{l,\consistencyrelation{CR}{i+1}}$, select an arbitrary $\conditionelement{c}{r} = \tupled{\object{o}{1}, \dots, \object{o}{s}} \in \condition{c}{r, \consistencyrelation{CR}{i+1}}$, such that $\tupled{\conditionelement{c}{l}, \conditionelement{c}{r}} \in \consistencyrelation{CR}{i+1}$.
    %Now select any objects $\object{o'}{1}, \dots, \object{o'}{s}$, such that $\forall i \in \setted{1, \dots, s}: \object{o}{i} \subseteq \object{o'}{i}$ and create a model set $\modelset{m'}$ by adding the objects $\object{o'}{1}, \dots, \object{o'}{s}$ to $\modelset{m}$.
    Now create a model tuple $\modeltuple{m'}$ by adding the objects $\object{o}{1}, \dots, \object{o}{s}$ to $\modeltuple{m}$.
    Since $\conditionelement{c}{l}$ is the only of the left condition elements of $\consistencyrelation{CR}{i+1}$ that $\modeltuple{m}$ contains, model tuple $\modeltuple{m'}$ is consistent to $\consistencyrelation{CR}{i+1}$ per construction.
    $\modeltuple{m'}$ is also consistent to $\consistencyrelation{CR^T}{i+1}$, because due to the symmetry of $\consistencyrelation{CR}{i+1}$ and $\consistencyrelation{CR^T}{i+1}$, it is $\conditionelement{c}{r} \in \condition{c}{l,\consistencyrelation{CR^T}{i+1}}$ and due to $\tupled{\conditionelement{c}{r}, \conditionelement{c}{l}} \in \consistencyrelation{CR^T}{i+1}$, a consistent corresponding element exists in $\modeltuple{m'}$. 
    Furthermore, there cannot be any other $\conditionelement{c'}{} \in \condition{c}{l,\consistencyrelation{CR^T}{i+1}}$ with $\modeltuple{m'} \containsmath \conditionelement{c'}{}$, because otherwise there would have been another consistency relation $\consistencyrelation{CR'}{}$ that required the creation of $\conditionelement{c'}{}$, which means that there are two concatenations of consistency relations $\consistencyrelation{CR}{} \concat \dots \concat \consistencyrelation{CR'}{}$ and $\consistencyrelation{CR}{} \concat \dots \concat \consistencyrelation{CR}{i+1}$ that both relate instances of the same classes, which contradicts \autoref{def:relationtree} for a consistency relation tree.
    
    Additionally, due to \autoref{lemma:treehassequence}, for all  $\consistencyrelation{CR}{s}\; (s < i+1)$, we know that $\classtuple{C}{l,\consistencyrelation{CR}{s}} \cap \classtuple{C}{r,\consistencyrelation{CR}{i+1}} = \emptyset$. 
    Since the newly added elements $\conditionelement{c}{r}$ are part of $\condition{c}{r,\consistencyrelation{CR}{i+1}}$, these elements cannot match the left conditions of any of the consistency relations $\consistencyrelation{CR}{s}\; (s < i+1)$.
    So $\modeltuple{m'}$ is still consistent to all $\consistencyrelation{CR}{s}\; (s < i+1)$.
    Finally, due to \autoref{lemma:treehassequence}, for all  $\consistencyrelation{CR}{s}\; (s < i+1)$, we know that $\classtuple{C}{r,\consistencyrelation{CR}{s}} \cap \classtuple{C}{r,\consistencyrelation{CR}{i+1}} = \emptyset$.
    Again, since the newly added elements $\conditionelement{c}{r}$ are part of $\condition{c}{r,\consistencyrelation{CR}{i+1}}$, these elements cannot match the left conditions of any of the consistency relations $\consistencyrelation{CR^T}{s}\; (s < i+1)$.
    So $\modeltuple{m'}$ is still consistent to all $\consistencyrelation{CR^T}{s}\; (s < i+1)$.
    %Additionally, $\modelset{m'} \consistenttomath \consistencyrelation{CR^T}{i+1}$, because the only element of $\condition{c}{l,\consistencyrelation{CR^T}{i+1}}$ is $\conditionelement{c}{l}$. 
    %Otherwise another consistency relation would have required such an element to be updated, such there was another sequence of consistency relations next to $\consistencyrelation{CR}{}, \dots, \consistencyrelation{CR}{i+1}$ that created elements of $\classtuple{C}{r,\consistencyrelation{CR}{i+1}}$, which is not possible by \autoref{def:relationtree} for consistency relation trees.
    In consequence, we know that $\modeltuple{m'} \consistenttomath \setted{\consistencyrelation{CR}{1}, \consistencyrelation{CR^T}{1} \dots, \consistencyrelation{CR}{i+1}, \consistencyrelation{CR^T}{i+1}}$.
    
    Considering the example in \autoref{fig:compatibility:tree_construction_example}, we would select $\consistencyrelation{CR}{2}$ and add for the resident, which is in the left condition elements of $\consistencyrelation{CR}{2}$, an appropriate employee to make the model tuple consistent to $\consistencyrelation{CR}{2}$ (3).
    % If adding those elements could violate consistency to $\consistencyrelation{CR}{1}$, any element considered by $\consistencyrelation{CR}{1}$, in this case a resident, would have to be created, which is not possible, as the relations would not form a tree anymore.
    
    \paragraph{Conclusion}
    Taking the base case for $\consistencyrelation{CR}{}$ and the induction step for $\consistencyrelation{CR}{i+1}$, we have inductively shown that 
    \begin{align*}
        \formulaskip 
        \modeltuple{m'} \consistenttomath \setted{\consistencyrelation{CR}{1}, \consistencyrelation{CR^T}{1} \dots, \consistencyrelation{CR}{k}, \consistencyrelation{CR^T}{k}} = \consistencyrelationset{CR}
    \end{align*}
    Since the construction is valid for each condition element in every consistency relation in $\consistencyrelationset{CR}$, we know that a consistency relation tree $\consistencyrelationset{CR}$ is compatible.
    %
    % Now select an arbitrary $\consistencyrelation{CR'}{} \in \consistencyrelationset{CR} \setminus \setted{\consistencyrelation{CR}{}}$ with $\consistencyrelation{CR}{} \concat \consistencyrelation{CR'}{} \neq \emptyset$.
    % Such a relation must exist, because otherwise $\consistencyrelationset{CR}$ would not be connected.
    % For all condition elements $\conditionelement{c}{l} \in \condition{c}{l, \consistencyrelation{CR'}{}}$ with $\modelset{m} \containsmath \conditionelement{c}{l}$, select an arbitrary  $\conditionelement{c}{r} = \tupled{\object{o}{1}, \dots, \object{o}{s}} \in \condition{c}{r, \consistencyrelation{CR'}{}}$, such that $\tupled{\condition{c}{l, \consistencyrelation{CR'}{}}, \condition{c}{r, \consistencyrelation{CR'}{}}} \in \consistencyrelation{CR'}{}$.
    % Now select any objects $\object{o'}{1}, \dots, \object{o'}{s}$, such that $\forall i \in \setted{1, \dots, s}: \object{o}{i} \subseteq \object{o'}{i}$ and create a model set $\modelset{m'}$ by adding the objects $\object{o'}{1}, \dots, \object{o'}{s}$ to $\modelset{m}$.
    % Per construction, model set $\modelset{m'}$ is consistent to $\consistencyrelation{CR'}{}$.
    % It is also still consistent to $\consistencyrelation{CR}{}$, because if any of the newly added objects $\object{o'}{1}, \dots, \object{o'}{s}$ would violate consistency to $\consistencyrelation{CR}{}$, then there must be an overlap in the classes $\classtuple{C}{r,\consistencyrelation{CR'}{}}$ of $\consistencyrelation{CR'}{}$ and the classes $\classtuple{C}{l,\consistencyrelation{CR}{}}$ of $\consistencyrelation{CR}{}$.
    % Since $\consistencyrelation{CR}{} \concat \consistencyrelation{CR'}{} \neq \emptyset$ by construction, there is also an overlap in the class tuples $\classtuple{C}{l,\consistencyrelation{CR'}{}}$ and $\classtuple{C}{r,\consistencyrelation{CR}{}}$, so that there is a cycle in the classes, violating the definition of a consistency relation tree.
    % Considering the example in \autoref{fig:compatibility:formal:tree_construction_example}, we select $\consistencyrelation{CR}{2}$ and add an appropriate person to make the model set consistent to $\consistencyrelation{CR}{2}$ (3).
    % If adding those elements could violate consistency to $\consistencyrelation{CR}{1}$, any element considered by $\consistencyrelation{CR}{1}$, in this case a resident, would have to be created, which is not possible, as the relations would not form a tree anymore.
    %
    % Now, having added elements to make the model set $\modelset{m'}$ consistent to $\consistencyrelation{CR}{1}, \dots, \consistencyrelation{CR}{k]}$ inductively that way, select any $\consistencyrelation{CR''}{} \in \consistencyrelationset{CR} \setminus \setted{\consistencyrelation{CR}{1}, \dots, \consistencyrelation{CR}{k}}$ with $\exists i \in \setted{1, \dots, k} : \consistencyrelation{CR}{i} \concat \consistencyrelation{CR''}{} \neq \emptyset$.
    % Like before, this relation has to exist because $\consistencyrelationset{CR}$ is connected.
    % In the same way like for $\consistencyrelation{CR'}{}$, add elements to the model set $\modelset{m'}$ to form a new model set $\modelset{m''}$ that is consistent to $\consistencyrelation{CR''}{}$.
    % If these added elements would lead to $\modelset{m''}$ not being consistent to any $\consistencyrelation{CR}{1}, \dots, \consistencyrelation{CR}{k}$, due to the same argumentation as for $\consistencyrelation{CR'}{}$, there would be a sequence of consistency relations that relates one class with itself, thus $\consistencyrelationset{CR}$ would not fulfill the definition of consistency relation tree.
    % Thus, $\modelset{m''}$ is consistent to $\setted{\consistencyrelation{CR}{1}, \dots, \consistencyrelation{CR}{k}, \consistencyrelation{CR''}{}}$.
    %
    % Applying that argument inductively for all consistency relations in $\consistencyrelationset{CR}$, we are able to create a model set $\modelset{m}$ that is consistent to $\consistencyrelationset{CR}$ and that contains the initially selected condition element $\conditionelement{c}{l}$. 
    % Since the argument holds for any such condition element of any of the consistency relations, $\consistencyrelationset{CR}$ fulfills the definition of compatibility.
\end{proof}

%%
%% Summary: Independent trees are compatible
%%
Summarizing, \autoref{theorem:independencecompatibility} and \autoref{theorem:treecompatibility} have shown that consistency relation sets fulfilling a special notion of trees are compatible and that combining compatible independent sets of relations is compatibility-preserving.
In consequence, having a consistency relation set that consists of independent subsets that are consistency relation trees, this set of relations is inherently compatible.
An approach that evaluates whether a given set of consistency relations fulfills \autoref{def:independence} and \autoref{def:relationtree} for independence and trees can be used to prove compatibility of those relations.

%%
%% Transition: Actual sets may generally not be trees
%%
However, consistency relations fulfill such a structure only in specific cases.
In general, like in our motivational example in \autoref{fig:compatibility:three_persons_example_extended}, there may be different consistency relations putting the same elements into relation, such that the definition for consistency relation trees is not fulfilled.
In the following, we discuss how to find a consistency relation tree that is equivalent to a given set of consistency relations, such that this equivalence witnesses compatibility.



\subsection{Redundancy of Consistency Relations}
\label{chap:compatibility:formal_approach:redundancy}

%\todoDiss{Add a definition for a \emph{compatibility-preserving consistency relation} that states that is preserved compatibility and use that for clarifying the following lemmas and theorems.}

%%
%% Problem: Not having a compatible structure, compatibility is unclear
%%
We have introduced specific structures of consistency relations that are inherently compatible.
If a given set of consistency relations does not represent one of those structures, especially because there are multiple consistency relations putting the same classes into relation, it is unclear whether such a set is compatible.

%%
%% Idea: Find and virtually remove redundant relations
%%
In the following, we present an approach to reduce a set of consistency relations to a structure of independent consistency relation trees.
The essential idea is to find relations within the set, which do not change compatibility of the consistency relation set whether or not they are contained in it.
An approach that finds such relations and---virtually---removes them from the set until the remaining relations form a set of independent consistency relation trees, proves compatibility of the given set of relations.
We first define the term of a \emph{compatibility-preserving} relation.

\begin{definition}
    \label{def:compatibilitypreserving}
    Let $\consistencyrelationset{CR}$ be a compatible set of consistency relations and let $\consistencyrelation{CR}{}$ be a consistency relation. We say that:
    \begin{align*}
        \formulaskip &
        \consistencyrelation{CR}{} \compatibilitypreservingtomath \consistencyrelationset{CR} \equivalentperdefinition \\
        & \formulaskip
        \consistencyrelationset{CR} \cup \setted{\consistencyrelation{CR}{}} \compatiblemath
    \end{align*}
\end{definition}

To be able to find such a compatibility-preserving relation, we introduce the notion of \emph{redundant} relations and prove the property of being compatibility preserving.
Informally speaking, a relation is redundant if it is expressed transitively across others, i.e., if it does not restrict or relax consistency compared to a combination of other relations.
We precisely specify a notion of redundancy in the following.

% \begin{itemize}
%     \item There may also be cycles in the fine-grained relations, such that the relation graph induced by the specification cannot witness compatibility.
%     \item If it is possible to find a set of trees that is equivalent to the given graph (\formalize{what equivalent means here!}), this serves as a witness for compatibility.
%     \item Finding such an equivalent representation can be achieved by (virtually) removing relations that have no impact on the valid instances (i.e. do not reduce the degree of consistency) (\formalize{that with the definition})
%     \item A relation is redundant if it is transitively expressed across the others, i.e. if it does not restrict the valid instances of two metamodels that are considered consistent in addition to those allowed by all other relations (\formalize{with extensional definitions of consistency}).
%     \item Virtually removing such relations leads to an equivalent representation and if that representation finally forms a tree, we have a witness for compatibility.
%     \item We explain this in detail in \autoref{sec:redundancies} and discuss how theorem proving can be used to find such redundant relations.
% \end{itemize}

\begin{definition}[Redundant Consistency Relation]
\label{def:redundancy}
    Let $\consistencyrelationset{CR}$ be a set of consistency relations for a tuple of metamodels $\metamodeltuple{M}$. %  = \setted{\metamodel{M}{1}, \dots, \metamodel{M}{k}}$.
    For a consistency relation $\consistencyrelation{CR}{} \in \consistencyrelationset{CR}$, we say that:
    \begin{align*}
        \formulaskip &
        \consistencyrelation{CR}{} \redundantinmath \consistencyrelationset{CR} \equivalentperdefinition\\
        & \formulaskip 
        \exists \consistencyrelation{CR'}{} \in \transitiveclosure{(\consistencyrelationset{CR} \setminus \setted{\consistencyrelation{CR}{}})} : 
        \forall \modeltuple{m} \in \metamodeltupleinstanceset{M} :\\
        & \formulaskip\formulaskip
        \modeltuple{m} \consistenttomath \consistencyrelation{CR'}{} \Rightarrow \modeltuple{m} \consistenttomath \consistencyrelation{CR}{}
    \end{align*}
    % \begin{align*}
    %     \formulaskip &
    %     \consistencyrelation{CR}{} \in \consistencyrelationset{CR} \mathtext{is redundant in} \consistencyrelationset{CR} \equivalentperdefinition\\
    %     & \formulaskip 
    %     \consistencyrelationset{CR} \mathtext{equivalent to} \consistencyrelationset{CR} \setminus \setted{\consistencyrelation{CR}{}}
    % \end{align*}
    % \begin{align*}
    %     \formulaskip &
    %     \consistencyrelation{CR}{} \in \consistencyrelationset{CR} \mathtext{is redundant in} \consistencyrelationset{CR} \equivalentperdefinition\\
    %     %& \formulaskip
    %     %\forall \modelset{m} = \setted{\model{m}{1}, \dots, \model{m}{k}} \mid \model{m}{i} \in \metamodelinstances{\metamodel{M}{i}} : \\
    %     & \formulaskip 
    %     \exists \consistencyrelation{CR'}{} \in \transitiveclosure{\consistencyrelationset{CR}} : \consistencyrelation{CR'}{} \mathtext{overlapping with} \consistencyrelation{CR}{} \\
    %     & \formulaskip
    %     \land \forall \consistencyrelation{CR'}{} \in \transitiveclosure{\consistencyrelationset{CR}} \mid \consistencyrelation{CR'}{} \mathtext{overlapping with} \consistencyrelation{CR}{} :\\
    %     & \formulaskip\formulaskip
    %     \bigl(\forall \tupled{\conditionelement{c}{l}, \conditionelement{c}{r}} \in \consistencyrelation{CR}{} : \exists \tupled{\conditionelement{c'}{l}, \conditionelement{c'}{r}} \in \consistencyrelation{CR'}{} : \forall \modelset{m} \in \metamodelinstances{\metamodelset{M}} : \\
    %     & \formulaskip\formulaskip\formulaskip\modelset{m} \mathtext{contains} \tupled{\conditionelement{c}{l}, \conditionelement{c}{r}} \equivalent \modelset{m} \mathtext{contains} \tupled{\conditionelement{c'}{l}, \conditionelement{c'}{r}} \\
    %     & \formulaskip\formulaskip
    %     \land \forall \tupled{\conditionelement{c'}{l}, \conditionelement{c'}{r}} \in \consistencyrelation{CR'}{} : \exists \tupled{\conditionelement{c}{l}, \conditionelement{c}{r}} \in \consistencyrelation{CR}{} : \forall \modelset{m} \in \metamodelinstances{\metamodelset{M}} : \\
    %     & \formulaskip\formulaskip\formulaskip\modelset{m} \mathtext{contains} \tupled{\conditionelement{c}{l}, \conditionelement{c}{r}} \equivalent \modelset{m} \mathtext{contains} \tupled{\conditionelement{c'}{l}, \conditionelement{c'}{r}} \bigr)
    % \end{align*}
\end{definition}

\todoLater{Add examples for redundancy! How do the elements of the redundant relation have to be related to the ones in $\consistencyrelation{CR'}{}$?}
\todoLater{Can we define an even more general notion of redundancy, not stating about the relation to a single consistency relation but the set of consistency relation, abstracting the implication to consistency to the whole set of relations?}
The definition of redundancy of a consistency relation $\consistencyrelation{CR}{}$ ensures that there is another consistency relation, possibly transitively expressed across others, such that if a model is consistent to that other relation, it is also consistent to $\consistencyrelation{CR}{}$.
This means that there are no model tuples that are considered inconsistent to $\consistencyrelation{CR}{}$, but not to another relation, thus $\consistencyrelation{CR}{}$ does not restrict consistency.
Actually, the definition of redundancy implies that the set of consistency relations with and without the redundant one are equivalent according to \autoref{def:equivalence}, thus both consider the same model tuples as consistent.

% The definition of redundancy ensures that the redundant relation does not provide any relaxation (first equivalence) or restriction (second equivalence) regarding existing consistency relations and that there is at least one overlapping consistency relation, as otherwise the relation will always restrict consistent models.
% Intuitively, redundancy could also be defined by requiring equivalence of the set of consistency relations with and without the redundant relation. In that case, exactly the same models would be considered consistent with and without the redundant relation.
% However, we want to use the redundancy definition to make statements about compatibility of sets of consistency relations, which requires this more restricted notion of redundancy.
% Actually, the definition of redundancy always implies equivalence.
\todoLater{Explain that we do not require equality of elements in CR and CR' because, e.g., CR might only related names, whereas CR' related names and addresses, thus we only require that there are elements that are co-indicating consistency.}

\begin{lemma} \label{lemma:redundancyimpliesequivalence}
    Let $\consistencyrelation{CR}{} \in \consistencyrelationset{CR}$ be a redundant consistency relation in a relation set $\consistencyrelationset{CR}$. %, according to \autoref{def:redundancy}.
    Then $\consistencyrelationset{CR}$ is equivalent to $\consistencyrelationset{CR} \setminus \setted{\consistencyrelation{CR}{}}$. %, according to \autoref{def:equivalence}.
\end{lemma}

\begin{proof}
    Like discussed in \autoref{lemma:consistencytransitiveclosure}, adding a consistency relation to a set of consistency relations can never lead to a relaxation of consistency, i.e., models becoming consistent that were not considered consistent before. This is a direct consequence of \autoref{def:consistency} for consistency, which requires models be consistent to all consistency relations in a set to be considered consistent, thus restricting the set of consistent model tuples by adding further consistency relations.
    In consequence, it holds that:
    \begin{align*}
        \formulaskip
        \modeltuple{m} \consistenttomath \consistencyrelationset{CR} \Rightarrow 
        \modeltuple{m} \consistenttomath \consistencyrelationset{CR} \setminus \setted{\consistencyrelation{CR}{}}
    \end{align*}
    Additionally, a direct consequence of \autoref{def:redundancy} for redundancy is that a redundant consistency relation does not restrict consistency, as it considers all models to be consistent that are also considered consistent to another consistency relation in the transitive closure of the consistency relation set. Thus, all models that are considered consistent to the transitive closure of $\consistencyrelationset{CR} \setminus \setted{\consistencyrelation{CR}{}}$ are also consistent to $\consistencyrelation{CR}{}$ and thus to all relations in $\consistencyrelationset{CR}$:
    \begin{align*}
        \formulaskip
        \modeltuple{m} \consistenttomath \transitiveclosure{(\consistencyrelationset{CR} \setminus \setted{\consistencyrelation{CR}{}})} \Rightarrow 
        \modeltuple{m} \consistenttomath \consistencyrelationset{CR}
    \end{align*}
    According to \autoref{lemma:consistencytransitiveclosure}, each tuple of models that is consistent to a consistency relation set is also consistent to its transitive closure an vice versa.
    In consequence, the previous implication is also true for $\consistencyrelationset{CR} \setminus \setted{\consistencyrelation{CR}{}}$ rather than $\transitiveclosure{(\consistencyrelationset{CR} \setminus \setted{\consistencyrelation{CR}{}})}$.
    Summarizing, $\consistencyrelationset{CR}$ and $\consistencyrelationset{CR} \setminus \setted{\consistencyrelation{CR}{}}$ are equivalent.
\end{proof}

\todoLater{Possibly add that lemma}
% \begin{lemma}
%     Let $\consistencyrelationset{CR}$ be a set of consistency relation and let $\consistencyrelation{CR}{}$ be redundant in $\consistencyrelationset{CR}$.
%     Then it holds that:
%     \begin{align*}
%         \formulaskip &
%         \exists \consistencyrelation{CR'}{} \in \transitiveclosure{\consistencyrelationset{CR}} \setminus \setted{\consistencyrelation{CR}{}} : \consistencyrelation{CR}{} \mathtext{related to} \consistencyrelation{CR'}{}\\
%         &
%         \lor \forall \modelset{m} \in \metamodelinstances{\metamodelset{M}} : \modelset{m} \mathtext{consistent to} \consistencyrelation{CR}{}
%     \end{align*}
% \end{lemma}

% \begin{proof}
%     tba \todoHeiko{Add the proof}
% \end{proof}

\begin{figure}
    \centering
    \newcommand{\hdistance}{19em}
\newcommand{\vdistance}{1.5em}
\newcommand{\classwidth}{6em}
\renewcommand{\sameheight}{\vphantom{yR}}

\begin{tikzpicture}

% Resident
\umlclassvarwidth{resident}{}{Resident\sameheight}{
name
}{\classwidth}

% Employee
\umlclassvarwidth[, right=\hdistance of resident.north, anchor=north]{employee}{}{Employee\sameheight}{
name
}{\classwidth}

% Location
\umlclassvarwidth[, below=\vdistance of resident.south, anchor=north]{location}{}{Location\sameheight}{
street
}{\classwidth}

% Address
\umlclassvarwidth[, below=\vdistance of employee.south, anchor=north]{address}{}{Address\sameheight}{
street
}{\classwidth}

% CONSISTENCY RELATIONS
\draw[directed consistency relation] ([yshift=1em]resident.east) -- node[pos=0, above right] {$r$} node[pos=0.5, above] {$\consistencyrelation{CR}{1}$} node[pos=1, above left] {$e$} ([yshift=1em]employee.west);
\draw[directed consistency relation, -] ([yshift=1em]$(employee.west)!0.8!(employee.west-|resident.east)$) |- node[pos=1, above right] {$l$} (location.east);
\draw[directed consistency relation 2] ([yshift=-1em]resident.east) -- node[pos=0, above right] {$r$} node[pos=0.5, below] {$\consistencyrelation{CR}{2}$} node[pos=1, above left] {$e$} ([yshift=-1em]employee.west);
\draw[directed consistency relation 2] ([yshift=-1em]$(employee.west)!0.2!(employee.west-|resident.east)$) |- node[pos=1, above left] {$a$} (address.west);
\filldraw[consistency related element] ([yshift=1em]$(employee.west)!0.8!(employee.west-|resident.east)$) circle (0.15em);
\filldraw[consistency related element] ([yshift=-1em]$(employee.west)!0.2!(employee.west-|resident.east)$) circle (0.15em);

\node[consistency related element, anchor=north, inner sep=0em] (relation1) at ([yshift=-1em]$(location.south east)!0.5!(address.south west)$) {
$\begin{aligned}
    \consistencyrelation{CR}{1} =\; & \setted{\tupled{(r,l),e} \mid \mathvariable{r.name} \neq "" %\\
    %&
    \land (\mathvariable{r.name} = \mathvariable{e.name} \lor \mathvariable{r.name} = \mathvariable{e.name.toLower})} \\[0.3em]
    \consistencyrelation{CR}{2} =\; & \setted{\tupled{r,(e,a)} \mid \mathvariable{r.name} = \mathvariable{e.name} \land \mathvariable{a.street} \neq ""}
\end{aligned}$
};

\end{tikzpicture}
    %\includegraphics[width=\columnwidth]{figures/redundancy_relation_extremes.png}
    \caption[Redundant consistency relation]{Redundant consistency relation $\consistencyrelation{CR}{1}$ in $\setted{\consistencyrelation{CR}{1}, \consistencyrelation{CR}{2}}$. Taken from \owncite{klare2020compatibility-report}.}
    \label{fig:compatibility:redundancyrelationextremes}
\end{figure}

In general, to consider a consistency relation redundant in %a set of consistency relation
$\consistencyrelationset{CR}$, it has to define equal or weaker requirements for consistency than one of the other relations in $\consistencyrelationset{CR}$.
Informally speaking, such weaker requirements mean that the redundant relation must have weaker conditions, i.e., it must require consistency for less objects and consider the same or more objects consistent to each of the left condition elements. %, i.e., it must have a weaker left condition, and consider the same or more elements consistent to those of the left condition.

\begin{example}
Such weaker consistency requirements are exemplified in \autoref{fig:compatibility:redundancyrelationextremes}, which shows a consistency relation $\consistencyrelation{CR}{1}$ that is redundant in $\setted{\consistencyrelation{CR}{1}, \consistencyrelation{CR}{2}}$.
A redundant consistency relation, such as $\consistencyrelation{CR}{1}$, must have weaker requirements in the left condition, such that it requires consistent elements to exist in less cases.
This means that it may have a larger set of classes that are matched and that there may be less condition elements for which consistency is required.
In case of $\consistencyrelation{CR}{1}$, the left condition contains both a resident and a location, whereas the left condition of $\consistencyrelation{CR}{2}$ only contains residents.
Thus $\consistencyrelation{CR}{1}$ requires consistent elements, i.e., employees, only if a resident and a location exists, whereas $\consistencyrelation{CR}{2}$ requires that already for an existing resident.
Furthermore, the residents for which $\consistencyrelation{CR}{1}$ defines any consistency requirements are a subset of those for which $\consistencyrelation{CR}{2}$ defines consistency requirements, as $\consistencyrelation{CR}{1}$ does not make any statements about residents having an empty name.
Thus, the left condition elements of $\consistencyrelation{CR}{1}$ are a subset of those of $\consistencyrelation{CR}{2}$.
In consequence, if $\consistencyrelation{CR}{1}$ requires consistency for a resident and a location, $\consistencyrelation{CR}{2}$ requires it anyway, because it already defines consistency for the contained resident.

Additionally, a redundant consistency relation, such as $\consistencyrelation{CR}{1}$, must have weaker requirements for the elements at the right side, such that one of the consistent right condition elements is contained anyway because another relation already required them. 
This means that the relation may have a smaller set of classes, of whom instances are required to consider the models consistent, and there may be more condition elements of the right side that are considered consistent with condition elements of the left side to not restrict the elements considered consistent.
In case of $\consistencyrelation{CR}{1}$, it only requires an employee to exist for a resident compared to $\consistencyrelation{CR}{2}$, which also requires a non-empty address to exist. Additionally, $\consistencyrelation{CR}{1}$ does not restrict the employees that are considered consistent to employees compared to $\consistencyrelation{CR}{2}$, as it also considers employees with the same name as consistent, but additionally those having the name of the resident in lowercase.
\end{example}

\todoLater{Add proposition about redundancy properties}
% These informal insights on the properties of a redundant consistency relation can be formalized as follows.

% \begin{proposition}
%     Let $\consistencyrelationset{CR}$ be a set of consistency relations and let $\consistencyrelation{CR}{}$ be a consistency relation. Then it holds that:
%     \begin{align*}
%         \formulaskip &
%         \consistencyrelation{CR}{} \redundantinmath \consistencyrelationset{CR} \cup \setted{\consistencyrelation{CR}{}} \equivalent \\
%         & \formulaskip
%         \exists \consistencyrelation{CR'}{} \in \consistencyrelationset{CR} : %\\
%         %& \formulaskip
%         \classtuple{C}{l,\consistencyrelation{CR'}{}} \subseteq \classtuple{C}{l,\consistencyrelation{CR}{}} \land
%         \classtuple{C}{r,\consistencyrelation{CR}{}} \subseteq \classtuple{C}{r,\consistencyrelation{CR'}{}} \\
%         & \formulaskip
%         \land \forall \conditionelement{c}{l} \in \condition{c}{l,\consistencyrelation{CR}{}} : \exists \conditionelement{c'}{l} \in \condition{c}{l,\consistencyrelation{CR'}{}} : \bigl( 
%         \conditionelement{c'}{l} \subseteq \conditionelement{c}{l} \\
%         & \formulaskip\formulaskip
%         \land \forall \tupled{\conditionelement{c'}{l},\conditionelement{c'}{r}} \in \consistencyrelation{CR'}{} : \exists \tupled{\conditionelement{c}{l},\conditionelement{c}{r}} \in \consistencyrelation{CR}{} : \conditionelement{c}{r} \subseteq \conditionelement{c'}{r} \big)
%     \end{align*}
% \end{proposition}

% \begin{proof}
%     tba
% \end{proof}


\begin{figure}
    \centering
    \newcommand{\hdistance}{19em}
\newcommand{\vdistance}{1.5em}
\newcommand{\classwidth}{6em}
\renewcommand{\sameheight}{\vphantom{yR}}

\begin{tikzpicture}

% Employee
\umlclassvarwidth{employee}{}{Employee\sameheight}{
name
}{\classwidth}

% Person
\umlclassvarwidth[, right=\hdistance of employee.north, anchor=north]{person}{}{Person\sameheight}{
name
}{\classwidth}

% Resident
\umlclassvarwidth[, below=\vdistance of employee.south, anchor=north]{resident}{}{Resident\sameheight}{
name
}{\classwidth}


% CONSISTENCY RELATIONS
\draw[directed consistency relation] ([yshift=1em]employee.east) -- node[pos=0, above right] {$e$} node[pos=0.5, above] {$\consistencyrelation{CR}{1}$} node[pos=1, above left] {$p$} ([yshift=1em]person.west);
\draw[directed consistency relation, -] ([yshift=1em]$(person.west)!0.8!(person.west-|employee.east)$) |- node[pos=1, above right] {$l$} ([yshift=1em]resident.east);
\filldraw[consistency related element] ([yshift=1em]$(person.west)!0.8!(person.west-|employee.east)$) circle (0.15em);

\draw[directed consistency relation 2] ([yshift=-1em]employee.east) -- node[pos=0, above right] {$e$} node[pos=0.5, below] {$\consistencyrelation{CR}{2}$} node[pos=1, above left] {$p$} ([yshift=-1em]person.west);

\draw[directed consistency relation 2] (person.south) |- node[pos=0, below right] {$p$} node[pos=0.8, above] {$\consistencyrelation{CR}{3}$} node[pos=1, above right] {$r$} ([yshift=-1em]resident.east);

\node[consistency related element, below left=1em and 0em of resident.south west, anchor=north west, inner sep=0em] (relation1) {
$\begin{aligned}
    \consistencyrelation{CR}{1} =\; & \setted{\tupled{(e,r),p} \mid \mathvariable{e.name} = \mathvariable{r.name.toUpper} \land \mathvariable{e.name}  = \mathvariable{p.name}} \\[0.3em]
    \consistencyrelation{CR}{2} =\; & \setted{\tupled{e,p} \mid \mathvariable{e.name} = \mathvariable{p.name}}\\[0.3em]
    \consistencyrelation{CR}{3} =\; & \setted{\tupled{p,r} \mid \mathvariable{r.name} = \mathvariable{p.name.toLower}}
\end{aligned}$
};

\end{tikzpicture}
    %\includegraphics[width=\columnwidth]{figures/redundancy_compatibility_counterexample.png}
    \caption[Incompatibility with redundant consistency relation]{A consistency relation $\consistencyrelation{CR}{1}$ being redundant in  $\setted{\consistencyrelation{CR}{1}, \consistencyrelation{CR}{2}, \consistencyrelation{CR}{3}}$, with $\setted{\consistencyrelation{CR}{2}, \consistencyrelation{CR}{3}}$ being compatible and $\setted{\consistencyrelation{CR}{1}, \consistencyrelation{CR}{2},\consistencyrelation{CR}{3}}$ being incompatible. Taken from \owncite{klare2020compatibility-report}.}
    \label{fig:compatibility:redundancy_compatibility_counterexample}
\end{figure}

Our goal is to have a compatibility-preserving notion of redundancy, i.e., adding a redundant relation to a compatible relation set should preserve compatibility.
Unfortunately, our intuitive redundancy definition is not compatibility-preserving. % with our intuitive notion of redundancy, a consistency relation $\consistencyrelation{CR}{}$ that is redundant to a compatible set of consistency relations $\consistencyrelationset{CR}$ does not imply that $\consistencyrelationset{CR} \cup \setted{\consistencyrelation{CR}{}}$ is compatible.

\begin{proposition} \label{prop:redundantnotimpliescompatible}
    Let $\consistencyrelationset{CR}$ be a compatible set of consistency relations and let $\consistencyrelation{CR}{}$ be a consistency relation that is redundant in $\consistencyrelationset{CR} \cup \setted{\consistencyrelation{CR}{}}$.
    Then $\consistencyrelation{CR}{}$ is not necessarily compatibility-preserving, i.e., $\consistencyrelationset{CR} \cup \setted{\consistencyrelation{CR}{}}$ is not necessarily compatible.
    % it holds that:
    % \begin{align*}
    %     \formulaskip &
    %     \consistencyrelationset{CR} \compatiblemath \not\Rightarrow \consistencyrelationset{CR} \cup \setted{\consistencyrelation{CR}{}} \compatiblemath
    % \end{align*}
\end{proposition}

\begin{proof}
We prove the proposition by providing a counterexample. % for the implication.
Consider the example in \autoref{fig:compatibility:redundancy_compatibility_counterexample}. 
$\consistencyrelation{CR}{2}$ relates each employee to a person with the same name and $\consistencyrelation{CR}{3}$ relates each person to a resident with the same name in lowercase.
The consistency relation set $\setted{\consistencyrelation{CR}{2}, \consistencyrelation{CR}{3}}$ is obviously compatible, because for each employee and each person, which constitute the left condition elements of the consistency relations, a consistent model tuple containing the person respectively employee can be created by adding the appropriate person or employee with the same name and a resident with the name in lowercase.
Furthermore, $\consistencyrelation{CR}{1}$ is redundant in $\setted{\consistencyrelation{CR}{1}, \consistencyrelation{CR}{2}, \consistencyrelation{CR}{3}}$ according to \autoref{def:redundancy}, because if a model is consistent to $\consistencyrelation{CR}{2}$ it is also consistent to $\consistencyrelation{CR}{1}$, since $\consistencyrelation{CR}{1}$ also requires persons with the same name as an employee to be contained in a model tuple but in less cases, precisely only those in which the model also contains a resident such that the employee name is the one of the resident in uppercase.

However, $\setted{\consistencyrelation{CR}{1}, \consistencyrelation{CR}{2}, \consistencyrelation{CR}{3}}$ is not compatible.
Intuitively, this is due to the fact that $\consistencyrelation{CR}{1}$ and $\consistencyrelation{CR}{3}$ define an incompatible mapping between the names of residents and persons.
This is also reflected by \autoref{def:compatibility} for compatibility. Take a model with an employee and a resident named $A$. This is a condition element in $\condition{c}{l,\consistencyrelation{CR}{1}}$. 
Consequentially, $\consistencyrelation{CR}{1}$ requires a person $A$ to exist. Furthermore $\consistencyrelation{CR}{3}$ requires a resident with name $a$ to exist.
In consequence, there are two tuples of employees and residents, both with employee $A$ and one with resident $A$ respectively resident $a$ each, for which a consistent person with name $A$ is required by $\consistencyrelation{CR}{1}$.
However, $\consistencyrelation{CR}{1}$ actually forbids to have two residents, one having the lowercase name of the other, because both are condition elements in $\consistencyrelation{CR}{1}$ requiring an appropriate person to occur in a consistent model, but there is only one person that to which both can be mapped, namely the one with the uppercase name, so there is no witness structure with a unique mapping as required by \autoref{def:consistency} for consistency.
This example shows that adding a redundant consistency relation to a compatible set of consistency relations does not lead to a compatible consistency relation set.
\end{proof}



\subsection{Compatibility-Preserving Redundancy}

In consequence of \autoref{prop:redundantnotimpliescompatible}, we need a stronger definition of redundancy that is compatibility-preserving. 
%to be able to derive compatibility for a consistency relation set from adding a redundant relation to an already compatible consistency relation set.
In the example in \autoref{fig:compatibility:redundancy_compatibility_counterexample} showing \autoref{prop:redundantnotimpliescompatible}, we have seen that it is problematic if a redundant consistency relation considers more classes in its left condition than the relation it is redundant to.
Therefore, we restrict the left class tuple.

\begin{definition}[Left-equal Redundant Consistency Relation] \label{def:leftequalredundancy}
    Let $\consistencyrelationset{CR}$ be a set of consistency relations for a metamodel tuple $\metamodeltuple{M}$.
    For a consistency relation $\consistencyrelation{CR}{} \in \consistencyrelationset{CR}$, we say:
    \begin{align*}
        \formulaskip &
        \consistencyrelation{CR}{} \leftequalredundantinmath \consistencyrelationset{CR} \equivalentperdefinition \\
        & \formulaskip 
        \exists \consistencyrelation{CR'}{} \in \transitiveclosure{(\consistencyrelationset{CR} \setminus \setted{\consistencyrelation{CR}{}})} : 
        \forall \modeltuple{m} \in \metamodeltupleinstanceset{M} :\\
        & \formulaskip\formulaskip
        \modeltuple{m} \consistenttomath \consistencyrelation{CR'}{} \Rightarrow \modeltuple{m} \consistenttomath \consistencyrelation{CR}{} \\
        & \formulaskip\formulaskip
        \land \classtuple{C}{l,\consistencyrelation{CR}{}} = \classtuple{C}{l,\consistencyrelation{CR'}{}}
        %\consistencyrelation{CR}{} \redundantinmath \consistencyrelationset{CR} \\
        %& \formulaskip
        %\land \exists \consistencyrelation{CR'}{} \in \transitiveclosure{(\consistencyrelationset{CR} \setminus \setted{\consistencyrelation{CR}{}})} : 
        %\condition{c}{l,\consistencyrelation{CR}{}} \subseteq \condition{c}{l,\consistencyrelation{CR'}{}} 
        %\forall \conditionelement{c}{l} \in \condition{c}{l,\consistencyrelation{CR}{}} :
        %\exists \conditionelement{c'}{l} \in \condition{c}{l, \consistenyrelation{CR'}{}} :
        %\forall \modelset{m} \in \metamodelinstances{\metamodelset{M}} :
    \end{align*}
\end{definition}

%The definition of left-equal redundancy restricts the notion of redundancy to cases in which the left side of the redundant consistency relation $\consistencyrelation{CR}{}$ considers instances of the same classes as another relation in the set of consistency relations.
%As discussed before, redundancy in general allows that the left side of a redundant consistency relation $\consistencyrelation{CR}{}$ considers more classes than another relation in the set of consistency relations that induces consistency of a model set to $\consistencyrelation{CR}{}$, according to the definition of redundancy.

The definition of left-equal redundancy is similar to the redundancy definition but restricts the notion of redundancy to cases in which the left condition of the redundant consistency relation $\consistencyrelation{CR}{}$ considers the same classes than the other relation in the set of consistency relations that induces consistency of a model tuple to $\consistencyrelation{CR}{}$.
As discussed before, redundancy in general allows that the left condition of a redundant consistency relation can consider a superset of those classes. %than another relation in the set of consistency relations that induces consistency of a model set to $\consistencyrelation{CR}{}$, according to the definition of redundancy.

\begin{lemma} \label{lemma:leftequalredundancyimpliesredundancy}
    Let $\consistencyrelation{CR}{}$ be a consistency relation that is left-equal redundant in a set of consistency relations $\consistencyrelationset{CR}$. Then $\consistencyrelation{CR}{}$ is redundant in $\consistencyrelationset{CR}$.
\end{lemma}

\begin{proof}
    Since the definition of left-equal redundancy is equal to the one for redundancy, apart from the additional restriction for the class tuples, redundancy of a left-equal redundant relation is a direct implication of the definition.
\end{proof}


Before showing that left-equal redundancy is compatibility-preserving, we introduce an auxiliary lemma that shows that if a model tuple contains any left condition element of a left-equal redundant relation, i.e., if that redundant relation requires the model tuple to contain corresponding elements for that object tuple to be consistent, there is also another relation that requires corresponding elements for that object tuple.

%In the following, we will show that the notion of left-equal redundancy, in comparison to the weaker general redundancy property, can be used to inductively prove compatibility of a set of consistency relations.

\begin{lemma} \label{lemma:leftequalredundancysubset}
    Let $\consistencyrelation{CR}{}$ be a consistency relation that is left-equal redundant in a set of consistency relations $\consistencyrelationset{CR}$ for a tuple of metamodels $\metamodeltuple{M}$. Then it holds that: 
    \begin{align*}
        \formulaskip &
        \exists \consistencyrelation{CR'}{} \in \transitiveclosure{(\consistencyrelationset{CR} \setminus \setted{\consistencyrelation{CR}{}})} : 
        \forall \conditionelement{c}{l} \in \condition{c}{l, \consistencyrelation{CR}{}} : 
        \exists \conditionelement{c'}{l} \in \condition{c}{l,\consistencyrelation{CR'}{}} : \\
        & \formulaskip
        \forall \modeltuple{m} \in \metamodeltupleinstanceset{M} : 
        \modeltuple{m} \containsmath \conditionelement{c'}{l} \Rightarrow 
        \modeltuple{m} \containsmath \conditionelement{c}{l}
    \end{align*}
\end{lemma}

\begin{proof}
    Due to left-equal redundancy of $\consistencyrelation{CR}{}$ in $\consistencyrelationset{CR}$, we know per definition that:
    \begin{align*}
        \formulaskip &
        \exists \consistencyrelation{CR'}{} \in \transitiveclosure{(\consistencyrelationset{CR} \setminus \setted{\consistencyrelation{CR}{}})} :
        \forall \modeltuple{m} \in \metamodeltupleinstanceset{M} : \\
        & \formulaskip
        \modeltuple{m} \consistenttomath \consistencyrelation{CR'}{} \Rightarrow \modeltuple{m} \consistenttomath \consistencyrelation{CR}{} \\
        & \formulaskip
        \land 
        \classtuple{C}{l,\consistencyrelation{CR}{}} = \classtuple{C}{l,\consistencyrelation{CR'}{}}
    \end{align*}
    This implies that:
    \begin{align*}
        \formulaskip &
        \exists \consistencyrelation{CR'}{} \in \transitiveclosure{(\consistencyrelationset{CR} \setminus \setted{\consistencyrelation{CR}{}})} :
        %\forall \modelset{m} \in \metamodelinstances{\metamodelset{M}} : \\
        %& \formulaskip
        %\condition{c}{l,\consistencyrelation{CR}{}} \subseteq \condition{c}{l,\consistencyrelation{CR'}{}} 
        \forall \conditionelement{c}{l} \in \condition{c}{l,\consistencyrelation{CR}{}} :
        \conditionelement{c}{l} \in \condition{c}{l,\consistencyrelation{CR'}{}}
    \end{align*}
    Because if there was a $\conditionelement{c}{l} \in \condition{c}{l,\consistencyrelation{CR}{}}$ so that $\conditionelement{c}{l} \not\in \condition{c}{l,\consistencyrelation{CR'}{}}$, then the model tuple $\modeltuple{m}$ only consisting of $\conditionelement{c}{l}$ would be consistent to $\consistencyrelation{CR'}{}$, because it does not require any other elements to exist for considering the model tuple consistent, whereas there is at least one $\tupled{\conditionelement{c}{l}, \conditionelement{c}{r}} \in \consistencyrelation{CR}{}$, so that $\modeltuple{m}$ needs to contain $\conditionelement{c}{r}$ for considering $\modeltuple{m}$ consistent to $\consistencyrelation{CR}{}$, which is not given by construction.
    This shows that $\condition{c}{l,\consistencyrelation{CR'}{}}$ contains all elements in $\condition{c}{l,\consistencyrelation{CR}{}}$, so there is always at least one element from $\condition{c}{l,\consistencyrelation{CR'}{}}$ that a model tuple $\modeltuple{m}$ contains if it contains an element from $\condition{c}{l,\consistencyrelation{CR}{}}$, %namely the same one, 
    which proves the statement in the lemma.
\end{proof}

\todoLater{The following lemma derived the property of left-equal redundancy from redundancy, which was not correct. Maybe we can find a more general notion of redundancy from which we can derive the contains implication, reviving this lemma gain.}
% \begin{lemma} \label{lemma:redundancysubset}
%     Let $\consistencyrelation{CR}{}$ be a consistency relation that is redundant in a set of consistency relations $\consistencyrelationset{CR}$ for a set of metamodels $\metamodelset{M}$. Thus there exists a consistency relation $\consistencyrelation{CR'}{} \in \transitiveclosure{(\consistencyrelationset{CR} \setminus \setted{\consistencyrelation{CR}{}})}$ with:
%     \begin{align*}
%         \formulaskip & 
%         \forall \modelset{m} \in \metamodelinstances{\metamodelset{M}} : \modelset{m} \consistenttomath \consistencyrelation{CR'}{} \Rightarrow \modelset{m} \consistenttomath \consistencyrelation{CR}{}
%     \end{align*}
%     Then it holds that:
%     \begin{align*}
%         \formulaskip &
%         \forall \conditionelement{c}{l} \in \condition{c}{l, \consistencyrelation{CR}{}} : \exists \conditionelement{c'}{l} \in \condition{c}{l, \consistencyrelation{CR'}{}} : 
%         \forall{m} \in \metamodelinstances{\metamodelset{M}} : \\
%         & \formulaskip
%         \modelset{m} \containsmath \condition{c'}{l} \Rightarrow \modelset{m} \containsmath \condition{c}{l} %\\
%         % &
%         % \land \forall \conditionelement{c}{r} \in \condition{c}{r, \consistencyrelation{CR}{}} : \exists \conditionelement{c'}{r} \in \condition{c}{r, \consistencyrelation{CR'}{}} : 
%         % \forall{m} \in \metamodelinstances{\metamodelset{M}} : \\
%         % & \formulaskip
%         % \modelset{m} \containsmath \condition{c'}{r} \Rightarrow \modelset{m} \containsmath \condition{c}{r}
%         %\conditionelement{c}{l} \subseteq \conditionelement{c'}{l}
%     \end{align*}
% \end{lemma}

% \begin{proof}
%     % Due to symmetry of the statement for $\conditionelement{c}{l}$ and $\conditionelement{c}{r}$, the proof is also symmetric, which is why we restrict the proof to $\conditionelement{c}{l}$. 
%     We prove that
%     \begin{align*}
%         \formulaskip &
%         \forall \conditionelement{c}{l} \in \condition{c}{l, \consistencyrelation{CR}{}} : \exists \conditionelement{c'}{l} \in \condition{c}{l, \consistencyrelation{CR'}{}} : 
%         %\forall{m} \in \metamodelinstances{\metamodelset{M}} : \\
%         %& \formulaskip
%         \conditionelement{c}{l} \subseteq \conditionelement{c'}{l}
%     \end{align*}
%     which directly implies the statement according to \autoref{def:conditionelementcontainment} for the containment of condition elements.
%     Let us assume the contrary, such that:
%     \begin{align*}
%         \formulaskip &
%         \exists \conditionelement{c}{l} \in \condition{c}{l, \consistencyrelation{CR}{}} : \forall \conditionelement{c'}{l} \in \condition{c}{l, \consistencyrelation{CR'}{}} : \conditionelement{c}{l} \not\subseteq \conditionelement{c'}{l}
%     \end{align*}
%     Consider that $\conditionelement{c}{l} = \tupled{\object{o}{1}, \dots \object{o}{n}} \in \condition{c}{l, \consistencyrelation{CR}{}}$.
%     Now select a model set $\modelset{m} \in \metamodelinstances{\metamodelset{M}}$, which only contains objects $\object{o'}{1}, \dots, \object{o'}{n}$, such that $\forall i \in \setted{1, \dots, n} : \object{o}{i} \subseteq \object{o'}{i}$. In other words, we select a minimal model set that contains $\conditionelement{c}{l}$.
%     Per definition of $\consistencyrelation{CR}{}$, there must exist at least one consistency relation pair $\tupled{\conditionelement{c}{l}, \conditionelement{c}{r}} \in \consistencyrelation{CR}{}$, in which $\conditionelement{c}{l}$ occurs.
%     Since $\modelset{m}$ does not contain any $\conditionelement{c}{r}$, $\neg (\modelset{m} \consistenttomath \consistencyrelation{CR}{})$ per definition.
%     Since $\forall \conditionelement{c'}{l} \in \condition{c}{l, \consistencyrelation{CR'}{}} : \conditionelement{c}{l} \not\subseteq \conditionelement{c'}{l}$, there is no such $\conditionelement{c'}{l}$ with $\modelset{m} \containsmath \conditionelement{c'}{l}$.
%     \dots
%     \todoHeiko{Correct and finish proof}
% \end{proof}

\begin{theorem} \label{theorem:redundancycompatibility}
    Let $\consistencyrelationset{CR}$ be a compatible set of consistency relations for a tuple of metamodels $\metamodeltuple{M}$ and let $\consistencyrelation{CR}{}$ be a consistency relation that is left-equal redundant in $\consistencyrelationset{CR} \cup \setted{\consistencyrelation{CR}{}}$. Then $\consistencyrelationset{CR} \cup \setted{\consistencyrelation{CR}{}}$ is compatible. 
    % If two sets of consistency relations $\set{\consistencyrelation[1]{CR}}$ and $\set{\consistencyrelation[2]{CR}}$ are equivalent and $\set{\consistencyrelation[1]{CR}}$ is compatible, then $\set{\consistencyrelation[2]{CR}}$ is compatible as well.
\end{theorem}

\begin{proof}
    Due to left-equal redundancy of $\consistencyrelation{CR}{}$ in $\consistencyrelationset{CR} \cup \setted{\consistencyrelation{CR}{}}$, which also implies general redundancy according to \autoref{def:redundancy}, $\consistencyrelationset{CR}$ and $\consistencyrelationset{CR} \cup \setted{\consistencyrelation{CR}{}}$ are equivalent, according to \autoref{lemma:redundancyimpliesequivalence}.
    Due to that equivalence, we know that for any model tuple $\modeltuple{m} \in \metamodeltupleinstanceset{M}$:
    \begin{equation} \label{eq:redundancyconsistency}
        \formulaskip 
        \modeltuple{m} \mathtext{consistent to} \consistencyrelationset{CR} \equivalent     \modeltuple{m} \mathtext{consistent to} \consistencyrelationset{CR} \cup \setted{\consistencyrelation{CR}{}}
    \end{equation}
    It follows from \autoref{def:compatibility} for compatibility and \autoref{eq:redundancyconsistency}:
    \begin{align} \label{eq:redundancycompatibleexisting}
        \formulaskip & \nonumber
        \forall \consistencyrelation{CR'}{} \in \consistencyrelationset{CR} : \forall \conditionelement{c}{l} \in \condition{c}{l, \consistencyrelation{CR'}{}} %\cup \condition{c}{r, \consistencyrelation{CR'}{}} 
        : \exists \modeltuple{m} \in \metamodeltupleinstanceset{M} : \\
        & \formulaskip
        \modeltuple{m} \containsmath \conditionelement{c}{l} \land \modeltuple{m} \containsmath \consistencyrelationset{CR} \cup \setted{\consistencyrelation{CR}{}}
    \end{align}
    This already shows that for $\consistencyrelationset{CR}$ the compatibility definition is fulfilled, so we need to prove that the compatibility definition is fulfilled for $\consistencyrelation{CR}{}$ as well.
    % \begin{align*}
    %     \formulaskip &
    %     \forall \tupled{\conditionelement{c}{l}, \conditionelement{c}{r}} \in \consistencyrelation{CR}{} : \exists \modelset{m} \in \metamodelinstances{\metamodelset{M}}: \\
    %     & \formulaskip
    %     \modelset{m} \mathtext{contains} \tupled{\conditionelement{c}{l}, \conditionelement{c}{r}} \land \modelset{m} \mathtext{consistent to} \consistencyrelationset{CR} \cup \setted{\consistencyrelation{CR}{}}
    % \end{align*}
    Due to compatibility of $\consistencyrelationset{CR}$ and \autoref{lemma:compatibilitytransitiveclosure} showing equality of compatibility for a consistency relation set and its transitive closure, we know that:
    \begin{align} \label{eq:compatibilityclosure}
        \formulaskip & \nonumber
        \forall \consistencyrelation{CR'}{} \in \transitiveclosure{\consistencyrelationset{CR}} : \forall \conditionelement{c}{l} \in \condition{c}{l, \consistencyrelation{CR'}{}} %\cup \condition{c}{r, \consistencyrelation{CR'}{}} 
        : \exists \modeltuple{m} \in \metamodeltupleinstanceset{M} : \\
        & \formulaskip
        \modeltuple{m} \containsmath \conditionelement{c}{l} \land \modeltuple{m} \consistenttomath \transitiveclosure{\consistencyrelationset{CR}}
    \end{align}
    Due to left-equal redundancy of $\consistencyrelation{CR}{}$ in $\consistencyrelationset{CR} \cup \setted{\consistencyrelation{CR}{}}$, we have shown in \autoref{lemma:leftequalredundancysubset} that the following is true:
    \begin{align} \label{eq:redundancycontainment}
        \formulaskip & \nonumber 
        \exists \consistencyrelation{CR'}{} \in \transitiveclosure{\consistencyrelationset{CR}} : \forall \conditionelement{c}{l} \in \condition{c}{l, \consistencyrelation{CR}{}} : \exists \conditionelement{c'}{l} \in \condition{c}{l,\consistencyrelation{CR'}{}} : \forall \modeltuple{m} \in \metamodeltupleinstanceset{M} : \\
        & \formulaskip
        \modeltuple{m} \containsmath \conditionelement{c'}{l} \Rightarrow \modeltuple{m} \containsmath \conditionelement{c}{l}
    \end{align}
    The combination of \autoref{eq:compatibilityclosure} and \autoref{eq:redundancycontainment} gives:
    \begin{align*}
        \formulaskip & \nonumber 
        \exists \consistencyrelation{CR'}{} \in \transitiveclosure{\consistencyrelationset{CR}} : \forall \conditionelement{c}{l} \in \condition{c}{l, \consistencyrelation{CR}{}} : \exists \conditionelement{c'}{l} \in \condition{c}{l,\consistencyrelation{CR'}{}} : \\
        & \formulaskip
        (\forall \modeltuple{m} \in \metamodeltupleinstanceset{M} : \modeltuple{m} \containsmath \conditionelement{c'}{l} \Rightarrow \modeltuple{m} \containsmath \conditionelement{c}{l}) \\
        & \formulaskip
        \land (\exists \modeltuple{m} \in \metamodeltupleinstanceset{M} :
        \modeltuple{m} \containsmath \conditionelement{c'}{l} \land \modeltuple{m} \mathtext{consistent to} \transitiveclosure{\consistencyrelationset{CR}})
    \end{align*}
    A simplification by combining the two last lines of that statement leads to:
    \begin{align*}
        \formulaskip & \nonumber 
        \forall \conditionelement{c}{l} \in \condition{c}{l, \consistencyrelation{CR}{}} : \exists \modeltuple{m} \in \metamodeltupleinstanceset{M} : \\
        & \formulaskip
        \modeltuple{m} \containsmath \conditionelement{c}{l} \land \modeltuple{m} \consistenttomath \transitiveclosure{\consistencyrelationset{CR}}
    \end{align*}
    Due to \autoref{eq:redundancyconsistency} and \autoref{lemma:consistencytransitiveclosure}, which shows equality of consistency for a consistency relation set and its transitive closure, this is equivalent to:
    \begin{align} \label{eq:redundancycompatiblenew}
        \formulaskip & \nonumber 
        \forall \conditionelement{c}{l} \in \condition{c}{l, \consistencyrelation{CR}{}} : \exists \modeltuple{m} \in \metamodeltupleinstanceset{M} : \\
        & \formulaskip
        \modeltuple{m} \containsmath \conditionelement{c}{l} \land \modeltuple{m} \consistenttomath \consistencyrelationset{CR} \cup \setted{\consistencyrelation{CR}{}}
    \end{align}
    % Together with the symmetric argumentation for $\conditionelement{c}{r}$ rather than $\conditionelement{c}{l}$, we have shown that the compatibility definition holds for $\consistencyrelation{CR}{}$:
    % \begin{align} \label{eq:redundancycompatiblenew}
    %     \formulaskip & \nonumber 
    %     \forall \conditionelement{c}{} \in \condition{c}{l, \consistencyrelation{CR}{}} \cup \condition{c}{r,\consistencyrelation{CR}{}}: \exists \modelset{m} \in \metamodelinstances{\metamodelset{M}} : \\
    %     & \formulaskip
    %     \modelset{m} \mathtext{contains} \conditionelement{c}{} \land \modelset{m} \mathtext{consistent to} \consistencyrelationset{CR} \cup \setted{\consistencyrelation{CR}{}}
    % \end{align}
    The combination of \autoref{eq:redundancycompatibleexisting} and \autoref{eq:redundancycompatiblenew} shows that $\consistencyrelationset{CR} \cup \setted{\consistencyrelation{CR}{}}$ fulfills \autoref{def:compatibility} for compatibility.
    % Assume that given equivalent consistency relation sets $\set{\consistencyrelation[1]{CR}}$ and $\set{\consistencyrelation[2]{CR}}$ are equivalent and $\set{\consistencyrelation[1]{CR}}$ is compatible, whereas $\set{\consistencyrelation[2]{CR}}$ is not.
    % Then there is a consistency relation $\consistencyrelation{CR} \in \set{\consistencyrelation[2]{CR}}$ such that for all pairs of tuples $\bigtupled{\tupled{e_{l1}, \ldots, e_{ln}}, \tupled{e_{r1}, \ldots, e{rm}}} \in \consistencyrelation{CR}$ there is no set of models $\tupled{\model[1]{m}, \ldots, \model[k]{m}}$ such that for all models $\model[i]{m}, \model[j]{m}$ in that tuple either (i) $\{e_{l1}, \ldots e_{ln} \} \not\subseteq \model[i]{m} \lor \{e_{r1}, \ldots e_{rm} \not\subseteq \model[j]{m}$ or (ii) $\{ \model[1]{m}, \ldots, \model[k]{m} \} \text{not consistent according to} \set{\consistencyrelation[2]{CR}} \setminus \{ \consistencyrelation{CR} \}$. 
\end{proof}

% \begin{proof}
%     %In the proof, we always consider model sets $\modelset{m}$ to be sets of instances of the metamodels that are related by the consistency relations in $\consistencyrelationset{CR} \cup \setted{\consistencyrelation{CR}{}}$, without further mentioning that.
%     Due to the redundancy of $\consistencyrelation{CR}{}$ in $\consistencyrelationset{CR} \cup \setted{\consistencyrelation{CR}{}}$, $\consistencyrelationset{CR}$ and $\consistencyrelationset{CR} \cup \setted{\consistencyrelation{CR}{}}$ are equivalent, according to \autoref{corollary:redundancyimpliedequivalence}.
%     Due to that equivalence, it holds that for any model set $\modelset{m}$:
%     \begin{equation} \label{eq:redundancyconsistency}
%     \formulaskip 
%     \modelset{m} \mathtext{consistent to} \consistencyrelationset{CR} \equivalent \modelset{m} \mathtext{consistent to} \consistencyrelationset{CR} \cup \setted{\consistencyrelation{CR}{}}
%     \end{equation}
%     Due to \autoref{def:compatibility} for compatibility and \autoref{eq:redundancyconsistency}, it holds that:
%     \begin{align} \label{eq:redundancycompatibleexisting}
%         \formulaskip & \nonumber
%         \forall \consistencyrelation{CR'}{} \in \consistencyrelationset{CR} : \forall \tupled{\conditionelement{c}{l}, \conditionelement{c}{r}} \in \consistencyrelation{CR'}{} : \exists \modelset{m} \in \metamodelinstances{\metamodelset{M}}: \\
%         & \formulaskip
%         \modelset{m} \mathtext{contains} \tupled{\conditionelement{c}{l}, \conditionelement{c}{r}} \land \modelset{m} \mathtext{consistent to} \consistencyrelationset{CR} \cup \setted{\consistencyrelation{CR}{}}
%     \end{align}
%     This already shows that for $\consistencyrelationset{CR}$ the compatibility definition holds, so we need to prove that the compatibility definition holds for $\consistencyrelation{CR}{}$.
%     % \begin{align*}
%     %     \formulaskip &
%     %     \forall \tupled{\conditionelement{c}{l}, \conditionelement{c}{r}} \in \consistencyrelation{CR}{} : \exists \modelset{m} \in \metamodelinstances{\metamodelset{M}}: \\
%     %     & \formulaskip
%     %     \modelset{m} \mathtext{contains} \tupled{\conditionelement{c}{l}, \conditionelement{c}{r}} \land \modelset{m} \mathtext{consistent to} \consistencyrelationset{CR} \cup \setted{\consistencyrelation{CR}{}}
%     % \end{align*}
%     Due to compatibility of $\consistencyrelationset{CR}$ and \autoref{lemma:compatibilitytransitiveclosure} and \autoref{lemma:consistencytransitiveclosure}, it holds that:
%     \begin{align} \label{eq:compatibilityclosure}
%         \formulaskip & \nonumber
%         \forall \consistencyrelation{CR'}{} \in \transitiveclosure{\consistencyrelationset{CR}} : \forall \tupled{\conditionelement{c'}{l}, \conditionelement{c'}{r}} \in \consistencyrelation{CR'}{} : \exists \modelset{m} \in \metamodelinstances{\metamodelset{M}} :\\
%         & \formulaskip
%         \modelset{m} \mathtext{contains} \tupled{\conditionelement{c'}{l}, \conditionelement{c'}{r}} \land \modelset{m} \mathtext{consistent to} \transitiveclosure{\consistencyrelationset{CR}}
%     \end{align}
%     Due to the redundancy of $\consistencyrelation{CR}{}$ in $\consistencyrelationset{CR} \cup \setted{\consistencyrelation{CR}{}}$, it holds that:
%     \begin{align} \label{eq:redundancycontainment}
%         \formulaskip & \nonumber 
%         \exists \consistencyrelation{CR'}{} \in \transitiveclosure{\consistencyrelationset{CR}} : \forall \tupled{\conditionelement{c}{l}, \conditionelement{c}{r}} \in \consistencyrelation{CR}{} : \exists \tupled{\conditionelement{c'}{l}, \conditionelement{c'}{r}} \in \consistencyrelation{CR'}{} : \forall \modelset{m} \in \metamodelinstances{\metamodelset{M}} : \\
%         & \formulaskip 
%         \modelset{m} \mathtext{contains} \tupled{\conditionelement{c}{l}, \conditionelement{c}{r}} \equivalent
%         \modelset{m} \mathtext{contains} \tupled{\conditionelement{c'}{l}, \conditionelement{c'}{r}}
%     \end{align}
%     \autoref{eq:compatibilityclosure} especially holds for the selected $\consistencyrelation{CR'}{}$ and $\tupled{\conditionelement{c'}{l}, \conditionelement{c'}{r}}$ in \autoref{eq:redundancycontainment}, such that the following holds:
%     \begin{align*}
%         \formulaskip & 
%         \exists \consistencyrelation{CR'}{} \in \transitiveclosure{\consistencyrelationset{CR}} : \forall \tupled{\conditionelement{c}{l}, \conditionelement{c}{r}} \in \consistencyrelation{CR}{} : \exists \tupled{\conditionelement{c'}{l}, \conditionelement{c'}{r}} \in \consistencyrelation{CR'}{} : \\
%         & \formulaskip 
%         \forall \modelset{m} \in \metamodelinstances{\metamodelset{M}} : \modelset{m} \mathtext{contains} \tupled{\conditionelement{c}{l}, \conditionelement{c}{r}} \equivalent
%         \modelset{m} \mathtext{contains} \tupled{\conditionelement{c'}{l}, \conditionelement{c'}{r}} \\
%         & \formulaskip
%         \land \exists \modelset{m} \in \metamodelinstances{\metamodelset{M}} : \modelset{m} \mathtext{contains} \tupled{\conditionelement{c'}{l}, \conditionelement{c'}{r}} \land \modelset{m} \mathtext{consistent to} \transitiveclosure{\consistencyrelationset{CR}}
%     \end{align*}
%     Combining the last two lines leads to:
%     \begin{align*}
%         \formulaskip & 
%         \exists \consistencyrelation{CR'}{} \in \transitiveclosure{\consistencyrelationset{CR}} : \forall \tupled{\conditionelement{c}{l}, \conditionelement{c}{r}} \in \consistencyrelation{CR}{} : \exists \tupled{\conditionelement{c'}{l}, \conditionelement{c'}{r}} \in \consistencyrelation{CR'}{} : \\
%         & \formulaskip
%         \exists \modelset{m} \in \metamodelinstances{\metamodelset{M}} : \modelset{m} \mathtext{contains} \tupled{\conditionelement{c}{l}, \conditionelement{c}{r}} \land \modelset{m} \mathtext{consistent to} \transitiveclosure{\consistencyrelationset{CR}}
%     \end{align*}
%     This implies with \autoref{lemma:consistencytransitiveclosure} and \autoref{eq:redundancyconsistency} that:
%     \begin{align} \label{eq:redundancycompatiblenew}
%         \formulaskip & \nonumber
%         \forall \tupled{\conditionelement{c}{l}, \conditionelement{c}{r}} \in \consistencyrelation{CR}{} : \exists \modelset{m} \in \metamodelinstances{\metamodelset{M}} : \\
%         & \formulaskip
%         \modelset{m} \mathtext{contains} \tupled{\conditionelement{c}{l}, \conditionelement{c}{r}} \land \modelset{m} \mathtext{consistent to} \consistencyrelationset{CR} \cup \setted{\consistencyrelation{CR}{}}
%     \end{align}
%     The combination of \autoref{eq:redundancycompatibleexisting} and \autoref{eq:redundancycompatiblenew} shows that the compatibility definition is fulfilled for $\consistencyrelationset{CR} \cup \setted{\consistencyrelation{CR}{}}$.
%     % Assume that given equivalent consistency relation sets $\set{\consistencyrelation[1]{CR}}$ and $\set{\consistencyrelation[2]{CR}}$ are equivalent and $\set{\consistencyrelation[1]{CR}}$ is compatible, whereas $\set{\consistencyrelation[2]{CR}}$ is not.
%     % Then there is a consistency relation $\consistencyrelation{CR} \in \set{\consistencyrelation[2]{CR}}$ such that for all pairs of tuples $\bigtupled{\tupled{e_{l1}, \ldots, e_{ln}}, \tupled{e_{r1}, \ldots, e{rm}}} \in \consistencyrelation{CR}$ there is no set of models $\tupled{\model[1]{m}, \ldots, \model[k]{m}}$ such that for all models $\model[i]{m}, \model[j]{m}$ in that tuple either (i) $\{e_{l1}, \ldots e_{ln} \} \not\subseteq \model[i]{m} \lor \{e_{r1}, \ldots e_{rm} \not\subseteq \model[j]{m}$ or (ii) $\{ \model[1]{m}, \ldots, \model[k]{m} \} \text{not consistent according to} \set{\consistencyrelation[2]{CR}} \setminus \{ \consistencyrelation{CR} \}$. 
    
%     % Need to redefine compatibility to proceed \dots
% \end{proof}
% \todoHeiko{the proof does not require the second part of the redundancy definition, so can we omit it? Or is there a mistake in the proof?}

\begin{corollary} \label{corollary:transitiveredundancycompatibility}
    Let $\consistencyrelationset{CR}$ be a compatible set of consistency relations and let $\consistencyrelation{CR}{1}, \dots, \consistencyrelation{CR}{k}$ be consistency relations with:
    \begin{align*}
        \formulaskip &
        \forall i \in \setted{1, \dots, k} : \\
        & \formulaskip 
        \consistencyrelation{CR}{i} \leftequalredundantinmath \consistencyrelationset{CR} \cup \setted{\consistencyrelation{CR}{1}, \dots, \consistencyrelation{CR}{i}}
    \end{align*}
    Then $\consistencyrelationset{CR} \cup \setted{\consistencyrelation{CR}{1}, \dots, \consistencyrelation{CR}{k}}$ is compatible.
\end{corollary}

\begin{proof}
    This is an inductive implication of \autoref{theorem:redundancycompatibility}, because $\consistencyrelationset{CR}$ is compatible and sequentially adding $\consistencyrelation{CR}{i}$ to $\consistencyrelationset{CR} \cup \setted{\consistencyrelation{CR}{1}, \dots, \consistencyrelation{CR}{i-1}}$ ensures that $\consistencyrelationset{CR} \cup \setted{\consistencyrelation{CR}{1}, \dots, \consistencyrelation{CR}{i}}$ is compatible, because $\consistencyrelationset{CR} \cup \setted{\consistencyrelation{CR}{1}, \dots, \consistencyrelation{CR}{i-1}}$ was compatible as well.
\end{proof}

With \autoref{corollary:transitiveredundancycompatibility}, we have shown that if we have a set of consistency relations $\consistencyrelationset{CR}$ and are able to find a sequence of redundant consistency relations $\consistencyrelation{CR}{1}, \dots {\consistencyrelation{CR}{k}}$ according to \autoref{corollary:transitiveredundancycompatibility} such that we know that $\consistencyrelationset{CR} \setminus \setted{ \consistencyrelation{CR}{1}, \dots {\consistencyrelation{CR}{k}}}$ is compatible, then it is proven that $\consistencyrelationset{CR}$ is compatible.



\subsection{An Algorithm to Prove Compatibility} % Trees, Independence and Redundancy for Witnessing Compatibility
\label{chap:compatibility:formal_approach:algorithm}

In the previous sections, we have proven the following three central insights:
\begin{longenumerate}
    \item Compatibility is composable: If independent sets of consistency relations are compatible, then their union is compatible as well (\autoref{theorem:independencecompatibility}).
    \item Consistency relation trees are compatible: If there are no two concatenations of consistency relations in a consistency relation set that relate the same classes, then that set is compatible (\autoref{theorem:treecompatibility}).
    \item Left-equal redundancy is compatibility-preserving: Adding a left-equal redundant consistency relation to a compatible set of consistency relations, that set unified with the redundant relation is still compatible (\autoref{corollary:transitiveredundancycompatibility}).
\end{longenumerate}

These insights enable us to define a formal approach for proving compatibility of a set of consistency relations.
Given a set of relations for which compatibility shall be proven, we search for consistency relations in that set that are left-equal redundant to it.
If iteratively removing such redundant relations---virtually---from the set leads to a set of independent consistency relation trees, it is proven that the initial set of consistency relations is compatible.

\begin{algorithm}
    %\let\oldalgorithmic\algorithmic 
%\renewcommand\algorithmic{\ttfamily\fontseries{l}\selectfont\oldalgorithmic}
%\renewcommand{\Comment}[2][.3\linewidth]{\leavevmode\hfill\makebox[#1][l]{//~#2}}

    %\algrenewcommand\alglinenumber[1]{\texttt{\footnotesize #1}}
    \makeatletter
    \algnewcommand{\LineComment}[1]{\Statex \hskip\ALG@thistlm \(// \) #1}
\makeatother

\begin{algorithmic}[1]
\Procedure{\function{ProveCompatibility}}{$\consistencyrelationset{CR}$}
    \State $\mathvariable{isTree}$ $\leftarrow$ \function{IsRelationTree}($\consistencyrelationset{CR}$) \label{algo:compatibility:starttree}
    \If{$\mathvariable{isTree}$}
        \State \Return{\textsc{true}}
    \EndIf \label{algo:compatibility:endtree}

    \State $\mathvariable{hasIndependentSubsets}$ $\leftarrow$ \function{HasIndependentSubsets}($\consistencyrelationset{CR}$) \label{algo:compatibility:startindependence}
    \If{$\mathvariable{hasIndependentSubsets}$}
        \State $\setted{\consistencyrelationset{CR}_{1}, \consistencyrelationset{CR}_{2}}$ $\leftarrow$ \function{FindIndependentSubsets}($\consistencyrelationset{CR}$)
        \State $\mathvariable{isFirstSetCompatible}$ = \function{ProveCompatibility}($\consistencyrelationset{CR}_{1}$)
        \State $\mathvariable{isSecondSetCompatible}$ = \function{ProveCompatibility}($\consistencyrelationset{CR}_{2}$)
        \State \Return{$\mathvariable{isFirstSetCompatible} \land \mathvariable{isSecondSetCompatible}$}
    \EndIf \label{algo:compatibility:endindependence}

    \State $\mathvariable{redundantRelation}$ $\leftarrow$ \function{FindRedundantRelation}($\consistencyrelationset{CR}$) \label{algo:compatibility:startredundancy}
    \If{$\mathvariable{redundantRelation}$ $\neq \varnothing$}
        \State $\consistencyrelationset{CR}'$ = $\consistencyrelationset{CR} \setminus \mathvariable{redundantRelation}$
        \State \Return{\function{ProveCompatibility}($\consistencyrelationset{CR}'$)}
    \EndIf \label{algo:compatibility:endredundancy}

    \State \Return{\textsc{false}} \label{algo:compatibility:failure}
\EndProcedure
\end{algorithmic}
    \caption[Proof for compatibility of consistency relations]{Proof for compatibility of consistency relations.}
    \label{algo:compatibility}
\end{algorithm}

An algorithm that realizes this procedure is given in \autoref{algo:compatibility}.
It executes the described steps and assumes appropriate procedures to find out whether the given set of relations is a relation tree, whether it consists of independent subsets and whether there it contains a redundant relations.
It is easy to see that this algorithm is correct, as is implements the proven findings of the previous sections.
This does, however, not mean that implementing the sub-procedures is trivial.
We will provide a practical approach to realize them in the subsequent section.

\begin{theorem}[Compatibility Algorithm Correctness]
    \label{theorem:compatibilityalgorithmcorrectness}
    \autoref{algo:compatibility} is correct, i.e., it only returns \textsc{true} if the given consistency relations set $\consistencyrelationset{CR}$ is actually compatible.
\end{theorem}

\begin{proof}
    We make a case distinction for the situation in which the algorithm returns a result:
    \begin{longenumerate}
        \item When the consistency relation set is a tree, the algorithm directly returns \textsc{true} (Lines \ref{algo:compatibility:starttree}--\ref{algo:compatibility:endtree}), which is correct according to \autoref{theorem:treecompatibility}.
        \item When the consistency relation set can be split into independent sets, the algorithm returns \textsc{true} when both independent sets are identified as compatible by recursive application of the algorithm (Lines \ref{algo:compatibility:startindependence}--\ref{algo:compatibility:endindependence}), which is correct according to \autoref{theorem:independencecompatibility}.
        \item When the consistency relation set contains a redundant relation, the algorithm returns \textsc{true} when the set without the redundant relation is identified as compatible by recursive application of the algorithm (Lines \ref{algo:compatibility:startredundancy}--\ref{algo:compatibility:endredundancy}), which is correct according to \autoref{corollary:transitiveredundancycompatibility}.
        \item In all other cases, the algorithm returns \textsc{false} (Line \ref{algo:compatibility:failure}).
        \qedhere
    \end{longenumerate}
\end{proof}

The algorithm is, however, also \emph{conservative}.
%Such an approach to prove compatibility of consistency relations is \emph{conservative}.
If the approach finds redundant relations, such that a consistency relation set can be reduced to a set of independent consistency relations trees, the set is proven compatible, as we have shown by proof.
If the approach is not able to find such relations, the set may still be compatible, but the approach is not able to prove that.
Conceptually, this can be due to the fact that there may be compatibility-preserving relations that do not fulfill the definition of left-equal redundancy.
Furthermore, an actual technique to identify left-equal redundant relations may not be able to find all of them automatically, as we will see later.

\begin{theorem}[Compatibility Algorithm Conservativeness]
    \autoref{algo:compatibility} is conservative, i.e., it is correct, but if it returns \textsc{false} the given consistency relations set $\consistencyrelationset{CR}$ is not necessarily incompatible.
\end{theorem}

\begin{proof}
    We know that the algorithm is correct due to \autoref{theorem:compatibilityalgorithmcorrectness}.
    Additionally, it is easy to find examples for which the algorithm cannot prove compatibility, although the relations are compatible.
    Let us assume a consistency relation $\consistencyrelation{CR}{}$. 
    Then we construct a consistency relation $\consistencyrelation{CR}{}'$ by taking $\consistencyrelation{CR}{}$, adding an arbitrary class $C$ to the left-hand side class tuple of the relation and constructing the relation elements by taking the ones in $\consistencyrelation{CR}{}$, each complemented by all instances of $C$.
    Then $\setted{\consistencyrelation{CR}{},\consistencyrelation{CR}{}'}$ is, by construction, compatible, but the two relations are neither independent or a consistency relation tree, as they relate the same classes, nor are they redundant according to \autoref{def:leftequalredundancy}, because the left-side class tuples are not equal.
\end{proof}

The example given in the proof for conservativeness shows that the strictness of our definition for left-equal redundancy (\autoref{def:leftequalredundancy}).
We will, however, see in the evaluation that it is still sufficient in realistic cases, although such special cases as discussed in the proof are not supported.

In the following, we discuss how such an approach can be operationalized.
First, we discuss how actual transformations, at the example of \qvtr, can be represented in a graph-based structure, such that it conforms to our formal notion and allows to check whether the structure is an independent set of consistency relation trees.
Second, we present an approach for finding consistency relations that are left-equal redundant, by the means of an SMT solver applied to the constraints defined in \qvtr relations.


\todoLater{We partly did that by the expressiveness section for model-level and fine grained relations: Construction of valid models. Valid models may restrict the usable instances of a metamodel. Discuss impact on definitions and theorems and especially the constructive discussions within the proofs. Especially consider the meaning of references in models.}

%\subsection{TODO}
%\begin{itemize}
%    \item Afterthought: Construction of valid models. Valid models may restrict the usable instances of a metamodel. Discuss impact on definitions and theorems and especially the constructive discussions within the proofs. Especially consider the meaning of references in models.
%    \item This does not yet consider if the same meta element is used in different contexts. Thus there may be two paths between elements on the meta element level, although there will never be two paths in the instantiated models, because they occur in different contexts.
%    \item Explain conservativeness
%\end{itemize}

\end{copiedFrom} % SoSym MPM4CPS
\section{A Practical Approach to Prove Compatibility}
\label{chap:compatibility:practical_approach}

\mnote{Operationalization of compatibility proof}
We have presented a formal and proven correct approach for validating compatibility of consistency relations in the previous section.
It comprises the reduction of a given set of consistency relations by removing redundant relations to result in independent consistency relation trees.
In this section, we propose an algorithm that turns the formal approach into an operational procedure.
For the most part, this approach is based on results developed and described in detail in the Master's thesis of \textowncite{pepin2019ma}, which was supervised by the author of this thesis, and published in a report~\owncite{klare2020compatibility-report}.

\mnote{Approach for \qvtr}
We call the process of removing redundant relations from a consistency relation set to generate independent consistency relation trees \emph{decomposition}.
A decomposition procedure requires a representation of consistency relations present in actual model transformations that allows to validate their redundancy, more specifically the property of left-equal redundancy (see~\autoref{def:leftequalredundancy}).
We have decided to employ the transformation language \qvtr for the operationalization.
First, \qvtr is standardized~\cite{qvt} and well researched.
Second, it provides a level of abstraction at which consistency relations are explicitly represented.
In contrast, imperative languages would first require to extract consistency relations from their implicit specification as the image of the transformation rules.

\mnote{Procedure overview}
In the following, we first present a mapping between the formalization of the previous sections to the \qvtr transformation language through the use of \emph{predicates}.
We then propose a fully automated decomposition procedure that takes a set of \qvtr transformations, called a \emph{consistency specification}, as an input and removes redundant consistency relations as far as possible.
To find a redundant relation, the procedure identifies an alternative concatenation of consistency relations relating the same classes, according to \autoref{def:leftequalredundancy}, and performs a \emph{redundancy test} with respect to that alternative concatenation.
We explicitly separate the identification of candidates for the alternative concatenation from the redundancy test to assert exchangeability of the redundancy test approach.


\subsection{Practical Specification of Consistency Relations}

\mnote{Intensional consistency specifications}
In \autoref{chap:correctness:notions_consistency:intensional_extensional}, we have discussed the distinction of intensional and extensional specifications of consistency.
We have used an extensional specification for formalizing consistency relations in \autoref{def:consistencyrelation}.
Developers, however, define transformations with intensional specifications of the constraints that have to hold.
In relational transformation languages, such as \qvtr, they define consistency as a set of conditions that models must fulfill.
Such conditions are expressed with metamodel elements, like attributes and references.
For example, an employee and a resident are considered consistent if their $\mathvariable{name}$ values are equal.

\mnote{Predicates expressing conditions}
Conditions represent predicates, i.e., Boolean-valued filter functions.
Consistency relations are then defined as sets of condition element pairs for which the predicate evaluates to $\truemath$.
In \autoref{chap:correctness:notions_consistency:intensional_extensional}, we have already shown that this type of specification has equal expressiveness and can be transformed into an extensional specification.
We define such a predicate based on combinations of properties, selected from each metamodel, which we introduce in the following.

\begin{definition}[Property Set]
A property set $\propertyset{P}{\class{C}{}}$ for a class $\class{C}{}$ is a subset of properties of $\class{C}{}$, i.e., $\propertyset{P}{\class{C}{}} = \setted{P_{\class{C}{}, 1}, \dots, P_{\class{C}{}, n}}$ such that $P_{\class{C}{}, i} \in \class{C}{}$.
\end{definition}

\mnote{Consistency-relevant properties}
A property set represents a selection of properties of a class that are relevant for the definition of a predicate in order to distinguish consistent and non-consistent condition elements. For a consistency relation, not all properties of a class may be relevant and thus need to be considered. In a case of an extensional specification at the level of classes rather than properties, such as the one defined in \autoref{def:consistencyrelation}, this is expressed by enumerating all objects with all possible values of the irrelevant properties. Thus, expressing the relations at the level of classes or properties have equal expressiveness.

\begin{definition}[Tuple of Property Sets]
For a class tuple $\classtuple{C}{} = \tupled{\class{C}{1}, \dots, \class{C}{n}}$, we denote a tuple of property sets for every class as a property tuple $\propertysettuple{P}{\classtuple{C}{}}$:
\begin{align*}
    \propertysettuple{P}{\classtuple{C}{}} = \tupled{\propertyset{P}{\class{C}{1}}, \dots, \propertyset{P}{\class{C}{n}}} = \tupled{\setted{P_{\class{C}{1}, 1}, \dots, P_{\class{C}{1}, m}}, \dots, \setted{P_{\class{C}{n}, 1}, \dots, P_{\class{C}{n}, k}}}
\end{align*}
\end{definition}

\mnote{Property tuples for class tuples}
Since condition elements in consistency relations consist of objects that instantiate multiple classes, property set tuples generalize the use of property sets to class tuples.

\begin{definition}[Property Value Set]
A property value set $\propertyvalueset{p}{\class{C}{}}$ for a property set $\propertyset{P}{\class{C}{}} = \setted{P_{\class{C}{}, 1}, \dots, P_{\class{C}{}, n}}$ is a set in which each property in $\propertyset{P}{\class{C}{}}$ is instantiated, i.e., $\propertyvalueset{p}{\class{C}{}} = \setted{p_{\class{C}{}, 1}, \dots, p_{\class{C}{}, n}}$ with $p_{\class{C}{}, i} \in I_{P_{\class{C}{}, i}}$. 
Analogously, a tuple of property value sets is built from a tuple of property sets by instantiating each property set in it.
\end{definition}

\mnote{Relevant property values}
A property value set is a subset of property values of an object $\object{o}{}$ that instantiates $\class{C}{}$, like a property set is a subset of properties of a class $\class{C}{}$. Such a property value set represents the information of an object $\object{o}{}$ that is relevant for consistency according to a specific consistency relation.

\begin{definition}[Predicate]
A predicate for two class tuples $\classtuple{C}{l} = \tupled{\class{C}{l,1}, \dots, \class{C}{l,n}}$ and $\classtuple{C}{r}$ is a triple $\pi = \tupled{\propertysettuple{P}{\classtuple{C}{l}}, \propertysettuple{P}{\classtuple{C}{r}}, \function{f}_\pi}$ where $\propertysettuple{P}{\classtuple{C}{l}} = \tupled{\propertyset{P}{\class{C}{l,1}}, \dots, \propertyset{P}{\class{C}{l,n}}}$ (resp. $\propertysettuple{P}{\classtuple{C}{r}}$) is a tuple of property sets of $\classtuple{C}{l}$ (resp.\ $\classtuple{C}{r}$) and $\function{f}_\pi$ is a Boolean-valued function for instances of $\tuple{P}_{\classtuple{C}{l}}$ and $\tuple{P}_{\classtuple{C}{r}}$, i.e.,
\begin{align*}
    \function{f}_\pi : \instances{\propertysettuple{P}{\classtuple{C}{l}}} \times \instances{\propertysettuple{P}{\classtuple{C}{r}}} \to \setted{\truemath, \falsemath}
\end{align*}
\end{definition}

\mnote{Property union}
For better readability, we define the property collection $\propertycollection{\pi}$ of a predicate $\pi = (\propertysettuple{P}{\classtuple{C}{l}}, \propertysettuple{P}{\classtuple{C}{r}}, \function{f}_\pi)$ as the union of all properties in that predicate:
\begin{align*}
    &
    \propertycollection{\pi} \equalsperdefinition \bigcup_{j} \propertyset{P}{\class{C}{l,j}} \cup \bigcup_{k} \propertyset{P}{\class{C}{r,k}}
\end{align*}

\mnote{Conditions by predicates}
The definition of a predicate requires the selection of properties of the classes within the class tuples related by a consistency relation $CR$ and the definition of a function $\function{f}_\pi$ that defines whether two instances of these properties are considered consistent.
If $\function{f}_\pi$ evaluates to $\truemath$ for given property values of two object tuples, they match the predicate and are considered consistent, i.e., they represent the condition elements of a consistency relation pair according to \autoref{def:consistencyrelation}.

\mnote{Consistency relations by predicates}
Predicates thus model how consistency relations are defined in model transformation languages in terms of conditions to evaluate for object tuples, i.e., condition elements, rather than enumerating all consistent pairs of condition elements.
We define when we consider property values to match objects and then derive how consistency relations can be defined by predicates.

\begin{definition}[Property Matching]
Let $\propertyvalueset{p}{\class{C}{}} = \setted{p_{\class{C}{}, 1}, \dots, p_{\class{C}{}, n}}$ be a property value set. We say that:
\begin{align*}
    &
    \propertyvalueset{p}{\class{C}{}} \matchesmath \object{o}{} \equivalentperdefinition
    \object{o}{} \in I_{\class{C}{}} \land \forall p_{\class{C}{}, i} \in \propertyvalueset{p}{\class{C}{}} : p_{\class{C}{}, i} \in \object{o}{}
\end{align*}
%
Similarly, let $\propertyvaluesettuple{p}{\classtuple{C}{}} = \tupled{\propertyvalueset{p}{\class{C}{1}}, \dots, \propertyvalueset{p}{\class{C}{n}}}$ be a tuple of property value sets and $\tuple{o} = \tupled{\object{o}{1}, \dots, \object{o}{n}}$ a tuple of objects. We say that:
\begin{align*}
    &
    \propertyvaluesettuple{p}{\classtuple{C}{}} \matchesmath \tuple{o} \equivalentperdefinition
    \forall \propertyvalueset{p}{\class{C}{i}} \in \propertyvaluesettuple{p}{\classtuple{C}{}} : \propertyvalueset{p}{\class{C}{i}} \matchesmath \object{o}{i}
\end{align*}
\end{definition}

\begin{definition}[Predicate-Based Consistency Relation] \label{def:predicatebasedconsistencyrelation}
Let $\condition{c}{l}$ and $\condition{c}{r}$ be two conditions for two class tuples $\classtuple{C}{\condition{c}{l}}$ and $\classtuple{C}{\condition{c}{r}}$. 
Let $\Pi$ be a set of predicates for $\classtuple{C}{\condition{c}{l}}$ and $\classtuple{C}{\condition{c}{r}}$. A $\Pi$-based consistency relation $CR_{\Pi}$ is a subset of pairs of condition elements:
\parameterizeformat{
\begin{align*}
    &
    CR_{\Pi} \equalsperdefinition
    \setted{\tupled{\conditionelement{c}{l}, \conditionelement{c}{r}} \in \condition{c}{l} \times \condition{c}{r} \mid \forall \tupled{\propertysettuple{P}{\classtuple{C}{\condition{c}{l}}}, \propertysettuple{P}{\classtuple{C}{\condition{c}{r}}}, \function{f}_\pi} \in \Pi :
    #2
    \exists \propertyvalueset{p}{\classtuple{C}{\condition{c}{l}}} \in \instances{\propertysettuple{P}{\classtuple{C}{\condition{c}{l}}}},
    \propertyvalueset{p}{\classtuple{C}{\condition{c}{r}}} \in \instances{\propertysettuple{P}{\classtuple{C}{\condition{c}{r}}}}:\\
    & \formulaskip
    \big(
        \propertyvaluesettuple{p}{\classtuple{C}{\condition{c}{l}}} \matchesmath \conditionelement{c}{l}
        \land \propertyvaluesettuple{p}{\classtuple{C}{\condition{c}{r}}} \matchesmath \conditionelement{c}{r}
        \land \function{f}_{\pi}(\propertyvaluesettuple{p}{\classtuple{C}{\condition{c}{l}}}, \propertyvaluesettuple{p}{\classtuple{C}{\condition{c}{r}}}) = \truemath 
    \big)
    }
\end{align*}
}{}{\\ & \formulaskip\formulaskip}%
\end{definition}

\mnote{Intensional consisteny relations}
The construction of consistency relations by means of predicates is comparable to the one discussed in \autoref{chap:correctness:notions_consistency:intensional_extensional} at the level of models.
\autoref{def:predicatebasedconsistencyrelation} extends that construction to fine-grained consistency relations.
It expresses how consistency relations enumerating consistent object tuples are defined by means of predicates.
The construction of the consistency relation fully amounts to the evaluation of the predicate function.

\begin{example}
We construct a consistency relation $\consistencyrelation{CR}{PR}$ based on predicates between persons and residents, according to \autoref{fig:compatibility:three_persons_example_extended}. 
$\consistencyrelation{CR}{PR}$ ensures that the $\mathvariable{name}$ of a resident concatenates the $\mathvariable{firstname}$ and $\mathvariable{lastname}$ of a person and that both have the same address. 
$\consistencyrelation{CR}{PR}$ involves one class in each metamodel, resulting in two class tuples $\classtuple{C}{P} = \tupled{\class{C}{\mathvariable{Person}}}$ and $\classtuple{C}{R} = \tupled{\class{c}{\mathvariable{Resident}}}$. 
Two predicates need to represent consistency conditions, which are equal names and equal addresses.
The first predicate considers $\mathvariable{firstname}$ and $\mathvariable{lastname}$ of a person and $\mathvariable{name}$ of a resident, so $\propertysettuple{P}{\classtuple{C}{P}, 1} = \tupled{\setted{\mathvariable{firstname}, \mathvariable{lastname}}}$ and $\propertysettuple{P}{\classtuple{C}{R}, 1} = \tupled{\setted{\mathvariable{name}}}$. 
Similarly, $\propertysettuple{P}{\classtuple{C}{P}, 2} = \tupled{\setted{\mathvariable{address}}}$ and $\propertysettuple{P}{\classtuple{C}{R}, 2} = \tupled{\setted{\mathvariable{address}}}$.
The functions of the predicate, shortly denoting $\mathvariable{name}$ as $n$, $\mathvariable{firstname}$ as $\mathvariable{fn}$, $\mathvariable{lastname}$ as $\mathvariable{ln}$, as well as $\mathvariable{address}$ of a person as $a_P$ and of a resident as $a_R$, are:
\begin{align*}
   &
   \function{f}_{\pi, 1}(\tupled{\setted{\mathvariable{fn}, \mathvariable{ln}}}, \tupled{\setted{n}}) = 
   \begin{cases} 
      \truemath, & \ifmath n = \mathvariable{fn} + "\text{\textvisiblespace}" + \mathvariable{ln} \\
      \falsemath, & \otherwisemath
   \end{cases} \\
   &
   \function{f}_{\pi, 2}(\tupled{\setted{a_P}}, \tupled{\setted{a_R}}) = \begin{cases} 
      \truemath, & \ifmath a_P = a_R \\
      \falsemath, & \otherwisemath
   \end{cases}
\end{align*}
$\consistencyrelation{CR}{PR}$ is a $\Pi$-based consistency relation where $\Pi$ is the set of the two predicates for names and addresses $\setted{\tupled{\propertysettuple{P}{\classtuple{C}{P}, 1}, \propertysettuple{P}{\classtuple{C}{R}, 1},\function{f}_{\pi, 1}}, \tupled{\propertysettuple{P}{\classtuple{C}{P}, 2}, \propertysettuple{P}{\classtuple{C}{R}, 2}, \function{f}_{\pi, 2}}}$.
\end{example}

\lstinputlisting[float,
    language=QVT, 
    numbers=none, 
    mathescape=true, 
    caption={[Simplified structure of a \acrshort{QVTR} transformation]Simplified structure of a \qvtr transformation. Adapted from~\owncite[Lst.~1]{klare2020compatibility-report}.},
    label={lst:compatibility:qvtr_structure}
]{listings/correctness/compatibility/qvtr_structure.tex}

\mnote{Relations in \qvtr}
We have decided to use \qvtr as the relational language of the \gls{QVT} standard~\cite{qvt} for implementing the formal approach for validating compatibility.
The defined relations can be interpreted as predicates defining $\Pi$-based consistency relations.
The language can be executed in \emph{checkonly} mode to check models for fulfillment of consistency relations, or in \emph{enforce} mode to repair consistency in a specified direction if not all relations are fulfilled.
The relevant parts of the structure of a \qvtr transformation are as follows and also exemplified in \autoref{lst:compatibility:qvtr_structure}. 

\mnote{\qvtr structure}
A \qvtr \texttt{transformation} receives models, which conform to defined metamodels, and checks or repairs their consistency.
Each \texttt{transformation} is composed of \texttt{relation}s, which define when objects of both models are considered consistent. These relations are only invoked if they are prefixed by the \texttt{top} keyword, if they belong to the precondition (\texttt{when}) of a relation to be invoked, or if they belong to the invariant (\texttt{where}) of a relation that was already invoked. 
The \qvtr mechanism for checking consistency is based on pattern matching.
The relations between information in the different models are represented by variables assigned to class properties. 
These variables contain values that must remain consistent from one object to another. 
To consider models consistent, there must exist some assignment that matches all patterns at the same time. 

\lstinputlisting[float,
    language=QVT, 
    numbers=none, 
    mathescape=true, 
    caption={[Two \acrshort{QVTR} domains, each with one domain pattern]Two \qvtr domains, each with one domain pattern. Adapted from~\owncite[Lst.~2]{klare2020compatibility-report}.},
    label={lst:compatibility:qvtr_domain_patterns}
]{listings/correctness/compatibility/qvtr_domain_patterns.tex}

\mnote{Constraints in \qvtr}
More precisely, each \qvtr \texttt{relation} contains two \texttt{domain}s, which in turn contain \emph{domain patterns}. 
In \gls{QVT} terminology, a domain pattern is a variable instantiating a class. 
This variable can take values that are constrained by conditions on its properties, known as \emph{property template items}.
These conditions are expressed by \gls{OCL} constraints~\cite{ocl}.
We give an example for the domains of persons and employees according to the running example in \autoref{lst:compatibility:qvtr_domain_patterns}, in which each domain has one pattern.
These patterns filter $\mathvariable{Person}$ objects with three property template items for $\mathvariable{firstname}$, $\mathvariable{lastname}$, and $\mathvariable{income}$, and $\mathvariable{Employee}$ objects with two property template items for $\mathvariable{name}$ and $\mathvariable{salary}$, respectively. 
For two objects to be consistent, there must exist values of $\mathvariable{fstn}$, $\mathvariable{lstn}$, and $\mathvariable{inc}$ that match property values of these objects, thus ensuring that the employee $\mathvariable{name}$ equals the concatenation of the $\mathvariable{firstname}$ and the $\mathvariable{lastname}$ of the person and that both have the same $\mathvariable{income}$. 
If objects are inconsistent, e.g., if the person and the employee have different incomes, then there is no such variable assignment.
The \qvtr transformations for all three relations of the running example introduced with \autoref{fig:compatibility:three_persons_example_extended} are depicted in \autoref{lst:compatibility:qvtr_running_example}.

\lstinputlisting[float,
    language=QVT, 
    numbers=none, 
    mathescape=true,
    basicstyle=\ifisafour\small\else\footnotesize\fi\ttfamily,
    caption={[\acrshort{QVTR} transformations for the running example]Three binary \qvtr transformations forming a consistency specification, based on the relations in \autoref{fig:compatibility:three_persons_example_extended}. Adapted from~\owncite[Fig.~10]{klare2020compatibility-report}.},
    label={lst:compatibility:qvtr_running_example}
]{listings/correctness/compatibility/qvtr_running_example.tex}

\mnote{\qvtr to formalism}
In \emph{checkonly} mode, \qvtr evaluates the existence of a value that fulfills all property template items in domain patterns. These patterns can be regarded as predicates.
To transfer \qvtr relations into our formalism, each relation is translated into one or multiple predicates.
A predicate is formed by the properties that are bound to the same \qvtr variables, because having \qvtr variables in common means that values of these properties are interrelated and thus need to fulfill some consistency constraints.
The properties of each domain form one of the property sets of a predicate.
Extracting the \gls{OCL} constraints of the property template items generates the predicate function. 
The property sets together with the predicate function represent a predicate. 
We subsequently present a formal construction of predicates from \qvtr and their representation in a graph.


\subsection{Consistency Relations Represented as Graphs}

\mnote{Redundancy in predicate-based consistency relations}
In the following, we introduce the decomposition procedure for proving compatibility, which relies on an algorithmic way to detect redundant consistency relations. 
We have defined the notion of left-equal redundancy for extensionally specified consistency relations in \autoref{def:leftequalredundancy}.
That notion is based on classes, whereas predicate-based consistency relations are defined for properties.
We have, however, already discussed that both kinds of specification have equal expressiveness.
Comparing predicate-based consistency relations of different transformations to evaluate redundancy is what we call a \emph{redundancy test}.

\mnote{Graph representation}
Consistency specifications induce a graph of class properties, which are related by edges that are labeled with the predicates that define the consistency relations.
Such a graph representation enables the application of graph algorithms to identify independent and redundant consistency relations.
The decomposition procedure thus creates such a graph, denoted as a \emph{property graph}, out of \qvtr transformations and detects redundant relations in that graph.
It represents properties and predicates as a hypergraph with labeling.

\begin{definition}[Property Graph] \label{def:propertygraph}
Let $\consistencyrelationset{CR} = \setted{\consistencyrelation{CR}{1}, \dots, \consistencyrelation{CR}{n}}$ be a set of consistency relations where each consistency relation $\consistencyrelation{CR}{i}$ is based on a set of predicates $\Pi_i$. 
A property graph is a couple $\mathcal{M}=\tupled{\mathcal{H}, \function{l}}$, such that $\mathcal{H} = \tupled{V_{\mathcal{H}}, E_{\mathcal{H}}}$ is a hypergraph and $\function{l}$
is a hyperedge labeling:
$V_{\mathcal{H}}$ is the set of vertices, i.e., the union of properties in all predicates:
\begin{align*}
    V_{\mathcal{H}} \equalsperdefinition \bigcup_{i = 1}^{n} \bigcup_{\pi \in \Pi_{i}} \propertycollection{\pi}
\end{align*}
%
$E_{\mathcal{H}}$ is the set of hyperedges, i.e., $E_{\mathcal{H}} \subseteq \mathcal{P}(V_{\mathcal{H}}) \setminus \setted{\varnothing}$. Each hyperedge consists of the properties of one predicate:
\begin{align*}
    E_{\mathcal{H}} \equalsperdefinition \bigcup_{i = 1}^{n} \bigcup_{\pi \in \Pi_{i}} \setted{\propertycollection{\pi}}
\end{align*}
$\function{l}$ labels each hyperedge with its corresponding predicate function:
\begin{align*}
    &\forall i \in \setted{1, \dots, n} : \forall \pi = \tupled{\propertysettuple{P}{\classtuple{C}{l}}, \propertysettuple{P}{\classtuple{C}{r}}, \function{f}_\pi} \in \Pi_{i} :
    \function{l}(\propertycollection{\pi}) = \function{f}_\pi
\end{align*}
\end{definition}

\mnote{Independence and redundancy in property graph}
A property graph groups properties that are used in the same predicate.
Each hyperedge with its labeling represents a predicate, which, in turn, represents a consistency relation.
Thus, such a graph is useful for detecting independent sets of consistency relations and potential redundancies.
When there are sets of hyperedges that do not share any vertices, they relate independent sets of properties.
According to \autoref{def:independence}, the consistency relations represented by the hyperedges are independent.
On the contrary, if multiple sequences of hyperedges relate the same properties, the represented consistency relations form a cycle and may thus be incompatible or redundant.

\begin{figure}
    \centering
    \begin{tikzpicture}[every node/.style={font=\small, inner sep=0.2em}]
    %% METAGRAPH EDGES
    \filldraw[fill=kit-orange30, draw=kit-orange100, thin, fill opacity=0.6] (-1em, -1em)
        to[out=0, in=270, looseness=0.2] (16em, 8em)
        to[out=90, in=0, looseness=0.2] (12em, 12em)
        to[out=180, in=180, looseness=0.2] node[midway, name=orangeN] {} (-1em, -1em);

    \filldraw[fill=kit-yellow30, draw=kit-yellow100, thin, fill opacity=0.6] (-1em, 0)
        to[out=-45, in=225, looseness=0.3] node[midway, name=yellowN] {} (19em, -2.2em)
        to[out=45, in=0, looseness=0.2] (12em, 1em)
        to[out=180, in=90, looseness=0.2] (-1em, 0);

    \coordinate (greenN) at (9, 2);
    \filldraw[fill=kit-green30, draw=kit-green100, thin, fill opacity=0.6] (18.2em, -3em)
        to[out=135, in=270, looseness=0.4] (11.4em, 10.8em)
        to[out=0, in=90, looseness=0.4] (18em, 4em)
        to[out=270, in=0, looseness=0.4] (18.2em, -3em);

    \filldraw[fill=kit-blue30, draw=kit-blue100, thin, fill opacity=0.6] (-3em, 3em)
        to[out=0, in=300, looseness=0.3] (9em, 13em)
        to[out=180, in=90, looseness=0.2] (2em, 10.4em)
        to[out=225, in=90, looseness=0.3] (-3em, 3em);
    
    \filldraw[fill=kit-blue30, draw=kit-blue100, thin, fill opacity=0.6] (0.8em, -4em)
        to[out=90, in=90, looseness=0.2] (17.4em, -6.2em)
        to[out=270, in=0, looseness=0.3] (9em, -6.4em)
        to[out=180, in=-45, looseness=0.3] (0.8em, -4em);
        
    \filldraw[fill=kit-blue30, draw=kit-blue100, thin, fill opacity=0.6] (20em, 9em)
        to[out=225, in=135, looseness=0.2] (22em, -1em)
        to[out=0, in=270, looseness=0.3] (23.2em, 6em)
        to[out=100, in=0, looseness=0.4] (20em, 9em);

    %% METAGRAPH NODES
    \node[circle, blueshading, name=circleResN] at ($(0, 0)$) {};
    \node[above = 0.1em of circleResN] {\texttt{Resident::name}};

    \node[circle, blueshading, name=circleEmpN] at ($(18em, -2em)$) {};
    \node[above = 0.1em of circleEmpN] {\texttt{Employee::name}};

    \node[circle, blueshading, name=circlePersF] at ($(14em, 8em)$) {};
    \node[below = 0.1em of circlePersF] {\texttt{Person::firstname}};

    \node[circle, blueshading, name=circlePersN] at ($(12em, 10em)$) {};
    \node[above = 0.1em of circlePersN] {\texttt{Person::lastname}};
    
    %% METAGRAPH NODES (part 2)
    \node[circle, blueshading, name=circlePersA] at ($(8em, 12em)$) {};
    \node[above = 0.1em of circlePersA] {\texttt{Person::address}};

    \node[circle, blueshading, name=circlePersI] at ($(21em, 8em)$) {};
    \node[above = 0.1em of circlePersI] {\texttt{Person::income}};
    
    \node[circle, blueshading, name=circleEmpSo] at ($(16em, -6em)$) {};
    \node[above = 0.1em of circleEmpSo] {\texttt{Employee::socsecnumber}};
    
    \node[circle, blueshading, name=circleEmpSa] at ($(22em, 0)$) {};
    \node[above left = 0.1em and -3.7em of circleEmpSa] {\texttt{Employee::salary}};
    
    \node[circle, blueshading, name=circleResA] at ($(-2em, 4em)$) {};
    \node[above right = 0.1em and -2.5em of circleResA] {\texttt{Resident::address}};
    
    \node[circle, blueshading, name=circleResS] at ($(2em, -4em)$) {};
    \node[above = 0.1em of circleResS] {\texttt{Resident::socsecnumber}};
    
    \node at (8em, 6em) {\textbf{1}};
    \node at (16em, 4em) {\textbf{2}};
    \node at (10em, -1em) {\textbf{3}};
    \node at (2em, 8em) {\textbf{4}};
    \node at (22em, 4em) {\textbf{5}};
    \node at (8em, -5em) {\textbf{6}};
\end{tikzpicture}
    \begin{tikzpicture}[
    every node/.style={font=\small, inner sep=0.5em},
    predicate/.style={thin, minimum height=4em, minimum width=14.8em, text width=13.8em, align=left, anchor=west}]
    
    % METAGRAPH SYSTEMS
    \node[predicate, draw=kit-orange100, name=n1] at (0, 0) {%
        \texttt{Person::firstname = fstn} \\
        \texttt{Person::lastname = lstn} \\
        \lstinline[language=QVT]{Resident::name = fstn + ' ' + lstn}
    };

    \node[predicate, draw=kit-green100, name=n2] at (0, -4.5em) {%
        \texttt{Person::firstname = fstn} \\
        \texttt{Person::lastname = lstn} \\
        \lstinline[language=QVT]{Employee::name = fstn + ' ' + lstn}
    };

    \node[predicate, draw=kit-yellow100, name=n3] at (0, -9em) {%
        \texttt{Resident::name = n} \\
        \texttt{Employee::name = n}
    };
    
    \node[predicate, draw=kit-blue100, name=n4, anchor=west] at (15.3em, 0) {%
        \texttt{Resident::address = addr} \\
        \texttt{Employee::address = addr}
    };
    
    \node[predicate, draw=kit-blue100, name=n5, anchor=west] at (15.3em, -4.5em) {%
        \texttt{Person::income = inc} \\
        \texttt{Employee::salary = inc}
    };
    
    \node[predicate, draw=kit-blue100, name=n6, anchor=west] at (15.3em, -9em) {%
        \texttt{Person::socsecnumber = ssn} \\
        \texttt{Employee::socsecnumber = ssn}
    };

    \node[below left= 0 of n1.north east, anchor=north east] {\textbf{1}};
    \node[below left= 0 of n2.north east, anchor=north east] {\textbf{2}};
    \node[below left= 0 of n3.north east, anchor=north east] {\textbf{3}};
    \node[below left= 0 of n4.north east, anchor=north east] {\textbf{4}};
    \node[below left= 0 of n5.north east, anchor=north east] {\textbf{5}};
    \node[below left= 0 of n6.north east, anchor=north east] {\textbf{6}};
\end{tikzpicture}
    \caption[Property graph for the running example]{Property graph for the \qvtr example in \autoref{lst:compatibility:qvtr_running_example} based on relations in \autoref{fig:compatibility:three_persons_example_extended}. Hyperedges are represented as shaded areas. Constraints for the predicate functions are annotated in boxes. Adapted from~\owncite[Fig.~11]{klare2020compatibility-report}.}
    \label{fig:compatibility:propertygraph_example}
\end{figure}

\mnote{Need for hypergraph}
A property graph needs to be a hypergraph, because a predicate can relate more than two properties, so an edge must be able to relate more than two vertices.
The consistency relation $\consistencyrelation{CR}{PE}$ of the running example relates an employee's $\mathvariable{name}$ to the concatenation of the $\mathvariable{firstname}$ and $\mathvariable{lastname}$ of a person and thus contains three properties.
We depict the hypergraph for the running example in \autoref{fig:compatibility:propertygraph_example}.
In the following, we discuss the construction of a property graph from \qvtr transformations.
The identification of redundancies in the represented relations is part of the subsequent subsection.

\mnote{Transformation and relation traversal}
The construction of the property graph for a given set of \qvtr transformations requires each of them to be processed.
Since transformations are not executed but only transformed into a property graph, the processing order is not relevant.
Each transformation consists of a set of \qvtr relations, of which each usually only defines consistency for small parts of the metamodels.
Those relations depend on each other and can thus not be processed in arbitrary order.
Only those relations that may be invoked during the execution of transformations need to be considered, which could be derived from a call graph.
While top-level relations are always invoked during execution, other relations are only invoked in \texttt{where} or \texttt{when} clauses of other relations similar to function calls. 
Since \texttt{when} and \texttt{where} clauses are dual to each other, we restrict ourselves to relations that are invoked in \texttt{where} clauses. 
Then, starting from top-level relations, relevant relations can simply be identified by a depth-first traversal. 

\mnote{Graph construction}
The property graph construction starts with an empty graph.
For every processed \qvtr relation, vertices and a hyperedge may be added.
Each \qvtr relation needs to be translated into a set of predicates, which are represented by labeled hyperedges, in accordance with \autoref{def:propertygraph}.
As an example, we consider the relation $\mathvariable{PE}$ of our running example in \autoref{lst:compatibility:qvtr_running_example}, which relates the domains for persons and employees.
For each domain of a relation, the class tuples of the predicates are specified in the domain patterns.
In the example, these class tuples are $\classtuple{C}{\mathvariable{person}} = \tupled{\class{C}{\mathvariable{Person}}}$ and $\classtuple{C}{\mathvariable{employee}} = \tupled{\class{C}{\mathvariable{Employee}}}$.
Each class in each class tuple is associated with a set of property template items.
A property template item relates a property to an \gls{OCL} expression. 
For example, the property template item $\mathvariable{name} = \mathvariable{fstn} + "\text{\textvisiblespace}" + \mathvariable{lstn}$ defines that the property $\mathvariable{name}$ must match the \gls{OCL} expression $\mathvariable{fstn} + "\text{\textvisiblespace}" + \mathvariable{lstn}$.
The \gls{OCL} expressions, in turn, contain \qvtr variables, such as $\mathvariable{fstn}$ and $\mathvariable{lstn}$.
Predicates are supposed to relate those properties that actually share a consistency relation, i.e., that are actually put into relation by the \qvtr relation.
Such a relation is only given if two properties are related by the same \qvtr variables, because in such a case a value assignment to that variable must satisfy the property template items of both properties.
In such a case, a hyperedge is created and labeled with a function that realizes the conditions of the property template item.
For example, $\mathvariable{Person.firstname}$, $\mathvariable{Person.lastname}$ and $\mathvariable{Employee.name}$ are related by the \qvtr variables $\mathvariable{fstn}$ and $\mathvariable{lstn}$, thus a hyperedge is generated between them.
In contrast, constraints on $\mathvariable{Employee.salary}$ and $\mathvariable{Employee.name}$ are independent, because the property template items relate them to disjoint sets of \qvtr variables. 
Thus consistency of one does not depend on consistency of the other.
In addition to property template items, \gls{OCL} expressions relating properties occur in \texttt{when} and \texttt{where} clauses, of which we, again, focus on invariants of \texttt{where} clauses.
Like for property template items, properties related by shared \qvtr variables in these clauses have to be grouped into a hyperedge.

\begin{algorithm}
    \algrenewcommand\alglinenumber[1]{\texttt{\footnotesize #1}}
\algdef{SE}[DOWHILE]{Do}{doWhile}{\algorithmicdo}[1]{\algorithmicwhile\ #1}%

\begin{algorithmic}[1]

\Procedure{\function{MergeConsistencyVariables}}{$\{({p}, V_{\{p\}})\}$}
    \State $stopMerge$ $\leftarrow$ \textsc{true} 
    \State $entries$ $\leftarrow$ $\tupled{(\{p\}, V_{\{p\}})}$\\

    \Do
        \State $stopMerge$ $\leftarrow$ \textsc{true}
        \State $results$ $\leftarrow$ $\tupled{}$\\
        
        \While{$entries \neq \tupled{}$}
        % \While{not entries.isEmpty()}
            \State $ref = (P_{\textup{ref}}, V_{p_{\textup{ref}}}) \leftarrow$ $entries[0]$ %\Comment{Reference pair to be compared with others}
            \State $others$ $\leftarrow$ $entries[1:]$
            \State $entries$ $\leftarrow$ $\tupled{}$\\
            
            \For{$(P, V_P) \in others$}
                \If{$V_P \cap V_{P_\textup{ref}} = \varnothing$}
                    \State $entries$ $\leftarrow$ $entries$ $\cup$ $\{(P, V_P)\}$
                \Else
                    \State $stopMerge$ $\leftarrow$ \textsc{False} \label{lst:line:stopMerge}%\Comment{Not all $V_{e_i}$ are pairwise disjoint}
                    \State $ref$ $\leftarrow$ $(P \cup P_{\textup{ref}}, V_P \cup V_{P_{\textup{ref}}})$
                \EndIf
            \EndFor
            \State $results$ $\leftarrow$ $results$ $\cup$ $\{$ref$\}$
        \EndWhile
        \State $entries$ $\leftarrow$ $results$
    \doWhile{$\neg stopMerge$}\\
    \State \Return{set($entries$)}
\EndProcedure

\end{algorithmic}
    \caption[Merge of properties to predicates]{Merge to predicates. Adapted from~\owncite[Alg.~1]{klare2020compatibility-report}.}
    \label{algo:compatibility:merge_properties}
\end{algorithm}

\mnote{Merge algorithm}
\autoref{algo:compatibility:merge_properties} expresses the sketched procedure of merging properties to predicates that finally represent hyperedges of the property graph.
It manages couples, called \emph{entries}, of properties and \qvtr variables.
These entries denote that a set of properties is related by the according set of \qvtr variables.
The algorithm starts with a set of couples, of which each couple $\tupled{\setted{p}, V_{\setted{p}}}$ consists of a singleton $\setted{p}$ that presents a property $p$ and the \qvtr variables $V_{\setted{p}}$ it is related to by its property template item.
In each iteration, the algorithm chooses one reference entry and merges it with all other entries to which the intersection of their \qvtr variables is not empty.
The algorithm terminates when all sets of \qvtr variables are pairwise disjoint.

\begin{example}
The relation $\mathvariable{PE}$ of the \qvtr transformation $\mathvariable{PersonEmployee}$ in \autoref{lst:compatibility:qvtr_running_example} contains five properties, which can be described with these entries:
\begin{align*}
&
    \tupled{\setted{\propdisplay{firstname}}, \setted{\mathvariable{fstn}}}, 
    \tupled{\setted{\propdisplay{lastname}}, \setted{\mathvariable{lstn}}},
    \tupled{\setted{\propdisplay{income}}, \setted{\mathvariable{inc}}}, \\ 
&
    \tupled{\setted{\propdisplay{name}}, \setted{\mathvariable{fstn}, \mathvariable{lstn}}},
    \tupled{\setted{\propdisplay{salary}}, \setted{\mathvariable{inc}}}
\end{align*}
After the execution of the algorithm, properties are merged into two sets:
\begin{align*}
&
    \tupled{\setted{\propdisplay{firstname}, \propdisplay{lastname}, \propdisplay{name}}, \setted{\mathvariable{fstn}, \mathvariable{lstn}}},
    \tupled{\setted{\propdisplay{income}, \propdisplay{salary}}, \setted{\mathvariable{inc}}}
\end{align*}
\end{example}

\mnote{Hypergraph from algorithm}
Each entry delivered by the algorithm can be transformed into a hyperedge.
To this end, the properties are grouped into two tuples according to the domains they originally belonged to.
The predicate function is given by the conjunction of all \gls{OCL} expressions associated with properties of the entry, i.e., property template items and invariants.
For the subsequent identification of redundant relations, we only need to operate on this hypergraph rather than the original metamodels or \qvtr transformations.


\subsection{Decomposition of Consistency Relations}

\mnote{General decomposition procedure}
The decomposition procedure for proving compatibility of consistency relations aims at removing redundant relations until, in case of success, the remaining relations form sets of independent consistency relation trees.
For a property graph $\mathcal{M} = \tupled{\mathcal{H}, \function{l}}$, this is achieved by removing the hyperedges of $\mathcal{H}$ that represent redundant consistency relations until no further redundant relations can be found.
Redundancy according to \autoref{def:leftequalredundancy} is given if for a consistency relation an alternative concatenation of consistency relations that relates the at least partly same classes does not restrict consistency.
In terms of a graph, this means that there must be two paths between the same properties.
Independence of consistency relations is then given by connected components of the hypergraph, because they represent the properties that are related by constraints involving the same \qvtr variables.
According to \autoref{theorem:independencecompatibility}, consistency relations are compatible if they are composed of independent, compatible subsets.
Thus if compatibility can be shown for the relations in each connected component of the hypergraph, their union is also compatible.

\mnote{Dual of property graph}
While the hypergraph representation of predicates in consistency relations is well suited for reasons of expressiveness, the drawback of hypergraphs is the increased complexity of graph algorithms, such as graph traversal.
We therefore replace the property graph with its dual, i.e., an equivalent simple graph, for the realization of the redundancy test.
This dual graph contains the hyperedges of the property graph as vertices and edges between two vertices when their hyperedges in the property graph share at least one property.
\autoref{fig:compatibility:dual_propertygraph_example} shows the dual of a property graph of the running example.

\begin{definition}[Dual of a Property Graph]
Let $\mathcal{M} = (\mathcal{H}, \function{l})$ be a property graph. The dual of the property graph $\mathcal{M}$, denoted $\mathcal{M^{*}}$, is a tuple $\tupled{\mathcal{G}, \function{v}, \function{l}}$ with a simple graph $\mathcal{G}$ and two functions $\function{v}$ and $\function{l}$ such that:
    \begin{itemize}
        \item $V_{\mathcal{G}} \equalsperdefinition E_{\mathcal{H}}$
        \item $E_{\mathcal{G}} \equalsperdefinition \setted{\setted{E_1, E_2} \mid \forall \tupled{E_1, E_2} \in E_{\mathcal{H}}^2 : E_1 \cap E_2 \neq \varnothing}$
        \item $\forall \tupled{E_1, E_2} \in E_{\mathcal{G}} : \function{v}(\setted{E_1, E_2}) = E_1 \cap E_2$
    \end{itemize}
\end{definition}

\begin{figure}
    \centering
    \begin{tikzpicture}[every node/.style={font=\small, inner sep=0.2em}]
    %% GRAPH NODES
    \node[circle, fill=kit-green30, draw=kit-green100, name=greenN] at (20em, 6em) {};
    \node[circle, fill=kit-orange30, draw=kit-orange100, name=orangeN] at (0, 4em) {};
    \node[circle, fill=kit-yellow30, draw=kit-yellow100, name=yellowN] at (10em, -4em) {};

    \node[circle, fill=kit-blue30, draw=kit-blue100, name=salary] at (-1em, -4em) {};
    
    \node[circle, fill=kit-blue30, draw=kit-blue100, name=socsecnumber] at (-4em, 0) {};
    
    \node[circle, fill=kit-blue30, draw=kit-blue100, name=address] at (20em, -2em) {};

    \draw[-] (greenN) -- (orangeN) node [midway, above, sloped] (T1) {\texttt{Person::firstname, Person::lastname}};
    \draw[-] (greenN) -- (yellowN) node [midway, above, sloped] (T2) {\texttt{Resident::name}};
    \draw[-] (orangeN) -- (yellowN) node [midway, above, sloped] (T1) {\texttt{Employee::name}};

    \node[left = 0.4em of orangeN] {\textbf{1}};
    \node[right = 0.4em of greenN] {\textbf{2}};
    \node[right = 0.4em of yellowN] {\textbf{3}};
    \node[left = 0.4em of address] {\textbf{4}};
    \node[right = 0.4em of salary] {\textbf{5}};
    \node[right = 0.4em of socsecnumber] {\textbf{6}};

    \node[below=1em of yellowN] () {};
\end{tikzpicture}
    \begin{tikzpicture}[
    every node/.style={font=\small, inner sep=0.5em},
    predicate/.style={thin, minimum height=4em, minimum width=14.8em, text width=13.8em, align=left, anchor=west}]
    
    % METAGRAPH SYSTEMS
    \node[predicate, draw=kit-orange100, name=n1] at (0, 0) {%
        \texttt{Person::firstname = fstn} \\
        \texttt{Person::lastname = lstn} \\
        \lstinline[language=QVT]{Resident::name = fstn + ' ' + lstn}
    };

    \node[predicate, draw=kit-green100, name=n2] at (0, -4.5em) {%
        \texttt{Person::firstname = fstn} \\
        \texttt{Person::lastname = lstn} \\
        \lstinline[language=QVT]{Employee::name = fstn + ' ' + lstn}
    };

    \node[predicate, draw=kit-yellow100, name=n3] at (0, -9em) {%
        \texttt{Resident::name = n} \\
        \texttt{Employee::name = n}
    };
    
    \node[predicate, draw=kit-blue100, name=n4, anchor=west] at (15.3em, 0) {%
        \texttt{Resident::address = addr} \\
        \texttt{Employee::address = addr}
    };
    
    \node[predicate, draw=kit-blue100, name=n5, anchor=west] at (15.3em, -4.5em) {%
        \texttt{Person::income = inc} \\
        \texttt{Employee::salary = inc}
    };
    
    \node[predicate, draw=kit-blue100, name=n6, anchor=west] at (15.3em, -9em) {%
        \texttt{Person::socsecnumber = ssn} \\
        \texttt{Employee::socsecnumber = ssn}
    };

    \node[below left= 0 of n1.north east, anchor=north east] {\textbf{1}};
    \node[below left= 0 of n2.north east, anchor=north east] {\textbf{2}};
    \node[below left= 0 of n3.north east, anchor=north east] {\textbf{3}};
    \node[below left= 0 of n4.north east, anchor=north east] {\textbf{4}};
    \node[below left= 0 of n5.north east, anchor=north east] {\textbf{5}};
    \node[below left= 0 of n6.north east, anchor=north east] {\textbf{6}};
\end{tikzpicture}
    \caption[Dual of the property graph for the running example]{Dual of the property graph for the \qvtr example in \autoref{lst:compatibility:qvtr_running_example} based on the relations in \autoref{fig:compatibility:three_persons_example_extended}. Adapted from~\owncite[Fig.~12]{klare2020compatibility-report}.}
    \label{fig:compatibility:dual_propertygraph_example}
\end{figure}

\mnote{Dual graph expressiveness}
The function $\function{v}$ labels each edge $\setted{E_1, E_2}$ in the dual with the set of properties that occur both in $E_1$ and $E_2$.
Since the property graph and its dual have equal expressiveness, the property graph can be constructed out of the dual again.
Given a dual $\mathcal{M^{*}} = \tupled{\mathcal{G}, \function{v}, \function{l}}$, the property graph $\mathcal{M} = \tupled{\mathcal{H}, \function{l}}$ can be built by defining $V_{\mathcal{H}} = \bigcup_{V \in V_{\mathcal{G}}} V$ and $E_{\mathcal{H}} = V_{\mathcal{G}}$. 

\mnote{Independence in property graphs}
Independence of consistency relations in the property graph is characterized by the existence of two (or more) subhypergraphs\footnote{A subhypergraph of a hypergraph $\mathcal{H} = (V_{\mathcal{H}}, E_{\mathcal{H}})$ is a hypergraph $\mathcal{S} = \tupled{V_{\mathcal{S}}, E_{\mathcal{S}}}$ such that $E_{\mathcal{S}} \subseteq E_{\mathcal{H}}$ and $V_{\mathcal{S}} = \bigcup_{E \in E_{\mathcal{S}}} E$} such that there is no path (i.e., sequence of incident hyperedges) from one to the other.
In the dual of the property graph, such a situation is represented by two subgraphs that are not connected to each other.
This conforms to the notion of connected components, which are maximal subgraphs such that any two vertices are connected by a path of edges and reflects the notion of independence given in \autoref{def:independence}.
Per definition, each subgraph relates disjoint sets of properties, as otherwise an edge between two vertices that contain an intersection of these properties would exist.
These property sets occur in independent sets of consistency relations, as otherwise there would be a vertex in the dual of the property graph for a hyperedge of the property graph that relates the properties that are linked by an \gls{OCL} expression and according \qvtr variables.
We use Tarjan's algorithm to compute the connected components of the dual of the property graph in linear time~\cite{tarjan1972depth}.
These independent subgraphs can be processed independently, since their compatibility composes (see~\autoref{theorem:independencecompatibility}).

\mnote{Trees in property graphs}
In addition to independence, \autoref{theorem:treecompatibility} stating that consistency relation trees are compatible also applies to the dual of the property graph.
When there are no two paths relating the same classes or properties, respectively, the notion of a consistency relation tree is fulfilled, thus the represented consistency relations are inherently compatible.
Consequently, if the dual of the property graph is only composed of independent trees, i.e., if it is a \emph{forest}, it is inherently compatible.

\mnote{Redundancy in property graphs}
Finally, \autoref{corollary:transitiveredundancycompatibility} has shown that adding left-equal redundant consistency relations to a compatible consistency relation set preserves its compatibility.
According to \autoref{def:leftequalredundancy} for redundancy, we consider a predicate and its representing hyperedge, respectively, redundant if there is another concatenation of predicates that are always fulfilled if the redundant one is fulfilled.
In the hypergraph, this conforms to an alternative sequence of hyperedges that represent those predicates, which relates the same properties as the possibly redundant one.
In our operationalization, we only consider the case when the exact same classes are related by both the possibly redundant and the alternative concatenation of predicates, although the definition only requires the classes at the left side to be equal.
The existence of such an alternative path is, however, only a necessary but not a sufficient condition.
The predicates must also relate the properties in the same way, as, for example, one predicate may ensure that two string attributes are equal, whereas an alternative sequence of predicates only ensures that they have the same length.
This is the reason why we perform a redundancy test for redundancy candidates given by such an alternative path, which we explain in the subsequent subsection.

\mnote{Alternative paths in property graphs}
An alternative path for a hyperedge $E$, which represents a predicate in the property graph, is a sequence of pairwise incident hyperedges, of which the first and last edge are incident to $E$.
In the dual of the property graph, these hyperedges are represented by vertices.
Thus, in the dual such an alternative sequence is given by a cycle including the vertex $E$.
Let $\sequenced{E, E_1, \dots, E_n, E}$ be the vertex sequence of such a cycle, then $\sequenced{E_1, \dots, E_n}$ is the alternative path.
The generation of redundant paths amounts to the enumeration of pairs $\tupled{E, \sequence{E}_i}$, where $E$ is a possibly redundant predicate, i.e., a vertex in the dual of the property graph, and $\sequence{E}_i$ is an alternative sequence of predicates that may replace $E$.
There may be multiple such possible alternative paths for a single predicate, thus all simple cycles in the dual of the property graph need to be considered.
The problem of finding all simple cycles in an undirected graph is called \emph{cycle enumeration}.

\begin{algorithm}
    \makeatletter
\algnewcommand{\LineComment}[1]{\Statex \hskip\ALG@thistlm \(// \) #1}
\makeatother

\begin{algorithmic}[1]
\Procedure{\function{RemoveRedundantPredicates}}{$\text{Dual}\ \mathcal{M}^{*}, \mathvariable{pred} \in V_{\mathcal{M}^{*}}$}
    \algindentskip
    \State $\setted{\mathvariable{base}_1,\dots,\mathvariable{base}_n} \gets \function{PatonAlgorithm}(\mathcal{M}^{*})$
    \State $\mathvariable{foundCycles} \gets \setted{\mathvariable{base}_1}$
    \State $\mathvariable{currentCycles} \gets \varnothing$, $\mathvariable{currentCycles}^{*} \gets \varnothing$
    \algblockskip
    
    \For{$base \in \setted{\mathvariable{base}_2, \dots, \mathvariable{base}_n}$}
        \algindentskip

        \For{$\mathvariable{foundCycle} \in \mathvariable{foundCycles}$}
            \State $\mathvariable{newCycle} \gets \mathvariable{foundCycle} \oplus \mathvariable{base}$
            \If{$\mathvariable{foundCycle} \cap \mathvariable{base} \neq \varnothing$}
                \State $\mathvariable{currentCycles} \gets \mathvariable{currentCycles} \cup \setted{\mathvariable{newCycle}}$
            \Else
                \State $\mathvariable{currentCycles}^{*} \gets \mathvariable{currentCycles}^{*} \cup \setted{\mathvariable{newCycle}}$
            \EndIf
        \EndFor
        \algblockskip

        \LineComment{Remove non-simple cycles from $\mathvariable{currentCycles}$}
        \For{$\mathvariable{cycle}_1, \mathvariable{cycle}_2 \in \mathvariable{currentCycles}$}
            \If{$\mathvariable{cycle}_2 \subset \mathvariable{cycle}_1$}
                \State $\mathvariable{currentCycles} \gets \mathvariable{currentCycles} \setminus \setted{\mathvariable{cycle}_1}$
                \State $\mathvariable{currentCycles}^{*} \gets \mathvariable{currentCycles}^{*} \cup \setted{\mathvariable{cycle}_1}$
            \EndIf
        \EndFor
        \algblockskip

        \LineComment{New valid cycles are in $\mathvariable{currentCycles} \cup \setted{\mathvariable{base}}$}
        \For{$\mathvariable{cand} \in \mathvariable{currentCycles} \cup \setted{\mathvariable{base}}$}\label{algo:compatibility:cycle_enumeration:line:newCycles}
            \If{$\mathvariable{pred} \in \mathvariable{cand} \land \function{IsRedundant}(\mathvariable{pred}, \mathvariable{cand})$}
                \State remove $\mathvariable{pred}$ and its incident edges from $\mathcal{M}^{*}$
                \State \textbf{break}
            \EndIf
        \EndFor
        \algblockskip

        \State $\mathvariable{foundCycles} \gets \mathvariable{foundCycles} \cup \mathvariable{currentCycles}$
        \State $\mathvariable{foundCycles} \gets \mathvariable{foundCycles} \cup \mathvariable{currentCycles}^{*} \cup \setted{\mathvariable{base}}$
        \State $\mathvariable{currentCycles} \gets \varnothing$, $\mathvariable{currentCycles}^{*} \gets \varnothing$
        \algindentskip
    \EndFor
    %\State \Return{$S$}
    \algindentskip
\EndProcedure
\end{algorithmic}
    \caption[Removal of redundant predicates]{Predicates removal. Adapted from~\owncite[Alg.~2]{klare2020compatibility-report}.}
    \label{algo:compatibility:cycle_enumeration}
\end{algorithm}

\mnote{Redundant relation removal algorithm}
\autoref{algo:compatibility:cycle_enumeration} implements the enumeration of alternative paths for predicates and their removal in case they are redundant.
The implementation of identifying a candidate predicate as actually redundant within a cycle is assumed to be available as an external function $\function{IsRedundant}$.
As discussed before, this allows us to plug in different implementations for the redundancy test, of which we depict one in the subsequent subsection.
The algorithm is mainly concerned with the enumeration of alternative paths.

\mnote{Cycle enumeration}
The algorithm relies on the computation of a \emph{cycles basis}, which is a set of simple cycles from which all other simple cycles of the graph can be derived by combination.
This cycle basis is computed using Paton's algorithm~\cite{paton1969algorithm}.
For a given predicate, the enumeration processes each cycle from the cycle basis and merges it with all cycles that have been processed so far. 
Every cycle is represented as a set of edges.
We denote the symmetric difference with the $\oplus$ sign, i.e., $A \oplus B$ is the set of edges that are in $A$ or in $B$ but not in both.
The set $\mathvariable{foundCycles}$ contains all linear combinations of cycles that have been processed so far.
Merged with cycles of the basis $\mathvariable{base}_1, \dots, \mathvariable{base}_n$, these linear combinations are used to merge more than two cycles of the basis.
In each iteration of \autoref{algo:compatibility:cycle_enumeration}, processing a new cycle $\mathvariable{base}$ from the cycle basis, new simple cycles are in $\mathvariable{currentCycles} \cup \setted{\mathvariable{base}}$. Edge-disjoint or non-simple cycles are stored in $\mathvariable{currentCycles}^*$.

\mnote{Redundancy test}
The redundancy test is performed in \autoref{algo:compatibility:cycle_enumeration:line:redundancyTest} whenever new cycles are generated.
It checks for the given predicate $\mathvariable{pred}$ whether one of the newly generated cycles is redundant, i.e., whether it contains $\mathvariable{pred}$ and whether $\mathvariable{pred}$ can be replaced by the concatenation of other predicates.
If the test succeeds for an alternative sequence of predicates, the candidate can be removed.
The algorithm then proceeds with further possibly redundant predicates.
It terminates as soon as all predicates have been tested.
If the connected component of the graph becomes a tree after a predicate removal, the dual of the connected component does not contain cycles anymore, thus no further redundancy tests have to be performed.
In the following, we discuss how such a redundancy test can be realized.


\subsection{Redundancy Check for Consistency Relations}
\label{chap:compatibility:practical_approach:redundancies}

\mnote{Redundancy test as black box}
We have so far considered the redundancy test of predicates in the decomposition procedure as a black box, which can be realized by any approach that is able to prove redundancy of predicates.
This fosters independent reuse of the proposed decomposition procedure and the redundancy test to be presented in the following.
\autoref{algo:compatibility:cycle_enumeration} contains the function \function{IsRedundant} that needs to realize this check.

\mnote{Redundancy undecidability}
Since \gls{OCL} expressions have equal expressiveness than fist-order logic, reasoning about satisfiability of \gls{OCL} constraints is undecidable~\cite{beckert2002ocltranslation}.
Deciding whether a predicate is redundant reduces to deciding satisfiability, which is why no strategy that always decides redundancy can be defined.
In the following, we first discuss how predicates can be generally compared to prove compatibility.
We then present an approach that translates \gls{OCL} constraints of the predicates into first-order formulae and applies a theorem prover.
Finally, we discuss the limitations of the approach especially arising from the translation to first-order logic and the use of a theorem prover.

\mnote{Redundancy notion for predicates}
A redundancy test takes a couple $\tupled{E, \sequenced{E_1, \dots, E_n}}$ and returns \textsc{true} whenever the predicate $E$ is proven to be redundant to the sequence of predicates $\sequenced{E_1, \dots, E_n}$.
Redundancy as defined in \autoref{def:leftequalredundancy} requires the set of consistency relations, which are defined by the predicates, to be equivalent with and without the redundant relation.
This especially means that removing the redundant relation must not weaken consistency, i.e., it must not lead to models being considered consistent without that relation that are not considered consistent with that relation.
This is equivalent for a property graph, in which a redundant predicate may not restrict consistency by considering a model with specific property values inconsistent that are considered consistent by an alternative sequence of predicates.
A predicate $E$ can thus only be removed if all instances matching the predicate also match predicates $\sequenced{E_1, \dots, E_n}$.
In fact, \autoref{def:leftequalredundancy} limits redundancy to relations in which the left-side classes are equal.
We do, however, only consider relations between the same sets of properties, thus being restricted to relations between the same sets of classes anyway.

\mnote{Extensional redundancy test}
In consequence, a redundancy test realizes the comparison of two sets of instances of models or, in particular, property values.
A predicate can, however, be fulfilled by an infinite number of property values, i.e., condition elements in terminology of consistency relations, such as consistency of person incomes and employee salaries by an infinite number of integer pairs.
An extensional element-wise comparison is thus generally impossible.

\mnote{Intensional redundancy test}
For that reason, we consider the intensional specification of consistency relations by means of \gls{OCL} constraints.
These constraints are annotated to the property graph as hyperedge labels.
The redundancy test can thus be realized by a static analysis of these labels and \qvtr relation conditions in \texttt{when} and \texttt{where} clauses.
One such strategy is the transformation of \gls{OCL} expressions into first-order logic and the reasoning about the resulting first-order formulae~\cite{beckert2002ocltranslation, berardi2005umlreasoning}.
We set up the first-order formulae such that they are valid, i.e., \textsc{true} under every possible interpretation, whenever the redundancy test is positive.
This transformation benefits from the availability of theorem proving tools for reasoning about first-order formulae.

\mnote{First-order logic representation}
Since first-order logic is generally undecidable, redundancy of a relation cannot be proven for every derived formula.
Thus, the result quality of the decomposition procedure depends on the quality of the theorem prover.
The transformation of \gls{OCL} to logic formulae requires a representation of all constructs, such as arithmetic operations, strings, arrays, etc., in formulae.
Objects, such as strings, floats, sequences, and others can be represented by theories of theorem provers.
With theories, the satisfiability problem equates to assigning values to variables in first-order logic sentences such that their evaluation returns \textsc{true}.
For example, the formula $(a \times b = 6) \land (a + b > 0)$ is satisfiable given the assignment $\setted{a = 2, b = 3}$.
This extension is known as \glsfirst{SMT}.
Formulae for the \gls{SMT} problem are called \emph{\gls{SMT} instances}.
\emph{Theory-based theorem provers} provide built-in theories, to which we translate \gls{OCL} constraints for our redundancy test.

\mnote{Redundancy of predicates}
The information that is necessary for a redundancy test is given by the predicates passed to the test.
Let $E = \tupled{\propertysettuple{P}{\classtuple{C}{l}}, \propertysettuple{P}{\classtuple{C}{r}}, \function{f}_E}$ be a predicate for two class tuples $\classtuple{C}{l}$ and $\classtuple{C}{r}$. 
During the construction of the property graph, a hyperedge composed of all properties in $\propertysettuple{P}{\classtuple{C}{l}}$ and $\propertysettuple{P}{\classtuple{C}{r}}$ is labeled with the description of the predicate function $\function{f}_E$.
Such a predicate $E$ can be replaced by a sequence of other predicates $\sequenced{E_1, \dots, E_n}$ if $\function{f}_E$ evaluates to \textsc{true} whenever $\function{f}_{E_1} \wedge \dots \wedge \function{f}_{E_n}$ evaluates to \textsc{true}.
In that case, the removal of the consistency relation given by $E$ does not weaken consistency, because it is fulfilled only when  the relation given by the concatenation of $\sequenced{E_1, \dots, E_n}$ is fulfilled anyway.
In consequence, $E$ is redundant.
This redundancy test can be encoded as a formula in the following way:
\begin{align*}
    &
    (\function{f}_{E_1} \wedge \dots \wedge \function{f}_{E_n}) \Rightarrow \function{f}_{E}
\end{align*}

\mnote{Horn clauses}
This formula is a \emph{Horn clause}.
According to common terminology, we call terms at the left-hand side of the clause \emph{facts} and the term at the right-hand side \emph{goal}.
The assignment of values to variables in the Horn clause also models the instantiation of properties, i.e., the assignment of property values.
If the Horn clause is valid, the alternative sequence of predicates can replace the other predicate for every instance.
Variables in Horn clauses are usually implicitly quantified universally.
Predicate functions of \gls{OCL} expressions, however, need to contain existentially quantified \qvtr variables, as the pattern matching of the expressions requires the existence of values for these variables.

\begin{example}
\autoref{fig:compatibility:dual_propertygraph_example} depicts the dual of the property graph for the motivational example in \autoref{lst:compatibility:qvtr_running_example}. 
It contains four connected components, of which three contain only one predicate. 
These three components are trivial trees, so compatibility for them is proven.
The other component consists of three predicates and contains a cycle ($\sequenced{1, 2, 3}$).
Let $3$ be the possibly redundant predicate.
Then, the alternative combination of predicates is composed of $1$ and $2$. 
This leads to the following formula with facts $1$ and $2$ and goal $3$:
\begin{align*}
    &
        \big(\propdisplay{Person.firstname} = \mathvariable{fstn1} \land \propdisplay{Person.lastname} = \mathvariable{lstn1}\\
    &
        \formulaskip \land \propdisplay{Resident.name} = \mathvariable{fstn1} + "\text{\textvisiblespace}" + \mathvariable{lstn1} \\
    & 
        \land 
            \propdisplay{Person.firstname} = \mathvariable{fstn2} \land \propdisplay{Person.lastname} = \mathvariable{lstn2}\\
    &
        \formulaskip \land \propdisplay{Employee.name} = \mathvariable{fstn2} + "\text{\textvisiblespace}" + \mathvariable{lstn2} \big)\\
    & 
        \Rightarrow
        \big(
            \exists n : (
                \propdisplay{Resident.name} = \mathvariable{n} \land \propdisplay{Employee.name} = \mathvariable{n}
            )
        \big)
\end{align*}
\qvtr variables have been renamed to avoid conflicts, because they are no longer isolated as they were before in distinct \qvtr relations.
The formula is valid and will be identified as such by an \gls{SMT} solver.
For that reason, predicate~$3$ can be removed from the property graph and its dual. 
Since the component then only consists of two predicates and thus forms a tree, the represented consistency relations are compatible.
Since all independent consistency relation sets, represented by the independent connected components of predicates, are compatible, the complete consistency specification is compatible.
\end{example}

\begin{figure}
    \centering
    \begin{tikzpicture}[
  oclbox/.style={draw, fill=lightlightgray, text width=9.3em, inner sep=3pt, align=left, anchor=center},
  z3box/.style={text width=8.8em, inner sep=3pt, align=center, anchor=center}
]
%% FRAMES
% \draw[dotted, anchor=west] (-0.3, -1) rectangle (6, 7.5);
%\draw[dotted, anchor=west] (7.2, -1) rectangle (12.3, 7.5);

\node (r1) at (0em,18em) [draw, darkgreen, thick, dashed, anchor=north west, minimum width=11.5em,minimum height=21em,inner sep=0em] {};
\node (r2) at ({12.5em+0.1*\difftoafiveimage},18em) [draw, darkgreen, thick, dashed, anchor=north west, minimum width=11em,minimum height=21em] {};
\draw[-latex, thick, darkgreen] (r1) -- (r2);

%% OCL EXPRESSIONS
\node[oclbox, name=persEmp] at (5.75em, 12.25em) {%
  $\mathvariable{Person.firstname} = \mathvariable{fstn}$\\
  $\land \; \mathvariable{Person.lastname} = \mathvariable{lstn}$\\
  $\land \; \mathvariable{Employee.name}$\\
  $\hspace{1em} = \mathvariable{fstn} + "\text{\textvisiblespace}" + \mathvariable{lstn}$%
};
    
\node[oclbox, name=empRes] at (5.75em, 4em) {%
  $\mathvariable{Resident.name} = n$\\
  $\land \; \mathvariable{Employee.name} = n$
};
      
\node[oclbox, name=persRes] at (5.75em, 0) {%
  $\mathvariable{Person.firstname} = \mathvariable{fstn}$\\
  $\land \; \mathvariable{Person.lastname} = \mathvariable{lstn}$\\
  $\land \; \mathvariable{Resident.name}$\\
  $\hspace{1em} = \mathvariable{fstn} + "\text{\textvisiblespace}" + \mathvariable{lstn}$
};
  
%% LABELS    
\node[above=0.4em of persEmp.north, anchor=south, align=center] (possiblyredundant_label) {Possibly\\ Redundant Predicate};
\node[above=0.4em of empRes.north, anchor=south, align=center] (alternative_label) {Alternative Sequence\\ of Predicates};

%% SMT FORMULAE
\node[name=smtPersEmp, right=12.25em+0.1*\difftoafiveimage of persEmp.center, yshift=-0.55em, z3box] {%
\begin{lstlisting}[language=framedQVT, basicstyle=\footnotesize\ttfamily]
(not
  (and
    (= firstname fstn)
    (= lastname lstn)
    (= name
      (str.++ fst
        (str.++ ' ' lstn)
      )
    )
  )
)
\end{lstlisting}
};
      
\node[name=smtEmpRes, right=12.25em+0.1*\difftoafiveimage of empRes.center, z3box] {%
\begin{lstlisting}[language=framedQVT, basicstyle=\footnotesize\ttfamily]
...
\end{lstlisting}
};
      
\node[name=smtPersRes, right=12.25em+0.1*\difftoafiveimage of persRes.center, z3box] {%
\begin{lstlisting}[language=framedQVT, basicstyle=\footnotesize\ttfamily]
...
\end{lstlisting}
};
      
\draw[-latex] (persEmp) to ([yshift=0.5em]smtPersEmp.west);
\draw[-latex] (empRes) to (smtEmpRes);
\draw[-latex] (persRes) to (smtPersRes);

\node[above=-1em of smtEmpRes] (wedge1) {$\land$};
\node[above=-1em of smtPersRes] (wedge2) {$\land$};
    
%% SMT SOLVER
\node[draw,rectangle, minimum width=4em, minimum height=3em, fill=darkblue!20, above right=9em and 1.25em+0.5*\difftoafiveimage of r2.east, anchor=west, align=center] (smtsolver) {SMT\\ Solver};
\draw[-latex, thick, darkgreen] ([yshift=9em]r2.east) -- (smtsolver);

%% OUTCOME
\node[text width=5.75em+0.7*\difftoafiveimage, font=\footnotesize, below=1.25em of smtsolver.south, anchor=north] (sat) {\textbf{SAT}. The initial Horn clause is not always valid, so the predicate is not entirely redundant.\\ $\rightarrow$ \emph{No removal}};
\node[text width=5.75em+0.7*\difftoafiveimage, font=\footnotesize, below=0.2em of sat.south west, anchor=north west] (unknown) {\textbf{UNKNOWN}. By conservativeness.\\ $\rightarrow$ \emph{No removal}};
\node[text width=5.75em+0.7*\difftoafiveimage, font=\footnotesize, below=0.2em of unknown.south west, anchor=north west] (unsat) {\textbf{UNSAT}. The initial Horn clause is valid, so the predicate is redundant.\\ $\rightarrow$ \emph{Removal}};

\draw[-latex] (smtsolver.south) -- (smtsolver.south|-sat.north);
%\draw[-latex] (smtsolver) -- (unknown);
%\draw[-latex] (smtsolver) -- (unsat);

\node[below = 0.2em of r1] {\textbf{OCL Expressions}};
\node[below = 0.2em of r2, name=smtf] {\textbf{SMT Formula}};
\node[anchor=center] at (smtf.center-|sat.center) {\textbf{Result}};

\end{tikzpicture}
    \caption[Redundancy test overview]{Overview of the redundancy test from \gls{OCL} expressions to the \gls{SMT} solver results. Adapted from~\owncite[Fig.~14]{klare2020compatibility-report}.}
    \label{fig:compatibility:redundancytest}
\end{figure}

\mnote{Satisfiability of negation}
Whenever such a Horn clause is valid, i.e., \textsc{true} under every interpretation, redundancy of the consistency relation represented by the predicate given as the clause goal is proven.
The \gls{SMT} solver takes the clause as an \gls{SMT} instance and verifies its satisfiability whenever possible.
Proving that a Horn clause $H$ is valid is equivalent to proving that its negation $\neg H$ is unsatisfiable.
Therefore, we actually let the \gls{SMT} solver prove that the \gls{SMT} instance $\function{f}_{E_1} \wedge \dots \wedge \function{f}_{E_n} \wedge \neg \function{f}_{E}$ is unsatisfiable.
The complete process of the redundancy test is depicted in \autoref{fig:compatibility:redundancytest}.
The solver can provide the following three results.
\begin{properdescription}
    \item[Satisfiable:] If $\neg H$ is satisfiable, $H$ is not valid. An interpretation exists, i.e., an instantiation of properties, that fulfills the possibly redundant predicate but not the alternative sequence of predicates. Thus, the predicate is not redundant and cannot be removed.
    \item[Unsatisfiable:] If $\neg H$ is unsatisfiable, $H$ is valid. Thus, when the alternative sequence is fulfilled, the predicate is fulfilled as well. It is redundant and can be removed.
    \item[Unknown:] First-order logic being undecidable, a theorem prover cannot evaluate satisfiability of all formulae, thus also returning \emph{Unknown}. To ensure conservativeness, the redundancy test is considered negative. As a result, the predicate is not removed.
\end{properdescription}

\mnote{Translation to logic formulae}
For the actual translation of \gls{OCL} expressions in \qvtr relations into \gls{SMT} instances, we refer to existing work on translating \gls{OCL} to first-order formulae~\cite{beckert2002ocltranslation} and, in particular, to our work presenting the specific translation for proving compatibility~\owncite{klare2020compatibility-report}.
\qvtr uses a subset of \gls{OCL} called \emph{EssentialOCL}~\cite{qvt}, which is a side-effect-free sublanguage that provides primitive data types, data structures and operations to express constraints on models.
Several \gls{OCL} constructs have a direct equivalent in theories of the theorem prover or can be mapped to a combination of primitive constructs.
We employ the SMT-LIB specification, which is a standard that provides an input language for \gls{SMT} solvers~\cite{smtlib2017}, and the Z3 theorem prover~\cite{z32008} to realize the redundancy test.
A complete reference of translated constructs has been developed in the Master's thesis of \textowncite{pepin2019ma}.

\mnote{Untranslatability}
In addition to undecidability of \gls{OCL}, some \gls{OCL} operations are said to be \emph{untranslatable}, because no mapping to features of \gls{SMT} solvers were found yet.
Thus, some \qvtr relations cannot be processed automatically by the proposed decomposition procedure.
For example, string operations like $\mathvariable{toLower}$ and $\mathvariable{toUpper}$ cannot be easily translated into logic formulae for \gls{SMT} solvers without several used-defined axioms.
Although decision procedures for such a case exist~\cite{veanes2012transducers}, they are not yet integrated into solvers.

\mnote{Practical approach summary}
In this section, we have discussed how the formal approach for proving compatibility as depicted in \autoref{algo:compatibility:formal_proof} can be realized for \qvtr.
We have defined a representation of consistency relations in graphs and explained how they can be derived from \qvtr transformations.
We have discussed how a consistency relation tree and independent relation sets manifest in such a graph and how candidates for redundancies can be found in it.
Finally, we have presented a redundancy test based on transforming \gls{OCL} expressions of potentially redundant relations into Horn clauses that are validated by \gls{SMT} solvers.

\section{Summary}

In this chapter, we have discussed the challenge regarding compatibility of consistency relations, which are encoded in transformations.
We have derived a well-founded notion of compatibility, precisely formalized this notion and presented a formal approach that is able to validate compatibility of given relations.
The approach is proven to be correct.
Based on the formal approach, we developed a practical approach for \gls{QVTR}, which is able to validate compatibility of consistency relations defined in \gls{QVTR} and \gls{OCL}.
We conclude this section with the following central insight.

\begin{insight}[Compatibility]
    Transformations that are supposed to preserve contradictory consistency relations easily lead to problems when combining them to a network, because (some of) their relations cannot be fulfilled at the same time.
    The relations preserved by transformations should thus be \emph{compatible}, i.e., they should not restrict consistency for elements such that no consistent set of models can be found by the transformation network.
    That notion of compatibility can be proven for given transformations by considering their preserved consistency relations, finding redundant relations and removing them until only a tree of relations remains. Since we were able to prove that consistency relation trees are inherently compatible and removing redundant relations is compatibility-preserving, this approach is proven correct.
    Compatibility is a property of the network and not a single transformation, thus it cannot be achieved by construction of the individual transformations but only analyzed for a given transformation network.
\end{insight}

%\todo{Discuss early that compatibility is a network property, not achievable by construction}
