\chapter{Orchestrating Transformation Networks 
    \pgsize{60 p.}
}
\label{chap:orchestration}

\mnote{Application function in transformation networks}
A transformation network is composed of transformations and an application function, which executes the transformations in an order determined by an orchestration function.
In the previous chapters, we have discussed how the individual transformations can be defined and which properties they have to fulfill to be properly usable in a transformation network.
In this chapter, we discuss how the combination of transformations, as the second essential part of a transformation network, can be realized by an application function.

\mnote{Correctness of application functions}
Although the behavior of an application function has already been defined in \autoref{def:applicationfunction}, we have shortly discussed that we cannot require correctness for such a function in the sense that it always yields consistent models for every given models and changes to them.
We will prove that statement and show that this can either be because there is no execution order of the given transformations that yields consistent models for given models and changes to them or, even if it exists, it may not be possible to find it.

\mnote{Requirements for transformations}
In this chapter, we thus discuss under which conditions we can require an application function to return consistent models.
We derive an algorithm that realizes an application function and prove that it is not possible to ensure its termination without further restrictions to the transformations or the cases in which the algorithm is expected to return consistent models.
The discussion of different restriction options gives us the insight that none of them is practically applicable, because they restrict expressiveness of transformations and transformation networks too much.
Thus, we finally propose an algorithm that operates conservatively, i.e., if it returns models they are consistent, but it may not always return consistent models although an execution order of transformations that yields them exists.
That algorithm is supposed to improve the ability of a transformation developer to identify why no execution order of transformations could be found although it existed.
We have envisioned this as the \emph{comprehensibility} property in \autoref{chap:introduction:consistency:orchestration}.

\mnote{Undecidability of orchestration problem, practical algorithm}
This chapter thus constitutes our contribution \autoref{contrib:correctness:orchestration}, which consists of four subordinate contributions:
a discussion of the design of an application function with possible bounds for the number of executions and a notion of optimality leading to the definition of the \emph{orchestration problem}; the derivation of an algorithm for an application function, for which we discuss termination, prove undecidability of the orchestration problem and discuss different strategies to restrict transformations such that the orchestration problem becomes decidable; a gradual definition of optimality of an application function and a discussion of its systematic improvement; and finally the proposal of an algorithm that operates conservatively based on well-defined properties that ensure its termination and help to find the reasons whenever no execution order of transformations yielding consistent models is found.
It answers the following research question:

\researchquestionrepeat{rq:correctness:orchestration}

\mnote{Benefits: systematic knowledge and concrete algorithm}
While existing approaches to orchestrate transformations are restricted to specific network topologies, our approach is supposed to not restrict the supported topology in any way.
Existing work proposes, for example, to define an execution order explicitly~\cite{pilgrim2008a, vanhooff2007UniTI-MODELS} or to derive a topological order~\cite{stevens2020BidirectionalTransformationLarge-SoSym}, which restricts the topologies to those in which a transformation needs to be executed only once.
We prove that it is not possible to orchestrate arbitrary transformations such that they always yield consistent models whenever that is possible, i.e., when an according execution order of the transformations exists.
We do, however, provide an algorithm that is able to process transformation networks of arbitrary topology, which follows a specific orchestration strategy: It does not necessarily find an execution order that yields consistent models whenever it exists, but instead is defined in way that it supports the transformation developer or user in finding the reason for the inability to find such an order.
On the one hand, this gives transformation developers systematic knowledge about limitations regarding the possibility to orchestrate transformations and, on the other hand, gives them an algorithm for the orchestration to be readily applied.

\mnote{Publication of contributions}
Selected insights presented in this chapter have been developed in a scientific internship together with Joshua Gleitze, which was supervised by the author of this thesis, and have already been published~\owncite{gleitze2020orchestration}.

% 1. Application function can return $\bot$ (or not terminate), what are the reasons that it needs to return $\bot$ or does not terminate?
% 1.1. In best case algorithm returns consistent models
% 1.2. If it does not, it may return $\bot$, return inconsistent models (is excluded by construction) or not terminate at all
% 1.3. When should it return $\bot$? Design Space for Functions
% 1.4. Why does it not terminate? Divergence/Alternation
% 2. Restriction to the Application Function
% 2.1. Execution Bound
% 2.2. Show Undecidability -> cannot find an orchestration although it exists
% 3. What restrictions can we make to transformation to, first, ensure that an orchestration exists and, second, it can be found?
% 3.1. Monotony

%\todo{Do we discuss somewhere that orchestration/application should be generic, independent from the concrete network? Maybe define this as the central goal in this section. The general problem in the notion chapter may also allow problem specific realizations. In fact, a parameterized function as in the formalization is specific to the transformations anyway. An algorithm, however, may be generic or specific. So discuss that property in the algorithm section.}

%\todo{Introduce term of \emph{interoperability} of transformations (i.e., that they select the same options and things like that), but rather informally}

\todoLater{Who returns $\bot$ when no order exists? Orc or app?}
%\todo{Define the "orchestration problem"} % and any short form of "orchestration that yields consistent models}

%\todo{Theorem: The shortest consistent orchestration can be arbitrarily long (unbounded executions)}

%\todo{Theorem for single execution: Only orchestrations in O(n) excludes consistent orchestrations}

\section{Orchestration Goals and Problem Statement}

\mnote{Application function results}
To recapitulate, an application function $\appfunction{\orcfunction{\transformationset{T}}}$ for transformation networks, as defined in \autoref{def:applicationfunction}, accepts models and changes to them and yields either a tuple of models or $\bot$.
Whenever it returns a tuple of models, they must be the result of applying the transformations in $\transformationset{T}$ of the network in an order determined by the orchestration function $\orcfunction{\transformationset{T}}$.
We then say that this execution order is an \emph{orchestration} of the transformations and that the execution of transformations in that order \emph{yields} those models.
The notion of correctness for the application function given in \autoref{def:applicationfunctioncorrectness} additionally requires the returned models to be consistent.
We did, however, not yet define when we expect the function to return consistent models and when we allow it to return $\bot$, as this requires a further discussion of the alternatives, which we provide in the following.

\mnote{Dependence on orchestration function}
The application function highly depends on the results of the orchestration function.
If that function does not deliver an orchestration that yields consistent models, a correct application function may only return $\bot$.
Thus, we are particularly concerned with ensuring that the orchestration function finds an orchestration that yields consistent models as often as possible.
We call an orchestration that yields consistent models a \emph{consistent orchestration}.
Precisely, we define an orchestration and a consistent orchestration as follows.
\begin{definition}[Orchestration]
    Let $\transformationset{T}$ be a transformation set.
    We call a sequence $\sequenced{\transformation{t}_{1}, \transformation{t}_{2}, \dots} \in \transformationset{T}^{< \mathbb{N}} \equalsperdefinition \bigcup_{i=0}^{\infty} \transformationset{t}^i$ of these transformations an \emph{orchestration} of them.

    For models $\modeltuple{m} \in \metamodeltuple{M}$ and changes $\changetuple{\metamodeltuple{M}} \in \changeuniverse{\metamodeltuple{M}}$, we say that an orchestration $\sequenced{\transformation{t}_{1}, \dots, \transformation{t}_{n}}$ is \emph{consistent} if, and only if, the subsequent application of the transformations to $\modeltuple{m}$ and $\changetuple{\metamodeltuple{M}}$ is consistent, i.e.:
    \begin{align*}
        &
        \exists \changetuple{\metamodeltuple{M}}' \in \changeuniverse{\metamodeltuple{M}} : 
        \big(
            \generalizationfunction{\metamodeltuple{M},\transformation{t}_{n}} \concatfunction \dots \concatfunction \generalizationfunction{\metamodeltuple{M},\transformation{t}_{1}}(\modeltuple{m}, \changetuple{\metamodeltuple{M}}) = (\modeltuple{m}, \changetuple{\metamodeltuple{M}}') \\
            & \formulaskip
            \land \changetuple{\metamodeltuple{M}}'(\modeltuple{m}) \consistenttomath \transformationset{T}
        \big)
    \end{align*}
\end{definition}

\mnote{Length of orchestrations}
The definition of an orchestration function allows it to determine an arbitrarily long sequence of transformations, also including each transformation multiple times.
We have introduced this general notion to avoid unnecessary restrictions.
In the following, we show the necessity of having this unrestricted notion rather than allowing each transformation to be executed only once, as proposed in existing work~\cite{stevens2020BidirectionalTransformationLarge-SoSym}.
From the insight that we need to allow transformations to be executed multiple times, we derive and discuss when we expect the application function to return consistent models to finally come up with a notion of \emph{optimality} for the orchestration function determining the execution order.
This leads to the definition of the central \emph{orchestration problem} that we want a transformation network to solve.


\subsection{Single Transformation Execution}

\mnote{Ranges for executions times}
The possible numbers of executions for transformations of a network range from a selected execution of a subset, e.g., in terms of an induced spanning tree, over the execution of each transformation for one or a fixed number of times, to an arbitrary number of executions per transformation.
In the following, we demonstrate why a single execution of each transformation is not sufficient in practice and prove that it is not sufficient in general.

\mnote{Spanning trees insufficient}
The even stronger restriction to spanning trees is obviously insufficient.
Consider the following consistency relations. For simplicity reasons, we use model-level relations (\autoref{def:modellevelconsistencyrelation}) instead of fine-grained ones:
\begin{align*}
    & 
    \consistencyrelation{CR}{12} = \setted{\tupled{\model{m}{1}, \model{m}{2}}, \tupled{\model{m}{1}, \model{m}{2}'}, \tupled{\model{m}{1}', \model{m}{2}'}, \tupled{\model{m}{1}', \model{m}{2}''}} \\
    & 
    \consistencyrelation{CR}{13} = \setted{\tupled{\model{m}{1}, \model{m}{3}}, \tupled{\model{m}{1}, \model{m}{3}''}, \tupled{\model{m}{1}', \model{m}{3}}, \tupled{\model{m}{1}', \model{m}{3}'}} \\
    & 
    \consistencyrelation{CR}{23} = \setted{\tupled{\model{m}{2}, \model{m}{3}}, \tupled{\model{m}{2}', \model{m}{3}'}, \tupled{\model{m}{2}', \model{m}{3}''}, \tupled{\model{m}{2}'', \model{m}{3}}} 
\end{align*}
This set of relations $\setted{\consistencyrelation{CR}{12}, \consistencyrelation{CR}{13}, \consistencyrelation{CR}{23}}$ is compatible according to \autoref{def:compatibility}, because for each model there is a containing tuple of models that is consistent.
For the initial tuple of models $\tupled{\model{m}{1}, \model{m}{2}, \model{m}{3}}$, we consider a change that changes $\model{m}{1}$ to $\model{m}{1}'$.
Then we can distinguish three possible spanning trees, each of two transformations that try to restore consistency.
We denote the transformations as $\transformation{t}_{12}$, $\transformation{t}_{13}$, and $\transformation{t}_{23}$ for the according consistency relations.
Each tree consists of two transformations:
\begin{properdescription}
    \item[$\transformation{t}_{12}$, $\transformation{t}_{13}$:] 
    $\transformation{t}_{12}$ may change $\model{m}{2}$ to $\model{m}{2}'$. $\transformation{t}_{13}$ does nothing, because $\model{m}{1}'$ and $\model{m}{3}$ are already consistent to $\consistencyrelation{CR}{13}$, but $\model{m}{2}'$ and $\model{m}{3}$ are not consistent to $\consistencyrelation{CR}{23}$.
    \item[$\transformation{t}_{12}$, $\transformation{t}_{23}$:] 
    Like before, $\transformation{t}_{12}$ may change $\model{m}{2}$ to $\model{m}{2}'$. 
    $\transformation{t}_{23}$ may then change $\model{m}{3}$ to $\model{m}{3}''$. 
    $\model{m}{1}'$ and $\model{m}{3}''$ are, however, not consistent to $\consistencyrelation{CR}{13}$.
    \item[$\transformation{t}_{13}$, $\transformation{t}_{23}$:]
    $\transformation{t}_{13}$ may do nothing, because $\model{m}{1}'$ and $\model{m}{3}$ are already consistent to $\consistencyrelation{CR}{13}$.
    $\transformation{t}_{23}$ does also nothing, because $\model{m}{2}$ and $\model{m}{3}$ are still consistent to $\consistencyrelation{CR}{23}$.
    $\model{m}{1}'$ and $\model{m}{2}$ are, however, not consistent to $\consistencyrelation{CR}{12}$.
\end{properdescription}

\mnote{Necessity to execute each transformation}
Thus, we need to execute each transformation at least once, because each transformation is only responsible for restoring consistency to its consistency relations.
We cannot expect the resulting models to be consistent if some transformations were not executed, although the involved models were changed by other transformations.
However, restricting the execution to each transformation once is not appropriate either.
To show that, we consider examples that we derived from those we have already presented in previous work~\owncite{gleitze2021orchestration-FASE}, which use a different scenario context.

\begin{figure}
    \centering
    \newcommand{\pcmclasswidth}{5.2em}
\newcommand{\umlclasswidth}{4.3em}
\newcommand{\vdistance}{(\pcmclasswidth+4.7em)}
\newcommand{\hdistance}{(\vdistance+0.4*\difftoafiveimage)}
\newcommand{\labeldistance}{1.0em}

\begin{tikzpicture}

% First
\umlclassvarwidth{pcm1_i}{}{\umlinterfacelabel\\ I}{
    \dots
}{\pcmclasswidth}
\umlclassvarwidth[, below=0.4*\vdistance of pcm1_i.north, anchor=north]{pcm1_c}{}{\umlcomponentlabel\\ C}{
}{\pcmclasswidth}

\umlclassvarwidth[, right=1\hdistance of pcm1_i.north, anchor=north]{uml1_i}{}{\umlinterfacelabel\\ I}{
    \dots
}{\umlclasswidth}
\umlclassvarwidth[, below=0.4*\vdistance of uml1_i.north, anchor=north]{uml1_c}{}{CImpl}{
    CImpl()
}{\umlclasswidth}

\node[above right=0.0*\vdistance and 1.25*\hdistance of uml1_i.north, anchor=north, text width=10em, inner sep=0em] (java1_code) {
\begin{lstlisting}[language=java, numbers=none, basicstyle=\footnotesize\ttfamily]
interface I { §\dots§ }

class CImpl implements I {
    CImpl() { §\dots§ }
}
\end{lstlisting}
};
\coordinate (java1_south) at ([yshift=1em]java1_code.south);

% Second
\node[below=0.22*\vdistance of java1_code.south, anchor=north, text width=10em, inner sep=0em] (java2_code) {
\begin{lstlisting}[language=java, numbers=none, basicstyle=\footnotesize\ttfamily]
interface I { §\dots§ }

class CImpl implements I {
    §\textcolor{consistencypreservationcolor}{I f;}§
    CImpl() { §\dots§ }
}
\end{lstlisting}
};

\umlclassvarwidth[, left=1.25*\hdistance of java2_code.north, anchor=north]{uml2_i}{}{\umlinterfacelabel\\ I}{
    \dots
}{\umlclasswidth}
\umlfullclassvarwidth[, below=0.4*\vdistance of uml2_i.north, anchor=north]{uml2_c}{}{CImpl}{
    \textcolor{consistencypreservationcolor}{f : I}
}{
    CImpl()
}{\umlclasswidth}

\umlclassvarwidth[, below left=0.52*\vdistance and 1*\hdistance of uml2_i.north, anchor=north]{pcm2_i}{}{\umlinterfacelabel\\ I}{
    \dots
}{\pcmclasswidth}
\umlclassvarwidth[, below=0.5*\vdistance of pcm2_i.north, anchor=north]{pcm2_c}{}{\umlcomponentlabel\\ C}{
}{\pcmclasswidth}
\pcmrequirerole{pcm2_c}{pcm2_i}{right, stereotype, consistencypreservationcolor}

% Third
\umlclassvarwidth[, below right=0.52*\vdistance and 1*\hdistance of pcm2_i.north, anchor=north]{uml3_i}{}{\umlinterfacelabel\\ I}{
    \dots
}{\umlclasswidth}
\umlfullclassvarwidth[, below=0.4*\vdistance of uml3_i.north, anchor=north]{uml3_c}{}{CImpl}{
    f : I
}{
    CImpl(\textcolor{consistencypreservationcolor}{f : I})
}{\umlclasswidth}

\node[above right=0.0*\vdistance and 1.25*\hdistance of uml3_i.north, anchor=north, text width=10em, inner sep=0em] (java3_code) {
\begin{lstlisting}[language=java, numbers=none, basicstyle=\footnotesize\ttfamily]
interface I { §\dots§ }

class CImpl implements I {
    I f;
    CImpl(§\textcolor{consistencypreservationcolor}{I f}§) { §\dots§ }
}
\end{lstlisting}
};

\node[mmlabel, above=\labeldistance of pcm1_i.north, anchor=center, font=\small\bfseries] (pcm_label) {\acrshort{PCM}};
\node[mmlabel, above=\labeldistance of uml1_i.north, anchor=center, font=\small\bfseries] (uml_label) {\acrshort{UML}};
\node[mmlabel, anchor=south, font=\small\bfseries] (uml_label) at (pcm_label.south-|java1_code.north) {Java};

\draw[correspondence] (pcm1_i) -- (pcm1_i-|uml1_i.west);
\draw[correspondence] (pcm1_c) -- (pcm1_c-|uml1_c.west);
\draw[correspondence] (uml1_i) -- ([xshift=-0.1em,yshift=1.9em]java1_code.west);
\draw[correspondence] (uml1_c) -- ([xshift=-0.1em,yshift=-0.2em]java1_code.west);

\draw[consistency execution] 
    ([xshift=-1em]java1_code.south)
    --
    node[right, align=center] {add field} % \\ \enquote{\texttt{I f}}}
    ([xshift=-1em]java2_code.north);

\draw[consistency execution] 
    ([xshift=-0.5em]java2_code.west)
    --
    node[above, align=center] {add field} % \\ \enquote{\texttt{f : I}}}
    ([xshift=0.5em]uml2_i.east|-java2_code.west);
\draw[consistency execution] 
    ([xshift=-0.5em]uml2_i.west|-java2_code.west)
    -|
    node[above left=0.3em and 0em, pos=1, align=center] {add\\ requi-\\ res}
    ([xshift=0.63*\hdistance,yshift=-0.1*\vdistance]pcm2_i.south)
    --
    ([xshift=0.1*\hdistance+0.5*\pcmclasswidth,yshift=-0.1*\vdistance]pcm2_i.south);
\draw[consistency execution] 
    ([xshift=0.63*\hdistance,yshift=-0.1*\vdistance]pcm2_i.south)
    |-
    node[pos=0.4, below left=0em and -0.8em, align=center] {add\\ constructor\\ parameter}
    ([xshift=-0.5em]uml3_i.west|-java3_code.west);
\draw[consistency execution]
    ([xshift=0.5em]uml3_i.east|-java3_code.west)
    --
    node[above, align=center] {add\\ constructor\\ parameter}
    ([xshift=-0.5em]java3_code.west);

\end{tikzpicture}
    %\includegraphics[width=\textwidth]{figures/correctness/orchestration/necessity_multiple_executions.png}
    \caption[Necessity of executing a transformation multiple times]{Necessity of executing a transformation multiple times. For initially consistent models, the Java code is changed, requiring the \acrshort{UML} and \acrshort{PCM} models to be updated accordingly. Blue lines without arrowheads connect initially corresponding elements, and green lines with arrowheads indicate changes performed by a user or consistency preservation.}
    \label{fig:orchestration:necessity_multiple_executions}
\end{figure}

\mnote{Example scenario for \gls{PCM}, \gls{UML} and Java}
Consider the example in \autoref{fig:orchestration:necessity_multiple_executions}, which depicts the introductory one of \autoref{fig:introduction:scenario_duplicate_execution} more precisely.
In the example, interfaces in the \gls{UML} and Java are related to architectural interfaces in a \gls{PCM} model.
\gls{PCM} components are realized by equally named classes in the \gls{UML} and Java.
Additionally, when a \gls{PCM} component requires an interface, this is realized by a field of the interface type and an appropriate constructor parameter in the component realization class in the \gls{UML} and Java.
Consistency is defined by transformations between \gls{PCM} and \gls{UML}, as well as between \gls{UML} and Java.

\mnote{Duplicate execution of transformation}
In the scenario in \autoref{fig:orchestration:necessity_multiple_executions}, we begin with a consistent state of one interface and component, each realized by an interface and class, respectively, in both the \gls{UML} and Java.
A user then introduces a change of the Java code, in which he or she adds a field of the interface type to the component realization class.
The transformation between \gls{UML} and Java propagates this change to the \gls{UML} model, such that both models are consistent again.
The transformation between \gls{PCM} and \gls{UML} then detects that the added field is of the type of an architectural interface, thus representing a requires relation between the corresponding component and the architectural interface. 
It adds the appropriate requires relation to the \gls{PCM} model but also adds an appropriate parameter to the constructor of the component realization class in the \gls{UML}.
This introduces a further inconsistency between the \gls{UML} and the Java model, which requires the transformation between \gls{UML} and Java to be executed again to also add that constructor parameter in the Java code.

\mnote{Cycles in transformation networks}
We have simplified the example to the necessary core, although in practice a further transformation between \gls{PCM} and Java may be required, e.g., to ensure that the field is set within the constructor.
One might argue that having such a cycle in the graph induced by the transformations between \gls{PCM}, \gls{UML}, and Java resolves the problem, as the second execution of the transformation between \gls{UML} and Java is not necessary if the information is propagated from the \gls{PCM} to Java.
This is, however, only true if exactly this execution order is chosen and if the transformation between \gls{PCM} and Java does not add further information to the Java model that must then be propagated to the \gls{UML}.

\mnote{Synchronizing transformations change already processed models}
In general, it is always possible that transformations need to react to the changes performed by other transformations if they are not in some way aligned with each other.
This is because a synchronizing transformation may change both models.
Thus, if one transformation restores consistency between two models and another transformation reacts to this by restoring consistency between one of these models and another one, then both these models become changed, which requires the first transformation to process the newly created changes again.

\begin{figure}
    \centering
    \newcommand{\distance}{13.8em}
\newcommand{\classwidth}{2em}

\begin{tikzpicture}

\umlclassvarwidth{A}{}{A}{
n
}{\classwidth}
\umlclassvarwidth[, right=\distance of A.north, anchor=north]{B}{}{B}{
n
}{\classwidth}
\umlclassvarwidth[, right=\distance of B.north, anchor=north]{C}{}{C}{
n
}{\classwidth}

\draw[consistency relation]
    ([yshift=0.7em]A.east)
    --
    node[pos=0, below right] {$a$}
    node[pos=1, below left] {$b$}
    node[above, align=center] {$\consistencyrelation{CR}{AB} = \{\tupled{a,b} \mid a.n, b.n \geq 0$ \\ 
        $\land \; b.n = a.n + 1 \land b.n \neq x\}$}
    ([yshift=0.7em]B.west);
\draw[consistency relation]
    ([xshift=0.7em]A.south)
    --
    node[pos=0, below left] {$a$}
    ++(0,-0.4*\distance)
    -|
    node[pos=1, below right] {$c$}
    node[above, pos=0.25] {$\consistencyrelation{CR}{AC} = \setted{\tupled{a,c} \mid a.n = c.n}$}
    ([xshift=-0.7em]C.south);
\draw[consistency relation]
    ([yshift=0.7em]B.east)
    --
    node[pos=0, below right] {$b$}
    node[pos=1, below left] {$c$}
    node[above, align=center] {$\consistencyrelation{CR}{BC} = \setted{\tupled{b,c} \mid b.n = c.n}$}
    ([yshift=0.7em]C.west);


\draw[transformation]
    ([yshift=-0.7em]A.east)
    --
    node[below, align=left] {$+a \xrightarrow{a.n \;\neq \; x-1} +b(n = a.n + 1)$\\
        $+b \xrightarrow{b.n \; \neq \; x} +a(n = b.n - 1)$}
    ([yshift=-0.7em]B.west);
\draw[transformation]
    ([xshift=-0.7em]A.south)
    --
    ++(0,-0.4*\distance-1.4em)
    -|
    node[below, pos=0.25, align=left] {$+a \rightarrow +c(n = a.n)$\\
        $+c \rightarrow +a(n = c.n)$}
    ([xshift=0.7em]C.south);
\draw[transformation]
    ([yshift=-0.7em]B.east)
    --
    node[below, align=left] {$+b \rightarrow +c (n = b.n)$\\
        $+c \rightarrow +b (n = c.n)$}
    ([yshift=-0.7em]C.west);

\end{tikzpicture}
    %\includegraphics[width=\textwidth]{figures/correctness/orchestration/no_upper_bound_example.png}
    \caption[Example for arbitrary bounds of transformation execution]{Example of consistency relations and associated transformations with an arbitrary bound of necessary transformation executions depending on the value of $x$.}
    \label{fig:orchestration:no_upper_bound}
\end{figure}

\mnote{Example generalization}
We can generalize the previous example to the one of \autoref{fig:orchestration:no_upper_bound}.
It is an extension of the example given in \autoref{fig:synchronization:multiple_unidirectional_execution} for the necessity to execute the consistency preservation rules of a bidirectional transformation multiple times.
This also applies to the case in which multiple synchronizing transformations are combined.
The depicted relations and the sketched consistency preservation rules require that elements \modelelement{A}, \modelelement{B}, and \modelelement{C} with the same value of $n$ exist, and that for each \modelelement{A} with value $n$, a \modelelement{B} and \modelelement{C} with $n$ incremented by $1$ exist except for the case that $n = x-1$.
Thus, for an \modelelement{A} with $n = i$, all \modelelement{A}, \modelelement{B}, and \modelelement{C} with $i \leq n < x$ must exist.
This, obviously, requires the transformations to be executed $x-1-i$ times.

\mnote{Definition of example transformation network}
We prove the informally given statement with the following precise definition of the transformations for a fixed but arbitrary value of $x$.
Let \modelelement{A}, \modelelement{B}, and \modelelement{C} be the classes depicted in \autoref{fig:orchestration:necessity_multiple_executions}.
\parameterizeformat{
\begin{align*}
    & 
    \metamodelinstanceset{M}{1} \equalsperdefinition \mathcal{P}(\instances{\class{A}{}}), \;
    \metamodelinstanceset{M}{2} \equalsperdefinition \mathcal{P}(\instances{\class{B}{}}), \;
    \metamodelinstanceset{M}{3} \equalsperdefinition \mathcal{P}(\instances{\class{C}{}}) \\[0.5em]
    &
    \consistencyrelation{CR}{12} \equalsperdefinition \setted{\tupled{a,b} \in \instances{\class{A}{}} \times \instances{\class{B}{}} \mid a.n, b.n \ge 0 \land b.n = a.n + 1 \neq x}\\
    &
    \consistencyrelationset{CR}_{12} \equalsperdefinition \setted{\consistencyrelation{CR}{12}, \consistencyrelation{CR}{12}^T} \\
    &
    \consistencypreservationrule{\consistencyrelationset{CR}_{12}}^{\rightarrow}(\model{m}{1}, \model{m}{2}, \change{\metamodel{M}{1}}) \equalsperdefinition \change{\metamodel{M}{2}} \\
    & \formulaskip
    \withmath \change{\metamodel{M}{2}}(\model{m}{2}) \equalsperdefinition \setted{b \in \instances{\class{B}{}} \mid \exists a \in \change{\metamodel{M}{1}}(\model{m}{1}) : b.n = a.n + 1 \neq x} \\
    & 
    \consistencypreservationrule{\consistencyrelationset{CR}_{12}}^{\leftarrow}(\model{m}{2}, \model{m}{1}, \change{\metamodel{M}{2}}) \equalsperdefinition \change{\metamodel{M}{1}} \\
    & \formulaskip
    \withmath \change{\metamodel{M}{1}}(\model{m}{1}) \equalsperdefinition \setted{a \in \instances{\class{A}{}} \mid \exists b \in \change{\metamodel{M}{2}}(\model{m}{2}) : b.n = a.n + 1 \neq x \land a \geq 0} \\
    &
    \transformation{t}_{12} \equalsperdefinition \tupled{\consistencyrelationset{CR}_{12}, \consistencypreservationrule{\consistencyrelationset{CR}_{12}}^{\rightarrow}, \consistencypreservationrule{\consistencyrelationset{CR}_{12}}^{\leftarrow}} \\[0.5em]
    & 
    \consistencyrelation{CR}{13} \equalsperdefinition \setted{\tupled{a,c} \in \instances{\class{A}{}} \times \instances{\class{C}{}} \mid a.n = c.n}, \;
    \consistencyrelationset{CR}_{13} \equalsperdefinition \setted{\consistencyrelation{CR}{13}, \consistencyrelation{CR}{13}^T} \\
    & 
    \consistencypreservationrule{\consistencyrelationset{CR}_{13}}^{\rightarrow}(\model{m}{1}, \model{m}{3}, \change{\metamodel{M}{1}}) \equalsperdefinition \change{\metamodel{M}{3}} #2
    \withmath \change{\metamodel{M}{3}}(\model{m}{3}) \equalsperdefinition \setted{c \in \instances{\class{C}{}} \mid \exists a \in \change{\metamodel{M}{1}}(\model{m}{1}) : a.n = c.n} \\
    & 
    \consistencypreservationrule{\consistencyrelationset{CR}_{13}}^{\leftarrow}(\model{m}{3}, \model{m}{1}, \change{\metamodel{M}{3}}) \equalsperdefinition \change{\metamodel{M}{1}} #2
    \withmath \change{\metamodel{M}{1}}(\model{m}{1}) \equalsperdefinition \setted{a \in \instances{\class{A}{}} \mid \exists c \in \change{\metamodel{M}{3}}(\model{m}{3}) : a.n = c.n} \\
    & 
    \transformation{t}_{13} \equalsperdefinition \tupled{\consistencyrelationset{CR}_{13}, \consistencypreservationrule{\consistencyrelationset{CR}_{13}}^{\rightarrow}, \consistencypreservationrule{\consistencyrelationset{CR}_{13}}^{\leftarrow}} \\[0.5em]
    &
    \consistencyrelation{CR}{23} \equalsperdefinition \setted{\tupled{b,c} \in \instances{\class{B}{}} \times \instances{\class{C}{}} \mid b.n = c.n}, \;
    \consistencyrelationset{CR}_{23} \equalsperdefinition \setted{\consistencyrelation{CR}{23}, \consistencyrelation{CR}{23}^T} \\
    & 
    \consistencypreservationrule{\consistencyrelationset{CR}_{23}}^{\rightarrow}, \consistencypreservationrule{\consistencyrelationset{CR}_{23}}^{\leftarrow}, \andmath \transformation{t}_{23} \mathtextspacearound{accordingly} \\[0.5em]
    &
    \consistencyrelationset{CR} \equalsperdefinition \consistencyrelationset{CR}_{12} \cup \consistencyrelationset{CR}_{13} \cup \consistencyrelationset{CR}_{23} \\
    &
    \transformationset{T}_\mathvariable{inc} \equalsperdefinition \setted{\transformation{t}_{12}, \transformation{t}_{13}, \transformation{t}_{23}}
\end{align*}
}{}{\\ & \formulaskip}%

\mnote{Minimal number of executions in example}
For these transformations, we can show that the transformation $\transformation{t}_{12}$ needs to be executed a minimal number of times depending on $x$ for a specific input.
Thus, it is not sufficient to execute each transformation only once in this network, and, even worse, we can enforce the necessity for an arbitrary high number of executions by proper selection of $x$.

\begin{lemma}[Minimal Number of Transformation Executions]
    \label{lemma:minimal_executions}
    Let $\transformationset{T}_\mathvariable{inc}$ be the previously defined transformation set, let $\model{m}{1} = \model{m}{2}= \model{m}{3} = \emptyset$ be empty models, and let $\change{\metamodel{M}{1}} \in \changeuniverse{\metamodel{M}{1}}$ be a change with $\change{\metamodel{M}{1}}(\model{m}{1}) = \setted{a \in \instances{\class{A}{}} \mid a.n = 0}$.
    Then every orchestration function $\orcfunction{\transformationset{T}_\mathvariable{inc}}$ with $\appfunction{\orcfunction{\transformationset{T}_\mathvariable{inc}}}(\tupled{\model{m}{1},\model{m}{2},\model{m}{3}},\tupled{\change{\metamodel{M}{1}},\identitychange,\identitychange}) \consistenttomath \consistencyrelationset{CR}$ yields an orchestration that contains $\transformation{t}_{12}$ at least $x-1$ times.
\end{lemma}

\begin{proof}
    $\appfunction{\orcfunction{\transformationset{T}_\mathvariable{inc}}}$ can only return consistent models when it applies the transformations in the order delivered by $\orcfunction{\transformationset{T}_\mathvariable{inc}}$ by \autoref{def:applicationfunction}.
    We thus consider every orchestration, as delivered by any orchestration function, to show that it contains $\transformation{t}_{12}$ at least $x-1$ times to deliver consistent models.
    
    Let $\mathvariable{max}_n(\model{m}{1},\model{m}{2},\model{m}{3}) \equalsperdefinition \mathvariable{max}\setted{e.n \mid e \in \model{m}{1} \cup \model{m}{2} \cup \model{m}{3}}$ be the maximal value of $n$ among all instances of \modelelement{A}, \modelelement{B}, and \modelelement{C} in the given models $\model{m}{1}$, $\model{m}{2}$, and $\model{m}{3}$. In the following, we shortly note $\mathvariable{max}_n$ whenever the actual models are not relevant. We show three statements that together prove the lemma.

    \begin{properdescription}
        \item[Executing $\transformation{t}_{13}$ and $\transformation{t}_{23}$ does not increase $\mathvariable{max}_n$:]
        The transformations only ensure that for given models the returned models contain all elements with the same values of $n$ and do not introduce new elements with values of $n$ larger than the existing ones.
        \item[One execution of $\transformation{t}_{12}$ increases $\mathvariable{max}_n$ by at most $1$:]
        There is no \modelelement{A} or \modelelement{B} with $n > \mathvariable{max}_n$.
        For every \modelelement{A} with $n < \mathvariable{max}_n$, $\transformation{t}_{12}$ creates, if necessary, a \modelelement{B} with value $n + 1 \leq \mathvariable{max}_n$, thus not increasing $\mathvariable{max}_n$.
        For every \modelelement{B} with $n \leq \mathvariable{max}_n$, it creates, if necessary, an \modelelement{A} with value $n-1 < \mathvariable{max}_n$.
        For every \modelelement{A} with $n = \mathvariable{max}_n$, a \modelelement{B} with value $n+1 = \mathvariable{max}_n + 1$ is created, as long as $n \neq x-1$.
        For the newly created \modelelement{B}, no further elements need to be created to fulfill the relations.
        Thus, $\mathvariable{max}_n$ is, at most, increased by $1$.
        \item[$\mathvariable{max}_n(\model{m}{1},\model{m}{2},\model{m}{3}) < x-1 \Rightarrow \tupled{\model{m}{1},\model{m}{2},\model{m}{3}} \mathtextspacearound{inconsistent to} \consistencyrelationset{CR}$:]
        There is at least one element with $n = \mathvariable{max}_n$ within the models.
        If the element with $n = \mathvariable{max}_n$ is an \modelelement{A}, there must be a \modelelement{B} with value $n+1$ due to $\consistencyrelationset{CR}_{12}$ and $n < x-1$.
        But due to $n = \mathvariable{max}_n$ such a \modelelement{B} cannot exist, because otherwise $\mathvariable{max}_n = n+1$, so this is a contradiction. 
        If the element with $n = \mathvariable{max}_n$ is a \modelelement{C}, $\consistencyrelationset{CR}_{13}$ requires an \modelelement{A} with the same value of $n$ to exist and the same argument as before leads to a contradiction.
        Finally, if the element with $n = \mathvariable{max}_n$ is a \modelelement{B}, then because of $\consistencyrelationset{CR}_{23}$, a \modelelement{C} with the same value must exist and the same argument as before leads to a contradiction.
    \end{properdescription}

    In summary, we have shown that models $\model{m}{1}$, $\model{m}{2}$, and $\model{m}{3}$ are only consistent to $\consistencyrelationset{CR}$ when $\mathvariable{max}_n(\model{m}{1},\model{m}{2},\model{m}{3}) \geq x-1$.
    Additionally, only $\transformation{t}_{12}$ increases $\mathvariable{max}_n$ and with each execution it only increases it by at most $1$.
    In consequence, starting with $\mathvariable{max}_n = 0$, we need at least $x-1$ executions of $\transformation{t}_{12}$ in an arbitrary sequence of the transformations in $\transformationset{T}_\mathvariable{inc}$ to achieve consistent models.
\end{proof}

\mnote{Arbitrary number of necessary executions}
We have proven that transformation networks can require an arbitrary high number of executions of each transformation.
By selecting an appropriate $x$ in the example network, we can force the network to perform at least $x-1$ executions of one transformation to yield a consistent tuple of models.
With this insight, it directly follows that we cannot find an approach to define orchestration functions that deliver sequences containing each transformation only once if we want to ensure that the approach delivers a consistent orchestration of transformations if it exists. 

\begin{theorem}[Orchestration with Single Execution]
    \label{theorem:orchestration_single}
    For a set of transformations $\transformationset{T}$, there can be models $\modeltuple{m}$ and changes $\changetuple{}$ to them for which each possible orchestration function $\orcfunction{\transformationset{T}}$ with whom $\appfunction{\orcfunction{\transformationset{T}}}(\modeltuple{m}, \changetuple{})$ is consistent delivers a sequence as $\orcfunction{\transformationset{T}}(\modeltuple{m}, \changetuple{})$ that contains at least one transformation twice.
\end{theorem}

\begin{proof}
    According to \autoref{lemma:minimal_executions}, $\transformationset{T}_\mathvariable{inc}$ requires at least two executions of $\transformation{t}_{12}$ for the inputs in \autoref{lemma:minimal_executions} and $x \geq 3$.
    This proves the theorem by example.
\end{proof}

\mnote{Single execution insufficient}
Of course, for a specific set of transformations it may be possible that there is an orchestration for all possible models and changes to them leading to a consistent state and only requiring each transformation to be executed once.
\autoref{theorem:orchestration_single} shows, however, that this cannot be assumed in general.
If we execute each transformation only once, we may exclude cases for which multiple executions of transformations would have led to a consistent tuple of models.
The example we have given in \autoref{fig:orchestration:necessity_multiple_executions} is a simplification of a realistic transformation scenario, which we generalized to the previous network with transformations $\transformationset{T}_\mathvariable{inc}$.
For that reason, the insight is likely to be relevant in realistic scenarios.
We should not restrict orchestration to execute each transformation only once, as there can be realistic scenarios that require multiple executions to find consistent models.
In the following, we thus allow an arbitrary number of executions of each transformation.

\mnote{Authoritative models}
In addition, the examples, both the concrete one and the generalized abstract one, demonstrate that it can be necessary to modify the model that was originally changed by the user again.
This contradicts the notion of \emph{authoritative} models as, for example, introduced by \textcite{stevens2020BidirectionalTransformationLarge-SoSym}.
With that notion, specific models are defined authoritative and cannot be changed, for example, because they are immutable or because they were changed by the user, and reverting those changes shall be avoided.
While that behavior may be a desired, forbidding the modification of a whole model is not a proper solution as shown in the examples, which is why we do not consider a notion of authoritative models.


\subsection{Orchestration Function Behavior}
\label{chap:orchestration:problem:function_behavior}

\mnote{Returning $\bot$}
An application function is defined to return models only when they can be derived by applying transformations in an order delivered by the orchestration function and otherwise to return $\bot$.
In addition, we expect a \emph{correct} application function only to deliver consistent models.
We have, however, not yet defined under which conditions we expect the function not to return $\bot$, because there are different reasons why the function may not be able to deliver consistent models, although we could expect it to do so.
In fact, with the current definition, the function is even considered correct if it always returns $\bot$, which is obviously not practical.
Thus, we need to define when exactly we expect the function to return $\bot$.

\mnote{Reasons for not finding consistent orchestration}
It might be intuitive to expect an application function to always return consistent models when the input models are consistent and when there is an execution order of the transformations, i.e., an orchestration, that delivers consistent models.
This, in consequence, would lead to the requirement that the orchestration function delivers a sequence of transformations whose application delivers consistent models whenever such a sequence exists for the given models and changes to them.
There can be the following reasons why the orchestration function may not deliver such a sequence.
\begin{properdescription}
    \item[Incompatible Relations:] If the consistency relations are incompatible, a user change may introduce an element for which no consistent models exist. In consequence, the transformations cannot be executed in an order returning models that are consistent and still reflect the user change.
    \item[No Consistent Orchestration Exists:] Even if relations are compatible, transformations may be defined in a way that they make contradictory decisions for locally consistent solutions. Thus, for a given change the consistency relations provide different ways of restoring consistency, of which each transformation selects one that is not consistent to one of the other relations.
    Then, no order of the transformations can restore consistency, although consistent models exist for the given change.
    \item[No Consistent Orchestration Found:] Even if an order of transformations for given changes that delivers consistent models exists, the orchestration function may not deliver it. 
\end{properdescription}

\mnote{Implication hierarchy of reasons}
These reasons can be considered to reside at different levels, because each of them induces the next.
This means, if there is no orchestration, it cannot be found, and having contradictory relations, there exists no orchestration for some of the changes.
In the end, all of them lead to the situation that no orchestration can be found and, thus, the orchestration function is not able to deliver it.

\mnote{Assumption of consistent orchestration existence}
The intuitive requirement that the orchestration function delivers a consistent orchestration whenever it exists would ensure the third level and needs to assume fulfillment of the first two levels to avoid situations in which no consistent orchestration is found.
While we can assume compatibility of the relations, for which we proposed an analysis in \autoref{chap:compatibility}, we cannot assume that an orchestration does always exists, as we see in the following.

\mnote{Compatibility not ensuring consistent orchestration}
Although compatibility reduces the chance that an orchestration function does not deliver a consistent orchestration, as we have motivated with the scenario depicted in \autoref{fig:compatibility:unwanted_behavior}, it does not ensure that there is always such a sequence of transformations that the orchestration function can find.
In general, this is always the case when consistency relations define different options for consistency, i.e., if they allow the existence of different corresponding elements to consider the models consistent.
Compatibility ensures that there is an overlap of these corresponding elements, such that for every element, for which consistency is restricted, consistent models can be found.
If, however, the transformations always restore consistency by introducing corresponding elements that are not in this overlap, each transformation will restore consistency locally to its consistency relation, but they can, together, never restore consistency to all consistency relations.

\mnote{Overlapping options in consistency relations}
Consider the situation that we have three metamodels $\metamodel{A}{}$, $\metamodel{B}{}$, and $\metamodel{C}{}$ with instances $\model{a}{i}$, $\model{b}{k}$, and $\model{c}{l}$.
Let us assume that these models are uniquely indexed by $i$, $k$, and $l$, and that we defined the following \modellevelconsistencyrelations:
\begin{align*}
    &
    \consistencyrelation{CR}{AB} = \setted{\tupled{\model{a}{i}, \model{b}{k}} \mid k = i} \\
    &
    \consistencyrelation{CR}{AC} = \setted{\tupled{\model{a}{i}, \model{c}{l}} \mid l = i \lor l = i+1} \\
    &
    \consistencyrelation{CR}{BC} = \setted{\tupled{\model{b}{k}, \model{c}{l}} \mid l = k+1 \lor l = k+2}
\end{align*}
This induces the model tuples $\setted{\tupled{\model{a}{i}, \model{b}{k}, \model{c}{l}} \mid  i = k = l-1}$, which are consistent to all three consistency relations.
Thus, for any given model we are able to find instances of the other metamodels that are consistent to all consistency relations.
If we define consistency preservation rules for these consistency relations, the ones for $\consistencyrelation{CR}{AC}$ and $\consistencyrelation{CR}{BC}$ may decide between two models to restore consistency, because their conditions define two options for consistent models.
The set of consistent models, however, contains only those models fulfilling the first of these two conditions.
If each consistency preservation rule selects the models that fulfill the second condition, the resulting models are locally consistent to its consistency relation, but they will never become globally consistent to all three relations.
More precisely, if the consistency preservation rules for $\consistencyrelation{CR}{AC}$ select $\model{c}{i}$ for $\model{a}{i}$ and vice versa, and if the rules for $\consistencyrelation{CR}{BC}$ select $\model{c}{i+2}$ for $\model{a}{i}$ and vice versa, no orchestration of the transformations will yield consistent models, because they never select those models that are in the overlap of the consistency relations.

\begin{figure}
    \centering
    \newcommand{\distance}{24em}
\newcommand{\classwidth}{4em}

\begin{tikzpicture}

\umlclassvarwidth{person}{}{Person}{
firstname\\
lastname
}{\classwidth}
\umlclassvarwidth[, right=\distance of person.north, anchor=north]{resident}{}{Resident}{
name
}{\classwidth}
\umlclassvarwidth[, below right=0.4*\distance and 0.5*\distance of person.north, anchor=north]{employee}{}{Employee}{
name
}{\classwidth}

\draw[consistency relation] 
    ([yshift=-2.0em]person.north east)
    --
    node[pos=0, above right] {$p$}
    node[pos=1, above left] {$r$}
    node[below, align=center] {$\{\tupled{p,r} \mid \mathvariable{r.name} = \mathvariable{p.firstname} + \textnormal{"\textvisiblespace"} + \mathvariable{p.lastname} $\\
    $\lor \mathvariable{r.name} = \mathvariable{p.lastname} + \textnormal{",\textvisiblespace"} + \mathvariable{p.firstname}\}$}
    ([yshift=-2.0em]resident.north west);
\draw[consistency relation] 
    ([xshift=0.7em]person.south)
    |-
    node[pos=0, below left] {$p$}
    node[pos=1, below left] {$e$}
    node[right, pos=0.35, align=center] {$\{\tupled{p,e} \mid\mathvariable{e.name} = \mathvariable{p.firstname} + \textnormal{"\textvisiblespace"} + \mathvariable{p.lastname} $\\
    $\lor \mathvariable{e.name} = \mathvariable{p.lastname} + \textnormal{"\textvisiblespace"} + \mathvariable{p.firstname} \}$}
    ([yshift=0.7em]employee.west);
\draw[consistency relation] 
    ([xshift=-0.7em]resident.south)
    |-
    node[pos=0, below right] {$r$}
    node[pos=1, below right] {$e$}
    node[left, pos=0.22, align=center] {$\{\tupled{e,r} \mid \mathvariable{r.name} = \mathvariable{e.name}\}$}
    ([yshift=0.7em]employee.east);

\draw[transformation, -latex] 
    ([yshift=-0.8em]person.north east)
    --
    node[above=0.5em, align=center] {
        $+p \rightarrow r(\mathvariable{name} = \mathvariable{p.lastname} + \textnormal{",\textvisiblespace}" + \mathvariable{p.firstname})$\\[0.2em]
        $\begin{aligned}
            \mathtextspaceafter{\textup{Alternative 1:}} \;
            +r \rightarrow p(
            &
                \mathvariable{firstname} = \mathvariable{r.name.substringAfter}(\textnormal{"\textvisiblespace"})\\
            &
                \mathvariable{lastname} = \mathvariable{r.name.substringBeforeFirst}(\textnormal{","}, \textnormal{"\textvisiblespace"})
            )\\
            \mathtextspaceafter{\textup{Alternative 2:}}
            +r \rightarrow \mathvariable{if}(
            &
                \mathvariable{r.name.contains}(\textnormal{","})) \mathtextspacearound{then Alternative 1, else} \\
            p(
            &
                \mathvariable{firstname} = \mathvariable{r.name.substringBefore}(\textnormal{"\textvisiblespace"}) \\
            &
                \mathvariable{lastname} = \mathvariable{r.name.substringAfter}(\textnormal{"\textvisiblespace"})
            )
        \end{aligned}$
    }
    ([yshift=-0.8em]resident.north west);
\draw[transformation, -latex] 
    ([xshift=-0.7em]person.south)
    |-
    node[below left=0.5em and 2.3em, pos=0.5, anchor=north west, align=center] {
        $\begin{aligned}
            +p \rightarrow +e(
            &
                \mathvariable{name} = \mathvariable{p.lastname} + \textnormal{"\textvisiblespace"} + \mathvariable{p.firstname}
            ) \\
            +e \rightarrow +p(
            &
                \mathvariable{firstname} = \mathvariable{e.name.substringAfter}(\textnormal{"\textvisiblespace"})\\
            &
                \mathvariable{lastname} = \mathvariable{e.name.substringBefore}(\textnormal{"\textvisiblespace"})
            )
        \end{aligned}$
    }
    ([yshift=-0.7em]employee.west);
\draw[transformation, latex-]
    ([xshift=0.7em]resident.south)
    |-
    node[below right=0.5em and 2.3em, pos=0.5, anchor=north east, align=center] {
        $+e \rightarrow +r(\mathvariable{name} = \mathvariable{e.name})$\\
        $+r \rightarrow +e(\mathvariable{name} = \mathvariable{r.name})$
    }
    ([yshift=-0.7em]employee.east);


\node[consistency preservation element, below=5em of employee.south, anchor=north] {
    $\begin{aligned}
    \mathtextspaceafter{\textup{with:}} \;
    & 
    \mathvariable{x.substringBefore}(\mathvariable{separator}) \equalsperdefinition \mathvariable{x.substring}(0, \mathvariable{x.indexOf}(\mathvariable{separator}))\\
    & 
    \mathvariable{x.substringBeforeFirst}(\mathvariable{separator1}, \mathvariable{separator2}) \equalsperdefinition\\
    &
    \formulaskip \mathvariable{x.substring}(\mathvariable{min}(\mathvariable{x.indexOf}(\mathvariable{separator1}), \mathvariable{x.indexOf}(\mathvariable{separator2})))\\
    & 
    \mathvariable{x.substringAfter}(\mathvariable{separator}) \equalsperdefinition \mathvariable{x.substring}(\mathvariable{x.indexOf}(\mathvariable{separator}) + 1)\\
    \end{aligned}$
};

\end{tikzpicture}
    %\includegraphics[width=0.85\textwidth]{figures/correctness/orchestration/no_orchestration.png}
    \caption[Consistency preservation rules without orchestration]{Consistency relations with options for corresponding elements leading to consistency preservation rules for which no consistent orchestration exists.}
    \label{fig:orchestration:no_orchestration}
\end{figure}

\mnote{Running example scenario}
\autoref{fig:orchestration:no_orchestration} demonstrates this situation at a derivation of the running example.
The consistency relation between employees and residents ensures that for each resident and employee there is a corresponding other element with the same $\mathvariable{name}$.
The consistency relations between employees and persons, as well as between residents and persons ensure that for each person there is a corresponding employee and resident, respectively, but they allow different relations of their names.
While both consider elements corresponding if the $\mathvariable{name}$ of an employee and resident, respectively, are the concatenation of the $\mathvariable{firstname}$ and $\mathvariable{lastname}$ of a person, an employee is also allowed to have the inverse concatenation of $\mathvariable{lastname}$ and $\mathvariable{firstname}$, whereas a resident is also allowed to have this inverse concatenation but with an additional separation of the $\mathvariable{lastname}$ and $\mathvariable{firstname}$ with a comma.
These options for the consistency relations provide further degrees of freedom for each transformation on its own, as they allow, for example, employee names to be encoded differently.
This can be reasonable if the order of $\mathvariable{firstname}$ and $\mathvariable{lastname}$ is not relevant in a model managing employees.
In combination with the other consistency relations, however, the only employees, residents, and persons that are considered consistent to all of the consistency relations are those having the same names with the concatenation of $\mathvariable{firstname}$ and $\mathvariable{lastname}$.
Nevertheless, these consistency relations are compatible, because for each possible condition element, i.e., for every possible employee, person, and resident, consistent models exist that contain them.

\mnote{Relations without consistent orchestration}
Consistency preservation rules for these consistency relations need to choose one of the given options for the names of corresponding employees, residents, and persons.
\autoref{fig:orchestration:no_orchestration} sketches consistency preservation rules that make such a selection.
The rules with alternative 1 ensure that for each employee, resident, and person corresponding elements exist, which fulfill those relations of the names that are conflicting.
This means, the employee's $\mathvariable{name}$ is the concatenation of the $\mathvariable{lastname}$ and $\mathvariable{firstname}$ of a person, whereas the resident's $\mathvariable{name}$ contains an additional comma in that concatenation.
In the other direction, the names of employees and residents are split at the appropriate indices given by the whitespace and comma, respectively, to calculate the required $\mathvariable{firstname}$ and $\mathvariable{lastname}$ of a person.
In consequence, there is no execution sequence of the transformations that results in consistent models, because the execution of the transformation between employees and persons always leads to a violation of the consistency relation between residents and persons and vice versa.
This is because the transformation between persons and residents always introduces a comma in the resident's $\mathvariable{name}$, which is then appended to the $\mathvariable{lastname}$ by the transformation between employees and persons.
A repeated execution of the transformation repeatedly appends that comma.
On the other hand, the execution of any of the transformations does never lead to the introduction of a person that fulfills the non-conflicting conditions of both consistency relations by simply containing a $\mathvariable{firstname}$ and $\mathvariable{lastname}$ that is represented as a concatenation of $\mathvariable{firstname}$ and $\mathvariable{lastname}$ in both an employee and a resident.
This is a concrete example for the abstract situation that of different options in consistency relations always the non-overlapping ones are chosen by the consistency preservation rules.

\mnote{Relations with consistent orchestration}
If we consider alternative 2 for the consistency preservation rule between persons and residents, we can always find a consistent orchestration.
The alternative rule decides how consistency is ensured based on the existence of a comma within the resident's $\mathvariable{name}$.
If a comma is present, the $\mathvariable{name}$ relation containing a comma is used, and otherwise the simple concatenation of $\mathvariable{firstname}$ and $\mathvariable{lastname}$ is assumed.
After adding an employee, first executing the transformation from employees to residents and afterwards the one from residents to persons ensures that all consistency relations are fulfilled, because the one between residents and persons sets the $\mathvariable{firstname}$ and $\mathvariable{lastname}$ of a person according to the relation that is also fulfilled between the person and the employee, because the $\mathvariable{name}$ does not contain a comma.
After adding a person, first executing the transformation from persons to employees and then the one from employees to residents results in an employee and a resident with inverse $\mathvariable{firstname}$ and $\mathvariable{lastname}$.
Since this resident is not consistent to the person, the transformation from residents to persons adds another person, which then also contains the swapped $\mathvariable{firstname}$ and $\mathvariable{lastname}$. Executing the same process again results in two persons, residents, and employees with both assignments of $\mathvariable{firstname}$ and $\mathvariable{lastname}$, which may not be intended but actually represents a consistent result.
Finally, after adding a resident we can, for example, first apply the transformation between residents and employees and then the one between residents and persons, resulting in consistent models due to the same reasons as above.

\mnote{Only specific orchestrations are consistent}
Although consistent orchestrations of the transformations with the consistency preservation rule defined as alternative 2 exist, not every execution order leads to consistent models.
In the scenarios discussed above, we have ensured that the transformation between residents and persons is executed after the addition of a resident.
If this transformation is executed after the addition of a person, a comma is added, which leads to the subsequent application of the same consistency preservation rules as with alternative 1 and implies that no further orchestration yields consistent models.

\mnote{Necessity to find consistent orchestrations}
No matter whether exactly these consistency relations and preservation rules for them may occur in an actual transformation network, they exemplify the general situation of having consistency preservation rules that select one of different options provided by the consistency relations to introduce corresponding elements to restore consistency.
The example shows that whether or not a consistent orchestration of transformations exists in such a situation depends on whether at least one transformation selects an option that is consistent to other consistency relations as well.
It also shows that even if a consistent orchestration exists, not all orchestrations yield consistent models.
Thus, we need to be able to find one that does.

\mnote{Resolvability}
In accordance with existing work~\cite{stevens2020BidirectionalTransformationLarge-SoSym}, we call a given tuple of models and changes \emph{resolvable} by a transformation network if a consistent orchestration exists.
%
%\mnote{Necessity to deal with unresolvability}
We have to accept that transformation networks may be unresolvable, i.e., that there is no consistent orchestration of the transformations.
Ensuring that a network is resolvable for every change would lead to restrictions for the individual transformations that would especially require different transformations to be aligned with each other.
Since that conflicts our assumption of independent development and modular reuse, we accept unresolvability and instead focus on how we can find an orchestration if it exists. 

\mnote{Optimality property of orchestration function}
In conclusion, we expect the application function to deliver consistent models whenever a consistent orchestration, i.e., an execution order that yields consistent models, exists.
Thus, we want to ensure that the orchestration function is able to always find such an orchestration if it exists.
We define this as an \emph{optimality} property in the following.


\subsection{Optimal Orchestration}

\mnote{Optimal orchestration function}
To ensure that an application function delivers consistent models whenever a consistent orchestration exists, we need to find an orchestration function that fulfills this property.
We denote this as an \emph{optimal} orchestration function.
Recall that $\generalizationfunction{\metamodeltuple{M},\transformation{t}}$ is the generalization function that applies a transformation to a model tuple that instantiates all metamodels in a tuple $\metamodeltuple{M}$.

\begin{definition}[Optimal Orchestration Function]
    Let $\transformationset{T}$ be a set of transformations for consistency relations $\consistencyrelationset{CR}$ and metamodels $\metamodeltuple{M}$.
    We say that an orchestration function $\orcfunction{\transformationset{T}}$ for these transformations is \emph{optimal} if, and only if, it returns a consistent orchestration whenever it exists:
    \begin{align*}
        &
        \forall \modeltuple{m} \in \metamodeltupleinstanceset{M} \mid \modeltuple{m} \consistenttomath \consistencyrelationset{CR} : 
        \forall \changetuple{\metamodeltuple{M}} \in \changeuniverse{\metamodeltuple{M}} : \\
        & \formulaskip
        \big[ 
            \exists \transformation{t}_{1}, \dots, \transformation{t}_{i} \in \transformationset{T} : 
            \exists \changetuple{\metamodeltuple{M}}' \in \changeuniverse{\metamodeltuple{M}} : 
            \big(
                \changetuple{\metamodeltuple{M}}'(\modeltuple{m}) \consistenttomath \consistencyrelationset{CR}\\
                & \formulaskip \formulaskip
                \land \generalizationfunction{\metamodeltuple{M}, \transformation{t}_{i}} \concatfunction \dots \concatfunction \generalizationfunction{\metamodeltuple{M}, \transformation{t}_{1}}(\modeltuple{m}, \changetuple{\metamodeltuple{M}}) = (\modeltuple{m}, \changetuple{\metamodeltuple{M}}')
            \big)
            \\
            & \formulaskip
            \Rightarrow
            \exists \transformation{t}_{1}', \dots, \transformation{t}_{k}' \in \transformationset{T} : 
            \exists \changetuple{\metamodeltuple{M}}'' \in \changeuniverse{\metamodeltuple{M}} : 
            \big(
                \changetuple{\metamodeltuple{M}}''(\modeltuple{m}) \consistenttomath \consistencyrelationset{CR}\\
                & \formulaskip \formulaskip
                \land \generalizationfunction{\metamodeltuple{M},\transformation{t}_{k}'} \concatfunction \dots \concatfunction \generalizationfunction{\metamodeltuple{M},\transformation{t}_{1}'}(\modeltuple{m}, \changetuple{\metamodeltuple{M}}) = (\modeltuple{m}, \changetuple{\metamodeltuple{M}}'') \\
                & \formulaskip \formulaskip
                \land \orcfunction{\transformationset{T}}(\modeltuple{m}, \changetuple{\metamodeltuple{M}}) = \sequenced{\transformation{t}_{1}', \dots, \transformation{t}_{k}'} 
            \big)
        \big]
    \end{align*}
\end{definition}

\mnote{Behavior without consistent orchestration existence}
Note that we allow an optimal orchestration function to return a sequence even when there is no consistent orchestration.
This is reasonable, because an application function may also support finding the reasons when no consistent orchestration is found by delivering a sequence of transformations that leads to a failure, as we discuss in \autoref{chap:orchestration:algorithm}.

\mnote{Application function optimality}
Finally, the result of the application function is what is relevant in the process of consistency preservation.
Thus, we apply the notion of \emph{optimality} to that function accordingly by requiring it to deliver consistent models whenever a consistent orchestration exists.

\begin{definition}[Optimal Application Function]
    \label{def:optimalapplicationfunction}
    Let $\transformationset{T}$ be a set of transformations for consistency relations $\consistencyrelationset{CR}$ and metamodels $\metamodeltuple{M}$.
    We say that an application function $\appfunction{\orcfunction{\transformationset{T}}}$ for these transformations is \emph{optimal} if, and only if, it returns models that are consistent whenever there is a consistent orchestration of the transformations:
    \begin{align*}
        &
        \forall \modeltuple{m} \in \metamodeltupleinstanceset{M} \mid \modeltuple{m} \consistenttomath \consistencyrelationset{CR} : 
        \forall \changetuple{\metamodeltuple{M}} \in \changeuniverse{\metamodeltuple{M}} : \\
        & \formulaskip
        \big[
            \exists \transformation{t}_{1}, \dots, \transformation{t}_{i} \in \transformationset{T} : 
            \exists \changetuple{\metamodeltuple{M}}' \in \changeuniverse{\metamodeltuple{M}} : 
            \big(
                \changetuple{\metamodeltuple{M}}'(\modeltuple{m}) \consistenttomath \consistencyrelationset{CR}\\
                & \formulaskip \formulaskip
                \land \generalizationfunction{\metamodeltuple{M}, \transformation{t}_{i}} \concatfunction \dots \concatfunction \generalizationfunction{\metamodeltuple{M}, \transformation{t}_{1}}(\modeltuple{m}, \changetuple{\metamodeltuple{M}}) = (\modeltuple{m}, \changetuple{\metamodeltuple{M}}')
            \big)
            \\
            & \formulaskip
            \Rightarrow \appfunction{\orcfunction{\transformationset{T}}}(\modeltuple{m},\changetuple{\metamodeltuple{M}}) \consistenttomath \consistencyrelationset{CR}
        \big]
    \end{align*}
\end{definition}

\mnote{Optimality dependency}
According to the defined behavior of an application function, an optimal application function requires an optimal orchestration function.

\begin{lemma}[Application / Orchestration Function Optimality]
    \label{lemma:optimalapplicationfunction}
    An application function $\appfunction{\orcfunction{\transformationset{T}}}$ can only be optimal if $\orcfunction{\transformationset{T}}$ is optimal.
\end{lemma}
\begin{proof}
    Let us assume that the condition in \autoref{def:optimalapplicationfunction} is fulfilled, i.e., that the input models are consistent and that a consistent orchestration of the transformations exists for them.
    Then, to be optimal, the application function needs to return models that are consistent.
    According to the definition of an application function (see \autoref{def:applicationfunction}), the sequence of transformations delivered by $\orcfunction{\transformationset{T}}$ for that input must yield the same model tuple as $\appfunction{\orcfunction{\transformationset{T}}}$.
    Thus, the orchestration function must deliver a sequence for such inputs that yields consistent models, which is equivalent to $\orcfunction{\transformationset{T}}$ being optimal.
\end{proof}


\subsection{The Orchestration Problem}

\mnote{Orchestration problem}
The problem to find a consistent orchestration whenever it is exists, i.e., to find an optimal orchestration function, is the central subject of the following sections.
This is what we denote as the \emph{orchestration problem}.
We prove that the problem is undecidable, discuss how we can make it decidable, and propose strategies to deal with its undecidability.
Finally, we come up with a discussion of conservatively approximating a solution to the problem.
We define the problem as follows.
\begin{definition}[Orchestration Problem]
    \label{def:orchestrationproblem}
    The problem to find a consistent orchestration of transformations for given inputs (models and changes to them) if it exists is called the \emph{orchestration problem}.
\end{definition}

\mnote{Orchestration existence problem}
Often, the more general problem of deciding whether a consistent orchestration exists is sufficient.
\begin{definition}[Orchestration Existence Problem]
    \label{def:orchestrationexistenceproblem}
    The question whether a consistent orchestration of transformations for given inputs (models and changes to them) exists is called the \emph{orchestration existence problem}.
\end{definition}

\mnote{Problem equivalence}
In fact, both these problems are equivalent in the sense that having a solution for one of them also delivers a solution for the other.
\begin{theorem}[Orchestration / Existence Problem Equivalence]
    \label{theorem:orchestrationproblemequivalence}
    The orchestration problem can be solved if, and only if, the orchestration existence problem can be solved.
\end{theorem}

\begin{proof}
    If a solution for the orchestration problem exists, it directly induces a solution for the orchestration existence problem, because if we find a consistent orchestration whenever it exists, we also know whether it exists.
    If a solution for the orchestration existence problem exists and we know that a consistent orchestration exists, we can find it by systematically testing all orchestrations of growing size until a consistent orchestration is found, since models are of finite size. Since we know that such an orchestration exists, this test must terminate, even though it may take an impractically long time.
\end{proof}

\mnote{Problem to optimality equivalence}
Since the orchestration problem is derived from the goal of finding an optimal application function, it is obviously equivalent to find an optimal application function or to solve the orchestration (existence) problem.

\begin{theorem}[Optimality / Orchestration Problem Equivalence]
    \label{theorem:optimal_application_function_orchestration_problem}
    An optimal application function $\appfunction{\orcfunction{\transformationset{T}}}$ can be defined if, and only if, a solution for the orchestration (existence) problem exists.
\end{theorem}

\begin{proof}
    We give the proof for the orchestration existence problem, which is, according to \autoref{theorem:orchestrationproblemequivalence}, equivalent to the orchestration problem.
    An optimal $\appfunction{\orcfunction{\transformationset{T}}}$ returns consistent models whenever there is a consistent orchestration.
    With such a function, we are able to decide whether such an orchestration exists or not.
    \begin{align*}
        \function{ExistsOrc}(\transformationset{T},\modeltuple{m},\changetuple{\metamodeltuple{M}}) \equalsperdefinition
            \begin{cases}
                \truemath, & \ifmath \appfunction{\orcfunction{\transformationset{T}}}(\modeltuple{m},\changetuple{\metamodeltuple{M}}) \consistenttomath \transformationset{T} \\
                \falsemath, & \otherwisemath
            \end{cases}
    \end{align*}
    $\function{ExistsOrc}$ returns \textsc{true} if, and only if, a consistent orchestration exists.
    Since $\appfunction{\orcfunction{\transformationset{T}}}$ is optimal, it returns consistent models in exactly those cases in which a consistent orchestration that yields them exists.

    If a solution for the orchestration existence problem exists, we know whether a consistent orchestration exists for an input.
    In that case, we can define $\appfunction{\orcfunction{\transformationset{T}}}$ to apply an according orchestration, which can be found by exhaustively testing different orchestration as discussed in the proof for \autoref{theorem:orchestrationproblemequivalence}, and otherwise to return $\bot$.
\end{proof}


\section{Decidability of the Orchestration Problem}

\mnote{Different approaches to achieve optimality}
We introduced the orchestration problem as the problem to find a consistent orchestration if it exists.
This is equivalent the existence of an optimal orchestration function.
We can distinguish two approaches to ensure that the orchestration function is optimal, i.e., that it does always find a consistent orchestration if it exists.
Let $P$ be the problem space, i.e., all possible transformation execution orders for given transformations and let $S_{i}$ be the solution space with those orders that yield consistent models for a specific input of models and a change to them.
\begin{properdescription}
    \item[Strategy Definition:] Define a strategy that explores the problem space $P$ to find one of the sequences in the solution space $S_{i}$, if $S_{i} \neq \emptyset$.
    \item[Transformation Restriction:] Define a \emph{well-behavedness} property for the transformations that ensures that executing the transformations in any order often enough, they yield consistent models if $S_{i} \neq \emptyset$, i.e., for any given input $i$ there is an $n \in \mathbb{N}$ such that $\forall s \in P : \abs{s} > n \Rightarrow s \in S_{i}$.
\end{properdescription}

%Unfortunately, optimality is a property that we cannot request from an orchestration function. Optimality would mean that the orchestration function can decide whether there is sequence of transformations that leads to consistent models and thus terminate.

\mnote{No restrictions to transformations}
In the latter case, the orchestration function may return any order of the transformations, as long as the sequence is long enough to be optimal.
This means, performing an iterative execution of the transformations leads to a consistent result, comparable to a fixed-point iteration.
Since optimality is a property of an orchestration function with respect to a set of transformations, defining a \emph{well-behavedness} property as a restriction for transformations to ease finding an optimal orchestration function will potentially not concern a single transformation but the set of them.
This can easily contradict our assumption of independent development and reuse or lead to restrictions of transformation that are not practical anymore.

\mnote{Section summary}
In the following, we first investigate the possibility to find an optimal orchestration function without restriction the transformations.
We define a general algorithm that realizes an application function, as in practice the function will be realized in terms of an algorithm that dynamically selects the next transformation to execute rather than being an ordinary mathematical function.
We then discuss its correctness and termination and relate it to the orchestration problem.
After proving undecidability of the orchestration problem, we discuss the possibilities to restrict transformations such that the problem get decidable.
Finally, we shortly discuss confluence as a considerable property of transformation networks.

% Thus, we first follow the former approach and investigate the possibility to find an optimal orchestration function without restricting the transformations.
% %Optimality of an orchestration function means that it can decide whether there is a sequence of transformations that leads to consistent models and thus terminates.
% In the following, we therefore define a general algorithm that realizes an application function and investigate how we can ensure that its orchestration function is optimal.


\subsection{An Algorithm for the Application Function}

%Start with defining an algorithm that realizes the application function. (different options depending on when to return $\bot$, discussed later)
%First option: we assume an oracle that returns the transformation to execute next according to the orchestration function and we stop when consistent models are achieved

\begin{algorithm}
   % \begin{algorithmic}[1]
    \Procedure{\function{FindCorresponding}}{$\consistencyrelation{CR}{}, \conditionelement{c}{l}, \model{m}{2}, \model{traces}{}$}
        \State $\mathvariable{tracedElements}$ $\leftarrow$ $\setted{\conditionelement{c}{r} \mid \tupled{\conditionelement{c}{l}, \conditionelement{c}{r}} \in \model{traces}{}}$
        \For{$\conditionelement{c}{r} \in \mathvariable{tracedElements}$} \label{algo:synchronization:findcorrespondingelements:line:explicit}
            \If{$\tupled{\conditionelement{c}{l}, \conditionelement{c}{r}} \in \consistencyrelation{CR}{}$}
                \State \Return{\textsc{true}}
            \EndIf
        \EndFor
        \For{$\conditionelement{c}{r} \in \mathcal{P}(\model{m}{2})$} \label{algo:synchronization:findcorrespondingelements:line:implicit}
            \If{$\tupled{\conditionelement{c}{l}, \conditionelement{c}{r}} \in \consistencyrelation{CR}{}$}
                \State $\model{trace}{}$ $\leftarrow$ $\model{trace}{} \cup \setted{\tupled{\conditionelement{c}{l},\conditionelement{c}{r}}}$
                \State \Return{\textsc{true}}
            \EndIf 
        \EndFor
        \State \Return{\textsc{false}}
    \EndProcedure
\end{algorithmic}
    \begin{algorithmic}[1]
        \Procedure{$\function{Apply}_{\transformationset{T}}$}{$\modeltuple{m}, \changetuple{\metamodeltuple{M}}$}
            \State $\mathvariable{isConsistent}$ $\leftarrow$ $\function{CheckConsistency}(\transformationset{t}, \modeltuple{m}$)
            \If{$\neq \mathvariable{isConsistent}$}
                \State \Return{$\bot$}
            \EndIf
            \State $\mathvariable{executedTransformations} \leftarrow \tupled{}$
            \State $\mathvariable{generatedChanges} \leftarrow \tupled{}$
            \State $\transformation{t}_{next}$ $\leftarrow$ $\function{Orchestrate}_{\transformationset{T}}(\modeltuple{m}, \changetuple{\metamodeltuple{M}}, \mathvariable{executedTransformations}, \mathvariable{generatedChanges})$ \label{algo:orchestration:application:line:startorchestrate}
            \While{$\transformation{t}_{next} \neq \bot$}
                \State $(\modeltuple{m}, \changetuple{\metamodeltuple{M}})$ $\leftarrow$ $\generalizationfunction{\metamodeltuple{M}, \transformation{t}_{next}}(\modeltuple{m}, \changetuple{\metamodeltuple{M}})$ \label{algo:orchestration:application:line:stepcalculation}
                \State $\mathvariable{executedTransformation}.add(\transformation{t}_{next})$
                \State $\mathvariable{generatedChanges}.add(\changetuple{\metamodeltuple{M}})$
                \State $\transformation{t}_{next}$ $\leftarrow$ $\function{Orchestrate}_{\transformationset{T}}(\modeltuple{m}, \changetuple{\metamodeltuple{M}}, \mathvariable{executedTransformations}, \mathvariable{generatedChanges})$
            \EndWhile \label{algo:orchestration:application:line:endorchestrate}
            \State $\tupled{\model{m}{1}, \dots, \model{m}{n}} \leftarrow \modeltuple{m}$
            \State $\tupled{\change{\metamodel{M}{1}}, \dots, \change{\metamodel{M}{n}}} \leftarrow \changetuple{\metamodeltuple{M}}$
            \State $\modeltuple{m}_{res} \leftarrow \tupled{\change{\metamodel{M}{1}}(\model{m}{1}), \dots, \change{\metamodel{M}{n}}(\model{m}{n})}$
            \State $\mathvariable{isConsistent}$ $\leftarrow$ $\function{CheckConsistency}(\transformationset{t}, \modeltuple{m}_{res}$) \label{algo:orchestration:application:line:startconsistencycheck}
            \If{$\neq \mathvariable{isConsistent}$}
                \State \Return{$\bot$}
            \EndIf \label{algo:orchestration:application:line:endconsistencycheck}
            % \For{$\transformation{t} \in \transformationset{T}$} \label{algo:orchestration:application:line:startcheckconsistency}
            %     %\State $(\consistencyrelation{CR}{}, \consistencypreservationrule{\consistencyrelation{CR}{}}) \leftarrow \transformation{t}$
            %     \State $\mathvariable{isConsistent}$ $\leftarrow$ $\function{CheckConsistency}_{\metamodeltuple{M}}(\modeltuple{m}_{res}, \transformation{t}$) %\consistencyrelation{CR}{})$
            %     \If{$\neq \mathvariable{isConsistent}$}
            %         \State \Return{$\bot$}
            %     \EndIf
            % \EndFor \label{algo:orchestration:application:line:endcheckconsistency}
            \State \Return{$\modeltuple{m}_{res}$} \label{algo:orchestration:application:line:returnresult}
        \EndProcedure
    \end{algorithmic}
    \caption[Application function for transformations]{Application function for transformation set.}
    \label{algo:orchestration:application}
\end{algorithm}

\mnote{Algorithm for the application function}
We have yet discussed the orchestration and application functions as purely mathematical functions.
In practice, however, they need to be implemented in terms of algorithms.
In \autoref{algo:orchestration:application}, we propose an algorithm for the application function.
It also encodes the orchestration function, because in contrast to the mathematical definition, an algorithm for the orchestration function will not determine a complete sequence of transformations for given models and changes, but dynamically select the next transformation to execute.
As soon as all transformation delivered by the orchestration are executed, it returns the resulting models if they are consistent or otherwise returns $\bot$.

\mnote{Algorithm dynamically selects next transformation}
The dynamic selection of transformations is realized by an \function{Orchestrate} function and stops as soon as no further transformations to apply are delivered.
The latter may be the case because the models are already consistent or because no further transformations can be applied.
The complete logic of the orchestration function is combined with the application of the delivered sequence in Lines~\ref{algo:orchestration:application:line:startorchestrate}--\ref{algo:orchestration:application:line:endorchestrate}.
Since, in practice, the selection of transformation has to be performed dynamically anyway, an implementation of the orchestration function always needs to apply the transformations.
Thus a separation of the orchestration function into a separate algorithm, which performs the same steps as in Lines~\ref{algo:orchestration:application:line:startorchestrate}--\ref{algo:orchestration:application:line:endorchestrate} leads to a redundancy by applying the transformations both in the separate orchestration algorithm as well as in the given algorithm.

\mnote{Orchestration needs history of changes and transformations}
The function receives the history of executed transformations and generated changes, because if the complete orchestration function was implemented in a separate method, it would also be able to use that information to determine a proper orchestration.
Otherwise, its expressiveness would be restricted with respect to the definition of an orchestration function, because that function makes a global decision for all transformations to execute base on the original input, which is not available for the \function{Orchestrate} function after its first execution anymore.
In a practical implementation of that function, the history may, however, not be considered or truncated, depending on the information necessary for the concrete implemented orchestration strategy.

\mnote{Strategies for selecting next transformation}
%The \function{Orchestrate} function is responsible for selecting the next transformation.
The \function{Orchestrate} function may implement different strategies, which we will later discuss in more detail.
The most simple strategy would be to execute the same order of transformations iteratively, thus always executing that transformation who was not executed for the longest time.
Another reasonable strategy would be to manage a queue of transformation and after executing one transformation to enqueue all transformations that are adjacent to the metamodels of the two models that were modified by the transformation if they are not yet enqueued.
This ensures that those transformation are executed next which can process changes that have just been produced by another transformation.
We will later discuss specific strategies.
Until then, the concrete strategy is not important and any of the exemplified ones can be imagined.

\mnote{Assumed further functions of algorithm}
Next to \function{Orchestrate}, the algorithm uses the external functions $\generalizationfunction{}$ and \function{CheckConsistency}.
The $\generalizationfunction{}$ function is the generalization function, which simply applies the given transformation to the appropriate models of the given tuple.
The \function{CheckConsistency} checks whether the given models are consistent to the set of transformations, according to \autoref{def:consistencytransformation}.
Due to their simplicity, we do not provide an explicit implementation of these two functions.

% Algorithm dynamically selects next transformation.
% It stops as soon as orchestration does provide further transformations.
% Then, either no transformation can be applied anymore, or the models are consistent.
% This moves some of the logic of the orchestration function into the apply function, as orchestrate only selects the next function rather than delivering a sequence in the beginning.
% Thus, Lines~\ref{algo:orchestration:application:line:startorchestrate}--\ref{algo:orchestration:application:line:endorchestrate} implement the orchestration function.
% This is reasonable for two reasons:
% First, this is how a practical algorithm will perform the selection anyway, by dynamically selecting the next transformation.
% Second, this is only a matter of implementation, as we could move the lines to a separate method, which acts as the orchestrate function by determining the sequence by dynamically applying the transformations, let it return the sequence and then let the apply function apply the transformations in that order again.
% %Second, although dynamic selection may sound more expressive than a a priori determination of the complete sequence like defined for the app and orc function in the previous definitions, this is not the case as an orc function according to that definition may be an oracle that, in a practical implementation, determines the sequence for a given input in the same way.
% Since applying the transformation in the orc function to determine a sequence and then simply applying them in the app function again is redundant, we directly implemented that dynamic selection in the app function.

% Apply is defined to be comprehensible, not efficient. For example, the consistency check can be improved by aborting as soon as one transformation to which the models are inconsistent is found.

% Gen is the generalization function, CheckConsistency is generalized consistency checker which just checks for the appropriate models in the tuple.

\todo{transformation set should be parameter, not one application/orchestration function per transformation set}


\subsection{Correctness and Termination of the Algorithm}

\mnote{Algorithms implements correct application function}
\autoref{algo:orchestration:application} is constructed to implement an application function according to \autoref{def:applicationfunction}.
It is designed to be correct, i.e., it only returns models when they are consistent.
We show that the algorithm fulfills these properties in the following theorem.

\begin{theorem}[Apply Function Correctness]
    The \function{Apply} function in \autoref{algo:orchestration:application} fulfills the functional behavior of an application function as defined in \autoref{def:applicationfunction} and is correct according to \autoref{def:applicationfunctioncorrectness}
\end{theorem}
\begin{proof}
    The \function{Apply} function fulfills the input and output requirements of an application function according to \autoref{def:applicationfunction}.
    It only returns a model tuple in Line \ref{algo:orchestration:application:line:returnresult}, which is achieved by applying the changes delivered by the sequence of transformations delivered by the orchestration function realized as a repeated call of the \function{Orchestrate} function in Lines~\ref{algo:orchestration:application:line:startorchestrate}--\ref{algo:orchestration:application:line:endorchestrate}.
    Thus, \function{Apply} fulfills the definition of an application function.

    Correctness of an application function according to \autoref{def:applicationfunctioncorrectness} requires the output models, if not returning $\bot$, to be consistent to the consistency relations of all transformations, as long as the input models were consistent.
    The algorithm only returns models in \autoref{algo:orchestration:application:line:returnresult}.
    These models are always consistent to the consistency relations of all transformations, because Lines~\ref{algo:orchestration:application:line:startconsistencycheck}--\ref{algo:orchestration:application:line:endconsistencycheck} ensure this and otherwise return $\bot$ before.
\end{proof}

\mnote{Termination is not guaranteed}
In addition to being correct, the algorithm needs to terminate always.
Non-termination can only occur because of the loop for orchestration transformations, as there are no recursions and the other loop is finite because the set of transformations is of finite size.
According to the definition, an orchestration function is defined to return a finite sequence of transformations, which would also result in a finite number of execution of the loop for orchestrating transformations.
The implementation by a dynamic selection of the next transformation to execute can, however, lead to an infinite sequence of transformations.
The \function{Orchestrate} function receives the list of previously executed transformations, as otherwise it would never be able to identify that, for example, always the same sequence of transformations is executed and leads to the same changes, thus the algorithms is only performing an infinite alternation.
We do, however, need to ensure that the \function{Orchestrate} function returns $\bot$ after a finite number of calls.

\mnote{Options to guarantee termination}
If we assume that we can achieve optimality for the orchestration function, we would have the guarantee that if a consistent orchestration exists, the function will find it.
There is, however, no restriction to what the orchestration function may return when there is no orchestration that yields consistent models at all.
Thus, we have two options to ensure termination:
\begin{enumerate}
    \item We enable the orchestration function to identify whether a consistent orchestration exists.
    \item We find an upper bound for the number of necessary transformation executions, such that if more transformations were executed, we cannot expect the algorithm to find consistent models anymore and thus abort. 
\end{enumerate}

\mnote{Termination and optimality are conflicting}
The simplest solution would be to find an upper bound for the number of necessary transformation executions.
We will, however, prove in the following that there is no such upper bound.
Afterwards, we will show that identifying whether a consistent orchestration exists is not possible either.
This will lead to the insight that we cannot guarantee termination of the algorithm with an optimal orchestration function.


%\subsection{Upper Execution Bound}

\mnote{No general upper bound for necessary number of executions}
With the example in \autoref{fig:orchestration:necessity_multiple_executions}, in which values are incremented by one upon each execution of one specific transformation until a fixed but arbitrary value $x$ is reached, we were able to show in \autoref{lemma:minimal_executions} that there can be transformation networks in which a transformation needs to be executed at least $x-1$ times for a fixed but arbitrary $x$ until consistent models are models.
Thus, any consistent orchestration contains that transformation at least $x-1$ times.
While we have used that to show that executing each transformation only once is, in general, not sufficient, we can also easily show the more general statement that we cannot find a maximal length for the orchestration of transformation networks of specific size.

\begin{theorem}[Orchestration with Fixed Number of Executions]
    \label{theorem:orchestration_fixed}
    For every $n$, there is a set of transformations $\transformationset{T}$ such that for specific models $\modeltuple{m}$ and changes $\changetuple{}$ to them for which each possible orchestration function $\orcfunction{\transformationset{T}}$ with whom $\appfunction{\orcfunction{\transformationset{T}}}(\modeltuple{m}, \changetuple{})$ is consistent, delivers a sequence as $\orcfunction{\transformationset{T}}(\modeltuple{m}, \changetuple{})$ with $\abs{\orcfunction{\transformationset{T}}(\modeltuple{m}, \changetuple{})} > n$.
\end{theorem}
\begin{proof}
    We know from \autoref{lemma:minimal_executions} that $\transformationset{T}_{inc}$ requires at least $x-1$ executions of $\transformation{T}_{12}$ for the inputs defined in \autoref{lemma:minimal_executions} and the fixed but arbitrary value $x$.
    Thus, with $x \geq n+2$ for $\transformationset{T}_{inc}$, we know that at least $x-1 = n+1$ executions of $\transformation{T}_{12}$ are necessary.
    Let $\modeltuple{m}$ and $\changetuple{}$ be the inputs defined in \autoref{lemma:minimal_executions}.
    Then for any orchestration function $\orcfunction{\transformationset{T}}$ that delivers a consistent orchestration for these inputs, we know that $\abs{\orcfunction{\transformationset{T}}(\modeltuple{m}, \changetuple{})} >= x-1 = n+1 > n$.
    This proves the theorem by example.
\end{proof}

\mnote{Upper bound cannot be used to decide whether consistent orchestration exists}
In consequence, it is not possible to find a fixed value or a value only depending on the transformation network size that defines an upper bound for the necessary number of transformation executions to yield consistent models.
Thus, even if we are able to ensure optimality of orchestration with the \function{Apply} and \function{Orchestrate} functions, there is no upper bound for the number of transformation execution that is necessary for a consistent orchestration.
We cannot abort the execution after a fixed number of loop iterations without the possibility that consistent models would have been found if the execution had proceeded and thus not ensuring optimality.

%To avoid non-termination of the algorithm, we thus need to be able to ensure that the \function{Orchestrate} function at some point returns $\bot$.
%This means that it must be able to identify when no orchestration that yields consistent models exists at all, and if it exists it must be able to find it.

%Requirement: Know whether orchestration exists, otherwise impossible to find sequence always if it exists, as we do not know whether it exists.

% This is due to the reason that the \function{Orchestrate} function does only receive the current models and changes but not the history of executed transformations.
% Thus, it cannot identify which and how many transformation have already been executed and whether the current changes have already been produced before such that 

% There are, however, still derivations of the stepwise orchestrate to the original orchestration function:
% The stepwise application of the orchestrate function cannot derive a complete sequence but only make a local decision based on the given models and changes. Thus, if the transformations produce the same changes repeatedly, i.e., there is an alternation in the produced changes, the orchestrate function cannot detect that and always determinates the same sequences of transformations to be executed next. This results in an infinite sequence of execute transformations, which is conforming to the orchestration function definition, but still is a result of the possibilities of an actual and reasonable implementation of that function.
%Additionally, the function may be able to determine beforehand whether an orchestration that yields consistent models exists and otherwise return $\bot$ in its first application.
%A practical implementation, however, will not act that way but instead determine the next transformation until no transformation can be applied anymore.
%In consequence, it is possible that several transformations are executed before detecting that no orchestration can be found.
%Since the orchestration is encoded in the apply function anyway and the function then still returns $\bot$, this is not a problem.
%This affects termination.

%Algorithm is correct if it returns only consistent models. Additionally, it must always terminate.
%ADD DEFINITION!

%Correctness is given by construction and easy to achieve.
%Termination if no transformation can be applied or consistent models are found.

%Thus: We need to find an order of transformation such that the result is consistent.

%Give example, where no execution order exists that terminates. Thus, the algorithm does not terminate.

%Approach: We want to decide whether an order exists or not.


\subsection{Undecidability of the Orchestration Problem} %Undecidability of Orchestration}

\mnote{Impossible to decide whether consistent orchestration exists}
To ensure termination of the \function{Apply} algorithm with an optimal orchestration function, we need to identify the case that no consistent orchestration exists, because that is the only situation in which otherwise an infinite number of transformation execution is possible.
Unfortunately, we will show that the problem to decide whether such an orchestration exists or not is undecidable.

%--> Undecidability

\begin{lemma}
    \label{lemma:networkfromturingmachine}
    For any Turing machine $\TM$, we can construct metamodels $\metamodeltuple{M}$ and a transformation network $\transformationnetwork{N} = \tupled{\transformationset{T}, \appfunction{\orcfunction{}}}$ with an optimal application function $\appfunction{\orcfunction{}}$, such that $\TM$ halts for input $x$ if, and only if, for models $\modeltuple{m}_{x}$ and changes $\changetuple{\metamodeltuple{M},x}$ it is $\appfunction{\orcfunction{}}(\modeltuple{m}_{x}, \changetuple{\metamodeltuple{M},x}) \consistenttomath \transformationset{T}$.
\end{lemma}
\begin{proof}
    \todo{Add proof with turing machine construction}
\end{proof}

\begin{theorem}
    \label{theorem:nooptimalapplication}
    Let $\appfunction{\orcfunction{}}$ be an application function. $\appfunction{\orcfunction{}}$ cannot be optimal.
\end{theorem}
\begin{proof}
    An optimal $\appfunction{\orcfunction{}}$ returns consistent models whenever there is a consistent orchestration.
    With such a function, we are able to decide whether such an orchestration exists or not.
    \begin{align*}
        \function{ExistsOrc}(\transformationset{T},\modeltuple{m},\changetuple{\metamodeltuple{M}}) =
            \begin{cases}
                \textsc{true}, & \appfunction{\orcfunction{}}(\transformationset{T}, \modeltuple{m},\changetuple{\metamodeltuple{M}}) \consistenttomath \transformationset{T} \\
                \textsc{false}, & otherwise
            \end{cases}
    \end{align*}
    $\function{ExistsOrc}$ returns \textsc{true} if, any only if, a consistent orchestration exists.
    $\appfunction{\orcfunction{}}$ does, per definition, only return consistent models when there is an orchestration that yields them.
    Additionally, it does always return consistent models when an orchestration that yields them exists, because it is optimal.
    It follows from \autoref{lemma:networkfromturingmachine} that we can simulate a universal Turing machine with a transformation network in the sense that we can construct a transformation network whose application function delivers consistent models for a given input if, and only if, the Turing machine halts for corresponding inputs.
    Calculating \function{ExistsOrc} would thus decide whether the Turing machine halts and thus decide the halting problem.
\end{proof}

\mnote{Algorithm cannot terminate and be optimal}
From this theorem, it directly follows that we cannot implement \function{Orchestrate} within the \function{Apply} algorithm in a way that it realizes an optimal application function and terminates always.

\begin{corollary}
    \function{Apply} according to \autoref{algo:orchestration:application} cannot terminate and return consistent models if an orchestration exists that yields them for every possible input.
\end{corollary}
\begin{proof}
    If \function{Apply} always terminated and returned consistent models whenever there is an orchestration that yields them, it would implement an optimal application function and could be used to realize \function{ExistsOrc} from the proof of \autoref{theorem:nooptimalapplication}.
    But according to \autoref{theorem:nooptimalapplication} an application function cannot be optimal.
\end{proof}

\mnote{Restrict expressiveness of transformations or accept conservativeness}
In consequence, we only have the two options to either restrict the expressiveness of the transformations such that they cannot be used to simulate a Turing machine anymore or to accept the situation that \function{Apply} may either not terminate in some cases or return $\bot$ although there is an orchestration that yields consistent models.
We call this behavior \emph{conservative}, because the algorithm does never return consistent models although there is no orchestration that yields them, but may not return consistent models in some cases in which actually such an orchestration existed.

\mnote{Discuss both options in the following}
%In the following, we first introduce alternation as a special case of non-termination that can be avoided by construction.
In the following, we discuss options to restrict transformations to make the orchestration problem solvable and finally conclude that this is not an option for solving the above discussed problem.
Afterwards, we discuss how we can realize \function{Apply} in a way that it always terminates and produces reasonable outputs.

\begin{corollary}
    The orchestration problem (see \autoref{def:orchestrationproblem}) is undecidable.
\end{corollary}
\begin{proof}
    \todo{Add proof. Maybe this should be the theorem from the Turing machine. Then we follow insights to the algorithm from that.}
\end{proof}
%Lemma: It is undecidable whether an orchestration exists for a given input.

%Theorem: We cannot guarantee termination of Algorithm, when we allow an arbitrary number of transformation execution. (combine with upper bound, see above/below)

% Consequence: Either we need to restrict transformations to make the problem decidable or we need to restrict the application function and allow it to return $\bot$ although a sequence returning consistency exists, thus, we need to deal with conservativeness.
% \textit{Weaker version:}
% Goal: Find a solution in as many cases as possible, abort in the others (conservatively). There are two approaches to achieve that: 
% 1. Reduce the number of cases in which there is no solution by adding assumptions to the relations and transformations (restrict input of app function)
% 2. Improve the ability to find a solution if it exists (improve capabilities of app function)
% Secondary goal: In cases, in which no solution is found, support the user in understanding why no solution was found.

%Since we find that we cannot guarantee to find an orchestration, we need to deal with the conservative case anyway. Thus, we can also deal with the conservative case for the second problem instead of requiring things from the transformation which may not be (easily) fulfillable.

% Conclude that we cannot guarantee termination for an optimal orchestration function


\subsection{Restriction of Transformation Networks}

\mnote{Restrictions either transformation-local or for complete network}
We have discussed that it is necessary to restrict the transformations as an input of the application function to avoid that it is undecidable whether a consistent orchestration of them exists for given models and changes.
Those restrictions can be at two levels:
\begin{properdescription}
    \item[Transformation:] Restrictions only concern the single transformation. Thus, if each transformation fulfills a specific property, the application function is able to decide whether a consistent orchestration exists.
    \item[Network:] Restrictions concern the complete network, i.e., the combination of transformations. Only a set of transformations can have an appropriate property that enables the application function to decide the orchestration problem, but not each transformation on its own.
\end{properdescription}

\mnote{Find that all restrictions are impractical}
Since we assume transformations to be developed and reused independently, restrictions to single transformations are of special interest.
It is, however, easy to see that it will unlikely be possible to define practical restrictions to single transformations that make the orchestration problem decidable.
We will see that even impractical restrictions do not make the problem decidable.
%Additionally, we discuss possible restrictions to complete networks.
%We discuss properties that obviously provide the ability to avoid divergence and alternation in the states of a network, namely monotony and confluence.
%Unfortunately, we will show that those restrictions are impractical.

% \todo{Maybe conflucence and monotony are rather restrictions of the application function (i.e. conservative operations) rather then restrictions of the transformation?}

% \begin{itemize}
%     \item Restrictions must, in the best case, be local to a transformation, i.e., they should be fulfillable without knowing with which other transformations the transformation shall be combined
%     \item Can this ever be the case?
% \end{itemize}

% \begin{itemize}
%     \item Discuss different restrictions we may apply to synchronizing transformations and networks
%     \item We conclude that none of them is practically
%     \item This does not mean that there is no restriction, but we were not able to find one
%     \item Maybe also discuss history-ignorance
% \end{itemize}

% Lösungsoptionen (Grad der Einschränkung an die Transformationen) --  überdeckt sich mit der Klassifizierung hierüber -> zusammenführen
% \begin{itemize}
%     \item Hohe Einschränkung: Jede beliebige Reihenfolge von ausgeführten Transformationen führt letztendlich zu einem korrekten Ergebnis (Fixpunktiteration -- Allquantifizierung) -- Hippokratie-Eigenschaft sorgt dafür, dass keine Transformation wieder etwas ändert, wenn Konsistenz bereits hergestellt ist.
%     Diese Eigenschaft ist in der Praxis möglicherweise zu strikt, da sie sehr starke Anforderungen an die Transformationen stellen müsste. Dafür wäre aber die Anwendungsfunktion trivial.
%     \item Mittlere Einschränkung: Es gibt eine Reihenfolge von ausgeführten Transformationen für jede Änderung die terminiert (Existenzquantifizierung) und die Ausführungsfunktion findet diese Reihenfolge.
%     Utopisch, dass die Anwendungsfunktion aus (potentiell sehr mächtigen) Transformationen die richtige Reihenfolge errechnen kann. Dafür aber (möglicherweise) weniger Anforderungen an die Transformationen (zumindest nicht mehr Anforderungen, denn die Allquantifizierung induziert die Existenzquantifizierung). Eine Funktion könnte dann zumindest nach best-effort versuchen, die richtige Reihenfolge zu finden und konservativ abbrechen, wenn sie diese nicht finden kann (also entweder konsistent terminieren oder terminieren mit der Aussage, dass es entweder keine solche Reihenfolge gibt -- bei relaxierten Anforderungen -- oder dass es sie nicht finden kann).  
%     \item Geringe Einschränkung: Es gibt potentiell keine Reihenfolge der Transformationen, die bei einer Änderung zu einer konsistenten Lösung kommt. Hier müsste die Ausführungsfunktion entsprechend einen Fehler ausgeben.
% \end{itemize}

%\subsection{Options in Consistency Relations}

\mnote{Consistency preservation rules may select contradictory options in consistency relations}
We have seen in the examples and the discussion in \autoref{chap:orchestration:problem:function_behavior} that an essential reason for the non-existence of a consistent orchestration is the existence of options within consistency relations.
This means that a condition element is allowed to correspond to to different condition elements to be considered consistent, like we have seen for the mapping of names in \autoref{fig:orchestration:no_orchestration}.
Different transformations can define different such options for specific elements, such that some of these options can never exist in globally consistent models, but only the ones that overlap between the consistency relations of all transformations can occur there.
Compatibility of the consistency relations ensures that there is at least one such element in the overlap of the consistency relations, because if there was no consistent set of models containing the condition element the relations would be considered incompatible.
Unfortunately, each transformation can only select one of these options to restore consistency when a condition element is added and if all transformations choose an element that is not in the overlap of the consistency relations, they will never find a consistent set of models.

\mnote{Reduce expressiveness by each condition element only in one consistency relation pair}
In consequence, an obvious option to reduce expressiveness of transformations in order to make the orchestration problem decidable by ensuring that a consistent orchestration does always exist would be to restrict consistency relations, such that each condition element is only allowed to occur in a single consistency relation pair of a consistency relation.
Thus, each condition element has a unique corresponding element to which it is considered consistent.
Then, the consistency preservation rules cannot select between different options to restore consistency and if the relations are compatible, all consistency relations relate elements in an equal way, thus the transformations must find exactly those elements.

\mnote{Consistency relation restriction does not solve the problem}
Although that approach will at least reduce the number of cases in which no consistent orchestration is found in our algorithm, there are still inputs for which no consistent orchestration exists.
Since we do not restrict the transformation in what they are allowed to do, they can perform arbitrary changes to restore consistency.
This especially includes that they may always returns changes that yield the same two models, which are consistent to that transformation but not to any models that can be delivered by the other transformations.
% Since we do not restrict the transformation in what they are allowed to do, they can perform arbitrary changes that restore consistency, such that alternations in the execution can occur.
% Consider three metamodels $\metamodel{A}, \metamodel{B}, \metamodel{C}$ with the following consistency relations and consistency preservation rules:
% \begin{align*}
%     &
%     \consistencyrelation{CR}{AB} = \setted{\tupled{\model{a}{1}, \model{b}{1}}, \tupled{\model{a}{2}, \model{b}{3}}, \tupled{\model{a}{3}, \model{b}{2}, \tupled{\model{a}{4}, \model{b}{4}} \\
%     &
%     \consistencypreservationrule{\consistencyrelation{AB}}(\model{a}, \model{b}, \change{\metamodel{A}}, \change{\metamodel{B}}) = \begin{cases}
%     \end{cases}
%     &
%     \consistencyrelation{CR}{BC} = \setted{\tupled{\model{b}{1}, \model{c}{1}}, \tupled{\model{b}{2}, \model{c}{3}}, \tupled{\model{b}{3}, \model{c}{2}, \tupled{\model{b}{4}, \model{c}{4}} \\
%     &
%     \consistencyrelation{CR}{AC} = \setted{\tupled{\model{a}{1}, \model{c}{1}}, \tupled{\model{a}{2}, \model{c}{3}}, \tupled{\model{a}{3}, \model{c}{2}, \tupled{\model{a}{4}, \model{c}{4}} \\
% \end{align*}

\mnote{Example with restriction but without consistent orchestration}
Let $\class{A}{}$, $\class{B}{}$, $\class{C}{}$ be three classes, each with one attribute $n$ storing a number.
We define the following metamodels, each only consisting of one of those classes, and consistency relations between them that define that for each element in one model a corresponding the others with the same value of $n$ has to exist.
These consistency relations are obviously compatible.
This is a further simplification of our running example that requires persons, residents and employees with the same names.
Additionally, we define consistency preservation rules, each of them delivering changes that always, i.e., for every input, yield the same models that only consist of one element with a specific value.
The resulting models are chosen in a way such that they are consistent to the according consistency relation, but not to any of the others.
\begin{align*}
    & 
    \metamodelinstanceset{M}{1} = \mathcal{P}(\metamodelinstances{\class{A}{}}), %\\
    %& 
    \metamodelinstanceset{M}{2} = \mathcal{P}(\metamodelinstances{\class{B}{}}), %\\
    %& 
    \metamodelinstanceset{M}{3} = \mathcal{P}(\metamodelinstances{\class{C}{}}) \\[1em]
    &
    \consistencyrelation{CR}{12} = \setted{\tupled{a,b} \in \metamodelinstances{\class{A}{}} \times \metamodelinstances{\class{B}{}} \mid a.n = b.n}, \consistencyrelationset{CR}_{12} = \setted{\consistencyrelation{CR}{12}, \consistencyrelation{CR}{12}^T} \\
    &
    \consistencypreservationrule{\consistencyrelationset{CR}_{12}}(\model{m}{1}, \model{m}{2}, \change{\metamodel{M}{1}}, \change{\metamodel{M}{2}}) = (\change{\metamodel{M}{1}}', \change{\metamodel{M}{2}}') \\
    & \formulaskip
        \mathtext{with} \change{\metamodel{M}{1}}'(\model{m}{1}) = \setted{a \in \metamodelinstances{\class{A}{}} \mid a.n = 1} \land \change{\metamodel{M}{2}}'(\model{m}{2}) = \setted{b \in \metamodelinstances{\class{B}{}} \mid b.n = 1} \\[1em]
    &
    \consistencyrelation{CR}{13} = \setted{\tupled{a,c} \in \metamodelinstances{\class{A}{}} \times \metamodelinstances{\class{C}{}} \mid a.n = c.n}, \consistencyrelationset{CR}_{13} = \setted{\consistencyrelation{CR}{13}, \consistencyrelation{CR}{13}^T} \\
    &
    \consistencypreservationrule{\consistencyrelationset{CR}_{13}}(\model{m}{1}, \model{m}{2}, \change{\metamodel{M}{1}}, \change{\metamodel{M}{3}}) = (\change{\metamodel{M}{1}}', \change{\metamodel{M}{3}}') \\
    & \formulaskip
        \mathtext{with} \change{\metamodel{M}{1}}'(\model{m}{1}) = \setted{a \in \metamodelinstances{\class{A}{}} \mid a.n = 2} \land \change{\metamodel{M}{3}}'(\model{m}{3}) = \setted{b \in \metamodelinstances{\class{C}{}} \mid c.n = 2} \\[1em]
    &
    \consistencyrelation{CR}{23} = \setted{\tupled{b,c} \in \metamodelinstances{\class{B}{}} \times \metamodelinstances{\class{C}{}} \mid b.n = c.n}, \consistencyrelationset{CR}_{23} = \setted{\consistencyrelation{CR}{23}, \consistencyrelation{CR}{23}^T} \\
    &
    \consistencypreservationrule{\consistencyrelationset{CR}_{23}}(\model{m}{2}, \model{m}{3}, \change{\metamodel{M}{2}}, \change{\metamodel{M}{3}}) = (\change{\metamodel{M}{2}}', \change{\metamodel{M}{3}}') \\
    & \formulaskip
        \mathtext{with} \change{\metamodel{M}{2}}'(\model{m}{3}) = \setted{b \in \metamodelinstances{\class{B}{}} \mid b.n = 3} \land \change{\metamodel{M}{3}}'(\model{m}{3}) = \setted{c \in \metamodelinstances{\class{C}{}} \mid c.n = 3}
\end{align*}

\mnote{Consistency preservation rules always yield models not consistent to other relations}
The given consistency relations are compatible, they contain each condition element only in one consistency relation pair, and the consistency preservation rules are correct, as their result is consistent to the consistency relation.
Still, there is no consistent orchestration of the transformation for any input that is not yet consistent.
This is because the consistency preservation rules always produce models that are not consistent to the consistency relations between the other models.

\mnote{Exemplary consistency preservation rules perform unreasonable changes}
One might argue that the defined consistency preservation rules are highly unreasonable and will never occur in that way in practice.
We would probably assume the consistency preservation rules to preserve the input models and changes in some way instead of returning models that are completely unrelated with the input.
However, we not yet have an appropriate notion for that.
Some were on transformations, such as \cite{cheney2017LeastChangeBx-JOT,macedo2016alloy}, proposes a notion of \emph{least change} to ensure that transformations do not perform arbitrary unrelated changes, which could exclude that situations.

\mnote{Restriction of consistency relations is impractical}
Although the given example is rather artificial and although there might be the additional property of least change, which could further reduce the cases in which no consistent orchestration exists, the essential drawback is that these restrictions are not reasonable.
Allowing a condition element to occur in multiple consistency relation pairs is essential, because options for elements to considered is necessary, especially if there is a gap in the abstraction of two related metamodels.
For example, a UML class needs to be able to correspond to all Java classes that provide different implementations of that class.
Requiring that there is exactly one Java class that is considered consistent to a UML class is obviously not applicable in practice, thus this restriction would make the consistency notion useless.

\mnote{Least change property does even not solve the problem}
If we, instead, only require some notion of least change, such as that only elements are changed which are involved in a violated consistency relation, this does also not solve the problem.
In the example in \autoref{fig:orchestration:no_orchestration}, relating the names of employees, residents and persons, we have defined consistency preservation rules that only require changes to elements that actually violate consistency.
Nevertheless, we have shown that for these consistency preservation rules only specific orchestrations are consistent and that with some modification even no consistent orchestration exists.

\mnote{Unlikely to find practical restrictions that solve the problem}
In consequence, we found that even a well-defined restriction that is too strong to be applied in practice still cannot ensure that a consistent orchestration exists for possible every input, even if the examples with used to show that on are rather artificial.
Although this does not serve as a proof for the impossibility for find a suitable restriction that solves the orchestration problem, which is even impossible because there is no unique notion of what an acceptable restriction would be, the investigated case shows that it is unlikely to find practical restrictions that solve the problem, if even impractical restrictions do not solve it.

% Es kann z.B. sein, dass ein Element a ein Element b braucht. Eine andere Transformation braucht zwei as und dafür ein (oder zwei) c. Dann ist (a,b) konsistent und es ist auch kompatibel, weil es mit dem a ein Modell gibt in dem es konsistent ist, nämlich das mit dem zweiten a, aber das Modell mit einem a ist nur lokal konsistent. Wenn das nun immer von der Transformation ausgewählt wird, findet man nie einen konsistenten Zustand.

% - We have seen that selection of options is a problem that leads to the non-existence of orchestrations.
% - Intended solution: Each object may only have one corresponding element. Then if the object exists, exactly one other has to exist. This ensure that for any model a there is only exactly one consistent b.
% It can, however, be that a1, b1 and c1 are consistent and a4, b4 and c4 are, but only (a2, b3), (a3, b2), (a2, c3), (a3, c2), (b2, c3), (c3, b2) are consistent. 
% The relations may still be compatible, because for example (a2, b3, a4), (a3, b2, a5) and so ond could be consistent.
% Since we have no requirements such as a least change requirement, the transformation may, for every input, return one of the last pairs, although this actually does not make any sense.
% Then the transformations do only alternate between those models.
% This seems to be a rather theoretical consideration, because transformations will usually not perform arbitrary unreasonable changes, such as always returning the same models.
% However, even if we found that we can define an additional property, such as least change, such that requiring each object to only have one corresponding element in the consistency relations together with that property, then still the latter requirement is impractical, as usually an element can correspond to different others.
% At least in the case when consistency relations gap abstraction levels, this is necessary.
% For example, a UML class needs to be able to correspond to a Java class with any implementation of its bodies. Restricting it to only one is not practical.

% Although the restriction of relations is already an impractical restriction, it does still not solve the problem.
% Thus it is unlikely to find restrictions that are practical and solve the problem, although, for sure, it is not formally proven.

% Even some notion of least change that ensure that only elements are changed which are involved in any violated consistency relations will not be sufficient, as the example in \autoref{fig:orchestration:no_orchestration} has shown.
% There, only elements that are involved in the violated consistency relations are changed.
%One might argue that in that example the consistency relations contain pairs of elements that can never be created by the consistency preservation rules and thus should not be part of the consistency relations.
%Gegenargument: Es gibt oft unendlich viele Paare (z.B. alle Implementierung einer Java Klasse mit einer UML-Klasse. Wenn auf der UML-Seite auch noch ein Freiheitsgrad ist, dann wird jede UML-Instanz der Klasse auf eine Standard Java-Klasse abgebildet und jede Java-Klasse auf eine Standard UML-Klasse, aber die Zusatzinformationen kommen erst durch manuellen Anreicherung dazu und sind dadurch vorhanden, aber nicht durch die Transformation)


\subsection{Confluence in Transformation Networks}
%\todo{Zeigen: Konfluenz führt zu Konvergenz. Aber konfluenz ist zu starke Anforderung, außerdem heißt Konvergenz, dass egal welche Reihenfolge der Transformationen man wählt am Ende immer das gleiche Ergebnis rauskommt. Das muss aber nicht so sein. Gebe Beispiel mit groß klein Schreibung, wo je nach Reihe folge verschieden elosungen rauskommen}
%\todo{Diss: Konvergenz einführen, hinreichende Eigenschaft? Notwendige Eigenschaft?}
%\todo{Discuss confluence and convergence!}

\begin{figure}
    \centering
    \includegraphics[width=\textwidth]{figures/correctness/orchestration/confluence.png}
    \caption[Confluence of transformations]{Counterexample for the practicality of confluence}.
    \label{fig:orchestration:confluence}
\end{figure}

\mnote{Confluence ensures that any consistent orchestration yields same result}
Confluence is an even stronger requirement than the existence of an optimal orchestration.
In literature~\cite{stevens2020BidirectionalTransformationLarge-SoSym}, confluence in a transformation network is described as the property that for given models and changes a consistent orchestration exists and that two consistent orchestrations for the same input always yield the same models.
Thus, executing transformations in any order such that the result is consistent will deliver the same result.
It is, however, easy to see that this is an impractical requirement.

\mnote{Counterexample for confluence}
In the example depicted in \autoref{fig:orchestration:confluence}, derived from the running example, three consistency relations expect for each person, employee and resident two corresponding others to exist.
They need to have the same name or, in case of the relations between persons and employees, as well as between residents and employees, the employee may have the same name in lower case.
The consistency preservation rule between persons and employees ensures that an employee with the same name exists, whereas the one between residents and employees ensures that an employee with the name in lower case exists.
Whenever a person is added, two consistent orchestrations can be distinguished.
First, the transformation between persons and employees can be executed, either followed or preceded by the one between persons and residents. Then all elements have the same name.
The models are also consistent to the relation between residents and employees, because the relation allows the names to be equal.
Second, the transformation between persons and residents can be executed, followed by the one between residents and employees.
Then the employee has the name in lower case, but still this is consistent to the relation between persons and employees.

\mnote{Confluence generally accepted as impractical}
Apart from that artificial example, such a situation can always occur if transformations have different options for elements to be consistent.
If there is not a single element that is in the overlap of consistent elements between all transformations, the result may be any of the elements in the overlap.
And the result may depend on which transformation made the first selection that fell into the overlap.
Finally, \textcite[p. 14]{stevens2020BidirectionalTransformationLarge-SoSym} also states explicitly that a network will only be confluent under very specific circumstances.

% Another intuitive notion of confluence would require that information flows together in a compatible way, i.e., the execution of a transformation does not violate consistency to other adjacent transformations.
% Formally this can be described as: A network is confluent when in every cycle of transformations t1,...tn the execution t1...ti-1 and tn,...,ti for any i leads to the same changes in the network.
% This means it does not matter in which order two transformations propagating information to the same model are executed.
% This, in consequence, means that the execution of a transformation cycles always yields models that are consistent to all transformations. Although two transformations affected the same model, consistency to the earlier executed one is not destroyed again.
% Finally, this reduces to the situation that executing each transformation once restores consistency to all transformations, which is, as we have seen, impractical.



\section{Conservative Approximation of the Orchestration Problem}

We found that we cannot achieve optimality (undecidability) for arbitrary transformations, we cannot restrict transformation such that we achieve decidability and thus optimality.
So, we need to deal with situation can we cannot resolve all networks.
Conclude: Conservativeness rather than correctness or achieving optimality (but improving the latter one)

\todo{Question whether having no upper bound is only a theoretical or also a practical problem. The given example is rather artificial, so we may stop in practice without excluding relevant cases}

\todo{Remove much of the following}

\mnote{Achieving a correct application function}
The definition of the application function basically ensures that the function either returns $\bot$ or executes the \modellevelconsistencypreservationrules given by the orchestration function to retrieve a changes tuple of models.
It is considered \emph{correct} if it ensures that its result is either $\bot$ or a consistent model tuple by executing the \modellevelconsistencypreservationrules given by the orchestration function.
% A correct application function thus has to ensure that its result is either $\bot$ or a consistent model tuple by executing the \modellevelconsistencypreservationrules given by the orchestration function.
In consequence, the application function can be realized by simply executing the result of the orchestration function and check whether the resulting model tuple is consistent or not and return an appropriate result.
Such a realization is generic and does not depend on the actual consistency preservation rules and orchestration function but represents a generic behavior.
Additionally, this gives an implementation of that function the ability to present a faulty result to the user, which eases finding out why no consistent state was reached.

\mnote{Correctness is not crucial}
Finally, correctness is not crucial, because correctness can easily be achieved by performing any execution of transformations and just ensuring that we terminate at some point in time and then decide whether the resulting models are consistent or not and appropriately deliver the result.

\mnote{How to define an orchestration function that is as optimal as possible?}
The remaining difficulty is how to define an orchestration function that fulfills the definition, i.e., to find a finite sequence of transformations, and also one that improves optimality, as an \emph{optimal} function can never be given.
Although the definition of the orchestration function proposes a closed description of that function, in practice such a function will not have a closed form but will be realized as an algorithm that dynamically decides which transformation to execute next.
Therefore the arising problem is that the length of the sequence to execute is not known a priori. Therefore, we need some abortion criterion. When a consistent result is found, this criterion is easy. But since we do not know whether a sequence exist, we need an abortion criterion that is reasonable and does not cut off the process although a consistent solution could be found, thus reducing optimality.
A simple realization for that algorithm to deliver a finite sequence of transformations would be to define a fixed termination criterion, such as a specific number of transformation executions. However, there is no upper bound for the number of executed transformations necessary to achieve consistency. Still, a fixed number (even 0) could be defined for the number of executed transformations to fulfill the definition. Hence, optimality would be 0 then as a consistent result is never reached. We therefore discuss in the following how to define an appropriate orchestration function and how to optimize it.

\mnote{Achieving a correct application function}
The definition of the application function basically ensures that the function either returns $\bot$ or executes the \modellevelconsistencypreservationrules given by the orchestration function to retrieve a changes tuple of models.
It is considered \emph{correct} if it ensures that its result is either $\bot$ or a consistent model tuple by executing the \modellevelconsistencypreservationrules given by the orchestration function.
% A correct application function thus has to ensure that its result is either $\bot$ or a consistent model tuple by executing the \modellevelconsistencypreservationrules given by the orchestration function.
In consequence, the application function can be realized by simply executing the result of the orchestration function and check whether the resulting model tuple is consistent or not and return an appropriate result.
Such a realization is generic and does not depend on the actual consistency preservation rules and orchestration function but represents a generic behavior.
Additionally, this gives an implementation of that function the ability to present a faulty result to the user, which eases finding out why no consistent state was reached.

\mnote{Correctness is not crucial}
Finally, correctness is not crucial, because correctness can easily be achieved by performing any execution of transformations and just ensuring that we terminate at some point in time and then decide whether the resulting models are consistent or not and appropriately deliver the result.

\mnote{How to define an orchestration function that is as optimal as possible?}
The remaining difficulty is how to define an orchestration function that fulfills the definition, i.e., to find a finite sequence of transformations, and also one that improves optimality, as an \emph{optimal} function can never be given.
Although the definition of the orchestration function proposes a closed description of that function, in practice such a function will not have a closed form but will be realized as an algorithm that dynamically decides which transformation to execute next.
Therefore the arising problem is that the length of the sequence to execute is not known a priori. Therefore, we need some abortion criterion. When a consistent result is found, this criterion is easy. But since we do not know whether a sequence exist, we need an abortion criterion that is reasonable and does not cut off the process although a consistent solution could be found, thus reducing optimality.
A simple realization for that algorithm to deliver a finite sequence of transformations would be to define a fixed termination criterion, such as a specific number of transformation executions. However, there is no upper bound for the number of executed transformations necessary to achieve consistency. Still, a fixed number (even 0) could be defined for the number of executed transformations to fulfill the definition. Hence, optimality would be 0 then as a consistent result is never reached. We therefore discuss in the following how to define an appropriate orchestration function and how to optimize it.






\subsection{A Gradual Notion of Optimality}
%\subsection{Reducing Conservativeness}

Due to Turing-completeness of the network this would mean that the orchestration function can decide whether a Turing machine halts, which is proven impossible.
Thus, our only goal can be to achieve optimality as far as possible in terms of reducing the degree of conservativeness, i.e., reduce the cases in which no sequence is found although it exists.

We can define a measure for the optimality of an orchestration function:
\begin{align*}
    &
    Optimality_{\orcfunction{\consistencypreservationruleset{}}} = \frac{\mathtext{\# of model / delta pairs for which the function finds an order that terminate consistently}}{\mathtext{\# of model / delta pairs for which an order that terminates consistently exists}}
\end{align*}

In fact, both these numbers usually infinite, an there is an infinite number of possible models and deltas. However, it does finally not matter for us what the actually value is, but only how to improve that value.
\todo{We have to map that value to compatibility, which reduces the number of potential false orders.}


\textbf{Overall Goal:} Find correct orchestration function that improves optimality.

There are two ways to improve optimality of the orchestration function:
\begin{enumerate}
    \item Optimize the orchestration function, i.e., find a good order (probably this is not possible), at least find an order that helps the developer to find problems
    \item Optimize the input, i.e., define requirements to the transformations and their relations representing the input to optimize optimality
\end{enumerate}
\todo{We need an example for that}

Both goes hand in hand, because restrictions to the input can never lead to an orchestration function that always terminates without leading to unsupported relevant cases.

This conform to two approaches:
\begin{enumerate}
    \item Dynamic decision about selected transformation and abortion criteria
    \item Constructive restrictions that ensure that appropriate order is (easily) found
\end{enumerate}

\todo{Application function can be generically defined, orchestration maybe not? We actually want to ensure that both are generic and none of them has to be defined for a specific project.}

\begin{itemize}
    \item We conclude that we need to deal with the situation of undecidability. In consequence, orchestration must operate conservatively, i.e., we cannot assume to always find a solution, but if find it, it must be consistent.
    \item One approach could be to reduce conservativeness. But even if we reduced conservativeness, we would not be able to completely eliminate it and thus have to deal with the situation that the strategy does not find a solution although it may exist.
    \item We thus propose an approach that helps to identify the reason when the strategy is in the conservative case, i.e., not able to find a solution although it exists.
\end{itemize}


\subsection{Systematic Improvement of Optimality}
%\paragraph{Avoidance of Non-Termination}

In consequence, we propose to dynamically deal with alternation / divergence.
To detect alternations, the execution can simply track if a state way already processed. Apart from spatial problems, this does always work.
Finding divergence is not that easy, because it is generally not possible to define an upper bound for the number of executions of a single transformation.
This is due to the reason that, again, this conflicts with the Halting problem.
% We can see this at the simple example in \autoref{fig:formal:noupperboundexample}.

% \begin{figure}
%     \centering
%     \includegraphics[width=\textwidth]{figures/correctness/orchestration/no_upper_bound_example_old.png}
%     \caption{Example for no upper bound}
%     \label{fig:formal:noupperboundexample}
% \end{figure}

% Depending on the value X, the transformations have to be executed X times to result in a consistent state. This value can be arbitrarily chosen, thus an arbitrary number of executions may be necessary to terminate in a consistent state.

% \todo{Moved from single execution section -> revise}
% \begin{theorem}[Orchestration with Unbounded Executions]
%     \label{theorem:unbounded_execution}
%     For any set of transformations $\transformationset{T}$, there can be models $\modeltuple{m}$ and changes $\changetuple{}$ to them for which each possible orchestration function $\orcfunction{\transformationset{T}}$ with whom $\appfunction{\orcfunction{\transformationset{T}}}(\modeltuple{m}, \changetuple{})$ is consistent, such that $\abs{\orcfunction{\transformationset{T}}(\modeltuple{m}, \changetuple{})} > \abs{\transformationset{T}}$.
% \end{theorem}
% \begin{proof}
%     We know from \autoref{lemma:minimal_executions} that $\transformationset{T}_{inc}$ requires at least $4$ executions of $\transformation{T}_{12}$ for the inputs defined in \autoref{lemma:minimal_executions} when selecting $x \geq 5$.
%     Thus, for any orchestration function, we know that $\abs{\orcfunction{\transformationset{T}}(\modeltuple{m}, \changetuple{})} > 4 > 3 = \abs{\transformationset{T}_{inc}}$.
%     This proves the theorem by example.
% \end{proof}

From an engineering perspective, this is still unwanted behavior. We claim that a transformation network that takes thousands of executions of the same transformation to find a consistent state works not as expected and if running into a failure would expose severe problems to find the reasons for that failures.
Thus, we propose to simply abort the execution after some time to be sure not to run in an endless loop.

Finally, this problem is comparable to ordinary programming, because there the same situations regarding alternation and divergence can occur that result in non-termination of a program.
As we all know, it is impossible to systematically avoid that, but just possible to carefully develop the program and apply best practices to avoid such situations.

\textbf{Conclusion:} We cannot systematically improve optimality / reduce conservativeness and also not dynamically terminate when divergence is detected. Especially, there is no guarantee wo when also alternation is detected. There can be an arbitrary high number of execution before alternation is detected.
Thus, we propose an algorithm that terminates deterministically and returns $\bot$ also when it is not able to find a solution in a fixed number of steps, but improves the ability for the transformation developer or user to find out why in specific situation no solution can be found.



\paragraph{Outlook}

We need to deal with the situation that no consistent orchestration exists and that we are not able to find it, even if it exists, as the problem space is arbitrarily large (arbitrary high number of transformation executions).
This forbids approaches such as backtracking to find an appropriate solution.
\todo{Discuss Backtracking}

We consider alternation as a possibility to reduce the number of cases in which non-termination can occur, thus improving optimality, but by its dynamic detection as well as its avoidance.
This is an easy way to improve optimality.
However, we can never achieve an optimality of 1, thus we have to deal with the situation that some executions will fail.
We therefore focus on how we may orchestrate the transformations such that the process of finding the reason for not finding a consistent orchestration is supported.





\subsection{Dynamic Detection of Alternation} % Divergence and alternation
\label{chap:orchestration:conservative:alternation}

The proposed algorithm, like any algorithm, is supposed to \emph{terminate} in a specific \emph{state} to be considered correct.
In our case, a correct state, as required by an application function it implements, is the return of consistent models or $\bot$, which the algorithm fulfills by construction.
In particular, the algorithm will never return models that are inconsistent, neither because it does not detect that they are inconsistent nor that it detects that they are inconsistent but still returns them.
From our previous findings regarding decidability, we know that we cannot expect the algorithm to realize an optimal application function.
Thus, we either need to implement \function{Orchestrate} such that it always returns $\bot$ after a finite number of executions to ensure termination, which results in returning $\bot$ although consistent models are expected, or we allow an arbitrary number of executions to improve the ability to find consistent results but accept that the algorithm may not terminate.

We have discussed that non-termination of the algorithm can occur because no consistent orchestration exists at all or because the algorithm is not able to find it.
A special case of non-termination is \emph{alternation}, which means that the same states are passed repeatedly. 
In case of transformation networks, alternation means that from some point in time the subsequent executions of the transformations in Line~\ref{algo:orchestration:application:line:stepcalculation} of \autoref{algo:orchestration:application} repeatedly produce the same sequence of results, i.e., of changes.
% Non-termination can, in general, manifest in terms of \emph{alternation} or \emph{divergence}, which means that either the same states are passed repeatedly or that an infinite sequence of different states is produced.
% In case of transformation networks, alternation means from some point in time the subsequent executions of the transformations in Line~\ref{algo:orchestration:application:line:stepcalculation} of \autoref{algo:orchestration:application} repeatedly produce the same sequence of results, i.e., of changes.
% Divergence means that from some point in time all results, i.e., changes, produced in Line~\ref{algo:orchestration:application:line:stepcalculation} differ.
In contrast to non-termination in general, the scenario of alternation can at least be avoided by construction.
% To this end, the history of change produced by the algorithm in Line~\ref{algo:orchestration:application:line:stepcalculation} has to be stored.
% It can either be used by the \function{Apply} function to detect alternation or by passing it to the \function{Orchestrate} function to influence the selection of transformations to avoid alternation.

\begin{definition}[Alternation of Apply Algorithmus]
    \label{def:applyalternation}
    Let there be a number $n$ of execution of the transformation execution loop in Lines~\autoref{algo:orchestration:application:line:startorchestrate}--\autoref{algo:orchestration:application:line:endorchestrate} of \autoref{algo:orchestration:application}, such that for all numbers of execution $> n$ there is a sequence of executed transformations and generated changes that occur at least two times subsequently at the end of the current states of $\mathvariable{executedTransformations}$ and $\mathvariable{generatedChanges}$.
\end{definition}

The \function{Orchestrate} function receives the history of transformations and changes and is thus able to identify the situation that the same sequence of transformations was already executed and produced equal changes in each application.
This allows it to implement the function in a way that it does not return the same sequence of transformations when it was already passed and produced the same changes.
If a concrete realization of the \function{Orchestrate} function is not implemented in a way that it can react to the detection of alternation and produce a different sequence of transformations, it can at least return $\bot$ to ensure termination of \function{Apply}, because repeated execution of the same transformations will still returns the same changes. 

Alternation produces orchestrations that can never yield consistent models, thus they are part of the problem space $P_{i}$ of finding an orchestration for a given input $i$ of models and changes but can never be part of the solution space $S_{i}$ containing the consistent orchestrations.
Avoid such alternations thus improves the possibility to find a consistent orchestration, because the number of consistent orchestrations to the number of orchestrations that can potentially be applied is improved.

\todo{Maybe add reference to conservative orchestration section for the dynamic detection of alternation, if appropriate.}

%Divergence: If it passes the same changes again, then it either does not pass those changes again 

% Derive from the previous insights that whenever the algorithm does not terminate, we can have two situations: divergence and alternation.
% Prove that no other options for occurring situation exist!

% Problemraum:
% \begin{itemize}
%     \item Ziel ist, dass ein Netzwerk von Transformationen nach einer Änderung in einem konsistenten Zustand terminiert. D.h. Korrektheit stellt Anforderungen an \emph{Terminierung}, sowie den \emph{Zustand} bei Terminierung.
%     \item Folgende Abweichungen davon können auftreten:
%     \begin{enumerate}
%         \item Nicht-Terminierung: Das Netzwerk terminiert nicht. Das bedeutet im Prinzip, dass die Ausführungsfunktion (bzw. der Laufzeit-Algorithmus, der die Funktion dynamisch emuliert) nicht \emph{sound} ist. Soundness der Ausführungsfunktion setzt voraus, dass die berechnet Aufrufsequenz endlich ist. Wenn die Ausführung nicht terminiert, bedeutet das, dass entweder die gleichen Zustände mehrfach durchlaufen werden oder eine Sequenz unendlich vieler Zustände produziert wird. Denn wenn beides nicht der Fall ist, gibt es eine endliche Sequenz unterschiedlicher Zustände, d.h. Terminierung. Das bedeutet, dass es folgende zwei Möglichkeiten gibt:
%         \begin{itemize}
%             \item Alternierung: Die gleichen Zustände werden mehrfach durchlaufen.
%             \item Divergenz: Es werden unendlich viele Zustände produziert.
%         \end{itemize}
%         \item Inkonsistente Terminierung: Die Ausführungsfunktion bzw. der Algorithmus beendet die Ausführung, aber in einem inkonsistenten Zustand. Hier lassen sich ebenfalls wieder zwei Fälle unterscheiden.
%         \begin{itemize}
%             \item Unerkannte Inkonsistenz: Der Algorithmus terminiert und denkt, der Zielzustand wäre konsistent. Dies bedeutet aber direkt, dass nicht alle Konsistenzrelationen erfüllt sind, was, zumindest in der Theorie, einfach zu prüfen wäre (entweder durch Prüfung der Relationen oder durch Ausführung der hippokratischen Transformationen, die alle nichts tun dürften)
%             \item Erkannte Inkonsistenz: Der Algorithmus terminiert, wissend dass die Lösung nicht konsistent ist. Dies kann entweder sein, weil eine Transformation für zwei Modelle in einem inkonsistenten Zustand nicht mehr anwendbar ist, oder weil irgendein anderes Abbruchkriterium erreicht ist.
%         \end{itemize}
%     \end{enumerate}
% \end{itemize}

% Assume we have an algorithm that sequentially applies transformations.
% It stops as soon as a transformation cannot be applied or the models are consistent.
% Then we need to guarantee termination.
% We need to avoid that transformations can be applied indefinitely never leading to consistent models.

% Reasons for this situation are alternation and divergence.
% Prove that if we do not pass the same model state again (alternation) and if there is no indefinite number of model states (divergence), the algorithm terminates.
% Thus, if we ensure that any execution order of transformations does never lead to alternation and divergence, we know that the algorithm terminates!!

% \begin{itemize}
%     \item Zeigen, dass es Beispiele gibt, in denen es unabhängig von der Ausführungsreihenfolge immer zu einer Alternierung kommt
%     \item Zeigen, dass es Beispiele gibt, in denen es unabhängig von der Ausführungsreihenfolge immer zu einer Divergenz kommt.
%     \item Die Beispiele sollten zeigen, dass wir keine Einschränkungen an die Transformationen machen können, was das Problem aushebelt. D.h. egal welche Einschränkungen ich an die Transformationen definiere, es lassen sich immer Beispiele konstruieren, in denen es keine Ausführungsreihenfolge gibt, in denen sie terminieren.
%     \item Mathematisch zeigen, dass Alternierung und Divergenz die einzigen Probleme sind. D.h. wenn nicht der gleiche Zustand mehrmals durchlaufen wird (Alternierung) und es nicht unendlich viele Zustände gibt (Divergenz), dann ist die Folge endlich.
%     %\item Außerdem mathematisch die Abbildung von Transformationen auf Turing-Maschinen zeigen und damit ableiten, dass allgemeine Netzwerke erstmal nicht terminieren müssen (Abbildung auf Halteproblem)
% \end{itemize}

% To avoid these problems by construction, we discussed before that we need to achieve that P = S, such that the application function can execute transformations in an arbitrary order to achieve consistency.

% Another possibility would be to allow the problems and detect them dynamically and react to them.
% We will finally discuss that in the last section.

%In the following, we discuss whether and how we may restrict synchronizing transformations, such that an arbitrary execution order can avoid divergence and alternation, such that the algorithm terminates.




\subsection{Monotony for Avoiding Alternation}

We have discussed in \autoref{chap:orchestration:conservative:alternation} that alternation, as a specific kind of non-termination scenario, can be avoided by construction of the orchestration function or at least can be detected by the \function{Apply} algorithm.
Instead of detecting alternation during orchestration, we may also restrict the transformation network such that no alternation can occur by construction.
We can achieve this by defining a notion of monotony for the transformations.

For the construction of synchronizing bidirectional transformations by unidirectional consistency preservation rules in \autoref{chap:synchronization:bidirectional:transformations}, we have have defined the property of \emph{partial consistency improvement}, which is a monotony notion for the two unidirectional consistency preservation rules of a synchronizing bidirectional transformation, as each execution of them improved that property.
We can, however, not define monotony in a similar way for the whole transformation network because of two reasons.
First, the notion of partial consistency is not applicable for transformation networks, because each transformation needs to restore consistency between two models completely.
Second, since each transformation is developed independently from all others, we cannot apply the notion of partial consistent improvement to the other models by restricting how far a transformation may violate consistency to the other transformations.

We thus define a different notion of monotony for transformations as follows.
\begin{definition}[Monotone Synchronizing Transformation]
    \label{def:monotonetransformation}
    Let $\metamodeltuple{M} = \metamodelsequence{M}{n}$ be metamodels and let $\transformation{t}$ be a synchronizing transformation. We call $\transformation{t}$ monotone if it does not change elements that were already changed, i.e.
    \begin{align*}
        &
        \forall \modeltuple{m} = \tupled{\model{m}{1}, \dots, \model{m}{n}} \in \metamodeltuple{M}, \changetuple{\metamodeltuple{M}} = \tupled{\change{\metamodel{M}{1}}, \dots, \change{\metamodel{M}{n}}} \in \changeuniverse{\metamodeltuple{M}} : \\
        &
        \bigl(\exists \changetuple{\metamodeltuple{M}}' = \tupled{\change{\metamodel{M}{1}}', \dots, \change{\metamodel{M}{n}}'} \in \changeuniverse{\metamodeltuple{M}} : \generalizationfunction{\metamodeltuple{M},\transformation{t}}(\modeltuple{m}, \changetuple{\metamodeltuple{M}}) = \changetuple{\metamodeltuple{M}}' \\
        & \formulaskip
        \Rightarrow
        % WE CANNOT ASSUME TRANSFORAMTION TO BE STRONG MONTONE, BECAUSE IF TRANSFORMATION IS EXECTED FOR ALREADY CONSISTENT MODELS, IT CANNOT CHANGE ANYTHING
        % (\changetuple{\metamodeltuple{M}}'(\changetuple{\metamodeltuple{M}}(\modeltuple{m})) = \changetuple{\metamodeltuple{M}}(\modeltuple{m}) \Rightarrow \modeltuple{m} \consistenttomath \transformationset{T}) \\
        % & \formulaskip
        % \land 
        \forall i \in \setted{1, \dots, n} : 
        (\change{\metamodel{M}{i}}(\model{m}{i}) \setminus \model{m}{i} \subseteq \change{\metamodel{M}{i}}'(\change{\metamodel{M}{i}}(\model{m}{i})) \\
        & \formulaskip\formulaskip
        \land
        (\model{m}{i} \setminus \change{\metamodel{M}{i}}(\model{m}{i})) \cap \change{\metamodel{M}{i}}'(\change{\metamodel{M}{i}}(\model{m}{i})) = \emptyset)
        \bigr)
        %\change{\metamodeltuple{M}}(\modeltuple{m}) \subseteq \modeltuple{m} \cup \changetuple{\metamodeltuple{M}}'(\changetuple{\metamodeltuple{M}}(\model{m}{}))
        %\land
        %\modeltuple{m} \cup \changetuple{\metamodeltuple{M}}'(\changetuple{\metamodeltuple{M}}(\model{m}{})) \subseteq \changetuple{\metamodeltuple{M}}(\model{m}{})\big)
    \end{align*}
\end{definition}

The definition is based on the idea that transformations are only supposed to append changes but not to revert previous changes.
This means that elements that were introduced by previous changes still need to be present after applying the transformation.
Additionally, elements that were removed are not allowed to be added by the transformation again.
Thus all elements of the originally changed models were either contained in the original models or are contained in the models yielded by the transformation application, which leads to the model relations in the definition.
Additionally, 

% \begin{definition}[Strongly Montone Synchronizing Transformation]
%     Let $\metamodeltuple{M}$ be metamodels and let $\transformation{t}$ be a monotone synchronizing transformation. We call $\transformation{t}$ strongly monotone if it does not perform any changes only when all models are already consistent does not change elements that were already changed, i.e.
% \end{definition}

Having only monotone transformations ensures that each orchestration, which does not apply a transformation to already consistent models, yields a sequence of pairwise different model states, if the transformations are sequentially applied.

\begin{lemma}
    \label{lemma:monotonetransformationsnosamestates}
    Let $\transformationset{T}$ be a set of correct monotone synchronizing transformations for a tuple of metamodels $\metamodeltuple{M}$.
    Then for all models and changes, as well as any orchestration $\tupled{\transformation{t}_{1}, \dots, \transformation{t}_{m}}  \; (\transformation{i} \in \transformationset{T})$ that does contain a transformation when its models are already consistent, then prefixes of that orchestration only yield the same models if those prefixes are consistent orchestrations, i.e.
    \begin{align*}
        &
        \forall \modeltuple{m} \in \metamodeltupleinstanceset{M}, \changetuple{\metamodeltuple{M}} \in \changetuple{\metamodeltuple{M}} : \forall i, k \in \setted{1, \dots, m} : \\
        &
        \generalizationfunction{\metamodeltuple{M}, \transformation{t}_{i}} \concatfunction \dots \concatfunction \generalizationfunction{\metamodeltuple{M}, \transformation{t}_{1}}(\modeltuple{m}, \changetuple{\metamodeltuple{M}}) = \generalizationfunction{\metamodeltuple{M}, \transformation{t}_{k}} \concatfunction \dots \concatfunction \generalizationfunction{\metamodeltuple{M}, \transformation{t}_{1}}(\modeltuple{m}, \changetuple{\metamodeltuple{M}}) \\
        & \formulaskip 
        \Rightarrow
        \generalizationfunction{\metamodeltuple{M}, \transformation{t}_{k}} \concat \dots \concat \generalizationfunction{\metamodeltuple{M}, \transformation{t}_{1}}(\modeltuple{m}, \changetuple{\metamodeltuple{M}}) \consistenttomath \transformationset{T}
        \end{align*}
\end{lemma}
\begin{proof}
    Assume that there are two prefixes $\tupled{\transformation{t}_{1}, \dots, \transformation{t}_{i}}$ and $\tupled{\transformation{t}_{1}, \dots, \transformation{t}_{k}}$ of an orchestration, $i < k$ without loss of generality, such that they yield the same inconsistent models, i.e., $\generalizationfunction{\metamodeltuple{M}, \transformation{t}_{i}} \concatfunction \dots \concatfunction \generalizationfunction{\metamodeltuple{M}, \transformation{t}_{1}}(\modeltuple{m}, \changetuple{\metamodeltuple{M}}) = \generalizationfunction{\metamodeltuple{M}, \transformation{t}_{k}} \concatfunction \dots \concatfunction \generalizationfunction{\metamodeltuple{M}, \transformation{t}_{1}}(\modeltuple{m}, \changetuple{\metamodeltuple{M}})$ although $\generalizationfunction{\metamodeltuple{M}, \transformation{t}_{k}} \concatfunction \dots \concatfunction \generalizationfunction{\metamodeltuple{M}, \transformation{t}_{1}}(\modeltuple{m}, \changetuple{\metamodeltuple{M}})$ is not consistent to $\transformationset{T}$.
    We denote the change tuple delivered by any prefixes of length $l$ as $\changetuple{\metamodeltuple{M},l} = \tupled{\change{\metamodel{M}{1},l}, \dots, \change{\metamodel{M}{n}, l}}$ with $\tupled{\modeltuple{m}, \changetuple{\metamodeltuple{M},l}} = \generalizationfunction{\metamodeltuple{M}, \transformation{t}_{l}} \concatfunction \dots \concatfunction \generalizationfunction{\metamodeltuple{M}, \transformation{t}_{1}}(\modeltuple{m}, \changetuple{\metamodeltuple{M}})$.
    We know that the sequence of changes between the two prefixes does not perform any changes and thus acts like the identity function, i.e., $\generalizationfunction{\metamodeltuple{M}, \transformation{t}_{k}} \concatfunction \dots \concatfunction \generalizationfunction{\metamodeltuple{M}, \transformation{t}_{i+1}}(\modeltuple{m}, \changetuple{\metamodeltuple{M},i}) = \identitychange(\modeltuple{m}, \changetuple{\metamodeltuple{M},i})$
    and thus $\changetuple{\metamodeltuple{M},i}(\modeltuple{m}) = \changetuple{\metamodeltuple{M},k}(\modeltuple{m})$.
    We also know that all the transformations between the prefixes, i.e., all transformations $\transformation{t}_{l}$ for each $l$ with $i < l \leq k$, do not act like the identity function for their inputs, i.e., $\generalizationfunction{\metamodeltuple{M}, \transformation{t}_{l}}(\modeltuple{m}, \changetuple{\metamodeltuple{M},l-1}) \neq \identitychange(\modeltuple{m}, \changetuple{\metamodeltuple{M},l-1})$.
    Otherwise, the models affected by the transformation would either have been consistent before, which conflicts with the assumption that the orchestration does not contain a transformation when its models are already consistent, or they would not be consistent afterwards, which conflicts with the assumed correctness of the transformations.

    Thus, each transformation $\transformation{t}_{l} \; (i < l \leq k)$ performs modifications to the change tuple, i.e., adds or removed further elements.
    This especially applies to $\transformation{t}_{i+1}$.
    Let us assume that $\transformation{t}_{i+1}$ adds an element (analogous argumentation for the removal).
    %modifies the change tuple such that it adds or removes further elements.
    Then there is a model that contains the element after applying the change generated by the transformation, i.e., $\exists s \in \setted{1, \dots, n} : \exists \modelelement{e} : \modelelement{e} \in \change{\metamodel{M}{s},i+1}(\model{m}{s}) \setminus \change{\metamodel{M}{s},i}(\model{m}{s})$.
    Due to the transformations being monotone, we know that this element was not contained before, especially not in $\model{m}{s}$, as otherwise $\modelelement{e} \in \model{m}{s} \setminus \change{\metamodel{M}{s},i}(\model{m}{s})$ and thus $\model{m}{s} \setminus \change{\metamodel{M}{s},i}(\model{m}{s}) \cap \change{\metamodel{M}{s},i+1}(\model{m}{s}) \neq \emptyset$, which conflicts the definition of monotone transformations for $\transformation{t}_{i+1}$.

    Since $\change{\metamodel{M}{s},k}(\model{m}{s}) = \change{\metamodel{M}{s},i}(\model{m}{s})$, we know that $\modelelement{e} \not\in \change{\metamodel{M}{s},k}(\model{m}{s})$.
    Thus, there must be a transformation $\transformation{t}_{l}$ with $i+1 < l \leq k$ which, in turn, removes this element, i.e., $\modelelement{e} \in \change{\metamodel{M}{s},l-1}(\model{m}{s}) \setminus \change{\metamodel{M}{s},l}(\model{m}{s})$.
    Then $\modelelement{e} \in \change{\metamodel{M}{s},l-1}(\model{m}{s}) \setminus \model{m}{s}$ and thus $\change{\metamodel{M}{s},l-1}(\model{m}{s}) \setminus \model{m}{s} \not\subseteq \change{\metamodel{M}{s},l}(\model{m}{s})$, which conflicts the definition of monotone transformations for $\transformation{t}_{l}$.

    In consequence, each transformation $\transformation{t}_{l} \; (i < l \leq k)$ can neither add nor remove an element, thus our assumption that two prefixes that yield the same inconsistent models does not hold, which proves the lemma.
\end{proof}

With that insight, it is easy to see that given only monotone transformation, no alternation can occur in our algorithm \autoref{algo:orchestration:application}.

\begin{theorem}
    Given a set of correct, monotone synchronizing transformations $\transformationset{T}$.
    Then \autoref{algo:orchestration:application} cannot contain an alternation according to \autoref{def:applyalternation}, as long as $\function{Orchestrate}$ does not return a transformation whose models are already consistent.
\end{theorem}
\begin{proof}
    According to \autoref{lemma:monotonetransformationsnosamestates}, monotone transformations ensure that in an orchestration that does not contains transformations that need to be applied to already consistent models the application of two prefixes never yields the same changes.
    In consequence, the sequence of $\mathvariable{generatedChanges}$ in the transformation application in 
    Lines~\autoref{algo:orchestration:application:line:startorchestrate}--\autoref{algo:orchestration:application:line:endorchestrate} of \autoref{algo:orchestration:application} can never contain the same two changes.
    This would, however, be necessary to fulfill \autoref{def:applyalternation} for alternation.
\end{proof}

In fact, the guarantee of not producing the same state twice is even stronger than non-alternation, because alternation allows to pass the same state multiple times, as long as the same sequence of states is not passed repeatedly and infinitely.
It does, however, only make sense to pass the same state twice if the orchestration algorithm that selects the next transformation to execute is able to process that situation by trying different execution orders if an alternation occurs.
Thus, the less strict requirement for alternation is suited to make statements about the orchestration strategy but not about the individual transformations, as it is unlikely to find a property for a single transformation that gives a guarantee that depends on the execution order of transformations, like alternation does.

While monotone transformation give the guarantee of non-alternation, monotony according to \autoref{def:monotonetransformation} is not a property that we cannot assume to be fulfilled by all transformations.
Although is seems intuitive that a transformation should not remove elements that were added before and vice versa, this does also mean that, for example, an attribute value may only be changed once by the transformations.
This would, however, require the transformations to always make a choice for attributes that fits for all other transformations as well.
We have seen in different examples, such as the one depicted in \autoref{fig:orchestration:no_upper_bound} and \autoref{fig:orchestration:no_orchestration}, that it may be necessary to change elements multiple times, because the transformations select values with which the models only fulfill their own consistency relation but not those of the other transformations.
It may take several executions to find a value selection with which the models are consistent to all transformations.
We might say that the transformations need to \emph{negotiate} a consistent solution.

Still, the given examples were rather artificial, so they cannot be seen as an indicator for monotony to be not practically achievable.
It may, at least in some cases, be possible to specify transformations that are monotone.
Even if only some of the transformations are monotone, or if only specific rules of them are monotone, it improves the chance that an orchestration strategy finds a consistent orchestration.
Having the knowledge about the benefits of monotony gives a transformation developer the ability to implement it as often as possible.

Finally, the possibility to avoid alternation by construction can be combined with the ability of an orchestration strategy to react to alternation.
We have discussed in \autoref{chap:orchestration:conservative:alternation} that an orchestration strategy can detect alternation and adapt its strategy of selecting the next transformation in that case.
In addition, if monotony is given at least for some transformations, the orchestration strategy needs to try less execution orders and thus improves the chance of finding a consistent orchestration.

% \begin{figure}
%     \centering
%     \includegraphics[width=\textwidth]{figures/correctness/orchestration/monotony_counterexample.png}
%     \caption{Counterexample for monotony}
%     \label{fig:formal:monotonycounterexample}
% \end{figure}


%\todo{Go on here}

% \begin{itemize}
%     \item Muss eine Transformation mit jedem beliebigen Delta umgehen können müssen? Eine Einschränkung auf Monotonie würde dies verhindern. Bzw. wir müssten zeigen, dass es Konsistenzrelationen gibt, die unter der Anforderung an Monotonie nicht wiederhergestellt werden können. Bspw. fügt eine andere Transformation 3 Elemente hinzu, wo zwei mit dem anderen entsprechend der Konsistenzrelationen korrelieren und somit keine Witness-Struktur aufgebaut werden kann, die Konsistenz beweist. Das lässt sich durch Hinzufügen weiterer Elemente potentiell nicht auflösen (siehe Beispiele im SoSym-Paper).
%     \item Refer to synchronization chapter, where we introduced a monotony notion based on transformations being partial-consistency-improving. Here, in contrast, the CPRs cannot be aligned, such that we cannot, for example, expect one transformation not to lead to a reduction of consistency regarding consistency relations of other, previously executed transformations.
% \end{itemize}
% \begin{itemize}
%     \item Im Allgemeinen könnte eine Transformation beliebige dieser Deltasequenzen modifizieren. Wir verlangen jedoch, dass eine Transformation nur Deltas anhängt, also die Sequenzen länger werden
%     \item Genauer beschränken wir auch, welche Sequenzen eine Transformation sehen und ändern darf, genau gesagt darf sie die Sequenzen von zwei Modellen sehen und eine davon verlängern.
%     \item Hier kommt bereits der Unterschied zu bisherigen Transformationen, denn die sehen nur Deltas an einem Modell und erzeugen Deltas an dem anderen. Das ist bei uns schon gänzlich anders. Bidirektionale Transformationen unterstützen das im Übrigen auch nicht, sondern sind nur Spezifikationen, aus denen sich Wiederherstellungsroutinen für beide Richtungen ableiten lassen (siehe Stevens 2010)
% \end{itemize}

% \paragraph{Idea:} Require monotony to avoid alternation

% We would have to relax the definition of transformation to be monotone, because if a transformation is monotone, it may only append information, but this is not always possible, as can be seen in the following example. A monotone transformation must be able to return bottom if it cannot make further changes to restore consistency to the relation.

% \begin{definition}[Monotone Transformation]
%     Transformation gets models M and deltas D and produces new deltas D'. Taking the union of the original models M and the new models D'(M), then D(M) must be a subset of that, because other elements would have been added and removed afterwards or elements would have been changes once by D and again in a different way by D'.

%     Generally, monotony could also mean that only the same complete model state is not passed twice. \todo{Why dont we do that?}
% \end{definition}

% This would mean that each transformation only appends changes, i.e., if an element was added/removed, the transformation may not do the inverse. The same applies to attribute/reference changes: if an attribute/reference was already changes it may not be changed again.
% This way, it is by design impossible to pass through the same state again. Actually, if a monotone transformation returns bottom, the network has to terminate with a failure.
% However, this is hard restriction to transformations. It leads to the fact that in some networks that actually have a simple solution no solution is found at all. This can be easily seen at the example in \autoref{fig:formal:monotonycounterexample}. In the example adding "aa" to the left model, any execution order of the transformations leads to the situation that a previous change must be revoked to result in a consistent state. However, it is possible to derive a consistent state for that input change.

% \begin{figure}
%     \centering
%     \includegraphics[width=\textwidth]{figures/correctness/orchestration/monotony_counterexample.png}
%     \caption{Counterexample for monotony}
%     \label{fig:formal:monotonycounterexample}
% \end{figure}

% One could now argue that there are binary relations in the example, which may never be fulfilled at all. We will later discuss how far relations that cannot be fulfilled should be restricted. However, in general, this is wanted behavior, because in general it may be necessary that transformations produce intermediate states that are not yet consistent with each other. Otherwise this would means that each transformation is always able to directly deliver a state that is consistent to all other relations, which is especially not possible, because other transformations may add further information to the models. More precisely, a relation may consider <a model consistent to all other models that contain any additional information not affected by the transformation. For example, a UML class model may be considered consistent to all Java models with any implementation of the specified methods, thus to an infinite number of models. Now saying that it should not be allowed that the transformation selects one with an empty implementation because that is not consistent to another relations induced by another transformation, such as the relationship to a component model, does not make any sense. Thus having those relation elements that may be considered locally consistent but will never occur in a globally consistent tuple of models does not make sense.
% In the example, we can see that such an inconsistent intermediate state is passed through and afterwards a consistent tuple of models is reached if not requiring monotony.
% In consequence, requiring monotony from transformations is a too strict requirement, because it is necessary to run through states that may be changed later on.

% \begin{theorem}
%     An application function for monotone transformations either returns a consistent model or produce a sequence of CPRs returning delta that return models of always growing size (i.e. it diverges).
% \end{theorem}

% \paragraph{Divergence cannot be avoided}

% There are rather equal network, one that terminates after a long time and one that never terminates. 
% Consider the example. The relations are defined in a way such that for any allocation for any of them a consistent tuple of models can be found. However, the transformations are not able to find it because they make "bad" choices from a set of choices that are conflicting. 
% This can be seen in the example that we have already given in \autoref{fig:correctness:no_execution_order}.

% Thus, systematically avoiding divergence is not possible. 

% \textbf{Central insight:} Alternation / Divergence cannot be avoided systematically (like in ordinary programming), if not restricting transformations in a way that may not be reasonable.



% \subsection{Unresolvability}

% Discuss why no execution order may exist although relations are compatible.
% If not even an order exists, the application function or the algorithm can, for sure, not find it.

% However, we found that we cannot always find an execution order if it exists and we were not able to find restrictions to transformations to ensure that it exists.
% We expect the same for the existence of an execution order at all.
% All restrictions we can make are likely to be too restrictive.
% The problem arises when there is an overlap of consistent models between some transformations, but they always decide for other elements that are not in the overlap of consistent models.
% It would, obviously, require the transformation to know about the others to ensure that this is not the case.
% This conflicts our assumption.

% Finally, it may be valid that for some changes no execution exists, because the change can not be processed on purpose \todo{Give example for that!}.
% Should this be the case if we assume compatibility?

% Although a more detailed investigation of the claim that we cannot define reasonable requirements to the transformations to ensure that they can always be ordered to restore consistency is a topic for further research, we did not investigate it in the scope of this thesis.
% Since we found it necessary to find a conservative algorithm that can deal with the case that no execution is found anyway, that algorithm covers the case that no execution order exists as well and thus is a solution for this problem as well.

% Beispiel:
% \begin{itemize}
%     \item Das ist im allgemeinen aber nicht Fall. Letztendlich trifft jede Transformation lokale Entscheidungen. Beispielsweise könnte jede einzelne Transformation gegeben eine beliebige Änderung immer dieselben Modelle (bzw. Änderungen die dazu führen) zurück liefern (im trivialsten Fall leere Modelle). Dann erfüllt jede Transformation ihre Korrektheitseigenschaft bzgl. ihrer Relation, aber das Netzwerk muss nicht korrekt sein, da bspw. T(A,B) und T(B,C) sich immer für verschiedene Instanzen von B entscheiden. Es gäbe somit nie eine konsistente Lösung für eine beliebige Ausführungsreihenfolge der Transformationen, auch wenn die Relationen das erlauben würden.
%     \item Beispiel mit Namen, wo eine Transformation immer den großen Namen zurück liefert, die andere immer den kleinen. T(A,B) bildet A auf gleiches B ab und beide auf kleine Schreibweise, obwohl beide erlaubt sind. Erzeuge A="a", dadurch B="a". T(B,C) bildet B auf C ab und beide auf große Schreibweise, obwohl beide erlaubt sind. Somit macht sie das zu B="A" und C="A". Nun wird T(A,B) wieder beide klein machen usw. Allerdings wäre eine insgesamt valide Lösung einfach alle groß oder alle klein zu machen, aber die Transformationen finden diesen Zustand nicht. 
% \end{itemize}




\subsection{A Conservative Application Algorithm}

Propose Provenance and Reactions Strategy.

\section{A Conservative Application Algorithm}
\label{chap:orchestration:algorithm}

\mnote{Correct algorithm that always terminates}
We have argued why it is inevitable that any algorithm realizing an application function cannot be optimal and thus will not be able to find a consistent orchestration although it exists and, in that case, either return $\bot$ or not even terminate at all.
Apart from minor improvements, such as the avoidance or detection of alternations, to improve the probability to find a consistent orchestration, or general strategies like backtracking for trying different orchestrations, we did not find systematic ways to improve optimality of the application function.
Nevertheless, we want to find an algorithm that is at least correct and does always terminate, even if it does implement a systematic way to improve optimality, thus which is conservative.

\mnote{Artificial execution bound can prevent from finding consistent orchestration}
\autoref{algo:orchestration:application} may not terminate, because it generates an infinitely long orchestrations, thus never leaving the loop in Lines~\ref{algo:orchestration:application:line:startorchestrate}--\ref{algo:orchestration:application:line:endorchestrate}.
Thus, to ensure termination we need to introduce an upper bound for the number of executed transformations.
We identified in \autoref{chap:orchestration:decidability:correctness_termination} that there is no upper bound for the number of necessary executions, thus even the shortest consistent orchestration for specific inputs can be arbitrarily long (see example in \autoref{fig:orchestration:no_upper_bound}).
Any arbitrary bound can prevent the algorithm from finding consistent orchestrations.

\mnote{High numbers of transformation executed are not wanted in practice}
From an engineers perspective, we may, however, consider the behavior that an arbitrary high number of transformations execution is required to yield consistent models as unwanted behavior.
Although the examples we have given are valid, they are rather artificial.
We claim that a transformation network that requires a rather high number of executions compared to the number of contained transformations to find consistent models does not operate as expected.
In particular, if such a high number of executions is required to find a consistent orchestration, it will be difficult to identify the reason for not finding a consistent execution in case the algorithm returns $\bot$.
Thus, we introduce an artificial upper bound for the number of transformation executions.
That bound will be well-defined, such that we can reasonably assume that no more execution are practically necessary.

\mnote{Design goals of orchestration, algorithm and termination}
In the following, we propose design goals for a conservative application algorithm and the so called \emph{provenance algorithm} itself and finally prove its correctness and termination properties.
The work was developed together with a student in a kind of scientific internship and published in \cite{gleitze2020orchestration}.

% FROM MODELS:
% Finally, one could question whether it is relevant if an executionstrategy can be guaranteed to terminate. Execution strategies will beused to tell users whether changes they made can be incorporatedinto the other models automatically. In consequence, users shouldreliably and timely get a response. We might compare this situationto merging changes in version control systems. There, users alsowant a reliable and timely response on whether their changes couldbe incorporated automatically, or whether they need to resolveconflicts manually

% From an engineering perspective, unbounded numbers of executions is still unwanted behavior. We claim that a transformation network that takes thousands of executions of the same transformation to find a consistent state works not as expected and if running into a failure would expose severe problems to find the reasons for that failures.
% Thus, we propose to simply abort the execution after some time to be sure not to run in an endless loop.

% Finally, this problem is comparable to ordinary programming, because there the same situations regarding alternation and divergence can occur that result in non-termination of a program.
% As we all know, it is impossible to systematically avoid that, but just possible to carefully develop the program and apply best practices to avoid such situations.

% \textbf{Conclusion:} Apart from minor discussed improvements, we cannot systematically improve optimality / reduce conservativeness and also not dynamically terminate when divergence is detected. Especially, there is no guarantee wo when also alternation is detected. There can be an arbitrary high number of execution before alternation is detected.
% Thus, we propose an algorithm that terminates deterministically and returns $\bot$ also when it is not able to find a solution in a fixed number of steps, but improves the ability for the transformation developer or user to find out why in specific situation no solution can be found.

% \todo{Question whether having no upper bound is only a theoretical or also a practical problem. The given example is rather artificial, so we may stop in practice without excluding relevant cases}

\subsection{Design Goals}
\label{chap:orchestration:algorithm:goals}

\mnote{Degrees of freedom in algorithm}
An adapted version of \autoref{algo:orchestration:application} that always terminates has two degrees of freedom.
First, the execution order of transformations needs to be determined by defining the function \function{Orchestrate}.
Second, an upper bound for the number of executions of transformations, thus the loop in Lines~\ref{algo:orchestration:application:line:startorchestrate}--\ref{algo:orchestration:application:line:endorchestrate}, needs to be defined.

\mnote{Execution order is important for identfying reasons for failing}
We have already discussed that improving optimality is not an achievable goal when determining the transformation execution order by the \function{Orchestrate} function.
Since we know that the algorithm will always produce false negatives, i.e., it will not find a consistent orchestration although it exists, it is important for a transformation developer or user to be able to identify the reasons for that in practice.
The algorithm can support them in this regard by delivering the final state of the models when the orchestration aborted.
The execution order that was chosen until that state was reached is of central importance of identifying the reasons for failing.
Consider that transformations are executed in an arbitrary order and then only some of the models of the final state are actually consistent.
Apart from investigating the complete sequence of executed transformations, there is clue for the user to find the reasons for the algorithm to fail, thus about \emph{provenance} of the error.

\mnote{Orchstration principle to incrementally restore consistency}
To improve identifying provenance whenever the algorithm fails, we propose the following principle for determining an orchestration:
\begin{quote}
    \enquote{Ensure consistency among the transformations that have already been executed before executing a transformation that has not been executed yet.} \cite{gleitze2020orchestration}
\end{quote}
The principle requires that consistency is ensured iteratively for subsets of the transformations and thus the models.
As long as the models are not consistent to all already executed transformations, the orchestration may only execute those transformations instead of new ones until the models are consistent to all of them.
This ensures that consistency is preserved after each change in an incremental way, iteratively increasing improving the set of models and transformations for which consistency is restored.

\mnote{Benefits of orchestration principle}
This approach improves identifying provenance, because it restricts the potentially causal transformations to consider.
If the algorithm fails after executing a subset of the transformations $\transformationset{T}_{exec} \subseteq \transformationset{T}$.
Then there is some transformation $\transformation{t} \in \transformationset{T}_{exec}$ that was executed for its first time last.
Thus, the algorithm found an orchestration of $\transformationset{T}_{exec} \setminus \transformation{t}$ such that the models were consistent to all those transformation, but is was not able to execute $\transformation{t}$ and the transformations in $\transformationset{T}_{exec}$ afterwards, such that the models become consistent to all these transformations.
This helps the transformation developer or user to understand and find the reason for failing in different ways.
First, he or she can ignore any transformation in $\transformationset{T} \setminus \transformationset{T}_{exec}$, as the algorithm already failed to preserve consistency according to the other transformations, which can significantly reduce the number of transformations to consider.
Second, the realization of $\transformation{t}$ is somehow conflicting with the other transformations in $\transformationset{T}_{exec}$. This does not necessarily mean that there is something wrong with $\transformation{t}$, but only that also considering that transformation does either induce the situation that no consistent orchestration anymore or that it cannot be found.
Third, having a state of the models that is consistent to $\transformationset{T}_{exec} \setminus \transformation{t}$ can be used as a starting point to either identify the occurring problem or to manually restore consistency of the models.

\mnote{Principle does not improve optimality}
If the algorithm operates according to the introduced principle and is not able to preserve consistency anymore after the orchestration considers an additional transformation $\transformation{t}$, the selected execution order provides the discussed benefits for identifying the reasons for failing.
There may, however, be another orchestration that is able to ensure consistency to $\transformationset{T}_{exec}$. Executing $\transformation{t}$ earlier or also integrating further transformations in $\transformationset{T}$ before ensuring consistency to all transformations in $\transformationset{T}_{exect}$ can, for sure, lead to the situation that the algorithm finds a consistent orchestration.
This can reduce optimality of the realized orchestration function, but we claim the discussed benefits to outweigh that.

\mnote{Transformations should react to each sequence of all other transformations}
We have shown that there is no inherent upper bound for the necessary number of transformation executions.
Rather than specifying a concrete depending, whether it be fixed or depending on the network size, we derive a reasonable artificial bound for the number of execution from a property that we assume reasonable for possible orchestrations of a set of transformations.
The idea of that property is that each transformation should be allowed to react to the execution of each possible sequence of all other transformations.
It should, however, not be necessary that a transformation must be executed again after the other transformations reacted the execution of that transformation.
Thus, if a transformation was executed after applying the other transformations in any possible order, we expect the models to be consistent to that transformation.

\begin{definition}[Reactive Converging Synchronizing Transformations]
    \label{def:reactiveconverging}
    A set of synchronizing transformations $\transformationset{T}$ is \emph{reactive converging} with respect to models $\modeltuple{m}$ and changes $\changetuple{\metamodeltuple{M}}$ if in any orchestration of $\transformationset{T}$ in which the last transformation $\transformation{t} \in \transformationset{T}$ of that sequence was executed after all other duplicate-free orders of transformations in that order, the yielded models are consistent to $\transformation{t}$.
\end{definition}

\mnote{Example for reactive converging transformations}
The property does not require the other transformation were executed in any order consecutively, but only that the orchestration contains any order of those transformations, but potentially with other transformations in between.
As an example, assume a set of transformations $\transformation{t}_{1}, \transformation{t}_{2}, \transformation{t}{3}$, which is reactive converging for some input of models and changes.
After executing them for these models and changes in the order $\sequenced{\transformation{t}_{1}, \transformation{t}_{2}, \transformation{t}{3}, \transformation{t}_{1}, \transformation{t}_{2}, \transformation{t}{3}}$, the models yielded by that orchestration may still be inconsistent to $\transformation{t}_{1}$, because it was not executed after the order of the transformations $\sequenced{\transformation{t}_{3}, \transformation{t}_{2}}$.
After executing $\transformation{t}_{1}$ once more, the orchestration must yield consistent models, because $\transformation{t}_{1}$ was executed after the two order of the other transformations $\sequenced{\transformation{t}_{2}, \transformation{t}_{3}}$ and $\sequenced{\transformation{t}_{3}, \transformation{t}_{2}}$.
Likewise, $\transformation{t}_{2}$ was executed after $\sequenced{\transformation{t}_{1}, \transformation{t}_{3}}$ and $\sequenced{\transformation{t}_{3}, \transformation{t}_{1}}$, and $\transformation{t}_{3}$ was executed after $\sequenced{\transformation{t}_{1}, \transformation{t}_{2}}$ and $\sequenced{\transformation{t}_{2}, \transformation{t}_{1}}$.


\subsection{The Provenance Algorithm}

\begin{algorithm}
	\caption[Provenance application algorithm]{The provenance algorithm. Adapted from \cite{gleitze2020orchestration}.}
	\label{algo:orchestration:provenance}
	
	%\SetDataSty{textsf}

    %\hspace*{\algorithmicindent} \textbf{Input:}\\
    %\hspace*{\algorithmicindent} \textbf{Output:} 
    \begin{algorithmic}[1]
    \Procedure{\function{ProvenanceApply}}{$\transformationset{T}, \modeltuple{m} %= \tupled{\model{m}{1}, \dots, \model{m}{n}}
    , \changetuple{\metamodeltuple{M}}$}
        \State $\mathvariable{isConsistent}$ $\leftarrow$ $\function{CheckConsistency}(\transformationset{T}, \modeltuple{m}$)
        \If{$\neg \mathvariable{isConsistent}$}
            \State \Return{$\bot$} \label{algo:orchestration:provenance:line:bot_input}
        \EndIf
        \State $\changetuple{\metamodeltuple{M},\mathvariable{res}} \gets %\tupled{\change{\metamodel{M}{1}, \mathvariable{res}}, \dots, \change{\metamodel{M}{1}, \mathvariable{res}}} \gets 
        \function{Propagate}(\transformationset{T}, \modeltuple{m}, \changetuple{\metamodeltuple{M}})$
        \If{$\changetuple{\metamodeltuple{M},\mathvariable{res}} = \bot$}
            \State \Return{$\bot$} \label{algo:orchestration:provenance:line:bot_orchestration}
        \EndIf
        \State $\modeltuple{m}_{res} \leftarrow \changetuple{\metamodeltuple{M},\mathvariable{res}}(\modeltuple{m})$
        \State \Return{$\modeltuple{m}_{res}$} \label{algo:orchestration:provenance:line:return_result}
    \EndProcedure
    \vspace{\baselineskip}
    \Procedure{\function{Propagate}}{$\transformationset{T}, \modeltuple{m} %= \tupled{\model{m}{1}, \dots, \model{m}{n}}
    , \changetuple{\metamodeltuple{M}} %= \tupled{\change{\metamodel{M}{1}}, \dots, \change{\metamodel{M}{n}}}
    $}
        \vspace{0.15\baselineskip}
        \State $\provalgexecuted \gets \emptyset$ \label{algo:orchestration:provenance:line:executed_init}
        %\State $\mathvariable{accumulatedChanges} \gets \changetuple{\metamodeltuple{M}}$         
        \vspace{0.3\baselineskip}
        
        \For{$\provalgcandidate \in \transformationset{T} \setminus \provalgexecuted \mid \changetuple{\metamodeltuple{M}}.\mathvariable{affects}(\provalgcandidate)$} \label{algo:orchestration:provenance:line:loop_start}
            %\vspace{0.15\baselineskip}
            \State $\mathvariable{appResult} \gets \generalizationfunction{\metamodeltuple{M}, \transformation{t}_{candidate}}(\modeltuple{m}, \changetuple{\metamodeltuple{M}})$ \label{algo:orchestration:provenance:line:first_execution}
            \If{$\mathvariable{appResult} = \bot$}
                \State \Return{$\bot$} \label{algo:orchestration:provenance:line:bot_application}
            \EndIf
            \State $\tupled{\modeltuple{m}, \changetuple{\metamodeltuple{M},\mathvariable{candidate}}} \gets \mathvariable{appResult}$
            %\State $\transformationset{T}_{\mathvariable{subnetwork}} \gets \transformationset{T}.\mathvariable{edgeInducedSubgraph}(\transformationset{T}_\mathvariable{executed})$
            %\State $\changetuple{\metamodeltuple{M},\mathvariable{propagation}} \gets \function{Propagate}(\transformationset{T}_{\mathvariable{subnetwork}}, \modeltuple{m}, \changetuple{\metamodeltuple{M},\mathvariable{candidate}})$ \label{line:recursive-call}
            \State $\changetuple{\metamodeltuple{M},\mathvariable{propagation}} \gets \function{Propagate}(\provalgexecuted, \modeltuple{m}, \changetuple{\metamodeltuple{M},\mathvariable{candidate}})$ \label{algo:orchestration:provenance:line:recursive_call}
            \If{$\changetuple{\metamodeltuple{M},\mathvariable{propagation}} = \bot$}
                \State \Return{$\bot$} \label{algo:orchestration:provenance:line:bot_recursion}
            \EndIf
            %\State $\changetuple{\metamodeltuple{M},\mathvariable{candidate}} \gets \mathvariable{candidate}.\mathvariable{execute}(\mathvariable{accumulatedChanges})$ \label{line:check-execution}
            \State $\modeltuple{m}_{\mathvariable{propagation}} \gets \changetuple{\metamodeltuple{M},\mathvariable{propagation}}(\modeltuple{m})$ %\tupled{\change{\metamodel{M}{1},\mathvariable{propagation}}(\model{m}{1}), \dots, \change{\metamodel{M}{n},\mathvariable{propagation}}(\model{m}{n})}$
            \State $\mathvariable{isConsistent}$ $\leftarrow$ $\function{CheckConsistency}(\provalgcandidate, \modeltuple{m}_{\mathvariable{propagation}})$
            \If{$\neg \mathvariable{isConsistent}$} \label{algo:orchestration:provenance:line:bot_failcheck}
                \State \Return{$\bot$} \label{algo:orchestration:provenance:line:bot_nonreactiveconverging}
            \EndIf
            %\vspace{0.3\baselineskip}
            \State $\changetuple{\metamodeltuple{M}} \gets \changetuple{\metamodeltuple{M},\mathvariable{propagation}}$
            \State $\provalgexecuted \gets \provalgexecuted \cup \setted{\provalgcandidate}$ \label{algo:orchestration:provenance:line:executed_update}
        \EndFor
        \State \Return{$\changetuple{\metamodeltuple{M}}$}
    \EndProcedure
    \end{algorithmic}
	% \Procedure{\(\Propagate\,\mathsf{(}\network,\changes\mathsf{)}\)}{
	% \vspace{0.15\baselineskip}		
	% \(\executed\gets\emptyset\)\;																		\label{line:executed-init}
	% \(\accumulatedChanges\gets\changes\)\;         
	% \vspace{0.3\baselineskip}
	
	% \ForEach{\(\candidate\in\network\mid\candidate\notin\executed\land\accumulatedChanges.\adjacentTo\text{\,(}\candidate\text{)}\)}{	\label{line:loop-start}
	% \vspace{0.15\baselineskip}
	% 		\(\candidateChanges\gets\candidate.\execute\,\text{(}\accumulatedChanges\text{)}\)\;			\label{line:first-execution}
	% 		\(\subnetwork\gets\network.\edgeInducedSubgraph\,\text{(}\executed\)\()\)\;
	% 		\(\propagationChanges\gets\Propagate\,\text{(}\subnetwork,\candidateChanges\text{)}\)\;			\label{line:recursive-call}
	% 		\(\candidateChanges\gets\candidate.\execute\,\text{(}\accumulatedChanges\text{)}\)\;			\label{line:check-execution}
	% 		\lIf{\(\candidateChanges.\adjacentToAny\,\text{(}\executed\text{)}\)}{\(\fail\,\text{()}\)}		\label{line:fail}				
	% \vspace{0.3\baselineskip}
	% 		\(\accumulatedChanges\cupgets\propagationChanges\cup\candidateChanges\)\;
	% 		\(\executed\cupgets\candidate\)\;																\label{line:executed-update}
	% 	}
	% \Return\(\accumulatedChanges\)\;
	% }
\end{algorithm}

\mnote{Provenance algorithm performs orchestration recursively}
We propose an algorithm that realizes the discussed design goal in \autoref{algo:orchestration:provenance}.
The algorithm is a derivation of the general algorithm implementing an application function depicted in \autoref{algo:orchestration:application}.
It first checks for consistency of the given models as a prerequisite for executing the transformations.
Then the algorithm calls the recursive function \function{Propagate}, which implements the orchestration of transformations and returns a change tuple that is yielded by the determined orchestration, which delivers consistent models if applied to the input models.
While this behavior is equal to the one in \autoref{algo:orchestration:application}, the orchestration itself is implemented differently in a recursive rather than an iterative manner, which implicitly ensures termination.

\mnote{Recursive orchestration function fulfills principle of incremental consistency restoration}
The function \function{Propagate} implementing the orchestration in a recursive manner acts as follows:
It selects one of the transformations as a candidate to execute next.
This selection ensures that a transformation is selected whose models are affected by any already performed change, such that the transformation needs to perform changes.
Models are affected by a change if any of the two changes in $\change{\metamodeltuple{M}}$ for either of the models that are kept consistent by the selected transformation is not the identity function $\identitychange$.
It then applies the transformation using the generalization function $\generalizationfunction{}$.
If the selected transformation is not defined for the given models and changes, the function may return $\bot$ so that the complete algorithm terminates with $\bot$.
Afterwards, it recursively executes the function \function{Propagate} with the subnetwork given by the transformations that have already been executed and are stored in $\provalgexecuted$.
After that recursive execution it is checked whether the models yielded by the resulting changes are still consistent to the candidate transformation.
If this consistency check fails, the transformations do not fulfill the definition of being reactive converging according to \autoref{def:reactiveconverging}, as we will prove later.
If the models are consistent to the transformation, the next candidate is picked.
In effect, the strategy realizes the defined principle in a recursive manner, because after executing a new transformations, the recursive execution ensures consistency to all yet executed transformations.

\begin{figure}
    \centering
    \resizebox{\textwidth}{!}{\usetikzlibrary{arrows.meta}

\newcommand{\modelradius}{0.1em}
\newcommand{\hmodeldistance}{0.98em}
\newcommand{\vmodeldistance}{0.9em}
\newcommand{\hnetworkdistance}{4*\hmodeldistance}
\newcommand{\vnetworkdistance}{4*\vmodeldistance}
\newcommand{\hborder}{0.4*\hmodeldistance}
\newcommand{\vborder}{0.4*\hmodeldistance}
\newcommand{\addnetworktopleft}[3]{
	\node[model,#2] (m#1left) {};
	\node[model, above right=\vmodeldistance and \hmodeldistance of m#1left.center, anchor=center] (m#1top) {};
	\draw[unconsidered, #3] (m#1left) -- (m#1top);
}
\newcommand{\addnetworkmiddle}[2]{
	\node[model, below right=\vmodeldistance and \hmodeldistance of m#1left.center, anchor=center] (m#1bottom) {};
	\draw[unconsidered, #2] (m#1top) -- (m#1bottom);
}
\newcommand{\addnetworktopright}[2]{
	\node[model, right=2*\hmodeldistance of m#1left.center, anchor=center] (m#1right) {};
	\draw[unconsidered, #2] (m#1top) -- (m#1right);
}
\newcommand{\addnetworkbottomleft}[2]{
	\draw[unconsidered, #2] (m#1left) -- (m#1bottom);
}
\newcommand{\addpseudobottom}[1]{
	\node[model, draw=none, fill=none, below right=\vmodeldistance and \hmodeldistance of m#1left.center, anchor=center] (m#1bottom) {};
}
\newcommand{\addpseudoright}[1]{
	\node[model, draw=none, fill=none, right=2*\hmodeldistance of m#1left.center, anchor=center] (m#1right) {};	
}

\newcommand{\drawnetworktopleft}[3]{
	\addnetworktopleft{#1}{#2}{#3}
	% Not necessary because no dependent networks, so save vertical space
	%\addpseudobottom{#1}
	%\addpseudoright{#1}
}
\newcommand{\drawnetworktopleftmiddle}[4]{
	\addnetworktopleft{#1}{#2}{#3}
	\addnetworkmiddle{#1}{#4}
	\addpseudoright{#1}
}
\newcommand{\drawnetworktop}[4]{
	\addnetworktopleft{#1}{#2}{#3}
	\addnetworktopright{#1}{#4}
	\addpseudobottom{#1}
}
\newcommand{\drawnetworktopmiddle}[5]{
	\addnetworktopleft{#1}{#2}{#3}
	\addnetworktopright{#1}{#4}
	\addnetworkmiddle{#1}{#5}
}
% #1: Number
% #2: Position of left model
% #3-#6: Properties of transformations (top left, top right, middle, bottom left)
\newcommand{\drawnetworkcomplete}[6]{
	\addnetworktopleft{#1}{#2}{#3}
	\addnetworktopright{#1}{#4}
	\addnetworkmiddle{#1}{#5}
	\addnetworkbottomleft{#1}{#6}
}
% #1: First network
% #2: Second network
\newcommand{\iteration}[2]{
	\draw[thick,-{>[scale=0.6]}] ($(#1right.east)+(0.5em,0)$) -- node[above,font=\scriptsize] {it} ($(#2left.west)+(-0.5em,0)$);
}
% #1: First network
% #2: Second network
\newcommand{\recursion}[2]{
	\draw[thick,-{>[scale=0.6]}] ($(#1bottom.south)+(0,-0.35em)$) -- node[right,font=\scriptsize] {rec} ($(#2top.north)+(0,0.35em)$);
}

\begin{tikzpicture}[
	model/.style={draw, circle, fill=black, inner sep=\modelradius},
	executed/.style={transformation, solid, very thin, {Triangle[open,angle=45:0.35em]}-{Triangle[open,angle=45:0.35em]}},
	candidate/.style={transformation, solid},
	unconsidered/.style={transformation, densely dotted, -}
]

\node[draw, fill=lightgray!10, above left=1*\vmodeldistance+1*\vborder and \hborder, anchor=north west, minimum width=2*\hmodeldistance+2*\hborder, minimum height=2*\vmodeldistance+2*\vborder] {};

\drawnetworkcomplete{2}{
}{candidate}{}{}{}

\node[draw, fill=lightgray!10, above right=1*\vmodeldistance+1*\vborder and \hnetworkdistance-\hborder of m2left.center, anchor=north west, minimum width=2*\hmodeldistance+2*\hborder, minimum height=\vnetworkdistance+1*\vmodeldistance+2*\vborder] {};

\drawnetworkcomplete{3}{right=\hnetworkdistance of m2left.center, anchor=center}{executed}{candidate}{}{}
\iteration{m2}{m3}
\drawnetworktopleft{3-1}{below=\vnetworkdistance of m3left.center,anchor=center}{candidate}
\recursion{m3}{m3-1}

\node[draw, fill=lightgray!10, above right=1*\vmodeldistance+1*\vborder and \hnetworkdistance-1*\hborder of m3left.center, anchor=north west, minimum width=\hnetworkdistance+2*\hmodeldistance+2*\hborder, minimum height=2*\vnetworkdistance+1*\vmodeldistance+2*\vborder] {};

\drawnetworkcomplete{4}{right=\hnetworkdistance of m3left.center, anchor=center}{executed}{executed}{candidate}{}
\iteration{m3}{m4}
\drawnetworktop{4-1}{below=\vnetworkdistance of m4left.center,anchor=center}{candidate}{}
\recursion{m4}{m4-1}
\node[draw, fill=lightgray!20, above right=1*\vmodeldistance+1*\vborder and \hnetworkdistance-1*\hborder of m4-1left.center, anchor=north west, minimum width=2*\hmodeldistance+2*\hborder, minimum height=\vnetworkdistance+1*\vmodeldistance+2*\vborder] {};
\drawnetworktop{4-2}{right=\hnetworkdistance of m4-1left.center,anchor=center}{executed}{candidate}
\iteration{m4-1}{m4-2}
\drawnetworktopleft{4-2-1}{below=\vnetworkdistance of m4-2left.center,anchor=center}{candidate}
\recursion{[yshift=0.1*\vnetworkdistance]m4-2}{[yshift=0.1*\vnetworkdistance]m4-2-1}

\node[draw, fill=lightgray!10, above right=1*\vmodeldistance+1*\vborder and 2*\hnetworkdistance-1*\hborder of m4left.center, anchor=north west, minimum width=3*\hnetworkdistance+2*\hmodeldistance+2*\hborder, minimum height=3*\vnetworkdistance+1*\vmodeldistance+2*\vborder] {};

\drawnetworkcomplete{5}{right=2*\hnetworkdistance of m4left.center, anchor=center}{executed}{executed}{executed}{candidate}
\iteration{[xshift=1*\hnetworkdistance]m4}{m5}
\drawnetworktopmiddle{5-1}{below=\vnetworkdistance of m5left.center,anchor=center}{candidate}{}{}
\recursion{m5}{m5-1}
\node[draw, fill=lightgray!20, above right=1*\vmodeldistance+1*\vborder and \hnetworkdistance-1*\hborder of m5-1left.center, anchor=north west, minimum width=2*\hmodeldistance+2*\hborder, minimum height=\vnetworkdistance+1*\vmodeldistance+2*\vborder] {};
\drawnetworktopmiddle{5-2}{right=\hnetworkdistance of m5-1left.center,anchor=center}{executed}{candidate}{}
\iteration{m5-1}{m5-2}
\drawnetworktopleft{5-2-1}{below=\vnetworkdistance of m5-2left.center,anchor=center}{candidate}
\recursion{m5-2}{m5-2-1}
\node[draw, fill=lightgray!20, above right=1*\vmodeldistance+1*\vborder and \hnetworkdistance-1*\hborder of m5-2left.center, anchor=north west, minimum width=\hnetworkdistance+2*\hmodeldistance+2*\hborder, minimum height=2*\vnetworkdistance+1*\vmodeldistance+2*\vborder] {};
\drawnetworktopmiddle{5-3}{right=\hnetworkdistance of m5-2left.center,anchor=center}{executed}{executed}{candidate}
\iteration{m5-2}{m5-3}
\drawnetworktopleftmiddle{5-3-1}{below=\vnetworkdistance of m5-3left.center,anchor=center}{candidate}{}
\recursion{m5-3}{m5-3-1}
\node[draw, fill=lightgray!30, above right=1*\vmodeldistance+1*\vborder and \hnetworkdistance-1*\hborder of m5-3-1left.center, anchor=north west, minimum width=2*\hmodeldistance+2*\hborder, minimum height=1*\vnetworkdistance+1*\vmodeldistance+2*\vborder] {};
\drawnetworktopleftmiddle{5-3-2}{right=\hnetworkdistance of m5-3-1left.center,anchor=center}{executed}{candidate}
\iteration{m5-3-1}{m5-3-2}
\drawnetworktopleft{5-3-2-1}{below=\vnetworkdistance of m5-3-2left.center,anchor=center}{candidate}
\recursion{m5-3-2}{m5-3-2-1}

% Legend
\newcommand{\vdistancelegend}{1*\vmodeldistance}
\coordinate (legend_left_column) at ([yshift=-1.6*\vnetworkdistance,xshift=0.2em]m2left.center);

\node[legendbg, minimum height=6.9*\vdistancelegend, minimum width=7.2em, above right=1.1*\vdistancelegend and -0.75em of legend_left_column, anchor=north west] {};

\draw[unconsidered] (legend_left_column) -- ([xshift=1*\hmodeldistance]legend_left_column);
\node[legend, right=1.3*\hmodeldistance of legend_left_column,anchor=west] {$\transformationset{T}$};
\draw[candidate] ([yshift=-\vdistancelegend]legend_left_column) -- ([xshift=1*\hmodeldistance,yshift=-\vdistancelegend]legend_left_column);
\node[legend, below right=\vdistancelegend and 1.3*\hmodeldistance of legend_left_column,anchor=west] {$\provalgcandidate$};
\draw[executed] ([yshift=-2*\vdistancelegend]legend_left_column) -- ([xshift=1*\hmodeldistance,yshift=-2*\vdistancelegend]legend_left_column);
\node[legend, below right=2*\vdistancelegend and 1.3*\hmodeldistance of legend_left_column,anchor=west] {$\provalgexecuted$};

\coordinate (legend_right_column) at ([yshift=-3.8*\vdistancelegend]legend_left_column);

\coordinate (legenditerationright) at ([xshift=-0.3*\hmodeldistance+\modelradius]legend_right_column);
\coordinate (legenditerationleft) at ([xshift=-0.3*\hmodeldistance-2*\hmodeldistance-\modelradius+\hnetworkdistance]legend_right_column);
\iteration{legenditeration}{legenditeration}
\node[legend, right=1.3*\hmodeldistance of legend_right_column, anchor=west] (legenditerationtext) {iteration step};

\coordinate (legendrecursionbottom) at ([yshift=-\vdistancelegend+0.5*\vnetworkdistance-\vmodeldistance-\modelradius]legend_right_column);
\coordinate (legendrecursiontop) at ([yshift=-\vdistancelegend-0.5*\vnetworkdistance+\vmodeldistance+\modelradius]legend_right_column);
\recursion{legendrecursion}{legendrecursion}
\node[legend, below=\vdistancelegend of legenditerationtext.west, anchor=west] {recursion step};

\end{tikzpicture}%
}
    \caption[Exemplary execution of the provenance algorithm]{%
    Exemplary execution of the provenance algorithm for a change in the topmost model. 
    The transformations present to the current execution of \function{Propagate}, as well as the executed and candidate transformations $\provalgexecuted$ and $\provalgcandidate$ are depicted for each iteration (horizontal) and recursion step (vertical).
}
    \label{fig:orchestration:provenance_example}
\end{figure}

\mnote{Exemplary execution of the provenance algorithm}
\autoref{fig:orchestration:provenance_example} depicts an exemplary execution of the \function{ProvenanceApply} algorithm for a set of four transformations between four metamodels.
We assume that the algorithm receives four initially consistent models and a change to the topmost one.
The example shows that in each recursion step only the subnetwork of the already executed transformations in $\provalgexecuted$ is considered.
Thus, the set of transformations gets smaller in each recursive call of \function{ProvenanceApply}.


\subsection{Correctness, Termination and Goal Fulfillment}

\mnote{Fulfillment of properties}
The provenance algorithm was intended to implement a correct application function and to always terminate.
Additionally, it is supposed to deliver consistent models whenever the given transformation fulfill \autoref{def:reactiveconverging} for being reactive converging.
In the following, we prove that the algorithm actually fulfills these properties.

\mnote{Provenance algorithm terminates and implements correct application function}
First, it is easy to see that the algorithm does always terminate and always either returns consistent models yielded by an orchestration of the given transformations or $\bot$, which realizes a correct application function according to \autoref{def:applicationfunction} and \autoref{def:applicationfunctioncorrectness}.

\begin{theorem}[Provenance Algorithm Termination]
    \autoref{algo:orchestration:provenance} terminates for every possible input.
\end{theorem}
\begin{proof}
    The algorithm terminates if \function{CheckConsistency} and \function{Propagate} terminate.
    We assume termination for the external function \function{CheckConsistency}, because it only validates consistency of the given models.
    Since \function{CheckConsistency} and $\generalizationfunction{}$ terminate, \function{Propagate} may only be non-terminating because of the loop in Line~\ref{algo:orchestration:provenance:line:loop_start} and the recursive call in Line~\ref{algo:orchestration:provenance:line:recursive_call}.
    The number of loop executions is limited by the number of given transformations, i.e., $\abs{\transformationset{T}}$, as each iteration selects another transformation and adds it to $\provalgexecuted$, thus after selecting each transformation once, all transformations are in $\provalgexecuted$ and thus the loop condition is not fulfilled.
    The recursive call receives a set of transformations that is at least one element smaller than the set of transformations given to the calling method, because if $\provalgexecuted = \transformationset{T}$ the loop precondition is not fulfilled. If the given set of transformations is empty, the loop is not entered and thus no recursive call is performed. In consequence, the recursion is never higher than $\abs{\transformationset{T}}$.
\end{proof}

\begin{theorem}[Provenance Algorithm Correctness] \label{theorem:provenance_correctness}
    \autoref{algo:orchestration:provenance} realizes a correct application function according to \autoref{def:applicationfunctioncorrectness}.
\end{theorem}
\begin{proof}
    The algorithm receives models and changes to them and returns models being instances of the same metamodels, thus it fulfills the signature of an application function.
    Additionally, if it returns models they are the result of a consecutive application of transformations in $\transformationset{T}$, as \function{Propagate} calculates the changes, which are applied to the input models to calculate the result, by a repeated application of the generalization function $\generalizationfunction{}$ to transformations in $\transformationset{T}$.
    Thus, \function{Propagate} implicitly implements an orchestration function according to \autoref{def:orchestrationfunction} and applies the transformations in the determined order to calculate the result delivered by \function{ProvenanceApply}.
    Thus \function{ProvenanceApply} fulfills \autoref{def:applicationfunction} for an application function.

    Let us assume that \autoref{algo:orchestration:provenance} does not realize a correct application function.
    \function{ProvenanceApply} may return $\bot$ in Line~\ref{algo:orchestration:provenance:line:bot_input} or Line~\ref{algo:orchestration:provenance:line:bot_orchestration} or models in Line~\ref{algo:orchestration:provenance:line:return_result}.
    Correctness requires the function to either return $\bot$ or consistent models, which may only be violated by \function{ProvenanceApply} by returning models that are not consistent.
    This means that for some input models and changes, \function{ProvenanceApply} returns models $\modeltuple{m}_\mathvariable{res}$ there is a transformation $\transformation{t} \in \transformationset{T}$, such that $\modeltuple{m}_\mathvariable{res}$, or more specifically two models $\model{m}{i}$ and $\model{m}{k}$, 
    whose metamodels are related by $\transformation{t}$, is not consistent to $\transformation{t}$.
    We distinguish three cases:
    \begin{properenumerate}
        \item $\transformation{t}$ was never executed by \function{Propagate}. This means that the changes $\change{\metamodel{M}{i}}$ and $\change{\metamodel{M}{k}}$ in $\change{\metamodeltuple{M}}$ of the two models that are kept consistent by $\transformation{t}$ were always empty, i.e., $\identitychange$, because otherwise $\transformation{t}$ would have been selected in the loop header. Since the initial models $\model{m}{i}$ and $\model{m}{k}$ were consistent to $\transformation{t}$, the returned models are still consistent, because only the identity function is applied to them.
        \item $\transformation{t}$ was executed and no other transformation that involves $\model{m}{i}$ or $\model{m}{k}$ was executed afterwards. Then the returned models are consistent by definition of correctness for $\transformation{t}$.
        \item $\transformation{t}$ was executed and another transformation $\transformation{t}' \in \transformationset{T}$ that involves $\model{m}{i}$ or $\model{m}{k}$ was executed afterwards.
        Since $\transformation{t}'$ was executed after $\transformation{t}$, $\transformation{t}$ was in $\provalgexecuted$ when $\transformation{t}'$ was the candidate transformation $\provalgcandidate$.
        Thus, $\transformation{t}$ is executed in the recursion after the first execution of $\transformation{t}'$, thus the result is consistent to $\transformation{t}$ and because of the check in Line~\ref{algo:orchestration:provenance:line:bot_failcheck} after returning from the recursion also to $\transformation{t}'$. Thus, the returned models are consistent to both $\transformation{t}$ and $\transformation{t}'$.
    \end{properenumerate}
    The third case can be applied inductively if a transformation is followed by multiple transformations that involve the same models. Thus, all cases lead to a contradiction.
\end{proof}

\mnote{Complexity of provenance algorithm}
In addition to these essential properties, we can also derive the upper bound for the number of transformation executions by the algorithm.

\begin{theorem}[Provenance Algorithm Complexity]
    \autoref{algo:orchestration:provenance} executes transformation at most $\mathcal{O}(2^\transformationset{T})$ times.
\end{theorem}
\begin{proof}
	Let $T(m)$ denote the number of transformation executions the algorithm invokes for a set of transformations $\transformationset{T}$ with $m = \abs{\transformationset{T}}$.
	The set $\provalgexecuted$ is initialized to be empty (Line~\ref{algo:orchestration:provenance:line:executed_init}) and grows by one transformation every iteration of the loop (Line~\ref{algo:orchestration:provenance:line:executed_update}).
    It follows that the recursive call in Line~\ref{algo:orchestration:provenance:line:recursive_call} receives a set of transformations that contains one more transformation in each iteration.
    Thus, given $m$ transformations, \function{Propagate} executes each of them in the loop and then makes recursive calls for $0$ to $m-1$ transformations:
	\begin{align*}
	T(m)&	=m+\sum_{i=0}^{m-1}T(i)&	T(0)&	=0\\
	&	=2+2\,T(m-1)\\
	&	=2*(2^{m}-1) \in \mathcal{O}(2^m) & & \qedhere
	\end{align*}
\end{proof}

\mnote{Provenance algorithm fulfills design principle}
Finally, the algorithm was constructed in order to implement the principle defined in \autoref{chap:orchestration:algorithm:goals} to ensure consistency among the transformations that have already been executed before executing a transformation that has not been executed yet.

\begin{theorem}[Provenance Algorithm Design Principle]
    \autoref{algo:orchestration:provenance} ensures consistency among the transformations that have already been executed before executing a transformation that has not been executed yet.
\end{theorem}
\begin{proof}
	After the recursive call in Line~\ref{algo:orchestration:provenance:line:recursive_call}, the current model tuple $\modeltuple{m}_{\mathvariable{propagation}}$ is consistent with all executed transformations in $\provalgexecuted$ (see \autoref{theorem:provenance_correctness}). % and
%	\Cref{thm:strategy-correct} applies for every recursive execution of the strategy. 
	%no changes to models involved  to an executed synx are allowed.% after the recursive call in \cref{line:recursive-call}.
%	Hence,
	% \executed is either empty or 
%	the current model assignment is consistent with all transformations in \executed whenever the algorithm executes a new transformation in \cref{line:first-execution}.
\end{proof}	

\mnote{Provenance algorithm implements optimal application function for reactive converging transformations}
We have given \autoref{def:reactiveconverging} for specifying the property of a transformation set to be reactive converging.
This property defines that we do not want transformations to be required to react to changes they performed themselves if all other transformation have been executed afterwards, as we assume this to be a reasonable property that induces an upper bound for the number of transformation executions.
We have used that property as a design goal for the proposed algorithm and can now show that the algorithm actually always returns consistent models when the transformations fulfill that property, which means that the algorithm implements an optimal application function.

\begin{theorem}[Provenance Algorithm Optimality]
    If the transformation set $\transformationset{T}$ passed to \autoref{algo:orchestration:provenance} is reactive converging according to \autoref{def:reactiveconverging} and if the consistency preservation rules of all transformations $\transformation{t} \in \transformationset{T}$ are total functions, the algorithm implements an optimal application function.
\end{theorem}
\begin{proof}
    We will show that the algorithm does not return $\bot$ when the input models are consistent, thus an orchestration is always found.
    This is even stronger than optimality, because it means that for every input with consistent models a consistent orchestration exists.

    Since optimality allows the algorithm to return $\bot$ when the input models are inconsistent, returning $\bot$ in Line~\ref{algo:orchestration:provenance:line:bot_input} is valid.
    The algorithm returns $\bot$ in Line~\ref{algo:orchestration:provenance:line:bot_orchestration} if \function{Propagate} returns $\bot$, thus we show that \function{Propagate} does not return $\bot$.
    \function{Propagate} returns $\bot$ in Line~\ref{algo:orchestration:provenance:line:bot_application} if the application of a selection transformation in Line~\ref{algo:orchestration:provenance:line:first_execution} returns $\bot$. By assumption, all transformations are total, thus the application can never return $\bot$.
    \function{Propagate} returns $\bot$ in Line~\ref{algo:orchestration:provenance:line:bot_recursion} if a recursive call returns $\bot$. If the loop in that recursive call is executed, the arguments for not returning $\bot$ apply recursively. If the loop is not executed in the recursion than the input changes are returned, which are not $\bot$.
    
    Finally, \function{Propagate} returns $\bot$ in Line~\ref{algo:orchestration:provenance:line:bot_nonreactiveconverging} if the models yielded by the recursive call are not consistent with the transformation that is the candidate $\provalgcandidate$ in that loop iteration.
    Since the transformation set is reactive converging, this can only be the case if not all orders of the transformations currently in $\provalgexecuted$ have been executed yet.
    We show that all transformation in $\provalgexecuted$ have been executed by induction.
    In the first iteration of the loop only the candidate of that iteration is executed and $\provalgexecuted$ is empty, thus the statement is trivially true.
    Let us assume that in a loop iteration with $\abs{\provalgexecuted} = i-1$ all orchestrations of transformations in $\provalgexecuted$ have been executed in Line~\ref{algo:orchestration:provenance:line:bot_nonreactiveconverging}, but that in the following loop iteration with $\abs{\provalgexecuted} = i$ this is not true.
    This means that there is an order $\sequenced{\transformation{t}_{1}, \dots, \transformation{t}_{i}}$ of the transformations in $\provalgexecuted$, in which they have not been executed yet.
    Let $\transformation{t}$ be the candidate $\provalgcandidate$ of the last iteration with $\abs{\provalgexecuted} = i-1$. Let $k$ be the index of $\transformation{t}$ in that sequence, i.e., $\transformation{t} = \transformation{t}_{k}$. Then per induction assumption the sequence $\sequenced{\transformation{t}_{1}, \dots, \transformation{t}_{k}}$ has been executed in one of the previous iterations of the loop. 
    Afterwards $\transformation{t}$ was executed in Line~\ref{algo:orchestration:provenance:line:first_execution}.
    Additionally, the sequence $\sequenced{\transformation{t}_{k+1}, \dots, \transformation{t}_{i}}$ has been executed in the recursive call in Line~\ref{algo:orchestration:provenance:line:recursive_call} by induction assumption.
    Hence, the transformations have been executed in the order $\sequenced{\transformation{t}_{1}, \dots, \transformation{t}_{i}}$, which is a contradiction to our assumption.

    In consequence, \function{Propagate} never returns $\bot$ and thus also \function{ProvenanceApply} does not return $\bot$, except for inconsistent input models. Since we have already proven that the algorithm terminates always and implements a correct application function, it also implements an optimal application function.
\end{proof}

\mnote{Assumptions for optimality are usually not fulfilled}
Optimality can, however, only be guaranteed under rather specific conditions.
Apart from the necessary fulfillment of the property to be reactive converging, the transformations need to be able to handle any input, thus any combination of models and changes, as otherwise selecting a transformation may lead to \function{Propagate} returning $\bot$, because the transformation cannot be applied.
In practice, this assumption will usually not be fulfilled.
Nevertheless, it is theoretically possible to define such transformations and at least it leads to well-defined conditions for when we can assume the algorithm to realize an optimal orchestration function.

\mnote{Orchestration problem is non-existent for given assumptions}
Although this means that under that specific conditions the algorithm is able to decide the orchestration problem, the problem is actually trivially solved in that case, because for every input there is a consistent orchestration, thus the problems is actually non-existent for these assumptions.

\mnote{Assumptions allow algorithm definition for well-defined requirements}
Finally, it is also an open question how far we can assume sets of transformations to be reactive converging in practice.
We did, however, not introduce this as a property that should be fulfilled by transformations, as it is obviously hard to ensure or even analyze that property.
In fact, it is only supposed to be a well-defined property that allows us to define a reasonable upper bound for the execution of transformations and thus to allow us to define an algorithm that always terminates without using a completely arbitrary upper bound for determining when to terminate.


\subsection{Provenance Identification Improvement}

\mnote{Algorithm does not yield information about failure state}
We have motivated the provenance algorithm with the idea to improve the ability of a transformation developer or user to find the reason for the algorithm not to yield consistent models for certain inputs.
The proposed \autoref{algo:orchestration:application} does only return $\bot$ in those situations and thus does not directly support that process.
The necessary information for improving the identification of provenance for the failure is, however, present in the algorithm and can be easily retrieved.

\mnote{Reasons for algorithm to fail}
The algorithm may fail because it is, at some point, not able to execute a candidate transformation (Line~\ref{algo:orchestration:provenance:line:bot_application}), or because after executing a new transformation consistency to the previously executed transformations cannot be achieved without letting one of the transformations react to the reaction of all other transformations to its own changes (Line~\ref{algo:orchestration:provenance:line:bot_failcheck}), which we defined as the property of reactive convergence.
In that case, we at least know that the after the previous loop iteration consistency regarding all yet executed transformations could be achieved.

\mnote{Relevant information about the failure state}
Whenever the \function{Propagate} function fails and returns $\bot$, we know that for the current transformations in $\provalgexecuted$ an orchestration exists that yields the current changes in $\changetuple{\metamodeltuple{M}}$ %= \tupled{\change{\metamodel{M}{1}}, \dots, \change{\metamodel{M}{n}}}$
, for which we know that applied to the original models the result $\changetuple{\metamodeltuple{M}}(\modeltuple{m})$%\tupled{\change{\metamodel{M}{1}}(\model{m}{1}), \dots, \change{\metamodel{M}{n}}(\model{m}{n})}$
is at least consistent to $\provalgexecuted$.
We also know that the algorithm was not able to ensure consistency to the current candidate transformation $\provalgcandidate$.
This is exactly the information for which we already discussed in \autoref{chap:orchestration:algorithm:goals} the benefits with respect to the underlying design principle of recursively ensuring consistency for subsets of the transformations for the ability to identify the reasons for not finding a consistent orchestration.
Thus, implementing the algorithm such that also delivers $\provalgcandidate$, $\provalgexecuted$ and the current changes $\changetuple{\metamodeltuple{M}}$ reduces the necessary model states and transformations to consider for a transformation user or developer to identify why no consistent orchestration was found.

%Unfortunately, the algorithm does help in finding whether no orchestration exists at all or whether only the algorithm was not able to find it.
%Maybe map that to unapplicability of transformations to non-existence vs. fail after recursive application to non-finding

\mnote{Possible improvement of orchestration to improve locality}
The algorithm and the ability to identify reasons for the algorithm to fail may be further improved by determining a reasonable order for the execution of transformations in the loop of the \function{Propagate} function.
The loop at least ensures that no transformations are executed that are not yet affected by any change and thus would not produce changes.
It can, however, also be reasonable to first select transformations for which both models have already modified before selecting transformations for which only one model has been modified.
This can further improve locality of the changes made until the algorithm fails, because less models may have been modified until the algorithm fails.

%\todo{Discuss that is may make sense to first use transformations between already involved models rather than passing through the complete network.}



\section{Summary}

In this chapter, we have discussed how we can realize an application function for transformation networks.
We have motivated optimality as a desired property of such a function, which ensures that an application function always delivers consistent models if there is an order of the transformations that leads to those models. 
From that optimality notion, we have derived the central orchestration problem, for which we haven proven undecidability, even when restricting transformation networks.
Finally, we have proposed strategies to reduce the cases in which no consistent models are found and an algorithm that has a well-defined order and bound for the transformation execution and, rather than improving optimality, ensures that in cases when no consistent models can be derived at least some information can be provided that helps developers or user of transformations to identify why no consistent models were found.
We conclude this section with the following central insight.

\begin{insight}[Orchestration]
    We have proven that whether an orchestration of modular and independently developed transformations exists that restores consistency for given models and changes, denoted as the \emph{orchestration problem}, is undecidable.
    We have shown that even impractical restrictions to the individual transformations do not make the problem decidable, such that we need to accept that the problem is undecidable.
    In consequence, every algorithm that realizes an application function for transformations can only implement a conservative approximation of the orchestration problem.
    Due to this conservativeness, every algorithm will fail in cases in which actually an orchestration of the transformations exists that leads to consistent models.
    Thus, it is useful to find an algorithm that orchestrates the transformations in a way such that the state of executed transformations and the up to now delivered changes can help the transformation developer or user to identify why the algorithm failed.
    We found that this can be achieved with a strategy of iteratively restoring consistency for subsets of the transformations, such that always a subset of the transformations for which consistency could be restored and a transformation for which it could not be restored anymore can be given to ease reasoning about the cause for failing.
    We have proposed an algorithm that implements that strategy and is proven to fulfill the desired property.
\end{insight}

% \section{Old Notes}

%\section{Restrictions vs. Conservativeness}

% \todo{First restriction: Input delta of APP only contains changes to one model -> no synchronisation}
% \todo{Second restriction: Input delta is not rejected}
% \todo{Third restriction: Generated deltas are monotone}

% From a theoretical perspective, it is always possible to a specify consistency relations according to the definition, as it is just a subset of elements.
% It is also always possible to define a consistency preservation rule for a consistency relation according to its definition, as can simply return any any element of the relation.
% \todo{This is not true: the source model may not be in the relation, then its not possible, at least with the current definition. With a synchronizing transformation, any modification can be made to both models, then its fine.}
% The generalization function is generic, so it can always be applied.
% Finally, the consistency preservation application function is an artifact that cannot be easily specified according to the definition for a given set of consistency preservation rules.
% It is always possible to have a set of consistency preservation rules for which no application function can be defined that returns a consistent result for at least one input model and change tuple, as there is not sequence of consistency preservation rules that achieves that.
% \todo{Example!}
% Even worse, the problem to define such a function is Turing-complete, which makes it impossible to decide whether such a function exists.
% \todo{Show Turing-completeness}
% Consequence: From a theoretical perspective, this function is the crucial part!

% Essential problem: One transformation may restore consistency between A and B and another between A and C. If then a transformation restores consistency between B and C, the resulting B' and C' may not be consistent A anymore.

% Alternative to an app function: Define a \emph{well-definedness} property for a set of transformations, requiring that they can be executed in any order to always terminate consistently. However, this is a very strict requirement, which can usually not be fulfilled, so we do not further investigate that.
% \todo{Give simple example why that does not work.}


% Best-behaved app function: Whenever there is a sequence of CPRs, the app function finds item. This is still not possible due to Turing-completeness. The function would need to decide whether the network terminates or not.

% Only achievable app function is a best-effort (i.e., conservative) function: A function that either returns a consistent set of models or that does return bottom. Not making a statement about how often a correct result is returned in comparison to how often it is possible.

% This approach is conservative. The question is then, how high the degree of conservativeness is. In the worst case, a function that always returns bottom would fulfill the definition, but that is not what we want. We want to reduce conservativeness.

% Goal: Find a solution in as many cases as possible, abort in the others (conservatively). There are two subgoals to achieve that:
% 1. Function must be correct, i.e., always terminate (no endless sequence of CPR) and terminate in a consistent state
% 2. Function must be as less conservative as possible

% It is clear that we cannot give a closed function for APP that just by a given change returns a sequence of CPR that results in a consistent state. APP has to be calculated dynamically during execution. Therefore we consider it as an algorithm in the following.

% Annahmen:
% \begin{itemize}
%     \item Nutzeränderungen dürfen nicht rückgängig gemacht werden.
%     \item Nutzeränderungen lassen sich so feingranular zerlegen, dass, falls durch die Erzeugung/Änderung eine Konsistenzrelation verletzt wird, es in jeder unabhängigen Teilmenge von Konsistenzrelationen eine verletzte Konsistenzrelation gibt, für die die geänderten Elemente einem Condition Elemente entsprechen, es also insbesondere keine Teilmenge der geänderten Element gibt, die bereits dieses Condition Element sind. Ansonsten ist durch unsere Kompatibilitäts-Definition nicht sichergestellt, dass eine konsistente Modellmenge gefunden werden kann.
% \end{itemize}

% \section{Considerations}

% Zu diskutierende Dinge:
% \begin{itemize}
%     \item Im Allgemeinen könnte eine Transformation beliebige dieser Deltasequenzen modifizieren. Wir verlangen jedoch, dass eine Transformation nur Deltas anhängt, also die Sequenzen länger werden
%     \item Genauer beschränken wir auch, welche Sequenzen eine Transformation sehen und ändern darf, genau gesagt darf sie die Sequenzen von zwei Modellen sehen und eine davon verlängern.
%     \item Hier kommt bereits der Unterschied zu bisherigen Transformationen, denn die sehen nur Deltas an einem Modell und erzeugen Deltas an dem anderen. Das ist bei uns schon gänzlich anders. Bidirektionale Transformationen unterstützen das im Übrigen auch nicht, sondern sind nur Spezifikationen, aus denen sich Wiederherstellungsroutinen für beide Richtungen ableiten lassen (siehe Stevens 2010)
% \end{itemize}


% Trivialisierung des Problems:
% \begin{itemize}
%     \item Ohne weitere Annahmen ist das immer dadurch erreichbar, dass die Transformationen einen beliebigen anderen Zustand der Modelle produzieren. Im einfachsten Fall liefert jede Transformation immer die gleichen konsistenten Modelle zurück, unabhängig von der Änderung. Dann ist der Endzustand der Modelle nach der Ausführung des Netzwerks immer der gleiche.
%     \item Das ist im allgemeinen aber nicht Fall. Letztendlich trifft jede Transformation lokale Entscheidungen. Beispielsweise könnte jede einzelne Transformation gegeben eine beliebige Änderung immer dieselben Modelle (bzw. Änderungen die dazu führen) zurück liefern (im trivialsten Fall leere Modelle). Dann erfüllt jede Transformation ihre Korrektheitseigenschaft bzgl. ihrer Relation, aber das Netzwerk muss nicht korrekt sein, da bspw. T(A,B) und T(B,C) sich immer für verschiedene Instanzen von B entscheiden. Es gäbe somit nie eine konsistente Lösung für eine beliebige Ausführungsreihenfolge der Transformationen, auch wenn die Relationen das erlauben würden.
%     \item Beispiel mit Namen, wo eine Transformation immer den großen Namen zurück liefert, die andere immer den kleinen. T(A,B) bildet A auf gleiches B ab und beide auf kleine Schreibweise, obwohl beide erlaubt sind. Erzeuge A="a", dadurch B="a". T(B,C) bildet B auf C ab und beide auf große Schreibweise, obwohl beide erlaubt sind. Somit macht sie das zu B="A" und C="A". Nun wird T(A,B) wieder beide klein machen usw. Allerdings wäre eine insgesamt valide Lösung einfach alle groß oder alle klein zu machen, aber die Transformationen finden diesen Zustand nicht. 
%     \item Allgemeiner ist zu sagen, dass ein Transformationsnetzwerk eine Turing-Maschine emulieren kann. \todo{Nachweisen!}
%     Im allgemeinen terminiert das Netzwerk somit nicht, schlimmer noch, es ist unentscheidbar, ob das Netzwerk hält (siehe Halteproblem).
%     \item Dies zeigt bereits, dass keine Ausführungsfunktion definiert werden kann, die immer ein konsistentes Ergebnis liefert.
%     \item Wir versuchen daher Annahmen an Transformationen zu finden, um diese Fälle auszuschließen bzw. systematisch zu verringern. 
%     \item Außerdem möchten wir eine Ausführungsfunktion haben, die ein konsistentes Ergebnis liefert oder einen Fehler, denn es muss nicht immer eine korrekte Lösung geben. Ziel ist es dann die Anzahl der Fälle, in denen sie einen Fehler zurückgibt, zu reduzieren.
% \end{itemize}


% Zielsetzung:
% \begin{itemize}
%     \item Korrekte Anwendungsfunktion finden (in bestehenden Arbeiten~\cite{stevens2017a}) auch "Resolution" genannt (formal definieren!):
%     \item Welche Anforderungen müssen wir dafür an die Transformationen stellen, damit solch eine Funktion definiert werden kann?
%     \item Wir bezeichnen das Transformationsnetzwerk, in dem eine Transformation eingesetzt wird, als "Kontext"
%     \item Welche dieser Eigenschaften kann die einzelne Transformation (ohne Kenntnis der anderen) erfüllen und für welche muss der Kontext (d.h. die anderen Transformationen) bekannt sein?
%     \item $\Rightarrow$ Interesse an "kontextfreien" Eigenschaften (lassen sich ohne Kenntnis der anderen Transformationen sicherstellen -> Wiederverwendbarkeit) und "kontextsensitiven" Eigenschaften (Erfüllung der Eigenschaft nur durch Kenntnis über das Transformationsnetzwerk möglich)
%     \item Kontextfreie Eigenschaften involvieren solche, die wir eh schon von Transformationen kennen (Korrektheit einer Transformation, Hippokratie etc.) und solche, die dadurch zustande kommen, dass man weiß, dass diese Transformation in einem Netzwerk eingesetzt werden soll.
%     \item Zielsetzungsoptionen:
%     \begin{itemize}
%         \item Wir schränken die Transformationen so ein, dass es immer mindestens eine Ausführungsreihenfolge der Transformationen gibt, sodass für jede beliebige Änderung ein konsistentes Ergebnis durch Anwenden der Transformationen gefunden werden kann
%         \item Wir akzeptieren, dass es Änderungen gibt, für die das Netzwerk kein konsistentes Ergebnis produzieren kann. Dann muss das Netzwerk (mindestens) in diesen Fällen mit einer Fehlermeldung terminieren.
%         \item Eine Option ist, dass das Netzwerk dieses Verhalten nur approximiert bzw. approximieren kann, dann muss es sich konservativ verhalten, d.h. im Fall, dass es keine Lösung gibt, auf jeden Fall eine Fehlermeldung geben, und im Fall, in dem es eine Lösung gibt, diese bestenfalls finden oder ausgeben, dass es keine finden kann (d.h. keine False Positives bzw. Nicht-Terminierung). Ziel ist es dann den Grad der Konservativität zu minimieren.
%     \end{itemize}
%     \item Lösungsoptionen (Grad der Einschränkung an die Transformationen):
%     \begin{itemize}
%         \item Hohe Einschränkung: Jede beliebige Reihenfolge von ausgeführten Transformationen führt letztendlich zu einem korrekten Ergebnis (Fixpunktiteration -- Allquantifizierung) -- Hippokratie-Eigenschaft sorgt dafür, dass keine Transformation wieder etwas ändert, wenn Konsistenz bereits hergestellt ist.
%         Diese Eigenschaft ist in der Praxis möglicherweise zu strikt, da sie sehr starke Anforderungen an die Transformationen stellen müsste. Dafür wäre aber die Anwendungsfunktion trivial.
%         \item Mittlere Einschränkung: Es gibt eine Reihenfolge von ausgeführten Transformationen für jede Änderung die terminiert (Existenzquantifizierung) und die Ausführungsfunktion findet diese Reihenfolge.
%         Utopisch, dass die Anwendungsfunktion aus (potentiell sehr mächtigen) Transformationen die richtige Reihenfolge errechnen kann. Dafür aber (möglicherweise) weniger Anforderungen an die Transformationen (zumindest nicht mehr Anforderungen, denn die Allquantifizierung induziert die Existenzquantifizierung). Eine Funktion könnte dann zumindest nach best-effort versuchen, die richtige Reihenfolge zu finden und konservativ abbrechen, wenn sie diese nicht finden kann (also entweder konsistent terminieren oder terminieren mit der Aussage, dass es entweder keine solche Reihenfolge gibt -- bei relaxierten Anforderungen -- oder dass es sie nicht finden kann).  
%         \item Geringe Einschränkung: Es gibt potentiell keine Reihenfolge der Transformationen, die bei einer Änderung zu einer konsistenten Lösung kommt. Hier müsste die Ausführungsfunktion entsprechend einen Fehler ausgeben.
%         \item Bestehende Arbeiten (\cite{stevens2017a}) schlagen auch vor eine Baumstruktur zu berechnen (Spannbaum), in dem nur entlang der Baumkanten die Transformationen ausgeführt werden. Dies ist jedoch eine starke Einschränkung daran, was die Transformationen ausdrücken können. Betrachtet man beispielsweise PCM, UML und Java, und hat eine Änderung in PCM. Dann könnte der Spannbaum entweder PCM -> UML -> Java sein, oder PCM -> UML + PCM -> Java. In ersterem Fall würde Verhaltensbeschreibung, die von PCM nach Java übertragen, aber in UML nicht dargestellt wird, nicht übertragen. Im zweiten Fall würde zusätzliche Information zwischen UML und Java nicht propagiert (Beispiel?) --> Hier sollte auf das Properties-Kapitel verwiesen werden, wo diese "Bottlenecks" erklärt sein sollten, inklusive einem Beispiel, die allgemein Baumstrukturen für Transformationsnetwerke ausschließen.
%     \end{itemize}
%     \item Dies setzt voraus, dass die Transformationen und die Anwendungsfunktion mit jeder beliebigen Nutzer-Änderung umgehen kann. Man kann jedoch auch verlangen, dass die Anwendungsfunktion genau dann, wenn es überhaupt eine Ausführungsreihenfolge gibt, diese findet, und sonst einen Fehler ausgibt.
%     \item \textbf{Wichtig:} Im Allgemeinen kann eine Ausführungsfunktion keine terminierende Reihenfolge berechnen, da die Transformationen Turing-vollständig sind und deshalb die Frage, welche Reihenfolge zu einer Terminierung führt, unentscheidbar ist (Halteproblem). Daher können wir nur einen konservativen Algorithmus angeben, der ein sinnvolles Abbruchkriterium definiert, mit dem die Ausführung beendet wird, auch wenn potentiell eine Lösung hätte gefunden werden können. Die Fragestellung ist also, wie die Ausführungsfunktion aussehen muss, damit sie in möglichst vielen Fällen, in denen es eine terminierenden Reihenfolge gibt, diese auch findet. Insbesondere lässt sich somit keine geschlossene Form für die Ausführungsfunktion angeben, sondern nur ein Algorithmus, der zur Laufzeit eine Reihenfolge (dynamisch) festlegt.
% \end{itemize}





% Notwendigkeit Transformationen oder Anwendungsfunktion einzuschränken:
% \begin{itemize}
%     \item Zeigen, dass es Beispiele gibt, in denen es keine einzige Ausführungsreihenfolge gibt (All-Quantifizierung), die zu einem konsistenten Ergebnis führt:
%     \item Zeigen, dass es Beispiele gibt, in denen es unabhängig von der Ausführungsreihenfolge immer zu einer Alternierung kommt
%     \item Zeigen, dass es Beispiele gibt, in denen es unabhängig von der Ausführungsreihenfolge immer zu einer Divergenz kommt.
%     \item Die Beispiele sollten zeigen, dass wir keine Einschränkungen an die Transformationen machen können, was das Problem aushebelt. D.h. egal welche Einschränkungen ich an die Transformationen definiere, es lassen sich immer Beispiele konstruieren, in denen es keine Ausführungsreihenfolge gibt, in denen sie terminieren.
%     \item Mathematisch zeigen, dass Alternierung und Divergenz die einzigen Probleme sind. D.h. wenn nicht der gleiche Zustand mehrmals durchlaufen wird (Alternierung) und es nicht unendlich viele Zustände gibt (Divergenz), dann ist die Folge endlich.
%     \item Außerdem mathematisch die Abbildung von Transformationen auf Turing-Maschinen zeigen und damit ableiten, dass allgemeine Netzwerke erstmal nicht terminieren müssen (Abbildung auf Halteproblem)
% \end{itemize}

% Zielsetzung die Zweite:
% \begin{itemize}
%     \item Wir definieren möglichst minimale Beschränkungen, die dazu führen, dass das Netzwerk terminiert. D.h. es terminiert entweder konsistent oder es terminiert mit einem Fehler, der sagt, dass entweder keine Konsistenz hergestellt werden kann (es gibt keine Ausführungsreihenfolge der Transformationen, die zu Konsistenz führt) oder dass die Anwendungsfunktion nicht in der Lage war eine passende Ausführungsreihenfolge zu finden (Konservativität)
%     \item Zwei Arten von Beschränkungen
%     \begin{itemize}
%         \item Beschränkungen an die Transformationen, die dazu führen, dass es in mehr Fällen mindestens eine Ausführungsreihenfolge gibt, in der das Netzwerk konsistent terminiert
%         \item Beschränkungen an die Ausführungsfunktion, sodass die Ausführung auf jeden Fall terminiert, wenn auch konservativ, d.h. mit Fehler, obwohl es eine korrekte Lösung gegeben hätte.
%     \end{itemize}
% \end{itemize}

% \section{No Execution Order with Consistent Result}

% It is obvious that we can define consistency preservation rules for which the orchestration function cannot find an execution order that returns a consistent tuple of models after certain changes. We already gave an example in \autoref{fig:correctness:no_execution_order}. There exists no execution order for any input value that terminates. The transformations will always increase the value, although the defined relations could be fulfilled for the input value, but the transformations never find that solution.

% Although we will discuss restrictions to relations and transformations that reduce the chance that no solution can be found, it will not be possible to ensure that such a solution can always be found. This is due to the reason that transformations can perform arbitrary changes given that transformations are Turing complete, which should not be restricted, because it is unclear which restrictions could be made without forbidding scenarios that should actually we supported. Thus, we assume that transformations are Turing complete.

% We explicitly allow the orchestration function to return a sequence that will, if applied to models and changes to them, not deliver a consistent tuple of models. As discusses, this is supposed to reflect cases in which no such sequence can be calculated.
% However, it may be useful to have some notion of \emph{optimality} that ensures that if a sequence that delivers a consistent result exists, the orchestration function is supposed to find it.
% Formally, this notion looks as follows.

% \begin{definition}[Optimal Consistency Preservation Orchestration Function]
%     \todo{Needs to be adapted to transformations rather than CPRs}
%     Let $\consistencypreservationruleset{}$ be a set of consistency preservation rules for a set of consistency relations $\consistencyrelationset{CR}$ on metamodels $\metamodeltuple{M} = \tupled{\metamodelsequence{M}{n}}$.
%     We say that an orchestration function $\orcfunction{\consistencypreservationruleset{}}$ for these rules is \emph{optimal} if it always returns a sequence that delivers a consistent set of models if possible, i.e.,
%     \begin{align*}
%         &
%         \forall \modeltuple{m} \in \metamodeltupleinstanceset{M} : \forall \changetuple{\metamodeltuple{M}} = \tupled{\change{\metamodel{M}{1}}, \dots, \change{\metamodel{M}{n}}} \in \changeuniverse{\metamodeltuple{M}} :
%         \modeltuple{m} \consistenttomath \consistencyrelationset{CR} \Rightarrow \\
%         & \formulaskip
%         \bigl[ \bigl(
%             \exists \consistencypreservationrule{1}, \dots, \consistencypreservationrule{m} \in \consistencypreservationruleset{} : 
%             \exists \changetuple{\metamodeltuple{M}}' = \tupled{\change{\metamodel{M}{1}}', \dots, \change{\metamodel{M}{n}}'} \in \changeuniverse{\metamodeltuple{M}} :\\
%             & \formulaskip \formulaskip
%             \generalizationfunction{\consistencypreservationrule{1}} \concatfunction \dots \concatfunction \generalizationfunction{\consistencypreservationrule{m}}(\modeltuple{m}, \changetuple{\metamodeltuple{M}}) = (\modeltuple{m}, \changetuple{\metamodeltuple{M}}')\\
%             & \formulaskip \formulaskip
%             \land \tupled{\change{\metamodel{M}{1}}'(\model{m}{1}), \dots, \change{\metamodel{M}{n}}'(\model{m}{n})} \consistenttomath \consistencyrelationset{CR} \bigr) \\
%             & \formulaskip
%             \Rightarrow \bigl(        
%             \exists \consistencypreservationrule{1}', \dots, \consistencypreservationrule{m}' \in \consistencypreservationruleset{} : 
%             \exists \changetuple{\metamodeltuple{M}}'' = \tupled{\change{\metamodel{M}{1}}'', \dots, \change{\metamodel{M}{n}}''} \in \changeuniverse{\metamodeltuple{M}} :\\
%             & \formulaskip \formulaskip
%             \orcfunction{\consistencypreservationruleset{}}(\modeltuple{m}, \changetuple{\metamodeltuple{M}}) = \tupled{\consistencypreservationrule{1}', \dots, \consistencypreservationrule{m}'} \\
%             & \formulaskip \formulaskip
%             \land \generalizationfunction{\consistencypreservationrule{1}'} \concatfunction \dots \concatfunction \generalizationfunction{\consistencypreservationrule{m}'}(\modeltuple{m}, \changetuple{\metamodeltuple{M}}) = (\modeltuple{m}, \changetuple{\metamodeltuple{M}}'')\\
%             & \formulaskip \formulaskip
%             \land \tupled{\change{\metamodel{M}{1}}''(\model{m}{1}), \dots, \change{\metamodel{M}{n}}''(\model{m}{n})} \consistenttomath \consistencyrelationset{CR}
%         \bigr) \bigr]
%     \end{align*}
% \end{definition}

% Unfortunately, optimality is a property that we cannot request from an orchestration function. Optimality would mean that the orchestration function can decide whether there is sequence of transformations that leads to consistent models and thus terminate.
% Due to Turing-completeness of the network this would mean that the orchestration function can decide whether a Turing machine halts, which is proven impossible.
% Thus, our only goal can be to achieve optimality as far as possible in terms of reducing the degree of conservativeness, i.e., reduce the cases in which no sequence is found although it exists.

% We can define a measure for the optimality of an orchestration function:
% \begin{align*}
%     &
%     Optimality_{\orcfunction{\consistencypreservationruleset{}}} = \frac{\mathtext{\# of model / delta pairs for which the function finds an order that terminate consistently}}{\mathtext{\# of model / delta pairs for which an order that terminates consistently exists}}
% \end{align*}

% In fact, both these numbers usually infinite, an there is an infinite number of possible models and deltas. However, it does finally not matter for us what the actually value is, but only how to improve that value.
% \todo{We have to map that value to compatibility, which reduces the number of potential false orders.}



% \subsection{Achieving an Always Correct Application Function}
% % This discusses the case that the application function is optimal
% \todo{These problem cannot occur if a function fulfills the definition, because it always finds a sequence. So the question is how to fulfill the definition.}

% It is easy to achieve that the APP function only terminates in a consistent state, because knowing the relations allows to check whether all relations are fulfilled. 
% \todo{Need to define that a transformation may not be able to process a specific change? Then there could be inconsistent terminiation because a transformation cannot be executed anymore.}

% Problems due to which the function does not terminate: Alternation and Divergence

% Alternation: Run through same state twice
% Divergence: Always produce new states without reaching a consistent state

% Two possibilities to avoid problems:
% 1. Make assumptions to transformations that avoid them
% 2. Detect them dynamically and abort


% \subsubsection{Avoiding Alternation / Divergence}

% Making assumptions that avoid them is rather hard, as we will show in the following.

% \paragraph{Idea:} Require monotony to avoid alternation

% We would have to relax the definition of transformation to be monotone, because if a transformation is monotone, it may only append information, but this is not always possible, as can be seen in the following example. A monotone transformation must be able to return bottom if it cannot make further changes to restore consistency to the relation.

% \begin{definition}[Monotone Transformation]
%     Transformation gets models M and deltas D and produces new deltas D'. Taking the union of the original models M and the new models D'(M), then D(M) must be a subset of that, because other elements would have been added and removed afterwards or elements would have been changes once by D and again in a different way by D'.

%     Generally, monotony could also mean that only the same complete model state is not passed twice. \todo{Why dont we do that?}
% \end{definition}

% This would mean that each transformation only appends changes, i.e., if an element was added/removed, the transformation may not do the inverse. The same applies to attribute/reference changes: if an attribute/reference was already changes it may not be changed again.
% This way, it is by design impossible to pass through the same state again. Actually, if a monotone transformation returns bottom, the network has to terminate with a failure.
% However, this is hard restriction to transformations. It leads to the fact that in some networks that actually have a simple solution no solution is found at all. This can be easily seen at the example in \autoref{fig:formal:monotonycounterexample}. In the example adding "aa" to the left model, any execution order of the transformations leads to the situation that a previous change must be revoked to result in a consistent state. However, it is possible to derive a consistent state for that input change.

% \begin{figure}
%     \centering
%     \includegraphics[width=\textwidth]{figures/correctness/orchestration/monotony_counterexample.png}
%     \caption{Counterexample for monotony}
%     \label{fig:formal:monotonycounterexample}
% \end{figure}

% One could now argue that there are binary relations in the example, which may never be fulfilled at all. We will later discuss how far relations that cannot be fulfilled should be restricted. However, in general, this is wanted behavior, because in general it may be necessary that transformations produce intermediate states that are not yet consistent with each other. Otherwise this would means that each transformation is always able to directly deliver a state that is consistent to all other relations, which is especially not possible, because other transformations may add further information to the models. More precisely, a relation may consider <a model consistent to all other models that contain any additional information not affected by the transformation. For example, a UML class model may be considered consistent to all Java models with any implementation of the specified methods, thus to an infinite number of models. Now saying that it should not be allowed that the transformation selects one with an empty implementation because that is not consistent to another relations induced by another transformation, such as the relationship to a component model, does not make any sense. Thus having those relation elements that may be considered locally consistent but will never occur in a globally consistent tuple of models does not make sense.
% In the example, we can see that such an inconsistent intermediate state is passed through and afterwards a consistent tuple of models is reached if not requiring monotony.
% In consequence, requiring monotony from transformations is a too strict requirement, because it is necessary to run through states that may be changed later on.

% \begin{theorem}
%     An application function for monotone transformations either returns a consistent model or produce a sequence of CPRs returning delta that return models of always growing size (i.e., it diverges).
% \end{theorem}


% \paragraph{Divergence cannot be avoided}

% There are rather equal network, one that terminates after a long time and one that never terminates. 
% Consider the example. The relations are defined in a way such that for any allocation for any of them a consistent tuple of models can be found. However, the transformations are not able to find it because they make "bad" choices from a set of choices that are conflicting. 
% This can be seen in the example that we have already given in \autoref{fig:correctness:no_execution_order}.

% Thus, systematically avoiding divergence is not possible. 



% \paragraph{Detecting Alternation / Divergence}

% In consequence, we propose to dynamically deal with alternation / divergence.
% To detect alternations, the execution can simply track if a state way already processed. Apart from spatial problems, this does always work.
% Finding divergence is not that easy, because it is generally not possible to define an upper bound for the number of executions of a single transformation.
% This is due to the reason that, again, this conflicts with the Halting problem.
% We can see this at the simple example in \autoref{fig:formal:noupperboundexample}.

% \begin{figure}
%     \centering
%     \includegraphics[width=\textwidth]{figures/correctness/orchestration/no_upper_bound_example.png}
%     \caption{Example for no upper bound}
%     \label{fig:formal:noupperboundexample}
% \end{figure}

% Depending on the value X, the transformations have to be executed X times to result in a consistent state. This value can be arbitrarily chosen, thus an arbitrary number of executions may be necessary to terminate in a consistent state.

% From an engineering perspective, this is still unwanted behavior. We claim that a transformation network that takes thousands of executions of the same transformation to find a consistent state works not as expected and if running into a failure would expose severe problems to find the reasons for that failures.
% Thus, we propose to simply abort the execution after some time to be sure not to run in an endless loop.

% Finally, this problem is comparable to ordinary programming, because there the same situations regarding alternation and divergence can occur that result in non-termination of a program.
% As we all know, it is impossible to systematically avoid that, but just possible to carefully develop the program and apply best practices to avoid such situations.

% In the following, we propose measures to reduce the number of cases in which problematic cases can occur.
% In a case study, we will see that using such measures already resolves most of the problems that can occur.
% Additionally, we propose an orchestration strategy that improves the possibility to find errors in case something goes wrong.

% \textbf{Central insight:} Alternation / Divergence cannot be avoided systematically (like in ordinary programming), if not restricting transformations in a way that may not be reasonable.

% \subsection{Reducing Conservativeness of the Application Function}

% Goal: Find a solution in as many cases as possible, abort in the others (conservatively). There are two approaches to achieve that: 
% 1. Reduce the number of cases in which there is no solution by adding assumptions to the relations and transformations (restrict input of app function)
% 2. Improve the ability to find a solution if it exists (improve capabilities of app function)
% Secondary goal: In cases, in which no solution is found, support the user in understanding why no solution was found.


% Regarding 1: Reduce problematic cases


% 1. reduce cases in which there is no such solution
% 1.1. On relation level: Only sets, so analysis possible.
% Ensure that relations are defined in a way such that they do not allow a locally correct set of CPRs that has no APP solution. If there is a pair of models (or elements of a fine-grained relation) in a relation, a CPR may return it. But if there is no consistent tuple of models containing these two, it does not make any sense to consider these elements (even worse, if we have monotony, adding these elements makes the network unsolvable). For that reason, we need compatibility. Avoids both alternation and divergence
% 1.2. On transformation level: Hard to perform analyses
% Require monotony to avoid alternation
% Give some example why divergence cannot easily be avoided, thus terminate at some point
% 2. find the solution in as many cases as possible -> reasonable orchestration strategy
% Focus on engineering solution 


% Thus, there arise two questions:
% - Although theoretically easy, how to practically define a CPR that is synchronizing?
% - How to define an APP function and which requirements does that impose?




% \section{Optimality rather than Correctness}

% \mnote{Achieving a correct application function}
% The definition of the application function basically ensures that the function either returns $\bot$ or executes the \modellevelconsistencypreservationrules given by the orchestration function to retrieve a changes tuple of models.
% It is considered \emph{correct} if it ensures that its result is either $\bot$ or a consistent model tuple by executing the \modellevelconsistencypreservationrules given by the orchestration function.
% % A correct application function thus has to ensure that its result is either $\bot$ or a consistent model tuple by executing the \modellevelconsistencypreservationrules given by the orchestration function.
% In consequence, the application function can be realized by simply executing the result of the orchestration function and check whether the resulting model tuple is consistent or not and return an appropriate result.
% Such a realization is generic and does not depend on the actual consistency preservation rules and orchestration function but represents a generic behavior.
% Additionally, this gives an implementation of that function the ability to present a faulty result to the user, which eases finding out why no consistent state was reached.

% \mnote{Correctness is not crucial}
% Finally, correctness is not crucial, because correctness can easily be achieved by performing any execution of transformations and just ensuring that we terminate at some point in time and then decide whether the resulting models are consistent or not and appropriately deliver the result.

% \mnote{How to define an orchestration function that is as optimal as possible?}
% The remaining difficulty is how to define an orchestration function that fulfills the definition, i.e., to find a finite sequence of transformations, and also one that improves optimality, as an \emph{optimal} function can never be given.
% Although the definition of the orchestration function proposes a closed description of that function, in practice such a function will not have a closed form but will be realized as an algorithm that dynamically decides which transformation to execute next.
% Therefore the arising problem is that the length of the sequence to execute is not known a priori. Therefore, we need some abortion criterion. When a consistent result is found, this criterion is easy. But since we do not know whether a sequence exist, we need an abortion criterion that is reasonable and does not cut off the process although a consistent solution could be found, thus reducing optimality.
% A simple realization for that algorithm to deliver a finite sequence of transformations would be to define a fixed termination criterion, such as a specific number of transformation executions. However, there is no upper bound for the number of executed transformations necessary to achieve consistency. Still, a fixed number (even 0) could be defined for the number of executed transformations to fulfill the definition. Hence, optimality would be 0 then as a consistent result is never reached. We therefore discuss in the following how to define an appropriate orchestration function and how to optimize it.

% \mnote{Achieving a correct application function}
% The definition of the application function basically ensures that the function either returns $\bot$ or executes the \modellevelconsistencypreservationrules given by the orchestration function to retrieve a changes tuple of models.
% It is considered \emph{correct} if it ensures that its result is either $\bot$ or a consistent model tuple by executing the \modellevelconsistencypreservationrules given by the orchestration function.
% % A correct application function thus has to ensure that its result is either $\bot$ or a consistent model tuple by executing the \modellevelconsistencypreservationrules given by the orchestration function.
% In consequence, the application function can be realized by simply executing the result of the orchestration function and check whether the resulting model tuple is consistent or not and return an appropriate result.
% Such a realization is generic and does not depend on the actual consistency preservation rules and orchestration function but represents a generic behavior.
% Additionally, this gives an implementation of that function the ability to present a faulty result to the user, which eases finding out why no consistent state was reached.

% \mnote{Correctness is not crucial}
% Finally, correctness is not crucial, because correctness can easily be achieved by performing any execution of transformations and just ensuring that we terminate at some point in time and then decide whether the resulting models are consistent or not and appropriately deliver the result.

% \mnote{How to define an orchestration function that is as optimal as possible?}
% The remaining difficulty is how to define an orchestration function that fulfills the definition, i.e., to find a finite sequence of transformations, and also one that improves optimality, as an \emph{optimal} function can never be given.
% Although the definition of the orchestration function proposes a closed description of that function, in practice such a function will not have a closed form but will be realized as an algorithm that dynamically decides which transformation to execute next.
% Therefore the arising problem is that the length of the sequence to execute is not known a priori. Therefore, we need some abortion criterion. When a consistent result is found, this criterion is easy. But since we do not know whether a sequence exist, we need an abortion criterion that is reasonable and does not cut off the process although a consistent solution could be found, thus reducing optimality.
% A simple realization for that algorithm to deliver a finite sequence of transformations would be to define a fixed termination criterion, such as a specific number of transformation executions. However, there is no upper bound for the number of executed transformations necessary to achieve consistency. Still, a fixed number (even 0) could be defined for the number of executed transformations to fulfill the definition. Hence, optimality would be 0 then as a consistent result is never reached. We therefore discuss in the following how to define an appropriate orchestration function and how to optimize it.

% \textbf{Overall Goal:} Find correct orchestration function that improves optimality.

% There are two ways to improve optimality of the orchestration function:
% \begin{enumerate}
%     \item Optimize the orchestration function, i.e., find a good order (probably this is not possible), at least find an order that helps the developer to find problems
%     \item Optimize the input, i.e., define requirements to the transformations and their relations representing the input to optimize optimality
% \end{enumerate}
% \todo{We need an example for that}

% Both goes hand in hand, because restrictions to the input can never lead to an orchestration function that always terminates without leading to unsupported relevant cases.

% This conform to two approaches:
% \begin{enumerate}
%     \item Dynamic decision about selected transformation and abortion criteria
%     \item Constructive restrictions that ensure that appropriate order is (easily) found
% \end{enumerate}

% \todo{Application function can be generically defined, orchestration maybe not? We actually want to ensure that both are generic and none of them has to be defined for a specific project.}

