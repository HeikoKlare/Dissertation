\section{Conclusions}

In the presented evaluation, we have discussed and provided empirical evidence for several statements regarding the categorization of errors in transformation networks and our approaches for synchronization, analyzing compatibility, and orchestration to avoid such errors, which we could not prove.
Arising from the assumptions that we made for this thesis and discussed in \autoref{chap:introduction:objective:assumptions}, our contributions and their evaluation have some general limitations, which we shortly discuss in the following.

\subsection{Overall Limitations}

In this thesis, we explicitly focus on a modular notion of consistency, defined in terms of binary consistency relations and synchronizing transformations, which are also restricted to the binary case.
In many cases, our considerations and contributions have this restriction without loss of generality and thus also apply to the case when transformations are combined that relate more than two models.
Without explicitly considering that case, however, we currently need to accept is as at limitation, for which we have to validate in future work whether and for which statement it actually is a limitation.
This also resolves the issue that our approaches can currently only be applied to relations that are denoted as \emph{binary-definable} by \textcite{stevens2020BidirectionalTransformationLarge-SoSym}.
We consider this generalization in future work (see \autoref{chap:futurework:correctness:multiary}).

In addition, we restricted ourselves to structural consistency relations (see \autoref{chap:networks:notions:types}).
We need to investigate how far our insights and approaches apply to behavioral and extra-functional consistency relations.
In fact, there is no conceptual limitation in our formalism that prevents it from being applied to behavioral relations.
A hypothesis from a Dagstuhl seminar~\cite{cleve2019dagstuhl} states that behavioral relations may be more likely to be multiary, whereas structural relations are more likely to be binary.
That would reduce this limitation to first discussed one.

Finally, we always assumed user changes to be only performed to one of the models.
Thus, we do not consider concurrent edits to multiple models by one or more users.
While this is, conceptually, covered by transformations being synchronizing, it requires different semantics.
If transformations perform conflicting changes the transformation network is assumed to be erroneous, whereas in case user make conflicting changes, they are invalid.
These cases must at least be distinguished and potentially lead to different processing.
We consider this in future work (see \autoref{chap:futurework:correctness:concurrent}).

% \begin{itemize}
%     \item No concurrent editing support. May partially be resolved by synchronizing transformations, however as we do not make assumptions to these current changes (rather than the assumptions we make to the transformations), the changes of two user can be conflicting. This has to be addressed -> future work
% \end{itemize}

% Only binary relations/transformations rather than BX, but can be transferred

% Only binary-definable relations (see Stevens): But, like above, insights can be transferred to networks of MX rather than BX

% Considered only structural relations (see foundations): Need to investigate behavioral or extra-functional relations. Potentially those are not binary but multiary (see Dagstuhl), so extension to multiary necessary



\subsection{Summary}

Summarize central findings and limitations, especially also relevance of errors that we resolve by construction, how many we can potentially (but not evaluated) find by analysis.


