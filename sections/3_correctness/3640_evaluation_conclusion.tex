\section{Conclusions}

In the presented evaluation, we have discussed and provided empirical evidence for several statements regarding the categorization of errors in transformation networks and our approaches for synchronization, analyzing compatibility, and orchestration to avoid such errors, which we could not prove.
Arising from the assumptions that we made for this thesis and discussed in \autoref{chap:introduction:objective:assumptions}, our contributions and their evaluation have some general limitations, which we shortly discuss in the following together with a derivation of general topics for future work.
We finally summarize the results of our evaluation.


\subsection{Overall Limitations and Future Work}

\mnote{Limitations by assumptions}
For the correctness of transformation networks, we have presented a formal notion based on a well-defined formalism and derived different properties of correct transformation networks.
This thesis especially provides a general formalization of the overall problem and a division into smaller sub-problems, for which it provides individual contributions and insights.
While we made some initial assumptions that lead to general limitations of our contributions, they also provide space for future work.

\paragraph{Binary Consistency}
\mnote{Extension to multiary relations}
As discussed in \autoref{chap:introduction:objective:assumptions}, we assume a development process in which modular transformations are developed and reused independently.
In \autoref{chap:correctness}, we have then introduced our central formalism based on a modular notion of consistency, for which we defined correctness of transformation networks.
We decided to focus on transformations that rely on a binary notion of consistency.
While this is a limitation, since not every multiary consistency relation can be decomposed into binary ones~\cite{stevens2020BidirectionalTransformationLarge-SoSym}, for most considerations we made this limitation is actually only for ease of understanding but without loss of generality.
Thus most of our considerations and contributions also apply to networks of transformations of which each relates more than two models.
Since we did not explicitly consider that case, however, we currently need to accept it as a limitation, until we validate whether and which statements generalize in future work.
This also resolves the issue that our approaches can currently only be applied to relations that are denoted as \emph{binary-definable} by \textcite{stevens2020BidirectionalTransformationLarge-SoSym}.

\paragraph{Structural Consistency}
\mnote{Considering behavioral consistency}
In addition, we restricted ourselves to structural consistency relations (see \autoref{chap:networks:notions:types}).
We need to investigate how far our insights and approaches apply to behavioral and extra-functional consistency relations as well.
In fact, there is no conceptual limitation in our formalism that prevents it from being applied to behavioral relations.
A hypothesis from a Dagstuhl seminar~\cite{cleve2019dagstuhl} states that behavioral relations may be more likely to be multiary, whereas structural relations are more likely to be binary.
That would reduce this limitation to the one regarding the restriction to a binary consistency notion and thus imply the same necessity for future work.

\paragraph{Concurrent Editing}
\mnote{Difference to synchronization}
Finally, we assumed that a user only changes one model, for which consistency then has to be preserved.
Thus, we do not consider concurrent edits to multiple models by one or more users.
Although, from a conceptual point of view, networks of synchronizing transformations can also handle concurrent edits in multiple models, as the transformations need to be synchronizing anyway, the process of dealing with problems must be different.
While failures that occur without concurrent user edits in different models indicate faults within the transformations, concurrent edits can also lead to failures just because conflicting changes were made and are thus invalid.
These cases must at least be distinguished and potentially lead to the necessity of different processing.
This topic requires further investigation in future work, also incorporating existing work on considering concurrent updates in single transformations~\cite{xiong2009parallelUpdates-ICMT,xiong2013SynchronizingConcurrentUpdates-SoSym}.


\subsection{Summary}

\mnote{Correctness approaches and their evaluation}
In the preceding chapters, we have introduced a notion for correctness of transformation networks and identified three specific problems to be discussed in detail.
We have proposed an approach to analyze compatibility of consistency relations, whose formal representation is proven correct and for whose practical realization we empirically validated correctness and completeness.
Transformations must be synchronizing to be used in transformation network.
We have derived properties that transformations which are specified in existing languages for bidirectional transformations need to fulfill, for which we haven proven that they ensure synchronization.
In an empirical evaluation, we have shown that the proposed approach to fulfill these properties is correct and complete.
Finally, we have discussed the orchestration problem of finding consistent orchestrations for transformations, for which we have proven undecidability.
We have proposed an algorithm that conservatively approximates a solution to that problem, for which we have also proven correctness and completeness and validated usefulness in a scenario-based discussion.

\mnote{Errors, their relevance and impact on independent development}
In addition, we have analyzed what happens if correctness notions are not fulfilled.
We have proposed a categorization of mistakes, faults, and failures, which assigns mistakes to different conceptual levels in the specification process of transformation networks and shows that specific failures can be avoided if certain mistake types are avoided.
We found that mistakes due to missing synchronization can be avoided by construction of a single transformation without knowledge about the other transformations to combine it with.
Mistakes due to incompatible consistency relations can be found by analysis, and other mistakes are only found by failures during execution.
An empirical evaluation has shown that this categorization is correct.
In particular, the evaluation has also revealed that most of the faults that are likely to occur in practice are due to missing synchronization and can thus be avoided by construction.
Of the remaining faults, most are due to incompatible constraints and can thus at least be found by analysis at design time.
This is a promising insight, because it fosters the independent development of transformations, as most failures can already be avoided without knowing about other transformations to combine it with.
Thus, as a central takeaway, it is particularly important to ensure that transformations are synchronizing by construction.

