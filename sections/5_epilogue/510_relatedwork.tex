\chapter{Related Work
    \pgsize{15 p.}
}
\label{chap:relatedwork}

\section{From Commonalities}

The \commonalities approach is related to the highly researched field of model consistency and especially of model 
transformations.
In the following, we compare our approach to others that rely on commonality specifications, to both multidirectional transformations and transformation networks that also allow consistency preservation between multiple models and finally to constraint solving, a different paradigm for preserving model consistency.

\subsection*{\commonality Approaches}
%Commonalities \metamodels offer a different approach to reduce the number of transformations and potential issues.
%In this paper, we propose a generic idea for that, whereas DUALLy~\cite{malavolta2010a, eramo2012a} uses a domain-specific commonalities \metamodel for architecture description languages.

The idea of defining commonalities to express consistency of multiple models was especially researched from a theoretical viewpoint.
That research is based on the idea of using an additional $n+1$-th \metamodel to decompose the $n$-ary consistency relation between $n$ \metamodels into $n$ binary relations~\cite{stunkel2018a, diskin2018a}.
%Equal ideas to Commonalities especially researched from a theoretical viewpoint (usually one big commonality, which solves the problem, as all n-ary relations between n metamodel can be expressed by binary relations to an additional n+1 th metamodel)~\cite{stunkel2018a, diskin2018a}.

Existing approaches to practically use commonalities for keeping multiple models consistent are domain-specific. 
The DUALLy approach~\cite{malavolta2010a, eramo2012a} uses a domain-specific concept \metamodel for architecture description languages, which is a fixed \metamodel to which relations of arbitrary architecture description languages can be defined.


\subsection*{Multidirectional Transformations}

Without defining additional \metamodels, multidirectional transformations are an approach to directly define the relations between multiple \metamodels.
The QVT-R standard~\cite{qvt} considers multidirectional transformations, but \textcite{macedo2014a} reveal several limitations of its applicability and propose strategies to circumvent them.
\acp{TGG} are a graph-based approach to define transformations, which has been extended to enable the specification of multidirectional rules~\cite{trollmann2015a, trollmann2016a}.
In contrast to our work, these approaches support the specification on $n$-ary relations between $n$ \metamodels, but do not provide means to improve their understandability as we expect the definition of Commonalities to do.


\subsection*{Networks of Bidirectional Transformations}

We introduced networks of bidirectional transformations as the state-of-the-art for specifying consistency relations between multiple \metamodels.
\textcite{stevens2017a} investigates the ability to decompose $n$-ary relations into binary ones and also discusses confluence issues, which arise from incompatibilites of transformations, as discussed in \autoref{sec:approach:benefits}.
Such a decomposition of relations is not always possible, thus such approaches are restricted to cases where all $n$-ary relations can be decomposed into binary ones.
Additionally, such networks are prone to compatibility errors or reduced modularity, as discussed in \autoref{sec:approach:benefits}.

Transformation composition and transformation chains deal with specific problems of transformation networks.
Composition techniques deal with internal composition of transformations~\cite{wagelaar2008a}, which are techniques that are integrated into a language, and external composition of transformations, which work independently from the language.
Those approaches especially comprise factorization and re-composition of transformations~\cite{cuadrado2008a} and investigations of compatibility of transformations for different versions of the same \metamodels.
Transformation chains deal with specific networks that occur when transformations from \metamodels with a high level of abstraction to those with a low level of abstraction are defined.
Specification languages for transformation chains %, such as FTG+PM,
allow to combine transformations to chains~\cite{lucio2013a} and to treat them as black-boxes%like in UniTI
~\cite{vanhooff2006a, vanhooff2007a}. 


\subsection*{Constraint Solving}

Consistency relations between multiple \metamodels can also be expressed as logical constraints.
Restoring consistency for a set of models can be achieved by constraint solving.
%A different approach for describing consistency relations between multiple \metamodels and restoring them for a set of instances is constraint solving, where consistency relations are expressed as logical constraints.
%Consistency relations can be expressed as logical constraints that have to be fulfilled by a set of models.
\textcite{eramo2008a} consider the usage of \ac{ASP} for that. %to define consistency relations between \metamodels.
The approach derives a set of candidates that fulfill the constraints after a model is modified. %, such that as few changes are performed on the models as possible to restore consistency.
However, that research focuses on solving constraints rather than designing an appropriate way how to define them, in contrast to our \commonalities approach.
