\chapter{Conclusions \pgsize{5 p.}}

\subsection{Quality Properties by Commonalities}

In this paper, we proposed the \commonalities approach, which allows to make common concepts of different \metamodels explicit.
The central idea of the presented approach 
%Its central idea 
is to define \emph{\conceptmetamodels} that represent the common concepts, i.e., the \commonalities of two or more existing \metamodels, and to define the relation of those \metamodels to the \conceptmetamodels.
\Conceptmetamodels can be hierarchically composed to enable the separate definition and combination of independent concepts.
We discussed different options for designing a language that supports the specification of such \commonalities and their relations, 
as well as for operationalizing such a specification to executable transformations.
We outlined a language for the \commonalities approach and explained %our selection of the aforementioned options. %
which of the aforementioned options we chose, and why.
%We have implemented the language as a proof-of-concept and applied it to simple scenarios.
Finally, we have applied an implementation of that language to simple scenarios as a proof-of-concept.
The results indicate the feasibility of applying the \commonalities approach and implementing an appropriate language.

The expected benefit of our approach is a better understandability of relations between \metamodels compared to their implicit encoding in transformations.
Additionally, we argued why our approach improves the essential properties \emph{compatibility} and \emph{modularity}, which usually contradict each other in other approaches to keep multiple models consistent, like networks of transformations that define the direct relations between \metamodels.
In ongoing work, we extend the capabilities of the language to perform a comprehensive evaluation of the functionality of the approach as well as its applicability to more sophisticated case studies.
We will validate our claim of improved understandability in a controlled experiment.
Nevertheless, the initial results of our proof-of-concept are a promising indicator for the applicability of the \commonalities approach.
