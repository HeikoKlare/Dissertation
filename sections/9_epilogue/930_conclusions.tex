\chapter{Conclusions \pgsize{5 p.}}

Insight:
- Correctness of transformations (for local consistency) can be achieved by construction -> synchronizing transformation can be build without knowing about others to combine with
- Compatibility can only be checked for given transformations (not without knowing about other transformation), but actually not having compatibility does not affect correctness
- Orchestration is solved independent from concrete network and must not be adapted
Conclusion: We have an approach to achieve correctness by construction, have a well-defined level of conservativeness for the orchestration and support developers in the conservative cases, and finally provide a static prover for compatibility. Thus we can define correct networks.

\begin{copiedFrom}{SoSym MPM4CPS}

\subsection{Compatibility}

In this article, we presented an approach to prove compatibility of consistency relations in transformation network.
We introduced a formal notion of compatibility, describing when consistency relations are considered contradictory, and we proved correctness of a formal approach that checks whether a transformation network is compatible.
We defined an operationalization of that approach for QVT-R and OCL, which uses the translation of OCL to first-order logic formulae and an SMT solver to prove compatibility.

Applying the approach to different scenarios in an evaluation, we found that the approach operates correctly in the sense that it produces \emph{conservative} results.
Further, we found that conservativeness is rather low, i.e., only few actually compatible transformation networks were not identified as such.
More precisely, only 20\% of the compatible transformations were not identified as such, which indicates the practical applicability of the approach.
The current limitations of the approach and the degree of conservativeness especially arise from limitations due to undecidability of OCL and missing translations of OCL constructs to first-oder logic formulae.
We did not identify conceptual issues that limit the expressiveness or applicability of our approach.

The presented approach enables developers of transformations to independently define their transformations and combine them afterwards, without the necessity to align the underlying consistency relations a priori or to check their compatibility manually when combining them.
This is an important contribution to the overall goal of being able to build properly working networks of independently developed transformations to foster the development of large software and cyber-physical systems that involve several models and views to describe that system under construction.

% Applying the approach to different evaluation networks has shown that it finds an appropriate decomposition and proves compatibility in most cases, indicating its practical applicability. Its limitations especially arise from an incomplete representation of all possible OCL expressions as logical formulas and the restricted capabilities of the used SMT solver.

\end{copiedFrom} % SoSym MPM4CPS



\begin{copiedFrom}{ICMT}

\subsection{Interoperability Issues}
Issues that can arise from the combination of independently developed \acp{BX} to networks have not been systematically investigated yet.
In this paper, we therefore categorized failures that can occur when executing faulty networks of \acp{BX}.
Additionally, we structured the process of specifying consistency into three levels: the global level, the modularization level and the operationalization level.
These levels carry the danger for different kinds of mistakes, which we categorized and related to potential failures they can result in.
We found that each of the levels is prone to different types of mistakes, and that each type of failure is specific for one category of mistake.
This enables developers to easily identify the kind of mistake they made when recognizing a failure.
Additionally, the systematic knowledge about potential mistakes, failures, and their relations makes it possible to further develop techniques to avoid them.
We have discussed two general avoidance strategies at the modularization and operationalization level in this paper.
In future work, we will especially investigate how far and under which assumptions \acp{BX} can be analyzed regarding contradictions at the modularization level when they are combined.

\end{copiedFrom} % ICMT



\begin{copiedFrom}{VoSE}

\subsection{Quality Properties by Commonalities}

In this paper, we proposed the \commonalities approach, which allows to make common concepts of different metamodels explicit.
The central idea of the presented approach 
%Its central idea 
is to define \emph{\conceptmetamodels} that represent the common concepts, i.e., the \commonalities of two or more existing metamodels, and to define the relation of those metamodels to the \conceptmetamodels.
\Conceptmetamodels can be hierarchically composed to enable the separate definition and combination of independent concepts.
We discussed different options for designing a language that supports the specification of such \commonalities and their relations, 
as well as for operationalizing such a specification to executable transformations.
We outlined a language for the \commonalities approach and explained %our selection of the aforementioned options. %
which of the aforementioned options we chose, and why.
%We have implemented the language as a proof-of-concept and applied it to simple scenarios.
Finally, we have applied an implementation of that language to simple scenarios as a proof-of-concept.
The results indicate the feasibility of applying the \commonalities approach and implementing an appropriate language.

The expected benefit of our approach is a better understandability of relations between metamodels compared to their implicit encoding in transformations.
Additionally, we argued why our approach improves the essential properties \emph{compatibility} and \emph{modularity}, which usually contradict each other in other approaches to keep multiple models consistent, like networks of transformations that define the direct relations between metamodels.
In ongoing work, we extend the capabilities of the language to perform a comprehensive evaluation of the functionality of the approach as well as its applicability to more sophisticated case studies.
We will validate our claim of improved understandability in a controlled experiment.
Nevertheless, the initial results of our proof-of-concept are a promising indicator for the applicability of the \commonalities approach.

\end{copiedFrom} % VoSE

