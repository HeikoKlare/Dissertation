\emph{In dieser Dissertation formalisieren und analysieren wir die Konsistenzerhaltung verschiedener Artefakte zur Beschreibung eines Softwaresystems durch die Kopplung von Transformationen zwischen diesen, und unterstützten sie mit geeigneten Methoden.}

% Kontext: Konsistenz von Softwaresystemen
Für die Entwicklung eines Softwaresystems nutzen Entwickelnde und weitere Beteiligte verschiedene Sprachen, oder allgemein Werkzeuge, zur Beschreibung unterschiedlicher Belange.
Meist stellt Programmcode das zentrale Artefakt dar, welches jedoch, implizit oder explizit, durch Spezifikationen von Architektur, Deployment, Anforderungen und anderen ergänzt wird.
Neben der Programmiersprache verwenden die Beteiligten weitere Sprachen zur Spezifikation dieser Artefakte, beispielsweise die \acrshort{UML} für Modelle des objektorientierten Entwurfs oder der Architektur, den OpenAPI-Standard für Schnittstellen-Definitionen, oder Docker für Deployment-Spezifikationen.
Zur Erstellung eines funktionsfähigen Softwaresystems müssen diese Artefakte das System einheitlich und widerspruchsfrei darstellen.
Beispielsweise müssen Dienst-Schnittstellen in allen Artefakten einheitlich repräsentiert sein.
Wir sagen, die Artefakte müssen \emph{konsistent} sein.

% Problem: Keine Transformationskombination für mehr als zwei Modelle
In der modellgetriebenen Entwicklung werden solche verschiedenen Artefakte allgemein \emph{Modelle} genannt und bereits als wesentliche zentrale Entwicklungsbestandteile genutzt, um auch Teile des Programmcodes aus ihnen abzuleiten.
Dies betrifft beispielsweise die Softwareentwicklung für Fahrzeuge. %~\owncite{guissouma2018study,sax2017survey} oder allgemein die Entwicklung cyber-physikalischer Systeme.
Zur Konsistenzerhaltung der Modelle werden hier oftmals Transformationen eingesetzt, die nach Änderungen eines Modells die anderen Modelle anpassen.
Die bisherige Forschung beschränkt sich jedoch auf Transformationen zur Konsistenzerhaltung zweier Modelle %~\cite{stevens2020BidirectionalTransformationLarge-SoSym} 
und die projektspezifische Kombination von Transformationen zur Konsistenzerhaltung mehrerer Modelle\footfullcite{cleve2019dagstuhl}. %,diskin2018MultiModelSynchronization-FASE}.
Ein systematischer Entwicklungsprozess, in dem einzelne Transformationen unabhängig und modular entwickelt und in verschiedenen Kontexten wiederverwendet werden können, wird hierdurch jedoch nicht unterstützt.

% Annahmen und Beitragsbereiche: Korrektheit und Qualitätseigenschaften
In dieser Dissertation erforschen wir, wie Entwickelnde mehrere Transformationen zu einem Netzwerk kombinieren können, welches die Transformationen in einer geeigneten Reihenfolge ausführen kann, sodass abschließend alle Modelle konsistent zueinander sind. %~\owncite{klare2021Vitruv-JSS,klare2018docsym}.
Dies geschieht unter der Annahme, dass einzelne Transformationen zwischen zwei Sprachen unabhängig voneinander entwickelt werden und daher nicht aufeinander abgestimmt werden können.
Unsere Beiträge unterteilen sich in die Untersuchung der Korrektheit einer solchen Kombination von Transformationen zu einem Netzwerk und die Optimierung von Qualitätseigenschaften solcher Netzwerke.

% Beiträge und Evaluation: Korrektheit
Wir diskutieren und definieren zunächst einen einen adäquaten Korrektheitsbegriff, welcher drei spezifische Anforderungen impliziert. Diese umfassen eine \emph{Synchronisations}-Eigenschaft für die einzelnen Transformationen, eine \emph{Kompatibilitäts}-Eigenschaft für das Transformationsnetzwerk, sowie das Finden einer geeigneten Ausführungsreihenfolge der Transformationen, einer \emph{Orchestrierung}.
Wir stellen ein Konstruktionsverfahren für Transformationen vor, mit welchem die Synchronisations-Eigenschaft bei der Verwendung bestehender Sprachen zur Transformationsspezifikation basierend auf einer formal bewiesenen Eigenschaft erfüllt werden kann.
Für dieses zeigen wir Vollständigkeit und Angemessenheit mit einer fallstudienbasierten empirischen Evaluation in der Domäne der komponentenbasierten Softwareentwicklung.
Weiterhin definieren wir formal die Eigenschaft der Kompatibilität von Transformationen% ~\cite{klare2020compatibility-report}
, für welche wir ein formales und bewiesen korrektes Analyseverfahren vorschlagen und eine praktische Realisierung ableiten, deren Anwendbarkeit wir ebenfalls in Fallstudien nachweisen.
Schlussendlich definieren wir das \emph{Orchestrierungsproblem} zum Finden einer Orchestrierung, d.h. einer Ausführungsreihenfolge der Transformationen nach der die Modelle konsistent sind, wann immer solch eine Reihenfolge existiert. %\cite{gleitze2021orchestration-FASE}.
Wir beweisen die Unentscheidbarkeit dieses Problems und diskutieren Indikatoren dafür, dass eine Einschränkung des Problems um Entscheidbarkeit zu erreichen die Anwendbarkeit unpraktikabel einschränken würde.
Daher schlagen wir einen Algorithmus vor, der das Problem konservativ behandelt, indem er eine solche Orchestrierung möglicherweise nicht findet, obwohl sie existiert.
Stattdessen beweisen wir dessen Korrektheit und eine Eigenschaft, die das Finden der Ursache im Fehlerfall unterstützt.
Zusätzlich kategorisieren wir Fehler, die auftreten können, falls ein Netzwerk den definierten Korrektheitsbegriff nicht erfüllt.
Daraus leiten wir mittels den bereits angesprochenen Fallstudien ab, dass die meisten möglichen Fehler per Konstruktion mit den in dieser Arbeit vorgeschlagenen Ansätzen vermieden werden können.

% Beiträge und Evaluation: Qualitätseigenschaften
Zur Untersuchung von Qualitätseigenschaften eines Netzwerkes von Transformationen klassifizieren wir zunächst relevante Eigenschaften, sowie den Effekt verschiedener Typen von Netzwerktopologien auf diese. %~\cite{klare2018docsym}.
Hierbei zeigt sich, dass insbesondere Korrektheit und Wiederverwendbarkeit im Widerspruch stehen, sodass die Wahl der Netzwerktopologie ein Abwägen bei der Optimierung dieser Eigenschaften erfordert.
Wir leiten hieraus ein Konstruktionsverfahren für Transformationsnetzwerke ab%~\cite{klare2019models}
, welches die Notwendigkeit einer Abwägung zwischen den Qualitätseigenschaften abmildert und, unter gewissen Voraussetzungen, Korrektheit per Konstruktion gewährleistet. %und gleichzeitig die modulare Wiederverwendung aller Teile des Transformationsnetzwerkes gewährleitet.
Wir unterstützen den Entwicklungsprozess für diesen Ansatz mithilfe einer spezialisierten Spezifikationssprache.
%Für dieses Verfahren stelle ich außerdem eine Spezifikationssprache vor~\cite{klare2019models}, die den Entwicklungsprozess strukturiert.
Während die Verminderung der Notwendigkeit einer Abwägung zwischen Qualitätseigenschaften durch den Ansatz per Konstruktion erreicht wird, zeigen wir die Erreichbarkeit der Voraussetzungen und die Vorteile der vorgeschlagenen Sprache in einer empirischen Evaluation mithilfe derselben Fallstudie aus der komponentenbasierten Softwareentwicklung wie für die anderen Evaluationen.
% Ich evaluiere die Anwendbarkeit des Verfahrens und der Sprache durch die Implementierung der bereits für die Korrektheitseigenschaften verwendeten Fallstudien aus der komponentenbasierten Softwareentwicklung. 
% Hierzu validiere ich insbesondere die Erfüllbarkeit der Voraussetzungen zur Anwendung des Verfahrens in realistischen Szenarien, sowie die Kompaktheit der vorgestellten Sprache.
%, in der insbesondere die Erfüllung der Voraussetzungen des Ansatzes in realen Szenarien gezeigt werden konnte.

% Nutzen
Die Beiträge dieser Dissertation unterstützen sowohl Forschende als auch Transformationsentwickelnde und Transformationsanwendende bei der Analyse und Konstruktion von Netzwerken von Transformationen. %~\owncite{klare2019dagstuhl}.
Sie stellen für Forschende und Transformationsentwickelnde systematisches Wissen über die Korrektheit und weitere Qualitätseigenschaften solcher Netzwerke bereit, und zeigen insbesondere welche Teile dieser Eigenschaften per Konstruktion erreicht werden können, welche per Analyse validiert werden können, und welche Fehler unvermeidbar bei der Ausführung erwartet werden müssen.
Zusätzlich zu diesen Einsichten stellen wir konkrete, praktisch nutzbare Verfahren bereit, mit denen korrekte, modulare Netzwerke konstruiert, analysiert und ausgeführt werden können, wovon sowohl Transformationsentwickelnde als auch Transformationsanwendende profitieren.

