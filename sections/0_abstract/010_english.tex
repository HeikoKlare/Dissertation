\emph{In this thesis, we formalize and analyze how to preserve consistency between multiple artifacts describing the same software system through the combination of transformations between them% their specification languages
and support it with appropriate methods.}

% Context: Consistency of software systems
During the development of a software system, the developers and further stakeholders employ multiple languages or, in general, tools to describe different concerns.
Code often represents the central artifact, which is, however, implicitly or explicitly complemented by specifications of the architecture, deployment, requirements and others.
In addition to the programming language, further languages are used to specify these artifacts, such as the \acrshort{UML} for object-oriented design or architecture models, the OpenAPI standard for interface definitions, or Docker for deployment specifications.
To achieve a functional software system, all these artifacts must depict a uniform, non-contradicting specification of the whole system.
Interfaces of services must, for example, be represented in all these artifacts uniformly.
We say that the artifacts have to be \emph{consistent}.

% Problem: No transformation combination for more than two models
In model-driven software development, such artifacts are denoted as \emph{models} and represent central units of the development process, from which also at least parts of the program code can be derived.
This is, for example, already applied in automotive software development.
A common means to preserve consistency between models are transformations, which adapt the other models after one of them was changed.
Existing research is restricted to transformations that preserve consistency between pairs of models or to project-specific combinations of transformations to preserve consistency of multiple models. %\footfullcite{cleve2019dagstuhl}.
A systematic development process that enables the independent development of transformations and their modular reuse in different contexts is, however, not yet supported.

% Assumptions and contribution areas: Correctness and quality properties
In this thesis, we research how developers can combine multiple transformations to a network that is able to execute these transformations in an order such that all resulting models are consistent.
To this end, we assume that each transformation between two languages is developed independently and that the transformations cannot be aligned with each other.
Our contributions are separated into those concerning the correctness of such a combination of transformations to a network and those concerning the optimization of quality properties of such a network.

% Contributions and evaluation: Correctness
We first derive and precisely define an appropriate notion of correctness for transformation networks.
It induces three specific requirements, which are a \emph{synchronization} property of the single transformations, a \emph{compatibility} property of a network of transformations, and finding an appropriate \emph{orchestration}, i.e., an execution order of the transformations.
We propose a construction approach for transformations to fulfill the synchronization property with existing transformation specification languages on a formally proven property.
For this approach, we show completeness and appropriateness with a case-study-based empirical evaluation in the domain of component-based software engineering.
We formally define compatibility of transformations, for which we propose a formal analysis, which is proven correct, and derive a practical analysis, whose applicability we demonstrate with case studies.
Finally, we define the orchestration problem of finding an orchestration that delivers consistent models whenever such an orchestration exists.
We prove undecidability of that problem and discuss that restrictions to achieve its decidability will likely limit practical applicability.
For that reason, we propose an algorithm that conservatively approaches the problem.
It guarantees to deliver an orchestration under specific, well-defined conditions and otherwise indicates an error.
We prove correctness of the algorithm and a property that supports finding the cause whenever the algorithm fails.
Additionally, we categorize errors that can occur if a transformation network does not fulfill the defined correctness notion, from which we derive by means of the mentioned case studies that most potential errors can be avoided by construction with the approaches that we propose in this thesis.

% Contributions and evaluation: Quality properties
The investigation of quality properties of transformation networks is based on a classification of relevant properties and of the effects of different types of network topologies on them.
It reveals that especially correctness and reusability are contradictory, thus the selection of a network topology induces a trade-off between these properties.
We derive a construction approach for transformation networks that mitigates the necessary trade-off decision and, under specific assumptions, guarantees correctness by construction.
We support the development process for this approach with a specialized specification language.
While trade-off mitigation is given by construction of the approach, we show achievability of the assumptions and benefits of the proposed language in an empirical evaluation using the case study from component-based software engineering.

% Benefits
The contributions of this thesis support researchers as well as transformation developers and users of transformations in analyzing and constructing networks of transformations.
They depict systematic knowledge about correctness and further quality properties of transformation networks for researchers and transformation developers.
In particular, they show precisely which parts of these properties can be achieved by construction, which can be validated by analysis, and which errors must inevitably be expected during execution. 
Along with these insights, we provide concrete, practically applicable approaches for the construction, analysis and execution of correct and modularly reusable transformation networks, from which developers and users of transformation networks both benefit.
