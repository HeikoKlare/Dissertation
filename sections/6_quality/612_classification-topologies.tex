\section{Topologies of Transformation Networks}

\subsection{Topology Categories}

Dense graphs vs. trees

\begin{figure}
    \centering
    \begin{minipage}[b]{0.49\columnwidth}
        \centering
        \newcommand{\hmmdistance}{3.6em}
\newcommand{\vmmdistance}{2.4em}

\begin{tikzpicture}[
    mm/.style={schematic metamodel},
]

\node[mm] (full_left) {};
\node[mm, above right=\vmmdistance and \hmmdistance of full_left.center, anchor=center] (full_top) {};
\node[mm, below right=\vmmdistance and \hmmdistance of full_left.center, anchor=center] (full_bottom) {};
\node[mm, right=2*\hmmdistance of full_left.center, anchor=center] (full_right) {};
\node[mm, below left=\vmmdistance and \hmmdistance of full_left.center, anchor=center] (full_bottomleft) {};

\draw[transformation] (full_left) -- (full_top);
\draw[transformation] (full_left) -- (full_right);
\draw[transformation] (full_left) -- (full_bottom);
\draw[transformation] (full_top) -- (full_right);
\draw[transformation] (full_top) -- (full_bottom);
\draw[transformation] (full_right) -- (full_bottom);
\draw[transformation] (full_left) -- (full_bottomleft);
\draw[transformation] (full_bottom) -- (full_bottomleft);
\draw[transformation] (full_top) to[bend right=30] (full_bottomleft);
\draw[transformation] (full_bottomleft) .. controls ++(1*\hmmdistance, -0.8*\vmmdistance) and ([yshift=-1.5*\vmmdistance]full_right.south) .. (full_right);

\end{tikzpicture}
        \subcaption{Dense graph}
        \label{fig:classification:topologies:full}
    \end{minipage}
    \hfill
    \begin{minipage}[b]{0.49\columnwidth}
        \centering
        \newcommand{\hmmdistance}{3.6em}
\newcommand{\vmmdistance}{2.4em}

\begin{tikzpicture}[
    mm/.style={schematic metamodel}
]

\node[mm, right=4*\hmmdistance of full_left.center, anchor=center] (tree_left) {};
\node[mm, above right=\vmmdistance and \hmmdistance of tree_left.center, anchor=center] (tree_top) {};
\node[mm, below right=\vmmdistance and \hmmdistance of tree_left.center, anchor=center] (tree_bottom) {};
\node[mm, right=2*\hmmdistance of tree_left.center, anchor=center] (tree_right) {};
\node[mm, below left=\vmmdistance and \hmmdistance of tree_left.center, anchor=center] (tree_bottomleft) {};

\draw[transformation] (tree_left) -- (tree_top);
\draw[transformation] (tree_left) -- (tree_bottom);
\draw[transformation] (tree_left) -- (tree_bottomleft);
\draw[transformation] (tree_top) -- (tree_right);

\end{tikzpicture}
        \vspace{1em}
        \subcaption{Tree}
        \label{fig:classification:topologies:tree}
    \end{minipage}
    \caption[Extremes of transformation network topologies]{Examples for extremes of transformation network topologies. Nodes depict metamodels and edges depict transformations. Adapted from~\owncite[Fig. 2]{klare2018docsym}.}
    \label{fig:classification:topologies}
\end{figure}



\begin{copiedFrom}{DocSym}

%As discussed during the presentation of the challenges in the previous section, 
We will focus on compatibility and modularity, as they are crucial for the applicability of transformations. %, whereas comprehensibility and development effort can be seen as usability problems. %, so we focus on those former two challenges.
Defining binary transformations %to express consistency preservation 
for a set of metamodels leads to a trade-off solution regarding those challenges, as they cannot be solved simultaneously.
The intuitive way to define transformations %preservation of consistency relations is achieved by expressing each relation in a transformation.
is to express each relation between two metamodels in one transformation, which leads to a dense graph of transformations. % between the metamodels.
In the extreme case, if all pairs of metamodels have non-empty consistency relations, the graph is complete, as shown in \autoref{fig:classification:topologies:full}.
In that case, modularity is high because each metamodel can be excluded without any drawback, %, comprehensibility is rather high as each relation is expressed directly between the involved metamodels, %(although it may be additionally expressed transitively), 
but relations are likely to be incompatible, as, in the worst case, each relation is specified over $(n-1)!$ transformation paths if $n$ metamodels are involved.
While comprehensibility is high, as each relation is explicitly expressed, adding a metamodel requires to define up to $n-1$ transformations, implying high evolution effort.
%Finally, this also implies a high development effort, as adding a metamodel requires to define transformations to all existing ones.

Another extreme case is to have each consistency relation only defined over a single path in the transformations graph, which results in a tree of transformations, as shown in \autoref{fig:classification:topologies:tree}.
In that case, compatibility of transformations is inherently given, as each relation is only defined once, but modularity is reduced, as only metamodels being leaves can be omitted.
Comprehensibility is low, as each relation may be defined in a path of up to $n-1$ transformations, but evolvability is rather good, as each relation must only be defined once. %, but in a way such that the tree structure is maintained. %requires a critical investigation about how to express the consistency relations to maintain the tree structure.
We summarize that impact of the topology in \autoref{tab:classification:topology_impact}.

%In the following sections, we discuss our proposed solutions on how to deal with those trade-offs.
\begin{table}
    \centering
    \begin{tabular} {L{8em} C{6em} C{5em}}
        \toprule
        \textbf{Challenge} & \textbf{Dense Graph} & \textbf{Tree}\\
        \midrule
    %    Uniqueness & - - & ++\\
        Compatibility & - & ++\\
        Modularity & ++ & -\\
        Comprehensibility & + & -\\
        Evolvability & - & +\\
        \bottomrule
    \end{tabular}
    \caption{Challenge fulfillment by transformation topology}
    \label{tab:classification:topology_impact}
    \vspace{-1.5em}
\end{table}

A tree topology has the drawback that it is not always applicable.
It requires that the transformation developers
%\todoErik{whom?}
find a subset of all consistency relations between the metamodels that induces a tree and whose transitive closure contains all other relations.
%This is necessary, because the transitive relations in the tree must also cover the direct relations between the metamodels that are omitted.
In general, such a subset cannot be found.
Of three metamodels, there must always be one containing the overlapping information of the two others, as only then the transitive closure of two consistency relations subsumes the third. 
In the example in \autoref{fig:classification:tree_generation}, it is necessary that two relations are contained in the transitive closure of the others to get a tree that covers all relations.
%An example for such a relation is given in the graphic.

\begin{figure}[b]
    \centering
    \newcommand{\mmdistance}{4em}

\begin{tikzpicture}[
    mm/.style={schematic metamodel},
]

% graph

\node[mm] (graph_top) {};
\node[mm, below=\mmdistance of graph_top.center, anchor=center] (graph_middle) {};
\node[mm, below left=0.5*\mmdistance and sqrt(3)/2*\mmdistance of graph_middle.center, anchor=center] (graph_bottomleft) {};
\node[mm, below right=0.5*\mmdistance and sqrt(3)/2*\mmdistance of graph_middle.center, anchor=center] (graph_bottomright) {};

\draw[consistency relation] (graph_top) -- node[left] {$R_2$} (graph_middle);
\draw[consistency relation] (graph_top) to[bend right=40] node[pos=0.3, above left] {$R_1$} (graph_bottomleft);
\draw[consistency relation] (graph_top) to[bend left=40] node[pos=0.3, above right] {$R_3$} (graph_bottomright);
\draw[consistency relation] (graph_middle) -- node[pos=0.4, above left] {$R_4$} (graph_bottomleft);
\draw[consistency relation] (graph_middle) -- node[pos=0.4, above right] {$R_5$} (graph_bottomright);

% tree

\node[mm, right=4*\mmdistance of graph_top] (tree_top) {};
\node[mm, below=\mmdistance of tree_top.center, anchor=center] (tree_middle) {};
\node[mm, below left=0.5*\mmdistance and sqrt(3)/2*\mmdistance of tree_middle.center, anchor=center] (tree_bottomleft) {};
\node[mm, below right=0.5*\mmdistance and sqrt(3)/2*\mmdistance of tree_middle.center, anchor=center] (tree_bottomright) {};

\draw[consistency relation] (tree_top) -- node[left] {$R_2$} (tree_middle);
\draw[consistency relation] (tree_middle) -- node[pos=0.4, above left] {$R_4$} (tree_bottomleft);
\draw[consistency relation] (tree_middle) -- node[pos=0.4, above right] {$R_5$} (tree_bottomright);


\draw[latex-latex] ([yshift=0.2*\mmdistance, xshift=1.1*\mmdistance]graph_middle.east) -- 
    node[above] {$R_2 \concat R_4 \subseteq R_1$} 
    node[below] {$R_2 \concat R_5 \subseteq R_3$}
    ([yshift=0.2*\mmdistance, xshift=-1*\mmdistance]tree_middle.west);

\end{tikzpicture}
    \caption{Reducing a consistency relation graph to a tree}
    \label{fig:classification:tree_generation}
\end{figure}

In practice, the used topology will potentially be a mixture between those extremes.
The natural way to foster the independent development of transformations %to specify consistency preservation 
would be to define one transformation for each consistency relation, resulting in a dense graph with a high potential for incompatible transformations.
To deal with that, mechanisms that analyze compatibility between transformations could be researched.
Nevertheless, high expressiveness of transformation languages allows only conservative approximations. % without reducing the expressiveness of transformation languages, this easily leads to problems similar to those in static code analysis, which rely on approximations, due to the Halting problem.
In our thesis, we will therefore investigate approaches that result in tree-like specifications that directly imply compatibility between the transformations, but with increased \emph{applicability} and \emph{modularity}. %of  try to achieve a tree of transformations, as this directly implies consistency between the transformations, but with increased modularity and with applicability.

\end{copiedFrom} % DocSym

\subsection{Property Effects}

Modularity, independence, correctness and comprehensibility directly induced by specific topologies
Universality not given for trees, has to be ensures for dense graphs (depends on orchestration strategy, see existing work)
Trade-off between Modularity/Independence (inherent in dense graph) and Correctness/Comprehensibility (inherent in tree)


Realizing consistency relations in transitive transformations can, for example, reduce comprehensibility, as a developer has to consider a set of transformations to understand a single consistency relation. 