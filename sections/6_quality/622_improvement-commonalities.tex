\section{The Commonalities Approach}
\label{chap:improvement:commonalities}

\subsection{Concept Metamodels}

\subsection{Composition of Concepts}

\subsection{Tree Topologies}

\subsection{Operationalization}


\begin{copiedFrom}{DocSym} % ABOUT TREES

The most crucial part of this approach is the necessity to build a tree of \glspl{CMM}.
This will not be possible if always considering whole metamodels and their relations, but can be possible if independent concepts are extracted to be treated individually.
For that, we will apply our findings on consistency relation decomposition (see \autoref{chap:improvement:approach}).
%This is similar to decomposing consistency relations, as introduced in \autoref{sec:approach:decomposition}, which is why we will apply our findings therefrom.
Additionally, the approach specifically aims to improve the specification of transformations for descriptive relations.
In consequence, it must be combined with transformations expressing the normative relations between \glspl{CMM} and other metamodels.

\end{copiedFrom} % DocSym


\begin{copiedFrom}{VoSE}

The state-of-the-art approach to keep models consistent automatically is the application of transformation languages.
If instances of multiple (i.e., more than two) metamodels are to be kept consistent, one can either use multidirectional transformation approaches, or compose bidirectional transformations to a network of transformations~\cite{cleve2019dagstuhl}.
% Following sentence moved to introduction
%Such a network can be regarded as a graph, formed by metamodels as its nodes and transformations as its edges.
When an instance of one metamodel is changed in such a network, the transformations are executed successively to propagate the change transitively across all models.
There are strategies to find one ordering of transformations to apply~\cite{stevens2020BidirectionalTransformationLarge-SoSym} and strategies to perform a fixpoint iteration until no further changes are conducted~\cite{klare2019icmt}.

% %Short introduction of transformation network, what nodes and edges are and so on.
% In this section, we propose a different approach for keeping two or more models consistent by specifying their common concepts rather than their direct consistency relations.
% This forms our contribution~\autoref{contrib:quality:improvement}.
% % \subsection{Defining Consistency Relations}
% % Different metamodels provide

% Consistency relations are usually defined declaratively (what constraints have to hold) or imperatively (how are constraints enforced) between two (or sometimes more) metamodels.
% This is a definition of when instances of that metamodels are considered consistent.
% Consistency constraints can be either \emph{declarative} or \emph{normative}. Declarative constraints exist (e.g. because they are somehow \enquote{natural} and have to be formalized to be checked or preserved by a tools. Normative constraints do not have to adhere to an existing notion of consistency and thus implicitly define what is considered consistent.
% In our running example, the constraints between UML and Java exist and only have to be specified in a declarative manner.
% On the other hand, the constraints between architecture and OO design (i.e. PCM and UML/Java) were normatively defined by \textcite{langhammerconstraints}. There is no single correct mapping between architecture and OO design but several possible.





\subsection*{Composing \commonalities}

We have explained how multiple metamodels can be kept consistent using one \conceptmetamodel.
This allows, theoretically, the definition of one large \conceptmetamodel that contains all \commonalities for all \concretemetamodels.
It would at first sight be similar to a \gls{SUMM}, as introduced by \textcite{atkinson2010a}.
However, it would be less complex than a \gls{SUMM}, which is able to express all information about the software system and thus contains the union of all \concretemetamodels.
%In fact, the resulting \conceptmetamodel would be comparable to a \summ according to \textcite{atkinson2010a}, which is able to express all information about the software system, whereas the instances of the \concretemetamodels would only serve as projectional views that do not provide further information.
Nevertheless, one large \conceptmetamodel would still become unmanageably large due to the fact that it had to contain the union of all pairwise intersections of the \concretemetamodels, as mentioned before.

\begin{figure}
    \centering
    \newcommand{\vdistance}{7.5em}
\newcommand{\hdistance}{11em}
\newcommand{\classwidth}{5.5em}
\newcommand{\labeldistance}{1.2em}
\newcommand{\labelshift}{0.3*\classwidth}
\newcommand{\representstext}{\emph{«manifests»}}
\newcommand{\mmborder}{0.9em}

\begin{tikzpicture}

\pgfdeclarelayer{bg}
\pgfsetlayers{bg,main}


% METACLASSES

\umlclassvarwidth{java_class}{}{Class}{
name\\
}{\classwidth}  

\umlclassvarwidth[, right=\hdistance of java_class.north, anchor=north]{uml_class}{}{Class}{
name\\
}{\classwidth} 

\umlclassvarwidth[, above right=\vdistance and 0.5*\hdistance of java_class.north, anchor=north]{oo_class}{}{Class\vphantom{p}}{
name\\
}{\classwidth} 

\umlclassvarwidth[, right=\hdistance of oo_class.north, anchor=north]{pcm_component}{}{Component}{
name\\
}{\classwidth} 

\umlclassvarwidth[, above right=\vdistance and 0.5*\hdistance of oo_class.north, anchor=north]{component_component}{}{Component}{
name\\
}{\classwidth}

% METAMODELS

\coordinate (java_label_coordinate) at ([xshift=-\labelshift,yshift=\labeldistance]java_class.north);
\node[mmlabel, anchor=center] (java_label) at (java_label_coordinate) {Java};

\coordinate (uml_label_coordinate) at ([xshift=\labelshift,yshift=\labeldistance]uml_class.north);
\node[mmlabel, anchor=center] (java_label) at (uml_label_coordinate) {UML};

\coordinate (oo_label_coordinate) at ([xshift=-4*\labelshift,yshift=\labeldistance]oo_class.north);
\node[mmlabel, anchor=west, align=left] (oo_label) at ([xshift=-0.7em, yshift=-0.8em]oo_label_coordinate) {Object-oriented\\ Design};

\coordinate (pcm_label_coordinate) at ([xshift=\labelshift,yshift=\labeldistance]pcm_component.north);
\node[mmlabel, anchor=center] (pcm_label) at (pcm_label_coordinate) {PCM};

\coordinate (component_label_coordinate) at ([yshift=\labeldistance]component_component.north);
\node[mmlabel, anchor=center] (component_label) at (component_label_coordinate) {Componend-based Design};

\begin{pgfonlayer}{bg}
    \node[mmbg, fit=(java_class)(java_label_coordinate), inner sep=\mmborder] (java) {};
    \node[mmbg, fit=(uml_class)(uml_label_coordinate), inner sep=\mmborder] (uml) {};
    \node[conceptmmbg, fit=(oo_class)(oo_label_coordinate), inner sep=\mmborder] (oo) {};
    \node[mmbg, fit=(pcm_component)(pcm_label_coordinate), inner sep=\mmborder] (pcm) {};
    \node[conceptmmbg, minimum width=12.5em, fit=(component_component)(component_label_coordinate), inner sep=\mmborder] (component) {};
\end{pgfonlayer}


% CONSISTENCY RELATIONS

\draw[directed consistency relation] (oo_class) -- node[above, sloped] {\representstext} (java_class);
\draw[directed consistency relation] (oo_class) -- node[above, sloped] {\representstext} (uml_class);
\draw[directed consistency relation] (component_component) -- node[above, sloped] {\representstext} (oo_class);
\draw[directed consistency relation] (component_component) -- node[above, sloped] {\representstext} (pcm_component);

\end{tikzpicture}

    \caption[Concept metamodels for the running example]{\Conceptmetamodels (dark) and their relations to \concretemetamodels (light) for the running example.}
    \label{fig:quality:composed_commonalities_example}
\end{figure}

To avoid the specification of such a monolithic \conceptmetamodel, we propose to compose \commonalities from different \conceptmetamodels.
Instead of having only \commonalities that relate to \metaclasses in \concretemetamodels, \commonalities may also have relations to other \commonalities.
Consider the \conceptmetamodel for component-based design in \autoref{fig:quality:composed_commonalities_example}.
It contains the \commonality \texttt{Component}, which is represented by an equally named \metaclass in \gls{PCM},
as well as in the \commonality \texttt{Class} in the \conceptmetamodel for object-oriented design, conforming to the relations proposed by \textcite{langhammer2015a}.
This induces a tree structure with \commonalities as inner nodes and \metaclasses of \concretemetamodels as leaves.
With such a composition structure, a \emph{«manifests»} relation may not only exist between a %concrete and a concept metamodel but also between two \conceptmetamodels.
\commonality of a \conceptmetamodel and a \metaclass in a \concretemetamodel but also between two \commonalities.
However, a concrete or \conceptmetamodel that is lower in the hierarchy is supposed to represent how a \metaclass or \commonality in the higher one manifests, which is why we call it a \emph{manifestation}.
%Due to that, we call concrete and \conceptmetamodels that contain \metaclasses or \commonalities with a \emph{«manifests»} relation to another \conceptmetamodel a \emph{manifestation} of that \conceptmetamodel.
For example, the object-oriented design \conceptmetamodel is a manifestation of the component-based design \conceptmetamodel.

\begin{figure}
    \centering
    \newcommand{\vdistance}{8em}
\newcommand{\hdistance}{11.5em}
\newcommand{\classwidth}{5.5em}
\newcommand{\smallclasswidth}{4em}
\newcommand{\labeldistance}{1.2em}
\newcommand{\labelshift}{0.3*\classwidth}
\newcommand{\representstext}{\emph{«manifests»}}
\newcommand{\mmborder}{1em}

\begin{tikzpicture}

\pgfdeclarelayer{bg}
\pgfsetlayers{bg,main}


% METACLASSES

\umlclassvarwidth{java_class}{}{Class\vphantom{p}}{
name\\
}{\smallclasswidth}  

\umlclassvarwidth[, right=0.65*\hdistance of java_class.north, anchor=north]{uml_class}{}{Class\vphantom{p}}{
name\\
}{\smallclasswidth}

\umlclassvarwidth[, right=1.3*\classwidth of uml_class.north, anchor=north]{uml_component}{}{Component}{
name\\
}{\classwidth} 

\umlclassvarwidth[, above right=\vdistance and 0.5*\hdistance of java_class.north, anchor=north]{oo_class}{}{Class\vphantom{p}}{
name\\
}{\classwidth} 

\umlclassvarwidth[, right=\hdistance of oo_class.north, anchor=north]{pcm_component}{}{Component}{
name\\
}{\classwidth} 

\umlclassvarwidth[, above right=\vdistance and 0.5*\hdistance of oo_class.north, anchor=north]{component_component}{}{Component}{
name\\
}{\classwidth}

% METAMODELS

\coordinate (java_label_coordinate) at ([xshift=-\labelshift,yshift=\labeldistance]java_class.north);
\node[mmlabel, anchor=center] (java_label) at (java_label_coordinate) {Java};

\coordinate (uml_label_coordinate) at ([xshift=\labelshift,yshift=\labeldistance]uml_component.north);
\node[mmlabel, anchor=center] (java_label) at (uml_label_coordinate) {UML};

\coordinate (oo_label_coordinate) at ([xshift=-4*\labelshift,yshift=\labeldistance]oo_class.north);
\node[mmlabel, anchor=west, align=left] (oo_label) at ([xshift=-0.7em, yshift=-0.8em]oo_label_coordinate) {Object-oriented\\ Design};

\coordinate (pcm_label_coordinate) at ([xshift=\labelshift,yshift=\labeldistance]pcm_component.north);
\node[mmlabel, anchor=center] (pcm_label) at (pcm_label_coordinate) {PCM};

\coordinate (component_label_coordinate) at ([yshift=\labeldistance]component_component.north);
\node[mmlabel, anchor=center] (component_label) at (component_label_coordinate) {Componend-based Design};

\begin{pgfonlayer}{bg}
    \node[mmbg, fit=(java_class)(java_label_coordinate), inner sep=\mmborder] (java) {};
    \node[mmbg, fit=(uml_class)(uml_component)(uml_label_coordinate), inner sep=\mmborder] (uml) {};
    \node[conceptmmbg, fit=(oo_class)(oo_label_coordinate), inner sep=\mmborder] (oo) {};
    \node[mmbg, fit=(pcm_component)(pcm_label_coordinate), inner sep=\mmborder] (pcm) {};
    \node[conceptmmbg, minimum width=12.5em, fit=(component_component)(component_label_coordinate), inner sep=\mmborder] (component) {};
\end{pgfonlayer}

\draw[-, color=gray] ($(uml_class.south east)!0.5!(uml_component.south west)-(0,\mmborder)$) -- ($(uml_class.north east)!0.5!(uml_component.north west)+(0,\mmborder+\labeldistance)$);


% CONSISTENCY RELATIONS

\draw[consistencyrel, <-] (java_class) -- node[above, sloped] {\representstext} (oo_class);
\draw[consistencyrel, <-] (uml_class) -- node[above, sloped] {\representstext} (oo_class);
\draw[consistencyrel, <-] (oo_class) -- node[above, sloped] {\representstext} (component_component);
\draw[consistencyrel, <-] (pcm_component) -- node[above, sloped] {\representstext} (component_component);
\draw[consistencyrel, <-] ([xshift=-1.5em]uml_component.north) -- node[below, sloped] {\representstext} (component_component);

\end{tikzpicture}

    \caption[Concept metamodels in the running example]{\Conceptmetamodels (dark) and \concretemetamodels (light) of the running example, extended by UML components, with their relations.}
    \label{fig:quality:extended_composed_commonalities_example}
\end{figure}

Our goal is to achieve a tree structure of commonalities. In the extended example in \autoref{fig:quality:extended_composed_commonalities_example}, a \texttt{Component} in the \conceptmetamodel for component-based design does not only manifest in a \gls{PCM} \texttt{Component} as well as a \texttt{Class} in object-oriented design, but also in \gls{UML}.
Since a \texttt{Class} in object-oriented design manifests both in Java and \gls{UML}, we do not have a tree structure of the induced relations between the metamodels anymore, due to \texttt{Class} and \texttt{Component} both being represented in \gls{UML}.
However, This still induces a tree structure between \metaclasses and \commonalities, with the \commonalities being inner nodes and \metaclasses of \concretemetamodels being leaves.

%
% How are Commonalities composed to keep multiple models consistent? What about overlaps?
% Apply to OO and PCM case
%
% \begin{figure}
%     \centering
% %    \includegraphics[width=\columnwidth]{figures/dag_example.pdf}\\
%     \newcommand{\vdistance}{4.8em}
\newcommand{\hdistance}{10.9em}
\newcommand{\mmwidth}{8.5em}
\newcommand{\mmheight}{3em}

\begin{tikzpicture}[
    mm/.style={draw, mmbg, minimum width=\mmwidth, minimum height=\mmheight, inner sep=0em}, %, font=\itshape},
    conceptmm/.style={mm, conceptmmbg},
]

% Metamodels

\node[mm] (java) {Java};
\node[mm, draw=gray!70, minimum height=0.5*\mmheight, right=\hdistance of java.south, anchor=south] (uml) {UML};
\node[mm, draw=gray!70, minimum width=0.5*\mmwidth, minimum height=0.5*\mmheight, above=0.5*\mmheight of uml.west, anchor=west] (uml_class) {\emph{Class}};
\node[mm, draw=gray!70, minimum width=0.5*\mmwidth, minimum height=0.5*\mmheight, above=0.5*\mmheight of uml.east, anchor=east] (uml_component) {\emph{Comp}};
\node[mm, fill=none, right=\hdistance of java.north, anchor=north] {}; % Redraw thick border of UML
\node[conceptmm, above right=\vdistance and 0.5*\hdistance of java.north, anchor=north, align=center] (oo) {Object-oriented\\ Design};
\node[mm, right=\hdistance of oo.north, anchor=north] (pcm) {PCM};
\node[conceptmm, above right=\vdistance and 0.5*\hdistance of oo.north, anchor=north, align=center] (component) {Component-based\\ Design};


% CONSISTENCY RELATIONS

\draw[consistencyrel, <-] (java) -- (oo);
\draw[consistencyrel, <-] (uml_class) -- (oo);
\draw[consistencyrel, <-] (oo) -- (component);
\draw[consistencyrel, <-] ([xshift=-0.15*\mmwidth]uml_component.north) -- (component);
\draw[consistencyrel, <-] (pcm) -- (component);


\end{tikzpicture}
 %\\[1em]
%     %\input{figures/dag_example_alternative.tex}
%     \caption{\Concretemetamodels (light) and \conceptmetamodels (dark) of the running example forming a \acs{DAG}}
%     \label{fig:quality:dag_example}
% \end{figure}

%Since we only assume a tree of \commonalities,  rather than a tree of \conceptmetamodels, 
A metamodel may have several \commonalities in different \conceptmetamodels with different other metamodels.
For example, in \autoref{fig:quality:composed_commonalities_example}, the UML metamodel contains a \texttt{Class} and a \texttt{Component} \metaclass, which have two different \commonalities in two different \conceptmetamodels.

%We currently assume that all concept metamodels and relations to their manifestations can be represented as a tree.
% In general, %, this will not be possible, 
% the assumption of having a tree of \conceptmetamodels may not be satisfiable if
% metamodels have several \commonalities with different metamodels.
% In fact, we can relax that assumption: 
% The essential requirement is that each change is only propagated across one path between two models, which is inherently given in a tree.
% If the \commonalities of two \conceptmetamodels manifest in disjoint sets of elements of the same manifestation, that manifestation can be---virtually---separated into two metamodels, for which the network forms a tree again.
% In consequence, if we consider the transformations as directed edges from a \conceptmetamodel to its manifestations, we only have to require a \ac{DAG}, % induced by these edges, 
% as long as the manifestations of all \commonalities are disjoint.
% %If we consider the transformations as directed edges from a \conceptmetamodel to its manifestations, we only have to require a \ac{DAG} induced by these edges, as long as the \conceptmetamodels represent disjoint sets of elements of their manifestations.
% %This relaxation is possible due to the fact that the essential requirement is to have only one path between two models across which a change can be propagated.
% %If different \conceptmetamodels relate to disjoint parts of their manifestations, these manifestations can be virtually separated into different metamodels, which then form a tree again.
% \autoref{fig:dag_example} exemplifies this relaxation: % on our running example:
% If the UML is also considered a manifestation of component-based design by providing the \texttt{Component} \metaclass, the network does not constitute a tree because of the relations between the UML and the \conceptmetamodels for component-based design and object-oriented design.
% However, it forms a \ac{DAG} and the redundant paths to the UML metamodel are unproblematic, because the relation to object-oriented design affects the part of the UML metamodel considering classes (\emph{Class}), whereas the relation to component-based design affects the part of the UML metamodel considering components (\emph{Comp}). 
% The elements in these parts of the UML metamodel are disjoint.

\subsection*{Transformation Operationalization}
\label{chap:commonalities:approach:operationalization}

To actually keep models consistent, the specification of a hierarchy of \conceptmetamodels has to be operationalized.
Two options for operationalization can be distinguished:

\begin{description}[leftmargin=\parindent]
    \item[\Conceptmetamodels as additional metamodels:] The \conceptmetamodels are actually instantiated and the transformations are executed as they are defined between the \conceptmetamodels and their manifestations. In consequence, instances of the \conceptmetamodels have to be maintained.
    \item[Transformations between \concretemetamodels:] The \conceptmetamodels and the relations between them and their manifestations are used to derive bidirectional transformations between the \concretemetamodels. For example, from the \conceptmetamodel for object-oriented design in \autoref{fig:improvement:commonalities_example}, a bidirectional transformation between Java and UML is derived.
\end{description}

A drawback of the first option is that additional models have to be managed and persisted. 
In consequence, the user has to version these models although they should be transparent to him or her, as long as no appropriate framework abstracts from such tasks.
%Additionally, it is not easily possible to derive the direct relations between two concrete metamodels from such a specification. For example, the direct relation between classes in UML and Java is not accessible but only implicitly expressed by the transitive relation across the concept metamodel for object-oriented design.
A drawback of the second option is that the types of supported relations that can be described in the transformations are limited.
First, only relations may be defined that can be composed with any other relation, such that a direct transformation between two metamodels can be derived.
Second, it is possible to define $n$-ary relations between more than two metamodels that cannot be decomposed into binary relations between them, but only into $n$ binary relations between those metamodels and an additional one~\cite{stevens2020BidirectionalTransformationLarge-SoSym}.
In consequence, the first option provides higher expressiveness.

While the first option can be realized without an additional language by just defining the \conceptmetamodels and the transformations with existing languages, the second option requires a mechanism that generates the transformations between the \concretemetamodels from those between the \conceptmetamodels and their manifestations.

\end{copiedFrom} % VoSE