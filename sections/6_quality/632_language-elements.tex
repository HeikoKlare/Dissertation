\section{The \Commonalities Language}

\subsection{Elements Overview}
UML diagram or the like

\begin{figure}
    \centering
    \newcommand{\hdistance}{7.6em}
\newcommand{\classwidth}{5.5em}
\newcommand{\vdistance}{5em}

\begin{tikzpicture}[
    existing/.style={fill=lightgray!20}
]

\umlclassvarwidth[, existing]{metamodel}{}{Metamodel}{
name : String\\
}{\classwidth} 

\umlclassvarwidth[, existing, right=\hdistance of metamodel.north, anchor=north]{class}{}{Class}{
name : String\\
}{\classwidth}  

\umlclassvarwidth[, existing, right=\hdistance of class.north, anchor=north]{reference}{}{Reference}{
name : String\\
}{\classwidth}

\umlclassvarwidth[, existing, right=\hdistance of reference.north, anchor=north]{attribute}{}{Attribute}{
name : String\\
}{\classwidth}

\umlclassvarwidth[, below=\vdistance of metamodel.north, anchor=north]{concept}{}{Concept}{
}{\classwidth}  

\umlclassvarwidth[, right=\hdistance of concept.north, anchor=north]{commonality}{}{Commonality}{
root : Boolean\\
}{\classwidth}

\umlclassvarwidth[, below=\vdistance of commonality.north, anchor=north]{manifestation}{}{Manifestation}{
}{\classwidth} 

\umlclassvarwidth[, right=\hdistance of manifestation.north, anchor=north]{commonality_reference}{}{Commonality\\Reference}{
}{\classwidth}

\umlclassvarwidth[, right=\hdistance of commonality_reference.north, anchor=north]{commonality_attribute}{}{Commonality\\Attribute}{
}{\classwidth}


\umlclassvarwidth[, below left=\vdistance and \hdistance of manifestation.north, anchor=north]{manifestation_class}{}{Manifestation\\Class}{
alias : String\\
single : Boolean\\
}{\classwidth}

\umlclassvarwidth[, below=\vdistance of manifestation.north, anchor=north]{manifestation_condition}{}{Manifestation\\Condition}{
dir : Direction\\
}{\classwidth}

\umlclassvarwidth[, below=\vdistance of commonality_attribute.north, anchor=north]{attribute_relation}{}{Attribute\\Relation}{
dir : Direction\\
}{\classwidth} 

\umlclassvarwidth[, below=\vdistance of commonality_reference.north, anchor=north]{reference_relation}{}{Reference\\Relation}{
dir : Direction\\
}{\classwidth} 

\umlclassvarwidth[, below=\vdistance of manifestation_condition.north, anchor=north]{manifestation_operator}{}{Manifestation\\Operator}{
}{\classwidth}

\umlclassvarwidth[, below=\vdistance of attribute_relation.north, anchor=north]{attribute_operator}{}{Attribute\\Operator}{
}{\classwidth} 

\umlclassvarwidth[, below=\vdistance of reference_relation.north, anchor=north]{reference_operator}{}{Reference\\Operator}{
}{\classwidth} 

\umlclassvarwidth[, below=\vdistance of reference_operator.north, anchor=north]{operand}{}{\textit{Operand}}{
}{\classwidth}

\umlclassvarwidth[, below left=\vdistance and \hdistance of operand.north, anchor=north]{manifestation_operand}{}{\textit{Manifestation}\\\textit{Operand}}{
}{\classwidth}

\umlclassvarwidth[, below right=\vdistance and 0em of operand.north, anchor=north]{literal_operand}{}{Literal\\Operand}{
}{\classwidth}

\umlclassvarwidth[, right=\hdistance of literal_operand.center, anchor=center]{direction}{}{\umlenumlabel\\Direction}{
\itshape
Bidirectional\\
Checkonly\\
Enforce\\
}{\classwidth}

\umlclassvarwidth[, below left=\vdistance and \hdistance of manifestation_operand.north, anchor=north]{manifestation_class_operand}{}{Manifestation\\Class\\Operand}{
}{\classwidth}

\umlclassvarwidth[, below left=\vdistance and 0em of manifestation_operand.north, anchor=north]{manifestation_attribute_operand}{}{Manifestation\\Attribute\\Operand}{
}{\classwidth}

\umlclassvarwidth[, below right=\vdistance and \hdistance of manifestation_operand.north, anchor=north]{manifestation_reference_operand}{}{Manifestation\\Reference\\Operand}{
}{\classwidth}


% INHERITANCE
\umlsubclassof{concept}{--}{metamodel}
\umlsubclassof{commonality}{--}{class}
\umlsubclassof{commonality_attribute}{--}{attribute}
\umlsubclassof{commonality_reference}{--}{reference}

\umlsubclassof{manifestation_operand}{|- ([yshift=-1.5em]operand.south) --}{operand}
\draw (literal_operand) |- ([yshift=-1.5em]operand.south);

\umlsubclassof{manifestation_class_operand}{|- ($(manifestation_attribute_operand.north)!0.5!(manifestation_operand.south)$) --}{manifestation_operand}
\draw (manifestation_attribute_operand) -- ($(manifestation_attribute_operand.north)!0.5!(manifestation_operand.south)$);
\draw (manifestation_reference_operand) |- ($(manifestation_attribute_operand.north)!0.5!(manifestation_operand.south)$);

% REFERENCES
%\umlassociationfromto{(class) -- node[uml cardinality end, pos=1, above right] {1} (metamodel)}
\umlcomposition{(metamodel.east) -- node[uml cardinality end, pos=1, above left] {*} (class.west|-metamodel.west)}
\umlcomposition{(concept.east) -- node[uml cardinality end, pos=1, above left] {*} (commonality.west|-concept.west)}
\umlcomposition{(class.east) -- node[uml cardinality end, pos=1, above left] {*} (reference.west|-class.east)}
\umlcomposition{([xshift=1em]class.north) -- ++(0,1em) -| node[uml cardinality end, pos=1, above right] {*} (attribute.north)}

\umlcomposition{(commonality) -- node[uml cardinality end, pos=1, above right] {*} (manifestation)}
\umlcomposition{([xshift=-1em]commonality.south east) -- ++(0,-1em) -|node[uml cardinality end, pos=1, above right] {*} ([xshift=1em]commonality_reference.north west)}
\umlcomposition{([yshift=0.5em]commonality.east) -| node[uml cardinality end, pos=1, above left] {*} ([xshift=-1.5em]commonality_attribute.north)}

\umlassociationfromto{([xshift=-1em]commonality_reference.north) |- node[uml cardinality end, pos=1, above right] {1} node[uml role end, pos=1, below right] {type} ([yshift=-1em]commonality.east)}

\umlcomposition{([xshift=1em]manifestation.south west) -- ++(0,-1.5em) -| node[uml cardinality end, pos=1, above left] {1..*} ([xshift=-1em]manifestation_class.north east)}
\umlcomposition{(manifestation) -- node[uml cardinality end, pos=1, above left] {*} (manifestation_condition)}
\umlcomposition{(commonality_attribute) -- node[uml cardinality end, pos=1, above left] {*} (attribute_relation)}
\umlcomposition{(commonality_reference) -- node[uml cardinality end, pos=1, above left] {*} (reference_relation)}

\umlaggregation{(manifestation_condition) -- node[uml cardinality end, pos=1, above left] {*} (manifestation_operator)}
\umlaggregation{(attribute_relation) -- node[uml cardinality end, pos=1, above left] {*} (attribute_operator)}
\umlaggregation{(reference_relation) -- node[uml cardinality end, pos=1, above left] {*} (reference_operator)}

\umlaggregation{([xshift=1em]manifestation_class.north west) |- node[uml cardinality start, pos=0, above right] {*} ++(-1.5em,\vdistance) |- ([xshift=-1em,yshift=1em]class.north) -- node[uml cardinality end, pos=1, above left] {1} ([xshift=-1em]class.north)}

\umlcomposition{(attribute_relation.west) -- ++(-0.8em,0) |- node[uml cardinality end, pos=1, below right] {*} (operand.east)}
\umlcomposition{(reference_relation.west) -- ++(-0.8em,0) |- node[uml cardinality end, pos=1, above left] {*} ([yshift=0.8em]operand.west)}
\umlcomposition{([yshift=-0.5em]manifestation_condition.west) -- ++(-0.8em,0) |- node[uml cardinality end, pos=1, above left] {*} node[uml cardinality end, pos=1, below left] {rightOperand} ([yshift=-0.8em]operand.west)}
\umlcomposition{([yshift=0.5em]manifestation_condition.west) -- ++(-1.1em,0) |- node[uml cardinality end, pos=1, above left] {*} node[uml cardinality end, pos=1, below left] {leftOperand} ([yshift=0.8em]manifestation_operand.west)}

\umlassociationfromto{([yshift=-0.8em]manifestation_operand.west) -| node[uml cardinality end, pos=1, below right] {1} ([xshift=-1em]manifestation_class.south)}
\umlassociationfromto{(manifestation_attribute_operand.south) -- ++(0,-0.8em) -| ([xshift=0.8em, yshift=-1.2em]attribute.south east) -| node[uml cardinality end, pos=1, below right] {1} ([xshift=1.5em]attribute.south)}
\umlassociationfromto{(manifestation_reference_operand.east) -| ([xshift=0.4em, yshift=-2em]attribute.south east) -| node[uml cardinality end, pos=1, below right] {1} ([xshift=1.5em]reference.south)}

\end{tikzpicture}
    \caption[\commonalities language elements]{Class diagram with the essential elements of the \commonalities language and their relations. Elements that exist independent from the language are depicted in the top row.}
    \label{fig:language:elements}
\end{figure}

Additional consolidation of Relations (manifestation Condition, Reference/Attribute Relation), of Operator and of Named elements possible

\subsection{Concept Metamodels}

\subsection{Commonalities and Manifestations}

Participations and participation conditions
call them participations or manifestation?

\subsection{Properties and Operators}
Implemented operators:
* Attribute mappings operators: Mappings between attributes
* Attribute reference operators: Mappings between attributes that serve as references, e.g., subpackages in UML and concept metamodel as explicit references whereas encoded as namespace attributes in classes in Java -> namespace attribute of Java class is mapped to package structure in UML
* Condition operators

References realized by participations and commonalities referencing other commonalities

Reuse mechanisms, libraries of operators

Discuss that in fact this is comparable to declarative transformation languages, esp. Mappings, but also QVT-R, so no deeper discussion.


\subsection{Textual Syntax}

Grammar or maybe only examples, due to not focusing on concrete language and complexity of the actual language.



\begin{copiedFrom}{VoSE}

\section*{Language Description}

As introduced before, our realization of the \commonalities language
%for the previously explained \commonalities approach
provides an internal concept definition and uses the \conceptmetamodels as additional metamodels in the operationalization.
An example for the syntax of the \commonalities language is depicted in \autoref{lst:quality:commonalities_language_example}.

% \begin{figure}
%     \centering
%     \todo{Potentially extend running example so that references are covered. At least, we do not have the package in the example}
%     \includegraphics[width=\columnwidth]{figures/commonalities_language_example.PNG}
%     \caption{An example for defining the common concept of components}
%     \label{fig:commonalities_language_example}
% \end{figure}

\lstinputlisting[language=commonalities, float, belowskip=-0.8 \baselineskip,
    caption={[Exemplary commonality for components]An exemplary specification of the \texttt{Component} \commonality between \gls{PCM}, UML and the object-oriented design concept in the \commonalities language},
    captionpos=b,
    label=lst:quality:commonalities_language_example,
]{listings/quality/commonalities_language_example.lst}

The language allows to define \conceptmetamodels by declaring \commonalities, each representing one commonality between different manifestations, such as the \texttt{Component} \commonality in our example.
Relations between the \conceptmetamodels and their manifestations are supposed to be specified \emph{declaratively}.
%, which is realized as a \metaclass in the \conceptmetamodel.
For every \commonality, the \metaclasses in the manifestations that realize them are specified.
In the example, the \texttt{Component} in \gls{PCM} and the \texttt{Class} in the object-oriented design \conceptmetamodel are %defined as representations of
related to the \texttt{Component} \commonality.
In our language, a \commonality is realized by a \metaclass in the metamodel that is generated for a concept, so the \texttt{Component} \commonality is realized by a \texttt{Component} \metaclass.

Within a \commonality, attributes and references can be defined, similar to an ordinary \metaclass.
The relations of an attribute to the manifestation are declared directly at the attribute.
In the example, a \texttt{name} attribute is specified, which maps to the name of the component in \gls{PCM} and the name appended with an \enquote{Impl} suffix in Java.
The language %defined by \textcite{gleitze2017a} 
provides several operators for attribute relations, apart from equality relations.
The example depicts a prefix operator that allows to compose a String attribute.
Such operators can be defined independently and added to the language dynamically.
References can be defined comparably to attributes but can be enriched with a definition of containment relations.

% Introduce the idea of \enquote{Concepts} and \enquote{Commonalities}, explain how attributes and references are mapped.
% \todo{We have to align the definition of Concept, Commonality etc. in the paper with the implementation}

The actually conceptualized and implemented language by \textcite{gleitze2017a} is far more sophisticated than the simple overview we provide here. 
It supports different kinds of bidirectional operators for attribute mappings, containment specifications (so-called \emph{participations}), attribute checks as preconditions for \commonality instantiation, and more.

\end{copiedFrom} % VoSE
