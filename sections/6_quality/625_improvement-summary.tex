\section{Summary}

In this section, we have discussed how the insights regarding effects of different network topologies on the quality properties of a transformation network can be used to mitigate trade-offs between them.
We have motivated a different way of considering consistency in terms of making common concepts explicit as \emph{commonalities} instead of implicitly encoding them into consistency relations.
We have used this way of specifying consistency to propose a construction approach for transformation networks that results in a tree topology providing inherent benefits regarding correctness, but also provide high reusability due to the actual metamodels, whose instances are used to describe a system, being only leaves of the tree induces by the transformation network.
We conclude this chapter with the following central insight.

\begin{insight}[Trade-off Mitigation]
    Quality properties of transformation networks are influences by the network's topology.
    Especially correctness and reusability are contrary properties, which induce a trade-off depending on whether the network topology is rather a dense or a sparse graph.
    The drawback regarding reusability in networks with tree topology arises from the fact that the metamodels represented by the inner nodes of the tree cannot be easily omitted, as consistency between several other metamodels is expressed across them.
    This can be mitigated by ensuring that the metamodels represented by the inner nodes are auxiliary artifacts and not the actual metamodels used by developers.
    This matches with a different way of thinking about consistency in terms of making the commonalities between metamodels to keep consistent explicit in addition metamodels rather than encoding them implicitly in consistency relations.
    Following such a specification approach leads to a network that improves both correctness and reusability, which are contradictory if only considering transformations between the metamodels whose instances are actually used by developers.
    Such an approach can even be used to define consistency partially for some of the metamodels and then combine it with other consistency specifications, such as ordinary transformations.
    To still have the same guarantees regarding correctness and reusability, such a specification can be encapsulated behind views, which provide projections of the information within the actual models and do only allow one of them to be updated at a time.
\end{insight}