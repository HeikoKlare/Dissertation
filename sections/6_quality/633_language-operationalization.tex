\section{Operationalization of Transformations}
\label{chap:language:operationalization}

\mnote{Operationalization according to \mappings language}
In the previous section, we have discussed relevant elements for the \commonalities language and their semantics.
We have also sketched an exemplary textual syntax, we which have also used for the prototypical implementation of the language.
In \autoref{chap:language:design:artifacts_process}, we have already depicted that such a specification needs to be compiled to \conceptmetamodels and transformations between them an existing \concretemetamodels to be used as an ordinary transformation network.
Since the semantics of relations defined between a \commonality and its manifestations is analogous to the semantics of bidirectional relations defined in the \mappings language~\cite[Chap. 7]{kramer2017a}, we refer to that detailed discussion for the operationalization of \commonalities specifications to transformations.

\mnote{Essential compiler responsibilities}
We still discuss essential responsibilities of the compiler process.
First, we consider the compilation to the \reactionslanguage, which is part of the \vitruv framework, implements delta-based consistency preservation rules according to our formalism in \autoref{chap:correctness} and is the same language to which artifacts of the \mappings language compile.
Then, we discuss how the scenarios of \commonality instantiation and deletion by means of matching manifestations, and of updates by means of operator execution are realized.


\subsection{Compiling to \reactions}

\mnote{Suitability of \reactions}
The operationalization of \commonalities specifications requires the derivation of transformations and, in particular, their consistency preservation rules according to \autoref{def:consistencypreservationrule}.
Thus, we need to derive rules that instantiate, delete or update \commonalities after changes to a manifestation, such that they are again consistent to the consistency relations implied by the manifestation relations defined in the \commonalities specification, and vice versa.
The \reactionslanguage (see \autoref{chap:foundations:transformations:reactions}) allows the definition of \reactions and routines that restore consistency after changes.
Each \reaction defines the type of change it reacts to and executes routines, which identify whether the consistency relation to which they preserve consistency is violated by that change and then execute actions to restore it.
Since that kind of specification fits to the goals of the operationalization of \commonalities specifications, we described the operationalization to \reactions and have also implemented it in our prototype.
An analogous operationalization has been developed for the \mapping language by \textcite[Sec. 7.7]{kramer2017a}.

\mnote{\Reactions for each change type}
The operationalization of \commonalities to \reactions requires that a \reaction is created for each change that may require the instantiation, deletion or update of a \commonality.
Thus, for each \metaclass and each feature references within a \commonality, a \reaction for their changes has to be created.
We discuss the details for the different scenarios of instantiation, deletion and update in the subsequent subsections.

\mnote{Benefits of \reactions}
A benefit of compiling to \reactions is that they have well-defined semantics~\cite[Sec. 6.7]{kramer2017a} and that they are proven complete and correct~\cite[Sec. 9.2.4 and 9.3]{kramer2017a}.
This means that they are able to preserve consistency according to any possible consistency relation and that their execution actually actually preserves consistency to the consistency relation that is implied by the specified consistency preservation rule.
Thus, the transformation language with which the manifestation relations of \commonalities are operationalized does especially not restrict expressiveness in any way.


\subsection{Matching Manifestations}

Conforms to instantiation (/deletion)

\cite[Sec. 7.7.4]{kramer2017a}.
Instantiation: \cite[Alg. 1]{kramer2017a}
Deletion: \cite[Alg. 2]{kramer2017a}
Feature update: \cite[Alg. 3]{kramer2017a}


\subsection{Executing Operators}

Conforms to update

Operators are executed as soon as one of the used values (parameters) is changed

Reuse mechanisms, libraries of operators

