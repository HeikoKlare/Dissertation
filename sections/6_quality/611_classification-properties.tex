\section{Properties of Transformation Networks}
\label{chap:classification:properties}

\mnote{Further functional properties}
The most essential property of transformation networks, which we have also considered in the last chapters, is correctness, or more precisely \emph{functional correctness} according to ISO standard 25010~\cite[p.~11]{iso25010}.
In addition to its correctness, functionality can be regarded in terms of \emph{completeness} and \emph{appropriateness}~\cite[p.~11]{iso25010}.
While completeness concerns the degree to which functions cover all intended objectives, appropriateness is the degree to which functions facilitate the conduction of tasks to achieve the intended objectives.
In terms of a transformation network, completeness represents whether the network is able to preserve all consistency relations, which requires transformations for all existing relations to keep consistent to be defined.
Since appropriateness especially concerns manual effort, it is not as relevant in a fully automated process. Appropriateness would especially be of interest if the user is involved in the consistency preservation process by clarifying its intent or making necessary decisions to adapt models for being consistent, which can influence how far the automation facilitates the process of consistency preservation.
Thus, in addition to functional correctness, we also discuss functional completeness as a relevant property and relate it to our requirement of \emph{universality}, as defined in \autoref{chap:introduction}.

\mnote{Quality properties regarding maintainability}
In our work, we focus on properties of transformation networks that are relevant for their developers (see \autoref{chap:introduction:objective:benefits}).
Thus, in addition to functional properties of such networks, we especially consider properties regarding their \emph{maintainability}~\cite[Tab.~2]{iso25010}, which describe the \enquote{degree of effectiveness and efficiency with which a product or system can be modified by the intended maintainers}~\cite[p.~14]{iso25010}.
Maintainability includes the properties \emph{modularity}, \emph{reusability}, \emph{analyzability}, \emph{modifiability}, and \emph{testability}~\cite[pp.~14]{iso25010}.
We have already covered the former two properties of modularity and reusability implicitly in our assumption of \emph{modular reuse} as well as analyzability in the goal of \emph{comprehensibility}.
In previous work~\owncite{klare2018docsym}, we have also discussed properties of transformation networks but without basing them on a common understanding defined by the mentioned ISO standard.


\subsection{Correctness}

\mnote{Central importance of correctness}
According to ISO standard 25010~\cite{iso25010}, functional correctness denotes to which degree a system, in our case a transformation network, provides correct results.
We have intensively discussed this property in the previous chapters, starting with a definition of \emph{correct results} in \autoref{chap:correctness} and discussing how to achieve transformation networks that fulfill such a correctness notion in \autoref{chap:compatibility} to \autoref{chap:orchestration}.
Thus, we do not discuss this property again but emphasize its central importance for a transformation network to be useful, as an incorrect transformation network leading to models of a system description that are inconsistent will hardly provide relevant benefits.


\subsection{Completeness}

\mnote{Universality}
According to ISO standard 25010~\cite{iso25010}, functional completeness describes to which degree provided functions cover all objectives.
Applied to transformation networks, this means to which degree such a network can preserve consistency according to consistency relations, be they explicitly defined or only intended by a transformation developer.
Completeness of the individual transformations as well as of the transformations are both covered by their notions of correctness (see \autoref{def:synchronizingtransformationcorrectness} and \autoref{def:transformationnetworkcorrectness}).
It does, however, assume an even broader notion of what we introduced as \emph{universality} in \autoref{chap:introduction}.
While we have introduced universality as the ability to process transformation networks of arbitrary topology, an even broader notion would require the applicability of transformation network to every project in which artifacts need to be kept consistent.
Thus, it would first require that the artifacts to keep consistent are represented in a form that is required to define transformations between them.
More precisely, the artifacts to keep consistent need to conform to some kind of modeling formalism, such as the one we have proposed in \autoref{chap:networks:models} based on the \gls{EMOF} standard~\cite{mof}.

\mnote{Non-modeled artifacts}
If the artifacts or, more generally, the models to keep consistent are not represented in a format conforming to such a modeling formalism, a metamodel for them needs to be defined, and their representation may need to be transformed into an instance of such a metamodel.
This is especially the case for proprietary tools that do not use a common format to represent their artifacts.
For many popular tools, however, metamodels based on the \gls{EMOF} or Ecore have already been reverse-engineered, such as 
MATLAB/Simulink~\cite{heinzemann2013Reconfiguration-CBSE,son2012Simulink-CAGGIE,armengaud2011Safety-WCE}.
In addition, the \gls{EMF} as a popular modeling framework provides an importer for \gls{XML}-based specifications of metamodels~\cite[pp.~86]{steinberg2009emf}.
Tools, especially from engineering domains, often provide \gls{XML}-based representations of their artifacts, such as the electronic circuit design tool EPLAN~\cite{eplan} or the exchange format for automation system design in AutomationML~\cite{automationML}.
Defining a metamodel for a specific modeling formalism, such as Ecore, and representing artifacts as models of it is always necessary when modeling tools for that formalism shall be applied, for which transformations are only one example.
Frameworks for generating graphical editors or model analyses could be further tools to be applied~\owncite{klare2017models}.
Thus, such an integration of artifacts into model-driven processes is part of separate research.

\mnote{Integration approach}
In our research, we have also developed and proposed such an approach to integrate artifacts into model-driven processes~\owncite{klare2017models, klare2019modelsward}.
It is based on the insight that code often contains models implicitly.
The tools, whose artifacts we want to keep consistent, usually have definitions of metamodels of their artifacts defined within their source code, but they are only represented as a simple structure of classes instead of am explicit metamodel according to some modeling formalisms.
For example, Java graph libraries need to contain a metamodel for representing graphs, but this is usually just represented by a set of classes and not an explicit metamodel according to some modeling framework.
This also applies to programming languages, for which parsers contain metamodels for their \gls{AST} representations.
We have proposed an approach that makes these implicit metamodels explicit to apply modeling tools, such as transformations for consistency preservation, to them~\owncite{klare2017models, klare2019modelsward}.
Since this topic and especially the proposed approach is important for applying transformation networks but also has further, broader application areas, we do not further discuss it in this thesis but refer to our previous work for details about it.


\subsection{Maintainability}

\mnote{Influencing factors}
We have identified maintainability as a dimension of quality properties with central importance for developers of transformations and transformation networks.
According to ISO standard 25010~\cite{iso25010}, maintainability includes modularity, reusability, analyzability, modifiability, and testability.
We discuss for these properties how they manifest in transformation networks and especially how they are related to each other.
We do not aim to measure these properties, which is why we do not propose specific metrics for them.
For source code, it has been shown that it is hard to assess its quality, e.g., to measure modifiability in terms of a correlation with the number of defects~\cite{Gyimothy2005ValidationMetrics-TSE, Yu2002FaultPrediction-ECSMR}, and that only few metrics provide a correlation to, for example, the number of defects.
We only aim at identifying the influencing factors for these properties instead of a measure for them anyway, especially with respect to topologies of the transformation network.

\begin{properdescription}
    \item[Modularity:] 
    Modularity is the degree to which a program, and thus also a transformation network, is composed of components such that changes only influence a part of it~\cite[p.~14]{iso25010}.
    This property degrades when having multiple paths of transformations expressing the same consistency relations, as then these paths depend on each other and may be contradictory. 
    Having such multiple paths can lead to incompatibilities (see \autoref{chap:compatibility}) or situations in which no consistent orchestration of the transformations exists (see \autoref{chap:orchestration}), and thus degrade modularity.
    \item[Reusability:]
    Reusability is the degree to which assets, such as the single transformations of a transformation network, can be used in more than one system~\cite[p.~15]{iso25010}.
    In terms of a transformation network, reusability of a transformation is given if it is independent from the other transformations and can be used together with others in a different context.
    This conforms to our notion of independent development and modular reuse, given as assumptions in \autoref{chap:introduction}.
    Reusability profits from having all relations between the involved metamodels expressed explicitly, i.e., directly between each pair of metamodels and not only transitively across others.
    This leads to multiple expressions of the same relations transitively across different paths of transformations, but it allows subsets of the transformations to be used in a different context in which only a subset of the metamodels is used.
    For example, having the relation between \gls{PCM} and Java expressed directly instead of only expressing it transitively across the \gls{UML} enables its reuse in other system development scenarios in which the \gls{UML} is not used at all.
    Thus, reusability degrades when modularity improves.
    \item[Analyzability:] 
    Analyzability is the degree to which the impact of a change can be assessed effectively and efficiently or to which defects can be identified afterwards~\cite[p.~15]{iso25010}.
    On the one hand, this is important for the single transformations, as analyzing the impact of a change especially concerns the intended change of the behavior of a transformation. That is, however, also a topic of dedicated research about transformation validation and verification~\cite{cabot2010VerificationInvariants-JSS, rahim2015SurveyTransformationVerification-SoSym, azizi2017ContractVerification-ICCKE, vallecillo2012FormalTesting-FMMDE}.
    On the other hand, this is important for the interplay of transformations, thus how a change to one transformation affects interoperability with the others.
    This is, again, directly related to the existence of multiple paths of transformations preserving consistency to the same relations, as it influences how many other transformations may be affected and potentially need to be updated due to the modification to one of them.
    Consequently, the more relations are preserved across multiple paths of transformations, the more transformations may be affected by a single change and introduce interoperability problems that may be hard to analyze (see \autoref{chap:orchestration}).
    Analyzability is also related to the notion of comprehensibility that we have introduced in \autoref{chap:introduction}. The lower analyzability is, the harder it becomes for a transformation developer to comprehend what the combination of transformations actually does, how an intended change can be performed, and what its impact is.
    We have also used comprehensibility to motivate the design of our orchestration algorithm in \autoref{chap:orchestration:algorithm}, which is driven by the goal of easing the analysis of failures of the transformation network, analogous to analyzability.
    Thus, analyzability improves with modularity.
    \item[Modifiability:]
    Modifiability is the degree to which a system can be modified without introducing defects or degrading quality~\cite[p.~15]{iso25010}.
    It is directly influenced by modularity and analyzability, as also stated by the ISO standard~\cite[p.~15]{iso25010}.
    In terms of a transformation network, this can include the adaptation of existing transformations or the extension of an existing network with further metamodels and transformations.
    The same arguments as for modularity and analyzability apply, and thus modifiability improves and degrades with modularity of the transformation network.
    For example, the complexity of adding a new transformation, which is covered by modifiability, depends on the number of transformations that already, and in particular transitively, preserve relations between the two metamodels related by the new transformation.
    \item[Testability:]
    Testability is the degree with which test criteria can be effectively and efficiently established and evaluated by test cases for a product~\cite[p.~15]{iso25010}, such as a transformation network.
    While there are many influencing factors for testability, such as encapsulation and coupling within the implementation, it is, again, also influenced by the number of transformation paths across which consistency relations are preserved.
    The more paths of transformations preserving the same consistency relations exist, the larger is the set of models to be considered and transformations to be executed for testing correctness of preserving consistency according to a certain relation.
    This increases complexity of the tests to perform.
    Testability is also highly related to the notion of comprehensibility that we have introduced in \autoref{chap:introduction}, as we have also discussed for analyzability. 
    The higher the number of transformations that need to be executed to detect a failure, the more complex we can expect the process of identifying the causing mistake to be (see \autoref{chap:errors}).
    Testability, just like analyzability and modifiability, thus improves with modularity.
\end{properdescription}

\mnote{Opposed properties}
The discussion shows that the existence of multiple paths of transformations preserving consistency to the same consistency relations reduces modularity, modifiability, analyzability, and testability, while it improves reusability.
This is because multiple representations of the same consistency relations induce dependencies, which reduce modularity, and can contain conflicts, which reduces modifiability.
The increased complexity reduces analyzability and testability.
Reusability is, however, improved, because relations are not only represented transitively.
In the following, we identify relevant topologies of transformation networks that reflect the effects on properties of having multiple transformation paths preserving consistency to the same relations and discuss their impact on properties.
