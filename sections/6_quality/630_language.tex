\chapter{Designing a Language for Expressing Commonalities 
    \pgsize{25 p.}
}
\label{chap:language}

\mnote{\Commonalities approach}
In the previous chapter, we have introduced the \commonalities approach, which defines a methodology for constructing transformation networks by means of auxiliary, so called \conceptmetamodels.
These \conceptmetamodels contain the commonalities of the metamodels whose instances are to be kept consistent, denoted as \concretemetamodels, as explicit entities, rather than encoding them implicitly in transformations between the metamodels to be kept consistent.
We have argued why this construction approach foster achieving a specific tree topology of the transformation network.
Such a topology improves correctness and reusability of the resulting transformation networks, which are contradictory properties when constructing networks only of transformations between the \concretemetamodels, at least if a specific tree topology of the network is achieved.

\mnote{\Commonalities language}
Although the construction methodology of the \commonalities approach itself provides significant benefits and is thus a distinct and independently usable contribution on its own, the construction can be further supported with an appropriate language.
While the approach requires the specification of \conceptmetamodels, as well as transformations realizing the manifestation relations between the metamodels, a language can combine these specification by integrating the definition of manifestation with those of the \commonalities.
This improves conciseness and locality of the related information to be defined.
While those improvements only foster usability, but provide no conceptual benefits, a language can also ensure the achievement of an appropriate tree topology.
This can either be achieved by restricting expressiveness to achieve it by construction or by defining constructs that are a analyzable.

\mnote{Subordinate contributions}
In this chapter, we discuss the design of such a language.
We focus on design options for such a language and given an overview of the process and artifacts involved in such a language.
We also depict a concrete language, for which we have developed the prototypical \commonalities language, with a focus on the relevant elements, their relations and their operationalization.
Although we also provide a prototypical realization of such a language, this chapter does not focus on the specifics of that language, but rather the concepts behind it.
It constitutes our contribution \autoref{contrib:quality:language}, which consists of three subordinate contributions: a discussion of design options and the resulting process and artifacts for such a language; a depiction of the structure of a concrete realization of such a language with a description of its semantics and operationalization into transformations; and finally a discussion of benefits that we can expect from such a language.
It answers the following research question:

\researchquestionrepeat{rq:quality:language}

\mnote{Benefits of contributions}
The insights in this chapter first give guidelines for developers of tools for construction transformation networks.
It especially clarifies the available design space for tools supporting the \commonalities approach.
In addition, the chapter makes concrete proposals for how to develop such a language, which elements it has to contain and how it can be operationalized.
Finally, it even provides an actual realization of such a language, which can be readily used with the \vitruv framework (see \autoref{chap:foundations:multiview:vitruv}).

\mnote{Publication of contributions}
An overview of the prototypical realization of the \commonalities language and relevant design options along with a proof-of-concept has already been published~\owncite{klare2019models}.
An initial prototype of the language was developed in the Bachelor's thesis of \textowncite{gleitze2017a} and extended for a case study evaluation in the Master's thesis of \textowncite{hennig2020ma}.
Since we focus on the concepts and design options for such a language in this thesis, we refer to the work of \textowncite{gleitze2017a} for details about the realization and capabilities of the \commonalities language.


\section{Design Options}

\mnote{Space of design options}
The development of a language for realizing the \commonalities approach offers several degrees of freedom.
They range from conceptual degrees of freedom, e.g., regarding the operationalization alternatives discussed in \autoref{chap:improvement:commonalities:operationalization}, over notation types, such as textual or graphical representations, to the specific syntax to use or even reuse from existing languages.
We, in particular, consider the conceptual degrees of freedom and give an overview of how an according textual syntax can look like.

\mnote{Operationalization options}
The conceptual degrees of freedom include options for operationalizing a specification in terms of using the \conceptmetamodels as additional metamodels with the manifestation relations constituting ordinary transformations or in terms of generating direct transformations between the \concretemetamodels from the \commonalities specification, as both discussed in \autoref{chap:improvement:commonalities:operationalization}.
This option selection is transparent to the developer of a transformation network, as it only affects its operationalization.

\mnote{Specification options}
In addition, we can distinguish \emph{internal} and \emph{external} specifications, depending on whether the specification is decomposed by the \commonalities or by the defined manifestation relations.
This decision affects the developer of a transformation network, as he or she is directly concerned with the way in which \commonalities are specified.
We discuss these two options in the following in more detail.
Furthermore, we derive an overview of the resulting process for specifying and executing artifacts in such a language.


\subsection{Internal and External Specification}
\label{chap:language:design:internal_external}

\begin{figure}
    \centering
    \newcommand{\distance}{4em}
\newcommand{\vdistance}{2.5em}
\newcommand{\scenariodistance}{4*\distance}
\newcommand{\overlapdistance}{0.3em}

\begin{tikzpicture}[
    manifests relation label/.style={manifests relation, above, sloped}
]

% EXTERNAL CONCEPT DEFINITION

\node[schematic conceptmetamodel] (cmm_concept1) {};
\node[schematic conceptmetamodel, right=\overlapdistance of cmm_concept1.center, anchor=center] (cmm_concept2) {};
\node[schematic conceptmetamodel, right=\overlapdistance of cmm_concept2.center, anchor=center] (cmm_concept3) {};

\node[schematic conceptmetamodel, right=\distance of cmm_concept1.center, anchor=center] (cmm_rel1_concept1) {};
\node[schematic conceptmetamodel, right=\overlapdistance of cmm_rel1_concept1.center, anchor=center] (cmm_rel1_concept2) {};
\node[schematic conceptmetamodel, right=\overlapdistance of cmm_rel1_concept2.center, anchor=center] (cmm_rel1_concept3) {};

\node[schematic metamodel, below left=\vdistance and 0.2*\distance of cmm_rel1_concept1.center, anchor=center] (cmm_rel1_concrete1) {};
\node[schematic metamodel, right=\overlapdistance of cmm_rel1_concrete1.center, anchor=center] (cmm_rel1_concrete2) {};
\node[schematic metamodel, right=\overlapdistance of cmm_rel1_concrete2.center, anchor=center] (cmm_rel1_concrete3) {};

\node[schematic conceptmetamodel, right=\distance of cmm_rel1_concept1.center, anchor=center] (cmm_rel2_concept1) {};
\node[schematic conceptmetamodel, right=\overlapdistance of cmm_rel2_concept1.center, anchor=center] (cmm_rel2_concept2) {};
\node[schematic conceptmetamodel, right=\overlapdistance of cmm_rel2_concept2.center, anchor=center] (cmm_rel2_concept3) {};

\node[schematic metamodel, below right=\vdistance and 0.2*\distance of cmm_rel2_concept1.center, anchor=center] (cmm_rel2_concrete1) {};
\node[schematic metamodel, right=\overlapdistance of cmm_rel2_concrete1.center, anchor=center] (cmm_rel2_concrete2) {};
\node[schematic metamodel, right=\overlapdistance of cmm_rel2_concrete2.center, anchor=center] (cmm_rel2_concrete3) {};

\draw[manifests relation] (cmm_rel1_concept1) -- (cmm_rel1_concrete1);
\draw[manifests relation] (cmm_rel1_concept2) -- (cmm_rel1_concrete2);
\draw[manifests relation] (cmm_rel1_concept3) -- (cmm_rel1_concrete3);

\draw[manifests relation] (cmm_rel2_concept1) -- (cmm_rel2_concrete1);
\draw[manifests relation] (cmm_rel2_concept2) -- (cmm_rel2_concrete2);
\draw[manifests relation] (cmm_rel2_concept3) -- (cmm_rel2_concrete3);

\node[below right=0.5*\vdistance and 0.5*\distance of cmm_concept2.center, anchor=center, font=\bfseries] {+};
\node[below right=0.5*\vdistance and 0.5*\distance of cmm_rel1_concept2.center, anchor=center, font=\bfseries] {+};


% INTERNAL CONCEPT DEFINITION

\node[schematic conceptmetamodel, right=\scenariodistance of cmm_concept1.center, anchor=center] (int_concept1) {};
\node[schematic metamodel, below left=\vdistance and 0.2*\distance of int_concept1.center, anchor=center] (int_concept1_concrete1) {};
\node[schematic metamodel, below right=\vdistance and 0.2*\distance of int_concept1.center, anchor=center] (int_concept1_concrete2) {};
\draw[manifests relation] (int_concept1) -- (int_concept1_concrete1);
\draw[manifests relation] (int_concept1) -- (int_concept1_concrete2);

\node[schematic conceptmetamodel, right=\distance of int_concept1.center, anchor=center] (int_concept2) {};
\node[schematic metamodel, below left=\vdistance and 0.2*\distance of int_concept2.center, anchor=center] (int_concept2_concrete1) {};
\node[schematic metamodel, below right=\vdistance and 0.2*\distance of int_concept2.center, anchor=center] (int_concept2_concrete2) {};
\draw[manifests relation] (int_concept2) -- (int_concept2_concrete1);
\draw[manifests relation] (int_concept2) -- (int_concept2_concrete2);

\node[schematic conceptmetamodel, right=\distance of int_concept2.center, anchor=center] (int_concept3) {};
\node[schematic metamodel, below left=\vdistance and 0.2*\distance of int_concept3.center, anchor=center] (int_concept3_concrete1) {};
\node[schematic metamodel, below right=\vdistance and 0.2*\distance of int_concept3.center, anchor=center] (int_concept3_concrete2) {};
\draw[manifests relation] (int_concept3) -- (int_concept3_concrete1);
\draw[manifests relation] (int_concept3) -- (int_concept3_concrete2);

\node[below right=0.5*\vdistance and 0.5*\distance of int_concept1.center, anchor=center, font=\bfseries] {+};
\node[below right=0.5*\vdistance and 0.5*\distance of int_concept2.center, anchor=center, font=\bfseries] {+};

\node[above=0.5em of cmm_rel1_concept2.north, anchor=south, font=\small\bfseries] (ext_label) {External Concept Definition};
\node[above=0.5em of int_concept2.north, anchor=south, font=\small\bfseries] (int_label) {Internal Concept Definition};

\node[below=1.7*\vdistance of cmm_rel1_concept2.center, anchor=center, font=\footnotesize\itshape] (ext_dim_label) {Decomposition Dimension: Relations};
\node[below=1.7*\vdistance of int_concept2.center, anchor=center, font=\footnotesize\itshape] (int_dim_label) {Decomposition Dimension: Commonalities};

\draw[gray, thin] ($(ext_label.north)!0.5!(int_label.north)$) -- ($(ext_dim_label.south)!0.5!(int_dim_label.south)$);


\end{tikzpicture}
    %\includegraphics[width=\textwidth]{figures/quality/language/design_options.png}
    \caption[Design options for \commonalities specification]{Exemplification of alternatives to specify \commonalities by means of a separate, external specification of complete \conceptmetamodels and manifestation relations or an integrated, internal definition of \commonalities with their manifestation relations. Circles denote \commonalities and manifestations, arrows denote manifestation relations.}
    \label{fig:language:design_options}
\end{figure}

\mnote{Decomposition dimensions}
We can distinguished two ways in which \conceptmetamodels and manifestation relations can be specified according to the \commonalities approach.
They depend on the dimension along which the specification is decomposed.
More precisely, the specification can either be decomposed along the \commonalities, such that each \commonality together with all its manifestations is defined at one place, or it can be decomposed along the manifestation relations, such that all manifestation relations between a \conceptmetamodel and its manifestation are defined at one place.
We refer to these specifications as \emph{internal} and \emph{external} specifications, which we have already proposed in previous work~\owncite{klare2019models} and which we illustrate in \autoref{fig:language:design_options}.
\begin{properdescription}
    \item[External Concept Definition:] \Conceptmetamodels are defined as ordinary metamodels and each manifestation relation is defined as an individual transformation, i.e., manifestation relations are defined externally to \conceptmetamodels and their \commonalities.
    \item[Internal Concept Definition:] Each \commonality of each \conceptmetamodel is defined together with its relations to manifestations, thus manifestation relations are defined internally with the \commonalities they belong to.
\end{properdescription}

\mnote{External specification}
Without developing an additional language, the \commonalities approach can be realized by developing \conceptmetamodels as if they are ordinary metamodels with appropriate modeling tools.
The manifestation relations can then be defined with any existing transformation language that is able to generate incremental transformations.
This conforms to an \emph{external} specification, in which \conceptmetamodels and manifestation relations are defined separately.
It decomposes the specification along the relations, such that there are as many separate artifacts as there are \conceptmetamodels and relations to be defined.
For example, for Java and the \gls{UML} an object-oriented design \conceptmetamodel as well as two manifestation relations to each of the \concretemetamodels would be defined separately.

\mnote{Internal specification}
Developing a specific language allows to integrate the definition of \commonalities with their manifestation relations.
The relations to manifestations of a \commonality are then defined at one place with the declaration of the \commonality, improving locality of this related information.
This conforms to an \emph{internal} specification.
It decomposes the specification along the \commonalities, thus as many separate specifications exist as \commonalities are defined.
For example, for Java and the \gls{UML} a class \commonality together with its manifestation as classes in both Java and the \gls{UML} with the according relations of attribute values and references would be defined at one place.

\mnote{Tyranny of dominant decomposition}
Selecting one of these types of specification suffers from the \enquote{tyranny of the dominant decomposition}~\cite{tarr1999Tyranny-ICSE}.
Thus, decomposition is only possible along one dimension of concerns, i.e., either the structural specification of \commonalities or the relational specification of manifestation relations, such that either one suffers from lacking separation of concerns in the other dimension.
Thus, while one approach improves locality when adding \commonalities, the other improves locality when adding manifestation relations.

\mnote{External specification benefits}
External specifications benefit from the separation of each manifestation relation into its own specification.
This reduces dependencies between the manifestations and especially allows each developer who is responsible for a specific \concretemetamodel to define the relation to each related \conceptmetamodel as a whole instead of distributing this specification among all \commonalities specifications describing a concept represented in the \concretemetamodel.
In consequence, adding a new \concretemetamodel only requires the addition and potentially adaptation of manifestation relations to \conceptmetamodels.
External specifications support this scenario well because of high locality of all information regarding a manifestation relation and because manifestation relations represent the largest part of the addition.
Additionally, they can be realized without developing a new language.

\mnote{Internal specification benefits}
Internal specifications require a dedicated language enabling the integrated specification of \commonalities and their manifestations.
This improves locality regarding the information about each \commonality, as each \commonality is represented along with all its manifestations.
In consequence, when initially developing \commonalities for a set of \concretemetamodels, it is easier to add each single \commonality, because all information about the \commonality and its relations to the manifestations can be defined at one place.
This can make it easier to understand the overall relation of that common concept among all \concretemetamodels.
In addition, it makes it less likely for a developer to miss the definition of one or more manifestations of a \commonality, as they are obviously missing in the specification of the \commonality, whereas in an external specification it is missing somewhere in the complete manifestation relation between the \conceptmetamodel and its manifestation.
Finally, the approach promises to be more concise, because the manifestation relations are defined within the \commonality they belong to instead of referencing the \commonality within a transformation again.

\mnote{Proposal of internal specification language}
To benefit from locality regarding each \commonality and a more concise specification, we have decided to design a language that supports internal specifications.
Depending on the usage context and usual change scenarios, an external specification may, however, be more appropriate.
Then, modeling \concretemetamodels with an existing modeling framework and the manifestation relations with existing transformation languages is sufficient.


\subsection{Artifacts and Process}
\label{chap:language:design:artifacts_process}

\mnote{Selected design options}
Regarding the design options in \autoref{chap:improvement:commonalities:operationalization} and \autoref{chap:language:design:internal_external}, we have made the following, already argued decisions.
First, we chose to operationalize a specification by treating \conceptmetamodels as ordinary metamodels, such that instances of them are created and kept consistent.
This option does especially not restrict expressiveness of the relations, and the generation of additional models can be hidden from the user by appropriate tooling.
Second, we chose to provide a language that supports an internal specification of concepts to improve locality of the information regarding each \commonality.
We expect this specification to be more concise and to better support the initial specification process for \commonalities.

\begin{figure}
    \centering
    \newcommand{\developstext}{«develops»}
\newcommand{\modifiestext}{«modifies»}
\newcommand{\vdistance}{5em}
\newcommand{\hdistance}{(\vdistance+0.2*\difftoafiveimage)}

\newcommand{\tikzsystemborder}[2]{
    \node[draw, runtime artifact, ellipse, minimum width=0.5*\vdistance, minimum height=0.5*\vdistance, anchor=center, #1] (#2) {};
}

\newcommand{\tikzconcretemetamodels}[1]{
    \node[draw, runtime artifact, ellipse, minimum width=0.05*\vdistance, minimum height=0.05*\vdistance, above left=0.02*\vdistance and 0.08*\vdistance of 
    #1.center, inner sep=0em] (#1model1) {};
    \node[draw, runtime artifact, ellipse, minimum width=0.05*\vdistance, minimum height=0.05*\vdistance, above right=0.08*\vdistance and 0.04*\vdistance of #1.center, inner sep=0em] (#1model2) {};
    \node[draw, runtime artifact, ellipse, minimum width=0.05*\vdistance, minimum height=0.05*\vdistance, below right=0.08*\vdistance and 0.04*\vdistance of #1.center, inner sep=0em] (#1model3) {};
}

\newcommand{\tikzconceptmetamodels}[1]{
    %\node[draw, orchestration artifact, ellipse, minimum width=0.05*\vdistance, minimum height=0.05*\vdistance, above left=0.08*\vdistance and 0.05*\vdistance of #1.center, inner sep=0em] (#1model1) {};
    \node[draw, orchestration artifact, ellipse, minimum width=0.05*\vdistance, minimum height=0.05*\vdistance, right=0.08*\vdistance of #1.center, inner sep=0em] (#1model2) {};
    \node[draw, orchestration artifact, ellipse, minimum width=0.05*\vdistance, minimum height=0.05*\vdistance, below left=0.06*\vdistance and 0.05*\vdistance of #1.center, inner sep=0em] (#1model3) {};
}

\newcommand{\tikzconceptandconcretemetamodels}[1]{
    \tikzconcretemetamodels{#1}
    \tikzconceptmetamodels{#1}
}

\newcommand{\tikzsystemwithconcepts}[2]{
    \tikzsystemborder{#1}{#2}
    \tikzconceptandconcretemetamodels{#2}
}

\newcommand{\machine}[2]{
    \draw[#2] ([xshift=-0.5em, yshift=-0.5em]#1) -- ++(0,1em) -- ++(-0.5em, 0.5em) -- ++(2em, 0) -- ++(-0.5em, -0.5em) -- ++(0,-1em) -- ++(-1em,0);
}


\begin{tikzpicture}[
    every node/.append style={font=\footnotesize},
    developer artifact/.style={consistencypreservationcolor},
    runtime artifact/.style={},
    orchestration artifact/.style={executioncolor},
    artifact role name/.style={font=\footnotesize\bfseries}
]

% Commonalities
\node[draw, developer artifact, fill=white, inner sep=0.15em, align=left] (trans1) {
    \begin{lstlisting}[language=commonalities, frame=none, numbers=none, backgroundcolor=\color{white}, linewidth=14pt, basicstyle=\fontsize{2}{2.4}\selectfont\ttfamily]
commonality R {
  with A
  with B
    
  has att {
    = A.att
    -> B.att
  }
}
    \end{lstlisting}
};
\node[draw, below right=0.2em of trans1.north, anchor=north, developer artifact, fill=white, inner sep=0.15em, align=left] (trans2) {
    \begin{lstlisting}[language=commonalities, frame=none, numbers=none, backgroundcolor=\color{white}, linewidth=14pt, basicstyle=\fontsize{2}{2.4}\selectfont\ttfamily]
commonality R {
  with A
  with B
    
  has att {
    = A.att
    -> B.att
  }
}
    \end{lstlisting}
};
\node[draw, below right=0.2em of trans2.north, anchor=north, developer artifact, fill=white, inner sep=0.15em, align=left] (trans3) {
    \begin{lstlisting}[language=commonalities, frame=none, numbers=none, backgroundcolor=\color{white}, linewidth=14pt, basicstyle=\fontsize{2}{2.4}\selectfont\ttfamily]
commonality R {
  with A
  with B
    
  has att {
    = A.att
    -> B.att
  }
}
    \end{lstlisting}
};

% Expert
\umlhuman{expert1}{}{above left=0.2em and 1.2*\hdistance of trans2.center, anchor=center}{}{0.3}
\draw[-latex] ([yshift=-0.2em]expert1.east) -- node[above, stereotype] {\developstext} ([xshift=-0.2em]trans2.west);

% Concrete metamodels
\coordinate (concrete_metamodels) at ([yshift=\vdistance]trans2);
\tikzconcretemetamodels{concrete_metamodels}

\draw[-latex, densely dashed] ([yshift=0.2em]trans2.north) -- node[left, stereotype, align=center] {«refers\\ to»} ([yshift=-1em]concrete_metamodels);

% Compiler
\coordinate (compiler) at ([xshift=0.8*\hdistance]trans2.center);
\machine{compiler}{}

\draw[-latex] ([xshift=0.5em]trans2.east) -- ([xshift=-1em]compiler);
\draw[-latex] ([yshift=-0.5em,xshift=0.8em]concrete_metamodels) -- ([yshift=1.2em,xshift=-0.7em]compiler);

% Generated BX
\node[draw, right=1.6*\hdistance of trans1.center, anchor=center, orchestration artifact, fill=white, inner sep=0.15em, align=left] (bx1) {
    \begin{lstlisting}[language=reactions, frame=none, numbers=none, backgroundcolor=\color{white}, linewidth=14pt, basicstyle=\fontsize{2}{2.4}\selectfont\ttfamily]
reaction RAB {
  after ...
  call S
}

routine S {
  match ... 
  action ...
}
    \end{lstlisting}
};
\node[draw, below right=0.2em of bx1.north, anchor=north, orchestration artifact, fill=white, inner sep=0.15em, align=left] (bx2) {
    \begin{lstlisting}[language=reactions, frame=none, numbers=none, backgroundcolor=\color{white}, linewidth=14pt, basicstyle=\fontsize{2}{2.4}\selectfont\ttfamily]
reaction RAB {
  after ...
  call S
}

routine S {
  match ... 
  action ...
}
    \end{lstlisting}
};
\node[draw, below right=0.2em of bx2.north, anchor=north, orchestration artifact, fill=white, inner sep=0.15em, align=left] (bx3) {
    \begin{lstlisting}[language=reactions, frame=none, numbers=none, backgroundcolor=\color{white}, linewidth=14pt, basicstyle=\fontsize{2}{2.4}\selectfont\ttfamily]
reaction RAB {
  after ...
  call S
}

routine S {
  match ...
  action ...
}
    \end{lstlisting}
};

% All metamodels
\coordinate (all_metamodels) at ([yshift=\vdistance]bx2);
\tikzconceptandconcretemetamodels{all_metamodels}

\draw[-latex, orchestration artifact] ([yshift=1.2em,xshift=0.7em]compiler) -- ([yshift=-0.5em,xshift=-0.8em]all_metamodels);
\draw[-latex] ([xshift=1em]compiler) -- ([xshift=-0.5em]bx2.west);

\coordinate (upperend_all_metamodels) at ([xshift=0.2*\hdistance,yshift=0.2*\vdistance]all_metamodels);
\draw [orchestration artifact, decorate,decoration={brace,amplitude=10pt},xshift=0pt,yshift=4pt]
([xshift=0.5em]bx3.south east|-upperend_all_metamodels) -- ([xshift=0.5em]bx3.south east);

% Machine
\coordinate (machine) at ([yshift=0.2*\vdistance,xshift=2.1*\hdistance]bx2);
\machine{machine}{orchestration artifact}

\coordinate (output_middle) at ($(bx3.south east)!0.5!(upperend_all_metamodels)$);
\draw[orchestration artifact, densely dashed, -latex] ([xshift=10pt+0.5em]output_middle) -- node[below=0.2em, stereotype, align=center] {Transformation\\ Network} ([yshift=0.3em, xshift=-0.9em]machine.west);

% Input
\tikzsystemwithconcepts{above left=0.6*\vdistance and 0.48*\hdistance of machine}{original_system}
\node[runtime artifact, above=1.2em of original_system.north, anchor=south, align=center, artifact role name] {System\\ (Concrete +\\ {\color{darkorange} Concept}\\ Models)};
\node[runtime artifact, below right=0em and \hdistance of original_system.center, anchor=north] (original_changes) {$\sequenced{\change{1}, \change{2}, \change{3}, \dots}$};
\node[runtime artifact, above=-0.3em of original_changes.north east, anchor=south east, artifact role name] {Changes};

\umlhuman{developer}{}{above right=0.5*\vdistance and 0.2*\hdistance of original_changes.north, anchor=south}{}{0.3}

\draw[-latex] (developer) -- node[stereotype, sloped, above] (modifies_label) {\modifiestext} (original_system);
\draw[dashed, -latex] (modifies_label.south) -- ([xshift=-1.2em]original_changes.north);

\draw [runtime artifact, decorate,decoration={brace,amplitude=10pt},xshift=0pt,yshift=-4pt]
(original_changes.south east) -- (original_system.west|-original_system.south);

\draw[gray, densely dashed, -latex] ([xshift=-0.2em]original_system.west) -- node[above, sloped, stereotype, align=center, pos=0.35] {«instan-\\ tiates»} ([xshift=0.2em+0.25*\hdistance]all_metamodels);

% Output
\tikzsystemwithconcepts{below=0.5*\vdistance of machine}{final_system}
\node[runtime artifact, left=0em of final_system.west, anchor=east, align=center, artifact role name] {Consistent\\ System\\ (Models)};
\draw[runtime artifact, -latex] ([yshift=-0.8em]machine.south) -- ([yshift=0.2em]final_system.north); 

% Labels
\node[above=1.5em of concrete_metamodels.north, anchor=south, artifact role name, align=center, inner sep=0em] (concrete_metamodels_label) {Concrete\\ Metamodels};
\node[above=1.5em of all_metamodels.north, anchor=south, artifact role name, align=center, inner sep=0em] (concrete_metamodels_label) {Concrete +\\ {\color{darkorange} Concept}\\ Metamodels};
\node[above right=0.3em and 0.1*\hdistance of expert1.north, anchor=south, artifact role name, align=center, inner sep=0em] (expert_text) {Domain\\ Expert};
\node[developer artifact, below=0.8em of trans2.south, anchor=north, artifact role name, align=center] (comm_spec_label) {Commonalities\\ Specifications};
\node[anchor=north, below=0.8em of compiler, anchor=north, artifact role name, align=center] %at (compiler|-comm_spec_label.north) 
{Compiler};
\node[orchestration artifact, below=0.8em of bx2.south, anchor=north, artifact role name, align=center] (bx_spec_label) {Bidirectional\\ Transformations};
\node[above=0.2em of developer.north, anchor=south, artifact role name, inner sep=0em, align=center] {System\\ Developer};


\end{tikzpicture}

    %\includegraphics[width=\textwidth]{figures/quality/language/overall_process.png}
    \caption[Process and artifacts using a language for \commonalities]{The process for developing, compiling, and executing specifications in a language for \commonalities. From \concretemetamodels and \commonalities specifications, additional \conceptmetamodels and transformations are generated, which are executed at runtime for preserving consistency of models.
    \Commonalities specifications by domain experts are marked orange, the generated artifacts (\conceptmetamodels and transformations forming a network) are marked green. Concrete systems and changes depict runtime artifacts.}
    \label{fig:language:process}
\end{figure}

\mnote{Specification and compilation}
The process of specifying, compiling, and executing artifacts in such a language is depicted in \autoref{fig:language:process}.
It is a specialization of the general process already depicted in \autoref{fig:introduction:process_overview}.
A domain expert or transformation developer defines \commonalities specifications using the language, which refers to \concretemetamodels that are to be kept consistent by the transformations derived from that specification.
The compiler of the language takes the \concretemetamodels together with the specifications to generate a set of \conceptmetamodels in addition to the existing \concretemetamodels, as well as a set of bidirectional transformations, which implement consistency preservation for the manifestation relations between the \conceptmetamodels and \concretemetamodels.
These artifacts together form a transformation network as introduced in \autoref{def:transformationnetwork}.

\mnote{Execution}
A system developer specifies a system by models that instantiate the \concretemetamodels of the \commonalities specification.
The complete system description consists of instances of these \concretemetamodels but also, in the best case hidden from developer, of instances of the \conceptmetamodels for means of consistency preservation.
Whenever the system developer produces changes to the instances of the \concretemetamodels, the transformation network can be applied to the changes together with the models.
It then returns a new set of instances of the \concretemetamodel and \conceptmetamodels that are consistent again, according to the proposed correctness notion of transformation networks in \autoref{def:transformationnetworkcorrectness}.

\section{The \Commonalities Language}

\mnote{Section overview}
In this section, we present an overview of the \commonalities language.
It constitutes one possible realization of a language for the \commonalities approach with the conceptual design choices that we have discussed in the previous section.
This especially includes an internal specification of concepts.
To give an impression of the language, we first introduce two examples for specifications in a prototypical realization of the language with a textual syntax, which we have already proposed in previous work~\owncite{klare2019models} and which was originally developed in the bachelor's thesis of \textowncite{gleitze2017a} and extended in the master's thesis of \textowncite{hennig2020ma}.
We then give an overview of the language elements before explaining the different categories of them in more detail at the given examples.
Since we focus on the concepts of the language rather than its detailed realization with a textual syntax, we refer for details of that realization to the theses of \textowncite{gleitze2017a} and \textowncite{hennig2020ma}.


\subsection{Examples in Textual Syntax}

\lstinputlisting[language=commonalities, float,
    caption={[Exemplary \commonality for clases]An exemplary specification for an extract of the \texttt{Class} \commonality between \gls{UML} and Java in the \commonalities language},
    captionpos=b,
    label=lst:language:class_example,
]{listings/quality/language/class_example.lst}

\lstinputlisting[language=commonalities, float,
    caption={[Exemplary \commonality for components]An exemplary specification for an extract of the \texttt{Component} \commonality between \gls{PCM}, UML and the object-oriented design \conceptmetamodel in the \commonalities language},
    captionpos=b,
    label=lst:language:component_example,
]{listings/quality/language/component_example.lst}

\mnote{Examples overview}
We depict two examples for specifications in our prototype of the \commonalities language with a textual syntax in \autoref{lst:language:class_example} and \autoref{lst:language:component_example}.
The specifications depict extracts of a \commonality for classes in \gls{UML} and Java, as well as extracts of a \commonality for components in \gls{PCM}, \gls{UML} and classes with their containing packages in the object-oriented design \conceptmetamodel.
The extracts are selected to reflect the different elements of the \commonalities language without introducing unnecessary complexity.
We sketch the meaning of the examples in the following and clarify them along with the subsequent introduction of the language elements more precisely.

\mnote{Class \commonality example}
The class \commonality, depicted in \autoref{lst:language:class_example}, is restricted to their names and methods.
In \gls{UML}, a class is represented by a class that is contained in a unique instance of a \gls{UML} model.
In Java, a class is also represented by class that is contained in a compilation unit, which depicts one file consisting of imports and class specifications as a single unit of compilation~\cite{heidenreich2009a}.
Names are represented equally in \gls{UML} and Java classes.
The name of the compilation unit is defined by the fully qualified name of the class, i.e., the concatenation of its namespace and the class name separated by a dot.
The specification expresses this as the class name to be the suffix of the compilation unit name after the namespace followed by a dot.
Methods are specified in a dedicated \commonality in the object-oriented design \conceptmetamodel, such that they are only referenced in the class \commonality, but without any specification of the relations of their contents.

\mnote{Component \commonality example}
The component \commonality, depicted in \autoref{lst:language:component_example}, is restricted to their names.
In \gls{PCM} and \gls{UML}, components are realized by explicit component or basic component classes, respectively, which share the same name.
In object-oriented design, components are defined to be represented by classes contained in a package.
Classes are only considered to represent components when their name has an \enquote{Impl} suffix and their name is then defined to be the component name with an \enquote{Impl} suffix.
The specification defines this as a prefix, analogous to the suffix for the name of a compilation unit, as it denotes that the component name is the prefix of the class name before \enquote{Impl}.
Finally, the package name is defined to be the component name but starting with a lowercase letter whereas the component name is start with an uppercase letter.
Analogous to the prefix definition for the class name, the specification defines a \texttt{firstUpper} operation as the component name shall be the package name with the first letter in uppercase.


\subsection{Elements Overview}

\mnote{Categories of elements}
The \commonalities language essentially consists of three categories of elements.
First, at a top level the structure of \commonalities needs to be defined in terms of defining for each of them the \conceptmetamodels they belong to, as well as the features in terms of attributes and references it described.
Second, each \commonality needs to define its manifestations, i.e., the \metaclasses of \concretemetamodels or other \conceptmetamodels being its manifestation, along with conditions defining when instances of \metaclasses are to be considered a manifestation.
This defines when a manifestation relation between a \commonality and \metaclasses of another \conceptmetamodel or \concretemetamodel exist.
Third, each \commonality needs to define the relations of its features to those of its manifestations.
This defines the manifestation relations, i.e., the conditions that have to hold for considering a manifestation consistent to a \commonality.

\begin{figure}
    \centering
    \newcommand{\hdistance}{7.6em}
\newcommand{\classwidth}{5.5em}
\newcommand{\vdistance}{5em}

\begin{tikzpicture}[
    existing/.style={fill=lightgray!20}
]

\umlclassvarwidth[, existing]{metamodel}{}{Metamodel}{
name : String\\
}{\classwidth} 

\umlclassvarwidth[, existing, right=\hdistance of metamodel.north, anchor=north]{class}{}{Class}{
name : String\\
}{\classwidth}  

\umlclassvarwidth[, existing, right=\hdistance of class.north, anchor=north]{reference}{}{Reference}{
name : String\\
}{\classwidth}

\umlclassvarwidth[, existing, right=\hdistance of reference.north, anchor=north]{attribute}{}{Attribute}{
name : String\\
}{\classwidth}

\umlclassvarwidth[, below=\vdistance of metamodel.north, anchor=north]{concept}{}{Concept}{
}{\classwidth}  

\umlclassvarwidth[, right=\hdistance of concept.north, anchor=north]{commonality}{}{Commonality}{
root : Boolean\\
}{\classwidth}

\umlclassvarwidth[, below=\vdistance of commonality.north, anchor=north]{manifestation}{}{Manifestation}{
}{\classwidth} 

\umlclassvarwidth[, right=\hdistance of manifestation.north, anchor=north]{commonality_reference}{}{Commonality\\Reference}{
}{\classwidth}

\umlclassvarwidth[, right=\hdistance of commonality_reference.north, anchor=north]{commonality_attribute}{}{Commonality\\Attribute}{
}{\classwidth}


\umlclassvarwidth[, below left=\vdistance and \hdistance of manifestation.north, anchor=north]{manifestation_class}{}{Manifestation\\Class}{
alias : String\\
single : Boolean\\
}{\classwidth}

\umlclassvarwidth[, below=\vdistance of manifestation.north, anchor=north]{manifestation_condition}{}{Manifestation\\Condition}{
dir : Direction\\
}{\classwidth}

\umlclassvarwidth[, below=\vdistance of commonality_attribute.north, anchor=north]{attribute_relation}{}{Attribute\\Relation}{
dir : Direction\\
}{\classwidth} 

\umlclassvarwidth[, below=\vdistance of commonality_reference.north, anchor=north]{reference_relation}{}{Reference\\Relation}{
dir : Direction\\
}{\classwidth} 

\umlclassvarwidth[, below=\vdistance of manifestation_condition.north, anchor=north]{manifestation_operator}{}{Manifestation\\Operator}{
}{\classwidth}

\umlclassvarwidth[, below=\vdistance of attribute_relation.north, anchor=north]{attribute_operator}{}{Attribute\\Operator}{
}{\classwidth} 

\umlclassvarwidth[, below=\vdistance of reference_relation.north, anchor=north]{reference_operator}{}{Reference\\Operator}{
}{\classwidth} 

\umlclassvarwidth[, below=\vdistance of reference_operator.north, anchor=north]{operand}{}{\textit{Operand}}{
}{\classwidth}

\umlclassvarwidth[, below left=\vdistance and \hdistance of operand.north, anchor=north]{manifestation_operand}{}{\textit{Manifestation}\\\textit{Operand}}{
}{\classwidth}

\umlclassvarwidth[, below right=\vdistance and 0em of operand.north, anchor=north]{literal_operand}{}{Literal\\Operand}{
}{\classwidth}

\umlclassvarwidth[, right=\hdistance of literal_operand.center, anchor=center]{direction}{}{\umlenumlabel\\Direction}{
\itshape
Bidirectional\\
Checkonly\\
Enforce\\
}{\classwidth}

\umlclassvarwidth[, below left=\vdistance and \hdistance of manifestation_operand.north, anchor=north]{manifestation_class_operand}{}{Manifestation\\Class\\Operand}{
}{\classwidth}

\umlclassvarwidth[, below left=\vdistance and 0em of manifestation_operand.north, anchor=north]{manifestation_attribute_operand}{}{Manifestation\\Attribute\\Operand}{
}{\classwidth}

\umlclassvarwidth[, below right=\vdistance and \hdistance of manifestation_operand.north, anchor=north]{manifestation_reference_operand}{}{Manifestation\\Reference\\Operand}{
}{\classwidth}


% INHERITANCE
\umlsubclassof{concept}{--}{metamodel}
\umlsubclassof{commonality}{--}{class}
\umlsubclassof{commonality_attribute}{--}{attribute}
\umlsubclassof{commonality_reference}{--}{reference}

\umlsubclassof{manifestation_operand}{|- ([yshift=-1.5em]operand.south) --}{operand}
\draw (literal_operand) |- ([yshift=-1.5em]operand.south);

\umlsubclassof{manifestation_class_operand}{|- ($(manifestation_attribute_operand.north)!0.5!(manifestation_operand.south)$) --}{manifestation_operand}
\draw (manifestation_attribute_operand) -- ($(manifestation_attribute_operand.north)!0.5!(manifestation_operand.south)$);
\draw (manifestation_reference_operand) |- ($(manifestation_attribute_operand.north)!0.5!(manifestation_operand.south)$);

% REFERENCES
%\umlassociationfromto{(class) -- node[uml cardinality end, pos=1, above right] {1} (metamodel)}
\umlcomposition{(metamodel.east) -- node[uml cardinality end, pos=1, above left] {*} (class.west|-metamodel.west)}
\umlcomposition{(concept.east) -- node[uml cardinality end, pos=1, above left] {*} (commonality.west|-concept.west)}
\umlcomposition{(class.east) -- node[uml cardinality end, pos=1, above left] {*} (reference.west|-class.east)}
\umlcomposition{([xshift=1em]class.north) -- ++(0,1em) -| node[uml cardinality end, pos=1, above right] {*} (attribute.north)}

\umlcomposition{(commonality) -- node[uml cardinality end, pos=1, above right] {*} (manifestation)}
\umlcomposition{([xshift=-1em]commonality.south east) -- ++(0,-1em) -|node[uml cardinality end, pos=1, above right] {*} ([xshift=1em]commonality_reference.north west)}
\umlcomposition{([yshift=0.5em]commonality.east) -| node[uml cardinality end, pos=1, above left] {*} ([xshift=-1.5em]commonality_attribute.north)}

\umlassociationfromto{([xshift=-1em]commonality_reference.north) |- node[uml cardinality end, pos=1, above right] {1} node[uml role end, pos=1, below right] {type} ([yshift=-1em]commonality.east)}

\umlcomposition{([xshift=1em]manifestation.south west) -- ++(0,-1.5em) -| node[uml cardinality end, pos=1, above left] {*} ([xshift=-1em]manifestation_class.north east)}
\umlcomposition{(manifestation) -- node[uml cardinality end, pos=1, above left] {*} (manifestation_condition)}
\umlcomposition{(commonality_attribute) -- node[uml cardinality end, pos=1, above left] {*} (attribute_relation)}
\umlcomposition{(commonality_reference) -- node[uml cardinality end, pos=1, above left] {*} (reference_relation)}

\umlaggregation{(manifestation_condition) -- node[uml cardinality end, pos=1, above left] {*} (manifestation_operator)}
\umlaggregation{(attribute_relation) -- node[uml cardinality end, pos=1, above left] {*} (attribute_operator)}
\umlaggregation{(reference_relation) -- node[uml cardinality end, pos=1, above left] {*} (reference_operator)}

\umlaggregation{([xshift=1em]manifestation_class.north west) |- node[uml cardinality start, pos=0, above right] {*} ++(-1.5em,\vdistance) |- ([xshift=-1em,yshift=1em]class.north) -- node[uml cardinality end, pos=1, above left] {1} ([xshift=-1em]class.north)}

\umlcomposition{(commonality_attribute.west) -- ++(-0.8em,0) |- node[uml cardinality end, pos=1, below right] {*} (operand.east)}
\umlcomposition{(commonality_reference.west) -- ++(-0.8em,0) |- node[uml cardinality end, pos=1, above left] {*} ([yshift=0.8em]operand.west)}
\umlcomposition{([yshift=-0.5em]manifestation_condition.west) -- ++(-0.8em,0) |- node[uml cardinality end, pos=1, above left] {*} node[uml cardinality end, pos=1, below left] {rightOperand} ([yshift=-0.8em]operand.west)}
\umlcomposition{([yshift=0.5em]manifestation_condition.west) -- ++(-1.1em,0) |- node[uml cardinality end, pos=1, above left] {*} node[uml cardinality end, pos=1, below left] {leftOperand} ([yshift=0.8em]manifestation_operand.west)}

\umlassociationfromto{([yshift=-0.8em]manifestation_operand.west) -| node[uml cardinality end, pos=1, below right] {1} ([xshift=-1em]manifestation_class.south)}
\umlassociationfromto{(manifestation_attribute_operand.south) -- ++(0,-0.8em) -| ([xshift=0.8em, yshift=-1.2em]attribute.south east) -| node[uml cardinality end, pos=1, below right] {1} ([xshift=1.5em]attribute.south)}
\umlassociationfromto{(manifestation_reference_operand.east) -| ([xshift=0.4em, yshift=-2em]attribute.south east) -| node[uml cardinality end, pos=1, below right] {1} ([xshift=1.5em]reference.south)}

\end{tikzpicture}
    \caption[\commonalities language elements]{Class diagram with the essential elements of the \commonalities language and their relations. Elements that exist independent from the language are depicted in the top row.}
    \label{fig:language:elements}
\end{figure}

\mnote{Structural elements}
\autoref{fig:language:elements} depicts the essential elements of the \commonalities language.
At the top, it depicts metamodels, classes, references and attributes as already existing in the notion of a general modeling formalism and as specified in \concretemetamodels.
The language introduces concepts, which represent the \conceptmetamodels, and \commonalities, of which such a concept consists.
In our realization, they can be considered specializations of metamodels and classes, as they can be considered as such but with the special semantics of being only auxiliary artifacts for the \commonalities approach.
A commonality consists of commonality references and attributes, which, again, can be considered specializations of ordinary references attributes.
In the given examples, we have name attributes and a reference to methods.
Additionally, a commonality contains manifestations.
Each manifestation represents the realization of the concept represented by the \commonality in another metamodel by one or more classes and potentially further conditions for these classes.
Such manifestation are, for example, a class and a compilation unit in Java for the class \commonality depicted in \autoref{lst:language:class_example}.
Each commonality reference and attribute is complemented by reference and attribute relations that define how these features are related to information in the manifestations.

\mnote{Relational elements}
In consequence, the manifestation conditions together with the attribute and reference relations define the consistency relations between the \commonality and its manifestations, which we have introduced as manifestation relations.
All these relations consists of operators, which define how elements are related, and operands, which define the involved elements to be considered by the operator. 
The operators can be considered specifications of transformation rules.
They have a direction, as they may enforce the defined relation either in both directions or only in one of them.
For example, the name of a class in \autoref{lst:language:class_example} is related to the Java class name bidirectionally (denoted by \enquote{=}), such that a change of the Java class name leads to the change of the name of the class \commonality, which then in consequence changes the \gls{UML} class name, but also a change of the class \commonality name, e.g., because of a change of the \gls{UML} class name, leads to a change of the Java class name.
The name of a compilation unit, however, is only enforced, because it is only derived from the Java class name, such that a change is propagated because of the changed Java class name anyway.

\mnote{Simplifications}
For reasons of simplicity, we omitted several elements of the actual language realization, which concerns generalizations as well as specializations of the depicted elements.
For example, manifestation conditions, reference relations and attribute relations all represent relations between the \commonality and its manifestations, especially comprising a direction, which an be represented in a common supertype \texttt{Relation}.
Likewise, the three operator types for manifestations, references and attributes can be derived from a common \texttt{Operator} supertype.


\subsection{Commonalities and Manifestations}

Concept metamodel it belongs to ...

Each manifestation represents the realization of the concept represented by the \commonality in another metamodel by one or more classes.
Such a manifestation class only references an ordinary class, but it may additionally have an alias for referencing it and it may be declared as \emph{single} to denote that there may only be a single instance of it, such as the model root containing all other elements.
The classes of a manifestation may either be ordinary classes of a \concretemetamodel, or they may be \commonalities of a \conceptmetamodel.


Participations and participation conditions
call them participations or manifestation?

conditions restrict at model level rather than type level

\subsection{Properties and Operators}
Implemented operators:
* Attribute mappings operators: Mappings between attributes
* Attribute reference operators: Mappings between attributes that serve as references, e.g., subpackages in UML and concept metamodel as explicit references whereas encoded as namespace attributes in classes in Java -> namespace attribute of Java class is mapped to package structure in UML
* Condition operators

References realized by participations and commonalities referencing other commonalities

Reuse mechanisms, libraries of operators

Discuss that in fact this is comparable to declarative transformation languages, esp. Mappings, but also QVT-R, so no deeper discussion.

Bidirectionalization -> Max' Diss




% \begin{copiedFrom}{VoSE}

% \section*{Language Description}

% As introduced before, our realization of the \commonalities language
% %for the previously explained \commonalities approach
% provides an internal concept definition and uses the \conceptmetamodels as additional metamodels in the operationalization.
% An example for the syntax of the \commonalities language is depicted in \autoref{lst:quality:commonalities_language_example}.

% % \begin{figure}
% %     \centering
% %     \todo{Potentially extend running example so that references are covered. At least, we do not have the package in the example}
% %     \includegraphics[width=\columnwidth]{figures/commonalities_language_example.PNG}
% %     \caption{An example for defining the common concept of components}
% %     \label{fig:commonalities_language_example}
% % \end{figure}

% % \lstinputlisting[language=commonalities, float, belowskip=-0.8 \baselineskip,
% %     caption={[Exemplary commonality for components]An exemplary specification of the \texttt{Component} \commonality between \gls{PCM}, UML and the object-oriented design concept in the \commonalities language},
% %     captionpos=b,
% %     label=lst:quality:commonalities_language_example,
% % ]{listings/quality/commonalities_language_example.lst}

% The language allows to define \conceptmetamodels by declaring \commonalities, each representing one commonality between different manifestations, such as the \texttt{Component} \commonality in our example.
% Relations between the \conceptmetamodels and their manifestations are supposed to be specified \emph{declaratively}.
% %, which is realized as a \metaclass in the \conceptmetamodel.
% For every \commonality, the \metaclasses in the manifestations that realize them are specified.
% In the example, the \texttt{Component} in \gls{PCM} and the \texttt{Class} in the object-oriented design \conceptmetamodel are %defined as representations of
% related to the \texttt{Component} \commonality.
% In our language, a \commonality is realized by a \metaclass in the metamodel that is generated for a concept, so the \texttt{Component} \commonality is realized by a \texttt{Component} \metaclass.

% Within a \commonality, attributes and references can be defined, similar to an ordinary \metaclass.
% The relations of an attribute to the manifestation are declared directly at the attribute.
% In the example, a \texttt{name} attribute is specified, which maps to the name of the component in \gls{PCM} and the name appended with an \enquote{Impl} suffix in Java.
% The language %defined by \textcite{gleitze2017a} 
% provides several operators for attribute relations, apart from equality relations.
% The example depicts a prefix operator that allows to compose a String attribute.
% Such operators can be defined independently and added to the language dynamically.
% References can be defined comparably to attributes but can be enriched with a definition of containment relations.

% % Introduce the idea of \enquote{Concepts} and \enquote{Commonalities}, explain how attributes and references are mapped.
% % \todo{We have to align the definition of Concept, Commonality etc. in the paper with the implementation}

% The actually conceptualized and implemented language by \textcite{gleitze2017a} is far more sophisticated than the simple overview we provide here. 
% It supports different kinds of bidirectional operators for attribute mappings, containment specifications (so-called \emph{participations}), attribute checks as preconditions for \commonality instantiation, and more.

% \end{copiedFrom} % VoSE

\section{Expected Benefits}

Especially usability benefits, no conceptual benefits (in contrast to the concept as such)

\subsection{Improving Comprehensibility}

\begin{figure}
    \centering
    \includegraphics[width=\textwidth]{figures/quality/language/benefit_comprehensibility.png}
    \caption[Benefit of \commonalities regarding comprehensibility]{Example for consistency relations between classes and components expressed with \qvtr and the \commonalities approach.}
    \label{fig:language:benefit_comprehensibility}
\end{figure}

\subsection{Reducing Specification Effort}

Further benefit: Specification effort, adding concrete metamodel, adapting commonality, easier with internal specification


\section{Summary}

\begin{insight}[Language]

\end{insight}


