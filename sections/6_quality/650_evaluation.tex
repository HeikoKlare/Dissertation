\chapter{Evaluation and Discussion 
    \pgsize{15 p.}
}
\label{chap:commonalities_evaluation}

\mnote{Summary of contributions}
In the preceding chapters \ref{chap:classification}--\ref{chap:language}, we have discussed quality property of transformation networks and how they can be systematically improved.
We have discussed the effect of the network topology on properties, and we have derived the \commonalities approach for constructing transformation networks, which uses the effects of topologies to optimize specific quality properties.
Finally, we have proposed the \commonalities language and discussed central aspects of it, which supports the process of applying the \commonalities approach to define a transformation network.

\mnote{Trade-off mitigation by construction}
The central benefit of the developed \commonalities approach and the supporting \commonalities language is given by construction.
The way in which the transformation network is defined inherently improves correctness, especially in terms of compatibility (cf.\ \autoref{chap:compatibility}), and reusability, which are contradicting quality properties in a network of transformations that are directly defined between the metamodels whose instances are to be kept consistent.
We have argued this trade-off mitigation in more detail in \autoref{chap:improvement:benefits:properties}.
In addition to this central benefit, we discussed some further benefits that we expect from both the \commonalities approach as well as the \commonalities language in \autoref{chap:improvement:benefits} and \autoref{chap:language:commonalities:benefits}.
We empirically evaluate these benefits with a case study presented in this chapter.

\mnote{Empirical evaluation of \commonalities approach}
In the discussions of \autoref{chap:improvement} and \autoref{chap:language}, two general issues affecting the \commonalities approach remained that may only be solved by empirical investigations.
First, although consistency relations and their preservation are only described in a different way by means of auxiliary models, it may be possible that the approach restricts the possible consistency relations that can be described in any way, especially under the goal of achieving a consistency relation tree.
Second, the achievement of a consistency relation tree with the approach is important to maximize the compatibility guarantee while ensuring maximal reusability (see \autoref{chap:improvement:commonalities:tree}), but it is unclear how far or under which conditions this tree can be achieved in practice.

\mnote{Empirical evaluation of \commonalities language}
The \commonalities language provides the benefits of the \commonalities approach, but, as discussed in \autoref{chap:language:commonalities:benefits}, we also expect it to reduce the specification effort, which would actually increase with the \commonalities approach in comparison to an ordinary transformation network when employing existing tools for defining the auxiliary metamodels and transformations to them.
We have thus developed a prototype of that language and evaluate its correctness, as well as the goal of reducing specification effort in a case study.


\section{Goals and Methodology}

\mnote{Completeness, applicability and benefits}
In this evaluation, we aim to validate relevant properties of the \commonalities approach and the \commonalities language that are not given by their construction but have to be analyzed empirically.
This especially concerns the applicability of the approach and specific benefits provided by the language, but also the general completeness of the approach, i.e., the ability to express every required set of consistency relations.

\begin{table}
    \renewcommand{\arraystretch}{1.4}%
    \rowcolors{1}{\secondlinecolor}{\firstlinecolor}
    \begin{tabular}{L{8.2em} L{20em}}
        \toprule
        \rowcolor{\headinglinecolor}
        \goal{Approach} & 
            Show that transformation developers can use the \commonalities approach to specify consistency and its preservation between multiple models. \\
        \question[eq:commonalities:completeness]{Completeness} & 
			\questiontext{How far are the \commonalities approach and the \commonalities language capable of defining arbitrary conistency relations?} \\
        \metric & 
			\metrictext{Definition ratio: Ratio of consistency relations for which consistency can successfully be defined} \\
		\question[eq:commonalities:practicality]{Practicality} & 
            \questiontext{How far is the goal of achieving a consistency relation tree with \commonalities achievable in practice?} \\
        \metric & 
            \metrictext{Cross-tree ratio: Number of cross-tree relations compared to the number of relations} \\
            \midrule
    \end{tabular}
    \rowcolors{1}{\secondlinecolor}{\firstlinecolor}
    \begin{tabular}{L{8.2em} L{20em}}
        \rowcolor{\headinglinecolor}
        \goal{Language} & 
			Show that transformation developers can define consistency in a concise way with the \commonalities language. \\
		\question[eq:language:correctness]{Correctness} & 
			\questiontext{Do transformations generated by specifications in the \commonalities language preserve consistency according to the defined relations?} \\
        \metric & 
            \metrictext{Preservation ratio: Ratio of scenarios in which consistency can successfully be preserved} \\
        \question[eq:language:benefit]{Benefit} & 
            \questiontext{How much more concise is a specification in the \commonalities language to a specification in the \reactionslanguage?} \\
        \metric & 
            \metrictext{Code ratio: Ratio between the number of \acrshort{SLOC} in a \commonalities specification compared to the number of \acrshort{SLOC} with a \reactions specification for the same transformations} \\
        \bottomrule
    \end{tabular}
    \caption[Goals, questions, metrics for \commonalities]{Goals, questions and metrics for \commonalities approach and language.}
    \label{tab:commonalities_evaluation:gqm}
\end{table}

\mnote{Empirical evaluation in case study}
In the following, we present an empirical evaluation based on a case study, in which we apply a prototypical realization of the \commonalities language to consistency relations and their preservation in the domain of component-based software engineering, which we have already employed for the evaluation of our contributions regarding the construction of correct transformation networks in \autoref{chap:correctness_evaluation:categorization}.
We summarize the general goals of the evaluation along with according questions and metrics as quantitative measures for answering them in \autoref{tab:commonalities_evaluation:gqm}.

\mnote{Completeness evaluation}
Regarding the \commonalities approach as such, we are interested in the possibility to be used by transformation developers to define consistency preservation.
In a first place, this comprises the validation of completeness according to \autoref{chap:classification:properties}.
We want to find out whether it is possible to define arbitrary consistency relations with the \commonalities approach. 
In fact, \citeauthor{stevens2020BidirectionalTransformationLarge-SoSym} shows that a multiary relation can be expressed by an auxiliary metamodel with binary relations between this auxiliary metamodels and the metamodels to describe consistency between.
This means that also any set of binary relations, which induce a multiary relation as discussed in \autoref{chap:correctness:notions_consistency:monolithic_modular}, can be expressed by an auxiliary metamodel and binary relation between it and the metamodels to define consistency between.
This conforms to general idea of the \commonalities approach and, if recursively applied, even to the hierarchic composition of \commonalities.
Despite this theoretical insight, we investigate whether such a specification is actually achievable in practice, especially under the specific goal of achieving a consistency relation tree in a specification of \commonalities.
Even if the \commonalities approach itself may not be restricted in expressiveness, the proposed \commonalities language may be because of the selected way in which \commonalities and their relations are defined.
This leads to \autoref{eq:commonalities:completeness}, which we aim to answer by measuring how many consistency relations of our case study we are able to implement:
\begin{align*}
    &
    \mathvariable{definition\ ratio} = \frac{\mathtextspacearound{\# of defined consistency relations}}{\mathtextspacearound{\# of total consistency relations}}
\end{align*}
The more consistency relations we are able to express, the higher it is an indicator for the completeness of the approach and the language. 
It does, however, especially indicate completeness of the \commonalities language, such that we derive by argumentation whether restrictions in expressiveness exist only because of the language or already because of restrictions of the \commonalities approach.
The language especially serves as a means to draw conclusions about completeness of the approach.

\mnote{Practicality evaluation}
For the \commonalities approach to provide the benefit of inherently guaranteeing compatibility, it must be possible to define a consistency relation tree by means of the additional \conceptmetamodels and their \commonalities.
In this first place, we aim to identify whether such a tree can be defined at all.
We do not aim to systematically find conditions under which this is possible or even how the \commonalities approach and the \commonalities language can systematically support this.
Knowing whether the specification of such a tree is achievable at all is a preliminary for these further investigations, which we refer to as future work.
It identifies practicality of the approach, as considered in \autoref{eq:commonalities:practicality}.
To this end, we measure in our case study how many of the defined relations are cross-tree relations, i.e., violate the definition of a consistency relation tree:
\begin{align*}
    &
    \mathvariable{cross\mbox{-}tree\ ratio} = \frac{\mathtextspacearound{\# of cross-tree consistency relations}}{\mathtextspacearound{\# of defined consistency relations}}
\end{align*}
In the best case, this ratio is $0$, such that the relations actually form a consistency relation tree.
Referring to \autoref{def:relationtree} for consistency relation trees, we consider the graph induced by the relations defined by the manifestation relations of a \commonalities specification between \metaclasses of the \concretemetamodels and \conceptmetamodels, in which they are called \commonalities.
We only consider the actually defined consistency relations, as we cannot make statements about the relations that we do not express by \commonalities in the case study.

\mnote{Correctness evaluation}
Regarding the \commonalities language, we are most interested in finding indicators for improving usability of the \commonalities approach by providing a concise way of specification.
First of all, this requires that language operates correctly, i.e., that is actually delivers transformations that preserve consistency according to the defined consistency relations, as defined in \autoref{eq:language:correctness}.
This actually evaluates two correctness notions. 
First, it identifies whether the language implementation is correct at a technical level.
Second, it identifies whether the concepts for operationalizing \commonalities into transformations defined with the \reactionslanguage, as proposed in \autoref{chap:language:commonalities:operationalization}, are correct.
We measure this by executing several change scenarios and identifying whether the results are consistent to the specified relations:
\begin{align*}
    &
    \mathvariable{preservation\ ratio} = \frac{\mathtextspacearound{\# of successfull scenarios}}{\mathtextspacearound{\# of total scenarios}}
\end{align*}
In the best case, this metric evaluates to $1$, such that in all scenarios consistency can successfully be preserved.
In failure cases, we manually investigate the cause, especially distinguishing between conceptual issues in the operationalization of the \commonalities language and technical faults in the compiler implementation.

\mnote{Benefit evaluation}
As an essential benefit of the \commonalities language, we motivated the reduction of specification effort (see \autoref{chap:language:commonalities:benefits}).
This is of particular importance, because developing a \commonalities specification for consistency between two metamodels by means of existing tools for metamodel and transformation definition requires the definition of three artifacts compared to a single artifacts when defining an ordinary transformation.
The \commonalities language aims to resolve this issue.
We consider the specification effort by means of conciseness, i.e., the size of a specification with \commonalities in comparison to a specification of ordinary transformations between the metamodels, as defined in \autoref{eq:language:benefit}.
Since the \commonalities language compiles to \reactions and a comparable implementation of the case study already exists for them (see \autoref{chap:correctness_evaluation:categorization:case_studies}), we compare the size of a \commonalities specification with the size of a specification in the \reactionslanguage in terms of the \gls{SLOC} and measure the following metric:
\begin{align*}
    &
    \mathvariable{code\ ratio} = \frac{\mathtextspacearound{\# of \acrshort{SLOC} with \commonalities}}{\mathtextspacearound{\# of \acrshort{SLOC} with \reactions}}
\end{align*}
The lower the value of this metric, the more concise a specification in the \commonalities language can be expected to be compared to a specification in the \reactionslanguage.
We expect this insight in conciseness to correlate with the required specification effort.


\section{Prototypical Implementation}


\section{Case Study}

* 15 Commonalities, 8 in OO, 7 in CBS, overview in A.2 of Hennig
\begin{table}
	\small
	\centering
	\rowcolors{1}{\firstlinecolor}{\secondlinecolor}
	\begin{tabular}{l S[table-format=2] S[table-format=2] S[table-format=1] S[table-format=2, table-column-width=2.5em] r}
		\toprule
		\multirow{2}{*}{\bfseries Element Type} & {\multirow{2}{*}{\bfseries Total}} & \multicolumn{4}{c}{\bfseries Covered} \\
		\cmidrule(lr){3-6}
		& & \multicolumn{1}{c}{Direct} & \multicolumn{1}{c}{Implicit} & \multicolumn{2}{c}{Overall} \\
		\midrule
		\Metaclasses 				& 16 & 7  & 2 		& 9 & \SI{56}{\percent}  \\
		Attributes 					& 15 & 7  & 1 		& 8 & \SI{53}{\percent} \\
		Containment References 		& 18 & 6  & 0 		& 6 & \SI{33}{\percent}  \\
		Non-containment References 	& 8  & 3  & 1  		& 4 & \SI{50}{\percent}  \\
		Enumerations 				& 1  & 0  & 1  		& 1 & \SI{100}{\percent}  \\
		\midrule
		Total 						& 58  & 23  & 5  	& 28 & \SI{48}{\percent}  \\
		\bottomrule
	\end{tabular}
	\caption[Number of case study elements of \gls{PCM}]{Numbers of elements from the \gls{PCM} metamodel used in the case study. Adapted from~\cite[Table 10.3]{hennig2020ma}.}
	\label{tab:commonalities_evaluation:coverage_pcm}
\end{table}

Total elements relevant for case studies (as depicted in \autoref{chap:correctness_evaluation:compatibility:case_study}), directly covered elements (those as manifestations / participations in \commonalities) and indirectly covered elements (referenced within operators but not as participations, because they are only relevant when transferring information of other elements but not on their own).
Those internal elements are primitive data types, enums etc., but also UML interface realizations due to indirect reference of generalization which cannot be mapped yet, same as for provided role in PCM

Many Java elements internally covered, because primitive types, type references and modifiers are represented as metaclasses, which are not mapped on their own but only in context of an operator

\begin{table}
	\small
	\centering
	\rowcolors{1}{\firstlinecolor}{\secondlinecolor}
	\begin{tabular}{l S[table-format=2] S[table-format=2] S[table-format=1] S[table-format=2, table-column-width=2.5em] r}
		\toprule
		\multirow{2}{*}{\bfseries Element Type} & {\multirow{2}{*}{\bfseries Total}} & \multicolumn{4}{c}{\bfseries Covered} \\
		\cmidrule(lr){3-6}
		& & \multicolumn{1}{c}{Direct} & \multicolumn{1}{c}{Implicit} & \multicolumn{2}{c}{Overall} \\
		\midrule
		\Metaclasses 				& 13	& 7		& 3	& 10	& \SI{77}{\percent}	\\
		Attributes 					& 27	& 19	& 1	& 20	& \SI{74}{\percent}	\\
		Containment References 		& 13	& 9		& 0	& 9		& \SI{69}{\percent}	\\
		Non-containment References 	& 4		& 2		& 2	& 4		& \SI{100}{\percent}	\\
		Enumerations				& 2		& 0		& 2	& 2		& \SI{100}{\percent}	\\
		\midrule
		Total 						& 59	& 37	& 8	& 45	& \SI{76}{\percent}	\\
		\bottomrule
	\end{tabular}
	\caption[Number of case study elements of \gls{UML}]{Numbers of elements from the \gls{UML} metamodel used in the case study. Adapted from~\cite[Table 10.4]{hennig2020ma}.}
	\label{tab:commonalities_evaluation:coverage_uml}
\end{table}


\begin{table}
	\small
	\centering
	\rowcolors{1}{\firstlinecolor}{\secondlinecolor}
	\begin{tabular}{l S[table-format=2] S[table-format=2] S[table-format=2] S[table-format=2, table-column-width=2.5em] r}
		\toprule
		\multirow{2}{*}{\bfseries Element Type} & {\multirow{2}{*}{\bfseries Total}} & \multicolumn{4}{c}{\bfseries Covered} \\
		\cmidrule(lr){3-6}
		& & \multicolumn{1}{c}{Direct} & \multicolumn{1}{c}{Implicit} & \multicolumn{2}{c}{Overall} \\
		\midrule
		\Metaclasses 				& 30	& 9		& 11	& 20	& \SI{67}{\percent}	\\
		Attributes 					& 13	& 11	& 0		& 11	& \SI{85}{\percent}	\\
		Containment References 		& 32	& 19	& 1		& 20	& \SI{63}{\percent}	\\
		Non-containment References 	& 1		& 0		& 0		& 0		& \SI{0}{\percent}	\\
		Enumerations				& 0		& 0		& 0		& 0		& \SI{100}{\percent}	\\
		\midrule
		Total 						& 76	& 39	& 12	& 51	& \SI{67}{\percent}	\\
		\bottomrule
	\end{tabular}
	\caption[Number of case study elements of Java]{Numbers of elements from the Java metamodel used in the case study. Adapted from~\cite[Table 10.5]{hennig2020ma}.}
	\label{tab:commonalities_evaluation:coverage_java}
\end{table}

FROM LUKAS: 
Of the total \textbf{193} identified element types, our commonalities cover \textbf{124} (\textbf{64.25\%}) elements in total, \textbf{99} of which are covered directly and \textbf{25} are covered internally by operators. The largest portion of elements that are not covered is associated with PCM (\textbf{52\%}). We mostly covered all UML and Java elements that we considered in our cases study (\textbf{98\%} and \textbf{96\%} respectively), but of the total identified elements we only cover \textbf{76\%} and \textbf{67\%}.


NOT COVERED:
* Composite Components: Same as basic ones, need to be distinguished (e.g. by naming schema, user interaction or containment of other components), but then are comparable to basic ones, so no further insights
* Provided/Require Role, Generalization internally use non-containment references (discussed above)
* System: no covered intentionally, is basically only a composite component
* CollectionDataType: complicated to map between multiplicities in Java and Collection data types in Java and PCM. Other data types are already mapped. Since each type can only be related to one kind of data type, there are not further problems to be expected regarding a tree structure. It is basically doing and potentially limitations of existing operators of the language, but no conceptual problems.


\section{Results and Interpretation}

Test cases always multiple of 6 (CBS) and multiple of 2 (OO) because each test scenario covered by six/two cases for each direction (?)

\begin{table}
	\small
	\centering
	\rowcolors{1}{\firstlinecolor}{\secondlinecolor}
	\begin{tabular}{L{10em} S[table-format=3, table-column-width=6em] S[table-format=3, table-column-width=5em] r}
		\toprule
		\multicolumn{1}{l}{\bfseries Test Group} & \multicolumn{1}{c}{\bfseries Test Cases} & \multicolumn{2}{c}{\bfseries Successful Test Cases} \\
		\midrule
		Repository 				& 6		& 6		& \SI{100}{\percent} \\
		ComponentInterface 		& 6		& 6		& \SI{100}{\percent} \\
		Component 				& 6		& 6		& \SI{100}{\percent} \\
		CompositeDataType 		& 48	& 48	& \SI{100}{\percent} \\
		Operation 				& 48	& 48	& \SI{100}{\percent} \\
		% PCM/UML -> Java
		ProvidedRole 			& 12	& 2		& \SI{17}{\percent} \\
		\midrule
		Total 					& 126	& 116	& \SI{92}{\percent} \\
		\bottomrule
	\end{tabular}
	\caption{Test groups for CBS. Adapted from~\cite[Table 10.1]{hennig2020ma}.}
	\label{tab:commonalities_evaluation:tests_cbs}
\end{table}

Provided role failures: Faulty definition of operator - provided interface of a role is only used in operator but not defined as parameter/participation, such that no reaction for its change is created
Limited expressiveness: Language cannot express mapping between non-containment reference (in Java) to containment reference with object (role) + non-containment reference (interface) (in PCM).

\begin{table}
	\small
	\centering
	\rowcolors{1}{\firstlinecolor}{\secondlinecolor}
	\begin{tabular}{L{10em} S[table-format=3, table-column-width=6em] S[table-format=3, table-column-width=5em] r}
		\toprule
		\multicolumn{1}{l}{\bfseries Test Group} & \multicolumn{1}{l}{\bfseries Test Cases} & \multicolumn{2}{c}{\bfseries Successful Test Cases} \\
		\midrule
		Package 			& 6		& 6		& \SI{100}{\percent} \\
		Interface 			& 10	& 10	& \SI{100}{\percent} \\
		InterfaceMethod 	& 28	& 28	& \SI{100}{\percent} \\
		Class 				& 26	& 26	& \SI{100}{\percent} \\
		Property 			& 20	& 20	& \SI{100}{\percent} \\
		ClassMethod 		& 40	& 40	& \SI{100}{\percent} \\
		% 1 scenario disabled: multi constructor
		Constructor 		& 24	& 23	& \SI{96}{\percent} \\
		\midrule
		Total 				& 154	& 153	& \SI{99}{\percent} \\
		\bottomrule
	\end{tabular}
	\caption{Test groups for OO. Adapted from~\cite[Table 10.2]{hennig2020ma}.}
	\label{tab:commonalities_evaluation:tests_oo}
\end{table}

OO Constructor: Problem in \vitruv framework -- after two constructor creations, first both constructors are propagated before propagating the insertion of parameters. This results in two indistinguishable constructors (with empty parameter lists) in Java, which results in problems of the subsequent parameter propagation for one of them.
Adding one constructor after another (either in the test, i.e., by the user, or as propagation of the framework) would solve the problem.


MEDIA STORE

* From PCM to UML
* Only elements already supported (components, interfaces, data types, signatures with parameters)
* Missing: provided and required roles


\begin{table}
	\small
	\centering
	\begin{tabular}{lrrrr}
		\toprule
		\multicolumn{1}{l}{\bfseries } & \multicolumn{1}{r}{$CS_{{UML}\leftrightarrow{Java}}$ (omitted)} & \multicolumn{1}{r}{$CS_{{UML}\leftrightarrow{Java}}$ (covered)} & \multicolumn{1}{r}{OO} & \multicolumn{1}{r}{$\Delta$}\\
		\midrule
		Specifications 			& 302	& 2390 	& 514 	& -1876 \\
		Utilities				& 445	& 2250	& 2523 	& 273 \\
		\midrule
		Total					& 747	& 4640 	& 3037 	& -1603\\
		\bottomrule
	\end{tabular}
	\caption[Lines of code in the consistency specifications for UML and Java]{Lines of code in the consistency specifications for UML and Java ($\Delta$ is the difference from $CS_{{UML}\leftrightarrow{Java}}$ (covered) to OO)}
	\label{tab:commonalities_evaluation:reactions_comparison}
\end{table}



\section{Discussion and Validity}

Tree only considered for defined relations: potentially further relations would have violated tree structure. However, if relations cannot be defined by the approach at all, which would violate the tree structure, that would at least prevent the definition of tree violations.
Same for correctness: only considered for the relations that we were able to express (completeness)

Tree is what we want to achieve, but it is not necessary. Violations of the tree structure only introduce potential incompatibility and potential cycles of propagation, requiring synchronization. This must, however, not be problematic, first, because a cycle in the relations tree must not necessarily lead to a cycle between the elements in an instance, as the transformation may always select other elements that do not lead to a cycle, and second, because even if there is a cycle, it can be implemented properly (compatible and with synchronization, see chapter), such that it properly works as an ordinary transformation network built according to what we discussed in \autoref{part:correctness}.

Benefit: reactions are imperative, more declarative languages exist -> no fair comparison. But reactions already compared to ordinary code (refer to paper) and show reduction there, so they are verbose but not arbitrarily verbose
Only one case study: high bias
Correlation between conciseness and specification effort unclear


\section{Limitations and Future Work}
Limited to descriptive specification

Language does not provide inheritance for commonalities, repeated definition of names, repeated definition of shared information, e.g., between basic and composite components etc.

Limited set of operators, e.g., no generic operator for mapping attributes to references (e.g. visibility is EnumLiteral in UML, whereas it is a reference to a type instance in Java), complex pattern mappings, e.g., type reference in UML is direct reference, whereas in Java a reference contains a reference object, which then references a further object that references the type; or primitive types in Java represented as types and in UML as predefined instances of a generic primitive type type.

Movement of objects treated as removal and insertion leading to deletion of contained elements and values


Granularity of operators: open research question how to define a reusable set with appropriate degree of abstraction, such that only for few cases custom operators need to be defined. Especially necessary to define a metrics for that (what are "few cases"?). Currently left open in language design (see diagram with arbitrary operators) and implemented specific for case-study (as focus on structure of language, not on realization of manifestation relations).

Tree: Developer has to ensure that a tree is achieved. Can be improved by the language. Integrate compatibility approach from \autoref{chap:compatibility} into the language. Due to rather declarative specification, graph of consistency relations can be derived and analyzed. Assumptions have to be made on operators, which can do arbitrary stuff, e.g. that only the elements explicitly given to the operator are used, and no further navigation through the model is done within the operator. With that assumption, we can construct the graph from the commonalities specifications and operator arguments and apply the analysis.
Who is responsible for the tree structure? How does the language support its achievement?

\section{Summary}



\begin{copiedFrom}{VoSE}

\section*{Proof-of-Concept}

We have proposed the \commonalities approach and a realizing language.
We have explained that we expect them to improve understandability of transformations and to reduce problems of transformation networks, such as compatibility and modularity.
Although we gave arguments that justify this expectation, it has to be evaluated empirically to increase evidence.
However, before evaluating the benefits of our approach, we first have to investigate its feasibility.
For that reason, we built an initial prototype of the language and applied it to a simple evaluation case as a proof-of-concept.
%to show the feasibility of the concept.
%This forms our contribution~\ref{contrib:proofofconcept}.

% Evaluation Goal: Show feasibility of the idea

% Evaluation Methodology: Build a proof-of-concept prototype of a language and apply it to a simple case

\subsection*{Case Study}

We have implemented a prototype of the \commonalities language, which allows to define \commonalities with simple attribute and reference mappings and to compose \commonalities.
The syntax is an extension of the examples shown in \autoref{lst:language:class_example} and \autoref{lst:language:component_example}.
The language comprises a compiler that derives a \conceptmetamodel, as well as a set of transformations from a specification in the language.
The generated transformations are in turn defined in the \reactionslanguage~\cite{klare2016b}, which is a delta-based transformation language that is, just like the \commonalities language itself, part of the \vitruv approach~\cite{kramer2013b}.
\vitruv is a view-based development approach that uses transformations to keep models consistent.
The implementation of the \commonalities language can be found in the GitHub repository of the \vitruv project \cite{vitruvFrameworkGithub}. %\footnote{\label{githubfootnote}\url{https://github.com/vitruv-tools/Vitruv}}.

We have applied the implementation to a simple case study that consists of four metamodels, each containing one \metaclass that represents a root element and one that represents a contained element. 
Both elements have an identifier and a name in all metamodels, and an additional single-valued and multi-valued feature of integers in two of the metamodels.
The root \metaclass additionally has a containment reference to the contained \metaclass.
We have defined two \commonalities, one for the root element and one for the contained element, which redundantly represent the same concepts in all the metamodels.
The root \commonality references the contained \commonality.
This results in one \conceptmetamodel with four manifestations.

To validate that the specifications in the \commonalities language are correctly defined and operationalized, we have
%defined 8 test cases that 
defined test cases that perform 21 different model modifications, which 
%
create and delete all possible types of elements and modify all attribute and reference values in instances of every metamodel.
They cover the set of all possible modifications that can be performed on instances of those metamodels.
This also includes change propagation across composed \commonalities.
The tests successfully validate that the modifications are correctly propagated to all other models in all cases.
The test cases and the used example metamodels are also available in the GitHub repository of the \vitruv project \cite{vitruvFrameworkGithub}. %\footnotemark[1].

% \begin{itemize}
%     \item First reference to concrete implementation here
%     \item Refer to simple example implementation showing the ability to define and run the Commonalities approach
% \end{itemize}

\subsection*{Discussion}
% Formerly: Threats to Validity

%Maybe we can omit this section, but we need to discuss that it is really, really only a proof-of-concept (some reviewers do not understand the difference to a complete benefit, scalability and effectiveness evaluation if you do not tell them again and again).

Our proof-of-concept validates the feasibility of the proposed \commonalities approach: 
It demonstrates that it is possible to apply the concept of defining consistency relations between multiple metamodels  through a central metamodel in a simple scenario. It also shows that an operationalization can be derived that preserves consistency of all instances of such metamodels.
%Our evaluation only serves as a proof-of-concept to validate feasibility of the approach that we present in this paper.
%In consequence, 
The results only give an indicator that the \commonalities concept can be applied and that a language with an internal concept definition can be designed.
To further evaluate the capabilities of such an approach, the language would have to be extended to be able to define more complex relationships.
Additionally, the approach has to be applied to larger parts of more complex metamodels and metamodels for different contexts to improve external validity of the results.
This could also reveal whether the assumption of having a tree of \commonalities is practical in realistic scenarios.

Since evaluating functional capabilities of the approach is only an---essential---first step, the evaluation of further properties such as applicability, appropriateness, effectiveness and scalability are part of ongoing work with further case studies.
As a central benefit of our approach, we claim to improve understandability of relations between metamodels, but can only give arguments for that by now.
An evaluation of that claim would require a user experiment that compares our approach to specifications of direct transformations between multiple metamodels.

Finally, one might argue that defining \conceptmetamodels leads to additional effort, as for two metamodels it is necessary to define one additional metamodel and two transformations rather than only a single transformation.
First, this is only true as long as only two metamodels are related by one \conceptmetamodel. 
If three metamodels shall be related, there would be a network of three transformations, which are not necessarily compatible without using the \commonalities approach, and one metamodel with three transformations using the \commonalities approach.
When the \commonalities approach is applied, the number of necessary transformations increases linearly with the number of metamodels that are related, whereas it increases quadratic without them.
Second, the effort for defining transformations can be reduced by using an appropriate language to define \conceptmetamodels and transformations, as we have proposed in \autoref{chap:language}. Our language only requires developers to write one specification that contains both the \conceptmetamodel and all transformations to its manifestations.

% \subsection{Limitations}
% Contradictory Commonalities specifications?

\end{copiedFrom} % VoSE


