\section{Summary}

In this chapter, we have discussed which software quality properties, as defined in an ISO standard~\cite{iso25010}, are relevant for developers of transformation networks.
In addition, we have identified two extremes of transformation network topologies and discussed their impacts on quality properties.
From this discussion, we were are able to derive necessary trade-offs between the properties induced by the topology of the network.
We conclude this chapter with the following central insight.

\begin{insight}
    In addition to functional correctness of transformation networks, further quality properties can be relevant for developers and user of such networks.
    For developers of transformations networks, in particular functional completeness, i.e., the ability to apply transformation networks to any situation in which consistency between models needs to be preserved, and different aspects of maintainability, such as modularity, reusability, analyzability, modifiability and testability, are important.
    Transformation networks induce a graph of metamodels and transformations between them that can, at one extreme, be a complete graph, in which each pair of metamodels is directly related by a transformation, and, at the other extreme, be a tree, in which each pair of metamodels is only related by one path of transformations.
    While networks inducing a complete graph inherently optimize completeness and reusability, those inducing a tree inherently optimize correctness.
    Although trees are particularly restrictive regarding completeness and in practice networks inducing a tree are thus hard to achieve, their inherent correctness guarantee makes them still interesting, as they avoid multiple challenges to achieve correctness.
\end{insight}