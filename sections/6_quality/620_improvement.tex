\chapter{Mitigating Trade-offs with Commonalities 
    \pgsize{35 p.}
}
\label{chap:improvement}

\mnote{Topologies and Properties}
We have identified in the previous chapter that the topology of the graph induced by the metamodels and transformations of a transformation network directly influences several of its quality properties, such as function correctness and completeness, as well as maintainability in terms of modularity and reusability.
The extreme topologies of complete graphs and trees imply extremes in the optimization or degradation of these properties, which induces a trade-off between these properties by means of the topology.

%\todo{Start discussion from topologies and properties.
%Then again start the discussion from the idea of common concept definition.
%Then unify the two argumentations.}

\mnote{Benefits of trees}
In \autoref{part:correctness}, we have focused on achieving correctness for networks of arbitrary topology, thus in general not inducing a tree but any graph topology that can be extended to a complete graph, which inherently optimizes reusability and completeness, but requires high effort for achieving completeness.
On the other hand, a tree structure, although not that easy to achieve, provides inherent correctness guarantees while reducing reusability and completeness, as identified in \autoref{chap:classification:topologies:effects}.
In this chapter, we discuss how a network having a tree topology can be constructed by introducing additional metamodels, such that correctness is still inherently given but reusability and completeness is improved.

\mnote{Common concepts}
The idea of adding metamodels is not just a conceptual necessity to improve quality properties, but also motivated by practical benefits.
Since consistency relations often define how common information represented in instances of several metamodels multiple times, we propose to represent this common information explicitly by means of additional metamodels instead.
Then, only the manifestation of this information in the models to keep consistent has to be defined rather than an implicit encoding of common information in the consistency relations between each pair of metamodel.
These manifestation relations can, of course, again be represented by transformation.
That way of specifying consistency relations in terms of explicit metamodels representing common information can inherently lead to a transformation network with a tree topology.

%\todo{Motivation for Comnmonalities idea: it is common in other domains (cf. Albers) to model the overlaps of models for preserving consistency (check slides of SFB and reference their publications)}

\mnote{Subordinate contributions}
This chapter constitutes our contribution \autoref{contrib:quality:improvement}, which consists of four subordinate contributions: a discussion of how common information can be represented explicitly in dedicated metamodels and when this is reasonable; a proposal of the \commonalities approach to construct such metamodels and transformations for describing the manifestations of common information in the original metamodels; a discussion of the expected benefits of the approach, especially in terms of mitigating trade-offs between quality properties; and finally an outlook to processes of applying the approach and of combining it with other transformations. It answers the following research question:

\researchquestionrepeat{rq:quality:topology}

\mnote{Benefits of contributions}
The insights in this chapter support transformation developers in constructing networks of correct, complete and reusable transformations.
It gives a different view on consistency and the possibilities to describe it besides consistency relations, which we expect to improve comprehensibility due to common concepts being represented explicitly rather than implicitly encoding them in consistency relations.
The proposed construction approach for transformation networks inherently improves several quality properties reducing the effort to achieve correctness of transformation networks as discussed in \autoref{part:correctness} and mitigating necessary trade-offs.
It especially improves reusability and completeness in comparison to an ordinary construction of a network having a tree topology.

\mnote{Publication of contributions}
The initial idea for the contributions in this chapter has already been published~\owncite{klare2018docsym}, as well as the proposed \commonalities approach with its expected benefits~\owncite{klare2019models}.
The approach along with a language that supports it, which we present in the subsequent chapter, has originally been developed in the Bachelor's thesis of \textcite{gleitze2017a}, which was supervised by the author of this thesis.


\section{Consistency of Common Concepts}
\label{chap:improvement:concpets}

\mnote{Alternative for consistency relations}
In \autoref{chap:introduction}, we have motivated that models describing the same system share an overlap of information that leads to dependencies or, in particular, redundancies between the models.
We have made these dependencies explicit by means of consistency relations.
In the following, we discuss an alternative consideration of redundancies, as a special case of dependencies, by means of common concepts.
We therefore provide an introductory example to be extended throughout the following considerations, explain the idea of \emph{\commonalities}, and discuss in which cases it can be reasonably applied.


\subsection{Introductory Example}

\mnote{Example metamodels}
We employ a running example from the case study introduced in \autoref{chap:foundations:case_studies} involving the \gls{PCM}, the \gls{UML} and Java.
Consistency relations comprise the common and mostly one-to-one mappings between \gls{UML} and Java as well as the mappings proposed by \textcite{langhammer2015a} to represent \gls{PCM} architecture models in Java code and in \gls{UML} class models.

\begin{figure}
	\centering
	\newcommand{\vdistance}{7.8em}
\newcommand{\hdistance}{17em}
\newcommand{\classwidth}{5.5em}
\newcommand{\labeldistance}{0.3em}
\newcommand{\mmborder}{1.7em}


\begin{tikzpicture}

\pgfdeclarelayer{bg}
\pgfsetlayers{bg,main}


% METACLASSES

\umlclassvarwidth{java_class}{}{Class}{
name\\
}{\classwidth}  

\umlclassvarwidth[, below=\vdistance of java_class.center, anchor=center]{uml_class}{}{Class}{
name\\
}{\classwidth} 

\umlclassvarwidth[, below right=0.5*\vdistance and \hdistance of java_class.center, anchor=center]{pcm_component}{}{Component}{
name\\
}{\classwidth} 


% METAMODELS

\coordinate (java_label_coordinate) at ([yshift=\labeldistance]java_class.north);
\node[mmlabel, anchor=south] (java_label) at (java_label_coordinate) {Java};

\coordinate (uml_label_coordinate) at ([yshift=-\labeldistance]uml_class.south);
\node[mmlabel, anchor=north] (java_label) at (uml_label_coordinate) {UML};

\coordinate (pcm_label_coordinate) at ([xshift=7*\labeldistance]pcm_component.east);
\node[mmlabel, left=6*\labeldistance of pcm_label_coordinate, anchor=west] (pcm_label) {PCM};

\begin{pgfonlayer}{bg}
    \node[mmbg, fit=(java_class)(java_label_coordinate), inner sep=\mmborder] (java) {};
    \node[mmbg, fit=(uml_class)(uml_label_coordinate), inner sep=\mmborder] (uml) {};
    \node[mmbg, fit=(pcm_component)(pcm_label_coordinate), inner sep=\mmborder] (pcm) {};
\end{pgfonlayer}


% CONSISTENCY RELATIONS

\draw[consistency relation] (java_class) -| node[pos=0, above right] {$cl$} node[above, pos=0.3, align=center] {$cl.name = co.name + \text{\enquote{Impl}}$} node[pos=1, above right] {$co$} (pcm_component);
\draw[consistency relation] (java_class) -- node[pos=0, below left] {$jcl$} node[right] {$jcl.name = ucl.name$} node[pos=1, above left] {$ucl$} (uml_class);
\draw[consistency relation] (uml_class) -| node[pos=0, above right] {$cl$} node[below, pos=0.3, align=center] {$cl.name = co.name + \text{\enquote{Impl}}$} node[pos=1, below right] {$co$} (pcm_component);

\end{tikzpicture}

	\caption[Consistency relations for extracts of Java, \acrshort{UML} and \acrshort{PCM}]{Simple metamodel extracts for Java, the \gls{UML}, and the \gls{PCM} and consistency relations between them. Adapted from~\owncite[Fig.~1]{klare2019models}.}
	\label{fig:improvement:running_example}
\end{figure}

\mnote{Example relations}
In the following, we start with limited subsets of the metamodels, namely the one-to-one mapping between components in the \gls{PCM} and classes in Java, whereby each component is mapped to a class but not vice versa, as depicted in \autoref{fig:improvement:running_example}.
Consistency relations require the existence of a class in the \gls{UML} and Java for each \gls{PCM} component having the component name with an \enquote{Impl} suffix by an according unidirectional consistency relation.
In addition, the consistency relations require an equally-named \gls{UML} class for each Java class and vice versa.
We extend the example in the following sections to explain the introduced concepts.


\subsection{Explicit Commonalities}
\label{chap:improvement:concepts:explicit}

\mnote{Common concepts}
In the given example, classes are redundantly represented in Java and the \gls{UML}.
This requires them to be kept consistent, which can, for example, specified by means of an according consistency relation.
As an alternative, redundant classes in a Java and a \gls{UML} model can also be considered representations of a \emph{common concept}, more precisely the common concept of a class in general object-oriented design.
Thus, rather than expressing this redundancy implicitly by means of a consistency relation and a transformation that preserves consistency to it, we propose to make the common concept explicit in an according metamodel and descriptions of how this concept \emph{manifests} in Java and the \gls{UML}.
Then, instead of saying that each \gls{UML} class should corresponding to a Java class and vice versa, we would say that classes in the \gls{UML} and Java are both representations of the same concept of a class in object-oriented design.

\mnote{Concepts and their relations}
We denote the actual metamodels that developers instantiate and want to keep consistent as \emph{\concretemetamodels}, whereas we denote metamodels that describe the concepts that such \concretemetamodels have in common as \emph{\conceptmetamodels}.
\autoref{fig:improvement:one_commonality_example} depicts the \concretemetamodels \gls{UML} and Java with their representations of classes.
In addition, it contains a \conceptmetamodel for object-oriented design, which contains the common concept of a class, shared by the \gls{UML} and Java.
We denote a single common concept, such as a class, as a \emph{\commonality}.
Further \commonalities in object-oriented design would be interfaces or methods.
In general, a \commonality can be considered a \metaclass with the specific semantics of describing the commonalities between elements of \concretemetamodels.
We say that an element in a \concretemetamodel, such as a class in the \gls{UML} and Java, is a \emph{manifestation} of a common concept.
The relation of a \commonality to these manifestations is denoted by a manifestation relation (\emph{\manifestslabel}).
In the example, the relations would especially define that each class manifestation conforms to a common class concept having the same name and vice versa, according to the relations in \autoref{fig:improvement:running_example}.

\begin{figure}
    \centering
    \newcommand{\vdistance}{7em}
\newcommand{\hdistance}{15em}
\newcommand{\classwidth}{5.5em}
\newcommand{\labeldistance}{1.0em}
\newcommand{\mmborder}{1em}

\begin{tikzpicture}

\pgfdeclarelayer{bg}
\pgfsetlayers{bg,main}


% METACLASSES

\umlclassvarwidth{java_class}{}{Class}{
name\\
}{\classwidth}  

\umlclassvarwidth[, right=\hdistance of java_class.north, anchor=north]{uml_class}{}{Class}{
name\\
}{\classwidth} 

\umlclassvarwidth[, above right=\vdistance and 0.5*\hdistance of java_class.north, anchor=north]{oo_class}{}{Class\vphantom{p}}{
name\\
}{\classwidth} 


% METAMODELS

\coordinate (java_label_coordinate) at ([yshift=\labeldistance]java_class.north west);
\node[mmlabel, anchor=west] (java_label) at (java_label_coordinate) {Java};

\coordinate (uml_label_coordinate) at ([yshift=\labeldistance]uml_class.north east);
\node[mmlabel, anchor=east] (uml_label) at (uml_label_coordinate) {\acrshort{UML}};

\coordinate (oo_label_coordinate) at ([yshift=\labeldistance]oo_class.north);
\node[mmlabel, anchor=center] (oo_label) at (oo_label_coordinate) {Object-Oriented Design};

\begin{pgfonlayer}{bg}
    \node[mmbg, fit=(java_class)(java_label.center), inner sep=\mmborder] (java) {};
    \node[mmbg, fit=(uml_class)(uml_label.center), inner sep=\mmborder] (uml) {};
    \node[conceptmmbg, fit=(oo_class)(oo_label.west)(oo_label.east), inner sep=\mmborder] (oo) {};
\end{pgfonlayer}


% CONSISTENCY RELATIONS

\draw[manifests relation] (oo_class) -- node[stereotype, above, sloped] {\manifestslabel} (java_class);
\draw[manifests relation] (oo_class) -- node[stereotype, above, sloped] {\manifestslabel} (uml_class);

\end{tikzpicture}

    \caption[One \commonality example for object-oriented design]{\Conceptmetamodel for object-oriented design with a \texttt{Class} \commonality and its relations to the \concretemetamodels \gls{UML} and Java. Adapted from~\owncite[Fig.~2]{klare2019models}.}
    \label{fig:improvement:one_commonality_example}
\end{figure}

\mnote{Specification effort}
In fact, these manifestation relations can be considered consistency relations that are preserved by ordinary transformations.
Thus, in a first place the representation of common concepts in terms of explicit \commonalities introduces further effort, because it requires the definition of one metamodel and two transformations instead of a single transformation that relates the \metaclasses directly.
This drawback is, however, reduced by several benefits, which we discuss in \autoref{chap:improvement:benefits}, such as mitigating trade-offs between correctness and reusability as well as improving comprehensibility.
Finally, such a specification can even reduce effort due to better scalability when adding further \concretemetamodels to keep consistent.
For example, if another object-oriented language such as \cplusplus shall be kept consistent, no matter whether only with the \gls{UML} or indeed even with Java, only the manifestation relation from \commonalities in the object-oriented design \conceptmetamodel to \cplusplus has to be added.
This may only come along with some extensions of the \conceptmetamodel for information shared between \cplusplus and the \gls{UML} as well as between \cplusplus and Java that was not already shared between Java and the \gls{UML}.
This reduces the effort in comparison to defining both relations from \cplusplus to the \gls{UML} and to Java.

\mnote{Size of concept metamodels}
In general, a \conceptmetamodel must contain \commonalities for all redundancies between the \concretemetamodels to keep consistent.
In a mathematical sense, this can be considered as the union of all pairwise intersections of the \concretemetamodels.
It can, however, not be precisely expressed as such, because elements may be similarly represented in the \concretemetamodels, but they are not the same.
One manifestation of the same \commonality may contain different information or encode it differently, such as using other units, than the others.
This already illustrates the essential difference to approaches in which one central model unifies all information about a system, called a \gls{SUM} (see \autoref{chap:foundations:multiview:osm}), from which the models used by different tools are derived by projections.
Such a \gls{SUM} can be seen as the union of all \concretemetamodels, whereas \conceptmetamodels represent the union of their pairwise intersections, as illustrated in \autoref{fig:improvement:commonalities_and_sums}.

\begin{figure}
    \centering
    \newcommand{\distance}{4.6em}
\newcommand{\circlesize}{3.9em}
\newcommand{\imagesdistance}{3.8*\distance}

\begin{tikzpicture}[
    concrete font/.style={font=\footnotesize},
    concept font/.style={font=\small\bfseries},
    concept color/.style={conceptmmbg}
]

% CONCEPTS
\def\concepta{(0,0) circle (\circlesize)}
\def\conceptb{(\distance,0) circle (\circlesize)}
\def\conceptc{(0.5*\distance,\distance) circle (\circlesize)}

% Filling
\begin{scope}
    \clip \conceptb;
    \fill[concept color] \concepta;
\end{scope}

\begin{scope}
    \clip \conceptc;
    \fill[concept color] \conceptb;
\end{scope}

\begin{scope}
    \clip \concepta;
    \fill[concept color] \conceptc;
\end{scope}

% Borders
\draw \concepta;
\draw \conceptb;
\draw \conceptc;


% SUM
\def\suma{(\imagesdistance,0) circle (\circlesize)}
\def\sumb{(\imagesdistance+\distance,0) circle (\circlesize)}
\def\sumc{(\imagesdistance+0.5*\distance,\distance) circle (\circlesize)}

% Filling
\fill[concept color] \suma;
\fill[concept color] \sumb;
\fill[concept color] \sumc;

% Borders
\tikzstyle{reverseclip}=[insert path={(-\circlesize-\pgflinewidth,-\circlesize-\pgflinewidth) --
  (0, \distance+\circlesize+\pgflinewidth) --
  (\imagesdistance+\distance+\circlesize+\pgflinewidth, \distance+\circlesize+\pgflinewidth) --
  (\imagesdistance+\distance+\circlesize+\pgflinewidth, -\circlesize-\pgflinewidth) --
  (-\circlesize-\pgflinewidth,-\circlesize-\pgflinewidth)}
]

\begin{scope}
    \begin{pgfinterruptboundingbox}
    \path [clip] \suma [reverseclip];
    \path [clip] \sumb [reverseclip];
    \end{pgfinterruptboundingbox}
    \draw \sumc;
\end{scope}

\begin{scope}
    \begin{pgfinterruptboundingbox}
    \path [clip] \sumb [reverseclip];
    \path [clip] \sumc [reverseclip];
    \end{pgfinterruptboundingbox}
    \draw \suma;
\end{scope}

\begin{scope}
    \begin{pgfinterruptboundingbox}
    \path [clip] \suma [reverseclip];
    \path [clip] \sumc [reverseclip];
    \end{pgfinterruptboundingbox}
    \draw \sumb;
\end{scope}


% LABELS
\node[concept font, anchor=south] at (\imagesdistance+0.5*\distance,\distance-\circlesize) {SUM};
\node[concrete font, anchor=center, align=center] at (-0.4*\circlesize,-0.1*\circlesize) {Concrete\\Metamodel};
\node[concrete font, anchor=center, align=center] at (0.5*\distance,\distance+0.2*\circlesize) {Concrete\\Metamodel};
\node[concrete font, anchor=center, align=center] at (\distance+0.4*\circlesize,-0.1*\circlesize) {Concrete\\Metamodel};
\node[concept font, anchor=west, align=center] at (0.5*\distance+1.1*\circlesize, \distance+0.2*\circlesize) (concept_mm_label) {Concept\\Metamodel(s)};

\draw (concept_mm_label) -- (\distance,0.7*\circlesize);

\end{tikzpicture}
    %\includegraphics[width=\textwidth]{figures/quality/improvement/commonalities_and_sums.png}
    \caption[Commonalities compared to \acrlongpl{SUM}]{Sketched comparison for the scope of contents of \conceptmetamodels and \glspl{SUM}.}
    \label{fig:improvement:commonalities_and_sums}
\end{figure}

\subsection{Consistency Specification Types}
\label{chap:improvement:concepts:specification}

\mnote{Descriptive and normative specifications}
In \autoref{chap:networks:notions:normative_descriptive}, we have discussed the distinction of descriptive and normative specifications of consistency, which can be summarized as follows.
\begin{properdescription}
    \item[Descriptive Specification:] Descriptive specifications describe consistency relations that are \enquote{naturally} given when two metamodels represent common concepts redundantly or with common or dependent properties. 
    In that case, a notion of consistency already exists, formally or informally, to which the given specification must conform.
    This is, for example, the case for \gls{UML} class models and Java realizing object-oriented design.
    \item[Normative Specification:] Normative specifications prescribe consistency for metamodels for which no existing or common notion for consistency exists.
    This is especially the case if metamodels represent different abstractions or domains of a system, which have no implicit relations and for which different possibilities to relate them exist, such as an architecture description in the \gls{PCM} and its implementation in Java.
\end{properdescription}
While descriptive consistency relations between two metamodels are usually definite, such as those for object-oriented design between the \gls{UML} and Java, normative consistency relations may vary depending on the project context.
For example, several possible relations can be defined between an architecture description in the \gls{PCM} and object-oriented design, such as the realization of each component as a class, as a bean in \glspl{EJB}, or as a complete project~\cite{langhammer2017a}.

\mnote{Suitability for descriptive specification}
Describing consistency by means of \commonalities and \conceptmetamodels especially promises to be useful for descriptive consistency specifications, where a \enquote{natural} relation exists due to elements representing common concepts.
It can, however, also be used to normatively define \commonalities in terms of a normative specification.
A component \commonality can, for example, define that a component manifests as a component in the \gls{PCM} and as a class in the \gls{UML} and in Java, or, more generally, in an object-oriented design \conceptmetamodel.
This will, however, unlikely fit well for rather complex dependencies, such as a consistency relation requiring an implementation to fulfill some performance requirement.
In such a case, the complexity is in the specification of the relation anyway, which would have to be replicated when defining a \commonality between performance requirement and the implementation.
Finally, this conforms to our distinction of structural and behavioral consistency relations given in \autoref{chap:networks:notions:types}, in which the \commonalities fit well for structural relations, on which we focus in this thesis anyway.

\mnote{Generalization}
In the following, we do not distinguish whether \commonalities are defined for common concepts that exist naturally or for those which are prescribed by the definition of \conceptmetamodels and their \commonalities.
We will see that even for normative specifications \commonalities can be reasonably defined.
In \autoref{chap:improvement:application}, we also discuss how to combine ordinary transformations with the idea of \conceptmetamodels.


\section{The \commonalities Approach}
\label{chap:improvement:commonalities}

\mnote{Approach overview}
We have motivated the idea of representing common concepts of different metamodels in terms of \commonalities in explicit \conceptmetamodels rather than implicitly encoding them in direct consistency relations between the \concretemetamodels.
In the following, we discuss the specification of \conceptmetamodels and the notion of manifestation relations in more detail.
We also depict how further benefits can be generated by composing \conceptmetamodels in terms of defining a hierarchy of them.
We call this approach of defining and composing \conceptmetamodels of \commonalities the \emph{\commonalities approach}.
The mitigation of trade-offs between quality properties as the central benefit of the approach is given by the inherent possibility to achieve a specific kind of tree topology, which we derive from the approach before discussing different options for its operationalization.


\subsection{Concept Metamodels}

\mnote{Structure of metamodels}
The inherent benefits of the \commonalities approach are given by the definition of additional \conceptmetamodels, across which consistency relations are expressed, instead of defining consistency relations between the \concretemetamodels.
Conceptually, it is not that relevant how these \conceptmetamodels and the manifestation relations between to the \concretemetamodels actually look like.
Still, we discuss how elements can be represented as \commonalities in a \conceptmetamodel and which relations beyond pure redundancies representing exactly the same information they may express.

\begin{figure}
    \centering
    \newcommand{\vdistance}{12em}
\newcommand{\hdistance}{23em}
\newcommand{\innerhdistance}{7.6em}
\newcommand{\classwidth}{4.7em}
\newcommand{\labeldistance}{0.9em}
\newcommand{\mmborder}{0.9em}
\newcommand{\referenceshift}{0.9em}

\begin{tikzpicture}

\pgfdeclarelayer{bg}
\pgfsetlayers{bg,main}


% METACLASSES

\umlclassvarwidth{java_class}{}{Class}{
name\\
packageName
}{\classwidth}  

\umlclassvarwidth[, right=\innerhdistance of java_class.north, anchor=north]{java_field}{}{Field}{
name\\
}{\classwidth} 

\umlassociationfromto{([yshift=\referenceshift]java_field.west-|java_class.east) -- node[uml cardinality start, pos=0, above right] {$1$} node[uml cardinality end, pos=1, above left] {$*$} ([yshift=\referenceshift]java_field.west)}
\umlassociationfromto{([yshift=-\referenceshift]java_field.west) -- node[uml cardinality start, pos=0, above left] {$*$} node[uml role end, pos=1, below right] {type} node[uml cardinality end, pos=1, above right] {$1$} ([yshift=-\referenceshift]java_field.west-|java_class.east)}

\umlclassvarwidth[, below right=5em and \hdistance of java_class.north, anchor=north]{uml_class}{}{Class}{
name\\
}{\classwidth} 

\umlclassvarwidth[, left=\innerhdistance of uml_class.north, anchor=north]{uml_association}{}{Association}{
name\\
}{\classwidth}

\umlclassvarwidth[, above=5em of uml_class.north, anchor=north]{uml_package}{}{Package}{
name\\
}{\classwidth}

\umlassociationfromto{([yshift=\referenceshift]uml_association.east) -- node[uml cardinality start, pos=0, above right] {$*$} node[uml role end, pos=1, below left] {from} node[uml cardinality end, pos=1, above left] {$1$} ([yshift=\referenceshift]uml_class.west)}
\umlassociationfromto{([yshift=-\referenceshift]uml_association.east) -- node[uml cardinality start, pos=0, above right] {$*$} node[uml role end, pos=1, below left] {to} node[uml cardinality end, pos=1, above left] {$1$} ([yshift=-\referenceshift]uml_class.west)}
\umlassociationfromto{(uml_package.south) -- node[uml cardinality start, pos=0, below left] {$1$} node[uml role end, pos=1, above right] {classes} node[uml cardinality end, pos=1, above left] {$*$} (uml_class.north)}

\umlclassvarwidth[, above right=\vdistance and 0.5*\hdistance-0.5*\innerhdistance of java_class.north, anchor=north]{oo_class}{}{Class\vphantom{p}}{
name\\
}{\classwidth} 

\umlclassvarwidth[, below=5em of oo_class.north, anchor=north]{oo_association}{}{Association}{
name\\
}{\classwidth}

\umlclassvarwidth[, right=1.2*\innerhdistance of oo_class.north, anchor=north]{oo_package}{}{Package}{
name\\
}{\classwidth}

\umlassociationfromto{([xshift=-0.8*\referenceshift]oo_class.south) -- node[uml cardinality start, pos=0, below right] {$1$} node[uml role end, pos=0, below left] {from} node[uml cardinality end, pos=0, above right] {$*$} ([xshift=-0.8*\referenceshift]oo_association.north)}
\umlassociationfromto{([xshift=1.2*\referenceshift]oo_association.north) -- node[uml cardinality start, pos=0, above left] {$*$} node[uml role end, pos=1, below right] {to} node[uml cardinality end, pos=1, below left] {$1$} ([xshift=1.2*\referenceshift]oo_class.south)}
\umlassociationfromto{(oo_package.west) -- node[uml cardinality start, pos=0, below left] {$1$} node[uml role end, pos=1, above right] {classes} node[uml cardinality end, pos=1, below right] {$*$} (oo_class.east)}


% METAMODELS

\coordinate (java_label_coordinate) at ([yshift=\labeldistance]java_class.north west);
\node[mmlabel, anchor=west] (java_label) at (java_label_coordinate) {Java};

\coordinate (uml_label_coordinate) at ([yshift=\labeldistance]uml_package.north east);
\node[mmlabel, anchor=east] (java_label) at (uml_label_coordinate) {UML};

\coordinate (oo_label_coordinate) at ([yshift=\labeldistance]$(oo_class.north)!0.5!(oo_package.north)$);
\node[mmlabel, anchor=center] (oo_label) at (oo_label_coordinate) {Object-oriented Design};

\begin{pgfonlayer}{bg}
    \node[mmbg, fit=(java_class)(java_field)(java_label_coordinate), inner sep=\mmborder] (java) {};
    \node[mmbg, fit=(uml_class)(uml_package)(uml_association)(uml_label_coordinate), inner sep=\mmborder] (uml) {};
    \node[conceptmmbg, minimum width=11.5em, fit=(oo_class)(oo_association)(oo_package)(oo_label_coordinate), inner sep=\mmborder] (oo) {};
\end{pgfonlayer}


% CONSISTENCY RELATIONS

\draw[manifests relation] ([xshift=-0.48*\classwidth]oo_class.south) -- node[manifests relation, above, sloped] {\manifestslabel} (java_class);
\draw[manifests relation] ([xshift=0.48*\classwidth]oo_class.south) -- node[manifests relation, above, sloped] {\manifestslabel} (uml_class);
\draw[manifests relation] (oo_association) -- node[manifests relation, above, sloped] {\manifestslabel} (java_field);
\draw[manifests relation] (oo_association) -- node[manifests relation, above, sloped] {\manifestslabel} (uml_association);
\draw[manifests relation] (oo_package) -- node[manifests relation, above, sloped] {\manifestslabel} (uml_package);

\end{tikzpicture}

    \caption[Multiple \commonality example for object-oriented design]{\Conceptmetamodel for object-oriented design with a \texttt{Class}, an \texttt{Association} and a \texttt{Package} \commonality and its relations to the \concretemetamodels \gls{UML} and Java with a different representation of associations as fields and packages as attributes of classes in Java.}
    \label{fig:improvement:multiple_commonalities_example}
\end{figure}

\mnote{Representation of packages}
\autoref{fig:improvement:multiple_commonalities_example} depicts an extension of the example given in \autoref{fig:improvement:one_commonality_example}.
In addition to classes, it contains the representation of packages and associations.
A package is represented as a dedicated \metaclass in \gls{UML}, which references the classes contained in that package.
Java, however, does not have an explicit representation of packages, but encodes them into the package names specified within classes and, additionally, represents them in a folder structure in which the source code files of the classes are persisted.
A \conceptmetamodel used to preserve consistency between packages represented in \gls{UML} and Java must represent this information in any way such that changes in Java code can be propagated into a \gls{UML} model to preserve their consistency and vice versa.
To sketch an extreme, this could even be achieved with some string attribute in the \conceptmetamodel that encodes this information in such a unique way that the necessary information for both instances of the \concretemetamodels can be generated.
Actually, a \conceptmetamodel should represent such information in a reasonable structure, whose concrete characteristics have to be defined by the transformation developer.
For packages, either the representation in Java as attributes of classes or the representation in \gls{UML} as a dedicated \metaclass can be chosen.
In the given example, we define packages in the \conceptmetamodel as explicit \metaclasses, as this makes the containment structure of classes in packages explicit.
In addition, in the complete \gls{UML} and Java metamodels packages are represented hierarchically, which is also easier to express as a relation between dedicated elements rather than their implicit encoding in the package names of classes.

\mnote{Representation of associations}
Associations in \gls{UML} are used to define that classes are related to each other.
Each association defines two classes, denoting from which class to which class the association is defined.
Java does not provide an explicit representation of associations, which usually results in their implicit representation as fields of the class from which the association is defined and having the type of the class to which it is defined.
In the example, we have chosen to represent an association explicitly in the \conceptmetamodel.
Fields can, in the complete Java and \gls{UML} metamodels, be related to further elements than associations, thus having this distinction within the \conceptmetamodel gives it more semantics.
In addition, we have chosen to have the class from which the association is defined reference the association instead of having this reference in the opposite direction as in the \gls{UML} metamodel.
No matter whether or not this is beneficial, still all necessary information to keep Java fields and \gls{UML} associations consistent is represented by the \conceptmetamodel.
It shows that for the \conceptmetamodel even a representation that differs from all its manifestations can be chosen.

\mnote{General rule for \conceptmetamodels}
As mentioned before, the only requirement to a \conceptmetamodel is that it must be able to represent all information that is necessary for defining manifestation relations to the \concretemetamodels, such that they are able to preserve consistency according to some consistency relation between the \concretemetamodels.
A general but rather informal rule, which has proven to be beneficial in the implementation of a case study for our evaluation, is to select the semantically richest among different representation options.
In the example, we have thus chosen to represent packages explicitly instead of implicitly encoding them in package names of classes.
This improves expressiveness of the \conceptmetamodel and makes its information easier to use for defining manifestation relations without interpreting implicitly encoded information in each of these relations.

\mnote{Reuse of existing metamodels}
Instead of defining a new \conceptmetamodels, it is, of course, also possible to use an existing metamodels as a \conceptmetamodel.
For example, the \gls{UML} may be considered a suitable \conceptmetamodel for object-oriented design.
Doing so does not conflict in any way with the goals of the \commonalities approach.
Such a metamodel can then either only be considered a \conceptmetamodel, whose instances are, by accident, also used by the developers, or it can be considered both a \conceptmetamodel and a \concretemetamodel with a one-to-one manifestation relation between them.
This is only a conceptual differentiation with no practical impact.
Only for the operationalization of the approach, which we discuss later, it has to be considered whether instances of a \conceptmetamodel may actually be relevant during productive use or not.


\subsection{Composition of Concepts}
\label{chap:improvement:commonalities:composition}

\mnote{Multiple \commonalities}
We have so far discussed the idea of defining an additional \conceptmetamodel to represent the common concepts of two or more \concretemetamodels.
For the depicted example for Java and \gls{UML}, it seems reasonable to group the common concepts in object-oriented design in such a metamodel.
In \autoref{fig:improvement:running_example}, we have also considered \gls{PCM} components and their consistency relations to classes in \gls{UML} and Java.
Although we could define a component \commonality for \gls{PCM} components and classes in \gls{UML} and Java, and consider this \commonality next to the class \commonality for classes in \gls{UML} and Java, we will likely not do so because of several drawbacks.
First, a component \commonality does, semantically, not fit into the discussed \conceptmetamodel for object-oriented design. Thus, the \conceptmetamodel would have to be considered broader, potentially only as one generic \conceptmetamodel.
Second, and more importantly, such a construction would introduce further redundancies, as the relation between classes in \gls{UML} and Java is expressed via two \commonalities, which are the class \commonality and the component \commonality.

\mnote{Monolithic \commonalities}
To solve the problem of a redundant specification of the relation between classes in \gls{UML} and Java via a class and a component \commonality, we could combine these two \commonalities to a single one, representing all necessary common information.
If, however, further elements share information with classes and components, they also have to be merged into the same \commonality.
In the extreme case, this could result in only having one large \commonality that is able to represent all related information.
The manifestation relations would then have to make all kinds of distinctions based on the information given in such a monolithic \commonality.

\mnote{Exemplary \commonality hierarchy}
An intuitive solution for the example scenario is to not consider classes in \gls{UML} and Java as manifestations of a component \commonality, but to consider the class \commonality as a manifestation of the component \commonality.
Then the relation between classes in \gls{UML} and Java is still represented across one specific class \commonality, whereas the manifestation relation of the component \commonality only has to be defined for the concept of classes instead of their concrete manifestations.

\begin{figure}
    \centering
    \newcommand{\vdistance}{7.5em}
\newcommand{\hdistance}{11em}
\newcommand{\classwidth}{5.5em}
\newcommand{\labeldistance}{1.2em}
\newcommand{\labelshift}{0.3*\classwidth}
\newcommand{\representstext}{\emph{«manifests»}}
\newcommand{\mmborder}{0.9em}

\begin{tikzpicture}

\pgfdeclarelayer{bg}
\pgfsetlayers{bg,main}


% METACLASSES

\umlclassvarwidth{java_class}{}{Class}{
name\\
}{\classwidth}  

\umlclassvarwidth[, right=\hdistance of java_class.north, anchor=north]{uml_class}{}{Class}{
name\\
}{\classwidth} 

\umlclassvarwidth[, above right=\vdistance and 0.5*\hdistance of java_class.north, anchor=north]{oo_class}{}{Class\vphantom{p}}{
name\\
}{\classwidth} 

\umlclassvarwidth[, right=\hdistance of oo_class.north, anchor=north]{pcm_component}{}{Component}{
name\\
}{\classwidth} 

\umlclassvarwidth[, above right=\vdistance and 0.5*\hdistance of oo_class.north, anchor=north]{component_component}{}{Component}{
name\\
}{\classwidth}

% METAMODELS

\coordinate (java_label_coordinate) at ([xshift=-\labelshift,yshift=\labeldistance]java_class.north);
\node[mmlabel, anchor=center] (java_label) at (java_label_coordinate) {Java};

\coordinate (uml_label_coordinate) at ([xshift=\labelshift,yshift=\labeldistance]uml_class.north);
\node[mmlabel, anchor=center] (java_label) at (uml_label_coordinate) {UML};

\coordinate (oo_label_coordinate) at ([xshift=-4*\labelshift,yshift=\labeldistance]oo_class.north);
\node[mmlabel, anchor=west, align=left] (oo_label) at ([xshift=-0.7em, yshift=-0.8em]oo_label_coordinate) {Object-oriented\\ Design};

\coordinate (pcm_label_coordinate) at ([xshift=\labelshift,yshift=\labeldistance]pcm_component.north);
\node[mmlabel, anchor=center] (pcm_label) at (pcm_label_coordinate) {PCM};

\coordinate (component_label_coordinate) at ([yshift=\labeldistance]component_component.north);
\node[mmlabel, anchor=center] (component_label) at (component_label_coordinate) {Componend-based Design};

\begin{pgfonlayer}{bg}
    \node[mmbg, fit=(java_class)(java_label_coordinate), inner sep=\mmborder] (java) {};
    \node[mmbg, fit=(uml_class)(uml_label_coordinate), inner sep=\mmborder] (uml) {};
    \node[conceptmmbg, fit=(oo_class)(oo_label_coordinate), inner sep=\mmborder] (oo) {};
    \node[mmbg, fit=(pcm_component)(pcm_label_coordinate), inner sep=\mmborder] (pcm) {};
    \node[conceptmmbg, minimum width=12.5em, fit=(component_component)(component_label_coordinate), inner sep=\mmborder] (component) {};
\end{pgfonlayer}


% CONSISTENCY RELATIONS

\draw[directed consistency relation] (oo_class) -- node[above, sloped] {\representstext} (java_class);
\draw[directed consistency relation] (oo_class) -- node[above, sloped] {\representstext} (uml_class);
\draw[directed consistency relation] (component_component) -- node[above, sloped] {\representstext} (oo_class);
\draw[directed consistency relation] (component_component) -- node[above, sloped] {\representstext} (pcm_component);

\end{tikzpicture}

    \caption[Hierarchic composition of \conceptmetamodels]{\Conceptmetamodels for component-based and object-oriented design and their manifestation relations between each other and to \concretemetamodels for the example introduced in \autoref{fig:improvement:running_example}. Adapted from \owncite[Fig.~3]{klare2019models}.}
    \label{fig:improvement:composed_commonalities_example}
\end{figure}

\mnote{Hierarchic composition of \commonalities}
Abstracting from this concrete example, we propose to define hierarchies of \commonalities and \conceptmetamodels, such that a manifestation of a \commonality must not necessarily be some classes of a \concretemetamodel, but can also be \commonalities of other \conceptmetamodels.
We depict such a structure for the example of classes and components in \autoref{fig:improvement:composed_commonalities_example}.
This allows to define one \conceptmetamodel for each kind of concept, such as object-oriented design or component-based design, and then compose these concepts hierarchically.
In consequence, this avoids the specification of a single \conceptmetamodel that may be become unmanageably large and again suffers from bad modularity as it needs to combine information from as many \concretemetamodels as are supposed to be kept consistent.

\mnote{Restriction to tree topology}
Since constructing such hierarchies induces a tree topology between the concrete and \conceptmetamodels, this construction suffers from the drawbacks regarding completeness, which we have already discussed in \autoref{chap:classification:topologies:effects}.
Given two concrete or \conceptmetamodels, there must be one that can be considered the manifestation of the other, or it must be possible to define a \conceptmetamodel for them, such that finally a tree of concrete and \conceptmetamodels is achieved.
First, this is actually an assumption and thus limitation of the approach, for which we provide preliminary results regarding applicability in our evaluation in \autoref{chap:commonalities_evaluation}.
Second, we further discuss these requirements regarding a tree structure in the following subsection to relax the restriction currently defined at the level of metamodels and consider a more fine-grained restriction at the level of \metaclasses.


\subsection{Tree Topology}
\label{chap:improvement:commonalities:tree}

\mnote{Correctness guarantee}
In \autoref{chap:classification:topologies:effects}, we have discussed the benefits of a tree topology induced by the metamodels and transformations of a transformation network, especially concerning inherent correctness.
We have proposed the hierarchic composition of \conceptmetamodels in the previous subsection to achieve a tree structure of manifestation relations in the \commonalities approach, which leads to a transformation network having a tree topology when realizing the manifestation relations as transformations.

\begin{figure}
    \centering
    \newcommand{\vdistance}{7em}
\newcommand{\hdistance}{13em}
\newcommand{\classwidth}{5.5em}
\newcommand{\smallclasswidth}{4em}
\newcommand{\labeldistance}{1.0em}
\newcommand{\labelshift}{0.3*\classwidth}
\newcommand{\mmborder}{1em}


\begin{tikzpicture}

\pgfdeclarelayer{bg}
\pgfsetlayers{bg,main}


% METACLASSES

\umlclassvarwidth{java_class}{}{Class\vphantom{p}}{
name\\
}{\smallclasswidth}  

\umlclassvarwidth[, right=0.65*\hdistance of java_class.north, anchor=north]{uml_class}{}{Class\vphantom{p}}{
name\\
}{\smallclasswidth}

\umlclassvarwidth[, right=1.3*\classwidth of uml_class.north, anchor=north]{uml_component}{}{Component}{
name\\
}{\classwidth} 

\umlclassvarwidth[, above right=\vdistance and 0.5*\hdistance of java_class.north, anchor=north]{oo_class}{}{Class\vphantom{p}}{
name\\
}{\classwidth} 

\umlclassvarwidth[, right=\hdistance of oo_class.north, anchor=north]{pcm_component}{}{Component}{
name\\
}{\classwidth} 

\umlclassvarwidth[, above right=\vdistance and 0.5*\hdistance of oo_class.north, anchor=north]{component_component}{}{Component}{
name\\
}{\classwidth}

% METAMODELS

\coordinate (java_label_coordinate) at ([yshift=\labeldistance]java_class.north west);
\node[mmlabel, anchor=west] (java_label) at (java_label_coordinate) {Java};

\coordinate (uml_label_coordinate) at ([yshift=\labeldistance]uml_component.north east);
\node[mmlabel, anchor=east] (uml_label) at (uml_label_coordinate) {\acrshort{UML}};

\coordinate (oo_label_coordinate) at ([xshift=-4*\labelshift,yshift=\labeldistance]oo_class.north);
\node[mmlabel, anchor=west, align=left] (oo_label) at ([xshift=-0.5em, yshift=-0.7em]oo_label_coordinate) {Object-oriented\\ Design};

\coordinate (pcm_label_coordinate) at ([yshift=\labeldistance]pcm_component.north east);
\node[mmlabel, anchor=east] (pcm_label) at (pcm_label_coordinate) {\acrshort{PCM}};

\coordinate (component_label_coordinate) at ([yshift=\labeldistance]component_component.north);
\node[mmlabel, anchor=center] (component_label) at (component_label_coordinate) {Component-based Design};

\begin{pgfonlayer}{bg}
    \node[mmbg, fit=(java_class)(java_label.west), inner sep=\mmborder] (java) {};
    \node[mmbg, fit=(uml_class)(uml_component)(uml_label.east), inner sep=\mmborder] (uml) {};
    \node[conceptmmbg, fit=(oo_class)(oo_label_coordinate), inner sep=\mmborder] (oo) {};
    \node[mmbg, fit=(pcm_component)(pcm_label.east), inner sep=\mmborder] (pcm) {};
    \node[conceptmmbg, fit=(component_component)(component_label.west)(component_label.east), inner sep=\mmborder] (component) {};
\end{pgfonlayer}

\draw[-, color=gray] ($(uml_class.south east)!0.5!(uml_component.south west)-(0,\mmborder)$) -- ($(uml_class.north east)!0.5!(uml_component.north west)+(0,\mmborder+\labeldistance)$);


% CONSISTENCY RELATIONS

\draw[manifests relation] (oo_class) -- node[stereotype, above, sloped] {\manifestslabel} (java_class);
\draw[manifests relation] (oo_class) -- node[stereotype, above, sloped] {\manifestslabel} (uml_class);
\draw[manifests relation] (component_component) -- node[stereotype, above, sloped] {\manifestslabel} (oo_class);
\draw[manifests relation] (component_component) -- node[stereotype, above, sloped] {\manifestslabel} (pcm_component);
\draw[manifests relation] (component_component) -- node[stereotype, below, sloped] {\manifestslabel} ([xshift=-1.5em]uml_component.north);

\end{tikzpicture}

    \caption[Example for tree topology of \commonalities]{\Conceptmetamodels for component-based and object-oriented design and their manifestation relations between each other and to \concretemetamodels for the example introduced in \autoref{fig:improvement:running_example} and extended by components in \gls{UML}. Adapted from \owncite[Fig.~3]{klare2019models}.}
    \label{fig:improvement:extended_composed_commonalities_example}
\end{figure}

\mnote{Impracticality of trees}
That approach does, however, assume that such a tree topology of \conceptmetamodels can always be achieved.
Since we have up to now discussed the topology at the level of complete metamodels and transformations between them, it is easy to see that a tree cannot be achieved in many situations.
This is always the case if one \concretemetamodel contains concepts that are to be represented in multiple \conceptmetamodels.
For example, the \gls{UML} contains concepts both from object-oriented design and component-based design, which easily conflicts with the goal of achieving a tree topology.
\autoref{fig:improvement:extended_composed_commonalities_example} depicts this example for classes and components in \gls{UML}.
\gls{UML} classes have a common concept with the \concretemetamodels Java in object-oriented design and \gls{UML} components have a common concept with the \concretemetamodel \gls{PCM} in component-based design, which both, in turn, share a manifestation relation. This breaks the tree topology at the level of metamodels and transformations between them.

\mnote{Metamodel bounds}
Although the bounds of metamodels are usually motivated by their necessity to fit for a specific purpose (see \autoref{chap:foundations:modeling:models}) and thus to represent specific concepts, metamodel bounds are, in general, arbitrary.
Especially if metamodels have a rather general purpose, such as \gls{UML} or programming languages like Java, they may contain elements representing multiple different concepts or the same elements may even be considered manifestations of multiple concepts.
The former case leads to the situation that the elements of a metamodel may be separated by the different concepts they represent, thus virtually forming multiple metamodels.
Usually, however, even elements representing concepts from different domains are still related, for example, by having the same super types like \texttt{NamedElement}, which makes their separation into different metamodels impossible.

\mnote{Non-interference as relaxed notion}
The benefit of inherent correctness guarantees of transformation networks with tree topology arises from the fact that there are no two paths of transformations between the same metamodels, as discussed in \autoref{chap:classification:properties}.
This is, however, already given if two paths of transformations affect disjoint sets of elements and thus do not interfere.
Such a notion of \emph{non-interference} has already been defined by \textcite{stevens2020BidirectionalTransformationLarge-SoSym}, which specifies that two transformations changing the same model do not interfere if changing their execution order does not change the result.
Since each transformation ensures consistency to its consistency relations and since the result is independent from the execution order of non-interfering transformations, it is guaranteed that the resulting models are consistent to both non-interfering transformations.

\mnote{Consistency relation trees}
This informally stated notion of having all pairs of paths of transformations affect disjoint sets of elements, given, for example, by non-interference, conforms to our notion of \emph{consistency relation trees} as specified in \autoref{def:relationtree} for proving compatibility of consistency relations.
It defines that for each pair of concatenations of consistency relations either the left class tuples or the right class tuples must be disjoint, such that sequences of transformations preserving consistency to these relations affect disjoint sets of objects.
In consequence, it is sufficient to ensure that the graph of consistency relations defined by the manifestation relations is a consistency relation tree to ensure compatibility of the network.
Due to the lack of multiple transformation paths affecting the same elements, it is also not necessary to ensure that transformations are synchronizing.
Thus, even for this relaxed notion in comparison to trees at the level of metamodels and transformations, as depicted in \autoref{chap:classification:topologies:effects}, correctness guarantees for the transformation network are given.

\mnote{Practicality of relaxed notion}
Still, this relaxed notion represents a requirement for the \commonalities approach to provide specific benefits.
We show at a case study in our evaluation in \autoref{chap:commonalities_evaluation} that it is actually possible to achieve such a structure in practical scenarios, which serves as an indicator for its general achievability and thus the possibility to have inherent correctness guarantees when applying the \commonalities approach for preserving consistency of multiple models.
Finally, the notion could even be further relaxed, as it must finally only be ensured that only one transformation path between two elements exists at runtime.
Even if there are two possible relations defined in the transformations, it can be the case that further constraints ensure that at runtime only one path is relevant, because the constraints are mutually exclusive.
%For example, a Java class may be related to a standalone component (denoted as \texttt{BasicComponent} in \gls{PCM}) or a composed component (denoted as \texttt{CompositeComponent} in \gls{PCM}), which internally is composed of other composed or standalone components.
%BAD EXAMPLE: Different elements at one end

% NOTE: Hierarchy of commonalities still relevant, because only between concept metamodels manifestation relations are allowed


\subsection{Operationalization}
\label{chap:improvement:commonalities:operationalization}

\begin{figure}
    \centering
    \newcommand{\distance}{3em}
\newcommand{\scenariodistance}{7*\distance+0.5*\difftoafiveimage}

\begin{tikzpicture}[
    manifests relation label/.style={stereotype, above, sloped}
]

\node[schematic conceptmetamodel] (concept) {$\metamodel{C}{}$};
\node[schematic metamodel, below left=2*\distance and 0.5+\distance of concept.center, anchor=center] (concrete1) {$\metamodel{A}{}$};
\node[schematic metamodel, below right=2*\distance and 0.5+\distance of concept.center, anchor=center] (concrete2) {$\metamodel{B}{}$};

\draw[manifests relation] (concept) -- node[manifests relation label] {\manifestslabel} (concrete1);
\draw[manifests relation] (concept) -- node[manifests relation label] {\manifestslabel} (concrete2);

\node[schematic metamodel, above right=-\distance+0.5*\distance+\distance and \scenariodistance of concept.center, anchor=center] (additional_concretetop) {$\metamodel{C}{}$};
\node[schematic metamodel, below left=1*\distance and 0.5+\distance of additional_concretetop.center, anchor=center] (additional_concretebottom1) {$\metamodel{A}{}$};
\node[schematic metamodel, below right=1*\distance and 0.5+\distance of additional_concretetop.center, anchor=center] (additional_concretebottom2) {$\metamodel{B}{}$};

\draw[transformation] (additional_concretetop) -- (additional_concretebottom1);
\draw[transformation] (additional_concretetop) -- (additional_concretebottom2);


\node[schematic metamodel, below right=\distance+\distance and \scenariodistance-\distance of concept.center, anchor=center] (generate_concrete1) {$\metamodel{A}{}$};
\node[schematic metamodel, right=2*\distance of generate_concrete1.center, anchor=center] (generate_concrete2) {$\metamodel{B}{}$};

\draw[transformation] (generate_concrete1) -- (generate_concrete2);


\draw[thick, ->] ([yshift=-0*\distance,xshift=1.5*\distance]concept.center) -- node[above, sloped, font=\footnotesize\itshape] {Option 1:} node[below, sloped, align=center, font=\footnotesize\bfseries] {Concept metamodels\\ as additional metamodels} ([yshift=0.5*\distance, xshift=-0.5*\distance]additional_concretebottom1.north west);
\draw[thick, ->] ([yshift=-1.5*\distance,xshift=1.5*\distance]concept.center) -- node[above, sloped, font=\footnotesize\itshape] {Option 2:} node[below, sloped, align=center, font=\footnotesize\bfseries] {Transformations between \\ concrete metamodels} ([xshift=-0.5*\distance]generate_concrete1.north west);

\draw[gray] ($(additional_concretebottom1.west)!0.5!(generate_concrete1.west)$) -- ($(additional_concretebottom2.east)!0.5!(generate_concrete2.east)$);

\end{tikzpicture}
    %\includegraphics[width=\textwidth]{figures/quality/improvement/operationalization_alternatives.png}
    \caption[Alternatives for \commonalities operationalization]{Exemplification of alternatives to operationalize \commonalities specifications by using \conceptmetamodels (such as $\metamodel{C}{}$) as ordinary metamodels or by deriving direct transformations between the \concretemetamodels (such as $\metamodel{A}{}$ and $\metamodel{B}{}$) from them.}
    \label{fig:improvement:operationalization_alternatives}
\end{figure}

\mnote{Deriving executable transformations}
Up to now, we have discussed how to express consistency by means of \conceptmetamodels with \commonalities and manifestation relations in the \commonalities approach.
To actually preserve consistency of instances of the \concretemetamodels, such a specification must also be operationalized, such that executable transformations that can be applied after changes to these models are present or derived.
%\mnote{Operationalization options}
We can distinguish two basic options for this operationalization, which are also depicted in \autoref{fig:improvement:operationalization_alternatives}:
\begin{properdescription}
    \item[\Conceptmetamodels as additional metamodels:] \Conceptmetamodels are considered as ordinary metamodels and manifestation relations as ordinary transformations, thus we consider a transformation network of \concretemetamodels and \conceptmetamodels, whose instances are kept consistent by transformations for their manifestation relations. %In consequence, instances of the \conceptmetamodels have to be maintained.
    \item[Transformations between \concretemetamodels:] \Conceptmetamodels and manifestation relations are only used as auxiliary specification artifacts, from which direct transformations between the \concretemetamodels are derived. For example, from the object-oriented design \conceptmetamodel in \autoref{fig:improvement:one_commonality_example}, a transformation between Java and \gls{UML} is derived.
\end{properdescription}

\mnote{\Conceptmetamodels as ordinary metamodels}
The benefit of treating \conceptmetamodels as ordinary, additional metamodels and the manifestation relations as transformations is easy achievability.
No specific languages or generators are required to derive the necessary artifacts, but existing tools for defining metamodels and transformations can be used to define \conceptmetamodels and manifestation relations that can be readily used to preserve consistency of their instances.
A drawback of this approach is that it requires the management and persistence of additional artifacts, namely the instances of the \conceptmetamodels, which are only auxiliary artifacts that should be transparent to the user.
This can, however, be hidden by an according framework that abstracts from these additional artifacts, such that developers are still only confronted with the models of the tools they use.
Such functionality is provided by tools like \vitruv~\cite{klare2020Vitruv-JSS} (see \autoref{chap:foundations:multiview:vitruv}) providing only views on instances of \concretemetamodels.

\mnote{Deriving direct transformations}
Deriving transformations between \concretemetamodels from a specification of \conceptmetamodels and manifestation relations benefits from not introducing further artifacts, such that a developer still only has to deal with instances of the \concretemetamodels he or she is concerned with.
This approach, however, suffers from reduced expressiveness, because not all multiary relations as expressed across additional \conceptmetamodels (see~\cite{diskin2018MultiModelSynchronization-FASE}) can be expressed by sets of binary relations and transformations preserving them~\cite{stevens2020BidirectionalTransformationLarge-SoSym}.
In addition, it requires the implementation of generators that derive transformations from specifications of \conceptmetamodels and manifestation relations.

\mnote{Correctness of derived transformation network}
Although with the second approach of deriving ordinary transformations the resulting transformation network contains cycles and does thus not provide correctness guarantees due to its topology, it still provides the guarantee due to the transformations being generated from a specification that ensures correctness.
For example, since a specification of \commonalities cannot contain incompatibilities, the derived transformations cannot contain them either, as long as the generator produces transformations that actually preserve consistency conforming to the defined manifestation relations.

\mnote{Multiple transformation executions}
For the orchestration of the generated transformations, no matter whether they are defined between \conceptmetamodels or derived between the \concretemetamodels, it is still necessary to allow the execution of each transformation multiple times.
Due to the situations identified in \autoref{chap:orchestration}, in which it is necessary to execute transformations multiple times to \enquote{negotiate} a result and repeatedly react to the changes of other transformations, such a behavior is still relevant for the \commonalities approach.
For example, propagating a class from Java across the object-oriented design \conceptmetamodels and the component-based design \conceptmetamodel to a component in \gls{PCM} can lead to further additions to the class as soon as it is identified as a representation of a component, which then needs to be propagated back to the class representation in Java.
This support this, transformations should still be synchronizing and thus allowed to modify both involved models to support such situations, which require this backpropagation of changes.


% \begin{copiedFrom}{DocSym} % ABOUT TREES

% The most crucial part of this approach is the necessity to build a tree of \glspl{CMM}.
% This will not be possible if always considering whole metamodels and their relations, but can be possible if independent concepts are extracted to be treated individually.
% %For that, we will apply our findings on consistency relation decomposition (see \autoref{chap:improvement:approach}).
% %This is similar to decomposing consistency relations, as introduced in \autoref{sec:approach:decomposition}, which is why we will apply our findings therefrom.
% Additionally, the approach specifically aims to improve the specification of transformations for descriptive relations.
% In consequence, it must be combined with transformations expressing the normative relations between \glspl{CMM} and other metamodels.

% \end{copiedFrom} % DocSym

% \begin{copiedFrom}{VoSE}

% The state-of-the-art approach to keep models consistent automatically is the application of transformation languages.
% If instances of multiple (i.e., more than two) metamodels are to be kept consistent, one can either use multidirectional transformation approaches, or compose bidirectional transformations to a network of transformations~\cite{cleve2019dagstuhl}.
% Following sentence moved to introduction
%Such a network can be regarded as a graph, formed by metamodels as its nodes and transformations as its edges.
% When an instance of one metamodel is changed in such a network, the transformations are executed successively to propagate the change transitively across all models.
% There are strategies to find one ordering of transformations to apply~\cite{stevens2020BidirectionalTransformationLarge-SoSym} and strategies to perform a fixpoint iteration until no further changes are conducted~\cite{klare2019icmt}.

% %Short introduction of transformation network, what nodes and edges are and so on.
% In this section, we propose a different approach for keeping two or more models consistent by specifying their common concepts rather than their direct consistency relations.
% This forms our contribution~\autoref{contrib:quality:improvement}.
% % \subsection{Defining Consistency Relations}
% % Different metamodels provide

% Consistency relations are usually defined declaratively (what constraints have to hold) or imperatively (how are constraints enforced) between two (or sometimes more) metamodels.
% This is a definition of when instances of that metamodels are considered consistent.
% Consistency constraints can be either \emph{declarative} or \emph{normative}. Declarative constraints exist (e.g., because they are somehow \enquote{natural} and have to be formalized to be checked or preserved by a tools. Normative constraints do not have to adhere to an existing notion of consistency and thus implicitly define what is considered consistent.
% In our running example, the constraints between UML and Java exist and only have to be specified in a declarative manner.
% On the other hand, the constraints between architecture and OO design (i.e., PCM and UML/Java) were normatively defined by \textcite{langhammerconstraints}. There is no single correct mapping between architecture and OO design but several possible.

% \subsection*{Composing \commonalities}

% We have explained how multiple metamodels can be kept consistent using one \conceptmetamodel.
% This allows, theoretically, the definition of one large \conceptmetamodel that contains all \commonalities for all \concretemetamodels.
% It would at first sight be similar to a \gls{SUMM}, as introduced by \textcite{atkinson2010a}.
% However, it would be less complex than a \gls{SUMM}, which is able to express all information about the software system and thus contains the union of all \concretemetamodels.
% %In fact, the resulting \conceptmetamodel would be comparable to a \summ according to \textcite{atkinson2010a}, which is able to express all information about the software system, whereas the instances of the \concretemetamodels would only serve as projectional views that do not provide further information.
% Nevertheless, one large \conceptmetamodel would still become unmanageably large due to the fact that it had to contain the union of all pairwise intersections of the \concretemetamodels, as mentioned before.

% \begin{figure}
%     \centering
%     \newcommand{\vdistance}{7.5em}
\newcommand{\hdistance}{11em}
\newcommand{\classwidth}{5.5em}
\newcommand{\labeldistance}{1.2em}
\newcommand{\labelshift}{0.3*\classwidth}
\newcommand{\representstext}{\emph{«manifests»}}
\newcommand{\mmborder}{0.9em}

\begin{tikzpicture}

\pgfdeclarelayer{bg}
\pgfsetlayers{bg,main}


% METACLASSES

\umlclassvarwidth{java_class}{}{Class}{
name\\
}{\classwidth}  

\umlclassvarwidth[, right=\hdistance of java_class.north, anchor=north]{uml_class}{}{Class}{
name\\
}{\classwidth} 

\umlclassvarwidth[, above right=\vdistance and 0.5*\hdistance of java_class.north, anchor=north]{oo_class}{}{Class\vphantom{p}}{
name\\
}{\classwidth} 

\umlclassvarwidth[, right=\hdistance of oo_class.north, anchor=north]{pcm_component}{}{Component}{
name\\
}{\classwidth} 

\umlclassvarwidth[, above right=\vdistance and 0.5*\hdistance of oo_class.north, anchor=north]{component_component}{}{Component}{
name\\
}{\classwidth}

% METAMODELS

\coordinate (java_label_coordinate) at ([xshift=-\labelshift,yshift=\labeldistance]java_class.north);
\node[mmlabel, anchor=center] (java_label) at (java_label_coordinate) {Java};

\coordinate (uml_label_coordinate) at ([xshift=\labelshift,yshift=\labeldistance]uml_class.north);
\node[mmlabel, anchor=center] (java_label) at (uml_label_coordinate) {UML};

\coordinate (oo_label_coordinate) at ([xshift=-4*\labelshift,yshift=\labeldistance]oo_class.north);
\node[mmlabel, anchor=west, align=left] (oo_label) at ([xshift=-0.7em, yshift=-0.8em]oo_label_coordinate) {Object-oriented\\ Design};

\coordinate (pcm_label_coordinate) at ([xshift=\labelshift,yshift=\labeldistance]pcm_component.north);
\node[mmlabel, anchor=center] (pcm_label) at (pcm_label_coordinate) {PCM};

\coordinate (component_label_coordinate) at ([yshift=\labeldistance]component_component.north);
\node[mmlabel, anchor=center] (component_label) at (component_label_coordinate) {Componend-based Design};

\begin{pgfonlayer}{bg}
    \node[mmbg, fit=(java_class)(java_label_coordinate), inner sep=\mmborder] (java) {};
    \node[mmbg, fit=(uml_class)(uml_label_coordinate), inner sep=\mmborder] (uml) {};
    \node[conceptmmbg, fit=(oo_class)(oo_label_coordinate), inner sep=\mmborder] (oo) {};
    \node[mmbg, fit=(pcm_component)(pcm_label_coordinate), inner sep=\mmborder] (pcm) {};
    \node[conceptmmbg, minimum width=12.5em, fit=(component_component)(component_label_coordinate), inner sep=\mmborder] (component) {};
\end{pgfonlayer}


% CONSISTENCY RELATIONS

\draw[directed consistency relation] (oo_class) -- node[above, sloped] {\representstext} (java_class);
\draw[directed consistency relation] (oo_class) -- node[above, sloped] {\representstext} (uml_class);
\draw[directed consistency relation] (component_component) -- node[above, sloped] {\representstext} (oo_class);
\draw[directed consistency relation] (component_component) -- node[above, sloped] {\representstext} (pcm_component);

\end{tikzpicture}

%     \caption[Concept metamodels for the running example]{\Conceptmetamodels (dark) and their relations to \concretemetamodels (light) for the running example.}
%     \label{fig:quality:composed_commonalities_example}
% \end{figure}

% To avoid the specification of such a monolithic \conceptmetamodel, we propose to compose \commonalities from different \conceptmetamodels.
% Instead of having only \commonalities that relate to \metaclasses in \concretemetamodels, \commonalities may also have relations to other \commonalities.
% Consider the \conceptmetamodel for component-based design in \autoref{fig:quality:composed_commonalities_example}.
% It contains the \commonality \texttt{Component}, which is represented by an equally named \metaclass in \gls{PCM},
% as well as in the \commonality \texttt{Class} in the \conceptmetamodel for object-oriented design, conforming to the relations proposed by \textcite{langhammer2015a}.
% This induces a tree structure with \commonalities as inner nodes and \metaclasses of \concretemetamodels as leaves.
% With such a composition structure, a \emph{«manifests»} relation may not only exist between a %concrete and a concept metamodel but also between two \conceptmetamodels.
% \commonality of a \conceptmetamodel and a \metaclass in a \concretemetamodel but also between two \commonalities.
% However, a concrete or \conceptmetamodel that is lower in the hierarchy is supposed to represent how a \metaclass or \commonality in the higher one manifests, which is why we call it a \emph{manifestation}.
% %Due to that, we call concrete and \conceptmetamodels that contain \metaclasses or \commonalities with a \emph{«manifests»} relation to another \conceptmetamodel a \emph{manifestation} of that \conceptmetamodel.
% For example, the object-oriented design \conceptmetamodel is a manifestation of the component-based design \conceptmetamodel.

% \begin{figure}
%     \centering
%     \newcommand{\vdistance}{8em}
\newcommand{\hdistance}{11.5em}
\newcommand{\classwidth}{5.5em}
\newcommand{\smallclasswidth}{4em}
\newcommand{\labeldistance}{1.2em}
\newcommand{\labelshift}{0.3*\classwidth}
\newcommand{\representstext}{\emph{«manifests»}}
\newcommand{\mmborder}{1em}

\begin{tikzpicture}

\pgfdeclarelayer{bg}
\pgfsetlayers{bg,main}


% METACLASSES

\umlclassvarwidth{java_class}{}{Class\vphantom{p}}{
name\\
}{\smallclasswidth}  

\umlclassvarwidth[, right=0.65*\hdistance of java_class.north, anchor=north]{uml_class}{}{Class\vphantom{p}}{
name\\
}{\smallclasswidth}

\umlclassvarwidth[, right=1.3*\classwidth of uml_class.north, anchor=north]{uml_component}{}{Component}{
name\\
}{\classwidth} 

\umlclassvarwidth[, above right=\vdistance and 0.5*\hdistance of java_class.north, anchor=north]{oo_class}{}{Class\vphantom{p}}{
name\\
}{\classwidth} 

\umlclassvarwidth[, right=\hdistance of oo_class.north, anchor=north]{pcm_component}{}{Component}{
name\\
}{\classwidth} 

\umlclassvarwidth[, above right=\vdistance and 0.5*\hdistance of oo_class.north, anchor=north]{component_component}{}{Component}{
name\\
}{\classwidth}

% METAMODELS

\coordinate (java_label_coordinate) at ([xshift=-\labelshift,yshift=\labeldistance]java_class.north);
\node[mmlabel, anchor=center] (java_label) at (java_label_coordinate) {Java};

\coordinate (uml_label_coordinate) at ([xshift=\labelshift,yshift=\labeldistance]uml_component.north);
\node[mmlabel, anchor=center] (java_label) at (uml_label_coordinate) {UML};

\coordinate (oo_label_coordinate) at ([xshift=-4*\labelshift,yshift=\labeldistance]oo_class.north);
\node[mmlabel, anchor=west, align=left] (oo_label) at ([xshift=-0.7em, yshift=-0.8em]oo_label_coordinate) {Object-oriented\\ Design};

\coordinate (pcm_label_coordinate) at ([xshift=\labelshift,yshift=\labeldistance]pcm_component.north);
\node[mmlabel, anchor=center] (pcm_label) at (pcm_label_coordinate) {PCM};

\coordinate (component_label_coordinate) at ([yshift=\labeldistance]component_component.north);
\node[mmlabel, anchor=center] (component_label) at (component_label_coordinate) {Componend-based Design};

\begin{pgfonlayer}{bg}
    \node[mmbg, fit=(java_class)(java_label_coordinate), inner sep=\mmborder] (java) {};
    \node[mmbg, fit=(uml_class)(uml_component)(uml_label_coordinate), inner sep=\mmborder] (uml) {};
    \node[conceptmmbg, fit=(oo_class)(oo_label_coordinate), inner sep=\mmborder] (oo) {};
    \node[mmbg, fit=(pcm_component)(pcm_label_coordinate), inner sep=\mmborder] (pcm) {};
    \node[conceptmmbg, minimum width=12.5em, fit=(component_component)(component_label_coordinate), inner sep=\mmborder] (component) {};
\end{pgfonlayer}

\draw[-, color=gray] ($(uml_class.south east)!0.5!(uml_component.south west)-(0,\mmborder)$) -- ($(uml_class.north east)!0.5!(uml_component.north west)+(0,\mmborder+\labeldistance)$);


% CONSISTENCY RELATIONS

\draw[consistencyrel, <-] (java_class) -- node[above, sloped] {\representstext} (oo_class);
\draw[consistencyrel, <-] (uml_class) -- node[above, sloped] {\representstext} (oo_class);
\draw[consistencyrel, <-] (oo_class) -- node[above, sloped] {\representstext} (component_component);
\draw[consistencyrel, <-] (pcm_component) -- node[above, sloped] {\representstext} (component_component);
\draw[consistencyrel, <-] ([xshift=-1.5em]uml_component.north) -- node[below, sloped] {\representstext} (component_component);

\end{tikzpicture}

%     \caption[Concept metamodels in the running example]{\Conceptmetamodels (dark) and \concretemetamodels (light) of the running example, extended by UML components, with their relations.}
%     \label{fig:quality:extended_composed_commonalities_example}
% \end{figure}

% Our goal is to achieve a tree structure of commonalities. In the extended example in \autoref{fig:quality:extended_composed_commonalities_example}, a \texttt{Component} in the \conceptmetamodel for component-based design does not only manifest in a \gls{PCM} \texttt{Component} as well as a \texttt{Class} in object-oriented design, but also in \gls{UML}.
% Since a \texttt{Class} in object-oriented design manifests both in Java and \gls{UML}, we do not have a tree structure of the induced relations between the metamodels anymore, due to \texttt{Class} and \texttt{Component} both being represented in \gls{UML}.
% However, This still induces a tree structure between \metaclasses and \commonalities, with the \commonalities being inner nodes and \metaclasses of \concretemetamodels being leaves.

%
% How are Commonalities composed to keep multiple models consistent? What about overlaps?
% Apply to OO and PCM case
%
% \begin{figure}
%     \centering
% %    \includegraphics[width=\columnwidth]{figures/dag_example.pdf}\\
%     \newcommand{\vdistance}{4.8em}
\newcommand{\hdistance}{10.9em}
\newcommand{\mmwidth}{8.5em}
\newcommand{\mmheight}{3em}

\begin{tikzpicture}[
    mm/.style={draw, mmbg, minimum width=\mmwidth, minimum height=\mmheight, inner sep=0em}, %, font=\itshape},
    conceptmm/.style={mm, conceptmmbg},
]

% Metamodels

\node[mm] (java) {Java};
\node[mm, draw=gray!70, minimum height=0.5*\mmheight, right=\hdistance of java.south, anchor=south] (uml) {UML};
\node[mm, draw=gray!70, minimum width=0.5*\mmwidth, minimum height=0.5*\mmheight, above=0.5*\mmheight of uml.west, anchor=west] (uml_class) {\emph{Class}};
\node[mm, draw=gray!70, minimum width=0.5*\mmwidth, minimum height=0.5*\mmheight, above=0.5*\mmheight of uml.east, anchor=east] (uml_component) {\emph{Comp}};
\node[mm, fill=none, right=\hdistance of java.north, anchor=north] {}; % Redraw thick border of UML
\node[conceptmm, above right=\vdistance and 0.5*\hdistance of java.north, anchor=north, align=center] (oo) {Object-oriented\\ Design};
\node[mm, right=\hdistance of oo.north, anchor=north] (pcm) {PCM};
\node[conceptmm, above right=\vdistance and 0.5*\hdistance of oo.north, anchor=north, align=center] (component) {Component-based\\ Design};


% CONSISTENCY RELATIONS

\draw[consistencyrel, <-] (java) -- (oo);
\draw[consistencyrel, <-] (uml_class) -- (oo);
\draw[consistencyrel, <-] (oo) -- (component);
\draw[consistencyrel, <-] ([xshift=-0.15*\mmwidth]uml_component.north) -- (component);
\draw[consistencyrel, <-] (pcm) -- (component);


\end{tikzpicture}
 %\\[1em]
%     %\input{figures/dag_example_alternative.tex}
%     \caption{\Concretemetamodels (light) and \conceptmetamodels (dark) of the running example forming a \acs{DAG}}
%     \label{fig:quality:dag_example}
% \end{figure}

%Since we only assume a tree of \commonalities,  rather than a tree of \conceptmetamodels, 
% A metamodel may have several \commonalities in different \conceptmetamodels with different other metamodels.
% For example, in \autoref{fig:improvement:composed_commonalities_example}, the UML metamodel contains a \texttt{Class} and a \texttt{Component} \metaclass, which have two different \commonalities in two different \conceptmetamodels.

%We currently assume that all concept metamodels and relations to their manifestations can be represented as a tree.
% In general, %, this will not be possible, 
% the assumption of having a tree of \conceptmetamodels may not be satisfiable if
% metamodels have several \commonalities with different metamodels.
% In fact, we can relax that assumption: 
% The essential requirement is that each change is only propagated across one path between two models, which is inherently given in a tree.
% If the \commonalities of two \conceptmetamodels manifest in disjoint sets of elements of the same manifestation, that manifestation can be---virtually---separated into two metamodels, for which the network forms a tree again.
% In consequence, if we consider the transformations as directed edges from a \conceptmetamodel to its manifestations, we only have to require a \ac{DAG}, % induced by these edges, 
% as long as the manifestations of all \commonalities are disjoint.
% %If we consider the transformations as directed edges from a \conceptmetamodel to its manifestations, we only have to require a \ac{DAG} induced by these edges, as long as the \conceptmetamodels represent disjoint sets of elements of their manifestations.
% %This relaxation is possible due to the fact that the essential requirement is to have only one path between two models across which a change can be propagated.
% %If different \conceptmetamodels relate to disjoint parts of their manifestations, these manifestations can be virtually separated into different metamodels, which then form a tree again.
% \autoref{fig:dag_example} exemplifies this relaxation: % on our running example:
% If the UML is also considered a manifestation of component-based design by providing the \texttt{Component} \metaclass, the network does not constitute a tree because of the relations between the UML and the \conceptmetamodels for component-based design and object-oriented design.
% However, it forms a \ac{DAG} and the redundant paths to the UML metamodel are unproblematic, because the relation to object-oriented design affects the part of the UML metamodel considering classes (\emph{Class}), whereas the relation to component-based design affects the part of the UML metamodel considering components (\emph{Comp}). 
% The elements in these parts of the UML metamodel are disjoint.

% \subsection*{Transformation Operationalization}
% \label{chap:commonalities:approach:operationalization}

% To actually keep models consistent, the specification of a hierarchy of \conceptmetamodels has to be operationalized.
% Two options for operationalization can be distinguished:

% \begin{description}[leftmargin=\parindent]
%     \item[\Conceptmetamodels as additional metamodels:] The \conceptmetamodels are actually instantiated and the transformations are executed as they are defined between the \conceptmetamodels and their manifestations. In consequence, instances of the \conceptmetamodels have to be maintained.
%     \item[Transformations between \concretemetamodels:] The \conceptmetamodels and the relations between them and their manifestations are used to derive bidirectional transformations between the \concretemetamodels. For example, from the \conceptmetamodel for object-oriented design in \autoref{fig:improvement:one_commonality_example}, a bidirectional transformation between Java and UML is derived.
% \end{description}

% A drawback of the first option is that additional models have to be managed and persisted. 
% In consequence, the user has to version these models although they should be transparent to him or her, as long as no appropriate framework abstracts from such tasks.
% %Additionally, it is not easily possible to derive the direct relations between two concrete metamodels from such a specification. For example, the direct relation between classes in UML and Java is not accessible but only implicitly expressed by the transitive relation across the concept metamodel for object-oriented design.
% A drawback of the second option is that the types of supported relations that can be described in the transformations are limited.
% First, only relations may be defined that can be composed with any other relation, such that a direct transformation between two metamodels can be derived.
% Second, it is possible to define $n$-ary relations between more than two metamodels that cannot be decomposed into binary relations between them, but only into $n$ binary relations between those metamodels and an additional one~\cite{stevens2020BidirectionalTransformationLarge-SoSym}.
% In consequence, the first option provides higher expressiveness.

% While the first option can be realized without an additional language by just defining the \conceptmetamodels and the transformations with existing languages, the second option requires a mechanism that generates the transformations between the \concretemetamodels from those between the \conceptmetamodels and their manifestations.

% \end{copiedFrom} % VoSE
\section{Expected Benefits}

\subsection{Improving Correctness and Reusability}
\todo{Instead of only discussion compatibility (which for sure is also true), discuss that in Commonalities there is always a consistent orchestration and that the shortest orchestration has an upper bound depending on the number of transformations. Discuss why this is not a drawback regarding such a bound in ordinary networks (especially discuss the previous example, where ping-pong was required, whereas here the required information is automatically added within the Commonalities and then only needs to be propagated back). However, there can still be scenarios that cannot be handled (due to the same reasons as for ordinary networks), but we already questioned there, although theoretically possible, whether they are practically relevant. In contrast to ordinary networks, Commonalities encode the limitation into the approach strategy and thus make it explicit for the developer rather then defining individual transformation without knowing how often they are executed later on.}


\subsection{Reducing Specification Effort}

\subsection{Improving Comprehensibility}




\begin{copiedFrom}{VoSE}

% BENEFITS
We expect several benefits from our approach, i.e. specifying \commonalities, in comparison to direct transformation specifications between metamodels.
First, we claim to achieve \emph{better understandability} of relations between metamodels, because common concepts are made explicit. % rather than encoding them implicitly in transformations.
Second, the approach \emph{reduces errors} when more than two metamodels are to be kept consistent.
Transformations usually relate two metamodels, especially because multidirectional relations are hard to express~\cite{stevens2020BidirectionalTransformationLarge-SoSym},
%\modified{either because the developer does not know about multidirectional relations or due to cognitive limitations~\cite{stevens2020BidirectionalTransformationLarge-SoSym},} %keep instances of two metamodels consistent 
and therefore have to be combined to a network of transformations to keep instances of more than two metamodels consistent.
% Following sentence moved from approach section
Such a network can be regarded as a graph, formed by metamodels as its nodes and transformations as its edges.
However, such a network can easily raise compatibility problems if there exists more than one path of transformations between two metamodels.
A hierarchy of \commonality specifications is, by design, not prone to such problems.
Finally, we \emph{improve reusability} in comparison to a network of transformations, %regarding a network of transformations, 
because an arbitrary subset of metamodels, between which \commonalities are defined, can be selected to keep their instances consistent.
In contrast, removing metamodels from a transformation network can easily lead to missing transformation paths between two metamodels.

% Benefits:
% \begin{itemize}
%     \item \emph{Better understandability} of relations, as common concepts are explicitly defined rather than implicitly encoding them in transformations
%     \item \emph{Improved reusability / partial usability} (regarding networks of binary transformations), because an arbitrary selection of concrete metamodels can be used and kept consistent
%     \item \emph{High expressiveness}, because no restriction due to predefined sets of statements, as expression can be added dynamically. This can be also improve analyzability of transformations, as additional metadata could be defined for each of the extensions.
% \end{itemize}

\subsection*{Benefits of \commonalities}
\label{chap:commonalities:approach:benefits}

We suppose the \commonalities approach to provide two kinds of benefits:
First, we expect that it improves understandability of relations between metamodels, because common concepts are not encoded in transformations implicitly but modelled explicitly.
This is even a benefit if instances of only two metamodels shall be kept consistent.
Second, it reduces problems that can occur if several bidirectional transformations are combined into a network of transformations to keep multiple models consistent.

% Two types of benefits:
% \begin{itemize}
%     \item Independent from network size: Understandability (explicit commonalities rather than implicit encoding in constraints)
%     \item Benefits for transformation networks (in the following)
% \end{itemize}

\begin{figure}
    \centering
    \begin{minipage}[b]{0.49\columnwidth}
        \centering
        \newcommand{\hmmdistance}{3.6em}
\newcommand{\vmmdistance}{2.4em}

\begin{tikzpicture}[
    mm/.style={schematic metamodel},
]

\node[mm] (full_left) {};
\node[mm, above right=\vmmdistance and \hmmdistance of full_left.center, anchor=center] (full_top) {};
\node[mm, below right=\vmmdistance and \hmmdistance of full_left.center, anchor=center] (full_bottom) {};
\node[mm, right=2*\hmmdistance of full_left.center, anchor=center] (full_right) {};
\node[mm, below left=\vmmdistance and \hmmdistance of full_left.center, anchor=center] (full_bottomleft) {};

\draw[transformation] (full_left) -- (full_top);
\draw[transformation] (full_left) -- (full_right);
\draw[transformation] (full_left) -- (full_bottom);
\draw[transformation] (full_top) -- (full_right);
\draw[transformation] (full_top) -- (full_bottom);
\draw[transformation] (full_right) -- (full_bottom);
\draw[transformation] (full_left) -- (full_bottomleft);
\draw[transformation] (full_bottom) -- (full_bottomleft);
\draw[transformation] (full_top) to[bend right=30] (full_bottomleft);
\draw[transformation] (full_bottomleft) .. controls ++(1*\hmmdistance, -0.8*\vmmdistance) and ([yshift=-1.5*\vmmdistance]full_right.south) .. (full_right);

\end{tikzpicture}
        \subcaption{Dense Graph}
        \label{fig:improvement:topologies:full}
    \end{minipage}
    \hfill
    \begin{minipage}[b]{0.49\columnwidth}
        \centering
        \newcommand{\hmmdistance}{3.6em}
\newcommand{\vmmdistance}{2.4em}

\begin{tikzpicture}[
    conceptmm/.style={schematic conceptmetamodel},
    concretemm/.style={schematic metamodel},
]

\node[conceptmm, right=4*\hmmdistance of full_left.center, anchor=center] (tree_left) {};
\node[conceptmm, above right=\vmmdistance and \hmmdistance of tree_left.center, anchor=center] (tree_top) {};
\node[concretemm, below right=\vmmdistance and \hmmdistance of tree_left.center, anchor=center] (tree_bottom) {};
\node[concretemm, right=2*\hmmdistance of tree_left.center, anchor=center] (tree_right) {};
\node[concretemm, below left=\vmmdistance and \hmmdistance of tree_left.center, anchor=center] (tree_bottomleft) {};

\draw[transformation] (tree_left) -- (tree_top);
\draw[transformation] (tree_left) -- (tree_bottom);
\draw[transformation] (tree_left) -- (tree_bottomleft);
\draw[transformation] (tree_top) -- (tree_right);

\end{tikzpicture}
        \vspace{1em}
        \subcaption{Tree}
        \label{fig:improvement:topologies:tree}
    \end{minipage}
    \caption[Extremes of transformation network topologies]{Extremes of transformation network topologies: nodes represent metamodels, edges represent transformations (\conceptmetamodels in a tree of \commonalities in dark gray). Adapted from~\owncite[Fig. 4]{klare2019models}.}
    \label{fig:improvement:topologies}
\end{figure}

Networks of transformations can have two extremes of topologies, as depicted in \autoref{fig:improvement:topologies}.
If transformations between all metamodels are defined, the network forms a dense graph (see \autoref{fig:improvement:topologies:full}).
In contrast, if there exists exactly one path of transformations between each pair of metamodels, the network forms a tree (see \autoref{fig:improvement:topologies:tree}).
Several properties for such %transformation 
networks have been identified by \textcite{gleitze2017a} and \textcite{klare2018docsym}.
Two essential properties %, defined in \cite{klare2018docsym}, 
are \emph{compatibility} and \emph{modularity}~\cite{klare2018docsym}, which, unfortunately, contradict each other.
The \commonalities approach, however, improves both of them. %, which is an essential benefit that we discuss in the following. %We discuss the benefits of the \commonalities approach regarding those properties.
\emph{Compatibility} means that transformations do not define contradictory constraints.
Consider the relations introduced for the running example in \autoref{fig:improvement:running_example}.
The names of the same class in Java and UML are defined to be equal.
If a class in Java and UML realizes a \gls{PCM} component, it shall have the same name appended with an \enquote{Impl} suffix.
If transformations realize the three relations between \gls{PCM}, UML and Java, and the one between \gls{PCM} and Java adds that suffix whereas the one between \gls{PCM} and UML omits it, the constraints can never be fulfilled.
%If the three relations between Java, UML and \ac{PCM} are defined in transformations and the one between \ac{PCM} and Java adds that suffix whereas the one between \ac{PCM} and UML omits it, the constraints can never be fulfilled.
In that case, the transformations are considered incompatible.
Incompatibility may arise whenever more than one transformation path between two metamodels exists.
In consequence, compatibility cannot be guaranteed in dense network, whereas it is inherently high if the network forms a tree.
\emph{Modularity} means that any subset of the metamodels can be used without loosing consistency because of missing transformations.
Modularity is high if any metamodel can be removed from the network and the remaining transformations still define consistency between all remaining metamodels.
In consequence, modularity is high in a dense network, because all metamodels are directly related, while it is low if the network is a tree, because inner nodes cannot be removed without their children not being related by a transformation anymore.
Since redundant paths between metamodels improve modularity but reduce compatibility, these properties are inherently contradicting.

The \commonalities approach improves both these properties due to the fact that additional metamodels are introduced in the specification.
The transformations between \metaclasses in \concretemetamodels and \commonalities in \conceptmetamodels induce a tree, thus compatibility is high.
Additionally, only the leaves of the tree are \concretemetamodels, which are actually used to describe a system and whose instances are modified, whereas the inner nodes only represent auxiliary metamodels, exemplarily marked in \autoref{fig:improvement:topologies:tree}. 
In consequence, taking an arbitrary subset of \concretemetamodels removes only leaves and can thus be done without removing any transformations that are necessary to keep instances of the remaining metamodels consistent.
This constitutes a major benefit of the \commonalities approach as compared to ordinary networks of transformations.


% \subsection{Limitations of \commonalities}
% \label{sec:approach:limitations}
% ONE PART MOVED TO COMPOSITION (DAG INSTEAD OF TREES), ONE MOVED TO LIMITATIONS IN EVALUATION SECTION

\end{copiedFrom} % VoSE


\section{Application Processes}
\label{chap:improvement:application}

\mnote{Definition process}
The application of the \commonalities approach requires a process for defining them as well a concept for combining them with other specifications of transformations.
In a specification using the \commonalities approach, the \conceptmetamodels and manifestation relations are not as independent as they are supposed to be in the definition of an ordinary transformation network forming a dense or even complete graph.
Due to the necessity to relate all elements only via one transformation path, even if \commonalities are separated into \conceptmetamodels by concerns and composed hierarchically, the developers must ensure that such a structure is achieved.
We thus subsequently discuss different options how \commonalities can be defined.

\mnote{Combination concept}
We have identified in \autoref{chap:improvement:concepts:specification} that the \commonalities approach is well-suited for structural and \enquote{natural} consistency relations, rather than arbitrarily complex, in particular behavioral dependencies.
%This especially involves structural rather than behavioral consistency relations.
In the following, we discuss options for combining a \commonalities specification with other specifications, in particular ordinary transformations.


\subsection{Defining \Commonalities}

\mnote{Hierarchic composition}
We have discussed in \autoref{chap:improvement:commonalities:composition} how \commonalities and the \conceptmetamodels encapsulating them can be composed hierarchically.
This allows to separate \commonalities by concerns, i.e., by the concepts they belong to.
In addition, it fosters the independent development and reuse of different \conceptmetamodels.

\mnote{Independent development vs. tree structure}
The \commonalities approach does, however, only provide an essential benefit regarding guaranteed correctness of the resulting transformation network if the manifestation relations specify consistency relations that form a consistency relation tree (see \autoref{chap:improvement:commonalities:tree}).
Thus, \commonalities and their \conceptmetamodels must be composed in a way that such a structure is achieved.
This can, in the worst case, require all \concretemetamodels to define consistency between and the according relations to be elicited a priori and thus conflict with our independent development assumption.

\mnote{Bottom-up specification}
An intuitive process to define \commonalities is a bottom-up approach.
Developers select \concretemetamodels that share common concepts and are, by custom definition, most related among the \concretemetamodels to define consistency between and define a \conceptmetamodel of \commonalities between them.
Then, they iteratively choose \conceptmetamodels, and potentially also \concretemetamodels, that share further higher-level commonalities and define an according \conceptmetamodel for them.
This ends up in a hierarchy of \conceptmetamodels.

\mnote{Driven by \concretemetamodels}
Since finally instances of the \concretemetamodels are to be kept consistent, it is important to always consider the information represented in the \concretemetamodels, even if consistency is defined between \conceptmetamodels, i.e., at a higher level in the hierarchy of \conceptmetamodels.
Consider the running example of classes in \gls{UML} and Java, as well as components in \gls{UML}.
We may define an object-oriented design \conceptmetamodel to define \commonalities between \gls{UML} and Java, as well as a component-based design \conceptmetamodel to define \commonalities between object-oriented design and \gls{PCM}, as sketched in \autoref{chap:improvement:commonalities:tree} and depicted in \autoref{fig:improvement:composed_commonalities_example}.
If these \conceptmetamodels are defined in a bottom-up manner, i.e., first defining the object-oriented design \conceptmetamodel and afterwards the component-based design \conceptmetamodels, it is not sufficient to only consider the information represented in the object-oriented design \conceptmetamodels for defining their \commonalities.
That metamodel does only contain the \commonalities that are relevant for object-oriented design, but for the relation to component-based design, further information that is only present in one of the \concretemetamodels may be relevant.
For example, Java contains a definition of behavior in terms of method bodies, which is not represented in the purely structural \gls{UML} class models.
Thus, the object-oriented design \conceptmetamodel does not represent this behavioral information, as it does represent a \commonality.
\gls{PCM}, however, also has an abstract representation of behavior used for predicting the system's performance, which needs to be kept consistent with the precise behavior specification in Java.
Thus, the component-based design \conceptmetamodel must either have an additional manifestation relation to Java for the behavioral information, or the object-oriented design \conceptmetamodel must also contain behavioral information, although not being a \commonality between the \concretemetamodels it represents.

\mnote{Union of all information}
In general, this problem occurs because \conceptmetamodels are supposed to represent the unions of all pairwise intersections of their \concretemetamodels, as those represent the \commonalities that have to be kept consistent.
Information that is unique to one of the \concretemetamodels is not represented in the \conceptmetamodel, but may be relevant for further concepts and thus the relations to define to them.
A first, general solution would require a \conceptmetamodel to contain the union of all information in the \concretemetamodels rather than the union of their pairwise intersections.
This does, however, not conform to the purpose of \conceptmetamodels to only describe \commonalities.
It leads to large and complex \conceptmetamodels and thus also to high effort, because for each \concretemetamodel a transformation, in terms of a manifestation relation, of all its information to a \conceptmetamodel would have to be defined.
In addition, the topmost \conceptmetamodel of the hierarchy would inherently contain the union of information defined in all \concretemetamodels, thus representing a \gls{SUMM}, i.e., a single metamodel that is capable of representing all information to define one system (see \autoref{chap:foundations:multiview}). %, as introduced in \autoref{chap:improvement:concepts:explicit}.
In consequence, it would be sufficient to only manage an instance of that topmost \conceptmetamodel, representing the \gls{SUMM}, and to consider the instances of all other concept and \concretemetamodels as projections from the instance of that central metamodel, according to \textcite{atkinson2010a}.

\begin{figure}
    \centering
    \newcommand{\vdistance}{8em}
\newcommand{\hdistance}{(14em+0.3*\difftoafiveimage)}
\newcommand{\classwidth}{5.5em}
\newcommand{\labeldistance}{1em}
\newcommand{\labelshift}{0.3*\classwidth}
\newcommand{\mmborder}{1em}

\begin{tikzpicture}

\pgfdeclarelayer{bg}
\pgfsetlayers{bg,main}


% METACLASSES

\umlclassvarwidth{java_class}{}{Class}{
name\\
visbility\\
** behavior **
}{\classwidth}

\umlclassvarwidth[, right=0.8*\hdistance of java_class.north, anchor=north]{uml_class}{}{Class}{
name\\
visbility\\
}{\classwidth} 

\umlclassvarwidth[, above right=1.1*\vdistance and 0.5*\hdistance of java_class.north, anchor=north]{oo_class}{}{Class\vphantom{p}}{
name\\
visibility\\
** behavior **
}{\classwidth} 

\umlclassvarwidth[, right=\hdistance of oo_class.north, anchor=north]{pcm_component}{}{Component}{
name\\
** behavior **
}{\classwidth} 

\umlclassvarwidth[, above right=1.1*\vdistance and 0.5*\hdistance of oo_class.north, anchor=north]{component_component}{}{Component}{
name\\
visbility\\
** behavior **
}{\classwidth}

% METAMODELS

\coordinate (java_label_coordinate) at ([yshift=\labeldistance]java_class.north west);
\node[mmlabel, anchor=west] (java_label) at (java_label_coordinate) {Java};

\coordinate (uml_label_coordinate) at ([yshift=\labeldistance]uml_class.north east);
\node[mmlabel, anchor=east] (uml_label) at (uml_label_coordinate) {\acrshort{UML}};

\coordinate (oo_label_coordinate) at ([xshift=-4*\labelshift,yshift=\labeldistance]oo_class.north);
\node[mmlabel, anchor=west, align=left] (oo_label) at ([xshift=-0.5em, yshift=-0.7em]oo_label_coordinate) {Object-Oriented\\ Design};

\coordinate (pcm_label_coordinate) at ([yshift=\labeldistance]pcm_component.north east);
\node[mmlabel, anchor=east] (pcm_label) at (pcm_label_coordinate) {\acrshort{PCM}};

\coordinate (component_label_coordinate) at ([yshift=\labeldistance]component_component.north);
\node[mmlabel, anchor=center] (component_label) at (component_label_coordinate) {Component-Based Design};

\begin{pgfonlayer}{bg}
    \node[mmbg, fit=(java_class)(java_label.west), inner sep=\mmborder] (java) {};
    \node[mmbg, fit=(uml_class)(uml_label.east), inner sep=\mmborder] (uml) {};
    \node[conceptmmbg, fit=(oo_class)(oo_label_coordinate), inner sep=\mmborder] (oo) {};
    \node[mmbg, fit=(pcm_component)(pcm_label.east), inner sep=\mmborder] (pcm) {};
    \node[conceptmmbg, fit=(component_component)(component_label.west)(component_label.east), inner sep=\mmborder] (component) {};
\end{pgfonlayer}


% CONSISTENCY RELATIONS

\draw[manifests relation] (oo_class) -- node[stereotype, above, sloped] {\manifestslabel} (java_class);
\draw[manifests relation] (oo_class) -- node[stereotype, above, sloped] {\manifestslabel} (uml_class);
\draw[manifests relation] (component_component) -- node[stereotype, above, sloped] {\manifestslabel} (oo_class);
\draw[manifests relation] (component_component) -- node[stereotype, above, sloped] {\manifestslabel} (pcm_component);

\end{tikzpicture}

    \caption[\Commonalities with union of all information]{Example for a hierarchy of \conceptmetamodels and their \commonalities, in which \conceptmetamodels represent the union of information in their manifestations. Behavior of classes and components is considered any, not further specified kind of behavioral information.}
    \label{fig:improvement:definition_option_sum}
\end{figure}

\mnote{Example for union}
For the example int \autoref{fig:improvement:composed_commonalities_example} depicting hierarchic \conceptmetamodels for classes and components, we derive an extension according to the discussed scheme in \autoref{fig:improvement:definition_option_sum}.
It additionally contains visibilities for classes and any kind of not further specified behavior description in Java classes and \gls{PCM} components.
Both \conceptmetamodels contain the union information in their manifestations, such that the component-based design \conceptmetamodel contains all information represented in all metamodels.
In consequence, the component-based design \conceptmetamodel represents the visibility of classes in object-oriented design, although it is not relevant for components and is not kept consistent via that \conceptmetamodel.

\mnote{Non-strict manifestations}
The previous considerations assume a kind of strict layered architecture (see~\cite{buschmann1996PatternsArchitecture-Book}) in which the manifestation relations induce a tree between the metamodels, thus no manifestation relation bypasses a \conceptmetamodel to whose \commonalities additional manifestation relations are defined.
Referring to a non-strict layered architecture, another solution would be to allow manifestation relations to the manifestations of \conceptmetamodels to which further manifestation relations are defined, e.g., the component-based design \commonalities may have manifestation relations to elements in Java and \gls{UML} in addition to manifestation relations to the object-oriented design \conceptmetamodels, which in turn has manifestation relations to those \concretemetamodels.
A drawback of this solution is that it can likely violate the goal of achieving a tree structure.
Considering a class in Java as a manifestation of a component in component-based design, as well as a class in object-oriented design, which in turn is a manifestation of a component in component-based design, would already violate the definition of a consistency relation tree, thus not giving guarantees regarding compatibility.

\begin{figure}
    \centering
    \input{figures/quality/improvement/definition_option_bypass.tex}
    \caption[\Commonalities with multiple manifestation]{Example for a hierarchy of \conceptmetamodels and their \commonalities, in which \commonalities may have several manifestations inducing consistency relations that do not form a tree structure. Behavior of classes and components is considered any, not further specified kind of behavioral information.}
    \label{fig:improvement:definition_option_bypass}
\end{figure}

\mnote{Example for non-strict manifestations}
\autoref{fig:improvement:definition_option_bypass} depicts this solution for the already discussed example.
The \conceptmetamodels contain only the information relevant for the \commonalities they represent.
The additional manifestation relation between components of the component-based design \conceptmetamodel and classes in Java induce the violation of a tree structure as sketched before.
Although behavior may actually be represented in terms of method bodies represented as separate \metaclasses in Java, still consistency relations defined by the manifestation relations between Java and the object-oriented design \conceptmetamodel would include both classes and methods, as methods do not share an isolated consistency relation between Java and \gls{UML} but only in the context of the class they belong to.

\mnote{Union including concepts}
A third option is to construct a \conceptmetamodel not only driven by the \commonalities shared between its manifestations, but also by its \commonalities with other metamodels.
Thus, whenever a \conceptmetamodel is used as a manifestation of another \conceptmetamodel, it may be extended by the information from its manifestations required for the \commonalities in another concept with other metamodels.
For example, as soon as the object-oriented design \conceptmetamodel is considered as a manifestation of component-based design, its manifestations, namely Java and \gls{UML}, are checked for \commonalities with component-based design that are not yet considered \commonalities regarding object-oriented design.
This could be a description of method bodies in Java to keep consistent with the behavior specification in \gls{PCM}.
If consequently followed, such an approach would result in \conceptmetamodels not only representing the union of the pairwise intersections of the manifestations, but the union of the pairwise intersections of their manifestations with all other \concretemetamodels to be kept consistent.
This still promises to lead to \conceptmetamodels that are significantly smaller and more precise than the union of all metamodels as in the first option, but still allow to achieve a tree structure, which is why we propose to use this option.
This approach is comparable to the situation in which a further manifestation shall be added, like we exemplarily discussed for adding \cplusplus as a manifestation of the object-oriented design \conceptmetamodel in \autoref{chap:improvement:benefits:specification_effort}.

\begin{figure}
    \centering
    \input{figures/quality/improvement/definition_option_super_union.tex}
    \caption[\Commonalities including information of their concepts]{Example for a hierarchy of \conceptmetamodels and their \commonalities, in which \commonalities represent information necessary for the concepts they are manifestations of in addition to the information shared by their manifestations. Behavior of classes and components is considered any, not further specified kind of behavioral information.}
    \label{fig:improvement:definition_option_super_union}
\end{figure}

\mnote{Example for union including concepts}
The application of this option to the already discussed example is depicted in \autoref{fig:improvement:definition_option_super_union}.
In this solution, still a tree structure between the \metaclasses and \commonalities is given and the \conceptmetamodels are still restricted to the information in the manifestations and, in addition, the information of the manifestations necessary for the \conceptmetamodels of which they are manifestations.
This is why the object-oriented design \conceptmetamodel contains information about the behavior of classes and components, although \gls{UML} and Java do not share behavioral concepts, but the component \commonality for component-based design does not contain the visibilities of classes as in the first option of representing the union of all information in the manifestations.

\mnote{Problem mitigation by cliques}
Finally, it is still an open question how problematic the actual dependencies in practical scenarios are.
Potentially, only subsets of few metamodels are highly related and share large parts of one or more concepts, and the relation to other such subsets is only given across one metamodel or one concept.
This could be seen as a graph of cliques, in which some metamodels are highly related whereas the relation to others is rather loose.
In that case, it can be reasonable to define relations in these cliques by means of \commonalities and then define the loose relations to other cliques by means of an ordinary transformation, as we discuss in the subsequent section.
We derive first insights on the achievability of the required tree structure for \commonalities in our evaluation in \autoref{chap:commonalities_evaluation}, but further evidence if one of the previously discussed strategies can be reasonably applied has to be gained in larger studies in practical scenarios with more metamodels of more tools.

% Discuss how commonalities can be defined. Which roles are involved, especially how different domain experts can communicate. E.g., bottom-up approach: Take the most related metamodels and define their commonalities. Then define higher-level commonalities relating these concept metamodels or even the concrete metamodels. 
% The problem is that combining two concept metamodels requires them to contain all necessary information, thus a concept metamodel design is not only driven by the metamodels to keep directly consistent, but also by the information that is needed to preserve consistency to other commonalities. This refers to the same scenario as if multiple commonality structures are encapsulated into projective view environments which are combined by BX. They require the exposed views to provide all information necessary to preserve consistency.

% Prozess zur Erzeugung von Commonalities beschreiben:
% - Optimalerweise kennt man alle MM vorher und kann sinnvolle Commonalities bauen (immer das am wenigstens redundante Konstrukt, z.B. eher Person in Familie als Persons mit Familiennamen (redundante Darstellung des Familiennamens)
% - Bei Erweiterung um zusätzliche Metamodelle sind ggf. Anpassungen an Commonalities notwendig. Hier ist wieder die Lokalität bei den Commonalities vorteilhaft (interne Spezifikation), weil man dann eine Commonality insgesamt ersetzen kann statt in jeder Transformation deren Manifestierung anzupassen.

% Evolutionsprozess:
% - Wie eignet sich welcher Ansatz zur Erweiterung mit MM. I.d.R. reicht es nicht eine Verbindung zu ergänzen, sondern die Konzeptmetamodelle für angepasst werden für ein neuen zu koppelndes Metamodell


\subsection{Combining \Commonalities}

\mnote{Necessity for combination}
We have up to now discussed how to construct \conceptmetamodels and manifestation relations in terms of the \commonalities approach, such that the topology of the defined relations fulfills the definition of a consistency relation tree to achieve inherent guarantees regarding correctness of the transformation network.
We have also derived how the \commonalities approach improves reusability in comparison to the construction of a transformation network with tree topology out of the \concretemetamodels.
Nevertheless, the approach has at least two limitations, which we have already identified.
First, it lacks completeness, as it requires a specific topology of consistency relations to be achievable, which is likely to get more complex the more metamodels are involved.
Second, it only fits well for structural relations in which actual commonalities can be described or prescribed.

\mnote{\Commonalities for subsets}
In consequence, to improve applicability of the approach, it should be applied for subsets of metamodels that inherently share commonalities, comparable to the cliques mentioned before, which are suited to be described with the proposed approach.
These specifications should then be combined with other consistency specifications, be they defined with the \commonalities approach or with ordinary transformations.
Such a combination would restrict the size and complexity of a hierarchy of \commonalities and could foster reuse of consistency specifications for specific concepts in different context, as motivated by our assumptions of independent development and modular reuse, as well as the process proposed in \autoref{chap:networks:specification_process}.

\begin{figure}
    \centering
    \newcommand{\mmdistance}{8em}

\begin{tikzpicture}[
    mm/.style={schematic metamodel},
    conceptmm/.style={schematic conceptmetamodel}
]

\node[mm] (original_left) {$\metamodel{A}{}$};
\node[mm, right=\mmdistance of original_left.center, anchor=center] (original_middle) {$\metamodel{B}{}$};
\node[mm, right=\mmdistance of original_middle.center, anchor=center] (original_right) {$\metamodel{C}{}$};
\node[conceptmm, above right=0.8*\mmdistance and 0.5*\mmdistance of original_left.center, anchor=center, align=center] (concept) {$\metamodel{AB}{\mathvariable{Concepts}}$};

\draw[consistency relation] (original_left) -- node[above] {$\consistencyrelation{CR}{AB}$} (original_middle);
\draw[consistency relation] (original_middle) to[bend left=30] node[above] {$\consistencyrelation{CR}{BC}$} (original_right);
\draw[consistency relation] (original_left) to[bend right=35] node[above] {$\consistencyrelation{CR}{AC}$} (original_right);

\draw[consistency relation, color=gray] (original_left) -- node[above left=-0.3em and 0.5em] {$\consistencyrelation{CR}{A}$} (concept);
\draw[consistency relation, color=gray] (original_middle) -- node[above right=-0.3em and 0.3em] {$\consistencyrelation{CR}{B}$} (concept);
\draw[consistency relation, color=gray] (original_right) to[bend right=30] node[pos=0.6, below left] {$\consistencyrelation{CR}{C}$} (concept);


\node[right=3*\mmdistance+0.4*\difftoafiveimage of concept.north, anchor=north east, align=left] (constraints_top) { 
$\consistencyrelation{CR}{AB} \concat \consistencyrelation{CR}{BC} \neq (\consistencyrelation{CR}{AB} \concat \consistencyrelation{CR}{BC}) \cap \consistencyrelation{CR}{AC} \neq \consistencyrelation{CR}{AC}$
};

\node[below=2em of constraints_top.north east, anchor=north east, align=left] {
$\consistencyrelation{CR}{A} \concat \consistencyrelation{CR}{B} = \consistencyrelation{CR}{AB}$\\
$\consistencyrelation{CR}{A} \concat \consistencyrelation{CR}{C} = \consistencyrelation{CR}{AC}$\\
$\consistencyrelation{CR}{B} \concat \consistencyrelation{CR}{C} = \consistencyrelation{CR}{BC}$\\
};

\end{tikzpicture}
    \caption[Partial transformation network of \commonalities]{Example for a \conceptmetamodel $\metamodel{AB}{\mathvariable{Concepts}}$ to replace a consistency relation, and the replacement of ordinary consistency relations to the \concretemetamodels with one to the \conceptmetamodel. Adapted from~\owncite[Fig.~5]{klare2018docsym}.}
    \label{fig:improvement:commonalities_combination_generic}
\end{figure}

\mnote{General combination requirements}
To preserve the benefits of a \commonalities specification, it can be combined with other specifications, be they ordinary transformations or another \commonalities specification, by considering any of the other metamodels as a manifestation or a \conceptmetamodel of one of the \conceptmetamodels of the \commonalities specifications.
This preserves the tree structures of the \commonalities specification and its benefits.
Consider the generic example in \autoref{fig:improvement:commonalities_combination_generic} with three metamodels, a \conceptmetamodel for two of them and consistency relations between them, which are considered \modellevelconsistencyrelations according to \autoref{def:modellevelconsistencyrelation} for reasons of simplicity.
The consistency relation $\consistencyrelation{CR}{AB}$ between metamodels $\metamodel{A}{}$ and $\metamodel{B}{}$ is expressed by a \conceptmetamodel $\metamodel{AB}{\mathvariable{Concepts}}$ and consistency relations for the according manifestation relations $\consistencyrelation{CR}{A}$ and $\consistencyrelation{CR}{B}$.
In addition, the metamodel $\metamodel{C}{}$ shares consistency relations with both other metamodels.
To preserve reusability and the necessary tree structure, these consistency relations $\consistencyrelation{CR}{AB}$ and $\consistencyrelation{CR}{AC}$ should be described in terms of a consistency relation $\consistencyrelation{CR}{C}$ to the concept metamodel.
This does, however, require the \conceptmetamodel to contain all information that is necessary to preserve consistency between $\metamodel{C}{}$ and the two others, as described with the required relations in \autoref{fig:improvement:commonalities_combination_generic}.
In contrast to the scenarios discussed in the previous section for how to define \conceptmetamodels and which information to put into them, if $\metamodel{C}{}$ is a part of a different consistency specification to combine the \commonalities specification with, or if the \commonalities specification covers more than two \concretemetamodels with one \conceptmetamodel, this can require an arbitrarily complex adaptation, which may even not be wanted of possible at all if modular reuse is desired.

\mnote{Virtualization by views}
To improve such a combination of specifications, virtualization concepts as known from \gls{OSM}~\cite{atkinson2010a} (see \autoref{chap:foundations:multiview:osm}) and the \vitruv approach~\owncite{klare2021Vitruv-JSS} (see \autoref{chap:foundations:multiview:vitruv}) can be applied.
Their idea is to encapsulate metamodels and their instances behind a facet of views and to enable access to the actual models only via these views.
Views are projections of the encapsulated models, i.e., they derive all information from the models and potentially aggregate them or arrange them differently.
The metamodels of these views are called \emph{\viewtypes}.
While those approaches were originally designed to provide a well-defined interface through views for developers and internally ensure consistency of the persisted artifacts by either avoiding or managing redundancy, they can also be used as an interface for consistency preservation.
In the \vitruv approach, a so called \gls{VSUM} is composed of models and rules for preserving their consistency, whose contents are exposed by views to be modified by developers.

\begin{figure}
    \centering
    % requires tikzvitruvius.sty
\begin{tikzpicture}[
    viewtype/.style={circle, draw, solid, fill=white, inner sep=.1em, font=\scriptsize},
    polarrow/.style={latex-latex, densely dotted},
    legendnode/.style={inner sep=.4em, legend},
    uniformly sized package/.style={minimum width=2.5em},
]

% V-SUM
\umlpackage[uniformly sized package]{oo}{}{OOD}
\umlpackage[uniformly sized package]{uml}{below left=4.5em and 0em of oo}{\acrshort{UML}}
\umlpackage[uniformly sized package]{java}{below right=4.5em and 0em of oo}{Java}

\node[circle,draw,thick,fit=(oo.center)(uml)(java),inner sep=2ex] (sum) {};

\draw[manifests relation] (oo) -- node[stereotype, above, sloped] (pcmJavaCCR) {\manifestslabel} (uml.north);
\draw[manifests relation] (oo) -- node[stereotype, above, sloped] (pcmJavaCCR) {\manifestslabel} (java.north);

\node[viewtype] (viewtypeOO) at (sum.60) {$\mathvariable{VT}_\mathvariable{OOD}$};
\node[viewtype] (viewtypeJava) at (sum.5) {$\mathvariable{VT}_\mathvariable{Java}$};
\draw[polarrow] (oo) -- (viewtypeOO);
\draw[polarrow] (java) -- (viewtypeJava);

\node[font=\bfseries\footnotesize] (sumtext) [above=0.6em of sum.270, anchor=south, align=center] {\vsum\\ Metamodel};

% Outer transformations
\umlpackage[uniformly sized package]{pcm}{above right=-0.5em and 13em of oo}{\acrshort{PCM}};
\draw[transformation] (viewtypeOO) -- node[stereotype, above, sloped] {Structure} (pcm);
\draw[transformation] (viewtypeJava) -- node[stereotype, above, sloped] {Behavior} (pcm);

% Legend
\node[legendbg, matrix, inner sep=0.7em, nodes=legendnode] (legend) at (17.5em,-5.8em) {%
    \umlpackage[minimum height=0.1em, inner sep=0.3em, yshift=0.5ex, xshift=1.25em]{legend_mm}{}{MM} & \node[anchor=base west] {Metamodel\sameheight};\\
    \node[viewtype, xshift=1.25em, yshift=0.5ex] {$\mathvariable{VT}$}; & \node[anchor=base west] {\ViewType\sameheight};\\
    \draw[polarrow] (0,.5ex)--(2.5em,.5ex); & \node[anchor=base west] {View Transformation\sameheight};\\
    \draw[transformation] (0,.5ex)--(2.5em,.5ex); & \node[anchor=base west] {Transformation\sameheight};\\
};

\end{tikzpicture}

    %\includegraphics[width=0.6\textwidth]{figures/quality/improvement/combination_external_metamodel.png}
    \caption[Combination of \commonalities with a transformation]{Example for the combination of a \commonalities specification for object-oriented design (OOD) with \gls{PCM} by encapsulation into a \vsumm.}
    \label{fig:improvement:combination_external_metamodel}
\end{figure}

\mnote{Encapsulation of \commonalities in \vsum}
Consider the example depicted in \autoref{fig:improvement:combination_external_metamodel}.
It comprises the \commonalities specification for Java and \gls{UML} using a single \conceptmetamodel for object-oriented design.
This consistency specification by means of \commonalities is encapsulated into a \vsum, which exposes the Java code via a Java view and the object-oriented structure represented in instances of the \conceptmetamodel as an object-oriented view.
These two views are then related to \gls{PCM} by means of ordinary consistency relations and transformations preserving them.
The relations between metamodels and \viewtypes can, again, be considered ordinary transformations.
Thus, the defined transformation network would actually contain cycles, such that it does not benefit from the \commonalities specification within the \vsum in terms of correctness.
If we do only consider the \vsum itself, it does, however, still have a tree structure, so if only one of the views is modified at the same time, it provides the benefits that we have discussed for a \commonalities specification in \autoref{chap:improvement:benefits}.
In addition, views of a \vsum by now actually assume that only one of them is changed at a time~\owncite{klare2021Vitruv-JSS}, as a developer is supposed to work on one specific view at a time.
Thus, if the transformations outside the \vsum ensure that only one of the views is changed at a time, the \vsum provides the discussed benefits of the \commonalities approach.

\mnote{Clarification of responsibilities}
This approach does, of course, not solve possible issues regarding synchronization and orchestration in the transformation network defined outside the \vsum, but only moves the problem of avoiding these issues away from the \commonalities specification by making according assumptions in terms of allowing only modifications of one view of a \vsum.
It does, however, clarify responsibilities, as there are precisely defined views across which other metamodels can be combined with those for which consistency is defined by means of \commonalities, rather than defining consistency to the metamodels within the \commonalities specification directly and thus breaking the necessary assumption for the intended benefits of that approach.
In the example, we have a clear separation into views for the structure of the object-oriented representation in Java, \gls{UML} and potentially more metamodels, and for its behavior.
It is up to the developer of the transformation network outside the \vsum to ensure that no problems like execution loops occur by assigning clear non-conflicting responsibilities to the two transformations for structure and behavior of the \vsum to \gls{PCM}.

\begin{figure}
    \centering
    % requires tikzvitruvius.sty
\begin{tikzpicture}[
    viewtype/.style={circle, draw, solid, fill=white, inner sep=.1em, font=\scriptsize},
    polarrow/.style={latex-latex, densely dotted},
    legendnode/.style={inner sep=.4em, legend},
    uniformly sized package/.style={minimum width=2.5em},
]

% V-SUM 1
\umlpackage[uniformly sized package]{oo}{}{OOD}
\umlpackage[uniformly sized package]{uml}{below left=4.5em and 0em of oo}{\acrshort{UML}}
\umlpackage[uniformly sized package]{java}{below right=4.5em and 0em of oo}{Java}

\node[circle,draw,thick,fit=(oo.center)(uml)(java),inner sep=2ex] (sum1) {};

\draw[manifests relation] (oo) -- node[stereotype, above, sloped] (pcmJavaCCR) {\manifestslabel} (uml.north);
\draw[manifests relation] (oo) -- node[stereotype, above, sloped] (pcmJavaCCR) {\manifestslabel} (java.north);

\node[viewtype] (viewtypeOO) at (sum1.40) {$\mathvariable{VT}_\mathvariable{OOD}$};
\node[viewtype] (viewtypeJava) at (sum1.5) {$\mathvariable{VT}_\mathvariable{Java}$};
\draw[polarrow] (oo) -- (viewtypeOO);
\draw[polarrow] (java) -- (viewtypeJava);

\node[font=\bfseries\footnotesize] (sum1text) [above=0.6em of sum1.270, anchor=south, align=center] {\vsum\\ Metamodel};

% V-SUM 2
\umlpackage[uniformly sized package]{cbd}{right=14.8em+0.5*\difftoafiveimage of oo}{CBD}
\umlpackage[uniformly sized package]{pcm}{below left=4.5em and 0em of cbd}{\acrshort{PCM}}
\umlpackage[uniformly sized package]{umlcomp}{below right=4.5em and 0em of cbd}{\acrshort{UML}}

\node[circle,draw,thick,fit=(cbd.center)(pcm)(umlcomp),inner sep=2ex] (sum2) {};

\draw[manifests relation] (cbd) -- node[stereotype, above, sloped] (pcmJavaCCR) {\manifestslabel} (pcm.north);
\draw[manifests relation] (cbd) -- node[stereotype, above, sloped] (pcmJavaCCR) {\manifestslabel} (umlcomp.north);

\node[viewtype] (viewtypeStruc) at (sum2.140) {$\mathvariable{VT}_\mathvariable{Structure}$};
\node[viewtype] (viewtypeBehav) at (sum2.175) {$\mathvariable{VT}_\mathvariable{Behavior}$};
\draw[polarrow] (cbd) -- (viewtypeStruc);
\draw[polarrow] (pcm.north west) -- (viewtypeBehav);

\node[font=\bfseries\footnotesize] (sum2text) [above=0.6em of sum2.270, anchor=south, align=center] {\vsum\\ Metamodel};

% Outer transformations
\draw[transformation] (viewtypeOO) -- (viewtypeStruc);
\draw[transformation] (viewtypeJava) -- (viewtypeBehav);

\coordinate (middle) at ($(oo)!0.5!(cbd)$);

% Legend
\node[legendbg, matrix, inner sep=0.7em, nodes=legendnode, anchor=north, below=10.5em of middle] (legend) {%
    \umlpackage[minimum height=0.1em, inner sep=0.3em, yshift=0.5ex, anchor=center]{legend_mm}{}{MM} & \node[anchor=base west] {Metamodel \hspace{0.5em}\sameheight}; &
    \draw[polarrow] (0em,0.5ex)--(2em,0.5ex); & \node[anchor=base west] {View Transformation\sameheight}; \\
    \node[viewtype, anchor=center, yshift=0.5ex] {$\mathvariable{VT}$}; & \node[anchor=base west] {\ViewType\sameheight};&
    \draw[transformation] (0em,0.5ex)--(2em,0.5ex); & \node[anchor=base west] {Transformation\sameheight};\\
};

\end{tikzpicture}

    %\includegraphics[width=\textwidth]{figures/quality/improvement/combination_two_vsums.png}
    \caption[Combination of two \commonalities specifications]{Example for the combination of two \commonalities specifications for object-oriented (OOD) and component-based design (CBD) by encapsulation into \vsumms.}
    \label{fig:improvement:combination_two_vsums}
\end{figure}

\mnote{Combination of encapsulating \vsums}
Instead of only \gls{PCM}, there could be a more complex transformation network, or another \commonalities specification, which may again be encapsulated into a \vsum and provide its own views, across which both \vsums can then be combined.
\autoref{fig:improvement:combination_two_vsums} depicts such an example, in which \gls{PCM} and \gls{UML} component models are related by a \conceptmetamodel for component-based design, encapsulated into a second \vsum.
This \vsum provides separate \viewtypes for the structure represented by both \gls{PCM} and \gls{UML} and thus reflected in the \conceptmetamodels, and for the behavior only represented in \gls{PCM}.
These \viewtypes can then be combined by means of ordinary transformations with those of the \vsum for object-oriented design.
Again, this approach does not prevent the occurrence of correctness issues as discussed in \autoref{part:correctness} due to the transformations outside the \vsum, but at least within each \vsum we can guarantee correctness.

\mnote{Hierachic composition of \vsums}
This approach can even be hierarchically be composed, such that several kinds of specifications, including encapsulating \vsums, are again encapsulated into another \vsum.
For example, the \vsums in \autoref{fig:improvement:combination_two_vsums} could be encapsulated into a \vsum for object-oriented and component-based design to be reused together.
If the transformation network between the inner \vsums is correct, which can also be achieved by defining \commonalities between the views of these \vsums again, the composed \vsum again guarantees correctness and can provide well-defined views for different concerns of component-based and object-oriented design.

\mnote{Required evidence}
The sketched approaches for combining \commonalities specification with other kinds of consistency specifications have to be considered as conceptual ideas which promise to provide the benefits of specifying modular, reusable specifications that ease the achievement of correctness.
They have, however, not been applied yet. 
Thus, their actual applicability still has to be practically evaluated in case studies.


% \mnote{Encapsulation of Commonalities with Views}
% \todo{Composing commonality structures: encapsulate them (by views?) -> refer to \autoref{chap:networks:specification_process} for different network developers composing different networks}

% Discuss here, that a commonalities structure can be encapsulated in views (ref to Vitruv), which are then used to combine with other such structures in an ordinary network of BX. E.g., let there be an OO commonality for Java and UML and one CBS commonality for UML and PCM. Both are encapsulated in a projective view-based approach, which, e.g., exposes the concept metamodel. These views can than be combined by ordinary bx. This allows to build concept metamodels for subsets of the problem (subsets of the metamodels), especially for scenarios in which descriptive relations exist, which are then combined by ordinary networks. This gives the benefits of commonalities, such as extendability, modularity (which are preserved even if the concept is combined with others by bx), but also provides the flexibility of bx networks, but reduced the proneness to errors in the networks as parts are handled by inherently compatible commonalities.

% See for example \autoref{fig:improvement:concept_metamodel_integration} for a scenario combining normative and descriptive relations. We could compare a scenario where Java, UML class, UML comp and PCM are connected in a network, connected in an overall commonality and with two commonalities combined in a network.

% Drawback of this approach is that the views exposed the structures have to provide all required information to be kept consistent with other structures. For example, the CBS commonality only contains the information shared between UML and PCM, thus if there is information in PCM to be shared with Java, but not with UML component, the concept metamodel does not contain that, but has to be exposed to be kept consistent with the OO concept. Thus, there may be more extensive views than only exposing the commonality. In fact, the structure would need to be a SUM, for which any information can be extracted. However, it is an open issue how consistency is preserved if information is derived to different views which are all modified, or if a heterogeneous view is created (ModelJoin). Imagine the consistency preservation derives the commonalities view for components to modify the information shared between UML and PCM and uses the PCM view to change information only present in PCM (e.g., functionality). If a change in Java requires modifications in both views, these changes both have to be propagated to the underlying models. If there are conflicts, they have to be resolved like in a synchronization scenario (several user modify views concurrently). This problem is yet unsolved.


% \begin{copiedFrom}{DocSym}

% Instead, we propose to make these common concepts explicit in so-called \emph{\glspl{CMM}} and define relations between them and the concrete metamodels.
% We illustrate this in \autoref{fig:improvement:concept_metamodel_integration}.
% The descriptive consistency relation \ref{fig:improvement:concept_metamodel_integration:R1} is converted into a \gls{CMM} for the metamodels \ref{fig:improvement:concept_metamodel_integration:A} and \ref{fig:improvement:concept_metamodel_integration:B} with new relations \ref{fig:improvement:concept_metamodel_integration:R4} and \ref{fig:improvement:concept_metamodel_integration:R5} between the concrete metamodels and the \gls{CMM}.
% The existing normative consistency relations \ref{fig:improvement:concept_metamodel_integration} and \ref{fig:improvement:concept_metamodel_integration:R3} to metamodel \ref{fig:improvement:concept_metamodel_integration:C} are replaced by a new relation \ref{fig:improvement:concept_metamodel_integration:R6} to the \gls{CMM}. % to the metamodel \ref{fig:concept:C}.
% The \gls{CMM} and its consistency relations have to be appropriately defined to replace the original ones, as depicted in \autoref{fig:improvement:concept_metamodel_integration}. %, have to be fulfilled by appropriately defining the \ac{CMM}.
% It will be part of our research to figure out how to define such a \gls{CMM}, so that it can also be combined with other metamodels. %, without estimating the additional information that may be necessary in the \ac{CMM} a-priori.
% While it basically has to contain the common concepts of the metamodels sharing a descriptive consistency relations, it may also need to contain additional information depending on consistency relations to other metamodels, which are not known a-priori.

% \begin{figure}
%     \centering
%     \newcommand{\mmdistance}{8em}

\begin{tikzpicture}[
    mm/.style={draw, circle, fill=lightgray, inner sep=0.25em},
    consistencyrel/.style={latex-latex,dashed}]


\node[mm] (original_left) {\mylabel{fig:quality:concept_metamodel_integration:A}{$A$}};
\node[mm, right=\mmdistance of original_left.center, anchor=center] (original_middle) {\mylabel{fig:quality:concept_metamodel_integration:B}{$B$}};
\node[mm, right=\mmdistance of original_middle.center, anchor=center] (original_right) {\mylabel{fig:quality:concept_metamodel_integration:C}{$C$}};

\draw[consistencyrel] (original_left) -- node[above] {\mylabel{fig:quality:concept_metamodel_integration:R1}{$R_1$}} node[below] {\textit{descriptive}} (original_middle);
\draw[consistencyrel] (original_middle) to[bend left=30] node[above] {\mylabel{fig:quality:concept_metamodel_integration:R2}{$R_2$}} node[below=0.2em] {\textit{normative}} (original_right);
\draw[consistencyrel] (original_left) to[bend right=40] node[above] {\mylabel{fig:quality:concept_metamodel_integration:R3}{$R_3$}} node[below] {\textit{normative}} (original_right);


\node[mm, fill=gray!10, above right=0.55*\mmdistance and 0.5*\mmdistance of original_left.center, anchor=center, align=center] (concept) {$AB-$\\$CMM$};

\draw[consistencyrel, color=gray] (original_left) -- node[above left=-0.3em and 0.5em] {\mylabel{fig:quality:concept_metamodel_integration:R4}{$R_4$}} (concept);
\draw[consistencyrel, color=gray] (original_middle) -- node[above right=-0.3em and 0.3em] {\mylabel{fig:quality:concept_metamodel_integration:R5}{$R_5$}} (concept);
\draw[consistencyrel, color=gray] (original_right) to[bend right=30] node[pos=0.6, below left] {\mylabel{fig:quality:concept_metamodel_integration:R6}{$R_6$}} (concept);



\node[right=2.5*\mmdistance of concept.north, anchor=north east, align=left] { 
$R_1 \concat R_2 \neq (R_1 \concat R_2) \cap R_3 \neq R_3$
};

\node[below right=2em and 2.5*\mmdistance of concept.north, anchor=north east, align=left] {
$R_4 \concat R_5 = R_1$\\
$R_4 \concat R_6 = R_3$\\
$R_5 \concat R_6 = R_2$\\
};


\end{tikzpicture}
%     \caption{Definition of a concept metamodel}
%     \label{fig:improvement:concept_metamodel_integration}
%     \todo{We can use this for showing how to integrate commonalities with ordinary direct relations, maybe there should be a section about that.}
% \end{figure}

% \end{copiedFrom} % DocSym
\section{Summary}

In this section, we have discussed how the insights regarding effects of different network topologies on the quality properties of a transformation network can be used to mitigate trade-offs between them.
We have motivated a different way of considering consistency in terms of making common concepts explicit as \emph{commonalities} instead of implicitly encoding them into consistency relations.
We have used this way of specifying consistency to propose a construction approach for transformation networks that results in a tree topology providing inherent benefits regarding correctness, but also provides high reusability due to the actual metamodels, whose instances are used to describe a system, being leaves of the tree induces by the transformation network.
We conclude this chapter with the following central insight.

\begin{insight}[Trade-Off Mitigation]
    Quality properties of transformation networks are influenced by the network's topology.
    Especially correctness and reusability are contrary properties, which induce a trade-off depending on whether the network topology is rather a dense or a sparse graph.
    The drawback regarding reusability in networks with tree topology arises from the fact that the metamodels represented by the inner nodes of the tree cannot be easily omitted, as consistency between several other metamodels is expressed across them.
    This can be mitigated by ensuring that the metamodels represented by the inner nodes are auxiliary artifacts and not the actual metamodels used by developers.
    This matches with a different way of thinking about consistency in terms of making the commonalities between metamodels to keep consistent explicit in addition metamodels rather than encoding them implicitly in consistency relations.
    Following such a specification approach leads to a network that improves both correctness and reusability, which are contradictory if only considering transformations between the metamodels whose instances are actually used by developers.
    Such an approach can even be used to define consistency partially for some of the metamodels and then combine it with other consistency specifications, such as ordinary transformations.
    To still have the same guarantees regarding correctness and reusability, such a specification can be encapsulated behind views, which provide projections of the information within the actual models and do only allow one of them to be updated at a time.
\end{insight}


% \begin{copiedFrom}{VoSE}

% CONTEXT
% What we need transformations for
% Modern %software and 
% software-intensive systems are usually described by different \emph{models}, also considered as \emph{views}, such as code, architecture, deployment specifications and other role- or concern-specific models.
% %\modified{Those models can also be considered as views of the system, which synthetically form the complete description of a system.}
% Since all these models describe the same system, but focus on specific properties or use different levels of abstraction, they typically share information that is represented in the models redundantly, or at least induces dependencies between them.
% Such redundancies have to be kept consistent to achieve a contradiction-free specification of the %software 
% system.

% \begin{itemize}
%     \item When several models are used to describe the same software system, they usually share some information
%     \item Two models can describe different properties of the systems, or describe the same properties on a different level of abstraction
%     \item This leads to dependencies or even redundancies between the models
%     \item While arbitrary dependencies often base on manually defined consistency relations (e.g. an architectural component shall be represented as a class in OO-design and implementation), redundancies occur because the same elements are represented in different models
% \end{itemize}

% PROBLEM
% What transformations are
% In practice, keeping redundancies consistent is a task that is often performed manually~\cite{sax2017survey, guissouma2018study}.
% A common means to automate consistency preservation are \emph{incremental model transformations}.
% Such transformations define how an instance of one metamodel has to be updated to restore consistency whenever an instance of another was modified.
% A transformation can be declared in different ways: %can be defined on different abstraction levels: it 
% it may either specify what has to be changed to restore consistency %after a modification 
% (\emph{imperative}), or only the consistency constraints that have to hold, from which the rules to restore consistency are derived automatically (\emph{declarative}).
% %A transformation can be rather imperative, defining what has to be changed to restore consistency after a modification, or rather declarative, defining only the consistency constraints that have to hold and automatically deriving rules to restore consistency when a constraint gets violated.
% But no matter how a transformation is defined, it specifies a \emph{relation} between two (or more) metamodels.
% However, redundant elements are representations of a \emph{common concept} rather than independent elements that have to be directly related.
% In consequence, we think that it is natural to make the common concept, which the redundant elements are supposed to describe, explicit. 
% We can then specify how this concept manifests itself in the different metamodels, instead of defining directly how the redundant elements are related.
% For example, %we assume that it is 
% it appears to be more natural to say that classes in UML and Java are different manifestations of the concept of a class in object-oriented design, rather than saying that a UML class should be related to a corresponding Java class.
% Such a common concept is what we call a \emph{\commonality}.

% Transformations:
% \begin{itemize}
%     \item Incremental transformations are a common means to keep several models consistent
%     \item A transformation defines, how elements in different models are related and what should be done if one of them is changed
%     \item Consistency relations are usually defined declaratively (what constraints have to hold) or imperatively (how are constraints enforced) between two (or sometimes more) metamodels. This is a definition of when instances of that metamodels are considered consistent.
%     \item However, if transformations have to preserve consistency especially of redundant elements, these redundant elements in fact represent a common element in different ways
%     \item Instead of defining how redundant elements are related, it would be more natural to define that there is common element to be describe, and how this common element manifests in the different models
%     \item This is what we call a \emph{Commonality}
% \end{itemize}

% APPROACH
% In this paper, we present the \emph{\commonalities approach}. %, which is based on the Bachelor's thesis of \textcite{gleitze2017a}.
% % Vorschlag:
% It defines \commonalities between metamodels explicitly and thereby allows to state clearly which common concepts they share.
% %, instead of encoding this information implicitly in a transformation.
% %The approach aims to explicitly define commonalities of metamodels for describing their consistency relations rather than implicitly encoding them in a transformation.
% From such a specification, transformations are derived that keep instances of those metamodels consistent.
% We discuss options how to derive such transformations, strategies to hierarchically compose \commonalities, as well as benefits and limitations of the approach.
% Additionally, we discuss design options for a language that supports the specification of \commonalities and present the \emph{\commonalities language}, which we have developed as a proof-of-concept.
% It is based on the Bachelor's thesis of \textcite{gleitze2017a}. 
% Commonalities approach, based on thesis by \textcite{gleitze2017a}:
% \begin{itemize}
%     \item We propose the \emph{Commonalities Approach} that aims to explicitly define commonalities of models for describing their consistency relations rather than encoding them in a transformation
%     \item From the description of such Commonalities, transformations to keep the models consistent can be derived
%     \item We discuss how such Commonalities can be defined and how they can be composed to describe consistency between multiple, heterogeneous models (i.e. instances of different metamodels)
% \end{itemize}

% CONTRIBUTIONS
% Our main contributions in this paper are:
% \begin{description}[leftmargin=\parindent]
%     \item[\contributionlabel{contrib:approach}{Commonalities Approach}{C1}:] We propose an approach for making common concepts of different metamodels explicit rather than encoding them implicitly in constraints of a transformation.
%     \item[\contributionlabel{contrib:language}{Commonalities Language}{C2}:] We discuss design options for a language to define \commonalities and outline one language to specify them.
%     %\item[Dynamic Extensibility:] We provide an approach that allows to dynamically extend a declarative specification language.
%     \item[\contributionlabel{contrib:proofofconcept}{Proof-of-Concept}{C3}:] We give an indicator for the applicability of the approach by providing a proof-of-concept implementation and applying it to a scenario with four simple metamodels sharing common concepts. % that have to be kept consistent.
% \end{description}

% BENEFITS
% We expect several benefits from our approach, i.e. specifying \commonalities, in comparison to direct transformation specifications between metamodels.
% First, we claim to achieve \emph{better understandability} of relations between metamodels, because common concepts are made explicit. % rather than encoding them implicitly in transformations.
% Second, the approach \emph{reduces errors} when more than two metamodels are to be kept consistent.
% Transformations usually relate two metamodels, especially because multidirectional relations are hard to express~\cite{stevens2020BidirectionalTransformationLarge-SoSym},
% %\modified{either because the developer does not know about multidirectional relations or due to cognitive limitations~\cite{stevens2020BidirectionalTransformationLarge-SoSym},} %keep instances of two metamodels consistent 
% and therefore have to be combined to a network of transformations to keep instances of more than two metamodels consistent.
% % Following sentence moved from approach section
% Such a network can be regarded as a graph, formed by metamodels as its nodes and transformations as its edges.
% However, such a network can easily raise compatibility problems if there exists more than one path of transformations between two metamodels.
% A hierarchy of \commonality specifications is, by design, not prone to such problems.
% Finally, we \emph{improve reusability} in comparison to a network of transformations, %regarding a network of transformations, 
% because an arbitrary subset of metamodels, between which \commonalities are defined, can be selected to keep their instances consistent.
% In contrast, removing metamodels from a transformation network can easily lead to missing transformation paths between two metamodels.

% % Benefits:
% % \begin{itemize}
% %     \item \emph{Better understandability} of relations, as common concepts are explicitly defined rather than implicitly encoding them in transformations
% %     \item \emph{Improved reusability / partial usability} (regarding networks of binary transformations), because an arbitrary selection of concrete metamodels can be used and kept consistent
% %     \item \emph{High expressiveness}, because no restriction due to predefined sets of statements, as expression can be added dynamically. This can be also improve analyzability of transformations, as additional metadata could be defined for each of the extensions.
% % \end{itemize}

% \end{copiedFrom} % VoSE

