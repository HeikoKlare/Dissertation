\section{Summary}

In this chapter, we have introduced the \commonalities language.
It supports the \commonalities approach for constructing transformation networks as proposed in \autoref{chap:improvement} with a dedicated language.
We have made the design choice of decomposing the specification in that language along \commonalities rather than transformations, which promises to improve comprehensibility of the specification and its conciseness, such that specification effort is even reduced when only few metamodels are to be related.
While we have discussed the relevant elements of that language in more detail and explained them at examples with a concrete textual syntax that we have developed for our prototypical implementation in the \vitruv framework~\cite{vitruvFrameworkGithub}, we refer to the \mappings language of \textcite{kramer2017a} for further details on its operationalization.
The proposed \commonalities language can be seen as an extension of that \mappings language for the purpose of relating metamodels to their concepts rather than relating metamodels with each other.
We close this chapter with the following central insight.

\begin{insight}[Language]
    In addition to the design options given by the \commonalities approach as a whole, a language supporting it additionally needs to define how to decompose the specification.
    This can be done along the \commonalities, such that each \commonality is specified at one place with its manifestations, or along the transformations, such that each \conceptmetamodel and each relation between a \conceptmetamodel and one of its manifestations is defined at one place.
    While depending on the usage context either of them can be beneficial, a decomposition along the \commonalities can only be realized with a dedicated language that derives the \conceptmetamodels and transformations between them from a specification in that language. This approach can especially improve conciseness and comprehensibility.
    Such a language consists of three categories of elements, one for the structure of \conceptmetamodels, one for the manifestations and one for the relations between both.
    %It requires elements for the structural specification of \conceptmetamodels, including \commonalities and their features in terms of attributes and references, elements for describing the manifestations, i.e., the classes representing a \commonality in a \concretemetamodel or another \conceptmetamodel, and elements for the relations between features of a \commonality and its manifestations.
    The operators that define how information is propagated along the relations to keep models consistent across their \commonalities should make their operands, i.e., the features of \commonalities and manifestations, explicit and not internally acquire further information from the models.
    Then, the graph induced by these operands can be used to identify whether the specified consistency relations fulfill the definition of a consistency relation tree, which is likely to be achieved with a \commonalities specification and inherently guarantees compatibility.
\end{insight}

