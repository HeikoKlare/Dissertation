\chapter{Foundations and Notation
    \pgsize{10 p.}
}
\label{chap:foundations}

\section{Modeling}

\subsection{Models and Model Theory}
\label{chap:foundations:modeling:models}
What are models? -> Stachowiak

Definition of metamodels, ingredients:
esp. static/execution semantics~\cite[p. 26]{voelter2013DslEngineering}:
static: set of constraints, type system rules language has to conform in addition to structure regarding abstract syntax
execution: meaning of the program/model when it is executed, for DSL realized by execution engine

Still Völter:
Abstract syntax: data structure holding the semantically relevant information of the model (not layout information), typically a tree or graph
Concrete syntax: notations with which user can express models, e.g. textural, graphical, tabular

Introduce the term "DSL" used in introduction. Refer to XML, as its used in introduction.
Say that we always talk about metamodels, which are the constructs on which languages rely.

\subsection{Model-Driven Software Engineering}
Describe what MDSD is supposed to be


\section{Modeling Formalisms and Frameworks}
\label{chap:foundations:formalisms}

\subsection{Meta-Object Facility}
\label{chap:foundations:formalisms:mof}
Discuss \gls{MOF}, especially \gls{EMOF} and how it is used to model things;
Discuss levels of UML;
Say why MOF is the relevant formalism for us which we base our considerations on.

Modeling Levels M1--M3

\begin{figure}
    \centering
    \begin{tikzpicture}[node distance=5ex and 5ex+0.3*\difftoafiveimage, every node/.style={font=\footnotesize}]

% BEGIN ROW 1
\umlclassvarwidth[]{eme}{[]}{\itshape Element}{%
}{19ex}%21ex

\umlclassvarwidth[]{ene}{[base right=of eme]}{\itshape NamedElement}{%
name : String
}{15ex}

\umlclassvarwidth[]{ete}{[base right=of ene]}{\itshape TypedElement}{%
}{16ex}

\umlsuperclassof{eme.east}{
--
}{ene.west |- eme.east}

\umlsuperclassof{ene.east}{
--
}{ete.west |- ene.east}

% BEGIN ROW 2
\umlclassvarwidth[]{t}{[below=of ete]}{\itshape Type}{%
}{16ex}

\umlclassvarwidth[]{ecf}{[base left=of t]}{\itshape EClassifier}{%
}{15ex}

\umlclassvarwidth[]{eel}{[base left=of ecf]}{EnumerationLiteral}{%
}{19ex}

\umlassociationfromto{
($(ete.south)+(3ex,0)$)
--
node[uml role end, pos=1,above right] {type}
node[uml cardinality end, pos=1,above left] {0..1}
($(t.north)+(3ex,0)$)
}

\draw ($(t.north)+(-3ex,0)$) -- ++(0ex,1.8ex) -- ++(-8ex,0);
\draw[dotted,thick] ($(t.north)+(-11.1ex,1.8ex)$) -- ++(-5ex,0);
\draw[-open triangle 90] ($(t.north)+(-16ex,1.8ex)$) -| (ene.south);

\umlsuperclassof{t.west}{
--
}{ecf.east}

\draw (eel.north) -- ++(0ex,2ex) -- ++(11ex,0);
\draw[dotted,thick] ($(eel.north)+(11.1ex,2ex)$) -- ++(5ex,0);
\draw[-open triangle 90] ($(eel.north)+(16ex,2ex)$) -| (ene.south);

% BEGIN ROW 3
\umlclassvarwidth[]{ee}{[below=of eel]}{Enumeration}{%
}{19ex}

\umlclassvarwidth[]{edt}{[base right=of ee]}{DataType}{%
}{15ex}

\umlclassvarwidth[]{ec}{[base right=of edt]}{Class}{%
abstract : bool
}{16ex}

\umlcomposition{%
($(ee.north)+(3ex,0)$)
-- 
node[uml role end, pos=1,below left] {ownedLiteral}
node[uml cardinality end, pos=1,below right] {0..*}
($(eel.south)+(3ex,0)$)
}

\umlsuperclassof{edt.west}{
--
}{ee.east}

\umlsuperclassof{ecf.south}{
--
}{edt.north}

\draw[-] ($(ec.north)+(-8ex,0)$) -- ++(0,2ex) -- ([yshift=2ex]edt.north);
% \umlsubclassof{$(ec.north)+(-8ex,0)$}{
% -- ++(0,2ex)
% -|
% }{ecf.south}

\umlsuperclassof{edt.east}{
--
}{ec.west |- edt.east}

\umlassociationfromto{
($(ec.north)+(-6ex,0)$)
-- ++(0,3ex)
-- ++(6ex,0)
-- ++(0,-3ex)
node[uml role end, pos=1,above right=-0.2ex and 0.5ex] {/superClass}
node[uml cardinality end, pos=1,above left=-.1ex and 0.5ex] {0..*}
}

% BEGIN ROW 4
\umlclassvarwidth[]{esf}{[below=of ec]}{AggregationKind}{%
none\newline
shared\newline
composite
}{16ex}

\umlclassvarwidth[]{pt}{[base left=of esf]}{PrimitiveType}{%
}{15ex}

\umlclassvarwidth[]{ea}{[base left=of pt]}{\itshape MultiplicityElement}{%
/lower : int\newline
/upper : int\newline
isOrdered : bool
}{19ex}

\umlclassvarwidth[,yshift=-3ex,anchor=north west]{er}{at (pt.west |- ea.south)}{Property}{%
aggregation : AggregationKind\newline
/isComposite : bool
}{26ex}

% \umlcomposition{%
% ($(ec.south)+(-3ex,0)$)
% -- 
% node[pos=1,above right] {eStructuralFeatures}
% node[pos=1,above left] {0..*}
% ++(0,-6ex)
% }

% \umlassociationfromto{
% (ea.north)
% -- ++(0,6ex)
% -|
% ($(edt.south)+(-6ex,0)$)
% node[pos=.95,below left] {/eAttributeType}
% node[pos=.95,below right] {1}
% }

\umlsuperclassof{edt.south}{
--
}{pt.north}

\umlsuperclassof{eme.west}{
-- ++(-2.5ex,0)
|-
}{ea.west}

\umlcomposedin{
($(er.north)+(5ex,0)$)
|-
node[uml role end, pos=0,above left] {ownedAttribute}
node[uml cardinality end, pos=0,above right] {1}
($(ec.south west)+(0,2ex)$)
}

\draw[open triangle 90-] (ea.south) -- (ea.south |- er.west) -- ++(6ex,0);
\draw[dotted,thick] let \p1 = ($(ea.south)+(6.1ex,0)$), \p2 = (er.west) in (\x1,\y2) -- ++(5ex,0);
\draw let \p1 = ($(ea.south)+(11ex,0)$), \p2 = (er.west) in (\x1,\y2) -- (er.west);
% 
\draw[open triangle 90-] let \p1 = ($(ete.east)+(3ex,0)$), \p2 = (er.east) in (ete.east) -- ++(3ex,0) -- (\x1,\y2) -- ++(-6ex,0);
\draw[dotted,thick] let \p1 = ($(ete.east)+(-3.1ex,0)$), \p2 = (er.east) in (\x1,\y2) -- ++(-5ex,0);
\draw let \p1 = ($(ete.east)+(-8ex,0)$), \p2 = (er.east) in (\x1,\y2) -- (er.east);

\end{tikzpicture}
    \caption[Simplified EMOF metamodelling language]{Simplified class diagram showing central metaclasses of the EMOF metamodelling language~\cite[p. 27]{mof} (dotted lines denote indirect inheritance). Adapted from \cite[Fig. 2.2]{kramer2017a}.}
    \label{fig:foundations:emof}
\end{figure}

\subsection{EMF and Ecore}
\label{chap:foundations:formalisms:ecore}
Discuss \gls{EMF}, Ecore and the relation EMOF

Say that we define an own modelling formalism without precise specifications of the elements models consist of. We therefore rely on EMOF as the most general relevant standard for a modeling formalism.

Add graphics of relevant part of EMOF metamodel here, from Max' Diss

\begin{figure}
    \centering
    \newcommand{\classwidth}{15ex}

\begin{tikzpicture}[node distance=6ex and 6ex+0.3*\difftoafiveimage, every node/.style={font=\small}]
% BEGIN ROW 1
\umlclassvarwidth[]{eme}{[]}{\itshape EModelElement}{%
}{\classwidth}

\umlclassvarwidth[]{ene}{[base right=of eme]}{\itshape ENamedElement}{%
name : String
}{\classwidth}

\umlclassvarwidth[]{ete}{[base right=of ene]}{\itshape ETypedElement}{%
lowerBound : int\newline
upperBound : int\newline
ordered : bool
}{\classwidth+2ex}

\umlsuperclassof{eme.east}{
--
}{ene.west |- eme.east}

\umlsuperclassof{ene.east}{
--
}{ete.west |- ene.east}

% BEGIN ROW 2
\umlclassvarwidth[,yshift=2ex]{ecf}{[below=of ene]}{\itshape EClassifier}{%
}{\classwidth}

\umlclassvarwidth[]{eel}{[base left=of ecf]}{EEnumLiteral}{%
}{\classwidth}

\umlassociationfromto{
(ete.south)
|-
node[uml role end,pos=1,above right] {eType}
node[uml cardinality end,pos=1,below right=0.5ex and 0ex] {0..1}
($(ecf.east)+(0,-2ex)$)
}

\umlsuperclassof{ene.south}{
--
}{ecf.north}

\draw ([yshift=-2.5ex]ene.south) -| (eel.north);

% BEGIN ROW 3
\umlclassvarwidth[]{ee}{[below=of eel]}{EEnum}{%
}{\classwidth}

\umlclassvarwidth[]{edt}{[base right=of ee]}{EDataType}{%
}{\classwidth}

\umlclassvarwidth[]{ec}{[base right=of edt]}{EClass}{%
abstract:bool
}{\classwidth+2ex}


\umlcomposition{%
(ee.north)
-- 
node[uml role end,pos=1,below left] {eLiterals}
node[uml cardinality end,pos=1,below right] {0..*}
(eel)
}

\umlsuperclassof{edt.west}{
--
}{ee.east}

\umlsuperclassof{ecf.south}{
--
}{edt.north}

\draw ($(ec.north)+(-8ex,0)$) |- ([yshift=2ex]edt.north);

\umlsuperclassof{edt.east}{
--
}{ec.west |- edt.east}

\umlassociationfromto{
($(ec.north)+(-6ex,0)$)
-- ++(0,3ex)
-- ++(6ex,0)
-- ++(0,-3ex)
node[uml role end,pos=1,above right=-0.2ex and 0.5ex] {eSuperTypes}
node[uml cardinality end,pos=1,above left=-.1ex and 0.5ex] {0..*}
}

% BEGIN ROW 4
\umlclassvarwidth[]{esf}{[below=of ec, yshift=-3ex]}{\itshape EStructuralFeature}{%
abstract : bool
}{\classwidth+2ex}

\umlclassvarwidth[]{er}{[base left=of esf]}{EReference}{%
containment : bool
}{\classwidth+1ex}

\umlclassvarwidth[]{ea}{[base left=of er]}{EAttribute}{%
id : bool
}{\classwidth}

\umlcomposition{%
($(ec.south)+(6ex,0)$)
-- 
node[uml role end,pos=1,above left] {eStructuralFeatures}
node[uml cardinality end,pos=1,above right] {0..*}
++(0,-9ex)
}
\umlsuperclassof{ete.east}{
-- ++(3.5ex,0)
|-
}{esf.east}

\umlassociationfromto{
(er.north)
-- ++(0,4.5ex)
-|
($(ec.south)+(-8.5ex,0)$)
node[uml role end,pos=.85,below left] {/eReferenceType}
node[uml cardinality end,pos=.85,below right] {1}
}

\umlassociationfromto{
(ea.north)
-- ++(0,6ex)
-|
($(edt.south)+(-6ex,0)$)
node[uml role end,pos=.95,below left] {/eAttributeType}
node[uml cardinality end,pos=.95,below right] {1}
}

\umlsuperclassof{esf.south}{
-- ++(0,-3ex)
-|
}{ea.south}

\draw[-] (er.south) -- ++(0,-3ex);


\end{tikzpicture}
    \caption[Simplified Ecore metamodelling language]{Simplified class diagram showing central metaclasses of the Ecore metamodelling language~\cite[p. 97]{steinberg2009emf}. Adapted from \cite[Fig. 2.3]{kramer2017a}.}
    \label{fig:foundations:ecore}
\end{figure}


\subsection{Differences}
\label{chap:foundations:formalisms:differences}

Discuss differences between Ecore and EMOF, especially those relevant for the changes later and how the differences in general manifest in our formal framework

FROM MAX:
Different information and case distinctions are necessary to describe all possible model changes for modelling languages that follow the Essential Meta Object Facility (EMOF) standard or the Ecore variant. 
Both meta-modelling languages and the differences between them are described in \autoref{chap:foundations:modeling:models}.
Only two differences have a major effect on our change modelling language and the specifications language that use them:
\begin{longenumerate}%[1.]
\item In EMOF, properties can be typed using metaclasses or using other data types, but in Ecore these are distinguished as references and attributes.
\item Ecore requires that all elements except for a root element are contained in exactly one container and EMOF only requires that all elements have at most one container~\cite[pp.\ 31-32]{mof}.
\end{longenumerate}
If we only consider these two differences, then Ecore can be seen as a refinement of EMOF, which only adds a more fine-grained distinction of properties and further containment restrictions.
Because of this refinement relation, we will first describe which information is necessary to represent model changes of EMOF-based models and then add further information and distinctions for Ecore-based models.
Finally, we briefly explain how we made all this information available in practice using a change modelling language.


\subsection{Simplification}
\label{chap:foundations:formalisms:simplification}

Discuss how far we simplify the modeling concepts (or maybe do that later in the notation / formalization discussion at the end of foundations or in networks chapter)



\section{Multi-View Modeling}
\label{chap:foundations:multiview}

Refer to multi-paradigm modeling, discuss essential idea of using multiple views, mainly refer to introduction


\subsection{Orthographic Software Modeling}
\label{chap:foundations:multiview:osm}
Focus on views, how to create and manage them. SUM idea


\subsection{The \vitruv Framework}
\label{chap:foundations:multiview:vitruv}
Focus on construction approach for SUM why sticking to projective view idea of OSM. 
Combine models by consistency mechanisms rather than having inherent consistency.

Discuss that \vitruv is the underlying motivation for this thesis. Consistency is needed to build a SUM, especially consistency between multiple models.
\vitruv puts an abstraction layers of views onto the contributions of this thesis and defines a process of using and applying it. 
\todo{Reflect this later in the Commonalities chapter by referring to the idea and saying how composition of Commonalities can even improve the approach.}

That that in this thesis the main connection to \vitruv is the motivation and that implementation work for validation purposes was done with the \vitruv framework.
Additionally, at some points (Commonalities), the usage of developed concepts provides even more benefits when not only used standalone but in combination with further ideas of \vitruv.

For \vitruv, we have a simple formalism defining consistency on which the formalism in this thesis bases.
\todo{Add cite to Journal paper}
However, the formalism in this thesis will be much more fine-grained.



\section{Model Transformations}

\subsection{Bidirectional Transformations}

\todo{Introduce Transformations}

Compare multidirectional transformations with networks of transformations.
Refer for other consistency approaches to related work.

\subsection{Transformation Languages}

\todo{Introduce especially QVT-R in detail to understand compatibility}
\todo{Introduce TGGs shortly}
\todo{Only refer to \gls{VIATRA}, but do not explain it in detail}


\section{Mathematical Notations}
\label{chap:foundations:notations}

\begin{table}
    \centering
    \small
    \renewcommand{\arraystretch}{1.4}%
    \rowcolors{1}{\firstlinecolor}{\secondlinecolor}
    \begin{tabular}{L{10em} L{17em}}
        \toprule
        %\multicolumn{2}{c}{Notations}\\
        $\set{S} = \set{s} = \setted{a, b, \ldots}$ 
            & A set $\set{S}$ or $\set{s}$ of elements\\
        $\tuple{T} = \tuple{t} = \tupled{a, b, \ldots}$ 
            & A tuple $\tuple{T}$ or $\tuple{t}$ of elements\\
        $\sequence{S} = \sequence{s} = \sequenced{a, b, \ldots}$ 
            & A sequence $\sequence{S}$ or $\sequence{s}$ of elements\\
        $\sequenceindex{S}{i}$ 
            & Element at index $i$ of sequence $\sequence{S}$\\
        $\function{Func}$ 
            & A function\\
        \bottomrule
    \end{tabular}
    \caption[Notations for sets, tuples, sequences and functions]{Notations for sets, tuples, sequences and functions}
    \label{tab:foundations:notation}
\end{table}

\mnote{Notations overview}
For the most of our definition, we use standard mathematical notations.
Whenever we derive from that within the thesis, we explicitly denote it and define the used constructed.
We use specific formatting especially for sets, tuples, sequences and functions to ease their distinction.
We introduce this notation in \autoref{tab:foundations:notation}.
Additionally, we define some additional shortcut operators for tuples, which we frequently require throughout the thesis.

\mnote{Sets, tuples and sequences}
We usually denote variables representing sets of any kinds of elements in blackboard bold font $\set{S}{}$ and the definition of a set of elements by putting them in curly brackets, e.g., $\setted{a, b, \dots}$.
Likewise, we denote variables representing tuples of any kinds of elements in gothic font $\tuple{T}{}$ and write elements forming a tuple in angle brackets, e.g., $\tupled{a, b, \dots}$.
Finally, we denote variables representing sequences of any kinds of elements by subsequent square brackets $\sequence{S}$ and the definition of a sequence of elements by putting them into square brackets, e.g., $\sequenced{a, b, \dots}$.
To access an element at index $i$ of a sequence $\sequence{S}$, we write $\sequenceindex{S}{i}$.
We denote the addition of an element $e$ to a sequence $\sequence{S} = \sequenced{s_1, \dots, s_n}$ as:
\begin{align*}
    \sequence{S} + e \equalsperdefinition \sequenced{s_1, \dots, s_n, e}
\end{align*}
Sequences are mathematically equal to tuples, but we write them differently to make explicitly that they represent an order of potentially equal elements, rather than combining different elements of potentially different types in tuples. This is why we explicitly define an access operator for contained elements of sequences.
We derive from the described formatting of sets and tuples in specific situations whenever the focus of the semantics of the variable is not that it is a set or a tuple.
For example, if we consider a relation, which is a set of tuples, we do not denote it in our set syntax, as its semantics is to be a relation and not a set.
If we consider a set of relations, however, we denote it in the described set syntax.
In any case, we ensure that the meaning of the variables stay clear from the context.

\mnote{Tuple operators}
We often use tuples to ensure that the elements can be indexed, although they cannot contain duplications and thus behave as sets if not interested in the order of elements.
Since we need to treat the tuples similar to sets in several situations, especially to describe that a tuple contains an element or that is has a specific relation to another tuple, we define several operators which treat them as sets.
For a tuple $\tuple{t} = \tupled{t_1, \dots, t_n}$, we say that:
\begin{align*}
    &
    t' \in \tuple{t}{} \equivalentperdefinition \exists i \in \setted{1, \dots, n} : t' = t_i
\end{align*}
For two tuples $\tuple{t}_{1}$ and $\tuple{t}_{2}$, we define:
\begin{align*}
    &
    \tuple{t}_{1} \subseteq \tuple{t}{2} \equivalentperdefinition \forall t' \in \tuple{t}_{1} : t' \in \tuple{t}_{2} \\
    &
    \tuple{t}_{1} \cap \tuple{t}_{2} := \setted{t' \mid t' \in \tuple{t}{1} \land t' \in \tuple{t}{2}}
\end{align*}
Note that the intersection of tuples is not a tuple but a set, because we are only interested in getting the elements contained in both tuples but do not need to match their order.

\mnote{Relation concatenation}
In several situation, we define binary relations, which are sets of pairs, i.e., tuples of two elements.
We define the concatenation of two relations to express their transitive relation.
For two binary relations $R_1 = \setted{\tupled{a_{l}, a_{r}}, %\tupled{a_{2,l}, a_{2,r}}, 
\dots}$ and $R_2 = \setted{\tupled{b_{l}, b_{r}}, %\tupled{b_{2,l}, b_{2,r}}, 
\dots}$, we define their concatenation $R_1 \concat R_2$ as:
\begin{align*}
    &
    R_1 \concat R_2 = \setted{\tupled{a,b} \mid \exists z : \tupled{a,z} \in R_1 \land \tupled{z,b} \in R_2}
\end{align*}
This conforms to the composition of relations often denoted as $R_1 ; R_2$.

\mnote{Functions and Composition}
We usually denote function names in small caps, e.g., $\function{Func}$.
For functions, we use the standard notation for their composition. For two functions $\function{F}_{1}$ and $\function{F}_{2}$, we denote their composition for an input $x$ as:
\begin{align*}
    &
    \function{F}_{1} \concatfunction \function{F}_{2}(x) \equalsperdefinition \function{F}_{1}(\function{F}_{2}(x))
\end{align*}




% \begin{copiedFrom}{ICMT}

%%% Define what a transformation networks is
% \section{Assumptions and Terminology from ICMT}
% \label{chap:properties:terminology}

% We shortly clarify our assumptions and introduce a terminology for consistency %and its preservation % based on definitions of models, consistency and consistency preservation 
% that we %later 
% use to explain our classification.
% %Short introduction of transformations (not deeply necessary on ICMT) -- leave out, put to assumptions
% %\subsection{Assumptions}
% %\label{sec:foundations:assumptions}
% %We consider incremental \acp{BX} for preserving consistency between models.
% %Furthermore, we 
% We assume that consistency of more than two types of models is specified using networks of \acp{BX} rather than multidirectional approaches for two reasons:
% First, it is easier to think about binary than about $n$-ary relations~\cite{stevens2017a}.
% Second, a domain expert usually only knows about consistency relations within a subset of all model types used to develop a system, so modularizing transformations is inevitable.
% It was also the result of a Dagstuhl seminar that \enquote{it seems likely that networks of bidirectional transformations suffice for specifying multidirectional transformations}~\cite[p. 7]{cleve2019dagstuhl}.
% Finally, we investigate of a subset of problems that can actually occur, as in a concrete scenario $n$-ary relations may exist that cannot be expressed by sets of binary relations.
% Although we limit our considerations to the assumed scenarios, most of our findings could also be extended to a modularization into smaller $n$-ary relations rather than binary relations.

%Furthermore, we focus on consistency preservation rather than only consistency checking.
%Therefore, we follow a \emph{normative} approach, which means that we always assume that a specification of consistency defines when models are consistent rather than having another, maybe information notion of consistency that has to be met and potentially validated.
%
% \begin{itemize}
%     \item Incremental
%     \item Bidirectional
%     \item Delta-based(!)
%     \item Normative
%     \item Binary modularization, with domain experts
% \end{itemize}
%
% \todoHeiko{Make assumptions explicit! (From Intro) Incremental, distributed development (divide and conquer), ...}
% \begin{itemize}
%     \item Repair, not only checking!
%     \item Incremental transformations for consistency preservation
%     \item Independent development of binary transformations by domain experts -> Necessity to independently develop and combine transformations afterwards
%     \item Normative: We define what is consistent (rather than defining consistency and checking it against a "real" relation)
% \end{itemize}
%
% \paragraph{Terminology}
%
% Failure, (Fault), Mistake, Cause - define?
%
%
%\subsection{Terminology}
%
%\subsection{Models}
% \label{sec:foundations:models}
%\todoHeiko{Nicht Metamodelle, sondern Model Sets sagen? Diese Mengen sind ja eigentlich keine Metamodelle, sondern werden von einem Metamodell induziert. Oder auch model type?}
%
%We provide short definitions for models and the specification of consistency, on which we build our classification to clearly separate the issues that we investigate.
%Provide a short formalization of what is necessary to later define the different issues, e.g., what do these issues mean in terms of a formal definition. \todoHeiko{do this here in ICMT paper or later in SoSym?}
%\subsubsection{Models and Metamodels}
%Our definition of models follows the one used by \textcite{stevens2017a}. 

% \begin{definition}[Model]
% A model $M = \{e_1, e_2, \ldots\}$ is a finite set of not further defined elements, such as objects, attribute and reference values.
% \end{definition}

% The exact representation of the model contents is not relevant for our work, which is why we use this lightweight definition. 
% It allows us to transfer the insights to arbitrary models, such as models that are conform to the \ac{EMOF}~\cite{mof}.

% \begin{definition}[Model Type]
% A model type $\mathcal{M} = \{M_1, M_2, \dots\}$ is the (usually but not necessarily infinite) set of all models $M_1, M_2, \dots$ that are instances of $\mathcal{M}$.
% \end{definition}

% %So for a model type $\mathcal{M}$, a model $M$ is considered an instance of $\mathcal{M}$ iff $M \in \mathcal{M}$.
% In the following, let a model $M_i$ be always an instance of model type $\mathcal{M}_i$.
% This definition constitutes an \emph{extensional description} of models and does not explicitly consider actual instantiation relations between classes and objects, attributes and their values etc., other than containment in the respective model type. 
% We also use the term \emph{metamodel} when referring to an abstract syntax of classes, attributes and associations, as defined in the OCL standard~\cite[A.1]{ocl}. 
% A metamodel constitutes an \emph{intensional description} of models, from which the model type could be derived by enumerating all valid instances, i.e., all models with arbitrary instantiations of classes, their attributes and associations.
%In fact, common definitions of metamodels require abstract specifications of elements and their relations, which can be instantiated. Our definition rather covers a description of models sets, which is appropriate for our case.

%\subsection{Consistency}
%\label{sec:foundations:consistency}

%In addition to models, we define the basic terms \emph{consistency specification}, % expressing the consistency constraints that have to hold, 
%and \emph{consistency preservation specification}: % , expressing the rules to preserve consistency after changes.

% \begin{definition}[Consistency Specification]
% \label{def:consistency_specification}
% A \emph{consistency specification} $\mathit{CS}$ for model types $\mathcal{M}_1, \ldots, \mathcal{M}_n$ is a relation $\mathit{CS} \subset \mathcal{M}_1 \times \ldots \times \mathcal{M}_n$ between models that are consistent. 
% % the tuples of instances of model types $\mathcal{M}_1 \times \ldots \times \mathcal{M}_n$ that are consistent.
% We denote a binary consistency specification for model types $\mathcal{M}_i$ and $\mathcal{M}_j$ as $\mathit{CS}_{i,j}$.
% \end{definition}

% %\todoHeiko{In der Relation zu sein heißt konsistent zu sein. D.h. wenn ich keine Einschränkungen bzgl. Konsistenz mache (alle Modelle sind konsistent zueinadner), enthält die Relation alle möglichen Paare von Modellen}

% %This way of defining consistency between models by enumerating consistent instances is comparable to \cite{stevens2017a}.
% Enumerating consistent instances to define consistency is comparable to \cite{stevens2017a}.
% %Enumerating consistent models is not practically applicable in contrast to constructive approaches that define how to construct consistent models, but it eases expressing properties of consistency. % mathematical statements about consistency.
% If there are no restrictions on when models are consistent, %all models are always consistent by definition, and 
% $\mathit{CS}$ % the consistency specification 
% contains all tuples of models.
% We denote restrictions for models to be in $\mathit{CS}$ as \emph{consistency constraints}.
% It would, in theory, also be possible to define $\mathit{CS}$ on an infinite number of model types. However, for ease of understanding and because of missing practical examples, we decided to fix the number of model types in a consistency specification.

% We primarily consider binary consistency specifications, which are the binary relations that define consistency pairs of models, %, which only specify consistent instances of two model types.
% and also binary specifications for consistency preservation, which are functions that restore consistency between two models after one of them was modified. 
% % We also consider binary specifications for consistency preservation, which concern the modification of one model and the update of it and a model of another type. 
% In the following, we introduce such consistency preservation specifications.
% %To simplify the composition of such functions between more than two model types, 
% Each consistency preservation specification concerns modifications in instances of two model types.
% However, instead of defining such a function on two model types, we define it on an arbitrary number of model types, but restrict modifications to instances of two of them.
% In consequence, a set of binary consistency preservation specifications for an arbitrary number of model types can be defined, whose signatures of input and output are all equal.
% This leads to a rather verbose definition of consistency preservation specifications, but eases the composition of such functions between more than two model types.
% If the function only considered the two involved model types, the composition definition would have to properly consider matching function signatures, whereas our definition allows the composition of all functions with each other.
% %Therefore, we define them on an arbitrary number of model types, but restrict modifications to two of them.
% A consistency preservation specification expects and returns a tuple of pairs, each representing a change by containing an original and a modified model.
% The original models in a tuple are always consistent, but a specification may update the modified models. % may be updated by the specification.

% \begin{definition}[Consistency Preservation Specification]
% \label{def:consistency_preservation_specification}
% % A consistency preservation specification $CPS_{CS}$ is a function for a consistency specification $CS$ that expects a tuple of original models, and one model having a modified state regarding one of the original ones, and maps it to a new set of models:
% % \begin{align*}
% %     CPS_{CS} : (\mathcal{M}_1), \ldots, \mathcal{M}_n, \mathcal{M}_i) \mapsto (\mathcal{M}_1, \times \ldots \times \mathcal{M}_n), i \in \{1, \ldots, n\}
% % \end{align*}
% % For a consistency specification $CS_{i, j}$, a \emph{consistency preservation specification} $CPS_{CS_{i.j}}$ is a function between a tuple of model pairs, each containing one original and one modified model of the same model type, and maps it to a new tuple of model pairs.
% % \begin{align*}
% %     CPS_{CS} : ((\mathcal{M}_1, \mathcal{M}_1), \ldots, (\mathcal{M}_n, \mathcal{M}_n) \mapsto ((\mathcal{M}_1, \mathcal{M}_1), \ldots, (\mathcal{M}_n, \mathcal{M}_n)
% % \end{align*}
% % so that for $M_1, \ldots, M_n$ with $(M_i, M_j) \in CS_{i,j}$ and modified instances $M'_i, M'_j$:
% % \begin{align*}
% %     & \forall M'_k \in \mathcal{M}_k, k \in \{1, \dots, n\}\backslash\{i\}:\\
% %     & \hspace{1em} ((M_1, M''_1), \ldots, (M_n, M''_n)) = CPS_{CS_{i,j}}((M_1, M'_1), \ldots, (M_n, M'_n)) \\
% %     & \hspace{2em} \Rightarrow CS_{i,j}(M''_i, M''_j) \land \forall k \in \{1, \dots, n\}\backslash\{i, j\} : M'_k = M''_k
% %     %
% % \end{align*}
% For a binary consistency specification $\mathit{CS}_{i, j}$, a \emph{consistency preservation specification} $\mathit{CPS}_{\mathit{CS}_{i,j}}$ is a partial function defined if $(M_i, M_j) \in \mathit{CS}_{i,j}$
% that maps a tuple of model pairs, each containing an original model $M_k \in \mathcal{M}_k$ and a modified model $M'_k \in \mathcal{M}_k$, to a new tuple of model pairs:
% \begin{align*}
%     \mathit{CPS}_{\mathit{CS}_{i,j}}: \hspace{0.3em} & \big( (\mathcal{M}_1, \mathcal{M}_1), \ldots, (\mathcal{M}_n, \mathcal{M}_n) \big) \rightarrow \big( (\mathcal{M}_1, \mathcal{M}_1), \ldots, (\mathcal{M}_n, \mathcal{M}_n) \big), \\[0.5em]
%     & \hspace{-3.1em} \big( (M_1, M'_1), \ldots, (M_i, M'_i), \ldots, (M_j, M'_j), \ldots, (M_n, M'_n) \big) \\
%     & \hspace{-3.1em} \mapsto \begin{cases}
%         \big( (M_1, M'_1), \ldots, (M_i, M''_i), \ldots, (M_j, M''_j), \ldots, (M_n, M'_n) \big) & (M_i, M_j) \in \mathit{CS}_{i,j} \\
%         \mathit{undefined} & \mathit{otherwise}
%     \end{cases}
% \end{align*}
% so that
% \begin{align*}
%     (M''_i, M''_j) \in \mathit{CS}_{i,j}
% \end{align*}
% %holds
% %\begin{align*}
%     %$(M''_1, \ldots, M''_n) \in CS$
% %\end{align*}
% % Furthermore, it holds that for models $M_1, \ldots, M_n$ with $(M_1, \ldots, M_n) \in CS$ and a modified instance $M'_i$ of model $M_i$ that 
% % \begin{align*}
% %     CPS_{CS}(M_1, \ldots, M_n, M'_i) \in CS
% % \end{align*}
% \end{definition}
% %\todoHeiko{Normativ klar machen: Wir definieren, was konsistent ist. Wenn wir CPS angeben, die alle auf leere Modelle abbilden, ist das valide}

% \noindent\textit{Remark:} %We assume a normative approach for defining consistency, so it is up to the developer to provide reasonable specifications. 
% A specification that always maps to empty models would be valid regarding our definition.
% It is up to the developer to provide reasonable specifications. 

%\noindent\textit{Remark:} %We assume a normative approach for defining consistency, so it is up to the developer to provide reasonable specifications. 
%A specification that always maps to empty models would be valid regarding our definition.
%It is up to the developer to provide reasonable specifications. 
%We assume a normative approach for defining consistency, so consistent is what a developer specifies as such. In consequence, a consistency preservation specification that always maps to a tuple of empty models would be valid in our definition.

%Usually, incremental model transformations are used to preserve consistency between models. 

% We consider binary consistency preservation specifications, which concern the modification of one model type and the update of that and one other type. 
% To able to concatenate those specifications, we restrict the number of models appropriately.

%This can be expressed by a consistency preservation specification that is specified on two model types, but to be able to easily concatenate binary consistency preservation specifications on different pairs of model types, we use the following definition that simply restricts the number of modified models appropriately.

% \begin{definition}[Binary Consistency Preservation Specification]
% \label{def:binary_consistency_preservation_specification}
% A binary consistency preservation specification $CPS_{CS{i,j}}$ for a binary consistency specification $CS_{i,j}$ is a function according to \autoref{def:consistency_preservation_specification}, which only changes $\mathcal{M}_i$ and $\mathcal{M}_j$, so that 
% %This means that for models $M_1, \ldots, M_n$ with $(M_1, \ldots, M_n) \in CS$ and modified instances $M'_1, \ldots, M'_n$
% %\begin{align*}
%     %((M_1, M''_1), \dots, (M_n, M''_n)) := CPS_{CS, \mathcal{M}_{i,j}}(M_1, \dots, M_n, M'_i)
% %\end{align*}
% %holds
% %\begin{align*}
%     $CS_{i,j}(M''_i, M''_j) \land \forall k \in \{1, \dots, n\}\backslash\{i, j\} : M'_k = M''_k$
% %\end{align*}
% \end{definition}

% We are interested in consistency preservation specifications that can be executed in arbitrary order, so that they finally terminate in a consistent state regarding all consistency specifications, comparable to a fixed-point iteration.
% Therefore, it is essential for all specifications to be hippocratic~\cite{stevens2010sosym}, so that no changes are performed when models are already consistent.
% Let $\mathcal{CPS}$ be a set of preservation specifications %\mathit{CPS}_1, \dots, \mathit{CPS}_k$ 
% for consistency specifications $\mathcal{CS}$. % = \{CS_1, \dots, CS_l\}$, 
% We denote the set of consistent model tuples regarding $\mathcal{CS}$ as $\mathfrak{M}_{\mathcal{CS}} = \{(M_1, \dots, M_n) \mid %\forall i, j, 0 \leq i,j \leq n : 
% \forall \mathit{CS}_{i,j} \in \mathcal{CS} : (M_i, M_j) \in \mathit{CS}_{i, j}\}$.
% We want to achieve that:
% \begin{align*}
%     & \forall (M_1, \dots, M_n) \in \mathfrak{M}_{\mathcal{CS}} : %\{(M_1, \dots, M_n) \mid \forall CS_{i,j} \in \mathcal{CS} \Rightarrow (M_i, M_j) \in CS_{i, j}\} : \\ %0 \leq i,j \leq n : \exists CS_{i,j} \in \mathcal{CS} \Rightarrow (M_i, M_j) \in CS_{i, j}\} : \\
%     % & \hspace{1em} 
%     \forall M'_1 \in \mathcal{M}_1, \dots, M'_n \in \mathcal{M}_n : \exists \mathit{CPS}_1, \dots, \mathit{CPS}_k \in \mathcal{CPS} : \\
%     %& \forall M_1, \dots, M_n \mid \big( \forall CS_{i,j} \in \mathcal{CS} : (M_i, M_j) \in CS_{i, j} \big) : \\ % (i,j) \mid 1 \leq i, j \leq n
%     %& \exists p \in \mathbb{N}: (CPS_1 \circ \dots \circ CPS_k)^p \big( (M_1, M'_1), \dots, (M_n, M'_n) \big) = \big( (M_1, M''_1), \dots, (M_n, M''_n) \big) \\
%     & \hspace{1em} \mathit{CPS}_1 \circ \dots \circ \mathit{CPS}_k \big( (M_1, M'_1), \dots, (M_n, M'_n) \big) = \big( (M_1, M''_1), \dots, (M_n, M''_n) \big) \\
%     & \hspace{2em} \land \forall \mathit{CS}_{i,j} \in \mathcal{CS} : (M''_i, M''_j) \in \mathit{CS}_{i, j} %\forall (i,j) \mid 1 \leq i, j \leq n : CS_{i, j}(M''_i, M''_j)
% \end{align*}

% This means that there is always a sequence of consistency preservation specification applications, potentially with multiple applications of the same specification, that ensures that the modified models in all tuples are consistent after applying it.

%\todoHeiko{Zunächst mal die Konkatenierung erklären. Wir nehmen an, dass in einer korrekten Spezifikation eine Konkatenation existiert, die für eine beliebige Änderung wieder ein konsistentes Modell ausspuckt. Dafür ist die Ausführungsreihenfolge der Spezifikationen egal. Da man in der Praxis nicht nur wissen muss, dass die Modelle nach einer ausreichend langen Ausführung der Spezifikationen konsistent sind, sondern auch terminieren muss, wird die hippocraticness Eigenschaft \cite{stevens2007a} vorausgesetzt, nach der Transformationen nichts tun, wenn die Modelle bereits konsistent sind. Cf. Fixpunktiteration} 

% Declarative transformation languages are usually well suited to define consistency specifications according to \autoref{def:consistency_specification}, 
% from which a consistency preservation specification is %, in the best case automatically, 
% derived. 
% Imperative transformation languages can be used to define consistency preservation specifications according to \autoref{def:consistency_preservation_specification}. 

% \end{copiedFrom} % ICMT

