\chapter{Verifiability}
\label{chap:appendix:verifiability}

\mnote{Repositories and reproducibility}
Along with the depiction of our approaches and their evaluation, we have referred to their realization artifacts, which are available at GitHub.
In particular, most of the evaluation results can be reproduced with the case studies provided in those repositories.
Since these repositories evolve, we have annotated the date of the repository state that we have last used for the evaluation in the bibliography.
In addition, we provide a reproduction package~\cite{klare2021diss-reproduction} that contains all artifacts in that state together with an according environment that eases the reproduction of the results to improve long-term reproducibility.

\mnote{Relevant artifacts}
We have developed four kinds of artifacts.
They comprise a realization of the approach for validating compatibility of transformations, a simulator for transformation networks, an evaluation of the categorization of errors in transformation networks and approaches to resolve them, and finally the \commonalities language and a comprehensive case study for its evaluation.

\mnote{Decomposition approach}
We have realized a decomposition approach for the validation of compatibility (see~\autoref{chap:compatibility:informal:prevention})~\cite{decompositionGithub}.
The implementation validates compatibility of given \qvtr transformations defined with the Eclipse implementation \gls{QVTd}~\cite{EclipseQVTd}.
The case study presented in \autoref{chap:correctness_evaluation:compatibility} is implemented in terms of test cases in the according repository.

\mnote{Transformation network simulator}
We have implemented a simulator for transformation networks, in which different scenarios of transformations and models to which they are applied can be executed step by step (see~\autoref{chap:orchestration:conservative})~\cite{gleitze2020OrchestrationSimulator}.
The implementation provides a predefined set of transformation networks and model states to apply them to, which can be extended by further scenarios.
It is realized as a web-based visualization of the network execution process.

\mnote{Catogorization, synchronization and \commonalities}
For the case studies on error categorization and synchronization of transformations (see~\autoref{chap:correctness_evaluation:categorization}), the prototypical implementation of the \commonalities language (see~\autoref{chap:language}), and its evaluation (see~\autoref{chap:commonalities_evaluation}), we have employed and extended the \vitruv framework~\cite{klare2021Vitruv-JSS}.
The \commonalities language has been realized as an additional language in the \vitruv framework repository~\cite{vitruvFrameworkGithub}, next to the existing \reactions and \mappings languages.
The case studies have been realized based on transformations and test scenarios implemented in the case study repository for component-based system development~\cite{vitruvCBSEGithub}.

\mnote{Artifact versions}
The contributions and case studies for the \vitruv framework have last been validated with release version $2.0.1$ of the framework and release version $0.2.0$ of the case studies repository.
Results may also be reproducible with later versions, but the framework behavior may change and the case studies may be developed further, such that the absolute result values will differ although the same conclusions should be derivable from them.
To support the reproduction of the results with the \vitruv framework presented in this thesis, the reproduction package~\cite{klare2021diss-reproduction} contains the depicted artifact versions and a Docker-based execution environment to ease their setup.

\mnote{Reproduction of processes}
Several evaluation results, especially regarding the categorization and resolution of errors in \autoref{chap:correctness_evaluation:categorization}, depend on the execution of a process.
This process starts with independently developed transformations, combines them to a network, and fixes faults revealed by occurring execution failures.
The states of this process during the development have been tagged in the case studies repository~\cite{vitruvCBSEGithub}, but they may be difficult to reproduce in detail, as they depended on the framework at that time.
However, the mentioned versions of the artifacts and especially the reproduction package contain the final state after performing the depicted process, in which all faults have been fixed.

