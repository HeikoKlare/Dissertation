\chapter{Compatibility Proofs}
\label{chap:appendix:compatibility_proofs}

\mnote{Compatibility of consistency relation trees}
In \autoref{chap:compatibility:formal_approach}, we have given \autoref{theorem:treecompatibility} for inherent compatibility of consistency relation trees as defined in \autoref{def:relationtree}.
Due to the complexity of the according proof, we have separated it into this appendix.

\mnote{Auxiliary lemma}
To prove the statement of \autoref{theorem:treecompatibility}, we first present a lemma that shows that in a consistency relation tree one can always find an order of the relations such that the classes at the right side of a relation do not overlap with the classes at the left side of a relation that preceded in the order, i.e., there is no cycle between classes in the relations.

\begin{lemma}[Consistency Relation Tree Unique Paths] \label{lemma:treehassequence}
    Let $\consistencyrelationset{CR} = \setted{\consistencyrelation{CR}{1}, \consistencyrelation{CR}{1}^T, \dots, \consistencyrelation{CR}{k}, \consistencyrelation{CR}{k}}^T$ be a symmetric, connected set of consistency relations.
    $\consistencyrelationset{CR}$ is a consistency relation tree if, and only if, for each $\consistencyrelation{CR}{} \in \consistencyrelationset{CR}$ exists a sequence $\sequence{\consistencyrelationset{CR}'} = \sequenced{\consistencyrelation{CR}{1}', \dots, \consistencyrelation{CR}{k}'}$ with $\consistencyrelation{CR}{1}' = \consistencyrelation{CR}{}$ that contains for each $i$ either $\consistencyrelation{CR}{i}$ or $\consistencyrelation{CR}{i}^T$, i.e.,
    \begin{align*}
        &
        \forall i \in \setted{1, \dots, k} :
        \bigl( \consistencyrelation{CR}{i} \in \sequence{\consistencyrelationset{CR}'}
        \land \consistencyrelation{CR}{i}^T \not\in \sequence{\consistencyrelationset{CR}'}
        \bigl)\\
        & \formulaskip 
        \lor \bigl(\consistencyrelation{CR}{i}^T \in \sequence{\consistencyrelationset{CR}'}
        \land \consistencyrelation{CR}{i} \not\in \sequence{\consistencyrelationset{CR}'}
        \bigl)
    \end{align*}
    such that:
    \parameterizeformat{
    \begin{align*}
        &
        \forall s \in \setted{1, \dots, k-1} : \forall t \in \setted{s+1, \dots, k} :
        #2
        \classtuple{C}{r,\consistencyrelation{CR}{s}'} \cap \classtuple{C}{r,\consistencyrelation{CR}{t}'} = \emptyset 
        \land
        \classtuple{C}{l,\consistencyrelation{CR}{s}'} \cap 
        \classtuple{C}{r,\consistencyrelation{CR}{t}'} = \emptyset
    \end{align*}
    }{}{\\ & \formulaskip}%
\end{lemma}

\begin{proof}
    We start with the forward direction.
    Given a consistency relation tree $\consistencyrelationset{CR}$, we show that there exists a sequence according to the requirements in \autoref{lemma:treehassequence} by constructing such a sequence $\sequence{\consistencyrelationset{CR}'} = \sequenced{\consistencyrelation{CR}{1}', \dots, \consistencyrelation{CR}{k}'}$ for any $\consistencyrelation{CR}{} \in \consistencyrelationset{CR}$.
    We begin with any $\consistencyrelation{CR}{1}' = \consistencyrelation{CR}{} \in \consistencyrelationset{CR}$ and inductively add further relations to that sequence.
    We take any consistency relation $\consistencyrelation{CR}{s} = \consistencyrelation{CR}{s,1} \concat \dots \concat \consistencyrelation{CR}{s,m} \in \transitiveclosure{\consistencyrelationset{CR}}$ with $\classtuple{C}{l,\consistencyrelation{CR}{s,1}} \subseteq \classtuple{C}{r,\consistencyrelation{CR}{}}$. 
    Such a sequence must exist because of $\consistencyrelationset{CR}$ being connected.
    Then we add all $\consistencyrelation{CR}{s,1}, \dots, \consistencyrelation{CR}{s,m}$ to the sequence, such that we have $\sequenced{\consistencyrelation{CR}{}, \consistencyrelation{CR}{s,1}, \dots, \consistencyrelation{CR}{s,m}}$, which fulfills both requirements to that sequence in \autoref{lemma:treehassequence} by definition.
    The addition of further consistency relations can be applied inductively.
    We take any other consistency relation $\consistencyrelation{CR}{t} = \consistencyrelation{CR}{t,1} \concat \dots \concat \consistencyrelation{CR}{t,n} \in \transitiveclosure{\consistencyrelationset{CR}}$ such that:
    \begin{align*}
        &
        \exists \consistencyrelation{CR}{}' \in \setted{\consistencyrelation{CR}{}, \consistencyrelation{CR}{s,1}, \dots, \consistencyrelation{CR}{s,m}} :
        \classtuple{C}{l,\consistencyrelation{CR}{t,1}} \subseteq \classtuple{C}{r,\consistencyrelation{CR}{}'}\\
        & \formulaskip
        \land
        \consistencyrelation{CR}{t,1}, \consistencyrelation{CR}{t,1}^T \not\in \setted{\consistencyrelation{CR}{}, \consistencyrelation{CR}{s,1}, \dots, \consistencyrelation{CR}{s,m}}
    \end{align*}
    In other words, we take any concatenation in the transitive closure of $\consistencyrelationset{CR}$ that starts with a relation with a left class tuple that is contained in a right class tuple of a relation already added to the sequence.
    Again, such a sequence must exist because of $\consistencyrelationset{CR}$ being connected and, again, we add all $\consistencyrelation{CR}{t,1}, \dots, \consistencyrelation{CR}{t,n}$ to the sequence.
    Per construction, for each $\consistencyrelation{CR}{}'$ in the sequence, a non-empty concatenation of relations within the sequence $\consistencyrelation{CR}{} \concat \dots \concat \consistencyrelation{CR}{}'$ exists, because relations were added in a way so that such a concatenation always exists. Since all relations in the sequence are contained in $\consistencyrelationset{CR}$, such a concatenation was also contained in $\transitiveclosure{\consistencyrelationset{CR}}$.
    First (1.), we show that the sequence still contains no duplicate elements, i.e., that none of the $\consistencyrelation{CR}{t,i}$ or $\consistencyrelation{CR}{t,i}^T$ is already contained in the sequence $\sequenced{\consistencyrelation{CR}{}, \consistencyrelation{CR}{s,1}, \dots, \consistencyrelation{CR}{s,m}}$. 
    Second (2. ,3.), we show that both further conditions for the sequence defined in \autoref{lemma:treehassequence} are still fulfilled for the sequence $\sequenced{\consistencyrelation{CR}{}, \consistencyrelation{CR}{s,1}, \dots, \consistencyrelation{CR}{s,m}, \consistencyrelation{CR}{t,1}, \dots, \consistencyrelation{CR}{t,n}}$.
    \begin{longenumerate}
        \item
    Let us assume that $\sequenced{\consistencyrelation{CR}{}, \consistencyrelation{CR}{s,1}, \dots, \consistencyrelation{CR}{s,m}}$ already contained one of the $\consistencyrelation{CR}{t,i}$ or $\consistencyrelation{CR}{t,i}^T$. If the sequence contained $\consistencyrelation{CR}{t,i}$, there is a non-empty concatenation $\consistencyrelation{CR}{} \concat \dots \concat \consistencyrelation{CR}{t,i}$ of relations in $\sequenced{\consistencyrelation{CR}{}, \consistencyrelation{CR}{s,1}, \dots, \consistencyrelation{CR}{s,m}}$. In addition, the concatenation $\consistencyrelation{CR}{} \concat \dots \concat \consistencyrelation{CR}{t,1} \concat \dots \concat \consistencyrelation{CR}{t,i}$ is non-empty by selection in our construction approach.
    Since $\consistencyrelation{CR}{t,1} \not\in \setted{\consistencyrelation{CR}{}, \consistencyrelation{CR}{s,1}, \dots, \consistencyrelation{CR}{s,m}}$ by construction, these two concatenations are not identical but relate the same class tuples, i.e., they contradict the definition of a consistency relation tree.
    If $\consistencyrelation{CR}{t,i}^T$ was contained in the sequence $\sequenced{\consistencyrelation{CR}{}, \consistencyrelation{CR}{s,1} \concat \dots \concat \consistencyrelation{CR}{s,m}}$, there is a non-empty concatenation $\consistencyrelation{CR}{} \concat \dots \concat \consistencyrelation{CR}{w} \concat \consistencyrelation{CR}{t,i}^T$ of relations in $\sequenced{\consistencyrelation{CR}{}, \consistencyrelation{CR}{s,1}, \dots, \consistencyrelation{CR}{s,m}}$, and, like before, the concatenation $\consistencyrelation{CR}{} \concat \dots \concat \consistencyrelation{CR}{t,1}, \dots, \consistencyrelation{CR}{t,i}$ is non-empty by construction.
    Due to $\classtuple{C}{r,\consistencyrelation{CR}{w}} \cap \classtuple{C}{l,\consistencyrelation{CR}{t,i}^T} \neq \emptyset$ (with $\classtuple{C}{l,\consistencyrelation{CR}{t,i}^T} = \classtuple{C}{r,\consistencyrelation{CR}{t,i}}$) and $\consistencyrelation{CR}{t,1}^T \not\in \setted{\consistencyrelation{CR}{}, \consistencyrelation{CR}{s,1}, \dots, \consistencyrelation{CR}{s,m}}$ by construction, the two concatenations $\consistencyrelation{CR}{} \concat \dots \concat \consistencyrelation{CR}{w}$ and $\consistencyrelation{CR}{} \concat \dots \concat \consistencyrelation{CR}{t,1} \concat \dots \concat \consistencyrelation{CR}{t,i}$ have an overlap in both their left and right class tuples, i.e., they contradict the definition of a consistency relation tree.
    In consequence, $\sequenced{\consistencyrelation{CR}{}, \consistencyrelation{CR}{s,1}, \dots, \consistencyrelation{CR}{s,m}}$ cannot have contained any $\consistencyrelation{CR}{t,i}$ or $\consistencyrelation{CR}{t,i}^T$ before.
        \item 
    Let us assume that $\sequenced{\consistencyrelation{CR}{}, \consistencyrelation{CR}{s,1}, \dots, \consistencyrelation{CR}{s,m}, \consistencyrelation{CR}{t,1}, \dots, \consistencyrelation{CR}{t,n}}$ contains any $\consistencyrelation{CR}{x}'$ and $\consistencyrelation{CR}{y}'$ such that $\classtuple{C}{r,\consistencyrelation{CR}{x}'} \cap \classtuple{C}{r,\consistencyrelation{CR}{y}'} \neq \emptyset$.
    As discussed before, for each of these relations exists a non-empty concatenation of relations $\consistencyrelation{CR}{} \concat \dots \concat \consistencyrelation{CR}{x}'$ and $\consistencyrelation{CR}{} \concat \dots \concat \consistencyrelation{CR}{y}'$ in the sequence $\sequenced{\consistencyrelation{CR}{}, \consistencyrelation{CR}{s,1}, \dots, \consistencyrelation{CR}{s,m}, \consistencyrelation{CR}{t,1}, \dots, \consistencyrelation{CR}{t,n}}$ that is contained in $\transitiveclosure{\consistencyrelationset{CR}}$.
    This contradicts the definition of a consistency relation tree, so there cannot be two such relations with overlapping right class tuple.
        \item
    Let us assume that $\sequenced{\consistencyrelation{CR}{}, \consistencyrelation{CR}{s,1}, \dots, \consistencyrelation{CR}{s,m}, \consistencyrelation{CR}{t,1}, \dots, \consistencyrelation{CR}{t,n}}$ contains any $\consistencyrelation{CR}{x}'$ and $\consistencyrelation{CR}{y}'$ with $x < y$ such that $\classtuple{C}{l,\consistencyrelation{CR}{x}'} \cap \classtuple{C}{r,\consistencyrelation{CR}{y}'} \neq \emptyset$.
    Again per construction, there must be a non-empty concatenation $\consistencyrelation{CR}{} \concat \dots \concat \consistencyrelation{CR}{w}' \concat \consistencyrelation{CR}{x}'$ with $w < x$. Since $\classtuple{C}{l,\consistencyrelation{CR}{x}'} \subseteq \classtuple{C}{r,\consistencyrelation{CR}{w}'}$ per definition, it holds that
    $\classtuple{C}{r,\consistencyrelation{CR}{w}'} \cap \classtuple{C}{r,\consistencyrelation{CR}{y}'} \neq \emptyset$.
    We have already shown in (2.) that this contradicts the definition of a consistency relation tree.
    \end{longenumerate}

    The previous strategy for adding relations to the sequence can be continued inductively by adding relations of the transitive closure of $\consistencyrelationset{CR}$ if their relations were not yet added to the sequence.
    This process can be continued until finally all relations in $\consistencyrelationset{CR}$ are added to the sequence.
    Inductively applying the same arguments as before, the final sequence still fulfills all requirements for the sequence in \autoref{lemma:treehassequence}.
        
    We proceed with the reverse direction, i.e., given a sequence according to the requirements in \autoref{lemma:treehassequence} for all $\consistencyrelation{CR}{} \in \consistencyrelationset{CR}$, we show that the set of consistency relations fulfills the definition of a consistency relation tree.
    Let us assume that the tree definition was not fulfilled, i.e., that there were two consistency relations $\consistencyrelation{CR}{s} = \consistencyrelation{CR}{s,1} \concat \dots \concat \consistencyrelation{CR}{s,m} \in \transitiveclosure{\consistencyrelationset{CR}}$ and $\consistencyrelation{CR}{t} = \consistencyrelation{CR}{t,1} \concat \dots \concat \consistencyrelation{CR}{t,n} \in \transitiveclosure{\consistencyrelationset{CR}}$ such that $\classtuple{C}{l,\consistencyrelation{CR}{s}} \cap \classtuple{C}{l,\consistencyrelation{CR}{t}} \neq \emptyset$ and $\classtuple{C}{r,\consistencyrelation{CR}{s}} \cap \classtuple{C}{r,\consistencyrelation{CR}{t}} \neq \emptyset$.
    Without loss of generality, we assume that $\consistencyrelation{CR}{s,m} \neq \consistencyrelation{CR}{t,n}$, because if these last relations are the same, the previous relations $\consistencyrelation{CR}{s,m-1}$ and $\consistencyrelation{CR}{t,n-1}$ must have an overlap in the classes at the right side and thus we could instead consider the sequences without those last relations and still fulfill the defined requirements.
    Any sequence according to \autoref{lemma:treehassequence} containing both $\consistencyrelation{CR}{s,m}$ and $\consistencyrelation{CR}{t,n}$ would contradict the assumption, because $\classtuple{C}{r,\consistencyrelation{CR}{s,m}} \cap \classtuple{C}{r,\consistencyrelation{CR}{t,n}} \neq \emptyset$ in contradiction to the assumptions regarding the sequence.
    Thus, the sequence has to contain either $\consistencyrelation{CR}{s,m}^T$ or $\consistencyrelation{CR}{t,n}^T$.
    Let us assume that the sequence contains $\consistencyrelation{CR}{s,m}^T$.
    Then the sequence cannot contain $\consistencyrelation{CR}{s,m-1}$, because $\classtuple{C}{r,\consistencyrelation{CR}{s,m}^T} \cap \classtuple{C}{r,\consistencyrelation{CR}{s,m-1}} \neq \emptyset$, which, again, would contradict the assumptions regarding the sequence.
    This argument can be inductively applied to all $\consistencyrelation{CR}{s,i}$, such that the sequence has to contain all $\consistencyrelation{CR}{s,i}^T$.
    Since the sequence contains $\consistencyrelation{CR}{s,1}^T$, it must contain $\consistencyrelation{CR}{t,1}$, because $\classtuple{C}{r,\consistencyrelation{CR}{s,1}^T} \cap \classtuple{C}{r,\consistencyrelation{CR}{t,1}^T} \neq \emptyset$.
    In consequence of $\consistencyrelation{CR}{t,1}$ being contained in the sequence, all $\consistencyrelation{CR}{t,i}$ have to be contained as well for the same reasons as before.
    So we have these conditions, which introduce a cycle in the overlaps of the class tuples of the relations within the sequence:
    \begin{align*}
        &
        \classtuple{C}{l,\consistencyrelation{CR}{s,i-1}^T} \cap \classtuple{C}{r,\consistencyrelation{CR}{s,i}^T} \neq \emptyset
        \land
        \classtuple{C}{l,\consistencyrelation{CR}{t,1}} \cap \classtuple{C}{r,\consistencyrelation{CR}{s,1}^T} \neq \emptyset\\
        & 
        \land 
        \classtuple{C}{l,\consistencyrelation{CR}{t,i}} \cap \classtuple{C}{r,\consistencyrelation{CR}{t,i-1}} \neq \emptyset
        \land
        \classtuple{C}{l,\consistencyrelation{CR}{s,m}^T} \cap \classtuple{C}{r,\consistencyrelation{CR}{t,n}} \neq \emptyset
    \end{align*}
    Because of that cycle in the overlap of class tuples, there is no order of these relations $\sequenced{\consistencyrelation{CR}{1}'', \dots, \consistencyrelation{CR}{m+n}''}$ such that for all of them it holds that $\classtuple{C}{l,\consistencyrelation{CR}{u}''} \cap \classtuple{C}{r,\consistencyrelation{CR}{w}''} \neq \emptyset\; (u < w)$, which contradicts the assumptions regarding the sequence in \autoref{lemma:treehassequence}.
    The analogous argument holds when we assume that the sequence contains $\consistencyrelation{CR}{t,n}^T$ instead of $\consistencyrelation{CR}{s,m}^T$.
    In consequence, there cannot be two such concatenations $\consistencyrelation{CR}{s}$ and $\consistencyrelation{CR}{t}$ without breaking the assumptions for the sequence in \autoref{lemma:treehassequence}.
\end{proof}

\mnote{Proof of tree compatibility theorem}
The previous lemma shows that the definition of consistency relation trees in \autoref{def:relationtree} is equivalent to the possibility to find sequences of the relations that do not contain cycles in the related class tuples.
We can now show that a consistency relation tree is always compatible by a constructive proof that requires the equivalent definition from \autoref{lemma:treehassequence}.
We have defined this statement in \autoref{theorem:treecompatibility} and now provide the according proof.

\newcommand{\smallerparagraph}[1]{\vspace{-1em}\paragraph{#1}}
\begin{proof}
    We prove the statement by constructing a tuple of models for each condition element in the left condition of each consistency relation.
    This model tuple contains the condition element and is consistent, i.e., it fulfills the compatibility definition.
    The basic idea is that because $\consistencyrelationset{CR}$ is a consistency relation tree, we can simply add necessary elements to get a model tuple that is consistent to all consistency relations by following an order of relations according to \autoref{lemma:treehassequence}.
    Thus, we explain an induction for constructing such a model tuple, which is also exemplified for a simple scenario in \autoref{fig:compatibility:tree_construction_example}, which is based on the relations in the consistency relation tree in \autoref{fig:compatibility:tree_example}.
    
    \smallerparagraph{Base Case:}
    Take any $\consistencyrelation{CR}{} \in \consistencyrelationset{CR}$ and any condition element of the left-side condition $\conditionelement{c}{l} = \tupled{\object{o}{l,1}, \dots, \object{o}{l,m}} \in \condition{c}{l, \consistencyrelation{CR}{}}$.
    Select any $\conditionelement{c}{r} = \tupled{\object{o}{r,1}, \dots, \object{o}{r,n}} \in \condition{c}{r, \consistencyrelation{CR}{}}$, such that $\conditionelement{c}{l}$ and $\conditionelement{c}{r}$ constitute a consistency relation pair $\tupled{\conditionelement{c}{l}, \conditionelement{c}{r,}} \in \consistencyrelation{CR}{}$.
    We now construct the model tuple $\modeltuple{m} = \setted{\object{o}{l,1}, \dots, \object{o}{l,m}, \object{o}{r,1}, \dots, \object{o}{r,n}}$.
    In consequence, we have a minimal model tuple $\modeltuple{m}$, such that $\modeltuple{m} \containsmath \conditionelement{c}{l}$ and $\modeltuple{m} \consistenttomath \consistencyrelation{CR}{}$.
    Additionally, $\modeltuple{m}$ is consistent to $\consistencyrelation{CR}{}^T$ due to symmetry of $\consistencyrelation{CR}{}$ and $\consistencyrelation{CR}{}^T$: It is $\conditionelement{c}{r} \in \condition{c}{l,\consistencyrelation{CR}{}^T}$ and $\tupled{\conditionelement{c}{r}, \conditionelement{c}{l}} \in \consistencyrelation{CR}{}^T$ and no other condition element of $\condition{c}{l,\consistencyrelation{CR}{}^T}$ is contained in $\modeltuple{m}$ by construction, thus $\modeltuple{m}$ is consistent to $\consistencyrelation{CR}{}^T$.
    In consequence, we know for all $\consistencyrelation{CR}{} \in \consistencyrelationset{CR}$ that $\setted{\consistencyrelation{CR}{}, \consistencyrelation{CR}{}^T}$ is compatible. 
    Considering the example in \autoref{fig:compatibility:tree_construction_example}, for the selection of any person as a condition element in $\condition{c}{l,\consistencyrelation{CR}{1}}$ (1), we select a resident in $\condition{c}{r,\consistencyrelation{CR}{1}}$ with the same $\mathvariable{name}$ (2), such that the elements are consistent to $\consistencyrelation{CR}{1}$.
    
    \smallerparagraph{Induction Assumption:}
    We know from \autoref{lemma:treehassequence} that for the relations in $\consistencyrelationset{CR}$ there is a sequence $\sequenced{\consistencyrelation{CR}{1}, \dots, \consistencyrelation{CR}{k}}$ with $\consistencyrelation{CR}{1} = \consistencyrelation{CR}{}$ such that:
    \parameterizeformat{
    \begin{align*}
        &
        \forall s \in \setted{1, \dots, k-1} : \forall t \in \setted{s+1, \dots, k} :
        #2
        \classtuple{C}{r,\consistencyrelation{CR}{s}'} \cap \classtuple{C}{r,\consistencyrelation{CR}{t}'} = \emptyset 
        \land
        \classtuple{C}{l,\consistencyrelation{CR}{s}'} \cap 
        \classtuple{C}{r,\consistencyrelation{CR}{t}'} = \emptyset
    \end{align*}
    }{}{\\ & \formulaskip}%
    Considering the example in \autoref{fig:compatibility:tree_construction_example}, such a sequence would be $\sequenced{\consistencyrelation{CR}{1}, \consistencyrelation{CR}{2}}$, because the elements in the right condition of $\consistencyrelation{CR}{2}$ are not represented in the left condition of $\consistencyrelation{CR}{1}$.
    We assume that for some $i < k$ we know that $\setted{\consistencyrelation{CR}{1}, \consistencyrelation{CR}{1}^T, \dots, \consistencyrelation{CR}{i}, \consistencyrelation{CR}{i}^T}$ is compatible.
    Then for every $\conditionelement{c}{l} \in \condition{C}{l,\consistencyrelation{CR}{}}$ we can find a model tuple $\modeltuple{m}$ that contains $\conditionelement{c}{l}$ and that is consistent to $\setted{\consistencyrelation{CR}{1}, \consistencyrelation{CR}{1}^T, \dots, \consistencyrelation{CR}{i}, \consistencyrelation{CR}{i}^T}$.
    We can especially create a minimal model by our construction for the base case and the following induction step.
    
    \smallerparagraph{Induction Step:}
    We consider $\consistencyrelation{CR}{i+1}$.
    There is at most one condition element $\conditionelement{c}{l} \in \condition{c}{l, \consistencyrelation{CR}{i+1}}$ with $\modeltuple{m} \containsmath \conditionelement{c}{l}$.
    If there were at least two condition elements $\conditionelement{c}{l}, \conditionelement{c}{l}' \in \condition{c}{l,\consistencyrelation{CR}{i+1}}$ that are both contained in $\modeltuple{m}$, then by construction there is a consistency relation $\consistencyrelation{CR}{s}\; (s < i+1)$ with $\conditionelement{c}{l},\conditionelement{c}{l}' \in \condition{c}{r,\consistencyrelation{CR}{j}}$.
    Let us assume there were two consistency relations $\consistencyrelation{CR}{s}, \consistencyrelation{CR}{t}$, each containing one of the condition elements in the right condition, then there would be non-empty concatenations $\consistencyrelation{CR}{} \concat \dots \concat \consistencyrelation{CR}{s}$ and $\consistencyrelation{CR}{}' \concat \dots \concat \consistencyrelation{CR}{t}$ with $\classtuple{C}{l,\consistencyrelation{CR}{}} \cap \classtuple{C}{l,\consistencyrelation{CR}{}'} \neq \emptyset$, because we started the construction with elements from the left condition of $\consistencyrelation{CR}{}$ and so every element is contained in the models because of a relation to those elements, and with $\classtuple{C}{r,\consistencyrelation{CR}{s}} \cap \classtuple{C}{r,\consistencyrelation{CR}{t}} \neq \emptyset$, because both condition elements $\conditionelement{c}{l}$ and $\conditionelement{c}{l}'$ instantiate the same classes, as they are both contained in $\condition{c}{l,\consistencyrelation{CR}{i+1}}$.
    This would violate \autoref{def:relationtree} for a consistency relation tree, thus there is only one such consistency relation $\consistencyrelation{CR}{s}$.
    Consequently, there must be two condition elements $\conditionelement{c}{ll}, \conditionelement{c}{ll}' \in \condition{c}{l,\consistencyrelation{CR}{s}}$ with $\tupled{\conditionelement{c}{ll},\conditionelement{c}{l}}, \tupled{\conditionelement{c}{ll}',\conditionelement{c}{l}'} \in \consistencyrelation{CR}{s}$, because, per construction, $\modeltuple{m}$ was consistent to $\consistencyrelation{CR}{s}$ and so there must be a witness structure with a unique mapping between condition elements contained in $\modeltuple{m}$.
    The above argument can be applied inductively until we find that there must be two condition elements $\conditionelement{c}{lll},\conditionelement{c}{lll}' \in \condition{c}{l,\consistencyrelation{CR}{}}$ that are contained in $\modeltuple{m}$.
    This is excluded by construction, as we started with only one element from $\condition{c}{l,\consistencyrelation{CR}{}}$, so there is only one such condition element $\conditionelement{c}{l} \in \condition{c}{l,\consistencyrelation{CR}{i+1}}$ with $\modeltuple{m} \containsmath \conditionelement{c}{l}$.
    
    For this $\conditionelement{c}{l} \in \condition{c}{l,\consistencyrelation{CR}{i+1}}$, we select an arbitrary $\conditionelement{c}{r} = \tupled{\object{o}{1}, \dots, \object{o}{s}} \in \condition{c}{r, \consistencyrelation{CR}{i+1}}$ such that $\tupled{\conditionelement{c}{l}, \conditionelement{c}{r}} \in \consistencyrelation{CR}{i+1}$.
    Now we create a model tuple $\modeltuple{m'} = \modeltuple{m} \cup \setted{\object{o}{1}, \dots, \object{o}{s}}$.
    Since $\conditionelement{c}{l}$ is the only left condition element of $\consistencyrelation{CR}{i+1}$ that $\modeltuple{m}$ contains, $\modeltuple{m'}$ is consistent to $\consistencyrelation{CR}{i+1}$ per construction.
    $\modeltuple{m'}$ is also consistent to $\consistencyrelation{CR}{i+1}^T$, since the symmetry of $\consistencyrelation{CR}{i+1}$ and $\consistencyrelation{CR}{i+1}^T$ implies $\conditionelement{c}{r} \in \condition{c}{l,\consistencyrelation{CR}{i+1}^T}$, and due to $\tupled{\conditionelement{c}{r}, \conditionelement{c}{l}} \in \consistencyrelation{CR}{i+1}^T$ a consistent corresponding element exists in $\modeltuple{m'}$. 
    Furthermore, there cannot be any other $\conditionelement{c}{}' \in \condition{c}{l,\consistencyrelation{CR}{i+1}^T}$ with $\modeltuple{m'} \containsmath \conditionelement{c}{}'$, because otherwise there would have been another consistency relation $\consistencyrelation{CR}{}'$ that required the creation of $\conditionelement{c}{}'$, which means that there are two concatenations of consistency relations $\consistencyrelation{CR}{} \concat \dots \concat \consistencyrelation{CR}{}'$ and $\consistencyrelation{CR}{} \concat \dots \concat \consistencyrelation{CR}{i+1}$ that both relate instances of the same classes, which contradicts \autoref{def:relationtree} for a consistency relation tree.
    
    Additionally, we know the following for all $\consistencyrelation{CR}{s}\; (s < i+1)$ due to \autoref{lemma:treehassequence}: First, it is $\classtuple{C}{l,\consistencyrelation{CR}{s}} \cap \classtuple{C}{r,\consistencyrelation{CR}{i+1}} = \emptyset$.
    Since the newly added elements $\conditionelement{c}{r}$ are part of $\condition{c}{r,\consistencyrelation{CR}{i+1}}$, these elements cannot match the left condition of $\consistencyrelation{CR}{s}$.
    So $\modeltuple{m'}$ is still consistent to all $\consistencyrelation{CR}{s}\; (s < i+1)$.
    Second, it is $\classtuple{C}{r,\consistencyrelation{CR}{s}} \cap \classtuple{C}{r,\consistencyrelation{CR}{i+1}} = \emptyset$.
    Again, since the newly added elements $\conditionelement{c}{r}$ are part of $\condition{c}{r,\consistencyrelation{CR}{i+1}}$, these elements cannot match the left condition of $\consistencyrelation{CR}{s}^T$.
    So $\modeltuple{m'}$ is still consistent to all $\consistencyrelation{CR}{s}^T\; (s < i+1)$.
    In consequence, we know that $\modeltuple{m'} \consistenttomath \setted{\consistencyrelation{CR}{1}, \consistencyrelation{CR}{1}^T, \dots, \consistencyrelation{CR}{i+1}, \consistencyrelation{CR}{i+1}^T}$.
    
    Considering the example in \autoref{fig:compatibility:tree_construction_example}, we would select $\consistencyrelation{CR}{2}$ and add for the resident, which is in the left condition elements of $\consistencyrelation{CR}{2}$, an appropriate employee to make the model tuple consistent to $\consistencyrelation{CR}{2}$ (3).
    
    \smallerparagraph{Conclusion:}
    Taking the base case for $\consistencyrelation{CR}{}$ and the induction step for $\consistencyrelation{CR}{i+1}$, we have inductively shown that 
    \begin{align*}
        \formulaskip 
        \modeltuple{m'} \consistenttomath \setted{\consistencyrelation{CR}{1}, \consistencyrelation{CR}{1}^T, \dots, \consistencyrelation{CR}{k}, \consistencyrelation{CR}{k}^T} = \consistencyrelationset{CR}
    \end{align*}
    Since the construction is valid for each condition element in every relation in $\consistencyrelationset{CR}$, we know that a consistency relation tree $\consistencyrelationset{CR}$ is compatible.
\end{proof}
