\selectlanguage{ngerman}
\renewcommand{\arraystretch}{1.3}

\section*{Persönliche Daten}
\begin{tabular}{@{}p{4.5cm} p{8cm}}
Name & Heiko Klare \\
Geburtsdatum-/ort & 26. Juni 1990 in Höxter \\
\ifprintprivateaddress Anschrift (privat) & KASTEL Reussner, \newline Alter Weinberg 5, \newline 76228 Karlsruhe\\\fi%
Anschrift (geschäftlich) & KASTEL Reussner, \newline Am Fasanengarten 5, \newline 76131 Karlsruhe\\
E-Mail (geschäftlich) & klare@kit.edu
\end{tabular}

\section*{Wissenschaftlicher Werdegang}
\begin{tabular}{@{}p{3.5cm} p{10.6cm}}
seit 07/2016 & Doktorand, Karlsruher Institut für Technologie\\

05/2019 & Teilnahme am Dagstuhl-Seminar 19191: \emph{Software Evolution in Time and Space: Unifying Version and Variability Management}\\

11/2018 & Teilnahme am Dagstuhl-Seminar 18491: \emph{Multidirectional Transformations and Synchronisations}\\

04/2014 -- 06/2016 & M.Sc. in Informatik (Note 1,0), \newline Karlsruher Institut für Technologie,\newline Abschluss mit Auszeichnung\\

Masterarbeit & \emph{Designing a Change-Driven Language for Model Consistency Repair Routines} (Note 1,0)\\

10/2014 -- 09/2015 & Stipendiat des Deutschlandstipendiums \\

10/2010 -- 03/2014 & B.Sc. in Informatik (Note 1,1), \newline Karlsruher Institut für Technologie,\newline Abschluss mit Auszeichnung\\

Bachelorarbeit & \emph{Personentracking und Gelenkwinkelschätzung in der Laparoskopie} (Note 1,0)\\
\end{tabular}

\section*{Beruflicher Werdegang}
\begin{tabular}{@{}p{3.1cm} p{11cm}}
seit 01/2021 & Wissenschaftlicher Mitarbeiter, Lehrstuhl Prof. Reussner, \newline KASTEL --  Institut für Informationssicherheit und Verlässlichkeit, \newline Karlsruher Institut für Technologie\\
07/2016 -- 12/2020 & Wissenschaftlicher Mitarbeiter, Lehrstuhl Prof. Reussner, \newline Institut für Programmstrukturen und Datenorganisation, \newline Karlsruher Institut für Technologie\\
Projekte & \vitruv: Sichtenbasierte und konsistenzerhaltende Entwicklung software-intensiver, technischer Systeme\newline Aufgaben: Entwicklung von Mechanismen zur Konsistenzerhaltung von mehr als zwei Modellen, (Neu-)Entwicklung von drei Sprachen zur Konsistenzspezifikation und Code-Generatoren zur Operationalisierung, Gesamtarchitekturelle Weiterentwicklung, Etablierung aktueller Entwicklungsprozessen und Werkzeuge inklusive CI/CD.\\
07/2005 -- 06/2016
& Technischer Zeichner und Softwareentwickler (Erweiterungen für AutoCAD und IT-Administration), Ingenieurbüro Klare\\
\end{tabular}

\section*{Lehrerfahrung}
\begin{tabular}{@{}p{3.1cm} p{11cm}}
08/2020 &
Abschluss des Hochschuldidaktik-Zertifikats (Module 1--3)\\
seit 10/2017 &
Jährliche Gastvorlesung in der Veranstaltung \emph{Modellgetriebene Software-Entwicklung}\\
seit 10/2016 & 
Betreuung des Teils \emph{Parallelverarbeitung} in der Vorlesung \emph{Programmierparadigmen}: Erstellung der Vorlesungsunterlagen, sowie Übungs- und Klausuraufgaben,\newline
Auszeichnung \emph{beste Pflichtvorlesung} im Wintersemester 2017/2018\\
seit 07/2016 &
Betreuung von $13$ Bachelor- und Masterarbeiten\\
seit 07/2016 & Betreuung von sechs Praktikanten in den Praktika \emph{Software Quality Engineering mit Eclipse} und \emph{Ingenieurmäßige Software-Entwicklung} \\
seit 07/2016 & Betreuung von fünf (Pro-)Seminaristen in den (Pro-)Seminaren \emph{Software-Katastrophen} und \emph{Daten in software-intensiven technischen Systemen -- Modellierung -- Analyse -- Schutz}\\
10/2016 -- 09/2017 &
Betreuung von acht Gruppen zur \emph{Praxis der Softwareentwicklung}\\
04/2012 -- 07/2015 & Tutor für die Vorlesung \emph{Softwaretechnik I} (jährlich April bis Juli)\\
10/2013 -- 02/2014 & Tutor für die Vorlesung \emph{Betriebssysteme}
\end{tabular}

\section*{Schulbildung und Grundwehrdienst}
\begin{tabular}{@{}p{3.1cm} p{11cm}}
07/2009 -- 07/2010 & Verlängerter Grundwehrdienst, Personalabteilung,\newline ABC-Abwehrbataillon 7, Höxter\\
2009 & Abitur (Note 1,3), \newline Petrus-Legge-Gymnasium, Brakel,\newline Preis der Deutschen Physikalischen Gesellschaft
\end{tabular}

\vspace{2\baselineskip}
\noindent Karlsruhe, den \disssubmissiondate

\vspace{2\baselineskip}
\noindent \dissauthor
