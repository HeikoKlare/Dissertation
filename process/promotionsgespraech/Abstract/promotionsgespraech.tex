\documentclass[a4paper,deutsch,CS,internal,11pt]{KITbrief}
\usepackage[utf8]{inputenc}
\usepackage[T1]{fontenc}
\usepackage[english,german,ngerman]{babel}

\clubpenalty=10000
\widowpenalty=10000

\usepackage{xspace}
\newcommand{\doktorand}{Heiko Klare}
\newcommand{\disstitle}{Building Transformation Networks for Consistent Evolution\\ of Multiple Models}
\newcommand{\referent}{Prof.~Dr.~Ralf Reussner\xspace}
\newcommand{\coreferentshort}{Prof.~Dr.~Colin Atkinson\xspace}
\newcommand{\coreferent}{\coreferentshort(Universität Mannheim)\xspace}


\begin{document}

Dissertationsvorhaben \doktorand

\vspace{-0.25em}
Erstgutachter: \referent

\vspace{-0.25em}
Zweitgutachter: \coreferent

\vspace{0.5em}

\textbf{\Large \disstitle}

\vspace{0.15em}


\emph{In meiner Promotion habe ich die Konsistenzhaltung verschiedener Artefakte zur Beschreibung eines Softwaresystems untersucht, indem ich die Kopplung von Transformationen zwischen den Spezifikationssprachen dieser Artefakte formalisiert, analysiert und mit Methoden unterstützt habe.}

Bei der Entwicklung eines Softwaresystems nutzen die verschiedenen Entwickler/-innen und weiteren Interessenvertreter/-innen diverse Sprachen zur Beschreibung unterschiedlicher Belange.
Meist stellt der Programmcode das zentrale Artefakt dar, wird jedoch, implizit oder explizit, durch Spezifikationen der Architektur, des Deployments oder der Anforderungen ergänzt.
Zur Spezifikation dieser Artefakte werden neben der Programmiersprache verschiedene andere Sprachen verwendet, beispielsweise UML für Architekturmodelle, OpenAPI für Schnittstellen-Spezifikationen oder Docker für Deployment-Spezifikationen von Microservices.
Zur Erstellung eines funktionsfähigen Softwaresystems müssen diese Artefakte eine einheitliche, widerspruchsfreie Spezifikation des gesamten Systems darstellen.
Wir sagen, die Artefakte müssen konsistent sein.

In der modellgetriebenen Entwicklung werden solch verschiedene Artefakte, dort allgemein \emph{Modelle} genannt, bereits als wesentliche zentrale Entwicklungsbestandteile genutzt, um auch Teile des Programmcodes aus ihnen abzuleiten.
Dies betrifft beispielsweise die Softwareentwicklung für Fahrzeuge oder allgemein die Entwicklung cyber-physikalischer Systeme.
Zur Konsistenzhaltung der Modelle werden hier oftmals Transformationen eingesetzt, die nach Änderungen eines Modells die anderen Modelle anpassen.
Die bisherige Forschung beschränkt sich jedoch auf die Konsistenzhaltung zweier Modelle, oder die aufeinander abgestimmte Konsistenzhaltung mehrerer Modelle.
Ein systematischer Entwicklungsprozess, in dem einzelne Transformationen unabhängig und modular entwickelt und in verschiedenen Kontexten wiederverwendet werden können, wird hierdurch jedoch nicht unterstützt.

In meiner Dissertation befasse ich mich daher mit der Frage, wie Entwickler mehrere Transformationen zu einem Netzwerk kombinieren können, welches die Transformationen in einer geeigneten Reihenfolge ausführen kann, sodass abschließend alle Modelle konsistent zueinander sind.
Dies geschieht unter der Annahme, dass einzelne Transformationen zwischen zwei Sprachen unabhängig voneinander entwickelt und daher nicht aufeinander abgestimmt werden.
Meine Beiträge unterteilen sich dabei in die Untersuchung der Korrektheit einer solchen Kombination und die Auswirkung auf und Optimierung von Qualitätseigenschaften eines solchen Netzwerkes.

Ich leite zunächst einen adäquaten Korrektheitsbegriff her und definiere ihn mittels eines angemessenen Formalismus.
Dieser Korrektheitsbegriff ergibt die Notwendigkeit dreier weiterer Betrachtungen. Erstens fordert er von Transformationen eine \emph{Synchronisations-Eigenschaft}, zweitens erfordert er eine Form von \emph{Kompatibilität} zwischen den Transformationen, und drittens erfordert er das Finden einer korrekten Ausführungsreihenfolge der Transformationen, einer \emph{Orchestrierung}.
Ich leite her, wie mit bestehenden Sprachen Transformationen entwickelt werden können, die die Synchronisations-Eigenschaft erfüllen, und schlage ein entsprechendes Konstruktionsverfahren vor, welches ich in Fallstudien im Kontext der komponentenbasierten Softwareentwicklung evaluiere.
Weiterhin definiere ich formal die Eigenschaft der Kompatibilität von Transformationen, für welche ich ein formales Analyseverfahren vorschlage und dessen Korrektheit beweise. Außerdem entwickele ich daraus eine praktische Realisierung, deren Anwendbarkeit ich ebenfalls in Fallstudien untersuche.
Außerdem untersuche ich, wie für Transformationen eine valide Orchestrierung, also eine Ausführungsreihenfolge nach der alle Modelle konsistent sind, gefunden werden kann.
Dabei beweise ich, dass das allgemeine zugrundeliegende Problem unentscheidbar ist und stelle Indikatoren dafür vor, dass Einschränkungen des Problems, sodass es entscheidbar wird, die praktische Anwendbarkeit zu stark einschränken.
Ich schlage daher einen Algorithmus vor, der die Lösung konservativ approximiert. % und auf einer Systematik basiert, die den Entwickler im Fehlerfall bei der Identifikation der Ursache unterstützen kann.
Des Weiteren kategorisiere ich Fehler, die auftreten können wenn die identifizierten Korrektheitseigenschaften nicht erfüllt werden, und analysiere deren Relevanz in oben genannten Fallstudien.
Hieraus leite ich zum einen ab, inwieweit das fehlerhafte Verhalten des Netzwerkes Rückschlüsse auf die Art des ursächlichen Fehler zulässt, und zeige zum anderen, dass sich die meisten potentiellen Fehler mit den vorgeschlagenen Verfahren per Konstruktion vermeiden lassen.

Zur Untersuchung von Qualitätseigenschaften eines Netzwerkes von Transformationen identifiziere ich zunächst relevante Eigenschaften und untersuche, wie sich verschiedene Typen von Netzwerktopologien auf diese auswirken.
Hierbei zeigt sich, dass insbesondere Korrektheit und Modularität im Widerspruch stehen und die Wahl der Topologie ein Abwägen bei der Optimierung dieser Eigenschaften induziert.
Ich leite hieraus ein Konstruktionsverfahren für Transformationsnetzwerke ab, welches die Notwendigkeit einer Abwägung zwischen den Qualitätseigenschaften abmildert und, unter gewissen Voraussetzungen, Korrektheit per Konstruktion gewährleistet. %und gleichzeitig die modulare Wiederverwendung aller Teile des Transformationsnetzwerkes gewährleitet.
Für dieses Verfahren stelle ich außerdem eine Spezifikationssprache vor, die den Entwicklungsprozess strukturiert.
Ich evaluiere die Anwendbarkeit des Verfahrens und der Sprache durch die Implementierung der bereits für die Korrektheitseigenschaften verwendete Fallstudie aus der komponentenbasierten Softwareentwicklung. %, in der insbesondere die Erfüllung der Voraussetzungen des Ansatzes in realen Szenarien gezeigt werden konnte.

Die Beiträge meiner Arbeit unterstützen sowohl Forscher/-innen, als auch Transformationsentwickler/-innen und Transformationsanwender/-innen bei der Analyse und Konstruktion von Netzwerken von Transformationen.
Sie stellen für Forscher/-innen und Transformationsentwickler/-innen systematisches Wissen über die Korrektheit und weitere Qualitätseigenschaften solcher Netzwerke bereit, und zeigen insbesondere, welche Teile dieser Eigenschaften per Konstruktion oder per Analyse erreicht bzw. nachgewiesen werden können, sowie mit welchen Fehlern bei der Ausführung gerechnet werden muss.
Zusätzlich stelle ich mit den Beiträgen konkrete, praktisch nutzbare Verfahren bereit, mit denen korrekte, modulare Netzwerke konstruiert, analysiert und ausgeführt werden können, wovon sowohl Transformationsentwickler/-innen als auch Transformationsanwender/-innen profitieren.

\end{document}