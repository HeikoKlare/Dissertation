% requires tikzvitruvius.sty
\begin{tikzpicture}[
    vtcaption/.style={font=\small},
    view/.style={uml box, densely dashed},
    consarrow/.style={latex-latex, thin, double},
    mir/.style={draw, fill=white, rectangle, rounded corners=4pt, inner sep=.25em, font=\scriptsize},
    viewtype/.style={circle, draw, solid, fill=white, inner sep=.1em, font=\scriptsize},
    polarrow/.style={latex-latex, densely dotted},
    mininode/.style={inner sep=.4em},
    %viewtype/.append style={blueshading},
    %mir/.append style={blueshading},
    ]
%% SUM
\node[uml small box] (pcm) {PCM};
\node[uml small box] (uml) [below left=2em and 1em of pcm] {UML};
\node[uml small box] (java) [above left=2em and 1em of uml] {Java};

\node[circle,draw,thick,fit=(pcm)(uml.north)(java),inner sep=2ex] (sum) {};

\draw[consarrow] (java) -- node[mir] (pcmJavaCCR) {CPR} (pcm);
\draw[consarrow] (pcm) to [bend left] node[mir] (pcmUMLCCR) {CPR} (uml);
% \draw[consarrow] (uml) to [bend left] node[left, font=\footnotesize] {cons$_2$} (java);
\draw[consarrow] (uml) to [bend left] node[mir] (umlJavaCCR) {CPR} (java);

%% Klassendiagramm

% Mittelpunkt
\node (cdmpoint) [below left=6em and -5.5em of sum] {};

\node[uml small box] (c1) [above=.3em of cdmpoint] {C$_1$};
\node[uml small box] (c2) [below left=1em and -.3em of c1] {C$_2$};
\node[uml small box] (c3) [right=1em of c2] {C$_3$};

\draw[open triangle 90-] (c1) -- ++ (0,-1.5em) -| (c2);
\draw[open triangle 90-] (c1) -- ++ (0,-1.5em) -| (c3);

\node[view,fit=(c1)(c2)(c3)] (classview) {};
\node[vtcaption] (classviewtext) [below=0.3em of classview] 
{UML Class Diagram View};

\path (classview) -- node [at end,viewtype] (viewtype1) {$\mathvariable{VT}_1$} (sum.290);

\path (classview) edge [-latex] (viewtype1);

\path (viewtype1) edge [polarrow, latex-latex] (uml);

%% Klasse-Komponenten-Diagramm

% Mittelpunkt
\node (ccmpoint) [above right=7.5em and 9.5em of classview] {};

{
\tiny
\pcmcomponentbody{}{comp1ccm}{comp$_1$}{at (ccmpoint.center)}
\pcmlolliwest{}{lolli1ccm}{(comp1ccm.west)}
}

\node[uml small box] (c1') [above right=0em and 3em of comp1ccm] {C$_1$};
\node[uml small box] (c2') [below right=0em and 3em of comp1ccm] {C$_2$};

\draw[open triangle 90-] (c1') -- (c2');

\draw[-latex, dashed] (c1') -- node[above,sloped,font=\tiny] {implements} (comp1ccm);
\draw[-latex, dashed] (c2') -- node[below,sloped,font=\tiny] {implements} (comp1ccm);

\node[view,fit=(comp1ccm)(lolli1ccm)(c1')(c2')] (classcompview) {};
\node[vtcaption] (classcompviewtext) [above=.3em of classcompview] 
{Component-Class Implementation View};

\path (classcompview) -- node [viewtype,at end] (viewtype2) {$\mathvariable{VT}_2$} (sum.335);
\path ([yshift=1em]classcompview.south west) edge [-latex] (viewtype2);
\path (viewtype2) edge [polarrow, latex-] (uml);
\path (viewtype2) edge [polarrow, latex-] (pcm);
\path (viewtype2) edge [polarrow, latex-] (pcmUMLCCR);

%% PCM-Komponenten-Diagramm
\node[] (pcmRepView) [above right=3.5em and 10.5em of sum] {};%
%\node (pcmpoint) [above right=10cm and 4em of sum] {};

{\tiny
\pcmcomponentbody{}{pcmcomp1}{comp$_1$}{at (pcmRepView)}
\pcmlolliwest{}{pcmcomp1lolli}{(pcmcomp1.west)}
\pcmcomponent{}{pcmcomp2}{comp$_2$}{[left=2.5em of pcmcomp1text]}
}

%\draw[->, dashed] (pcmcomp2arc) -- (pcmcomp1lolli);

\node[view,fit=(pcmcomp1)(pcmcomp2)(pcmcomp2lolli)] (pcmview) {};

\node[vtcaption] (pcmviewtext) 
[above=0.3em of pcmview] {PCM System View};

\path (pcmRepView) -- node[viewtype,at end] (viewtype3) {$\mathvariable{VT}_3$} (sum.40);
%\node[viewtype] (viewtype3) at (epoint3) {VT$_3$};
\path (pcmview) edge [-latex] (viewtype3);
\path (viewtype3) edge [polarrow, latex-latex] (pcm);

\node[view, font=\tiny\ttfamily] (javasource) [above left=4em and -10em of sum] {%
\begin{lstlisting}[%
numbers=none,
language=java,
backgroundcolor=\color{white},
basicstyle=\tiny\ttfamily,
frame=none,
aboveskip=0pt,
belowskip=0pt,
linewidth=3.7cm,
breaklines=true]
@ADLImplements(implements-component comp_1)
public class C2 extends C1 {
    public static void main (String[] args) {
        System.out.println ("Hello World!");
    }
}
\end{lstlisting}
};
\node[vtcaption] (javatext) [above=0.3em of javasource] 
{Annotated Java Source Code View};

\path (javasource) -- node [at end, viewtype] (viewtype4) {$\mathvariable{VT}_4$} (sum.100);
%\node[viewtype] (viewtype4) at (epoint4) {VT$_4$};
\draw[-latex] (javasource) -- (viewtype4);
\draw[polarrow] (viewtype4) -- (java);
\draw[polarrow] (viewtype4) -- (pcm.north west);
\draw[polarrow,latex-] (viewtype4) -- (pcmJavaCCR);
%\draw[polarrow,latex-] (viewtype4) to [bend right=10] (pcmUMLCCR);
%\draw[polarrow,latex-] (viewtype4) to [bend left=10] (umlJavaCCR);

% Legende
\node[draw, legendbg, matrix, font=\scriptsize, inner sep=0.6em, nodes=mininode] (legend) at (12.5em,-10em) {%
\node[uml small box, minimum height=0.1em, inner sep=0.3em, font=\scriptsize, xshift=2em] (legend_mm) {MM}; & \node[anchor=base west, font=\footnotesize] {Metamodel};\\
\node[viewtype, xshift=2em, font=\tiny] {$\mathvariable{VT}$}; & \node[anchor=base west, font=\footnotesize] {\ViewType};\\
\draw[-latex] (0,.5ex)--(4em,.5ex); & \node[anchor=base west, font=\footnotesize] {Instance of a \ViewType};\\
\draw[polarrow] (0,.5ex)--(4em,.5ex); & \node[anchor=base west, font=\footnotesize] {View Transformation};\\
\draw[consarrow] (0,.5ex) to node[midway,mir] {CPR} (4em,.5ex); & \node[anchor=base west, font=\footnotesize] {Consistency Preservation Rule};\\
};

\foreach \x in {pcm,uml,java,legend_mm} {
\draw (\x.north west) ++ (.2pt,0) -- ++ (0,.3em) -- ++ (0.6em,0) -- ++ (0,-.3em);
}

\node[font=\bfseries\scriptsize] (sumtext) [above right=0em of sum.240, anchor=south west, align=center] {\vsum\\ Metamodel};

\end{tikzpicture}

% %%
% %% requires: tikzkit.sty, tikzvitruvius.sty
% %% 
% \begin{tikzpicture}[
%     vtcaption/.style={font=\small},
%     metamodel/.style={draw},
%     instanceof/.style={dashed},
%     consarrow/.style={latex-latex, thin, double},
%     mir/.style={draw, fill=white, rectangle, rounded corners=4pt, inner sep=.25em, font=\scriptsize},
%     viewtype/.style={circle, draw, solid, fill=white, inner sep=.1em, font=\scriptsize},
%     polarrow/.style={latex-latex, densely dotted},
%     mininode/.style={inner sep=.25em},
%     view/.style={uml box, densely dashed},
%     %view/.append style={draw=gray},
%     %viewtype/.append style={blueshading},
%     %mir/.append style={blueshading},
% ]
% %% SUM
% \node[uml small box] (pcm) {ADL};
% \node[uml small box] (uml) [below left=2em and .5em of pcm] {UML};
% \node[uml small box] (java) [above left=2em and .5em of uml] {Java};
% % \node[metamodel, text width=5em] (omnet) [below left=2.5em and 1em of pcm] {Sensor Model};

% \foreach \x in {pcm,uml,java} {
% \draw (\x.north west) ++ (.2pt,0) -- ++ (0,.5em) -- ++ (1em,0) -- ++ (0,-.5em);
% }

% \node[circle,thick,draw,fit=(pcm)(uml.north)(java), inner sep=1.5em] (sum) {};

% \node[font=\bfseries\normalsize] (sumtext) [below=.5em of sum.north, align=center] {\vsum\\ Metamodel};

% \draw[consarrow] (pcm) to [bend left] node[mir] (mir1) {CPR} (uml);
% \draw[consarrow] (uml) to [bend left] node[mir] (mir2) {CPR} ([xshift=-0.5em]java.south);
% \draw[consarrow] (java) to node[mir] (mir3) {CPR} (pcm);
% % \draw[consarrow] (java) to node[mir] (mir3) {CPR} (pcm);
% % \draw[consarrow] (omnet) to [bend left] node[mir] (mir4) {CPR} (pcm);

% %% Methodologist
% %\node (user5) [below left=5em and -1em of sum] {\includegraphics[width=2em]{user-blue}};
% \umlhuman{user5}{[above right=-2em and 8em of sumtext]}{semithick}{}{1};
% \node (user5text) [below=0em of user5] {methodologist};
% \draw[-latex,thick] (user5) to [bend left=15] node[above, sloped, pos=0.4] {defines} (sumtext); 


% %% Klassendiagramm

% % Mittelpunkt
% \node (cdmpoint) [below left=2em and 6em of sum] {};

% \node[uml small box] (c1) [above=.3em of cdmpoint] {C\textsubscript{1}};
% \node[uml small box] (c2) [below left=.3em of cdmpoint] {C\textsubscript{2}};
% \node[uml small box] (c3) [below right=.3em of cdmpoint] {C\textsubscript{3}};

% \draw[open triangle 90-] (c1.south) -- ++ (0,-.75em) -| (c2);
% \draw[open triangle 90-] (c1.south) -- ++ (0,-.75em) -| (c3);

% \node[view,fit=(c1)(c2)(c3)] (classview) {};
% \node[vtcaption] (classviewtext) [below=.3em of classview] {UML class diagram view};

% \path (classview) edge [-latex, instanceof, shorten >=1em] node [viewtype,at end] (viewtype2) {VT\textsubscript{2}} (sum.220);
% \path (viewtype2) edge [polarrow] (uml);

% %% Klasse-Komponenten-Diagramm

% % Mittelpunkt
% \node (ccmpoint) [below right=4em and 7em of sum] {};

% {\tiny
% \pcmcomponentbody{}{comp1ccm}{comp\textsubscript{1}}{at (ccmpoint.center)}
% \pcmlolliwest{}{lolli1ccm}{(comp1ccm.west)}
% }

% \node[uml small box] (c1') [above right=-.5em and 3.5em of comp1ccm] {C\textsubscript{1}};
% \node[uml small box] (c2') [below right=-.5em and 3.5em of comp1ccm] {C\textsubscript{2}};

% \draw[open triangle 90-] (c1') -- (c2');

% \draw[->, dashed] (c1') -- node[above,sloped,font=\tiny] {implements} (comp1ccm);
% \draw[->, dashed] (c2') -- node[below,sloped,font=\tiny] {implements} (comp1ccm);

% \node[view,fit=(comp1ccm)(lolli1ccm)(c1')(c2')] (classcompview) {};
% \node[vtcaption] (classcompviewtext) [below=.3em of classcompview] {component-class implementation view};

% \path (classcompview) edge [-latex, instanceof, shorten >=1em] node[viewtype, at end] (viewtype3) {VT\textsubscript{3}} (sum.320);
% \path (viewtype3) edge [polarrow, latex-] (uml);
% \path (viewtype3) edge [polarrow] (pcm);

% % OO-Entwickler

% %\node (user1) [above right=2em and 0em of classview] {\includegraphics[width=2em]{user-green}};
% \umlhuman{user1}{[below=1.7em of sum]}{semithick}{}{1};
% \node (user1text) [below=0em of user1] {component developer};


% \draw[-latex,thick] (user1) to [bend right=15] node[above, sloped] {uses} (classcompview.190); 
% \draw[-latex,thick] (user1) to [bend left=15] node[above, sloped] {uses} (classview); 

% %% Java-Quelltext

% \node[view] (javasource) [above left=-4em and 6em of sum] {%
% \begin{lstlisting} [
% language=java,
% basicstyle=\tiny\ttfamily,
% frame=none,
% aboveskip=0pt,
% belowskip=0pt
% linewidth=12em,
% breaklines,
% numbers=none,
% ]
% /* @implements-component comp_1
% */
% public class C2 extends C1 {
% public static void main (String[] args) {
% System.out.println ("Hello World!");
% }
% }
% \end{lstlisting}
% };
% \node (javasourcetext) [below=0em of javasource] {Java source view};

% \path (javasource) edge [-latex, instanceof, shorten >=1em] node [viewtype,at end] (viewtype1) {VT\textsubscript{1}} (sum);
% \path (viewtype1) edge [polarrow] (java);
% \path (viewtype1) edge [polarrow, bend left=35] ([yshift=0.5em]pcm.west);

% %\node (user2) [right=3.5em of javasource] {\includegraphics[width=2em]{user-green}};
% \umlhuman{user2}{[below left=5em and -3em of javasource]}{semithick}{}{1};
% \node (user2text) [below=0em of user2] {programmer};
% \draw[-latex,thick] (user2) to [bend right=15] node[below, sloped] {uses} (javasourcetext); 

% %% Komponentenmodell
% \node (pcmpoint) [above right=-8em and 14em of sum] {};

% {\tiny
% \pcmcomponentbody{}{pcmcomp1}{comp\textsubscript{1}}{at (pcmpoint.center)}
% \pcmlolliwest{}{pcmcomp1lolli}{(pcmcomp1.west)}
% \pcmcomponent{}{pcmcomp2}{comp\textsubscript{2}}{[left=2.3em of pcmcomp1text]}
% }
% %\draw[->, dashed] (pcmcomp2arc) -- (pcmcomp1lolli);

% \node[view,fit=(pcmcomp1)(pcmcomp2)(pcmcomp2lolli)] (pcmview) {};
% \node[vtcaption] (pcmviewtext) [below=.3em of pcmview] {component diagram view};

% \path (pcmview) edge [-latex, instanceof, shorten >=1em] node [viewtype,at end] (viewtype4) {VT\textsubscript{4}} (sum);
% \path (viewtype4) edge [polarrow] (pcm);

% \umlhuman{user3}{[below right=0em and 6em of pcmview]}{semithick}{}{1};
% \node (user3text) [below=0em of user3] {system architect};
% \draw[-latex,thick] (user3) to [bend left=15] node[above, sloped] {uses} (pcmview.east); 
% \draw[-latex,thick] (user3text) to [bend left=15] node[above, sloped] {uses} (classcompview); 

% %% Simulation
% % \node[view] (chart) [above left=5em and -4em of sum] {
% % \begin{tikzpicture}[solid, minimum width=0em, minimum height=0em]
% % \draw[step=.4em,ultra thin] (-.1em,-.1em) grid (3.9em,2.9em);

% % \draw[thin,-latex] (0,-.1em) -- node[inner sep=0.1em, above=0em,sloped,font=\tiny] {Probability} (0,3em);
% % \draw[thin,-latex] (-.1em,0) -- node[inner sep=0.1em, below=0em,font=\tiny] {Time} (4em,0);

% % \foreach \x in {0.2,0.3,...,3.4}
% % \pgfmathrandominteger{\y}{0}{25}
% % \draw[line width=.1em] (\x em,0em) -- ++ ($\y *(0,0.1em)$);
% % \foreach \x in {0.2,0.4,...,3.4}
% % 	\pgfmathrandominteger{\y}{0}{25}
% % 	\draw[line width=.2em,gray] (\x em,0em) -- ++ ($\y *(0,0.1em)$);
% % \end{tikzpicture}
% %\includegraphics[width=5em]{simulation_chart}
% % };
% % \node[vtcaption] (charttext) [below=.3em of chart] {simulation results};
% % \path (charttext) edge [-latex, shorten >=1em] node [viewtype,at end] (viewtype5) {VT$_5$} (sum);
% % \path (viewtype5) edge [polarrow] (pcm);
% % \path (viewtype5) edge [polarrow] (omnet);

% %\node (user4) [above right=5em and -1.5em of chart] {\includegraphics[width=2em]{user-green}};
% % \umlhuman{user4}{[above right=5em and -1.5em of chart]}{semithick}{};
% % \node (user4text) [below=0em of user4] {performance engineer};
% % \draw[-latex,thick] (user4text) to [bend right] node[above, sloped] {uses} (pcmview.west); 
% % \draw[-latex,thick] (user4text) to [bend right] node[above, sloped] {uses} (chart); 

% %% Ende

% \node[draw, matrix, font=\scriptsize, nodes=mininode, fill=lightgray!30] (legend) at (20em, 3em) {%
% \draw[-latex, dashed](0,.5ex)--(4em,.5ex); & \node[anchor=base west] {«instance of»};\\
% \draw[polarrow](0,.5ex)--(4em,.5ex); & \node[anchor=base west] {view transformation};\\
% \draw[consarrow](0,.5ex) to node[midway,mir] {CPR} (4em,.5ex); & \node[anchor=base west] {consistency preservation rule};\\
% };

% \end{tikzpicture}
