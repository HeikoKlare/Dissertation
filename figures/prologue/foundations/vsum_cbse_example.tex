% requires tikzvitruvius.sty
\begin{tikzpicture}[
    vtcaption/.style={font=\small},
    view/.style={uml box, densely dashed},
    consarrow/.style={latex-latex, thin, double},
    mir/.style={draw, fill=white, rectangle, rounded corners=4pt, inner sep=.25em, font=\scriptsize},
    viewtype/.style={circle, draw, solid, fill=white, inner sep=.1em, font=\scriptsize},
    polarrow/.style={latex-latex, densely dotted},
    mininode/.style={inner sep=.4em},
    uniformly sized package/.style={minimum width=2.5em},
]
%% SUM
\umlpackage[uniformly sized package]{pcm}{}{PCM}
\umlpackage[uniformly sized package]{uml}{below left=2em and 1em of pcm}{UML}
\umlpackage[uniformly sized package]{java}{above left=2em and 1em of uml}{Java}

\node[circle,draw,thick,fit=(pcm)(uml.north)(java),inner sep=2ex] (sum) {};

\draw[consarrow] (java) -- node[mir] (pcmJavaCCR) {CPR} (pcm);
\draw[consarrow] (pcm) to [bend left] node[mir] (pcmUMLCCR) {CPR} (uml);
\draw[consarrow] (uml) to [bend left] node[mir] (umlJavaCCR) {CPR} (java);

%% Klassendiagramm

% Mittelpunkt
\node (cdmpoint) [below left=5.5em and -5.5em of sum] {};

\node[uml small box] (c1) [above=.3em of cdmpoint] {C$_1$};
\node[uml small box] (c2) [below left=1em and -.3em of c1] {C$_2$};
\node[uml small box] (c3) [right=1em of c2] {C$_3$};

\draw[open triangle 90-] (c1) -- ++ (0,-1.5em) -| (c2);
\draw[open triangle 90-] (c1) -- ++ (0,-1.5em) -| (c3);

\node[view,fit=(c1)(c2)(c3)] (classview) {};
\node[vtcaption] (classviewtext) [below=0.3em of classview] 
{UML Class Diagram View};

\path (classview) -- node [at end,viewtype] (viewtype1) {$\mathvariable{VT}_1$} (sum.290);

\path (classview) edge [-latex] (viewtype1);

\path (viewtype1) edge [polarrow, latex-latex] (uml);

%% Klasse-Komponenten-Diagramm

% Mittelpunkt
\node (ccmpoint) [above right=7.5em and 9.5em of classview] {};

{
\tiny
\pcmcomponentbody{}{comp1ccm}{comp$_1$}{at (ccmpoint.center)}
\pcmlolliwest{}{lolli1ccm}{(comp1ccm.west)}
}

\node[uml small box] (c1') [above right=0em and 3em of comp1ccm] {C$_1$};
\node[uml small box] (c2') [below right=0em and 3em of comp1ccm] {C$_2$};

\draw[open triangle 90-] (c1') -- (c2');

\draw[-latex, densely dashed] (c1') -- node[above,sloped,font=\tiny] {implements} (comp1ccm);
\draw[-latex, densely dashed] (c2') -- node[below,sloped,font=\tiny] {implements} (comp1ccm);

\node[view,fit=(comp1ccm)(lolli1ccm)(c1')(c2')] (classcompview) {};
\node[vtcaption] (classcompviewtext) [above=.3em of classcompview] 
{Component-Class Implementation View};

\path (classcompview) -- node [viewtype,at end] (viewtype2) {$\mathvariable{VT}_2$} (sum.335);
\path ([yshift=1em]classcompview.south west) edge [-latex] (viewtype2);
\path (viewtype2) edge [polarrow, latex-] (uml);
\path (viewtype2) edge [polarrow, latex-] (pcm);
\path (viewtype2) edge [polarrow, latex-] (pcmUMLCCR);

%% PCM-Komponenten-Diagramm
\node[] (pcmRepView) [above right=3.5em and 10.5em of sum] {};%

{\tiny
\pcmcomponentbody{}{pcmcomp1}{comp$_1$}{at (pcmRepView)}
\pcmlolliwest{}{pcmcomp1lolli}{(pcmcomp1.west)}
\pcmcomponent{}{pcmcomp2}{comp$_2$}{[left=2.5em of pcmcomp1text]}
}

\node[view,fit=(pcmcomp1)(pcmcomp2)(pcmcomp2lolli)] (pcmview) {};

\node[vtcaption] (pcmviewtext) 
[above=0.3em of pcmview] {PCM System View};

\path (pcmRepView) -- node[viewtype,at end] (viewtype3) {$\mathvariable{VT}_3$} (sum.40);
\path (pcmview) edge [-latex] (viewtype3);
\path (viewtype3) edge [polarrow, latex-latex] (pcm);

\node[view, font=\tiny\ttfamily] (javasource) [above left=3.5em and -10em of sum] {%
\begin{lstlisting}[%
numbers=none,
language=java,
backgroundcolor=\color{white},
basicstyle=\tiny\ttfamily,
frame=none,
aboveskip=0pt,
belowskip=0pt,
linewidth=3.7cm,
deletekeywords={implements},
breaklines=true]
@ADLImplements(implements-component comp_1)
public class C2 extends C1 {
    public static void main (String[] args) {
        System.out.println ("Hello World!");
    }
}
\end{lstlisting}
};
\node[vtcaption] (javatext) [above=0.3em of javasource] 
{Annotated Java Source Code View};

\path (javasource) -- node [at end, viewtype] (viewtype4) {$\mathvariable{VT}_4$} (sum.100);
\draw[-latex] (javasource) -- (viewtype4);
\draw[polarrow] (viewtype4) -- (java);
\draw[polarrow] (viewtype4) -- (pcm.north west);
\draw[polarrow,latex-] (viewtype4) -- (pcmJavaCCR);

% Legende
\node[draw, legendbg, matrix, font=\scriptsize, inner sep=0.7em, nodes=mininode] (legend) at (12em,-9.7em) {%
\umlpackage[minimum height=0.1em, inner sep=0.3em, yshift=0.5ex, font=\scriptsize, xshift=2em]{egend_mm}{}{MM} & \node[anchor=base west, font=\footnotesize] {Metamodel\sameheight};\\
\node[viewtype, xshift=2em, yshift=0.5ex, font=\scriptsize] {$\mathvariable{VT}$}; & \node[anchor=base west, font=\footnotesize] {\ViewType\sameheight};\\
\draw[-latex] (0,.5ex)--(4em,.5ex); & \node[anchor=base west, font=\footnotesize] {Instantiation of a \ViewType\sameheight};\\
\draw[polarrow] (0,.5ex)--(4em,.5ex); & \node[anchor=base west, font=\footnotesize] {View Transformation\sameheight};\\
\draw[consarrow] (0,.5ex) to node[midway,mir] {CPR} (4em,.5ex); & \node[anchor=base west, font=\footnotesize] {Consistency Preservation Rule\sameheight};\\
};

\node[font=\bfseries\scriptsize] (sumtext) [above right=0em of sum.240, anchor=south west, align=center] {\vsum\\ Metamodel};

\end{tikzpicture}
