\usetikzlibrary{arrows.meta}

\newlength{\transformationdoublespacing}
\newlength{\transformationlinewidth}
\newlength{\transformationshorten}
\newlength{\transformationarrowwidth}
\newlength{\transformationarrowheight}
\newlength{\transformationbaselength}
\setlength{\transformationbaselength}{0.5pt}
\newcommand*{\settransformationscale}[1]{%
	\setlength{\transformationdoublespacing}{\dimexpr #1\transformationbaselength * 3 / 2\relax}%
	\setlength{\transformationlinewidth}{\dimexpr #1\transformationbaselength\relax}%
	\setlength{\transformationshorten}{\dimexpr #1\transformationbaselength * 5\relax}
	\setlength{\transformationarrowheight}{\dimexpr #1\transformationbaselength * 10\relax}%
	\setlength{\transformationarrowwidth}{\dimexpr \transformationarrowheight * 8 / 5\relax}%
	\tikzset{%
		halfsyncxarrow/.tip = {Latex[round,left,length=\the\transformationarrowheight,width=\the\transformationarrowwidth,angle'=45]},%
		halfsyncxarrow'/.tip = {Latex[round,right,length=\the\transformationarrowheight,width=\the\transformationarrowwidth,angle'=45]},%
		syncxarrow/.tip={Latex[round,length=\the\transformationarrowheight,width=\the\transformationarrowwidth,angle'=45]}%
	}%
}
\settransformationscale{1}
\tikzstyle{halfsyncx}=[
	-halfsyncxarrow,
	line width=\transformationlinewidth,
	line cap=round
]
\NewDocumentCommand{\halfsyncx}{O{} r()O{0pt} r()O{0pt} D<>{}}{%
	\path (#2) -- (#4) coordinate[at start](h1) coordinate[at end](h2);
	\draw [
		halfsyncx,
		shorten < = \transformationshorten+#3,shorten > = \transformationshorten+#5,#1
	]
	($(h1)!\transformationdoublespacing!90:(h2)$) -- ($(h2)!\transformationdoublespacing!-90:(h1)$)
	node[midway, above]{#6};
}
\NewDocumentCommand{\syncx}{O{} r()O{0pt} r()O{0pt} D<>{}}{
	\halfsyncx[#1] (#2)[#3] (#4)[#5] <#6>
	\halfsyncx[#1] (#4)[#5] (#2)[#3]
}


\newcommand{\modelradius}{0.1em}
\newcommand{\hmodeldistance}{1.07em}
\newcommand{\vmodeldistance}{0.9em}
\newcommand{\hnetworkdistance}{3.6*\hmodeldistance}
\newcommand{\vnetworkdistance}{4.15*\vmodeldistance}
\newcommand{\hborder}{0.4*\hmodeldistance}
\newcommand{\vborder}{0.4*\hmodeldistance}
\newcommand{\addnetworktopleft}[3]{
	\node[model,#2] (m#1left) {};
	\node[model, above right=\vmodeldistance and \hmodeldistance of m#1left.center, anchor=center] (m#1top) {};
	\syncx[unconsidered, #3] (m#1left) (m#1top)
}
\newcommand{\addnetworkmiddle}[2]{
	\node[model, below right=\vmodeldistance and \hmodeldistance of m#1left.center, anchor=center] (m#1bottom) {};
	\syncx[unconsidered, #2] (m#1top) (m#1bottom)
}
\newcommand{\addnetworktopright}[2]{
	\node[model, right=2*\hmodeldistance of m#1left.center, anchor=center] (m#1right) {};
	\syncx[unconsidered, #2] (m#1top) (m#1right)
}
\newcommand{\addnetworkbottomleft}[2]{
	\syncx[unconsidered, #2] (m#1left) (m#1bottom)
}
\newcommand{\addpseudobottom}[1]{
	\node[model, draw=none, fill=none, below right=\vmodeldistance and \hmodeldistance of m#1left.center, anchor=center] (m#1bottom) {};
}
\newcommand{\addpseudoright}[1]{
	\node[model, draw=none, fill=none, right=2*\hmodeldistance of m#1left.center, anchor=center] (m#1right) {};	
}

\newcommand{\drawnetworktopleft}[3]{
	\addnetworktopleft{#1}{#2}{#3}
	% Not necessary because no dependent networks, so save vertical space
	%\addpseudobottom{#1}
	%\addpseudoright{#1}
}
\newcommand{\drawnetworktopleftmiddle}[4]{
	\addnetworktopleft{#1}{#2}{#3}
	\addnetworkmiddle{#1}{#4}
	\addpseudoright{#1}
}
\newcommand{\drawnetworktop}[4]{
	\addnetworktopleft{#1}{#2}{#3}
	\addnetworktopright{#1}{#4}
	\addpseudobottom{#1}
}
\newcommand{\drawnetworktopmiddle}[5]{
	\addnetworktopleft{#1}{#2}{#3}
	\addnetworktopright{#1}{#4}
	\addnetworkmiddle{#1}{#5}
}
% #1: Number
% #2: Position of left model
% #3-#6: Properties of transformations (top left, top right, middle, bottom left)
\newcommand{\drawnetworkcomplete}[6]{
	\addnetworktopleft{#1}{#2}{#3}
	\addnetworktopright{#1}{#4}
	\addnetworkmiddle{#1}{#5}
	\addnetworkbottomleft{#1}{#6}
}
% #1: First network
% #2: Second network
\newcommand{\iteration}[2]{
	\draw[thick,-{>[scale=0.6]}] ($(#1right.east)+(0.5em,0)$) -- node[above,font=\scriptsize] {it} ($(#2left.west)+(-0.5em,0)$);
}
% #1: First network
% #2: Second network
\newcommand{\recursion}[2]{
	\draw[thick,-{>[scale=0.6]}] ($(#1bottom.south)+(0,-0.35em)$) -- node[right,font=\scriptsize] {rec} ($(#2top.north)+(0,0.35em)$);
}

\settransformationscale{0.75}

\begin{tikzpicture}[
	model/.style={draw, circle, fill=black, inner sep=\modelradius},
	executed/.style={color=blue!50},
	candidate/.style={color=red!50},
	unconsidered/.style={color=gray},
	every node/.append style={font=\scriptsize}
]
%\drawnetworkcomplete{1}{}{}{}{}{}

\node[fill=gray!10, above left=1*\vmodeldistance+1*\vborder and \hborder, anchor=north west, minimum width=2*\hmodeldistance+2*\hborder, minimum height=2*\vmodeldistance+2*\vborder] {};

\drawnetworkcomplete{2}{%right=\hnetworkdistance of m1left.center, anchor=center
}{candidate}{}{}{}

\node[fill=gray!10, above right=1*\vmodeldistance+1*\vborder and \hnetworkdistance-\hborder of m2left.center, anchor=north west, minimum width=2*\hmodeldistance+2*\hborder, minimum height=\vnetworkdistance+1*\vmodeldistance+2*\vborder] {};

\drawnetworkcomplete{3}{right=\hnetworkdistance of m2left.center, anchor=center}{executed}{candidate}{}{}
\iteration{m2}{m3}
\drawnetworktopleft{3-1}{below=\vnetworkdistance of m3left.center,anchor=center}{candidate}
\recursion{m3}{m3-1}

\node[fill=gray!10, above right=1*\vmodeldistance+1*\vborder and \hnetworkdistance-1*\hborder of m3left.center, anchor=north west, minimum width=\hnetworkdistance+2*\hmodeldistance+2*\hborder, minimum height=2*\vnetworkdistance+1*\vmodeldistance+2*\vborder] {};

\drawnetworkcomplete{4}{right=\hnetworkdistance of m3left.center, anchor=center}{executed}{executed}{candidate}{}
\iteration{m3}{m4}
\drawnetworktop{4-1}{below=\vnetworkdistance of m4left.center,anchor=center}{candidate}{}
\recursion{m4}{m4-1}
\node[fill=gray!20, above right=1*\vmodeldistance+1*\vborder and \hnetworkdistance-1*\hborder of m4-1left.center, anchor=north west, minimum width=2*\hmodeldistance+2*\hborder, minimum height=\vnetworkdistance+1*\vmodeldistance+2*\vborder] {};
\drawnetworktop{4-2}{right=\hnetworkdistance of m4-1left.center,anchor=center}{executed}{candidate}
\iteration{m4-1}{m4-2}
\drawnetworktopleft{4-2-1}{below=\vnetworkdistance of m4-2left.center,anchor=center}{candidate}
\recursion{[yshift=0.1*\vnetworkdistance]m4-2}{[yshift=0.1*\vnetworkdistance]m4-2-1}

\node[fill=gray!10, above right=1*\vmodeldistance+1*\vborder and 2*\hnetworkdistance-1*\hborder of m4left.center, anchor=north west, minimum width=3*\hnetworkdistance+2*\hmodeldistance+2*\hborder, minimum height=3*\vnetworkdistance+1*\vmodeldistance+2*\vborder] {};

\drawnetworkcomplete{5}{right=2*\hnetworkdistance of m4left.center, anchor=center}{executed}{executed}{executed}{candidate}
\iteration{[xshift=1*\hnetworkdistance]m4}{m5}
\drawnetworktopmiddle{5-1}{below=\vnetworkdistance of m5left.center,anchor=center}{candidate}{}{}
\recursion{m5}{m5-1}
\node[fill=gray!20, above right=1*\vmodeldistance+1*\vborder and \hnetworkdistance-1*\hborder of m5-1left.center, anchor=north west, minimum width=2*\hmodeldistance+2*\hborder, minimum height=\vnetworkdistance+1*\vmodeldistance+2*\vborder] {};
\drawnetworktopmiddle{5-2}{right=\hnetworkdistance of m5-1left.center,anchor=center}{executed}{candidate}{}
\iteration{m5-1}{m5-2}
\drawnetworktopleft{5-2-1}{below=\vnetworkdistance of m5-2left.center,anchor=center}{candidate}
\recursion{m5-2}{m5-2-1}
\node[fill=gray!20, above right=1*\vmodeldistance+1*\vborder and \hnetworkdistance-1*\hborder of m5-2left.center, anchor=north west, minimum width=\hnetworkdistance+2*\hmodeldistance+2*\hborder, minimum height=2*\vnetworkdistance+1*\vmodeldistance+2*\vborder] {};
\drawnetworktopmiddle{5-3}{right=\hnetworkdistance of m5-2left.center,anchor=center}{executed}{executed}{candidate}
\iteration{m5-2}{m5-3}
\drawnetworktopleftmiddle{5-3-1}{below=\vnetworkdistance of m5-3left.center,anchor=center}{candidate}{}
\recursion{m5-3}{m5-3-1}
\node[fill=gray!30, above right=1*\vmodeldistance+1*\vborder and \hnetworkdistance-1*\hborder of m5-3-1left.center, anchor=north west, minimum width=2*\hmodeldistance+2*\hborder, minimum height=1*\vnetworkdistance+1*\vmodeldistance+2*\vborder] {};
\drawnetworktopleftmiddle{5-3-2}{right=\hnetworkdistance of m5-3-1left.center,anchor=center}{executed}{candidate}
\iteration{m5-3-1}{m5-3-2}
\drawnetworktopleft{5-3-2-1}{below=\vnetworkdistance of m5-3-2left.center,anchor=center}{candidate}
\recursion{m5-3-2}{m5-3-2-1}

% Legend
\newcommand{\vdistancelegend}{1.2*\vmodeldistance}
%\coordinate (legend_left_column) at ([yshift=-2.53*\vnetworkdistance]m2left.center);
\coordinate (legend_left_column) at ([yshift=-1.6*\vnetworkdistance,xshift=0.2em]m2left.center);

\node[draw, fill=lightgray!30, minimum height=5.9*\vdistancelegend, minimum width=7.2em, above right=1.1*\vdistancelegend and -0.75em of legend_left_column, anchor=north west] {};

\syncx[candidate] (legend_left_column) ([xshift=1*\hmodeldistance]legend_left_column)
\node[right=1.3*\hmodeldistance of legend_left_column,anchor=west] {candidate};
\syncx[executed] ([yshift=-\vdistancelegend]legend_left_column) ([xshift=1*\hmodeldistance,yshift=-\vdistancelegend]legend_left_column)
\node[below right=\vdistancelegend and 1.3*\hmodeldistance of legend_left_column,anchor=west] {executed};

\coordinate (legend_right_column) at ([yshift=-2.8*\vdistancelegend]legend_left_column); %([xshift=7em]legend_left_column);

\coordinate (legenditerationright) at ([xshift=-0.3*\hmodeldistance+\modelradius]legend_right_column);
\coordinate (legenditerationleft) at ([xshift=-0.3*\hmodeldistance-2*\hmodeldistance-\modelradius+\hnetworkdistance]legend_right_column);
\iteration{legenditeration}{legenditeration}
\node[right=1.3*\hmodeldistance of legend_right_column, anchor=west] (legenditerationtext) {iteration step};

\coordinate (legendrecursionbottom) at ([yshift=-\vdistancelegend+0.5*\vnetworkdistance-\vmodeldistance-\modelradius]legend_right_column);
\coordinate (legendrecursiontop) at ([yshift=-\vdistancelegend-0.5*\vnetworkdistance+\vmodeldistance+\modelradius]legend_right_column);
\recursion{legendrecursion}{legendrecursion}
\node[below=\vdistancelegend of legenditerationtext.west, anchor=west] {recursion step};

\end{tikzpicture}%