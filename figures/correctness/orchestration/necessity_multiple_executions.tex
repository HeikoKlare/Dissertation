\newcommand{\pcmclasswidth}{5em}
\newcommand{\umlclasswidth}{4.2em}
\newcommand{\distance}{(\pcmclasswidth+5em)}
\newcommand{\labeldistance}{1.0em}

\begin{tikzpicture}[
    correspondence/.style={consistency related element, -},
    invalid correspondence/.style={correspondence, darkred, dashed}
]

\pgfdeclarelayer{bg}
\pgfsetlayers{bg,main}

% First
\umlclassvarwidth{pcm1_i}{}{\umlinterfacelabel\\ I}{
    \dots
}{\pcmclasswidth}
\umlclassvarwidth[, below=0.4*\distance of pcm1_i.north, anchor=north]{pcm1_c}{}{\umlcomponentlabel\\ C}{
}{\pcmclasswidth}

\umlclassvarwidth[, right=1\distance of pcm1_i.north, anchor=north]{uml1_i}{}{\umlinterfacelabel\\ I}{
    \dots
}{\umlclasswidth}
\umlclassvarwidth[, below=0.4*\distance of uml1_i.north, anchor=north]{uml1_c}{}{CImpl}{
    CImpl()
}{\umlclasswidth}

\node[above right=0.0*\distance and 1.25*\distance of uml1_i.north, anchor=north, text width=10em, inner sep=0em] (java1_code) {
\begin{lstlisting}[language=java, numbers=none, basicstyle=\footnotesize\ttfamily]
interface I { §\dots§ }

class CImpl implements I {
    CImpl() { §\dots§ }
}
\end{lstlisting}
};
\coordinate (java1_south) at ([yshift=1em]java1_code.south);

% Second
\node[below=0.3*\distance of java1_code.south, anchor=north, text width=10em, inner sep=0em] (java2_code) {
\begin{lstlisting}[language=java, numbers=none, basicstyle=\footnotesize\ttfamily]
interface I { §\dots§ }

class CImpl implements I {
    §\textcolor{consistencypreservationcolor}{I f;}§
    CImpl() { §\dots§ }
}
\end{lstlisting}
};

\umlclassvarwidth[, left=1.25*\distance of java2_code.north, anchor=north]{uml2_i}{}{\umlinterfacelabel\\ I}{
    \dots
}{\umlclasswidth}
\umlfullclassvarwidth[, below=0.4*\distance of uml2_i.north, anchor=north]{uml2_c}{}{CImpl}{
    \textcolor{consistencypreservationcolor}{f : I}
}{
    CImpl()
}{\umlclasswidth}

\umlclassvarwidth[, below left=0.55*\distance and 1*\distance of uml2_i.north, anchor=north]{pcm2_i}{}{\umlinterfacelabel\\ I}{
    \dots
}{\pcmclasswidth}
\umlclassvarwidth[, below=0.5*\distance of pcm2_i.north, anchor=north]{pcm2_c}{}{\umlcomponentlabel\\ C}{
}{\pcmclasswidth}
\pcmrequirerole{pcm2_c}{pcm2_i}{right, consistencypreservationcolor}

% Third
\umlclassvarwidth[, below right=0.55*\distance and 1*\distance of pcm2_i.north, anchor=north]{uml3_i}{}{\umlinterfacelabel\\ I}{
    \dots
}{\umlclasswidth}
\umlfullclassvarwidth[, below=0.4*\distance of uml3_i.north, anchor=north]{uml3_c}{}{CImpl}{
    f : I
}{
    CImpl(\textcolor{consistencypreservationcolor}{f : I})
}{\umlclasswidth}

\node[above right=0.0*\distance and 1.25*\distance of uml3_i.north, anchor=north, text width=10em, inner sep=0em] (java3_code) {
\begin{lstlisting}[language=java, numbers=none, basicstyle=\footnotesize\ttfamily]
interface I { §\dots§ }

class CImpl implements I {
    I f;
    CImpl(§\textcolor{consistencypreservationcolor}{I f}§) { §\dots§ }
}
\end{lstlisting}
};

\node[mmlabel, above=\labeldistance of pcm1_i.north, anchor=center, font=\small\bfseries] (pcm_label) {\acrshort{PCM}};
\node[mmlabel, above=\labeldistance of uml1_i.north, anchor=center, font=\small\bfseries] (uml_label) {\acrshort{UML}};
\node[mmlabel, anchor=south, font=\small\bfseries] (uml_label) at (pcm_label.south-|java1_code.north) {Java};

\draw[correspondence] (pcm1_i) -- (pcm1_i-|uml1_i.west);
\draw[correspondence] (pcm1_c) -- (pcm1_c-|uml1_c.west);
\draw[correspondence] (uml1_i) -- ([xshift=-0.1em,yshift=1.9em]java1_code.west);
\draw[correspondence] (uml1_c) -- ([xshift=-0.1em,yshift=-0.2em]java1_code.west);

\draw[consistency execution,-latex] 
    ([xshift=-1em]java1_code.south)
    --
    node[right, align=center] {add field \\ \enquote{\texttt{I f}}}
    ([xshift=-1em]java2_code.north);

\draw[consistency execution,-latex] 
    ([xshift=-0.5em]java2_code.west)
    --
    node[above, align=center] {add field \\ \enquote{\texttt{f : I}}}
    ([xshift=0.5em]uml2_i.east|-java2_code.west);
\draw[consistency execution, -latex] 
    ([xshift=-0.5em]uml2_i.west|-java2_code.west)
    -|
    node[above left=0.3em and 0em, pos=1, align=center] {add\\ requi-\\ res}
    ([xshift=0.63*\distance,yshift=-0.1*\distance]pcm2_i.south)
    --
    ([xshift=0.4*\distance,yshift=-0.1*\distance]pcm2_i.south);
\draw[consistency execution, -latex] 
    ([xshift=0.63*\distance,yshift=-0.1*\distance]pcm2_i.south)
    |-
    node[pos=0.4, below left=0em and -0.8em, align=center] {add\\ constructor\\ parameter}
    ([xshift=-0.5em]uml3_i.west|-java3_code.west);
\draw[consistency execution,-latex]
    ([xshift=0.5em]uml3_i.east|-java3_code.west)
    --
    node[above, align=center] {add\\ constructor\\ parameter}
    ([xshift=-0.5em]java3_code.west);
% \draw[consistency execution, -latex] 
%     ([xshift=0.65*\distance,yshift=-0.1*\distance]pcm2_i.south)
%     |-
%     node[below, pos=0.75, align=center] {add \\constructor\\ parameter}
%     ([xshift=-0.5em]uml2_i.west|-java3_code.west);

% \draw[correspondence] (pcm1) -- (pcm1-|uml1.west);
% \draw[correspondence] ([xshift=0.5em]pcm1.north) -- ++(0,1em) -| (java1.north);
% \draw[correspondence] ([xshift=-0.5em]pcm1.north) -- ++(0,2em) -| ([xshift=1em]java2.east) -- (java2.east);
% \draw[correspondence] (pcm2) -- (pcm2-|uml2.west);
% \draw[correspondence] (pcm2.south) -- ++(0,-1em) -| (java2.south);
% \draw[correspondence] (java1) -- (java1-|uml1.east);
% \draw[invalid correspondence] (java1) -- (uml2.north east);
% \draw[correspondence] (java2) -- (java2-|uml2.east);
% \draw[correspondence] (java2) -- (uml1);

\end{tikzpicture}