\newcommand{\classwidth}{5em}
\newcommand{\mmborder}{1em}
\newcommand{\distance}{(\classwidth+2*\mmborder+2em)}
\newcommand{\labeldistance}{1.0em}


\begin{tikzpicture}[
    invalid correspondence/.style={correspondence, darkred, dashed}
]

\pgfdeclarelayer{bg}
\pgfsetlayers{bg,main}

% PCM
\umlclassvarwidth{pcm1_ia}{}{\umlinterfacelabel\\ IA}{
}{\classwidth}
\umlclassvarwidth[, below=0.5*\distance of pcm1_ia.north, anchor=north]{pcm1_a}{}{\umlcomponentlabel\\ A}{
}{\classwidth}
\pcmprovidedrole{[xshift=1em]pcm1_a.north}{[xshift=1em]pcm1_ia.south}{left, stereotype}

\umlclassvarwidth[, below=1.3*\distance of pcm1_ia.north, anchor=north]{pcm2_ia}{}{\umlinterfacelabel\\ IA}{
    method()
}{\classwidth}
\umlclassvarwidth[, below=0.6*\distance of pcm2_ia.north, anchor=north]{pcm2_a}{}{\umlcomponentlabel\\ A}{
}{\classwidth}
\pcmprovidedrole{[xshift=1em]pcm2_a.north}{[xshift=1em]pcm2_ia.south}{left, stereotype}

% UML
\umlclassvarwidth[, right=\distance of pcm1_ia.north, anchor=north]{uml1_ia}{}{\umlinterfacelabel\\ IA}{
}{\classwidth}
\umlclassvarwidth[, below=0.5*\distance of uml1_ia.north, anchor=north]{uml1_a}{}{A}{
    method()
}{\classwidth}
\umlsubclassof{uml1_a}{--}{uml1_ia}

\umlclassvarwidth[, right=\distance of pcm2_ia.north, anchor=north]{uml2_ia}{}{\umlinterfacelabel\\ IA}{
    method()
}{\classwidth}
\umlclassvarwidth[, below=0.6*\distance of uml2_ia.north, anchor=north]{uml2_a}{}{A}{
}{\classwidth}
\umlsubclassof{uml2_a}{--}{uml2_ia}

% JAVA
\node[above right=0.0*\distance and 1.17*\distance of uml1_a.south, anchor=south, text width=9em, inner sep=0em] (java1_code) {
\begin{lstlisting}[language=java, numbers=none, backgroundcolor=\color{white}, basicstyle=\footnotesize\ttfamily]
interface IA {}

class A implements IA {
    void method() {}
}
\end{lstlisting}
};
\coordinate (java1_south) at ([yshift=1em]java1_code.south);

\node[above right=0.75em and 1.17*\distance of uml2_ia.north, anchor=north, text width=9em, inner sep=0em] (java2_code) {
\begin{lstlisting}[language=java, numbers=none, backgroundcolor=\color{white}, basicstyle=\footnotesize\ttfamily]
interface IA {
    void method();
}

class A implements IA {
    void method() {
        // log something
    }
}
\end{lstlisting}
};
\coordinate (java2_south) at ([yshift=1em]java2_code.south);

\node[mmlabel, above=\labeldistance of pcm1_ia.north, anchor=center] (pcm1_label) {\acrshort{PCM} Model 1};
\node[mmlabel, above=\labeldistance of pcm2_ia.north, anchor=center] (pcm2_label) {\acrshort{PCM} Model 2};
\node[mmlabel, above=\labeldistance of uml1_ia.north, anchor=center] (uml1_label) {\acrshort{UML} Model 1};
\node[mmlabel, above=\labeldistance of uml2_ia.north, anchor=center] (uml2_label) {\acrshort{UML} Model 2};
\node[mmlabel, above=\labeldistance of java1_code.north, anchor=north] (java1_label) {Java Model 1};
\node[mmlabel, above=\labeldistance of java2_code.north, anchor=north] (java2_label) {Java Model 2};

\begin{pgfonlayer}{bg}
    \node[mmbg, fit=(pcm1_ia)(pcm1_a)(pcm1_label.center), inner sep=\mmborder] (pcm1) {};
    \node[mmbg, fit=(pcm2_ia)(pcm2_a)(pcm2_label.center), inner sep=\mmborder] (pcm2) {};
    \node[mmbg, fit=(uml1_ia)(uml1_a)(uml1_label.center), inner sep=\mmborder] (uml1) {};
    \node[mmbg, fit=(uml2_ia)(uml2_a)(uml2_label.center), inner sep=\mmborder] (uml2) {};
    \node[mmbg, fit=(java1_code.west)(java1_code.east)(java1_south)(java1_label.center), inner sep=\mmborder] (java1) {};
    \node[mmbg, fit=(java2_code.west)(java2_code.east)(java2_south)(java2_label.center), inner sep=\mmborder] (java2) {};
\end{pgfonlayer}


\draw[correspondence] (pcm1) -- (pcm1-|uml1.west);
\draw[correspondence] ([xshift=0.5em]pcm1.north) -- ++(0,1em) -| (java1.north);
\draw[correspondence] ([xshift=-0.5em]pcm1.north) -- ++(0,2em) -| ([xshift=1em]java2.east) -- (java2.east);
\draw[correspondence] (pcm2) -- (pcm2-|uml2.west);
\draw[correspondence] (pcm2.south) -- ++(0,-1em) -| (java2.south);
\draw[correspondence] (java1) -- (java1-|uml1.east);
\draw[invalid correspondence] (java1) -- (uml2.north east);
\draw[correspondence] (java2) -- (java2-|uml2.east);
\draw[correspondence] (java2) -- (uml1);

\end{tikzpicture}