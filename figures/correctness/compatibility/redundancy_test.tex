\begin{tikzpicture}
    %% FRAMES
    % \draw[dotted, anchor=west] (-0.3, -1) rectangle (6, 7.5);
    %\draw[dotted, anchor=west] (7.2, -1) rectangle (12.3, 7.5);

    \node (r1) at (2.85,3.25) [draw, kit-green100, thick, dashed, minimum width=6.3cm,minimum height=8.5cm] {};
    \node (r2) at (9.75,3.25) [draw, kit-green100, thick, dashed, minimum width=5.1cm,minimum height=8.5cm] {};
    \draw[->, ultra thick, kit-green100] (r1) -- (r2);
    
    %% OCL EXPRESSIONS
    \node[name=persEmp, anchor=west, minimum width=5.5cm, text width=5.5cm, align=left] at (0, 5) {%
\begin{embeddedqvtcode}[linewidth=5.5cm, mathescape=true]
Person::firstname = fstn
$\wedge$ Person::lastname = lstn
$\wedge$ Employee::name=fstn + ' ' + lstn
\end{embeddedqvtcode}
        };
        
    \node[name=empRes, anchor=west, minimum width=5.5cm, text width=5.5cm, align=left] at (0, 1.5) {%
\begin{embeddedqvtcode}[linewidth=5.5cm, mathescape=true]
Resident::name = n
$\wedge$ Employee::name = n
\end{embeddedqvtcode}
        };
        
    \node[name=persRes, anchor=west, minimum width=5.5cm, text width=5.5cm, align=left] at (0, 0) {%
\begin{embeddedqvtcode}[linewidth=5.5cm, mathescape=true]
Person::firstname = fstn
$\wedge$ Person::lastname = lstn
$\wedge$ Resident::name=fstn + ' ' + lstn
\end{embeddedqvtcode}
        };
    
    %% LABELS    
    \node[above=0.1cm of persEmp, align=center] (possiblyredundant_label) {\large Possibly\\ \large redundant predicate};
    \node[above=0.1cm of empRes, align=center] (alternative_label) {\large Alternative sequence\\ \large of predicates};
    
    %% SMT FORMULAE
    \node[name=smtPersEmp, anchor=west, right=2cm of persEmp, minimum width=4cm, text width=4cm, align=left] {%
\begin{z3code}[linewidth=4cm]
(not
  (and
    (= firstname fstn)
    (= lastname lstn)
    (= name
      (str.++ fst
        (str.++ ' ' lstn)
      )
    )
  )
)
\end{z3code}
        };
        
    \node[name=smtEmpRes, anchor=west, right=2cm of empRes, minimum width=4cm, text width=4cm, align=left] {%
\begin{z3code}[linewidth=4cm]
...
\end{z3code}
        };
        
    \node[name=smtPersRes, anchor=west, right=2cm of persRes, minimum width=4cm, text width=4cm, align=left] {%
\begin{z3code}[linewidth=4cm]
...
\end{z3code}
        };
        
    \draw[->] (persEmp) to (smtPersEmp);
    \draw[->] (empRes) to (smtEmpRes);
    \draw[->] (persRes) to (smtPersRes);
    
    \node[above=-0.3cm of smtEmpRes] (wedge1) {$\wedge$};
    \node[above=-0.3cm of smtPersRes] (wedge2) {$\wedge$};
        
    %% SMT SOLVER
    \node[draw,rectangle, minimum size=2cm, kit-blue100, fill=kit-blue30, align=center] (smtsolver) at (14.5, 3.15){\large SMT\\\large Solver};
    \draw[->, ultra thick, kit-green100] (r2) -- (smtsolver);
    
    %% OUTCOMES
    \node[text width=3cm] (sat) at (18, 6.5){\textbf{SAT}. The initial Horn clause is not always valid so the predicate is not entirely redundant. \textit{No removal.}};
    \node[text width=3cm] (unknown) at (18, 3.25){\textbf{UNKNWON}. By conservativeness. \textit{No removal.}};
    \node[text width=3cm] (unsat) at (18, 0){\textbf{UNSAT}. The initial Horn clause is valid so the predicate is redundant. \textit{Removal.}};
    
    \draw[->] (smtsolver) -- (sat);
    \draw[->] (smtsolver) -- (unknown);
    \draw[->] (smtsolver) -- (unsat);
    
    \node[below = 0.25cm of r1] {\textbf{OCL Expressions}};
    \node[below = 0.25cm of r2, name=smtf] {\textbf{SMT Formula}};
    \node[right = 4cm of smtf] {\textbf{Redundancy Test}};
\end{tikzpicture}