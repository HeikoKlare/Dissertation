\newcommand{\distance}{4em}
\newcommand{\vdistance}{2.5em}
\newcommand{\scenariodistance}{4*\distance}
\newcommand{\overlapdistance}{0.3em}

\begin{tikzpicture}[
    manifests relation label/.style={manifests relation, above, sloped}
]

% EXTERNAL CONCEPT DEFINITION

\node[schematic conceptmetamodel] (cmm_concept1) {};
\node[schematic conceptmetamodel, right=\overlapdistance of cmm_concept1.center, anchor=center] (cmm_concept2) {};
\node[schematic conceptmetamodel, right=\overlapdistance of cmm_concept2.center, anchor=center] (cmm_concept3) {};

\node[schematic conceptmetamodel, right=\distance of cmm_concept1.center, anchor=center] (cmm_rel1_concept1) {};
\node[schematic conceptmetamodel, right=\overlapdistance of cmm_rel1_concept1.center, anchor=center] (cmm_rel1_concept2) {};
\node[schematic conceptmetamodel, right=\overlapdistance of cmm_rel1_concept2.center, anchor=center] (cmm_rel1_concept3) {};

\node[schematic metamodel, below left=\vdistance and 0.2*\distance of cmm_rel1_concept1.center, anchor=center] (cmm_rel1_concrete1) {};
\node[schematic metamodel, right=\overlapdistance of cmm_rel1_concrete1.center, anchor=center] (cmm_rel1_concrete2) {};
\node[schematic metamodel, right=\overlapdistance of cmm_rel1_concrete2.center, anchor=center] (cmm_rel1_concrete3) {};

\node[schematic conceptmetamodel, right=\distance of cmm_rel1_concept1.center, anchor=center] (cmm_rel2_concept1) {};
\node[schematic conceptmetamodel, right=\overlapdistance of cmm_rel2_concept1.center, anchor=center] (cmm_rel2_concept2) {};
\node[schematic conceptmetamodel, right=\overlapdistance of cmm_rel2_concept2.center, anchor=center] (cmm_rel2_concept3) {};

\node[schematic metamodel, below right=\vdistance and 0.2*\distance of cmm_rel2_concept1.center, anchor=center] (cmm_rel2_concrete1) {};
\node[schematic metamodel, right=\overlapdistance of cmm_rel2_concrete1.center, anchor=center] (cmm_rel2_concrete2) {};
\node[schematic metamodel, right=\overlapdistance of cmm_rel2_concrete2.center, anchor=center] (cmm_rel2_concrete3) {};

\draw[manifests relation] (cmm_rel1_concept1) -- (cmm_rel1_concrete1);
\draw[manifests relation] (cmm_rel1_concept2) -- (cmm_rel1_concrete2);
\draw[manifests relation] (cmm_rel1_concept3) -- (cmm_rel1_concrete3);

\draw[manifests relation] (cmm_rel2_concept1) -- (cmm_rel2_concrete1);
\draw[manifests relation] (cmm_rel2_concept2) -- (cmm_rel2_concrete2);
\draw[manifests relation] (cmm_rel2_concept3) -- (cmm_rel2_concrete3);

\node[below right=0.5*\vdistance and 0.5*\distance of cmm_concept2.center, anchor=center, font=\bfseries] {+};
\node[below right=0.5*\vdistance and 0.5*\distance of cmm_rel1_concept2.center, anchor=center, font=\bfseries] {+};


% INTERNAL CONCEPT DEFINITION

\node[schematic conceptmetamodel, right=\scenariodistance of cmm_concept1.center, anchor=center] (int_concept1) {};
\node[schematic metamodel, below left=\vdistance and 0.2*\distance of int_concept1.center, anchor=center] (int_concept1_concrete1) {};
\node[schematic metamodel, below right=\vdistance and 0.2*\distance of int_concept1.center, anchor=center] (int_concept1_concrete2) {};
\draw[manifests relation] (int_concept1) -- (int_concept1_concrete1);
\draw[manifests relation] (int_concept1) -- (int_concept1_concrete2);

\node[schematic conceptmetamodel, right=\distance of int_concept1.center, anchor=center] (int_concept2) {};
\node[schematic metamodel, below left=\vdistance and 0.2*\distance of int_concept2.center, anchor=center] (int_concept2_concrete1) {};
\node[schematic metamodel, below right=\vdistance and 0.2*\distance of int_concept2.center, anchor=center] (int_concept2_concrete2) {};
\draw[manifests relation] (int_concept2) -- (int_concept2_concrete1);
\draw[manifests relation] (int_concept2) -- (int_concept2_concrete2);

\node[schematic conceptmetamodel, right=\distance of int_concept2.center, anchor=center] (int_concept3) {};
\node[schematic metamodel, below left=\vdistance and 0.2*\distance of int_concept3.center, anchor=center] (int_concept3_concrete1) {};
\node[schematic metamodel, below right=\vdistance and 0.2*\distance of int_concept3.center, anchor=center] (int_concept3_concrete2) {};
\draw[manifests relation] (int_concept3) -- (int_concept3_concrete1);
\draw[manifests relation] (int_concept3) -- (int_concept3_concrete2);

\node[below right=0.5*\vdistance and 0.5*\distance of int_concept1.center, anchor=center, font=\bfseries] {+};
\node[below right=0.5*\vdistance and 0.5*\distance of int_concept2.center, anchor=center, font=\bfseries] {+};

\node[above=0.5em of cmm_rel1_concept2.north, anchor=south, font=\small\bfseries] (ext_label) {External Concept Definition};
\node[above=0.5em of int_concept2.north, anchor=south, font=\small\bfseries] (int_label) {Internal Concept Definition};

\node[below=1.7*\vdistance of cmm_rel1_concept2.center, anchor=center, font=\footnotesize\itshape] (ext_dim_label) {Decomposition Dimension: Relations};
\node[below=1.7*\vdistance of int_concept2.center, anchor=center, font=\footnotesize\itshape] (int_dim_label) {Decomposition Dimension: Commonalities};

\draw[gray, thin] ($(ext_label.north)!0.5!(int_label.north)$) -- ($(ext_dim_label.south)!0.5!(int_dim_label.south)$);


\end{tikzpicture}