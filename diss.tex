%% Dissertationsvorlage
%%
%% Karlsruhe Institute of Technology
%% Institute for Program Structures and Data Organization
%% Chair for Software Design and Quality (SDQ)
%%
%% Dr.-Ing. Erik Burger
%% burger@kit.edu
%%
%% Siehe https://sdqweb.ipd.kit.edu/wiki/Dokumentvorlagen
%%
%% {$HeadURL: https://svnserver.informatik.kit.edu/i43/svn/presentations/SDQ-Dissertations-Vorlage/LaTeX-Dateien/sdqdiss.tex $}
%% {$LastChangedDate: 2018-08-02 14:23:11 +0200 (Do, 02 Aug 2018) $}
%% {$LastChangedRevision: 1078 $}
%% {$LastChangedBy: burger $}

%% Dokumentoptionen
%
% Für das Proposal bzw. vorgelegte Fassung (DIN A4):
%
% \documentclass{sdqdiss-a4}
%
% Für die Druckfassung im Format von KIT Scientific Publishing (DIN A5)
% bitte auch für die Bindekorrektur den Umfang des Dokuments eintragen:
%
% z.B. \documentclass[largediss]{sdqdiss-a5-ksp}
% 
% +------------+-----------------------+
% | Seitenzahl | Option                |
% +------------+-----------------------+
% | < 200      | smalldiss             |
% | 200–399    | mediumdiss (Standard) |
% | > 400      | largediss             |
% +------------+-----------------------+

\documentclass{sdqdiss-24x17-ksp} %5-ksp}
  
% Benötigt für die Typewriter-Schriftart
\usepackage[T1]{fontenc}
\usepackage[utf8]{inputenc}
  
% Für Sprachumschaltung
\usepackage[ngerman,english]{babel}

% Abstract
\newcommand{\Abstract}[1][Abstract]{\chapter*{#1}\addcontentsline{toc}{chapter}{#1}\markboth{#1}{#1}} 

%% Nur als Beispiel – im endgültigen Dokument bitte entfernen
\usepackage{blindtext}
% TikZ ist kein Zeichenprogramm
\usepackage{tikz}
% Schöne Tabellen
\usepackage{booktabs}
%% /Beispiel

%%
%% Beginn des Dokuments
%%

\title{Transformation Networks for Consistent Evolution of Multiple Heterogeneous Models}
\author{Heiko Klare}
\subject{}

\subtitle{
\vskip2em
zur Erlangung des akademischen Grades eines\\[1em]
{\Large Doktors der Ingenieurwissenschaften}\\[1em]
%{\Large Doktors der Naturwissenschaften}\\[1em]
der KIT-Fakultät für Informatik\\
des Karlsruher Instituts für Technologie (KIT)\\[.5em]
{vorgelegte}\\[.3em]
{\Large Dissertation}
}

\author{\normalsize{von}\\
{\LARGE Heiko Klare}\\
\normalsize{aus Höxter}
}

\publishers{%
\flushleft\small
Tag der mündlichen Prüfung: XX. Monat XXXX\\
Erster Gutachter: Prof.\ Dr.\ Ralf H.\ Reussner\\
Zweiter Gutachter: Prof.\ Dr.\ Colin Atkinson\\
}

\date{}

%%
%% Titelseite
%%

\begin{document}
\maketitle

%%
%% Abstract/Zusammenfassung
%%
% start with roman page numbers
\frontmatter

%% Englischer Abstract
\selectlanguage{english}
\Abstract{}
%\Blindtext[10]

%% Deutsche Zusammenfassung
\selectlanguage{ngerman}
\Abstract[Zusammenfassung]{}
%\Blindtext[10]

% Sprachumschaltung
\selectlanguage{english}

%%
%% Inhaltsverzeichnis
%%
\tableofcontents

% Linksbündige Einträge ohne Einzug
\selecttocstyleoption{tocflat}
%\listoffigures
%\listoftables

%%
%% Inhalt
%%
% restart with arabic page numbers
\mainmatter

\part{Prologue}
\chapter{Introduction}
\chapter{Foundations}

\part{Formalizing Transformation Networks}
\chapter{A Classification of Consistency Specification Levels}
\chapter{A Categorization of Errors in Transformation Networks}
\chapter{A Formalization of Transformation Networks}

\part{Building Correct Transformation Networks}
\chapter{Analyzing Transformations for Contradicting Consistency Relations}
\chapter{Ensuring Interoperability of Executed Transformations}
\chapter{Orchestrating the Execution of Transformation Networks}

\part{Improving Non-Functional Properties of Transformation Networks}
\chapter{Topologies and Properties of Transformation Networks}
\chapter{Specification of Consistency Relations with Commonalities}
\chapter{A Consistency Specification Language for Commonalities}

\part{Evaluation and State-of-the-Art}
\chapter{Evaluation and Discussion}
\chapter{Related Work}

\part{Epilogue}
\chapter{Future Work}
\chapter{Conclusion}


\end{document}